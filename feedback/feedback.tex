\documentclass[10pt,a4paper]{article}
\usepackage[utf8]{inputenc}
\usepackage{amsmath}
\usepackage{amsfonts}
\usepackage{amssymb}
\usepackage[left=3cm,right=3cm,top=2cm,bottom=2cm]{geometry}
\author{Marc Fabel}\date{\flushleft{Letzte Bearbeitung des Dokuments: \today}}
\title{Feedback}
\begin{document}
\maketitle

Items in bold are to be considered more important than others
\bigskip
\section{Verein für Sozialpolitik [Wien, September 2017]}
\subsection{Conceptional during presenation:}
\begin{itemize}
	\item Look up: does reform induce substitution for \textbf{formal/informal} care? [commentator mentioned that he found positive effects for Austria] 
	\item Nothing found for Micro Census: is that an issue of too few observations? Carry out \textbf{power calculations}. This should not be difficult as the size of the treatment effect is known from the hospital registry data.
	\item Correction of \textbf{p-values}: adjust for multiple outcomes (\texttt{Bonferroni correction}?). The issue is that 1 out of 20 outcome is significant by pure chance. This point was mentioned in association to significant coefficient of Dtrack2. [Lena Janys]
	\item How are the results \textbf{comparable to other studies} relevant in that field (as they look at similar reforms at roughly the same times)? What do they find, what do we find? Differences in scales, why?[Lena Janys]
	\item Channels: look at maternal health outcomes [D. Kühnle]
	\item we find very \textbf{large effects} [D. Kühnle]
\begin{itemize}
\item[-]how do we explain these given that other studies find little/nothing    
\item[-] introduction (expansion) positive (no) effects
\item[-] are the effects actually too large to be true?\footnote{Stefanie Schurer: Perry Preschool Program (heavy intervention): effect of 1 SD.}
\end{itemize}	
	\item hospital admission very extreme $\rightarrow$ less extreme outcomes? [D. Kühnle] RWI HI data
	\item what is with reduced form evidence $\rightarrow$ \textbf{RD plots}? [D. Kühnle]
	\item Suggestion to have \textbf{more focus} on one sort of disease, motivate more concisely [D. Kühnle]
\end{itemize}

\bigskip
\subsection{Presentation itself:}
\begin{itemize}
\item Outcome table is not readable $\rightarrow$ break up in multiple tables
\item Include more information on data: apparently the terminology of a hospital is very wide   
\end{itemize}


\newpage
\subsection{Input from other sources}
\begin{itemize}
\item Nice motivation why mental health problems are so important to focus on [D. Schnitzlein]
\item Include gender in predetermined variables (MZ) 
\item Pfeiffer (Hohenheim)
\begin{itemize}
\item identification: \textbf{check parallel trends}: Granger style regression/event study/leads and lags test
\item inference: clustered data structure, yet to few clusters $\rightarrow$ randomization inference (harder to reject the null)
\end{itemize}
\end{itemize}







\end{document}
