\documentclass[11pt,a4paper]{article}
\usepackage[utf8]{inputenc}
\usepackage{amsmath}
\usepackage{amsfonts}
\usepackage{amssymb}
\usepackage{graphicx}
\usepackage[left=3cm,right=3cm,top=2cm,bottom=2cm]{geometry}
\author{Marc Fabel}
\title{To Do Liste}
\date{\flushleft{Last revision of the document: \today}}
\begin{document}
\maketitle




%%%%%%%%%%%%%%%%%%%%%%%%%%%%%%%%%%%%%%%%%%%%%%%%%
Elemente mit der höchsten Priorität sind am obersten Ende der Liste:
\begin{enumerate}
\item \textbf{Pooling}:\newline Ist das  überhaupt in Ordnung? Impliziert, dass die Effekte über die Jahre diesselben sind. Alternativ: für ausgewählte	Jahre zeigen
\item \textbf{Parental covariate balance:}\newline sollte unabhängig davon sein ob das Kind noch im Haushalt lebt. Andere Datenquellen durchsuchen (z.B. IPUMS)
\end{enumerate}


\end{document}