\documentclass[11pt,a4paper]{article}
\usepackage[utf8]{inputenc}
\usepackage{amsmath}
\usepackage{amsfonts}
\usepackage{amssymb}
\usepackage{graphicx}
\usepackage[left=3cm,right=3cm,top=2cm,bottom=2cm]{geometry}
\author{Marc Fabel}
\title{To Do Liste}
\date{\flushleft{Last revision of the document: \today}}
\begin{document}
\maketitle




%%%%%%%%%%%%%%%%%%%%%%%%%%%%%%%%%%%%%%%%%%%%%%%%%
Elemente mit der höchsten Priorität sind am obersten Ende der Liste:
\begin{enumerate}
\item \textbf{Pooling}:\newline Ist das  überhaupt in Ordnung? Impliziert, dass die Effekte über die Jahre diesselben sind. Alternativ: für ausgewählte	Jahre zeigen

\item \textbf{Parental covariate balance:}\newline sollte unabhängig davon sein ob das Kind noch im Haushalt lebt. Andere Datenquellen durchsuchen (z.B. IPUMS)


\begin{quote}
\texttt{[Natalia Email 19.09.]}\newline
Beispiel: MZ 1980 

1.	Info zu Geburtsmonat   Geboren in den Monaten
Januar bis April = 1
Mai bis Dezember  =2 

$\rightarrow$	Zumindest kann man, da reform cut-off May 1, 1979 genau der geburtsmonatseinteilung entspricht, in pre and post-kids/families unterteilen.

2.	Datenerhebung: 
Beginn 21.04.1980 
Ende 27.04.1980 
3. Variable ef66 Alter von 00-99.

$\rightarrow$	Vielleicht kann man aus ef66 und interview-datum noch weitere infos ziehen….

\end{quote}

\item \textbf{KKH Daten Main Table}\newline dependent mean (sd) auf control anpassen

\item \textbf{Bandwidth}\newline 2 Monate als Standardfall nehmen? \newline
Siehe auch Baker \& Milligan, $N=8$
\end{enumerate}


















\end{document}