\documentclass[11pt,a4paper]{article}
\usepackage[utf8]{inputenc}
\usepackage{amsmath}
\usepackage{amsfonts}
\usepackage{amssymb}
\usepackage{graphicx}
\usepackage[left=3cm,right=3cm,top=2cm,bottom=2cm]{geometry}
\usepackage{color,soul}			% ermöglicht das highlighten von text
\usepackage{hyperref}
\usepackage{enumerate}
\author{Marc Fabel}
\title{To Do Liste}
\date{\flushleft{Last revision of the document: \today}}
\begin{document}
\maketitle


\hfill{\hyperlink{DONE}{\Large{\texttt{List of things that were done}}}
\bigskip
\label{TODO}

%%%%%%%%%%%       Deadlines         %%%%%%%%%%%%%%%%%%%%%%%%%%
\textbf{Deadlines:}\newline
\begin{tabular}{cccc}
\hline 
Event & deadline & time	& eingereicht \\ 
\hline 
IZA world labor (Berlin) & 15.11. & 28-29.06 & \checkmark\\
\hline 
\end{tabular} 
%%%%%%%%%%%%%%%%%%%%%%%%%%%%%%%%%%%%%%%%%%%%%%%%%%%%%%%%%%%%%%%%%%%%%%%%%%%%%%%%%%%%%%%%%%%%%%%%%%
\newline

\bigskip
\section{Kleine To-Do Liste:}
\begin{itemize}
	\item[-] \textbf{GWAP}
	\begin{itemize}
		\item[-] Motivation: 
		\begin{itemize}
		\item accumulated level: smaller diagnoses groups, other types of robutsness (GDR)
		\item lower regional level: increased power, heterogeneity analysis (ML approach such as causal forests),heteterogeneity rural/urban (Zuordnung BBSR) , weighting according to population size, use regional FE
		\end{itemize}
		
		\item[-] erst quick'n dirty irgendwie, dann auf AMR (2 files wichtig\footnote{siehe Email Natalia 22.03.2018, "G:\textbackslash Projekte\textbackslash Projekte\_ab2016\textbackslash EcUFam\textbackslash Daten\textbackslash Arbeitsmarktregionen"})
		
		\item[-] robustness: is the effect present for all AMR, states, und co. (i) for each or (ii) leave one out
		
		\item[-] development cases per thousand for different age groups over time, e.g. the number of cases for 17-21 year olds in 1995, 1996, ... $\rightarrow$ why does the differential increase?
	\end{itemize}
	
		\item[-] AMR-level:
		\begin{itemize}
		\item other estimation methods:  leave open for referee: some zeros (approx 20\% of observations) $\rightarrow$ 0 inflated poisson, hazard models, tobit 
		\item cluster: MxYxFRG oder MxYxbula
		\item contrast placebo DD and DDD: bug? or try DD with afterxFRG and leave out the cohort born in the previous year
		\item fweights did not work: try round(pop/100)
		\end{itemize}
		

\item[-] \textbf{Local Files}
\begin{itemize}
	\item[-] think about \textbf{bootstrapping} of standard errors/ does it make sense? What is with se correction in the spirit of BDM (2004,QJE) oder CGM (2008,REStat)
	\item[-] add line in life-course (graph \& table): how many cases per thousand, why does the differenital increase? Graph im Vergleich zum Verlauf des means zeigen
	
	\item[-] robustness: drop February
	
	\item[-] \textbf{FLFP:} MZ SUF/PUF/stat. Jahrbücher
	
	\item[-] Adapt Table with choice of CG, more columns that are associated with only C1 as C3 is per se unintersting (see comments Natalia made in pdf)
\end{itemize}	

\item[-] \textbf{Hiwi: }
\begin{itemize}
\item[-] Geburtsdaten (nach Bundeslaender) digitalisieren
\end{itemize}
\end{itemize}
%%%%%%%%%%%%%%%%%%%%%%%%%%%%%%%%%%%%%%%%%%%%%%%%%%%%%%%%%%%%%%%%%%%%%%%%%%%%%%%%%%%%%%%%%%%%%%%%%%
\newpage
\section{More long-term things}
\begin{enumerate}
\item \textbf{Pooling}:\newline Ist das  überhaupt in Ordnung? Impliziert, dass die Effekte über die Jahre diesselben sind. Alternativ: für ausgewählte	Jahre zeigen


\item \textbf{Identification:}\vspace{-1em}
\begin{itemize}
	\item[-]\textbf{check parallel trends}: Granger style regression/event study/leads and lags test
\end{itemize}


\item \textbf{\hl{Reduced form}}\newline 
RD graphs:\vspace{-1em}
\begin{enumerate}
	\item treat und control in einem graph
	\item pro Jahr
	\item synthetic control groups
	\item nonparametric ? 
	\item with residuals ? 
	\item in sd units 
	\item for difference TG-CG
\end{enumerate}


\item \textbf{Parental covariate balance:}\newline sollte unabhängig davon sein ob das Kind noch im Haushalt lebt. Andere Datenquellen durchsuchen (z.B. IPUMS, oder SHARE/SOEP)
\begin{quote}
	\texttt{[Natalia Email 19.09.]}\newline
	Beispiel: MZ 1980 \newline
	1.	Info zu Geburtsmonat   Geboren in den Monaten
	Januar bis April = 1
	Mai bis Dezember  =2 \newline
	$\rightarrow$	Zumindest kann man, da reform cut-off May 1, 1979 genau der 	geburtsmonatseinteilung entspricht, in pre and post-kids/families unterteilen.\newline
	2.	Datenerhebung: 
	Beginn 21.04.1980 
	Ende 27.04.1980 \newline
	3. Variable ef66 Alter von 00-99.\newline
	$\rightarrow$	Vielleicht kann man aus ef66 und interview-datum noch weitere infos ziehen….
\end{quote}


\item  \textbf{spatial heterogeneity nach counterfactual mode of care [Rafael Lalive]:} \newline eligibility (wo haben damals die meisten Frauen gearbeitet? rough approx. rural-urban); vielleicht gleich einen spatial treatment intensity index erstellen:\newline $inten_{r,t}=f(FLFP_{r,1979}, Share_foreigners_{r,t}$, ) \newline
with region $r$ at time $t$. 
\begin{itemize}
	\item[-] \textbf{\hl{FLFP in 1979:}}\newline Sefan Wöhrmüller Daten oder shares aus MZ ziehen (	Marc Piopiunik) Bastian 16.OKTOBER!
\end{itemize}


\item \textbf{Cost-Benefit Analysis [Rafael Lalive]}\newline Back-of-the-envelope calculation, welche Kosten werden dem Gesundheitssystem eingespaart durch X weniger Diagnosen von Y, in relation setzen zu was hat das damals gekostet (kann man da die \textit{DRG} ausnützen?). Argumentieren entlang der Linien von Hillary Hoynes: benefits materialize way later and in other dimensions which were early goals in mind


\item \textbf{KKH Daten Main Table}\newline dependent mean (sd) auf control anpassen


\item \textbf{Bandwidth}\newline 2 Monate als Standardfall nehmen? \newline
Siehe auch Baker \& Milligan, $N=8$ [Natalia]


\item \textbf{Effect size:}\vspace{-1em}
\begin{itemize}
	\item[-]in sd units (HH: metabolic syndrome and footstamps: 0.4 sd , corresponds to a famine 	shock) [EALE]
	\item[-] How are the results \textbf{comparable to other studies} relevant in that field (as 	they look at similar reforms at roughly the same times)? What do they find, what do we find? 	Differences in scales, why?[Verein, Lena Janys]
	\item[-] effects too large? how do we explain these given that other studies find little/	nothing (usually: introduction (expansion) positive (no) effects) [D Kühnle, Verein]
	\item[-] Natalia Email [19.09.2017]\vspace{-0.5em}
	\begin{itemize}
		\item Wichtiges Argument, auch für die dt PL reform: betrifft Kinder bzw Babys in sehr 	jungem 	Alter.
		\item literature: Barnett (2011) Science; Nores \& Barnett (2010) 
	\end{itemize}
\end{itemize}


\item \textbf{$\Delta Y$:}\newline
Unterschied pre/post cutoff, zumindest im descriptive teil


\item \textbf{robustness section:}\vspace{-1em}
\begin{itemize}
	\item[-] ratios mit den approx. Bevölkerungszahlen aus dem MZ oder Zensus 2011 ( bisher applied: Gewichte aus der Geburtenstatistik - da mehr Jahres zur Verfügung stehen). fight potential threat of different migration or mortality around threshold. 
\end{itemize}


\item \textbf{\hl{Life-course perspective:}}\vspace{-1em}
\begin{itemize}
	\item[-]andere Spezifikation: pooled lassen und nicht auf das Alter beschränken sondern 	vielmehr den ITT mit year dummies interagieren: \newline
	$\gamma\  treat \times after \times yeardummy$
\end{itemize}


\item \textbf{Effekte durch die reform:}\newline
Look up: does reform induce substitution for \textbf{formal/informal} care? [Verein, commentator mentioned that he found positive effects for Austria] 


\item \textbf{P-values/power:}\vspace{-1em}
\begin{itemize}
	\item[-] Nothing found for Micro Census: is that an issue of too few observations? Carry out \textbf{power calculations}. This should not be difficult as the size of the treatment effect is known from the hospital registry data. [Verein]
	\item[-] Correction of \textbf{p-values}: adjust for multiple outcomes (\texttt{Bonferroni correction}?). The issue is that 1 out of 20 outcome is significant by pure chance. This point was mentioned in association to significant coefficient of Dtrack2. [Verein, Lena Janys]
\end{itemize}


\item \textbf{Channels:}\vspace{-1em}
\begin{itemize}
	\item[-] Maternal health outcomes
\end{itemize}


\item \textbf{Econometric specification:}\vspace{-1em}
\begin{itemize}
	\item[-] include gender in vector of controls
	\item[-] Poisson model for rates [EcUFam, Jason Lindo presentation]
	\item[-] Introduce weighting: not every observation is equally important $\rightarrow$ account for the fact that the share in Mai is more informative than in December
	\item[-] regression: Aufsummieren der Jahre und Regression mit kleinem $N$ laufen lassen
\end{itemize}


\item \textbf{Validity:}\vspace{-1em}
\begin{itemize}
	\item[-] family and newborn characteristics smooth around threshold; number of births McCrary test [EcUFam, L. Gonzalez, presentation]\newline \texttt{Natalia's comment:}possible sources: human fertility database (MPI for demographic research) haben die Zugriff auf births per day? Alternativ: Landesämter
\end{itemize}


\item \textbf{Other data/outcomes:}\vspace{-1 em}
\begin{itemize}
	\item[-] hospital admission very extreme $\rightarrow$ less extreme outcomes? [D. Kühnle] RWI HI data, Amelie Wuppermann (Krankenkassenvereinigung Berlin und TKK)
	\item[-] INKAR daten (Indikatoren und Karten zur Raum- und Stadtentwicklung)\newline kann ich da etwas rausfinden für variation of FLFP in late 1970s, immigrants 
	\item[-] Health of mothers: \vspace{-.5em}
	\begin{enumerate}
		\item RKI: aber dort sind die Daten – zumindest für die 1979 Reform – wahrscheinclih eher nicht geeignet [Natalia]
		\item  Nationale Kohorte – falls hier die genaue Geburtenhistorie abgefragt wird, könnte man hier evtl Glück haben. Den fragebogen konnte ich auf die Schnelle nicht auf der Homepage finden. \texttt{http://nako.de/}
	\end{enumerate}
	\item[-] Standardize independent variable: coefficient indicates "effect sizes"( of one standard deviation change) [EcuFam, Ch. Ruhm presentation]
	\item[-] Wir finden was bei University degree; was ist mit Lehre/keine Ausbildung/ Abbrechen [Anna Raute]
	\item[-] Wir finden was bei Schwangerschaften $\rightarrow$ sind das cesarean sections? [Anna Raute]
	\item[-] Outcomes zu children: (i) \#  children (ii) \# children$|$child$>$0 (iii) childlessness [Sonia Bhalotra, ES]
	\item[-] KKH Diagnosedaten: noch die "R" Befunde zu den outcomes hinzuzählen (unspezifische Symptome)
\end{itemize}


\item \textbf{"Hospital"}\newline What is defined as a hospital, read up in documentation. wide definition


\item \textbf{Framing im paper:} \vspace{-1em}
\begin{itemize}
	\item[-] fetal origin hypothesis (Barker, 1990): chronic conditions later in life can be attributed back to in-utero and early-childhood setting
	\item[-] LR effects of policies have to be taken into account, were probably not the original goal when dsigning the policy instrument, larger and in different dimensions than originally thought [HH, EALE keynote]
	\item[-] differentiation of treatment (was ist bei der refrom passiert?). 1. time allocation due to reform (figures is DS und SL) und 2. counterfactual mode of care: maternal care (the moms would have stayed at home anyway) $\rightarrow$ they benefit from additional job protection and 750 DM.
	(KÜhnle \& Hübner paper).
	\item[-] What is the difference between our work and to DS und LS? Different populations (zeros for all women with less than 20hrs).
	\item[-] Is the effect an effect on health or health care consumption [EcUFam, L. Ganzalez, discussion during her presentation]\newline Gibt es vielleicht manche Diagnosen, die tatsächlich random passieren $\rightarrow$ könnte helfen die frage zu beantworten [Anna Raute]
	\item[-] Suprising that there is no sorting of parents with cesarean sections: what was the share of parents in the 1970s? [EcUFam, feedback]
	\item[-] Why these health outcomes? Better motivation than just it was in other studies
	\item[-] Hospital $\mathrel{\widehat{=}}$ "more severe cases": wenn wir dabei bleiben argumentieren warum das ein lower bound ist was wir finden [Anna Raute]
	\item[-] Stäkrer auf die life-cycle Geschichte auslegen, da dies noch nicht Teil der Literatur ist [Anna Raute]
\end{itemize}

\item \textbf{Motivation im Paper:}
\begin{itemize}
	\item[-] Show relevance of certain diseases: 
	\begin{enumerate}[(i)]
	\item Overview table Diag X for 20-25 year olds over 1995-2014
	\item Comparison with other EU States (EUROSTAT) or the US
	\end{enumerate}
\end{itemize}

\item \textbf{Reading list:}
\begin{itemize}
	\item[-] Molly Schnell - The role of Physician Behavior in the Opiod Epidemic [EALE]\newline
	good job market paper, kleines theoretisches model nebenbei
	\item[-]  Dahmann \& Schnitzlein: The Protective (?) Effect of Education on Mental Health   \newline
 	Nice motivation why mental health problems are so important to focus on 
 	\item[-] Paper [Natalia, before EALE] Maternal stress and child outcomes, Journal of Human Resources
\end{itemize}
\end{enumerate}




%%%%%%%%%%%%%%%%%%%%%%%%%%%%%%%%%%%%%%%%%%%%%%%%%%%%%%%%%%%%%%%%%%%%%%%%%%%%%%%%%%%%%%%%%%%%%%%%%%
%				DONE 
%%%%%%%%%%%%%%%%%%%%%%%%%%%%%%%%%%%%%%%%%%%%%%%%%%%%%%%%%%%%%%%%%%%%%%%%%%%%%%%%%%%%%%%%%%%%%%%%%%
\newpage
\hfill{\hyperlink{TODO}{\Large{\texttt{Link back to what still has to be done}}}
\bigskip

\label{DONE}
\section{List of things that were done already: }
\begin{itemize}
\item \textbf{Outcomes: }\vspace{-1em}
	\begin{itemize}
	\item[-]\textbf{Metabolic Syndrome}  from Hoynes, Schanzenbach \& Almond (2016, AER) [EALE Keynote HH]\newline
	\textit{
	equal weighted average of the $z$-score (normalisation $\mu=0$ and $\sigma=1$) of five dichotomous variables: obese, diabetic, high blood pressure, heart disease and heart attack. The multiple measures were aggregated in order to improve statistical power. The mean and standard devtion is taken from the "control group" of cohorts born before the intervention.}
	\end{itemize}

\item \textbf{2. Datenanfrage FDZ} Outcomes 1995 - 2014 \& andere Diagnosen [02.11.2017]\newline
	\textit{
	(i) noch fehlende Jahre beantragt (1995-2005 + 2014 $\rightarrow$ 20 Jahre) und (ii) neue Einzeldiagnosen: Metabolisches Syndrom, Index für Krankheiten des Atmungssystems und Verhaltensstörung aufrgund von Drogen (inklusive Alkohol\footnote{Für Alkohol alleine reichen die Fallzahlen nicht aus; muss dann am GWAP gemacht werden. Ist vielleicht aber auch das interessantere Maß.}. Dieses Mal habe ich mich nur auf die Reform 1 und nur auf Westdeutschland beschränkt; dafür aber pro Gescchlecht. Für eine etwaige DDD müssen wir dann auch an den GWAP. Das resultierende Panel hat 1,824 Beobachtungen.}\newline
	\hl{Es gibt zwischen den ICD Kodierungen einen strukurellen Bruch: e.g. \texttt{scatter Diag\_Index\_resp year}} 
	
\item \textbf{Andere Datensätze, die nicht in Frage kommen:}	
\begin{itemize}
	\item[-] \textbf{pairfam:} \textit{sample of 12,402 randomly selected anchor persons of the three birth cohorts 1991-93, 1981-83, and 1971-73; no information about breastfeeding}
	\item[-]
\end{itemize}
\end{itemize}




%\item \textbf{Outcomes: }\vspace{-1em}
%\begin{itemize}
%	\item[-] HC index \newline read up in Bailey, Rossin-Slater, Walker and Hoynes (WIP) [EALE 	Keynote HH] enthält: complete schooling, professional degree, professional occupation,...
%\end{itemize}








\end{document}