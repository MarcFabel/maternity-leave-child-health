\documentclass[11pt,a4paper]{article}
\usepackage[utf8]{inputenc}
\usepackage{amsmath}
\usepackage{amsfonts}
\usepackage{amssymb}
\usepackage{graphicx}
\usepackage[left=3cm,right=3cm,top=2cm,bottom=2cm]{geometry}
\usepackage{color,soul}			% ermöglicht das highlighten von text
\author{Marc Fabel}
\title{To Do Liste}
\date{\flushleft{Last revision of the document: \today}}
\begin{document}
\maketitle




%%%%%%%%%%%       Deadlines         %%%%%%%%%%%%%%%%%%%%%%%%%%
\textbf{Deadlines:}\newline
\begin{tabular}{cccc}
\hline 
Event & deadline & time	& angemeldet \\ 
\hline 
Royal Economic Society (Sussex) &  08.10.     &26.03. - 28.03. & \checkmark \\
dggö health econometrics workshop & 15.10. & 07 \& 08 Dezember 2017 \\ 
IAB (Nürnberg) & 15.10. & 01-02 Februar 2018 \\
SOLE (Toronto) & 31.10. & 04 \& 05 Mai 2018\\
IZA world labor (Berlin) & 15.11. & 28-29.06\\
dggö Tagung (Hamburg)& 15.11. & 05 \& 06 März 2018 \\ 
\hline 
\end{tabular} 
%%%%%%%%%%%%%%%%%%%%%%%%%%%%%%%%%%%%%%%%%%%%%%%%%
\begin{itemize}
\item[-] Do-files von generated variables reinigen $\rightarrow$ in prepare Teil auslagern
\end{itemize}



\begin{enumerate}

\item \textbf{Pooling}:\newline Ist das  überhaupt in Ordnung? Impliziert, dass die Effekte über die Jahre diesselben sind. Alternativ: für ausgewählte	Jahre zeigen


\item \textbf{Outcomes: }\vspace{-1em}
\begin{itemize}
\item[-]\hl{Metabolic Syndrome} \newline read up Hoynes, Schanzenbach \& Almond (2016, AER) [EALE Keynote HH], welche Diagnosen verstecken sich hinter dem Begriff: high blood pressure, obesity, heart disease, heart attack,...
\item[-] HC index \newline read up in Bailey, Rossin-Slater, Walker and Hoynes (WIP) [EALE Keynote HH] enthält: complete schooling, professional degree, professional occupation,...
\end{itemize}



\item \textbf{Identification:}\vspace{-1em}
\begin{itemize}
\item[-]\textbf{check parallel trends}: Granger style regression/event study/leads and lags test
\end{itemize}


\item \textbf{\hl{Reduced form}}\newline 
RD graphs:\vspace{-1em}
\begin{enumerate}
\item treat und control in einem graph
\item pro Jahr
\item mit absoluten Zahlen
\item mit ratios, die im Nenner die ursprünglichen Geburtszahlen beinhalten
\item \hl{'moving averages'}
\item nonparametric ? 
\item with residuals ? 
\end{enumerate}
\item \textbf{Parental covariate balance:}\newline sollte unabhängig davon sein ob das Kind noch im Haushalt lebt. Andere Datenquellen durchsuchen (z.B. IPUMS, oder SHARE/SOEP)
\begin{quote}
\texttt{[Natalia Email 19.09.]}\newline
Beispiel: MZ 1980 

1.	Info zu Geburtsmonat   Geboren in den Monaten
Januar bis April = 1
Mai bis Dezember  =2 

$\rightarrow$	Zumindest kann man, da reform cut-off May 1, 1979 genau der geburtsmonatseinteilung entspricht, in pre and post-kids/families unterteilen.

2.	Datenerhebung: 
Beginn 21.04.1980 
Ende 27.04.1980 
3. Variable ef66 Alter von 00-99.

$\rightarrow$	Vielleicht kann man aus ef66 und interview-datum noch weitere infos ziehen….
\end{quote}


\item  \textbf{spatial heterogeneity nach counterfactual mode of care [Rafael Lalive]:} \newline eligibility (wo haben damals die meisten Frauen gearbeitet? rough approx. rural-urban); vielleicht gleich einen spatial treatment intensity index erstellen:\newline $inten_{r,t}=f(FLFP_{r,1979}, Share_foreigners_{r,t}$, ) \newline
with region $r$ at time $t$. 
\begin{itemize}
\item[-] \textbf{\hl{FLFP in 1979:}}\newline Sefan Wöhrmüller Daten oder shares aus MZ ziehen (Marc Piopiunik) Bastian 16.OKTOBER!
\end{itemize}


\item \textbf{Cost-Benefit Analysis [Rafael Lalive]}\newline Back-of-the-envelope calculation, welche Kosten werden dem Gesundheitssystem eingespaart durch X weniger Diagnosen von Y, in relation setzen zu was hat das damals gekostet. Argumentieren entlang der Linien von Hillary Hoynes: benefits materialize way later and in other dimensions which were early goals in mind

\item \textbf{KKH Daten Main Table}\newline dependent mean (sd) auf control anpassen

\item \textbf{Bandwidth}\newline 2 Monate als Standardfall nehmen? \newline
Siehe auch Baker \& Milligan, $N=8$ [Natalia]

\item \textbf{Effect size:}\vspace{-1em}
\begin{itemize}
\item[-]in sd units (HH: metabolic syndrome and footstamps: 0.4 sd , corresponds to a famine shock) [EALE]
\item[-] How are the results \textbf{comparable to other studies} relevant in that field (as they look at similar reforms at roughly the same times)? What do they find, what do we find? Differences in scales, why?[Verein, Lena Janys]
\item[-] effects too large? how do we explain these given that other studies find little/nothing (usually: introduction (expansion) positive (no) effects) [D Kühnle, Verein]
\item[-] Natalia Email [19.09.2017]\vspace{-0.5em}
\begin{itemize}
\item Wichtiges Argument, auch für die dt PL reform: betrifft Kinder bzw Babys in sehr jungem Alter.
\item literature: Barnett (2011) Science; Nores \& Barnett (2010) 
\end{itemize}


\end{itemize}


\item \textbf{$\Delta Y$:}\newline
Unterschied pre/post cutoff, zumindest im descriptive teil


\item \textbf{robustness section:}\vspace{-1em}
\begin{itemize}
\item[-] ratios mit den Geburtszahlen im Nenner (DESTATIS): $\frac{\#\ of diagnosis}{\#\ people born_m}$
\item[-] cummulativ fälle im Zähler (über alle jahre) und people born, 
\end{itemize}


\item \textbf{\hl{Life-course perspective:}}\vspace{-1em}
\begin{itemize}
\item[-]andere Spezifikation: pooled lassen und nicht auf das Alter beschränken sondern vielmehr den ITT mit year dummies interagieren: \newline
$\gamma\  treat \times after \times yeardummy$
\item[-] nicht über survey jahr sondern Alter der patienten $\rightarrow$ treatment und control group sollten gleich alt sein
\end{itemize}


\item \textbf{Effekte durch die reform:}\newline
Look up: does reform induce substitution for \textbf{formal/informal} care? [Verein, commentator mentioned that he found positive effects for Austria] 


\item \textbf{P-values/power:}\vspace{-1em}
\begin{itemize}
\item[-] Nothing found for Micro Census: is that an issue of too few observations? Carry out \textbf{power calculations}. This should not be difficult as the size of the treatment effect is known from the hospital registry data. [Verein]
	\item[-] Correction of \textbf{p-values}: adjust for multiple outcomes (\texttt{Bonferroni correction}?). The issue is that 1 out of 20 outcome is significant by pure chance. This point was mentioned in association to significant coefficient of Dtrack2. [Verein, Lena Janys]
\end{itemize}


\item \textbf{Channels:}\vspace{-1em}
\begin{itemize}
\item[-] Maternal health outcomes
\end{itemize}


\item \textbf{Econometric specification:}\vspace{-1em}
\begin{itemize}
\item[-] include gender in vector of controls
\end{itemize}


\item \textbf{Other data/outcomes:}\vspace{-1 em}
\begin{itemize}
\item[-] hospital admission very extreme $\rightarrow$ less extreme outcomes? [D. Kühnle] RWI HI data
\item[-] INKAR daten (Indikatoren und Karten zur Raum- und Stadtentwicklung)\newline kann ich da etwas rausfinden für variation of FLFP in late 1970s, immigrants 
\item[-] Health of mothers: \vspace{-.5em}
\begin{enumerate}
\item RKI: aber dort sind die Daten – zumindest für die 1979 Reform – wahrscheinclih eher nicht geeignet [Natalia]
\item  Nationale Kohorte – falls hier die genaue Geburtenhistorie abgefragt wird, könnte man hier evtl Glück haben. Den fragebogen konnte ich auf die Schnelle nicht auf der Homepage finden. \texttt{http://nako.de/}
\end{enumerate}
\end{itemize}


\item \textbf{"Hopsital"}\newline What is defined as a hopsital, read up in documentation. wide definition


\item \textbf{Framing im paper:} \vspace{-1em}
\begin{itemize}
\item[-] fetal origin hypothesis (Barker, 1990): chronic conditions later in life can be attributed back to in-utero and early-childhood setting
\item[-] LR effects of policies have to be taken into account, were probably not the original goal when dsigning the policy instrument, larger and in different dimensions than originally thought [HH, EALE keynote]
\item[-] differentiation of treatment (was ist bei der refrom passiert?). 1. time allocation due to reform (figures is DS und SL) und 2. counterfactual mode of care: maternal care (the moms would have stayed at home anyway) $\rightarrow$ they benefit from additional job protection and 750 DM.
(KÜhnle \& Hübner paper).
\item[-] What is the difference between our work and to DS und LS? Different populations (zeros for all women with less than 20hrs).
\end{itemize}

\item \textbf{Reading list:}
\begin{itemize}
\item[-] Molly Schnell - The role of Physician Behavior in the Opiod Epidemic [EALE]\newline
good job market paper, kleines theoretisches model nebenbei
\item[-]  Dahmann \& Schnitzlein: The Protective (?) Effect of Education on Mental Health   \newline
 Nice motivation why mental health problems are so important to focus on 
 \item[-] Paper [Natalia, before EALE] Maternal stress and child outcomes, Journal of Human Resources

\end{itemize}

\end{enumerate}


















\end{document}