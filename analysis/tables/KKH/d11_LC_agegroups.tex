 \begin{table}[H] \centering \begin{threeparttable} \caption{Life-course approach - Table format} {\def\sym#1{\ifmmode^{#1}\else\(^{#1}\)\fi} \begin{tabular}{l*{5}{c}} \toprule \multicolumn{5}{l}{Dep. variable: \textbf{Diseases of the skin and subcutaneous tissue}} \\ & \multicolumn{4}{c}{Estimation window} \\ \cmidrule(lr){2-5}
            &\multicolumn{1}{c}{(1)}&\multicolumn{1}{c}{(2)}&\multicolumn{1}{c}{(3)}&\multicolumn{1}{c}{(4)}\\
            &\multicolumn{1}{c}{Age 17-21}&\multicolumn{1}{c}{Age 22-26}&\multicolumn{1}{c}{Age 27-31}&\multicolumn{1}{c}{Age 32-35}\\
\midrule
 \multicolumn{5}{l}{\emph{Panel A. Average causal effects}} \\ Abs. numbers        &       11.37         &       17.50\sym{**} &       15.80\sym{*}  &       8.333         \\
                    &     (9.336)         &     (7.420)         &     (9.509)         &     (9.554)         \\
 Ratio fertility     &      0.0567         &       0.186         &       0.152         &     -0.0149         \\
                    &     (0.188)         &     (0.171)         &     (0.184)         &     (0.247)         \\
 Ratio population    &       0.149         &       0.124         &    -0.00162         \\
                    &     (0.269)         &     (0.139)         &     (0.188)         \\
 Cum. numbers        &       46.50         &       116.2\sym{**} &       192.9\sym{***}&       252.5\sym{***}\\
                    &     (33.06)         &     (45.34)         &     (51.52)         &     (44.17)         \\
 Cum. ratio          &       0.235         &       0.734         &       1.478         &       1.912\sym{*}  \\
                    &     (0.631)         &     (0.974)         &     (0.931)         &     (1.127)         \\
 \midrule\multicolumn{5}{l}{\emph{Panel B. Treatment effect heterogeneity - Women}} \\ Abs. numbers        &       3.600         &       7.633         &       5.700         &       1.500         \\
                    &     (6.342)         &     (5.853)         &     (5.948)         &     (5.624)         \\
 Ratio fertility     &     -0.0342         &       0.158         &      0.0960         &     -0.0972         \\
                    &     (0.251)         &     (0.299)         &     (0.229)         &     (0.265)         \\
 Ratio population    &       0.160         &      0.0784         &     -0.0640         \\
                    &     (0.318)         &     (0.171)         &     (0.196)         \\
 Cum. numbers        &       17.37         &       41.63         &       79.27\sym{**} &       90.33\sym{***}\\
                    &     (24.20)         &     (33.14)         &     (39.20)         &     (34.40)         \\
 Cum. ratio          &     -0.0477         &      0.0403         &       0.933         &       0.713         \\
                    &     (0.971)         &     (1.539)         &     (1.749)         &     (1.684)         \\
 \midrule\multicolumn{5}{l}{\emph{Panel C. Treatment effect heterogeneity - Men}} \\ Abs. numbers        &       7.767\sym{*}  &       9.867\sym{*}  &       10.10         &       6.833         \\
                    &     (4.352)         &     (5.479)         &     (7.042)         &     (5.764)         \\
 Ratio fertility     &       0.144         &       0.219         &       0.213         &      0.0690         \\
                    &     (0.180)         &     (0.187)         &     (0.279)         &     (0.282)         \\
 Ratio population    &       0.144         &       0.176         &      0.0661         \\
                    &     (0.302)         &     (0.216)         &     (0.221)         \\
 Cum. numbers        &       29.13\sym{*}  &       74.53\sym{***}&       113.6\sym{**} &       162.2\sym{**} \\
                    &     (15.65)         &     (25.07)         &     (52.41)         &     (67.63)         \\
 Cum. ratio          &       0.504         &       1.409\sym{*}  &       2.049         &       3.135         \\
                    &     (0.589)         &     (0.837)         &     (1.835)         &     (2.606)         \\
 
\bottomrule \end{tabular} } \begin{tablenotes} \item \scriptsize \emph{Notes:} Clustered standard errors in parentheses. All regression are run with CG2 (i.e. the cohort prior to the reform) and with month-of-birth FEs. Ratios indicate cases per thousand; either approximated population (with weights coming from the original fertility distribution) or original number of births. Raqtio population muss eins nach rechts gerückt werden \end{tablenotes} \end{threeparttable} \end{table} 
