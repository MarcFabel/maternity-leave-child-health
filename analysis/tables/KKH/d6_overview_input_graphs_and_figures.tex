%---------------------------------
% INPUT FOR VARIABLE: d6
%---------------------------------
\subsection{d6}
% RD overview
\begin{landscape}
\begin{figure}[H]
	\centering
	\begin{minipage}{.95\linewidth}
	\includegraphics[width=\linewidth]{rd_d6_overview_panel1}
	{\scriptsize \emph{Notes:} The figures show monthly RD plots with averages obtained from a bin width of one month. The solid vertical line divides pre- and post-reform regime. The averages are taken over the period of at most 1995-2014. \par}
\end{minipage}
\end{figure}
\end{landscape}
\begin{landscape}
\begin{figure}[H]
	\centering
\begin{minipage}{.95\linewidth}
	\includegraphics[width=\linewidth]{rd_d6_overview_panel2}
	{\scriptsize \emph{Notes:} The figures show monthly RD plots with a moving average window width of 3 months. The solid vertical line divides pre- and post-reform regime. The averages are taken over the period of at most 1995-2014. \par}
\end{minipage}
\end{figure}
\end{landscape}
%---------------------------------
% TABELLEN
 \begin{table}[H] \begin{threeparttable} \centering \caption{Dep. variable: \textbf{Diseases of the nervous system}} {\def\sym#1{\ifmmode^{#1}\else\(^{#1}\)\fi} \begin{tabular}{l*{8}{c}} \toprule & \multicolumn{7}{c}{Estimation window} \\ \cmidrule(lr){2-8}
            &\multicolumn{1}{c}{(1)}&\multicolumn{1}{c}{(2)}&\multicolumn{1}{c}{(3)}&\multicolumn{1}{c}{(4)}&\multicolumn{1}{c}{(5)}&\multicolumn{1}{c}{(6)}&\multicolumn{1}{c}{(7)}\\
            &\multicolumn{1}{c}{1M}&\multicolumn{1}{c}{2M}&\multicolumn{1}{c}{3M}&\multicolumn{1}{c}{4M}&\multicolumn{1}{c}{5M}&\multicolumn{1}{c}{6M}&\multicolumn{1}{c}{Donut}\\
\midrule
 \multicolumn{8}{l}{\emph{Panel A. Average causal effects}} \\ Abs. numbers        &       9.900\sym{***}&       4.375         &       5.250         &       7.037\sym{*}  &       7.800\sym{**} &       10.02\sym{***}&       10.05\sym{***}\\
                    &  (5.44e-14)         &     (2.538)         &     (3.134)         &     (3.753)         &     (3.041)         &     (2.955)         &     (3.263)         \\
 Ratio fertility     &       0.178\sym{***}&      0.0367         &     -0.0167         &     0.00450         &      0.0105         &     0.00791         &     -0.0261         \\
                    &  (6.27e-16)         &    (0.0875)         &    (0.0753)         &    (0.0590)         &    (0.0531)         &    (0.0550)         &    (0.0594)         \\
 Ratio population    &       0.315\sym{***}&      0.0578         &     0.00587         &      0.0392         &      0.0728         &      0.0815         &      0.0348         \\
                    &  (7.90e-16)         &     (0.103)         &    (0.0923)         &    (0.0750)         &    (0.0711)         &    (0.0647)         &    (0.0719)         \\
 Ratio fert(03-14)   &       0.354\sym{***}&    -0.00692         &     -0.0841         &     -0.0261         &      0.0156         &      0.0164         &     -0.0510         \\
                    &  (7.96e-16)         &     (0.151)         &     (0.135)         &     (0.109)         &    (0.0976)         &    (0.0877)         &    (0.0937)         \\
 Cum. numbers        &       81.05\sym{***}&       78.12\sym{***}&       81.08\sym{***}&       82.41\sym{***}&       78.79\sym{***}&       93.66\sym{***}&       96.18\sym{***}\\
                    &  (1.00e-12)         &     (14.90)         &     (13.88)         &     (27.10)         &     (21.80)         &     (22.54)         &     (25.30)         \\
 Cum. ratio          &       1.316\sym{***}&       1.237\sym{***}&       0.897\sym{***}&       0.728\sym{***}&       0.632\sym{**} &       0.605\sym{*}  &       0.462         \\
                    &  (1.42e-14)         &     (0.341)         &     (0.267)         &     (0.237)         &     (0.259)         &     (0.320)         &     (0.376)         \\
 \midrule\multicolumn{8}{l}{\emph{Panel B. Treatment effect heterogeneity - Women}} \\ Abs. numbers        &       5.450\sym{***}&       6.750\sym{**} &       4.983\sym{**} &       4.525         &       4.280\sym{*}  &       5.425\sym{**} &       5.420\sym{***}\\
                    &  (2.66e-14)         &     (2.640)         &     (2.217)         &     (2.803)         &     (2.328)         &     (2.008)         &     (1.578)         \\
 Ratio fertility     &       0.183\sym{***}&       0.221         &      0.0769         &      0.0420         &     0.00870         &    -0.00703         &     -0.0451         \\
                    &  (1.56e-15)         &     (0.147)         &     (0.118)         &     (0.106)         &    (0.0873)         &    (0.0738)         &    (0.0545)         \\
 Ratio population    &       0.226\sym{***}&       0.147         &      0.0423         &      0.0312         &      0.0522         &      0.0886         &      0.0612         \\
                    &  (1.30e-15)         &     (0.146)         &     (0.132)         &     (0.129)         &     (0.103)         &    (0.0879)         &    (0.0667)         \\
 Ratio fert(03-14)   &       0.270\sym{***}&       0.217         &       0.120         &       0.101         &       0.119         &       0.156\sym{*}  &       0.133\sym{*}  \\
                    &  (7.86e-16)         &     (0.116)         &     (0.115)         &     (0.121)         &    (0.0968)         &    (0.0823)         &    (0.0662)         \\
 Cum. numbers        &       35.50\sym{***}&       66.90\sym{***}&       50.15\sym{***}&       43.04\sym{*}  &       34.49\sym{*}  &       33.88\sym{**} &       33.55\sym{**} \\
                    &  (5.52e-13)         &     (17.52)         &     (13.54)         &     (20.55)         &     (18.84)         &     (15.63)         &     (15.90)         \\
 Cum. ratio          &       1.047\sym{***}&       2.232\sym{*}  &       0.953         &       0.497         &     -0.0824         &      -0.724         &      -1.078         \\
                    &  (2.74e-14)         &     (0.983)         &     (0.877)         &     (0.796)         &     (0.734)         &     (0.693)         &     (0.740)         \\
 \midrule\multicolumn{8}{l}{\emph{Panel C. Treatment effect heterogeneity - Men}} \\ Abs. numbers        &       4.450\sym{***}&      -2.375         &       0.267         &       2.513         &       3.520         &       4.600\sym{*}  &       4.630         \\
                    &  (2.76e-14)         &     (2.883)         &     (2.638)         &     (2.278)         &     (2.198)         &     (2.256)         &     (2.689)         \\
 Ratio fertility     &       0.171\sym{***}&      -0.140         &      -0.107         &     -0.0318         &     0.00985         &      0.0186         &     -0.0119         \\
                    &  (1.72e-15)         &     (0.120)         &    (0.0973)         &    (0.0828)         &    (0.0928)         &    (0.0940)         &     (0.110)         \\
 Ratio population    &       0.348\sym{***}&      -0.113         &      -0.115         &     -0.0286         &      0.0220         &     0.00234         &     -0.0669         \\
                    &  (4.67e-16)         &     (0.194)         &     (0.132)         &     (0.112)         &     (0.113)         &     (0.106)         &     (0.112)         \\
 Ratio fert(03-14)   &       0.399\sym{***}&      -0.202         &      -0.217         &     -0.0895         &     -0.0250         &     -0.0631         &      -0.156         \\
                    &  (6.37e-16)         &     (0.244)         &     (0.173)         &     (0.146)         &     (0.144)         &     (0.134)         &     (0.145)         \\
 Cum. numbers        &       45.55\sym{***}&       11.23         &       30.93\sym{*}  &       39.38\sym{**} &       44.30\sym{**} &       59.78\sym{***}&       62.63\sym{**} \\
                    &  (5.56e-13)         &     (13.36)         &     (16.20)         &     (14.63)         &     (16.52)         &     (19.39)         &     (23.32)         \\
 Cum. ratio          &       1.447\sym{***}&       0.355         &       0.526         &       0.567\sym{*}  &       0.716         &       1.032         &       0.948         \\
                    &  (2.93e-15)         &     (0.476)         &     (0.377)         &     (0.297)         &     (0.521)         &     (0.602)         &     (0.716)         \\
 
\bottomrule \end{tabular} } \begin{tablenotes} \item \scriptsize \emph{Notes:} Clustered standard errors in parentheses. All regression are run with CG2 (i.e. the cohort prior to the reform) and with month-of-birth FEs. Ratios indicate cases per thousand; either approximated population (with weights coming from the original fertility distribution) or original number of births. \end{tablenotes} \end{threeparttable} \end{table} 

 \begin{table}[H] \begin{threeparttable} \centering \caption{Robustness with respect to the choice of \texttt{control group}} {\def\sym#1{\ifmmode^{#1}\else\(^{#1}\)\fi} \begin{tabular}{l*{10}{c}} \toprule & \multicolumn{9}{c}{Dependent variable: \textbf{Diseases of the nervous system}} \\ \cmidrule(lr){2-10}
            &\multicolumn{3}{c}{Average Causal Effects}&\multicolumn{3}{c}{Women}             &\multicolumn{3}{c}{Men}               \\\cmidrule(lr){2-4}\cmidrule(lr){5-7}\cmidrule(lr){8-10}
            &\multicolumn{1}{c}{(1)}&\multicolumn{1}{c}{(2)}&\multicolumn{1}{c}{(3)}&\multicolumn{1}{c}{(4)}&\multicolumn{1}{c}{(5)}&\multicolumn{1}{c}{(6)}&\multicolumn{1}{c}{(7)}&\multicolumn{1}{c}{(8)}&\multicolumn{1}{c}{(9)}\\
            &\multicolumn{1}{c}{C2}&\multicolumn{1}{c}{C1+C2}&\multicolumn{1}{c}{C1-C3}&\multicolumn{1}{c}{C2}&\multicolumn{1}{c}{C1+C2}&\multicolumn{1}{c}{C1-C3}&\multicolumn{1}{c}{C2}&\multicolumn{1}{c}{C1+C2}&\multicolumn{1}{c}{C1-C3}\\
\midrule
 \multicolumn{10}{l}{\emph{Panel A. 2 Month bandwidth}} \\ Abs. numbers        &      -1.056         &      -5.556         &      -4.574         &       4.167         &       2.806         &       3.704         &      -5.222         &      -8.361         &      -8.278         \\
                    &     (7.827)         &     (7.681)         &     (8.355)         &     (4.585)         &     (4.226)         &     (5.311)         &     (6.294)         &     (6.329)         &     (5.657)         \\
 Ratio fertility     &      0.0367         &      0.0459         &       0.105         &       0.221         &       0.274         &       0.353         &      -0.140         &      -0.172         &      -0.131         \\
                    &    (0.0875)         &     (0.143)         &     (0.155)         &     (0.147)         &     (0.189)         &     (0.207)         &     (0.120)         &     (0.153)         &     (0.144)         \\
 Ratio fertility     &    -0.00594         &     -0.0240         &    -0.00230         &       0.151         &       0.159         &       0.190         &      -0.155         &      -0.198         &      -0.185         \\
                    &     (0.119)         &     (0.127)         &     (0.147)         &     (0.142)         &     (0.146)         &     (0.174)         &     (0.176)         &     (0.179)         &     (0.177)         \\
 \midrule\multicolumn{10}{l}{\emph{Panel B. 4 Month bandwidth}} \\ Abs. numbers        &       6.389         &       3.389         &       2.500         &       4.611         &       3.625         &       3.241         &       1.778         &      -0.236         &      -0.741         \\
                    &     (7.955)         &     (7.940)         &     (7.562)         &     (5.085)         &     (4.390)         &     (4.439)         &     (4.367)         &     (5.015)         &     (4.615)         \\
 Ratio fertility     &     0.00450         &      0.0140         &      0.0668         &      0.0420         &       0.111         &       0.175         &     -0.0318         &     -0.0789         &     -0.0364         \\
                    &    (0.0590)         &    (0.0982)         &     (0.119)         &     (0.106)         &     (0.133)         &     (0.155)         &    (0.0828)         &     (0.109)         &     (0.115)         \\
 Ratio fertility     &      0.0266         &      0.0346         &      0.0331         &      0.0867         &       0.113         &       0.113         &     -0.0301         &     -0.0396         &     -0.0420         \\
                    &    (0.0891)         &     (0.101)         &     (0.116)         &     (0.130)         &     (0.121)         &     (0.132)         &     (0.109)         &     (0.130)         &     (0.141)         \\
 \midrule\multicolumn{10}{l}{\emph{Panel C. 6 Month bandwidth}} \\ Abs. numbers        &       10.02\sym{***}&       6.975\sym{*}  &       5.767         &       5.425\sym{**} &       5.004\sym{**} &       4.881\sym{**} &       4.600\sym{*}  &       1.971         &       0.886         \\
                    &     (2.955)         &     (3.750)         &     (3.498)         &     (2.008)         &     (2.301)         &     (2.363)         &     (2.256)         &     (2.642)         &     (2.380)         \\
 Ratio population    &      0.0188         &     0.00383         &     0.00330         &       0.113         &      0.0840         &      0.0608         &     -0.0698         &     -0.0798         &     -0.0596         \\
                    &     (0.120)         &     (0.125)         &     (0.129)         &     (0.162)         &     (0.147)         &     (0.150)         &     (0.189)         &     (0.186)         &     (0.190)         \\
 Ratio population    &      0.0288         &      0.0272         &      0.0323         &      0.0886         &       0.101         &       0.114         &     -0.0340         &     -0.0496         &     -0.0528         \\
                    &    (0.0705)         &    (0.0795)         &    (0.0878)         &    (0.0879)         &    (0.0951)         &    (0.0995)         &     (0.109)         &     (0.115)         &     (0.122)         \\
 \midrule\multicolumn{10}{l}{\emph{Panel D. Donut specification}} \\ Abs. numbers        &       10.05\sym{***}&       7.795\sym{*}  &       6.297\sym{*}  &       5.420\sym{***}&       4.970\sym{**} &       4.707\sym{***}&       4.630         &       2.825         &       1.590         \\
                    &     (3.263)         &     (3.940)         &     (3.353)         &     (1.578)         &     (2.074)         &     (1.693)         &     (2.689)         &     (2.922)         &     (2.730)         \\
 Ratio fertility     &     -0.0261         &     -0.0537         &     0.00111         &     -0.0451         &     -0.0424         &      0.0331         &     -0.0119         &     -0.0682         &     -0.0332         \\
                    &    (0.0594)         &    (0.0642)         &    (0.0838)         &    (0.0545)         &    (0.0852)         &    (0.0900)         &     (0.110)         &     (0.104)         &     (0.119)         \\
 Ratio fertility     &      0.0537         &      0.0764         &      0.0525         &       0.123\sym{*}  &       0.139\sym{*}  &       0.116\sym{*}  &     -0.0132         &      0.0158         &    -0.00865         \\
                    &    (0.0817)         &    (0.0859)         &    (0.0998)         &    (0.0674)         &    (0.0697)         &    (0.0680)         &     (0.118)         &     (0.130)         &     (0.149)         \\
 
\bottomrule \end{tabular} } \begin{tablenotes} \item \scriptsize \emph{Notes:} Clustered standard errors in parentheses. All regressions contain Birthmonth FE. Ratios indicate cases per thousand; either approximated population or original number of births. \end{tablenotes} \end{threeparttable} \end{table} 

%---------------------------------
% Life-course figure (Panel1)
\begin{landscape}
\begin{figure}[H]
\centering
\begin{minipage}{.9\linewidth}
\includegraphics[width=\linewidth]{lc_d6_overview_panel1}
{\scriptsize \emph{Notes:} The figures depict DDRD estimates and 90\% confidence intervals over the life-course. The years are harmonized such that the cohorts are in the same age when they are compared. All regressions are carried out with month-of-birth FE and make use of clustered standard errors. Furthermore, we used a bandwidth of half a year and only the control cohort that was born one year prior to the reform. Ratios indicate cases per thousand; using in the denominator the approximated population (with weights coming from the original fertility distribution) or original number of births. \par}
\end{minipage}
\end{figure}
\end{landscape}
%---------------------------------
% Life-course figure (Panel2)
\begin{landscape}
\begin{figure}[H]
\centering
\begin{minipage}{.9\linewidth}
\includegraphics[width=\linewidth]{lc_d6_overview_panel2}
{\scriptsize \emph{Notes:} The figures depict DDRD estimates and 90\% confidence intervals over the life-course. The years are harmonized such that the cohorts are in the same age when they are compared. All regressions are carried out with month-of-birth FE and make use of clustered standard errors. Furthermore, we used a bandwidth of half a year. Ratios indicate cases per thousand; using in the denominator the approximated population (with weights coming from the original fertility distribution) or original number of births. \par}
\end{minipage}
\end{figure}
\end{landscape}
%---------------------------------
% Life-course (panel 3 - 6)
\begin{figure}[H]%\vspace*{-2cm}
	\centering
	\includegraphics[width=.9\linewidth]{lc_d6_overview_panel3}
	\includegraphics[width=.9\linewidth]{lc_d6_overview_panel4}
\end{figure}
\begin{figure}[H]
	\centering	
	\includegraphics[width=.97\linewidth]{lc_d6_overview_panel5}
	\includegraphics[width=.97\linewidth]{lc_d6_overview_panel6}
\end{figure}
% Life-course TABLE Format
 \begin{table}[H] \centering \begin{threeparttable} \caption{Life-course approach - Table format} {\def\sym#1{\ifmmode^{#1}\else\(^{#1}\)\fi} \begin{tabular}{l*{5}{c}} \toprule \multicolumn{5}{l}{Dep. variable: \textbf{Diseases of the nervous system}} \\ & \multicolumn{4}{c}{Estimation window} \\ \cmidrule(lr){2-5}
            &\multicolumn{1}{c}{(1)}&\multicolumn{1}{c}{(2)}&\multicolumn{1}{c}{(3)}&\multicolumn{1}{c}{(4)}\\
            &\multicolumn{1}{c}{Age 17-21}&\multicolumn{1}{c}{Age 22-26}&\multicolumn{1}{c}{Age 27-31}&\multicolumn{1}{c}{Age 32-35}\\
\midrule
 \multicolumn{5}{l}{\emph{Panel A. Average causal effects}} \\ Abs. numbers        &       2.533         &       20.53\sym{***}&       13.07         &       13.12         \\
                    &     (5.342)         &     (5.446)         &     (10.20)         &     (12.58)         \\
 Ratio fertility     &     -0.0919         &       0.238         &      0.0374         &     -0.0190         \\
                    &     (0.129)         &     (0.149)         &     (0.191)         &     (0.256)         \\
 Ratio population    &       0.116         &      0.0420         &     0.00134         \\
                    &     (0.150)         &     (0.150)         &     (0.203)         \\
 Cum. numbers        &      -4.333         &       74.03\sym{*}  &       169.5\sym{***}&       215.1\sym{**} \\
                    &     (30.43)         &     (38.50)         &     (50.38)         &     (92.63)         \\
 Cum. ratio          &      -0.722         &      0.0343         &       0.957         &       0.682         \\
                    &     (0.617)         &     (1.055)         &     (1.312)         &     (2.020)         \\
 \midrule\multicolumn{5}{l}{\emph{Panel B. Treatment effect heterogeneity - Women}} \\ Abs. numbers        &      -2.367         &       10.60\sym{*}  &       8.833         &       10.71\sym{*}  \\
                    &     (3.358)         &     (5.934)         &     (9.153)         &     (5.943)         \\
 Ratio fertility     &      -0.274         &       0.221         &       0.106         &       0.115         \\
                    &     (0.178)         &     (0.277)         &     (0.336)         &     (0.275)         \\
 Ratio population    &       0.183         &      0.0931         &       0.102         \\
                    &     (0.181)         &     (0.252)         &     (0.203)         \\
 Cum. numbers        &      -13.13         &       15.23         &       72.60\sym{*}  &       102.3         \\
                    &     (14.38)         &     (35.11)         &     (42.80)         &     (73.16)         \\
 Cum. ratio          &      -1.285\sym{*}  &      -1.158         &      0.0326         &      -0.107         \\
                    &     (0.724)         &     (1.828)         &     (1.972)         &     (3.086)         \\
 \midrule\multicolumn{5}{l}{\emph{Panel C. Treatment effect heterogeneity - Men}} \\ Abs. numbers        &       4.900         &       9.933\sym{*}  &       4.233         &       2.417         \\
                    &     (6.026)         &     (5.411)         &     (6.688)         &     (10.74)         \\
 Ratio fertility     &      0.0787         &       0.250         &     -0.0319         &      -0.150         \\
                    &     (0.247)         &     (0.250)         &     (0.291)         &     (0.421)         \\
 Ratio population    &      0.0455         &     -0.0122         &      -0.102         \\
                    &     (0.254)         &     (0.231)         &     (0.342)         \\
 Cum. numbers        &       8.800         &       58.80\sym{*}  &       96.87\sym{**} &       112.7         \\
                    &     (30.53)         &     (33.17)         &     (47.94)         &     (83.29)         \\
 Cum. ratio          &      -0.196         &       1.142         &       1.791         &       1.369         \\
                    &     (1.183)         &     (1.454)         &     (2.259)         &     (3.625)         \\
 
\bottomrule \end{tabular} } \begin{tablenotes} \item \scriptsize \emph{Notes:} Clustered standard errors in parentheses. All regression are run with CG2 (i.e. the cohort prior to the reform) and with month-of-birth FEs. Ratios indicate cases per thousand; either approximated population (with weights coming from the original fertility distribution) or original number of births. Raqtio population muss eins nach rechts gerückt werden \end{tablenotes} \end{threeparttable} \end{table} 

%---------------------------------
% PLACEBO EXERCISES
\newpage
\begin{landscape}
\begin{figure}[H]
	\centering
    \begin{minipage}{.9\linewidth}
	\includegraphics[width=\linewidth]{placebo_graph_d6.pdf}
    {\scriptsize \emph{Notes:} The figures depict DDRD estimates and 95\% confidence intervals when the treatment cohort is shifted over time. The date on the abscissa indicates the starting date of the treated.  All regressions are carried out with month-of-birth FE and make use of clustered standard errors. Furthermore, we used a bandwidth of half a year. Ratios indicate cases per thousand; using in the denominator the approximated population (with weights coming from the original fertility distribution) or original number of births. \par}
    \end{minipage}
\end{figure}
\end{landscape}
 \begin{table}[H] \centering \begin{threeparttable} \caption{Placebo 1 (CONTROL1 ist TREAT) } {\def\sym#1{\ifmmode^{#1}\else\(^{#1}\)\fi} \begin{tabular}{l*{4}{c}} \toprule \multicolumn{4}{l}{Dep. variable: \textbf{Diseases of the nervous system}} \\ & \multicolumn{3}{c}{Choice of control group} \\ \cmidrule(lr){2-4}
            &\multicolumn{1}{c}{(1)}&\multicolumn{1}{c}{(2)}&\multicolumn{1}{c}{(3)}\\
            &\multicolumn{1}{c}{C2}&\multicolumn{1}{c}{C3}&\multicolumn{1}{c}{C2+C3}\\
\midrule
 \multicolumn{4}{l}{\emph{Panel A. Average causal effects}} \\ Abs. numbers        &       6.100\sym{**} &      -0.575         &       2.762         \\
                    &     (2.618)         &     (2.675)         &     (2.788)         \\
 Ratio fertility     &      0.0550         &       0.165\sym{**} &       0.110         \\
                    &    (0.0669)         &    (0.0636)         &    (0.0771)         \\
 Ratio population    &     0.00323         &      0.0168         &      0.0100         \\
                    &    (0.0464)         &    (0.0299)         &    (0.0497)         \\
 Cum. numbers        &       63.93\sym{*}  &       35.20         &       49.57         \\
                    &     (32.57)         &     (33.19)         &     (33.39)         \\
 Cum. ratio          &       0.662         &       2.272\sym{***}&       1.467\sym{*}  \\
                    &     (0.778)         &     (0.782)         &     (0.861)         \\
 \midrule\multicolumn{4}{l}{\emph{Panel B. Treatment effect heterogeneity - Women}} \\ Abs. numbers        &       0.842         &      0.0500         &       0.446         \\
                    &     (2.149)         &     (2.185)         &     (2.214)         \\
 Ratio fertility     &     -0.0348         &       0.190\sym{*}  &      0.0777         \\
                    &     (0.105)         &     (0.105)         &     (0.112)         \\
 Ratio population    &     -0.0245         &      0.0280         &     0.00176         \\
                    &    (0.0844)         &    (0.0597)         &    (0.0730)         \\
 Cum. numbers        &       2.375         &       30.88         &       16.63         \\
                    &     (24.24)         &     (25.41)         &     (24.81)         \\
 Cum. ratio          &      -0.604         &       2.844\sym{**} &       1.120         \\
                    &     (1.158)         &     (1.186)         &     (1.289)         \\
 \midrule\multicolumn{4}{l}{\emph{Panel C. Treatment effect heterogeneity - Men}} \\ Abs. numbers        &       5.258\sym{***}&      -0.625         &       2.317         \\
                    &     (1.447)         &     (1.680)         &     (1.670)         \\
 Ratio fertility     &       0.140\sym{*}  &       0.141\sym{**} &       0.141\sym{*}  \\
                    &    (0.0727)         &    (0.0670)         &    (0.0832)         \\
 Ratio population    &      0.0312         &     0.00592         &      0.0186         \\
                    &    (0.0643)         &    (0.0831)         &    (0.0901)         \\
 Cum. numbers        &       61.56\sym{***}&       4.317         &       32.94\sym{*}  \\
                    &     (16.83)         &     (13.00)         &     (16.77)         \\
 Cum. ratio          &       1.859\sym{**} &       1.730\sym{***}&       1.794\sym{**} \\
                    &     (0.780)         &     (0.543)         &     (0.714)         \\
 
\bottomrule \end{tabular} } \begin{tablenotes} \item \scriptsize \emph{Notes:} Clustered standard errors in parentheses. All regression are run with month-of-birth FEs and control cohort 2 is assigned with the treatment status. All regressions are carried out with a window width of half a year. \end{tablenotes} \end{threeparttable} \end{table} 

 \begin{table}[H] \centering \begin{threeparttable} \caption{Placebo 2 (CONTROL2 ist TREAT) } {\def\sym#1{\ifmmode^{#1}\else\(^{#1}\)\fi} \begin{tabular}{l*{4}{c}} \toprule \multicolumn{4}{l}{Dep. variable: \textbf{Diseases of the nervous system}} \\ & \multicolumn{3}{c}{Choice of control group} \\ \cmidrule(lr){2-4}
            &\multicolumn{1}{c}{(1)}&\multicolumn{1}{c}{(2)}&\multicolumn{1}{c}{(3)}\\
            &\multicolumn{1}{c}{C1}&\multicolumn{1}{c}{C3}&\multicolumn{1}{c}{C1+C3}\\
\midrule
 \multicolumn{4}{l}{\emph{Panel A. Average causal effects}} \\ Abs. numbers        &      -6.100\sym{**} &      -6.675\sym{**} &      -6.388\sym{*}  \\
                    &     (2.618)         &     (2.691)         &     (3.285)         \\
 Ratio fertility     &     -0.0550         &       0.110         &      0.0275         \\
                    &    (0.0669)         &    (0.0646)         &     (0.109)         \\
 Ratio population    &    -0.00323         &      0.0136         &     0.00517         \\
                    &    (0.0464)         &    (0.0578)         &    (0.0952)         \\
 Cum. numbers        &      -63.93\sym{*}  &      -28.73         &      -46.33         \\
                    &     (32.57)         &     (32.13)         &     (36.93)         \\
 Cum. ratio          &      -0.662         &       1.610\sym{*}  &       0.474         \\
                    &     (0.778)         &     (0.788)         &     (1.125)         \\
 \midrule\multicolumn{4}{l}{\emph{Panel B. Treatment effect heterogeneity - Women}} \\ Abs. numbers        &      -0.842         &      -0.792         &      -0.817         \\
                    &     (2.149)         &     (1.199)         &     (1.794)         \\
 Ratio fertility     &      0.0348         &       0.225\sym{***}&       0.130         \\
                    &     (0.105)         &    (0.0607)         &     (0.113)         \\
 Ratio population    &      0.0245         &      0.0525         &      0.0385         \\
                    &    (0.0844)         &    (0.0410)         &    (0.0779)         \\
 Cum. numbers        &      -2.375         &       28.51         &       13.07         \\
                    &     (24.24)         &     (19.12)         &     (24.25)         \\
 Cum. ratio          &       0.604         &       3.447\sym{***}&       2.026         \\
                    &     (1.158)         &     (0.960)         &     (1.414)         \\
 \midrule\multicolumn{4}{l}{\emph{Panel C. Treatment effect heterogeneity - Men}} \\ Abs. numbers        &      -5.258\sym{***}&      -5.883\sym{***}&      -5.571\sym{**} \\
                    &     (1.447)         &     (1.940)         &     (2.239)         \\
 Ratio fertility     &      -0.140\sym{*}  &    0.000902         &     -0.0696         \\
                    &    (0.0727)         &    (0.0835)         &     (0.127)         \\
 Ratio population    &     -0.0312         &     -0.0253         &     -0.0282         \\
                    &    (0.0643)         &    (0.0983)         &     (0.143)         \\
 Cum. numbers        &      -61.56\sym{***}&      -57.24\sym{***}&      -59.40\sym{***}\\
                    &     (16.83)         &     (18.25)         &     (18.91)         \\
 Cum. ratio          &      -1.859\sym{**} &      -0.130         &      -0.994         \\
                    &     (0.780)         &     (0.785)         &     (1.042)         \\
 
\bottomrule \end{tabular} } \begin{tablenotes} \item \scriptsize \emph{Notes:} Clustered standard errors in parentheses. All regression are run with month-of-birth FEs and control cohort 2 is assigned with the treatment status. All regressions are carried out with a window width of half a year. \end{tablenotes} \end{threeparttable} \end{table} 

 \begin{table}[H] \centering \begin{threeparttable} \caption{Placebo 3 (CONTROL3 ist TREAT) } {\def\sym#1{\ifmmode^{#1}\else\(^{#1}\)\fi} \begin{tabular}{l*{4}{c}} \toprule \multicolumn{4}{l}{Dep. variable: \textbf{Diseases of the nervous system}} \\ & \multicolumn{3}{c}{Choice of control group} \\ \cmidrule(lr){2-4}
            &\multicolumn{1}{c}{(1)}&\multicolumn{1}{c}{(2)}&\multicolumn{1}{c}{(3)}\\
            &\multicolumn{1}{c}{C1}&\multicolumn{1}{c}{C2}&\multicolumn{1}{c}{C1+C2}\\
\midrule
 \multicolumn{4}{l}{\emph{Panel A. Average causal effects}} \\ Abs. numbers        &       0.575         &       6.675\sym{**} &       3.625         \\
                    &     (2.675)         &     (2.691)         &     (3.796)         \\
 Ratio fertility     &      -0.165\sym{**} &      -0.110         &      -0.137\sym{*}  \\
                    &    (0.0636)         &    (0.0646)         &    (0.0781)         \\
 Ratio population    &     -0.0168         &     -0.0136         &     -0.0152         \\
                    &    (0.0299)         &    (0.0578)         &    (0.0664)         \\
 Cum. numbers        &      -35.20         &       28.73         &      -3.233         \\
                    &     (33.19)         &     (32.13)         &     (41.48)         \\
 Cum. ratio          &      -2.272\sym{***}&      -1.610\sym{*}  &      -1.941\sym{**} \\
                    &     (0.782)         &     (0.788)         &     (0.893)         \\
 \midrule\multicolumn{4}{l}{\emph{Panel B. Treatment effect heterogeneity - Women}} \\ Abs. numbers        &     -0.0500         &       0.792         &       0.371         \\
                    &     (2.185)         &     (1.199)         &     (2.044)         \\
 Ratio fertility     &      -0.190\sym{*}  &      -0.225\sym{***}&      -0.208\sym{**} \\
                    &     (0.105)         &    (0.0607)         &    (0.0924)         \\
 Ratio population    &     -0.0280         &     -0.0525         &     -0.0402         \\
                    &    (0.0597)         &    (0.0410)         &    (0.0597)         \\
 Cum. numbers        &      -30.88         &      -28.51         &      -29.70         \\
                    &     (25.41)         &     (19.12)         &     (25.59)         \\
 Cum. ratio          &      -2.844\sym{**} &      -3.447\sym{***}&      -3.146\sym{**} \\
                    &     (1.186)         &     (0.960)         &     (1.164)         \\
 \midrule\multicolumn{4}{l}{\emph{Panel C. Treatment effect heterogeneity - Men}} \\ Abs. numbers        &       0.625         &       5.883\sym{***}&       3.254         \\
                    &     (1.680)         &     (1.940)         &     (2.449)         \\
 Ratio fertility     &      -0.141\sym{**} &   -0.000902         &     -0.0710         \\
                    &    (0.0670)         &    (0.0835)         &    (0.0931)         \\
 Ratio population    &    -0.00592         &      0.0253         &     0.00968         \\
                    &    (0.0831)         &    (0.0983)         &     (0.111)         \\
 Cum. numbers        &      -4.317         &       57.24\sym{***}&       26.46         \\
                    &     (13.00)         &     (18.25)         &     (21.04)         \\
 Cum. ratio          &      -1.730\sym{***}&       0.130         &      -0.800         \\
                    &     (0.543)         &     (0.785)         &     (0.808)         \\
 
\bottomrule \end{tabular} } \begin{tablenotes} \item \scriptsize \emph{Notes:} Clustered standard errors in parentheses. All regression are run with month-of-birth FEs and control cohort 3 is assigned with the treatment status. All regressions are carried out with a window width of half a year. \end{tablenotes} \end{threeparttable} \end{table} 

%---------------------------------
% CUMMULATIVE APPROACH
\begin{landscape}
 \begin{table}[H] \begin{threeparttable} \centering \caption{Cummulative effects for upt to different points of age} {\def\sym#1{\ifmmode^{#1}\else\(^{#1}\)\fi} \begin{tabular}{l*{13}{c}} \toprule & \multicolumn{12}{c}{Dependent variable: \textbf{Diseases of the nervous system}} \\ \cmidrule(lr){2-13}
            &\multicolumn{4}{c}{Average Causal Effects}         &\multicolumn{4}{c}{Women}                          &\multicolumn{4}{c}{Men}                            \\\cmidrule(lr){2-5}\cmidrule(lr){6-9}\cmidrule(lr){10-13}
            &\multicolumn{1}{c}{(1)}&\multicolumn{1}{c}{(2)}&\multicolumn{1}{c}{(3)}&\multicolumn{1}{c}{(4)}&\multicolumn{1}{c}{(5)}&\multicolumn{1}{c}{(6)}&\multicolumn{1}{c}{(7)}&\multicolumn{1}{c}{(8)}&\multicolumn{1}{c}{(9)}&\multicolumn{1}{c}{(10)}&\multicolumn{1}{c}{(11)}&\multicolumn{1}{c}{(12)}\\
            &\multicolumn{1}{c}{2M}&\multicolumn{1}{c}{4M}&\multicolumn{1}{c}{6M}&\multicolumn{1}{c}{Donut}&\multicolumn{1}{c}{2M}&\multicolumn{1}{c}{4M}&\multicolumn{1}{c}{6M}&\multicolumn{1}{c}{Donut}&\multicolumn{1}{c}{2M}&\multicolumn{1}{c}{4M}&\multicolumn{1}{c}{6M}&\multicolumn{1}{c}{Donut}\\
\midrule
 \multicolumn{13}{l}{\emph{Panel A. 2 Up to the age of 21}} \\ Cum. numbers        &      -8.000         &      -1.250         &       11.17         &       22.60         &       6.000         &       0.750         &      -12.83         &      -10.40         &      -14.00         &      -2.000         &       24.00         &       33.00         \\
                    &     (60.08)         &     (34.24)         &     (23.79)         &     (23.92)         &     (22.02)         &     (16.68)         &     (13.75)         &     (16.57)         &     (51.86)         &     (33.29)         &     (26.89)         &     (28.01)         \\
 Cum. ratio          &      -0.344         &      -0.663         &      -0.704         &      -0.616         &     -0.0223         &      -0.633         &      -1.654\sym{**} &      -1.708\sym{*}  &      -0.646         &      -0.695         &       0.186         &       0.408         \\
                    &     (1.491)         &     (0.677)         &     (0.478)         &     (0.516)         &     (1.027)         &     (0.821)         &     (0.786)         &     (0.956)         &     (2.353)         &     (1.237)         &     (0.995)         &     (1.035)         \\
 \midrule\multicolumn{13}{l}{\emph{Panel B. Up to the age of 26}} \\ Cum. numbers        &       115.0\sym{*}  &       111.3\sym{***}&       113.8\sym{***}&       120.0\sym{***}&       94.50\sym{*}  &       62.75\sym{*}  &       40.17         &       42.00         &       20.50         &       48.50\sym{**} &       73.67\sym{**} &          78\sym{**} \\
                    &     (52.50)         &     (29.79)         &     (30.91)         &     (36.80)         &     (48.93)         &     (31.52)         &     (25.04)         &     (30.26)         &     (35.72)         &     (20.74)         &     (28.24)         &     (34.30)         \\
 Cum. ratio          &       1.900         &       0.992         &       0.488         &       0.329         &       3.376         &       1.256         &      -0.548         &      -0.787         &       0.487         &       0.734         &       1.438         &       1.354         \\
                    &     (1.669)         &     (0.826)         &     (0.760)         &     (0.915)         &     (2.404)         &     (1.538)         &     (1.321)         &     (1.600)         &     (2.020)         &     (0.967)         &     (1.171)         &     (1.430)         \\
 \midrule\multicolumn{13}{l}{\emph{Panel C. Up to the age of 31}} \\ Cum. numbers        &       131.5\sym{*}  &       151.8\sym{**} &       179.2\sym{***}&       184.0\sym{***}&       119.5         &       75.75         &       84.33\sym{*}  &       83.60\sym{**} &       12.00         &       76.00\sym{*}  &       94.83\sym{**} &       100.4\sym{*}  \\
                    &     (59.30)         &     (66.85)         &     (55.52)         &     (61.76)         &     (63.11)         &     (61.42)         &     (41.67)         &     (37.99)         &     (38.95)         &     (37.70)         &     (44.74)         &     (54.26)         \\
 Cum. ratio          &       1.915         &       0.975         &       0.675         &       0.294         &       4.075         &       0.923         &     -0.0182         &      -0.577         &      -0.157         &       1.012         &       1.278         &       1.065         \\
                    &     (2.271)         &     (1.149)         &     (1.069)         &     (1.171)         &     (3.592)         &     (2.393)         &     (1.592)         &     (1.397)         &     (2.047)         &     (1.606)         &     (1.892)         &     (2.295)         \\
 \midrule\multicolumn{13}{l}{\emph{Panel D. Up to the age of 34}} \\ Cum. numbers        &       127.5         &       175.7         &       231.7\sym{**} &       228.0\sym{**} &       161.0         &       116.2         &       127.2\sym{**} &       123.2\sym{***}&      -33.50         &       59.50         &       104.5         &       104.8         \\
                    &     (109.0)         &     (105.7)         &     (82.92)         &     (94.01)         &     (103.0)         &     (80.83)         &     (54.37)         &     (40.43)         &     (103.2)         &     (65.75)         &     (68.85)         &     (83.06)         \\
 Cum. ratio          &       1.513         &       0.628         &       0.599         &      -0.142         &       5.479         &       1.756         &       0.441         &      -0.460         &      -2.276         &      -0.455         &       0.680         &      0.0843         \\
                    &     (3.458)         &     (1.794)         &     (1.596)         &     (1.727)         &     (5.716)         &     (3.253)         &     (2.169)         &     (1.565)         &     (4.198)         &     (2.374)         &     (2.799)         &     (3.330)         \\
 
\bottomrule \end{tabular} } \begin{tablenotes} \item \scriptsize \emph{Notes:} Clustered standard errors in parentheses (MxY). All regressions contain Birthmonth FE. Ratios indicate cases per thousand; original number of births. \end{tablenotes} \end{threeparttable} \end{table} 

\end{landscape}
\begin{landscape}
 \begin{table}[H] \begin{threeparttable} \centering \caption{Cummulative effects for upt to different points of age - BOOTSTRAPPED} {\def\sym#1{\ifmmode^{#1}\else\(^{#1}\)\fi} \begin{tabular}{l*{13}{c}} \toprule & \multicolumn{12}{c}{Dependent variable: \textbf{Diseases of the nervous system}} \\ \cmidrule(lr){2-13}
            &\multicolumn{4}{c}{Average Causal Effects}         &\multicolumn{4}{c}{Women}                          &\multicolumn{4}{c}{Men}                            \\\cmidrule(lr){2-5}\cmidrule(lr){6-9}\cmidrule(lr){10-13}
            &\multicolumn{1}{c}{(1)}&\multicolumn{1}{c}{(2)}&\multicolumn{1}{c}{(3)}&\multicolumn{1}{c}{(4)}&\multicolumn{1}{c}{(5)}&\multicolumn{1}{c}{(6)}&\multicolumn{1}{c}{(7)}&\multicolumn{1}{c}{(8)}&\multicolumn{1}{c}{(9)}&\multicolumn{1}{c}{(10)}&\multicolumn{1}{c}{(11)}&\multicolumn{1}{c}{(12)}\\
            &\multicolumn{1}{c}{2M}&\multicolumn{1}{c}{4M}&\multicolumn{1}{c}{6M}&\multicolumn{1}{c}{Donut}&\multicolumn{1}{c}{2M}&\multicolumn{1}{c}{4M}&\multicolumn{1}{c}{6M}&\multicolumn{1}{c}{Donut}&\multicolumn{1}{c}{2M}&\multicolumn{1}{c}{4M}&\multicolumn{1}{c}{6M}&\multicolumn{1}{c}{Donut}\\
\midrule
 \multicolumn{13}{l}{\emph{Panel A. 2 Up to the age of 21}} \\ Cum. numbers        &      -8.000         &      -1.250         &       11.17         &       22.60         &       6.000         &       0.750         &      -12.83         &      -10.40         &      -14.00         &      -2.000         &       24.00         &       33.00         \\
                    &     (57.65)         &     (47.00)         &     (36.20)         &     (34.40)         &     (18.96)         &     (23.39)         &     (20.25)         &     (19.85)         &     (51.15)         &     (44.63)         &     (37.35)         &     (34.74)         \\
 Cum. ratio          &      -0.344         &      -0.663         &      -0.704         &      -0.616         &     -0.0223         &      -0.633         &      -1.654         &      -1.708         &      -0.646         &      -0.695         &       0.186         &       0.408         \\
                    &     (1.442)         &     (0.951)         &     (0.805)         &     (0.811)         &     (0.897)         &     (1.208)         &     (1.089)         &     (1.114)         &     (2.322)         &     (1.630)         &     (1.474)         &     (1.378)         \\
 \midrule\multicolumn{13}{l}{\emph{Panel B. Up to the age of 26}} \\ Cum. numbers        &       115.0\sym{**} &       111.3\sym{***}&       113.8\sym{***}&       120.0\sym{**} &       94.50\sym{**} &       62.75         &       40.17         &       42.00         &       20.50         &       48.50\sym{*}  &       73.67\sym{*}  &          78\sym{*}  \\
                    &     (50.48)         &     (40.83)         &     (44.05)         &     (53.42)         &     (42.94)         &     (43.36)         &     (40.97)         &     (42.04)         &     (33.84)         &     (26.77)         &     (38.14)         &     (44.55)         \\
 Cum. ratio          &       1.900         &       0.992         &       0.488         &       0.329         &       3.376         &       1.256         &      -0.548         &      -0.787         &       0.487         &       0.734         &       1.438         &       1.354         \\
                    &     (1.631)         &     (1.193)         &     (1.242)         &     (1.436)         &     (2.169)         &     (2.182)         &     (2.113)         &     (2.149)         &     (1.952)         &     (1.281)         &     (1.744)         &     (2.055)         \\
 \midrule\multicolumn{13}{l}{\emph{Panel C. Up to the age of 31}} \\ Cum. numbers        &       131.5\sym{**} &       151.8\sym{*}  &       179.2\sym{***}&       184.0\sym{**} &       119.5\sym{*}  &       75.75         &       84.33         &       83.60\sym{*}  &       12.00         &       76.00         &       94.83         &       100.4         \\
                    &     (58.26)         &     (90.02)         &     (67.24)         &     (75.29)         &     (61.19)         &     (81.13)         &     (59.78)         &     (48.86)         &     (33.52)         &     (51.65)         &     (61.63)         &     (67.56)         \\
 Cum. ratio          &       1.915         &       0.975         &       0.675         &       0.294         &       4.075         &       0.923         &     -0.0182         &      -0.577         &      -0.157         &       1.012         &       1.278         &       1.065         \\
                    &     (2.239)         &     (1.443)         &     (1.601)         &     (1.628)         &     (3.513)         &     (3.073)         &     (2.506)         &     (1.856)         &     (1.844)         &     (2.199)         &     (2.836)         &     (3.075)         \\
 \midrule\multicolumn{13}{l}{\emph{Panel D. Up to the age of 34}} \\ Cum. numbers        &       127.5         &       175.7         &       231.7\sym{**} &       228.0\sym{*}  &       161.0         &       116.2         &       127.2\sym{*}  &       123.2\sym{**} &      -33.50         &       59.50         &       104.5         &       104.8         \\
                    &     (99.85)         &     (142.0)         &     (103.9)         &     (117.1)         &     (101.6)         &     (103.6)         &     (75.53)         &     (52.88)         &     (90.11)         &     (92.18)         &     (95.37)         &     (104.5)         \\
 Cum. ratio          &       1.513         &       0.628         &       0.599         &      -0.142         &       5.479         &       1.756         &       0.441         &      -0.460         &      -2.276         &      -0.455         &       0.680         &      0.0843         \\
                    &     (3.302)         &     (2.289)         &     (2.311)         &     (2.344)         &     (5.643)         &     (4.104)         &     (3.268)         &     (2.068)         &     (3.617)         &     (3.296)         &     (4.109)         &     (4.410)         \\
 
\bottomrule \end{tabular} } \begin{tablenotes} \item \scriptsize \emph{Notes:} \textbf{BOOTSTRAPPED} standard errors in parentheses (MxY), with 400 replications. All regressions contain Birthmonth FE. Ratios indicate cases per thousand; original number of births. \end{tablenotes} \end{threeparttable} \end{table} 

\end{landscape}
%---------------------------------
\newpage
FEBRUAR CASES:
 \begin{table}[H] \begin{threeparttable} \centering \caption{Dep. variable: \textbf{Diseases of the nervous system}} {\def\sym#1{\ifmmode^{#1}\else\(^{#1}\)\fi} \begin{tabular}{l*{13}{c}} \toprule year & \multicolumn{12}{c}{Month of birth} \\ \cmidrule(lr){2-13} 
            &          11&          12&           1&           2&           3&           4&           5&           6&           7&           8&           9&          10\\
1995        &         145&         128&         133&         127&         127&         138&         146&         141&         149&         130&         136&         144\\
1996        &         143&         142&         135&         138&         145&         149&         156&         167&         158&         140&         134&         155\\
1997        &         166&         128&         139&         133&         132&         143&         159&         141&         177&         170&         149&         156\\
1998        &         153&         154&         160&         154&         141&         144&         178&         171&         193&         178&         152&         165\\
1999        &         160&         182&         177&         156&         183&         155&         186&         156&         173&         192&         159&         169\\
2000        &         170&         144&         160&         165&         198&         172&         174&         193&         206&         207&         172&         189\\
2001        &         191&         192&         199&         191&         197&         204&         195&         213&         190&         193&         214&         189\\
2002        &         220&         176&         220&         195&         197&         204&         213&         172&         185&         174&         195&         192\\
2003        &         196&         195&         197&         170&         211&         172&         166&         175&         220&         185&         182&         213\\
2004        &         184&         199&         156&         189&         208&         204&         213&         216&         224&         208&         199&         223\\
2005        &         228&         200&         201&         210&         237&         201&         179&         212&         240&         209&         201&         213\\
2006        &         232&         228&         224&         194&         247&         241&         243&         243&         238&         180&         195&         239\\
2007        &         203&         224&         203&         230&         280&         245&         225&         266&         239&         226&         238&         226\\
2008        &         234&         220&         242&         227&         275&         255&         262&         265&         238&         239&         255&         216\\
2009        &         257&         275&         218&         249&         246&         267&         291&         262&         274&         269&         245&         261\\
2010        &         245&         235&         246&         259&         297&         272&         282&         289&         281&         229&         256&         251\\
2011        &         261&         293&         266&         272&         317&         304&         281&         321&         279&         264&         286&         307\\
2012        &         278&         299&         281&         286&         301&         324&         307&         332&         322&         310&         305&         299\\
2013        &         312&         356&         316&         318&         344&         301&         361&         355&         345&         306&         318&         319\\
2014        &         341&         298&         331&         302&         366&         333&         385&         394&         356&         337&         324&         362\\
 \bottomrule \end{tabular} } \begin{tablenotes} \item \scriptsize \emph{Notes:} Number of cases per year and MOB in treatment cohort. \end{tablenotes} \end{threeparttable} \end{table} 

 \begin{table}[H] \begin{threeparttable} \centering \caption{Dep. variable: \textbf{Diseases of the nervous system}} {\def\sym#1{\ifmmode^{#1}\else\(^{#1}\)\fi} \begin{tabular}{l*{13}{c}} \toprule year & \multicolumn{12}{c}{Month of birth} \\ \cmidrule(lr){2-13} 
            &          11&          12&           1&           2&           3&           4&           5&           6&           7&           8&           9&          10\\
1995        &          -7&          -4&         -40&          13&         -22&          12&           4&         -10&          20&         -29&           1&          23\\
1996        &           3&          16&         -10&           7&          18&          20&           7&          27&          19&           0&          -5&          -2\\
1997        &          29&         -20&         -50&          -7&         -26&         -15&           5&         -19&           4&          10&          -2&           5\\
1998        &           2&          12&           9&           2&           4&           6&          31&           2&           3&           9&          -2&          28\\
1999        &          11&          22&          14&          11&          22&         -28&          35&         -27&          -6&          12&          12&           1\\
2000        &          14&         -22&           7&          17&          17&          30&          -9&           5&          -2&          10&          -9&         -19\\
2001        &          15&          10&          41&          19&          21&           4&          -4&         -24&         -29&           5&          33&         -25\\
2002        &          57&         -28&          26&          11&          16&           1&         -12&         -37&         -27&         -56&         -29&          -1\\
2003        &          25&          -6&          27&           5&          16&          12&         -43&         -18&          -3&         -26&           1&           1\\
2004        &           1&          32&         -26&         -16&           8&          14&         -18&         -21&          18&          -3&          -4&          10\\
2005        &          57&          27&           5&          37&           4&          21&         -24&          28&           5&          -9&          13&         -36\\
2006        &          34&          26&          19&           1&          24&          31&          15&          22&         -43&         -77&         -15&          50\\
2007        &         -20&          38&         -21&          18&          28&           8&         -19&          50&           3&         -12&           5&         -27\\
2008        &          18&         -12&          20&          -9&          27&          34&           8&          41&         -31&         -16&          15&         -23\\
2009        &          13&          31&         -28&         -11&         -34&          18&          51&          -5&         -28&          -9&         -25&          -8\\
2010        &          23&         -29&         -41&         -20&           1&          35&          31&          -6&           0&         -21&           5&         -44\\
2011        &           4&          11&           2&         -38&          16&          35&         -24&          70&         -25&         -79&         -36&          33\\
2012        &          28&          16&          10&         -19&         -20&          45&         -14&          -8&           3&          20&          -6&         -12\\
2013        &          50&          86&          -2&          32&         -11&          -4&          29&          32&          28&          -3&          34&          -3\\
2014        &          41&         -12&           8&          -9&           0&          13&          45&          30&          -5&          -4&          -9&          24\\
 \bottomrule \end{tabular} } \begin{tablenotes} \item \scriptsize \emph{Notes:} Difference of cases (control - treatment) per year and MOB in treatment cohort. \end{tablenotes} \end{threeparttable} \end{table} 

