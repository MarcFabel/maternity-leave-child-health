%---------------------------------
% INPUT FOR VARIABLE: d9
%---------------------------------
\subsection{d9}
% RD overview
\begin{landscape}
\begin{figure}[H]
	\centering
	\begin{minipage}{.95\linewidth}
	\includegraphics[width=\linewidth]{rd_d9_overview_panel1}
	{\scriptsize \emph{Notes:} The figures show monthly RD plots with averages obtained from a bin width of one month. The solid vertical line divides pre- and post-reform regime. The averages are taken over the period of at most 1995-2014. \par}
\end{minipage}
\end{figure}
\end{landscape}
\begin{landscape}
\begin{figure}[H]
	\centering
\begin{minipage}{.95\linewidth}
	\includegraphics[width=\linewidth]{rd_d9_overview_panel2}
	{\scriptsize \emph{Notes:} The figures show monthly RD plots with a moving average window width of 3 months. The solid vertical line divides pre- and post-reform regime. The averages are taken over the period of at most 1995-2014. \par}
\end{minipage}
\end{figure}
\end{landscape}
%---------------------------------
% TABELLEN
 \begin{table}[H] \begin{threeparttable} \centering \caption{Dep. variable: \textbf{Diseases of the respiratory system}} {\def\sym#1{\ifmmode^{#1}\else\(^{#1}\)\fi} \begin{tabular}{l*{8}{c}} \toprule & \multicolumn{7}{c}{Estimation window} \\ \cmidrule(lr){2-8}
            &\multicolumn{1}{c}{(1)}&\multicolumn{1}{c}{(2)}&\multicolumn{1}{c}{(3)}&\multicolumn{1}{c}{(4)}&\multicolumn{1}{c}{(5)}&\multicolumn{1}{c}{(6)}&\multicolumn{1}{c}{(7)}\\
            &\multicolumn{1}{c}{1M}&\multicolumn{1}{c}{2M}&\multicolumn{1}{c}{3M}&\multicolumn{1}{c}{4M}&\multicolumn{1}{c}{5M}&\multicolumn{1}{c}{6M}&\multicolumn{1}{c}{Donut}\\
\midrule
 \multicolumn{8}{l}{\emph{Panel A. Average causal effects}} \\ Abs. numbers        &      -7.100\sym{***}&      -15.03\sym{**} &      -8.433\sym{*}  &       0.112         &       2.820         &       9.825\sym{*}  &       13.21\sym{**} \\
                    &  (7.15e-14)         &     (4.513)         &     (4.099)         &     (6.011)         &     (5.026)         &     (5.381)         &     (5.740)         \\
 Ratio fertility     &      -0.213\sym{***}&      -0.440\sym{**} &      -0.477\sym{***}&      -0.336\sym{***}&      -0.310\sym{***}&      -0.287\sym{***}&      -0.301\sym{***}\\
                    &  (1.45e-15)         &     (0.182)         &     (0.120)         &     (0.113)         &    (0.0920)         &    (0.0771)         &    (0.0866)         \\
 Ratio population    &       0.137\sym{***}&     -0.0202         &     -0.0477         &     -0.0335         &     -0.0421         &      0.0155         &    -0.00874         \\
                    &  (6.18e-16)         &    (0.0697)         &    (0.0554)         &    (0.0459)         &    (0.0428)         &    (0.0458)         &    (0.0525)         \\
 Ratio fert(03-14)   &       0.128\sym{***}&     -0.0192         &     -0.0445         &     -0.0315         &     -0.0371         &      0.0186         &    -0.00330         \\
                    &  (1.37e-15)         &    (0.0657)         &    (0.0520)         &    (0.0429)         &    (0.0406)         &    (0.0440)         &    (0.0506)         \\
 Cum. numbers        &      -194.6\sym{***}&      -228.0\sym{***}&      -133.9\sym{**} &      -17.47         &       18.85         &       86.39         &       142.6\sym{**} \\
                    &  (1.54e-12)         &     (52.00)         &     (53.13)         &     (77.86)         &     (65.86)         &     (65.12)         &     (63.13)         \\
 Cum. ratio          &      -2.351\sym{***}&      -2.979\sym{**} &      -2.824\sym{***}&      -1.665\sym{**} &      -1.315\sym{*}  &      -1.130\sym{*}  &      -0.885         \\
                    &  (3.81e-14)         &     (1.062)         &     (0.735)         &     (0.777)         &     (0.656)         &     (0.555)         &     (0.630)         \\
 \midrule\multicolumn{8}{l}{\emph{Panel B. Treatment effect heterogeneity - Women}} \\ Abs. numbers        &       10.25\sym{***}&      -0.225         &       4.150         &       7.737\sym{**} &       6.050\sym{**} &       8.167\sym{***}&       7.750\sym{**} \\
                    &  (7.95e-14)         &     (4.791)         &     (3.667)         &     (3.421)         &     (2.834)         &     (2.575)         &     (2.801)         \\
 Ratio fertility     &       0.306\sym{***}&      0.0382         &      0.0463         &       0.115         &      0.0354         &     0.00160         &     -0.0592         \\
                    &  (2.21e-15)         &     (0.140)         &    (0.0931)         &    (0.0767)         &    (0.0761)         &    (0.0657)         &    (0.0702)         \\
 Ratio population    &       0.347\sym{***}&       0.221\sym{**} &       0.210\sym{***}&       0.215\sym{***}&      0.0968         &       0.110         &      0.0627         \\
                    &  (1.14e-15)         &    (0.0844)         &    (0.0660)         &    (0.0493)         &    (0.0696)         &    (0.0658)         &    (0.0744)         \\
 Ratio fert(03-14)   &       0.335\sym{***}&       0.210\sym{**} &       0.200\sym{***}&       0.205\sym{***}&      0.0953         &       0.111\sym{*}  &      0.0660         \\
                    &  (6.18e-16)         &    (0.0815)         &    (0.0636)         &    (0.0475)         &    (0.0661)         &    (0.0635)         &    (0.0721)         \\
 Cum. numbers        &       84.20\sym{***}&      -34.33         &       37.60         &       88.31\sym{*}  &       73.22\sym{*}  &       87.79\sym{**} &       88.51\sym{**} \\
                    &  (1.39e-12)         &     (63.75)         &     (52.79)         &     (48.65)         &     (39.50)         &     (35.00)         &     (34.90)         \\
 Cum. ratio          &       2.597\sym{***}&      -0.448         &       0.247         &       1.290         &       0.498         &      -0.273         &      -0.847         \\
                    &  (3.82e-14)         &     (1.917)         &     (1.286)         &     (1.067)         &     (0.958)         &     (0.880)         &     (0.939)         \\
 \midrule\multicolumn{8}{l}{\emph{Panel C. Treatment effect heterogeneity - Men}} \\ Abs. numbers        &      -17.35\sym{***}&      -14.80\sym{***}&      -12.58\sym{***}&      -7.625\sym{**} &      -3.230         &       1.658         &       5.460         \\
                    &  (3.28e-14)         &     (1.221)         &     (1.258)         &     (3.251)         &     (3.381)         &     (3.663)         &     (3.713)         \\
 Ratio fertility     &      -0.676\sym{***}&      -0.684\sym{***}&      -0.759\sym{***}&      -0.598\sym{***}&      -0.431\sym{***}&      -0.326\sym{***}&      -0.255\sym{**} \\
                    &  (1.90e-15)         &     (0.119)         &    (0.0840)         &    (0.0980)         &     (0.110)         &     (0.105)         &     (0.116)         \\
 Ratio population    &      -0.129\sym{***}&      -0.347\sym{***}&      -0.413\sym{***}&      -0.361\sym{***}&      -0.244\sym{***}&      -0.152\sym{*}  &      -0.156         \\
                    &  (1.39e-15)         &    (0.0894)         &    (0.0650)         &    (0.0600)         &    (0.0786)         &    (0.0801)         &    (0.0965)         \\
 Ratio fert(03-14)   &     -0.0667\sym{***}&      -0.238\sym{***}&      -0.277\sym{***}&      -0.257\sym{***}&      -0.164\sym{**} &     -0.0691         &     -0.0695         \\
                    &  (7.90e-16)         &    (0.0664)         &    (0.0505)         &    (0.0490)         &    (0.0642)         &    (0.0707)         &    (0.0836)         \\
 Cum. numbers        &      -278.8\sym{***}&      -193.6\sym{***}&      -171.5\sym{***}&      -105.8\sym{**} &      -54.37         &      -1.400         &       54.08         \\
                    &  (2.04e-13)         &     (33.21)         &     (24.25)         &     (42.27)         &     (42.66)         &     (43.66)         &     (40.81)         \\
 Cum. ratio          &      -7.075\sym{***}&      -5.327\sym{***}&      -5.684\sym{***}&      -4.382\sym{***}&      -2.980\sym{***}&      -1.942\sym{**} &      -0.915         \\
                    &  (3.13e-14)         &     (0.858)         &     (0.700)         &     (0.812)         &     (0.919)         &     (0.907)         &     (0.923)         \\
 
\bottomrule \end{tabular} } \begin{tablenotes} \item \scriptsize \emph{Notes:} Clustered standard errors in parentheses. All regression are run with CG2 (i.e. the cohort prior to the reform) and with month-of-birth FEs. Ratios indicate cases per thousand; either approximated population (with weights coming from the original fertility distribution) or original number of births. \end{tablenotes} \end{threeparttable} \end{table} 

 \begin{table}[H] \begin{threeparttable} \centering \caption{Robustness with respect to the choice of \texttt{control group}} {\def\sym#1{\ifmmode^{#1}\else\(^{#1}\)\fi} \begin{tabular}{l*{10}{c}} \toprule & \multicolumn{9}{c}{Dependent variable: \textbf{Diseases of the respiratory system}} \\ \cmidrule(lr){2-10}
            &\multicolumn{3}{c}{Average Causal Effects}&\multicolumn{3}{c}{Women}             &\multicolumn{3}{c}{Men}               \\\cmidrule(lr){2-4}\cmidrule(lr){5-7}\cmidrule(lr){8-10}
            &\multicolumn{1}{c}{(1)}&\multicolumn{1}{c}{(2)}&\multicolumn{1}{c}{(3)}&\multicolumn{1}{c}{(4)}&\multicolumn{1}{c}{(5)}&\multicolumn{1}{c}{(6)}&\multicolumn{1}{c}{(7)}&\multicolumn{1}{c}{(8)}&\multicolumn{1}{c}{(9)}\\
            &\multicolumn{1}{c}{C2}&\multicolumn{1}{c}{C1+C2}&\multicolumn{1}{c}{C1-C3}&\multicolumn{1}{c}{C2}&\multicolumn{1}{c}{C1+C2}&\multicolumn{1}{c}{C1-C3}&\multicolumn{1}{c}{C2}&\multicolumn{1}{c}{C1+C2}&\multicolumn{1}{c}{C1-C3}\\
\midrule
 \multicolumn{10}{l}{\emph{Panel A. 2 Month bandwidth}} \\ Abs. numbers        &       0.833         &      -6.444         &      -13.91         &       9.444\sym{**} &      -0.306         &      -0.389         &      -8.611         &      -6.139         &      -13.52\sym{**} \\
                    &     (8.277)         &     (7.891)         &     (9.809)         &     (3.926)         &     (4.961)         &     (5.723)         &     (4.667)         &     (4.650)         &     (6.228)         \\
 Ratio population    &      -0.287\sym{**} &      -0.200         &      -0.229         &     -0.0490         &      -0.167         &      -0.189         &      -0.508\sym{***}&      -0.227         &      -0.255\sym{*}  \\
                    &     (0.106)         &     (0.182)         &     (0.174)         &     (0.285)         &     (0.296)         &     (0.312)         &     (0.113)         &     (0.193)         &     (0.139)         \\
 Ratio population    &      -0.134         &      -0.174         &      -0.168         &      0.0781         &     -0.0970         &     0.00599         &      -0.347\sym{***}&      -0.252         &      -0.342\sym{**} \\
                    &     (0.102)         &     (0.132)         &     (0.126)         &     (0.128)         &     (0.156)         &     (0.181)         &    (0.0894)         &     (0.144)         &     (0.137)         \\
 \midrule\multicolumn{10}{l}{\emph{Panel B. 4 Month bandwidth}} \\ Abs. numbers        &       7.861         &       3.569         &      -0.954         &       11.53\sym{***}&       6.153\sym{*}  &       5.028         &      -3.667         &      -2.583         &      -5.981         \\
                    &     (4.848)         &     (6.398)         &     (7.931)         &     (2.179)         &     (3.516)         &     (4.167)         &     (3.034)         &     (3.696)         &     (4.836)         \\
 Ratio fertility     &      -0.126\sym{*}  &      -0.144\sym{*}  &      -0.139\sym{**} &       0.115         &     0.00657         &      0.0119         &      -0.351\sym{***}&      -0.283\sym{***}&      -0.279\sym{***}\\
                    &    (0.0613)         &    (0.0742)         &    (0.0620)         &    (0.0767)         &     (0.109)         &     (0.109)         &    (0.0574)         &    (0.0693)         &    (0.0748)         \\
 Ratio fertility     &   -0.000637         &      0.0130         &     -0.0260         &       0.254\sym{***}&       0.178\sym{**} &       0.163\sym{*}  &      -0.244\sym{**} &      -0.146         &      -0.207\sym{*}  \\
                    &    (0.0717)         &    (0.0815)         &    (0.0748)         &    (0.0686)         &    (0.0787)         &    (0.0858)         &    (0.0854)         &     (0.111)         &     (0.106)         \\
 \midrule\multicolumn{10}{l}{\emph{Panel C. 6 Month bandwidth}} \\ Abs. numbers        &       9.825\sym{*}  &      0.0125         &      -2.625         &       8.167\sym{***}&       1.775         &       0.589         &       1.658         &      -1.762         &      -3.214         \\
                    &     (5.381)         &     (6.248)         &     (7.694)         &     (2.575)         &     (3.798)         &     (5.607)         &     (3.663)         &     (3.249)         &     (3.210)         \\
 Ratio fertility     &     -0.0685         &      -0.114\sym{*}  &     -0.0909         &     0.00160         &     -0.0852         &     -0.0407         &      -0.135\sym{*}  &      -0.142\sym{*}  &      -0.138         \\
                    &    (0.0473)         &    (0.0596)         &    (0.0556)         &    (0.0657)         &    (0.0860)         &    (0.0964)         &    (0.0757)         &    (0.0734)         &    (0.0839)         \\
 Ratio fertility     &      0.0273         &   -0.000813         &    -0.00798         &       0.145         &      0.0472         &      0.0737         &     -0.0852         &     -0.0469         &     -0.0858         \\
                    &    (0.0635)         &    (0.0701)         &    (0.0709)         &    (0.0876)         &    (0.0857)         &    (0.0912)         &    (0.0816)         &     (0.100)         &     (0.107)         \\
 \midrule\multicolumn{10}{l}{\emph{Panel D. Donut specification}} \\ Abs. numbers        &       13.04\sym{*}  &       5.622         &       2.881         &       9.311\sym{**} &       4.122         &       3.815         &       3.733         &       1.500         &      -0.933         \\
                    &     (6.266)         &     (6.765)         &     (7.840)         &     (3.461)         &     (3.680)         &     (4.370)         &     (3.907)         &     (4.201)         &     (4.942)         \\
 Ratio fertility     &     -0.0685         &      -0.144\sym{**} &     -0.0974         &     -0.0592         &      -0.141         &     -0.0744         &     -0.0771         &      -0.146\sym{*}  &      -0.118         \\
                    &    (0.0566)         &    (0.0678)         &    (0.0644)         &    (0.0702)         &    (0.0948)         &     (0.109)         &    (0.0850)         &    (0.0825)         &    (0.0951)         \\
 Ratio fertility     &     -0.0320         &     -0.0517         &     -0.0248         &      0.0604         &    -0.00546         &      0.0548         &      -0.121         &     -0.0964         &      -0.101         \\
                    &    (0.0636)         &    (0.0764)         &    (0.0810)         &    (0.0907)         &    (0.0942)         &     (0.105)         &    (0.0924)         &     (0.113)         &     (0.125)         \\
 
\bottomrule \end{tabular} } \begin{tablenotes} \item \scriptsize \emph{Notes:} Clustered standard errors in parentheses. All regressions contain Birthmonth FE. Ratios indicate cases per thousand; either approximated population (with weights coming from the original fertility distribution) or original number of births. \end{tablenotes} \end{threeparttable} \end{table} 

%---------------------------------
% Life-course figure (Panel1)
\begin{landscape}
\begin{figure}[H]
\centering
\begin{minipage}{.9\linewidth}
\includegraphics[width=\linewidth]{lc_d9_overview_panel1}
{\scriptsize \emph{Notes:} The figures depict DDRD estimates and 90\% confidence intervals over the life-course. The years are harmonized such that the cohorts are in the same age when they are compared. All regressions are carried out with month-of-birth FE and make use of clustered standard errors. Furthermore, we used a bandwidth of half a year and only the control cohort that was born one year prior to the reform. Ratios indicate cases per thousand; using in the denominator the approximated population (with weights coming from the original fertility distribution) or original number of births. \par}
\end{minipage}
\end{figure}
\end{landscape}
%---------------------------------
% Life-course figure (Panel2)
\begin{landscape}
\begin{figure}[H]
\centering
\begin{minipage}{.9\linewidth}
\includegraphics[width=\linewidth]{lc_d9_overview_panel2}
{\scriptsize \emph{Notes:} The figures depict DDRD estimates and 90\% confidence intervals over the life-course. The years are harmonized such that the cohorts are in the same age when they are compared. All regressions are carried out with month-of-birth FE and make use of clustered standard errors. Furthermore, we used a bandwidth of half a year. Ratios indicate cases per thousand; using in the denominator the approximated population (with weights coming from the original fertility distribution) or original number of births. \par}
\end{minipage}
\end{figure}
\end{landscape}
%---------------------------------
% Life-course (panel 3 - 6)
\begin{figure}[H]%\vspace*{-2cm}
	\centering
	\includegraphics[width=.9\linewidth]{lc_d9_overview_panel3}
	\includegraphics[width=.9\linewidth]{lc_d9_overview_panel4}
\end{figure}
\begin{figure}[H]
	\centering	
	\includegraphics[width=.97\linewidth]{lc_d9_overview_panel5}
	\includegraphics[width=.97\linewidth]{lc_d9_overview_panel6}
\end{figure}
% Life-course TABLE Format
 \begin{table}[H] \centering \begin{threeparttable} \caption{Life-course approach - Table format} {\def\sym#1{\ifmmode^{#1}\else\(^{#1}\)\fi} \begin{tabular}{l*{5}{c}} \toprule \multicolumn{5}{l}{Dep. variable: \textbf{Diseases of the respiratory system}} \\ & \multicolumn{4}{c}{Estimation window} \\ \cmidrule(lr){2-5}
            &\multicolumn{1}{c}{(1)}&\multicolumn{1}{c}{(2)}&\multicolumn{1}{c}{(3)}&\multicolumn{1}{c}{(4)}\\
            &\multicolumn{1}{c}{Age 17-21}&\multicolumn{1}{c}{Age 22-26}&\multicolumn{1}{c}{Age 27-31}&\multicolumn{1}{c}{Age 32-35}\\
\midrule
 \multicolumn{5}{l}{\emph{Panel A. Average causal effects}} \\ Abs. numbers        &       10.47         &       13.20         &       6.367         &       17.21         \\
                    &     (21.20)         &     (11.41)         &     (12.47)         &     (15.55)         \\
 Ratio fertility     &      -0.369         &      -0.199         &      -0.273         &     -0.0842         \\
                    &     (0.387)         &     (0.151)         &     (0.215)         &     (0.319)         \\
 Ratio population    &      0.0400         &      -0.184         &     -0.0421         \\
                    &     (0.192)         &     (0.171)         &     (0.250)         \\
 Cum. numbers        &       48.77         &       76.23         &       138.0         &       196.0         \\
                    &     (76.44)         &     (130.1)         &     (165.0)         &     (205.1)         \\
 Cum. ratio          &      -1.799         &      -3.808         &      -4.618\sym{*}  &      -5.346         \\
                    &     (1.698)         &     (2.482)         &     (2.607)         &     (3.305)         \\
 \midrule\multicolumn{5}{l}{\emph{Panel B. Treatment effect heterogeneity - Women}} \\ Abs. numbers        &       10.20         &       12.40\sym{*}  &       6.900         &       8.375         \\
                    &     (15.87)         &     (6.588)         &     (8.981)         &     (8.917)         \\
 Ratio fertility     &      -0.354         &     -0.0438         &      -0.149         &      -0.118         \\
                    &     (0.604)         &     (0.337)         &     (0.399)         &     (0.390)         \\
 Ratio population    &       0.292         &     -0.0860         &     -0.0630         \\
                    &     (0.311)         &     (0.294)         &     (0.296)         \\
 Cum. numbers        &       53.63         &       74.47         &       134.2         &       166.7         \\
                    &     (56.63)         &     (79.75)         &     (97.79)         &     (107.1)         \\
 Cum. ratio          &      -1.575         &      -3.932         &      -3.776         &      -4.462         \\
                    &     (2.400)         &     (3.291)         &     (3.952)         &     (4.568)         \\
 \midrule\multicolumn{5}{l}{\emph{Panel C. Treatment effect heterogeneity - Men}} \\ Abs. numbers        &       0.267         &       0.800         &      -0.533         &       8.833         \\
                    &     (9.850)         &     (9.662)         &     (7.193)         &     (9.163)         \\
 Ratio fertility     &      -0.406         &      -0.351         &      -0.391         &     -0.0488         \\
                    &     (0.337)         &     (0.315)         &     (0.241)         &     (0.363)         \\
 Ratio population    &      -0.213         &      -0.281         &     -0.0190         \\
                    &     (0.385)         &     (0.204)         &     (0.290)         \\
 Cum. numbers        &      -4.867         &       1.767         &       3.800         &       29.29         \\
                    &     (34.03)         &     (73.95)         &     (102.0)         &     (126.7)         \\
 Cum. ratio          &      -2.122         &      -3.856         &      -5.598         &      -6.350         \\
                    &     (1.570)         &     (2.685)         &     (3.498)         &     (4.042)         \\
 
\bottomrule \end{tabular} } \begin{tablenotes} \item \scriptsize \emph{Notes:} Clustered standard errors in parentheses. All regression are run with CG2 (i.e. the cohort prior to the reform) and with month-of-birth FEs. Ratios indicate cases per thousand; either approximated population (with weights coming from the original fertility distribution) or original number of births. Raqtio population muss eins nach rechts gerückt werden \end{tablenotes} \end{threeparttable} \end{table} 

%---------------------------------
% PLACEBO EXERCISES
\newpage
\begin{landscape}
\begin{figure}[H]
	\centering
    \begin{minipage}{.9\linewidth}
	\includegraphics[width=\linewidth]{placebo_graph_d9.pdf}
    {\scriptsize \emph{Notes:} The figures depict DDRD estimates and 95\% confidence intervals when the treatment cohort is shifted over time. The date on the abscissa indicates the starting date of the treated.  All regressions are carried out with month-of-birth FE and make use of clustered standard errors. Furthermore, we used a bandwidth of half a year. Ratios indicate cases per thousand; using in the denominator the approximated population (with weights coming from the original fertility distribution) or original number of births. \par}
    \end{minipage}
\end{figure}
\end{landscape}
 \begin{table}[H] \centering \begin{threeparttable} \caption{Placebo 1 (CONTROL1 ist TREAT) } {\def\sym#1{\ifmmode^{#1}\else\(^{#1}\)\fi} \begin{tabular}{l*{4}{c}} \toprule \multicolumn{4}{l}{Dep. variable: \textbf{Diseases of the respiratory system}} \\ & \multicolumn{3}{c}{Choice of control group} \\ \cmidrule(lr){2-4}
            &\multicolumn{1}{c}{(1)}&\multicolumn{1}{c}{(2)}&\multicolumn{1}{c}{(3)}\\
            &\multicolumn{1}{c}{C2}&\multicolumn{1}{c}{C3}&\multicolumn{1}{c}{C2+C3}\\
\midrule
 \multicolumn{4}{l}{\emph{Panel A. Average causal effects}} \\ Abs. numbers        &       19.63\sym{***}&       1.900         &       10.76         \\
                    &     (4.034)         &     (4.221)         &     (7.484)         \\
 Ratio fertility     &       0.275\sym{**} &       0.522\sym{***}&       0.398\sym{***}\\
                    &     (0.101)         &     (0.133)         &     (0.120)         \\
 Ratio population    &       0.176\sym{***}&       0.180\sym{**} &       0.178\sym{***}\\
                    &    (0.0503)         &    (0.0680)         &    (0.0629)         \\
 Cum. numbers        &       229.2\sym{***}&       68.61         &       148.9\sym{*}  \\
                    &     (53.68)         &     (50.76)         &     (77.79)         \\
 Cum. ratio          &       3.273\sym{**} &       6.806\sym{***}&       5.039\sym{***}\\
                    &     (1.404)         &     (1.664)         &     (1.600)         \\
 \midrule\multicolumn{4}{l}{\emph{Panel B. Treatment effect heterogeneity - Women}} \\ Abs. numbers        &       12.78\sym{***}&       2.833         &       7.808         \\
                    &     (1.811)         &     (2.558)         &     (5.689)         \\
 Ratio fertility     &       0.404\sym{***}&       0.677\sym{***}&       0.541\sym{***}\\
                    &    (0.0829)         &     (0.148)         &     (0.141)         \\
 Ratio population    &       0.265\sym{***}&       0.287\sym{***}&       0.276\sym{***}\\
                    &    (0.0690)         &    (0.0966)         &    (0.0961)         \\
 Cum. numbers        &       142.3\sym{***}&       43.55         &       92.92         \\
                    &     (23.17)         &     (32.26)         &     (67.35)         \\
 Cum. ratio          &       4.439\sym{***}&       8.515\sym{***}&       6.477\sym{***}\\
                    &     (1.000)         &     (1.750)         &     (1.680)         \\
 \midrule\multicolumn{4}{l}{\emph{Panel C. Treatment effect heterogeneity - Men}} \\ Abs. numbers        &       6.842\sym{**} &      -0.933         &       2.954         \\
                    &     (3.248)         &     (2.980)         &     (3.297)         \\
 Ratio fertility     &       0.152         &       0.376\sym{**} &       0.264         \\
                    &     (0.145)         &     (0.146)         &     (0.167)         \\
 Ratio population    &      0.0857         &      0.0740         &      0.0798         \\
                    &    (0.0826)         &    (0.0993)         &    (0.0901)         \\
 Cum. numbers        &       86.96\sym{*}  &       25.06         &       56.01         \\
                    &     (43.80)         &     (33.25)         &     (39.26)         \\
 Cum. ratio          &       2.176         &       5.199\sym{**} &       3.688         \\
                    &     (2.037)         &     (1.874)         &     (2.417)         \\
 
\bottomrule \end{tabular} } \begin{tablenotes} \item \scriptsize \emph{Notes:} Clustered standard errors in parentheses. All regression are run with month-of-birth FEs and control cohort 2 is assigned with the treatment status. All regressions are carried out with a window width of half a year. \end{tablenotes} \end{threeparttable} \end{table} 

 \begin{table}[H] \centering \begin{threeparttable} \caption{Placebo 2 (CONTROL2 ist TREAT) } {\def\sym#1{\ifmmode^{#1}\else\(^{#1}\)\fi} \begin{tabular}{l*{4}{c}} \toprule \multicolumn{4}{l}{Dep. variable: \textbf{Diseases of the respiratory system}} \\ & \multicolumn{3}{c}{Choice of control group} \\ \cmidrule(lr){2-4}
            &\multicolumn{1}{c}{(1)}&\multicolumn{1}{c}{(2)}&\multicolumn{1}{c}{(3)}\\
            &\multicolumn{1}{c}{C1}&\multicolumn{1}{c}{C3}&\multicolumn{1}{c}{C1+C3}\\
\midrule
 \multicolumn{4}{l}{\emph{Panel A. Average causal effects}} \\ Abs. numbers        &      -19.62\sym{***}&      -17.72\sym{***}&      -18.68\sym{*}  \\
                    &     (4.034)         &     (3.566)         &     (9.230)         \\
 Ratio fertility     &      -0.275\sym{**} &       0.247\sym{**} &     -0.0136         \\
                    &     (0.101)         &    (0.0992)         &     (0.129)         \\
 Ratio population    &      -0.176\sym{***}&     0.00462         &     -0.0855         \\
                    &    (0.0503)         &    (0.0608)         &    (0.0678)         \\
 Cum. numbers        &      -229.2\sym{***}&      -160.6\sym{***}&      -194.9\sym{*}  \\
                    &     (53.68)         &     (51.45)         &     (104.3)         \\
 Cum. ratio          &      -3.273\sym{**} &       3.533\sym{**} &       0.130         \\
                    &     (1.404)         &     (1.301)         &     (1.633)         \\
 \midrule\multicolumn{4}{l}{\emph{Panel B. Treatment effect heterogeneity - Women}} \\ Abs. numbers        &      -12.78\sym{***}&      -9.950\sym{***}&      -11.37         \\
                    &     (1.811)         &     (2.199)         &     (7.485)         \\
 Ratio fertility     &      -0.404\sym{***}&       0.273\sym{*}  &     -0.0657         \\
                    &    (0.0829)         &     (0.152)         &     (0.226)         \\
 Ratio population    &      -0.265\sym{***}&      0.0213         &      -0.122         \\
                    &    (0.0690)         &     (0.104)         &     (0.114)         \\
 Cum. numbers        &      -142.3\sym{***}&      -98.73\sym{***}&      -120.5         \\
                    &     (23.17)         &     (29.31)         &     (95.27)         \\
 Cum. ratio          &      -4.439\sym{***}&       4.076\sym{**} &      -0.182         \\
                    &     (1.000)         &     (1.832)         &     (2.910)         \\
 \midrule\multicolumn{4}{l}{\emph{Panel C. Treatment effect heterogeneity - Men}} \\ Abs. numbers        &      -6.842\sym{**} &      -7.775\sym{***}&      -7.308\sym{**} \\
                    &     (3.248)         &     (2.697)         &     (3.193)         \\
 Ratio fertility     &      -0.152         &       0.223\sym{**} &      0.0356         \\
                    &     (0.145)         &    (0.0867)         &     (0.133)         \\
 Ratio population    &     -0.0857         &     -0.0118         &     -0.0487         \\
                    &    (0.0826)         &    (0.0809)         &    (0.0809)         \\
 Cum. numbers        &      -86.96\sym{*}  &      -61.90\sym{*}  &      -74.43\sym{*}  \\
                    &     (43.80)         &     (33.28)         &     (38.55)         \\
 Cum. ratio          &      -2.176         &       3.023\sym{**} &       0.424         \\
                    &     (2.037)         &     (1.153)         &     (2.023)         \\
 
\bottomrule \end{tabular} } \begin{tablenotes} \item \scriptsize \emph{Notes:} Clustered standard errors in parentheses. All regression are run with month-of-birth FEs and control cohort 2 is assigned with the treatment status. All regressions are carried out with a window width of half a year. \end{tablenotes} \end{threeparttable} \end{table} 

 \begin{table}[H] \centering \begin{threeparttable} \caption{Placebo 3 (CONTROL3 ist TREAT) } {\def\sym#1{\ifmmode^{#1}\else\(^{#1}\)\fi} \begin{tabular}{l*{4}{c}} \toprule \multicolumn{4}{l}{Dep. variable: \textbf{Diseases of the respiratory system}} \\ & \multicolumn{3}{c}{Choice of control group} \\ \cmidrule(lr){2-4}
            &\multicolumn{1}{c}{(1)}&\multicolumn{1}{c}{(2)}&\multicolumn{1}{c}{(3)}\\
            &\multicolumn{1}{c}{C1}&\multicolumn{1}{c}{C2}&\multicolumn{1}{c}{C1+C2}\\
\midrule
 \multicolumn{4}{l}{\emph{Panel A. Average causal effects}} \\ Abs. numbers        &      -1.900         &       17.73\sym{***}&       7.912         \\
                    &     (4.221)         &     (3.566)         &     (5.002)         \\
 Ratio fertility     &      -0.522\sym{***}&      -0.247\sym{**} &      -0.385\sym{***}\\
                    &     (0.133)         &    (0.0992)         &     (0.139)         \\
 Ratio population    &      -0.180\sym{**} &    -0.00462         &     -0.0924         \\
                    &    (0.0680)         &    (0.0608)         &    (0.0673)         \\
 Cum. numbers        &      -68.61         &       160.6\sym{***}&       46.01         \\
                    &     (50.76)         &     (51.45)         &     (65.65)         \\
 Cum. ratio          &      -6.806\sym{***}&      -3.533\sym{**} &      -5.170\sym{***}\\
                    &     (1.664)         &     (1.301)         &     (1.781)         \\
 \midrule\multicolumn{4}{l}{\emph{Panel B. Treatment effect heterogeneity - Women}} \\ Abs. numbers        &      -2.833         &       9.950\sym{***}&       3.558         \\
                    &     (2.558)         &     (2.199)         &     (3.405)         \\
 Ratio fertility     &      -0.677\sym{***}&      -0.273\sym{*}  &      -0.475\sym{**} \\
                    &     (0.148)         &     (0.152)         &     (0.187)         \\
 Ratio population    &      -0.287\sym{***}&     -0.0213         &      -0.154         \\
                    &    (0.0966)         &     (0.104)         &     (0.104)         \\
 Cum. numbers        &      -43.55         &       98.73\sym{***}&       27.59         \\
                    &     (32.26)         &     (29.31)         &     (45.94)         \\
 Cum. ratio          &      -8.515\sym{***}&      -4.076\sym{**} &      -6.296\sym{**} \\
                    &     (1.750)         &     (1.832)         &     (2.396)         \\
 \midrule\multicolumn{4}{l}{\emph{Panel C. Treatment effect heterogeneity - Men}} \\ Abs. numbers        &       0.933         &       7.775\sym{***}&       4.354         \\
                    &     (2.980)         &     (2.697)         &     (2.921)         \\
 Ratio fertility     &      -0.376\sym{**} &      -0.223\sym{**} &      -0.300\sym{**} \\
                    &     (0.146)         &    (0.0867)         &     (0.125)         \\
 Ratio population    &     -0.0740         &      0.0118         &     -0.0311         \\
                    &    (0.0993)         &    (0.0809)         &    (0.0899)         \\
 Cum. numbers        &      -25.06         &       61.90\sym{*}  &       18.42         \\
                    &     (33.25)         &     (33.28)         &     (34.28)         \\
 Cum. ratio          &      -5.199\sym{**} &      -3.023\sym{**} &      -4.111\sym{**} \\
                    &     (1.874)         &     (1.153)         &     (1.594)         \\
 
\bottomrule \end{tabular} } \begin{tablenotes} \item \scriptsize \emph{Notes:} Clustered standard errors in parentheses. All regression are run with month-of-birth FEs and control cohort 3 is assigned with the treatment status. All regressions are carried out with a window width of half a year. \end{tablenotes} \end{threeparttable} \end{table} 

%---------------------------------
% CUMMULATIVE APPROACH
\begin{landscape}
 \begin{table}[H] \begin{threeparttable} \centering \caption{Cummulative effects for upt to different points of age} {\def\sym#1{\ifmmode^{#1}\else\(^{#1}\)\fi} \begin{tabular}{l*{13}{c}} \toprule & \multicolumn{12}{c}{Dependent variable: \textbf{Diseases of the respiratory system}} \\ \cmidrule(lr){2-13}
            &\multicolumn{4}{c}{Average Causal Effects}         &\multicolumn{4}{c}{Women}                          &\multicolumn{4}{c}{Men}                            \\\cmidrule(lr){2-5}\cmidrule(lr){6-9}\cmidrule(lr){10-13}
            &\multicolumn{1}{c}{(1)}&\multicolumn{1}{c}{(2)}&\multicolumn{1}{c}{(3)}&\multicolumn{1}{c}{(4)}&\multicolumn{1}{c}{(5)}&\multicolumn{1}{c}{(6)}&\multicolumn{1}{c}{(7)}&\multicolumn{1}{c}{(8)}&\multicolumn{1}{c}{(9)}&\multicolumn{1}{c}{(10)}&\multicolumn{1}{c}{(11)}&\multicolumn{1}{c}{(12)}\\
            &\multicolumn{1}{c}{2M}&\multicolumn{1}{c}{4M}&\multicolumn{1}{c}{6M}&\multicolumn{1}{c}{Donut}&\multicolumn{1}{c}{2M}&\multicolumn{1}{c}{4M}&\multicolumn{1}{c}{6M}&\multicolumn{1}{c}{Donut}&\multicolumn{1}{c}{2M}&\multicolumn{1}{c}{4M}&\multicolumn{1}{c}{6M}&\multicolumn{1}{c}{Donut}\\
\midrule
 \multicolumn{13}{l}{\emph{Panel A. 2 Up to the age of 21}} \\ Cum. numbers        &      -227.5         &      -41.25         &       48.83         &       94.20         &      -94.50         &       25.00         &       50.67         &       51.60         &      -133.0         &      -66.25         &      -1.833         &       42.60         \\
                    &     (133.0)         &     (106.7)         &     (82.11)         &     (77.41)         &     (137.0)         &     (79.25)         &     (54.84)         &     (57.91)         &     (72.75)         &     (50.72)         &     (44.55)         &     (34.86)         \\
 Cum. ratio          &      -5.588         &      -3.585         &      -2.907\sym{*}  &      -2.599         &      -5.368         &      -2.403         &      -3.143         &      -3.828         &      -5.859         &      -4.729\sym{**} &      -2.829\sym{*}  &      -1.581         \\
                    &     (3.891)         &     (2.276)         &     (1.523)         &     (1.561)         &     (6.610)         &     (3.430)         &     (2.308)         &     (2.440)         &     (3.284)         &     (1.846)         &     (1.449)         &     (1.244)         \\
 \midrule\multicolumn{13}{l}{\emph{Panel B. Up to the age of 26}} \\ Cum. numbers        &      -230.5         &      -1.250         &       114.8         &       184.4\sym{*}  &      -25.50         &       111.0         &       112.7\sym{*}  &       116.2\sym{*}  &      -205.0\sym{*}  &      -112.3         &       2.167         &       68.20         \\
                    &     (124.1)         &     (136.2)         &     (108.9)         &     (100.8)         &     (127.2)         &     (83.29)         &     (56.72)         &     (57.39)         &     (91.90)         &     (77.87)         &     (75.79)         &     (68.95)         \\
 Cum. ratio          &      -6.286         &      -4.340         &      -3.902\sym{**} &      -3.512\sym{*}  &      -3.346         &      -0.564         &      -3.362         &      -4.357         &      -9.151\sym{*}  &      -7.943\sym{***}&      -4.586\sym{*}  &      -2.883         \\
                    &     (4.560)         &     (2.541)         &     (1.720)         &     (1.750)         &     (7.052)         &     (3.263)         &     (2.536)         &     (2.671)         &     (4.093)         &     (2.685)         &     (2.366)         &     (2.330)         \\
 \midrule\multicolumn{13}{l}{\emph{Panel C. Up to the age of 31}} \\ Cum. numbers        &      -320.5\sym{**} &       2.750         &       146.7         &       226.0         &      -19.00         &         159         &       147.2\sym{*}  &       144.4\sym{*}  &      -301.5\sym{***}&      -156.3         &      -0.500         &       81.60         \\
                    &     (113.7)         &     (174.4)         &     (138.3)         &     (143.4)         &     (175.6)         &     (114.1)         &     (78.43)         &     (82.86)         &     (79.66)         &     (94.68)         &     (95.07)         &     (91.70)         \\
 Cum. ratio          &      -8.722         &      -5.650\sym{*}  &      -5.267\sym{***}&      -5.021\sym{**} &      -3.752         &      0.0869         &      -4.105         &      -5.690         &      -13.51\sym{***}&      -11.12\sym{***}&      -6.542\sym{**} &      -4.558         \\
                    &     (5.081)         &     (2.765)         &     (1.765)         &     (1.951)         &     (9.531)         &     (4.239)         &     (3.288)         &     (3.448)         &     (2.804)         &     (2.725)         &     (2.698)         &     (2.812)         \\
 \midrule\multicolumn{13}{l}{\emph{Panel D. Up to the age of 34}} \\ Cum. numbers        &      -311.5         &       17.00         &       215.5         &       278.0         &       0.500         &       161.5         &       180.7\sym{**} &       170.8\sym{*}  &      -312.0\sym{***}&      -144.5         &       34.83         &       107.2         \\
                    &     (178.7)         &     (191.6)         &     (164.2)         &     (184.8)         &     (184.0)         &     (109.2)         &     (79.29)         &     (89.23)         &     (13.15)         &     (101.8)         &     (107.5)         &     (113.9)         \\
 Cum. ratio          &      -9.074         &      -6.649\sym{*}  &      -5.604\sym{**} &      -5.975\sym{**} &      -3.573         &      -1.144         &      -4.577         &      -6.705\sym{*}  &      -14.36\sym{***}&      -11.90\sym{***}&      -6.737\sym{**} &      -5.443         \\
                    &     (6.515)         &     (3.565)         &     (2.327)         &     (2.701)         &     (9.782)         &     (4.440)         &     (3.300)         &     (3.450)         &     (3.733)         &     (3.239)         &     (3.131)         &     (3.607)         \\
 
\bottomrule \end{tabular} } \begin{tablenotes} \item \scriptsize \emph{Notes:} Clustered standard errors in parentheses (MxY). All regressions contain Birthmonth FE. Ratios indicate cases per thousand; original number of births. \end{tablenotes} \end{threeparttable} \end{table} 

\end{landscape}
\begin{landscape}
 \begin{table}[H] \begin{threeparttable} \centering \caption{Cummulative effects for upt to different points of age - BOOTSTRAPPED} {\def\sym#1{\ifmmode^{#1}\else\(^{#1}\)\fi} \begin{tabular}{l*{13}{c}} \toprule & \multicolumn{12}{c}{Dependent variable: \textbf{Diseases of the respiratory system}} \\ \cmidrule(lr){2-13}
            &\multicolumn{4}{c}{Average Causal Effects}         &\multicolumn{4}{c}{Women}                          &\multicolumn{4}{c}{Men}                            \\\cmidrule(lr){2-5}\cmidrule(lr){6-9}\cmidrule(lr){10-13}
            &\multicolumn{1}{c}{(1)}&\multicolumn{1}{c}{(2)}&\multicolumn{1}{c}{(3)}&\multicolumn{1}{c}{(4)}&\multicolumn{1}{c}{(5)}&\multicolumn{1}{c}{(6)}&\multicolumn{1}{c}{(7)}&\multicolumn{1}{c}{(8)}&\multicolumn{1}{c}{(9)}&\multicolumn{1}{c}{(10)}&\multicolumn{1}{c}{(11)}&\multicolumn{1}{c}{(12)}\\
            &\multicolumn{1}{c}{2M}&\multicolumn{1}{c}{4M}&\multicolumn{1}{c}{6M}&\multicolumn{1}{c}{Donut}&\multicolumn{1}{c}{2M}&\multicolumn{1}{c}{4M}&\multicolumn{1}{c}{6M}&\multicolumn{1}{c}{Donut}&\multicolumn{1}{c}{2M}&\multicolumn{1}{c}{4M}&\multicolumn{1}{c}{6M}&\multicolumn{1}{c}{Donut}\\
\midrule
 \multicolumn{13}{l}{\emph{Panel A. 2 Up to the age of 21}} \\ Cum. numbers        &      -227.5\sym{*}  &      -41.25         &       48.83         &       94.20         &      -94.50         &       25.00         &       50.67         &       51.60         &      -133.0\sym{**} &      -66.25         &      -1.833         &       42.60         \\
                    &     (130.8)         &     (139.3)         &     (118.3)         &     (105.5)         &     (127.5)         &     (104.7)         &     (86.02)         &     (78.75)         &     (64.37)         &     (63.23)         &     (52.14)         &     (44.25)         \\
 Cum. ratio          &      -5.588         &      -3.585         &      -2.907         &      -2.599         &      -5.368         &      -2.403         &      -3.143         &      -3.828         &      -5.859\sym{*}  &      -4.729\sym{*}  &      -2.829         &      -1.581         \\
                    &     (3.836)         &     (3.316)         &     (2.420)         &     (2.100)         &     (6.280)         &     (4.797)         &     (3.511)         &     (3.200)         &     (3.032)         &     (2.642)         &     (2.054)         &     (1.628)         \\
 \midrule\multicolumn{13}{l}{\emph{Panel B. Up to the age of 26}} \\ Cum. numbers        &      -230.5\sym{*}  &      -1.250         &       114.8         &       184.4         &      -25.50         &       111.0         &       112.7         &       116.2         &      -205.0\sym{**} &      -112.3         &       2.167         &       68.20         \\
                    &     (122.7)         &     (168.6)         &     (149.6)         &     (141.8)         &     (119.5)         &     (105.9)         &     (82.10)         &     (81.62)         &     (80.06)         &     (94.63)         &     (95.84)         &     (88.32)         \\
 Cum. ratio          &      -6.286         &      -4.340         &      -3.902         &      -3.512         &      -3.346         &      -0.564         &      -3.362         &      -4.357         &      -9.151\sym{**} &      -7.943\sym{**} &      -4.586         &      -2.883         \\
                    &     (4.509)         &     (3.744)         &     (2.710)         &     (2.342)         &     (6.798)         &     (4.513)         &     (3.530)         &     (3.406)         &     (3.786)         &     (3.894)         &     (3.427)         &     (2.954)         \\
 \midrule\multicolumn{13}{l}{\emph{Panel C. Up to the age of 31}} \\ Cum. numbers        &      -320.5\sym{***}&       2.750         &       146.7         &       226.0         &      -19.00         &         159         &       147.2         &       144.4         &      -301.5\sym{***}&      -156.3         &      -0.500         &       81.60         \\
                    &     (110.3)         &     (213.0)         &     (179.4)         &     (203.3)         &     (163.3)         &     (144.3)         &     (108.5)         &     (120.0)         &     (69.44)         &     (111.6)         &     (114.1)         &     (118.2)         \\
 Cum. ratio          &      -8.722\sym{*}  &      -5.650         &      -5.267\sym{*}  &      -5.021\sym{*}  &      -3.752         &      0.0869         &      -4.105         &      -5.690         &      -13.51\sym{***}&      -11.12\sym{***}&      -6.542\sym{*}  &      -4.558         \\
                    &     (4.983)         &     (3.982)         &     (2.786)         &     (2.844)         &     (9.111)         &     (5.616)         &     (4.590)         &     (4.715)         &     (2.562)         &     (3.983)         &     (3.734)         &     (3.651)         \\
 \midrule\multicolumn{13}{l}{\emph{Panel D. Up to the age of 34}} \\ Cum. numbers        &      -311.5\sym{*}  &       17.00         &       215.5         &       278.0         &       0.500         &       161.5         &       180.7         &       170.8         &      -312.0\sym{***}&      -144.5         &       34.83         &       107.2         \\
                    &     (161.6)         &     (235.0)         &     (221.7)         &     (253.3)         &     (164.9)         &     (138.1)         &     (113.2)         &     (126.4)         &     (11.86)         &     (120.8)         &     (135.9)         &     (147.6)         \\
 Cum. ratio          &      -9.074         &      -6.649         &      -5.604         &      -5.975         &      -3.573         &      -1.144         &      -4.577         &      -6.705         &      -14.36\sym{***}&      -11.90\sym{**} &      -6.737         &      -5.443         \\
                    &     (6.214)         &     (5.123)         &     (3.644)         &     (3.835)         &     (9.148)         &     (6.030)         &     (4.734)         &     (4.723)         &     (3.678)         &     (4.822)         &     (4.559)         &     (4.768)         \\
 
\bottomrule \end{tabular} } \begin{tablenotes} \item \scriptsize \emph{Notes:} \textbf{BOOTSTRAPPED} standard errors in parentheses (MxY), with 400 replications. All regressions contain Birthmonth FE. Ratios indicate cases per thousand; original number of births. \end{tablenotes} \end{threeparttable} \end{table} 

\end{landscape}
%---------------------------------
\newpage
FEBRUAR CASES:
 \begin{table}[H] \begin{threeparttable} \centering \caption{Dep. variable: \textbf{Diseases of the respiratory system}} {\def\sym#1{\ifmmode^{#1}\else\(^{#1}\)\fi} \begin{tabular}{l*{13}{c}} \toprule year & \multicolumn{12}{c}{Month of birth} \\ \cmidrule(lr){2-13} 
            &          11&          12&           1&           2&           3&           4&           5&           6&           7&           8&           9&          10\\
1995        &         664&         691&         715&         735&         759&         731&         814&         682&         690&         682&         658&         716\\
1996        &         654&         633&         736&         645&         722&         670&         716&         677&         694&         720&         645&         683\\
1997        &         635&         611&         639&         654&         687&         655&         692&         677&         730&         648&         641&         717\\
1998        &         586&         618&         618&         619&         667&         651&         685&         676&         635&         626&         658&         620\\
1999        &         517&         555&         556&         563&         567&         606&         675&         615&         629&         636&         605&         611\\
2000        &         479&         508&         510&         522&         581&         539&         521&         530&         558&         533&         491&         565\\
2001        &         494&         487&         494&         510&         548&         525&         579&         546&         609&         516&         561&         561\\
2002        &         498&         497&         515&         509&         532&         483&         562&         540&         544&         518&         507&         515\\
2003        &         469&         447&         470&         498&         575&         489&         524&         502&         538&         488&         522&         489\\
2004        &         417&         459&         483&         485&         469&         467&         519&         463&         508&         480&         457&         469\\
2005        &         477&         435&         434&         442&         469&         437&         437&         468&         501&         468&         435&         457\\
2006        &         395&         388&         435&         453&         494&         428&         417&         479&         429&         414&         439&         462\\
2007        &         433&         428&         473&         493&         482&         487&         484&         425&         502&         456&         489&         441\\
2008        &         412&         388&         365&         444&         440&         458&         481&         474&         453&         443&         424&         442\\
2009        &         460&         447&         451&         490&         487&         462&         486&         481&         498&         449&         464&         476\\
2010        &         450&         424&         428&         466&         457&         410&         420&         425&         469&         398&         416&         408\\
2011        &         385&         445&         448&         439&         467&         461&         464&         452&         502&         453&         452&         414\\
2012        &         408&         393&         456&         427&         495&         489&         441&         485&         459&         478&         440&         445\\
2013        &         431&         404&         458&         430&         490&         478&         455&         483&         469&         473&         456&         436\\
2014        &         408&         428&         412&         406&         468&         452&         497&         401&         457&         442&         442&         449\\
 \bottomrule \end{tabular} } \begin{tablenotes} \item \scriptsize \emph{Notes:} Number of cases per year and MOB in treatment cohort. \end{tablenotes} \end{threeparttable} \end{table} 

 \begin{table}[H] \begin{threeparttable} \centering \caption{Dep. variable: \textbf{Diseases of the respiratory system}} {\def\sym#1{\ifmmode^{#1}\else\(^{#1}\)\fi} \begin{tabular}{l*{13}{c}} \toprule year & \multicolumn{12}{c}{Month of birth} \\ \cmidrule(lr){2-13} 
            &          11&          12&           1&           2&           3&           4&           5&           6&           7&           8&           9&          10\\
1995        &          63&          83&          77&          52&          54&         109&         198&          20&          54&          37&          58&          39\\
1996        &           3&          17&          99&         -27&           6&          -3&          12&          32&          -6&          31&          -6&         -31\\
1997        &          -4&         -57&         -67&          11&         -36&          38&          40&          11&         -30&         -84&         -45&           1\\
1998        &          34&          25&           0&         -24&         -43&         -10&          -8&          65&         -11&         -64&         -20&        -102\\
1999        &         -35&         -35&         -53&         -58&        -125&         -75&          21&          20&         -11&         -37&         -55&         -42\\
2000        &         -53&          -2&          10&           8&          -2&         -16&         -60&         -68&          -3&         -79&         -75&         -72\\
2001        &         -10&         -56&         -27&           5&         -24&         -55&           6&         -26&          40&         -38&           0&          -3\\
2002        &           4&           2&         -34&          -5&         -58&         -13&          -4&          28&         -28&         -77&         -50&         -58\\
2003        &          14&          -4&          -2&          16&          67&         -27&          42&           7&          14&         -50&          11&         -26\\
2004        &         -16&          32&           3&          50&          -4&           6&          13&          -3&         -15&          13&         -56&         -43\\
2005        &          44&         -29&         -24&         -27&         -15&           7&         -66&         -39&          -3&         -33&         -29&         -90\\
2006        &          11&         -37&          -2&          66&          30&         -26&         -16&          76&         -19&         -38&          15&          28\\
2007        &          57&          29&          41&          60&          20&          73&          26&          -2&          61&          11&          86&         -37\\
2008        &          12&          -6&         -36&          17&         -47&          -6&          23&          29&         -57&           2&         -20&         -30\\
2009        &          14&           5&         -34&          32&         -40&           4&         -14&          21&         -25&         -27&         -59&         -29\\
2010        &          66&         -17&           4&          15&          -7&         -62&          -9&         -36&           8&         -69&         -10&         -15\\
2011        &         -63&          13&          18&         -13&         -43&         -23&         -32&          -3&          38&         -22&         -14&         -64\\
2012        &           8&         -10&         -13&         -59&           5&          42&         -15&          25&         -26&          15&         -43&         -30\\
2013        &          57&         -19&          42&         -50&           3&          36&         -44&          12&         -31&         -16&          17&         -37\\
2014        &          23&           2&         -73&         -56&         -77&          -3&          25&         -46&         -32&         -61&         -42&         -27\\
 \bottomrule \end{tabular} } \begin{tablenotes} \item \scriptsize \emph{Notes:} Difference of cases (control - treatment) per year and MOB in treatment cohort. \end{tablenotes} \end{threeparttable} \end{table} 

