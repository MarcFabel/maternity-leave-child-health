 \begin{table}[H] \centering \begin{threeparttable} \caption{Life-course approach - Table format} {\def\sym#1{\ifmmode^{#1}\else\(^{#1}\)\fi} \begin{tabular}{l*{5}{c}} \toprule \multicolumn{5}{l}{Dep. variable: \textbf{Neoplasms}} \\ & \multicolumn{4}{c}{Estimation window} \\ \cmidrule(lr){2-5}
            &\multicolumn{1}{c}{(1)}&\multicolumn{1}{c}{(2)}&\multicolumn{1}{c}{(3)}&\multicolumn{1}{c}{(4)}\\
            &\multicolumn{1}{c}{Age 17-21}&\multicolumn{1}{c}{Age 22-26}&\multicolumn{1}{c}{Age 27-31}&\multicolumn{1}{c}{Age 32-35}\\
\midrule
 \multicolumn{5}{l}{\emph{Panel A. Average causal effects}} \\ Abs. numbers        &     -0.1000         &       23.60\sym{*}  &       17.40\sym{*}  &       4.625         \\
                    &     (16.98)         &     (13.94)         &     (10.20)         &     (17.37)         \\
 Ratio fertility     &      -0.198         &       0.336         &       0.121         &      -0.217         \\
                    &     (0.297)         &     (0.280)         &     (0.201)         &     (0.309)         \\
 Ratio population    &       0.127         &       0.106         &      -0.149         \\
                    &     (0.357)         &     (0.152)         &     (0.233)         \\
 Cum. numbers        &       3.767         &       77.77         &       148.4         &       219.1         \\
                    &     (73.66)         &     (109.5)         &     (131.2)         &     (184.5)         \\
 Cum. ratio          &      -0.874         &      -0.132         &       0.322         &       0.520         \\
                    &     (1.462)         &     (1.960)         &     (2.329)         &     (3.132)         \\
 \midrule\multicolumn{5}{l}{\emph{Panel B. Treatment effect heterogeneity - Women}} \\ Abs. numbers        &      -12.03         &       14.20         &       11.50\sym{**} &       15.75         \\
                    &     (11.07)         &     (9.615)         &     (5.805)         &     (12.94)         \\
 Ratio fertility     &      -0.789\sym{*}  &       0.387         &       0.162         &       0.188         \\
                    &     (0.421)         &     (0.351)         &     (0.230)         &     (0.429)         \\
 Ratio population    &     -0.0170         &       0.138         &       0.168         \\
                    &     (0.406)         &     (0.169)         &     (0.313)         \\
 Cum. numbers        &      -44.93         &      -19.20         &       21.23         &       93.54         \\
                    &     (50.71)         &     (85.57)         &     (93.97)         &     (121.4)         \\
 Cum. ratio          &      -3.165         &      -3.205         &      -2.837         &      -1.644         \\
                    &     (2.137)         &     (3.093)         &     (3.025)         &     (3.646)         \\
 \midrule\multicolumn{5}{l}{\emph{Panel C. Treatment effect heterogeneity - Men}} \\ Abs. numbers        &       11.93         &       9.400         &       5.900         &      -11.13         \\
                    &     (10.34)         &     (8.633)         &     (7.994)         &     (7.592)         \\
 Ratio fertility     &       0.352         &       0.282         &      0.0677         &      -0.627\sym{**} \\
                    &     (0.395)         &     (0.369)         &     (0.322)         &     (0.320)         \\
 Ratio population    &       0.265         &      0.0630         &      -0.486\sym{*}  \\
                    &     (0.420)         &     (0.252)         &     (0.250)         \\
 Cum. numbers        &       48.70         &       96.97         &       127.1         &       125.5         \\
                    &     (42.45)         &     (69.92)         &     (91.47)         &     (116.2)         \\
 Cum. ratio          &       1.267         &       2.721         &       3.207         &       2.371         \\
                    &     (1.674)         &     (2.892)         &     (3.853)         &     (4.761)         \\
 
\bottomrule \end{tabular} } \begin{tablenotes} \item \scriptsize \emph{Notes:} Clustered standard errors in parentheses. All regression are run with CG2 (i.e. the cohort prior to the reform) and with month-of-birth FEs. Ratios indicate cases per thousand; either approximated population (with weights coming from the original fertility distribution) or original number of births. Raqtio population muss eins nach rechts gerückt werden \end{tablenotes} \end{threeparttable} \end{table} 
