%---------------------------------
% INPUT FOR VARIABLE: d1
%---------------------------------
\subsection{d1}
% RD overview
\begin{landscape}
\begin{figure}[H]
	\centering
	\begin{minipage}{.95\linewidth}
	\includegraphics[width=\linewidth]{rd_d1_overview_panel1}
	{\scriptsize \emph{Notes:} The figures show monthly RD plots with averages obtained from a bin width of one month. The solid vertical line divides pre- and post-reform regime. The averages are taken over the period of at most 1995-2014. \par}
\end{minipage}
\end{figure}
\end{landscape}
\begin{landscape}
\begin{figure}[H]
	\centering
\begin{minipage}{.95\linewidth}
	\includegraphics[width=\linewidth]{rd_d1_overview_panel2}
	{\scriptsize \emph{Notes:} The figures show monthly RD plots with a moving average window width of 3 months. The solid vertical line divides pre- and post-reform regime. The averages are taken over the period of at most 1995-2014. \par}
\end{minipage}
\end{figure}
\end{landscape}
%---------------------------------
% TABELLEN
 \begin{table}[H] \begin{threeparttable} \centering \caption{Dep. variable: \textbf{Certain infectious and parasitic diseases}} {\def\sym#1{\ifmmode^{#1}\else\(^{#1}\)\fi} \begin{tabular}{l*{8}{c}} \toprule & \multicolumn{7}{c}{Estimation window} \\ \cmidrule(lr){2-8}
            &\multicolumn{1}{c}{(1)}&\multicolumn{1}{c}{(2)}&\multicolumn{1}{c}{(3)}&\multicolumn{1}{c}{(4)}&\multicolumn{1}{c}{(5)}&\multicolumn{1}{c}{(6)}&\multicolumn{1}{c}{(7)}\\
            &\multicolumn{1}{c}{1M}&\multicolumn{1}{c}{2M}&\multicolumn{1}{c}{3M}&\multicolumn{1}{c}{4M}&\multicolumn{1}{c}{5M}&\multicolumn{1}{c}{6M}&\multicolumn{1}{c}{Donut}\\
\midrule
 \multicolumn{8}{l}{\emph{Panel A. Average causal effects}} \\ Abs. numbers        &      -7.000\sym{***}&      -7.575\sym{***}&      -1.717         &       2.400         &       3.740         &       5.150\sym{**} &       7.580\sym{***}\\
                    &  (3.41e-14)         &     (0.572)         &     (2.891)         &     (3.014)         &     (2.482)         &     (2.202)         &     (2.185)         \\
 Ratio fertility     &      -0.168\sym{***}&      -0.206\sym{***}&      -0.153\sym{***}&     -0.0813\sym{*}  &     -0.0646         &     -0.0838\sym{**} &     -0.0669\sym{*}  \\
                    &  (4.68e-16)         &    (0.0289)         &    (0.0353)         &    (0.0452)         &    (0.0378)         &    (0.0327)         &    (0.0377)         \\
 Ratio population    &      -0.181\sym{***}&      -0.139\sym{***}&     -0.0765\sym{**} &     -0.0143         &     -0.0316         &     -0.0242         &     0.00730         \\
                    &  (5.45e-16)         &    (0.0165)         &    (0.0339)         &    (0.0407)         &    (0.0338)         &    (0.0283)         &    (0.0274)         \\
 Ratio fert(03-14)   &      -0.168\sym{***}&      -0.128\sym{***}&     -0.0697\sym{*}  &     -0.0115         &     -0.0268         &     -0.0193         &      0.0104         \\
                    &  (2.22e-16)         &    (0.0153)         &    (0.0318)         &    (0.0381)         &    (0.0316)         &    (0.0265)         &    (0.0255)         \\
 Cum. numbers        &      -61.40\sym{***}&      -64.90\sym{***}&      -10.25         &       11.70         &       37.44         &       47.86\sym{*}  &       69.71\sym{***}\\
                    &  (1.04e-12)         &     (1.998)         &     (38.46)         &     (31.04)         &     (27.41)         &     (23.68)         &     (23.48)         \\
 Cum. ratio          &      -1.486\sym{***}&      -1.861\sym{***}&      -1.476\sym{**} &      -1.147\sym{*}  &      -0.718         &      -1.012\sym{**} &      -0.917\sym{*}  \\
                    &  (2.57e-14)         &     (0.376)         &     (0.615)         &     (0.541)         &     (0.495)         &     (0.435)         &     (0.498)         \\
 \midrule\multicolumn{8}{l}{\emph{Panel B. Treatment effect heterogeneity - Women}} \\ Abs. numbers        &      -3.200\sym{***}&      -6.075\sym{***}&      -0.917         &       1.037         &       2.060         &       3.575\sym{*}  &       4.930\sym{**} \\
                    &  (2.27e-14)         &     (1.105)         &     (2.913)         &     (2.495)         &     (2.044)         &     (1.876)         &     (1.963)         \\
 Ratio fertility     &      -0.188\sym{***}&      -0.320\sym{***}&      -0.177\sym{*}  &      -0.106         &     -0.0832         &     -0.0791         &     -0.0573         \\
                    &  (1.13e-15)         &    (0.0623)         &    (0.0918)         &    (0.0754)         &    (0.0611)         &    (0.0508)         &    (0.0545)         \\
 Ratio population    &      -0.221\sym{***}&      -0.238\sym{***}&      -0.144\sym{**} &     -0.0832         &     -0.0784         &     -0.0443         &    -0.00902         \\
                    &  (1.77e-16)         &   (0.00691)         &    (0.0524)         &    (0.0478)         &    (0.0462)         &    (0.0423)         &    (0.0465)         \\
 Ratio fert(03-14)   &      -0.211\sym{***}&      -0.228\sym{***}&      -0.136\sym{**} &     -0.0773         &     -0.0717         &     -0.0381         &    -0.00345         \\
                    &  (3.06e-16)         &   (0.00647)         &    (0.0501)         &    (0.0460)         &    (0.0448)         &    (0.0410)         &    (0.0451)         \\
 Cum. numbers        &      -19.15\sym{***}&      -57.30\sym{***}&      -1.133         &       7.913         &       26.11         &       38.87\sym{*}  &       50.47\sym{**} \\
                    &  (5.54e-13)         &     (14.65)         &     (36.05)         &     (28.36)         &     (24.58)         &     (21.44)         &     (22.63)         \\
 Cum. ratio          &      -0.481\sym{***}&      -1.509\sym{***}&      -0.361         &      -0.252         &       0.195         &       0.179         &       0.311         \\
                    &  (1.79e-15)         &     (0.400)         &     (0.822)         &     (0.612)         &     (0.529)         &     (0.439)         &     (0.489)         \\
 \midrule\multicolumn{8}{l}{\emph{Panel C. Treatment effect heterogeneity - Men}} \\ Abs. numbers        &      -3.800\sym{***}&      -1.500         &      -0.800         &       1.362         &       1.680\sym{*}  &       1.575\sym{*}  &       2.650\sym{***}\\
                    &  (1.69e-14)         &     (1.055)         &     (0.845)         &     (1.170)         &     (0.942)         &     (0.785)         &     (0.686)         \\
 Ratio fertility     &     -0.0883\sym{***}&     -0.0406\sym{*}  &     -0.0558\sym{**} &     -0.0134         &    -0.00664         &     -0.0292         &     -0.0173         \\
                    &  (3.75e-16)         &    (0.0183)         &    (0.0238)         &    (0.0378)         &    (0.0320)         &    (0.0287)         &    (0.0303)         \\
 Ratio population    &      -0.167\sym{***}&     -0.0739\sym{*}  &     -0.0508         &      0.0239         &    -0.00840         &     -0.0325         &    -0.00551         \\
                    &  (2.20e-16)         &    (0.0368)         &    (0.0617)         &    (0.0639)         &    (0.0551)         &    (0.0471)         &    (0.0520)         \\
 Ratio fert(03-14)   &      -0.127\sym{***}&     -0.0333         &    -0.00547         &      0.0520         &      0.0168         &    -0.00121         &      0.0239         \\
                    &  (3.09e-16)         &    (0.0364)         &    (0.0657)         &    (0.0621)         &    (0.0538)         &    (0.0455)         &    (0.0485)         \\
 Cum. numbers        &      -42.25\sym{***}&      -7.600         &      -9.117         &       3.788         &       11.33         &       8.992         &       19.24\sym{**} \\
                    &  (3.89e-13)         &     (13.39)         &     (10.29)         &     (9.708)         &     (9.826)         &     (8.394)         &     (8.262)         \\
 Cum. ratio          &      -1.666\sym{***}&      -0.709         &      -1.430\sym{**} &      -1.033         &      -0.736         &      -1.223\sym{**} &      -1.134         \\
                    &  (2.64e-14)         &     (0.481)         &     (0.498)         &     (0.622)         &     (0.616)         &     (0.564)         &     (0.677)         \\
 
\bottomrule \end{tabular} } \begin{tablenotes} \item \scriptsize \emph{Notes:} Clustered standard errors in parentheses. All regression are run with CG2 (i.e. the cohort prior to the reform) and with month-of-birth FEs. Ratios indicate cases per thousand; either approximated population (with weights coming from the original fertility distribution) or original number of births. \end{tablenotes} \end{threeparttable} \end{table} 

 \begin{table}[H] \begin{threeparttable} \centering \caption{Robustness with respect to the choice of \texttt{control group}} {\def\sym#1{\ifmmode^{#1}\else\(^{#1}\)\fi} \begin{tabular}{l*{10}{c}} \toprule & \multicolumn{9}{c}{Dependent variable: \textbf{Certain infectious and parasitic diseases}} \\ \cmidrule(lr){2-10}
            &\multicolumn{3}{c}{Average Causal Effects}&\multicolumn{3}{c}{Women}             &\multicolumn{3}{c}{Men}               \\\cmidrule(lr){2-4}\cmidrule(lr){5-7}\cmidrule(lr){8-10}
            &\multicolumn{1}{c}{(1)}&\multicolumn{1}{c}{(2)}&\multicolumn{1}{c}{(3)}&\multicolumn{1}{c}{(4)}&\multicolumn{1}{c}{(5)}&\multicolumn{1}{c}{(6)}&\multicolumn{1}{c}{(7)}&\multicolumn{1}{c}{(8)}&\multicolumn{1}{c}{(9)}\\
            &\multicolumn{1}{c}{C2}&\multicolumn{1}{c}{C1+C2}&\multicolumn{1}{c}{C1-C3}&\multicolumn{1}{c}{C2}&\multicolumn{1}{c}{C1+C2}&\multicolumn{1}{c}{C1-C3}&\multicolumn{1}{c}{C2}&\multicolumn{1}{c}{C1+C2}&\multicolumn{1}{c}{C1-C3}\\
\midrule
 \multicolumn{10}{l}{\emph{Panel A. 2 Month bandwidth}} \\ Abs. numbers        &         -10\sym{**} &      -7.778\sym{**} &      -5.130         &      -9.056\sym{***}&      -4.000         &      -2.185         &      -0.944         &      -3.778\sym{*}  &      -2.944         \\
                    &     (3.302)         &     (3.240)         &     (5.337)         &     (2.425)         &     (3.491)         &     (4.425)         &     (1.559)         &     (2.101)         &     (1.804)         \\
 Ratio fertility     &      -0.206\sym{***}&     -0.0987         &     -0.0806         &      -0.320\sym{***}&     -0.0918         &     -0.0787         &     -0.0992\sym{***}&      -0.107         &     -0.0836         \\
                    &    (0.0289)         &    (0.0988)         &    (0.0861)         &    (0.0623)         &     (0.171)         &     (0.145)         &    (0.0248)         &    (0.0776)         &    (0.0698)         \\
 Ratio population    &      -0.188\sym{***}&      -0.132\sym{*}  &     -0.0767         &      -0.304\sym{***}&      -0.161         &     -0.0841         &     -0.0739\sym{*}  &      -0.105         &     -0.0706         \\
                    &    (0.0168)         &    (0.0709)         &    (0.0785)         &    (0.0245)         &     (0.112)         &     (0.117)         &    (0.0368)         &    (0.0680)         &    (0.0593)         \\
 \midrule\multicolumn{10}{l}{\emph{Panel B. 4 Month bandwidth}} \\ Abs. numbers        &       3.472         &       3.000         &       4.056         &      -0.500         &       0.944         &       1.361         &       3.972\sym{**} &       2.056         &       2.694         \\
                    &     (4.189)         &     (3.653)         &     (4.668)         &     (2.828)         &     (2.477)         &     (3.054)         &     (1.671)         &     (1.964)         &     (2.232)         \\
 Ratio fertility     &     -0.0148         &      0.0171         &      0.0143         &     -0.0142         &      0.0606         &      0.0430         &     -0.0134         &     -0.0217         &     -0.0108         \\
                    &    (0.0273)         &    (0.0312)         &    (0.0290)         &    (0.0509)         &    (0.0601)         &    (0.0585)         &    (0.0378)         &    (0.0345)         &    (0.0350)         \\
 Ratio population    &     -0.0596         &     -0.0281         &      0.0164         &      -0.144\sym{**} &     -0.0632         &    -0.00476         &      0.0239         &     0.00650         &      0.0373         \\
                    &    (0.0424)         &    (0.0489)         &    (0.0519)         &    (0.0506)         &    (0.0706)         &    (0.0774)         &    (0.0639)         &    (0.0593)         &    (0.0582)         \\
 \midrule\multicolumn{10}{l}{\emph{Panel C. 6 Month bandwidth}} \\ Abs. numbers        &       4.537         &       2.787         &       3.173         &       2.685         &       2.056         &       2.444         &       1.852         &       0.731         &       0.728         \\
                    &     (3.000)         &     (2.626)         &     (3.456)         &     (2.279)         &     (1.912)         &     (2.332)         &     (1.560)         &     (1.520)         &     (1.841)         \\
 Ratio population    &     -0.0687         &     -0.0703         &     -0.0264         &     -0.0365         &     -0.0465         &     -0.0144         &     -0.0987         &     -0.0927         &     -0.0381         \\
                    &    (0.0603)         &    (0.0561)         &    (0.0512)         &    (0.0998)         &    (0.0954)         &    (0.0905)         &    (0.0843)         &    (0.0716)         &    (0.0758)         \\
 Ratio fertility     &     -0.0137         &    -0.00337         &      0.0231         &    -0.00254         &      0.0131         &      0.0500         &     -0.0243         &     -0.0188         &    -0.00181         \\
                    &    (0.0310)         &    (0.0322)         &    (0.0306)         &    (0.0494)         &    (0.0511)         &    (0.0475)         &    (0.0456)         &    (0.0415)         &    (0.0447)         \\
 \midrule\multicolumn{10}{l}{\emph{Panel D. Donut specification}} \\ Abs. numbers        &       8.311\sym{***}&       4.911\sym{*}  &       5.015         &       5.133\sym{**} &       3.378         &       3.637         &       3.178\sym{*}  &       1.533         &       1.378         \\
                    &     (2.862)         &     (2.838)         &     (3.726)         &     (2.345)         &     (2.100)         &     (2.488)         &     (1.713)         &     (1.764)         &     (2.140)         \\
 Ratio fertility     &     -0.0669\sym{*}  &     -0.0926\sym{*}  &     -0.0470         &     -0.0573         &     -0.0687         &     -0.0183         &     -0.0788         &      -0.118\sym{**} &     -0.0773         \\
                    &    (0.0377)         &    (0.0489)         &    (0.0403)         &    (0.0545)         &    (0.0746)         &    (0.0626)         &    (0.0475)         &    (0.0461)         &    (0.0504)         \\
 Ratio fertility     &      0.0229         &      0.0105         &      0.0391         &      0.0532         &      0.0369         &      0.0759         &    -0.00576         &     -0.0143         &     0.00505         \\
                    &    (0.0273)         &    (0.0351)         &    (0.0330)         &    (0.0473)         &    (0.0570)         &    (0.0506)         &    (0.0526)         &    (0.0482)         &    (0.0532)         \\
 
\bottomrule \end{tabular} } \begin{tablenotes} \item \scriptsize \emph{Notes:} Clustered standard errors in parentheses. All regressions contain Birthmonth FE. Ratios indicate cases per thousand; either approximated population or original number of births. \end{tablenotes} \end{threeparttable} \end{table} 

%---------------------------------
% Life-course figure (Panel1)
\begin{landscape}
\begin{figure}[H]
\centering
\begin{minipage}{.9\linewidth}
\includegraphics[width=\linewidth]{lc_d1_overview_panel1}
{\scriptsize \emph{Notes:} The figures depict DDRD estimates and 90\% confidence intervals over the life-course. The years are harmonized such that the cohorts are in the same age when they are compared. All regressions are carried out with month-of-birth FE and make use of clustered standard errors. Furthermore, we used a bandwidth of half a year and only the control cohort that was born one year prior to the reform. Ratios indicate cases per thousand; using in the denominator the approximated population (with weights coming from the original fertility distribution) or original number of births. \par}
\end{minipage}
\end{figure}
\end{landscape}
%---------------------------------
% Life-course figure (Panel2)
\begin{landscape}
\begin{figure}[H]
\centering
\begin{minipage}{.9\linewidth}
\includegraphics[width=\linewidth]{lc_d1_overview_panel2}
{\scriptsize \emph{Notes:} The figures depict DDRD estimates and 90\% confidence intervals over the life-course. The years are harmonized such that the cohorts are in the same age when they are compared. All regressions are carried out with month-of-birth FE and make use of clustered standard errors. Furthermore, we used a bandwidth of half a year. Ratios indicate cases per thousand; using in the denominator the approximated population (with weights coming from the original fertility distribution) or original number of births. \par}
\end{minipage}
\end{figure}
\end{landscape}
%---------------------------------
% Life-course (panel 3 - 6)
\begin{figure}[H]%\vspace*{-2cm}
	\centering
	\includegraphics[width=.9\linewidth]{lc_d1_overview_panel3}
	\includegraphics[width=.9\linewidth]{lc_d1_overview_panel4}
\end{figure}
\begin{figure}[H]
	\centering	
	\includegraphics[width=.97\linewidth]{lc_d1_overview_panel5}
	\includegraphics[width=.97\linewidth]{lc_d1_overview_panel6}
\end{figure}
% Life-course TABLE Format
 \begin{table}[H] \centering \begin{threeparttable} \caption{Life-course approach - Table format} {\def\sym#1{\ifmmode^{#1}\else\(^{#1}\)\fi} \begin{tabular}{l*{5}{c}} \toprule \multicolumn{5}{l}{Dep. variable: \textbf{Certain infectious and parasitic diseases}} \\ & \multicolumn{4}{c}{Estimation window} \\ \cmidrule(lr){2-5}
            &\multicolumn{1}{c}{(1)}&\multicolumn{1}{c}{(2)}&\multicolumn{1}{c}{(3)}&\multicolumn{1}{c}{(4)}\\
            &\multicolumn{1}{c}{Age 17-21}&\multicolumn{1}{c}{Age 22-26}&\multicolumn{1}{c}{Age 27-31}&\multicolumn{1}{c}{Age 32-35}\\
\midrule
 \multicolumn{5}{l}{\emph{Panel A. Average causal effects}} \\ Abs. numbers        &       9.933         &       1.967         &       2.567         &       8.708         \\
                    &     (7.792)         &     (8.529)         &     (5.289)         &     (12.62)         \\
 Ratio fertility     &      0.0130         &      -0.162         &      -0.126         &     -0.0197         \\
                    &     (0.143)         &     (0.197)         &     (0.102)         &     (0.235)         \\
 Ratio population    &      -0.198         &     -0.0865         &    -0.00711         \\
                    &     (0.221)         &    (0.0761)         &     (0.180)         \\
 Cum. numbers        &       27.97         &       58.70         &       69.13         &       75.87         \\
                    &     (35.11)         &     (58.02)         &     (57.28)         &     (85.31)         \\
 Cum. ratio          &      -0.265         &      -0.676         &      -1.346         &      -2.079         \\
                    &     (0.663)         &     (1.164)         &     (1.294)         &     (1.567)         \\
 \midrule\multicolumn{5}{l}{\emph{Panel B. Treatment effect heterogeneity - Women}} \\ Abs. numbers        &       8.367         &       1.233         &       1.700         &       4.458         \\
                    &     (6.177)         &     (7.051)         &     (5.096)         &     (9.444)         \\
 Ratio fertility     &       0.107         &      -0.193         &      -0.132         &     -0.0378         \\
                    &     (0.230)         &     (0.304)         &     (0.161)         &     (0.373)         \\
 Ratio population    &      -0.343         &     -0.0876         &     -0.0172         \\
                    &     (0.324)         &     (0.121)         &     (0.275)         \\
 Cum. numbers        &       23.87         &       48.03         &       53.30         &       54.12         \\
                    &     (27.34)         &     (46.21)         &     (56.03)         &     (77.80)         \\
 Cum. ratio          &     -0.0738         &      -0.330         &      -1.140         &      -2.109         \\
                    &     (0.975)         &     (1.591)         &     (1.840)         &     (2.333)         \\
 \midrule\multicolumn{5}{l}{\emph{Panel C. Treatment effect heterogeneity - Men}} \\ Abs. numbers        &       1.567         &       0.733         &       0.867         &       4.250         \\
                    &     (3.972)         &     (6.625)         &     (6.356)         &     (5.521)         \\
 Ratio fertility     &     -0.0824         &      -0.135         &      -0.120         &    -0.00380         \\
                    &     (0.181)         &     (0.281)         &     (0.278)         &     (0.202)         \\
 Ratio population    &     -0.0546         &     -0.0860         &     0.00165         \\
                    &     (0.323)         &     (0.216)         &     (0.161)         \\
 Cum. numbers        &       4.100         &       10.67         &       15.83         &       21.75         \\
                    &     (21.22)         &     (26.82)         &     (23.17)         &     (24.83)         \\
 Cum. ratio          &      -0.473         &      -1.051         &      -1.597         &      -2.111         \\
                    &     (0.947)         &     (1.426)         &     (1.607)         &     (1.746)         \\
 
\bottomrule \end{tabular} } \begin{tablenotes} \item \scriptsize \emph{Notes:} Clustered standard errors in parentheses. All regression are run with CG2 (i.e. the cohort prior to the reform) and with month-of-birth FEs. Ratios indicate cases per thousand; either approximated population (with weights coming from the original fertility distribution) or original number of births. Raqtio population muss eins nach rechts gerückt werden \end{tablenotes} \end{threeparttable} \end{table} 

%---------------------------------
% PLACEBO EXERCISES
\newpage
\begin{landscape}
\begin{figure}[H]
	\centering
    \begin{minipage}{.9\linewidth}
	\includegraphics[width=\linewidth]{placebo_graph_d1.pdf}
    {\scriptsize \emph{Notes:} The figures depict DDRD estimates and 95\% confidence intervals when the treatment cohort is shifted over time. The date on the abscissa indicates the starting date of the treated.  All regressions are carried out with month-of-birth FE and make use of clustered standard errors. Furthermore, we used a bandwidth of half a year. Ratios indicate cases per thousand; using in the denominator the approximated population (with weights coming from the original fertility distribution) or original number of births. \par}
    \end{minipage}
\end{figure}
\end{landscape}
 \begin{table}[H] \centering \begin{threeparttable} \caption{Placebo 1 (CONTROL1 ist TREAT) } {\def\sym#1{\ifmmode^{#1}\else\(^{#1}\)\fi} \begin{tabular}{l*{4}{c}} \toprule \multicolumn{4}{l}{Dep. variable: \textbf{Certain infectious and parasitic diseases}} \\ & \multicolumn{3}{c}{Choice of control group} \\ \cmidrule(lr){2-4}
            &\multicolumn{1}{c}{(1)}&\multicolumn{1}{c}{(2)}&\multicolumn{1}{c}{(3)}\\
            &\multicolumn{1}{c}{C2}&\multicolumn{1}{c}{C3}&\multicolumn{1}{c}{C2+C3}\\
\midrule
 \multicolumn{4}{l}{\emph{Panel A. Average causal effects}} \\ Abs. numbers        &       1.825         &      -4.033\sym{*}  &      -1.104         \\
                    &     (1.613)         &     (2.127)         &     (2.641)         \\
 Ratio fertility     &     -0.0120         &      0.0954\sym{**} &      0.0417         \\
                    &    (0.0482)         &    (0.0408)         &    (0.0474)         \\
 Ratio population    &     -0.0541         &      0.0745         &      0.0102         \\
                    &    (0.0397)         &    (0.0464)         &    (0.0465)         \\
 Cum. numbers        &       21.78         &      -69.37\sym{***}&      -23.79         \\
                    &     (19.54)         &     (21.24)         &     (24.68)         \\
 Cum. ratio          &     -0.0776         &       0.435         &       0.179         \\
                    &     (0.605)         &     (0.489)         &     (0.625)         \\
 \midrule\multicolumn{4}{l}{\emph{Panel B. Treatment effect heterogeneity - Women}} \\ Abs. numbers        &       0.233         &      -2.783\sym{**} &      -1.275         \\
                    &     (1.662)         &     (1.115)         &     (2.033)         \\
 Ratio fertility     &     -0.0389         &      0.0878         &      0.0245         \\
                    &    (0.0819)         &    (0.0679)         &    (0.0757)         \\
 Ratio population    &      -0.102\sym{*}  &       0.113\sym{**} &     0.00549         \\
                    &    (0.0514)         &    (0.0463)         &    (0.0596)         \\
 Cum. numbers        &       2.225         &      -49.57\sym{***}&      -23.68         \\
                    &     (22.03)         &     (15.08)         &     (22.52)         \\
 Cum. ratio          &      -0.439         &       0.111         &      -0.164         \\
                    &     (1.050)         &     (0.908)         &     (0.979)         \\
 \midrule\multicolumn{4}{l}{\emph{Panel C. Treatment effect heterogeneity - Men}} \\ Abs. numbers        &       1.592         &      -1.250         &       0.171         \\
                    &     (1.159)         &     (1.742)         &     (1.527)         \\
 Ratio fertility     &      0.0137         &       0.103\sym{*}  &      0.0581         \\
                    &    (0.0569)         &    (0.0508)         &    (0.0621)         \\
 Ratio population    &    -0.00660         &      0.0363         &      0.0148         \\
                    &    (0.0503)         &    (0.0701)         &    (0.0602)         \\
 Cum. numbers        &       19.56         &      -19.79         &      -0.117         \\
                    &     (14.37)         &     (18.84)         &     (17.08)         \\
 Cum. ratio          &       0.266         &       0.738         &       0.502         \\
                    &     (0.733)         &     (0.531)         &     (0.792)         \\
 
\bottomrule \end{tabular} } \begin{tablenotes} \item \scriptsize \emph{Notes:} Clustered standard errors in parentheses. All regression are run with month-of-birth FEs and control cohort 2 is assigned with the treatment status. All regressions are carried out with a window width of half a year. \end{tablenotes} \end{threeparttable} \end{table} 

 \begin{table}[H] \centering \begin{threeparttable} \caption{Placebo 2 (CONTROL2 ist TREAT) } {\def\sym#1{\ifmmode^{#1}\else\(^{#1}\)\fi} \begin{tabular}{l*{4}{c}} \toprule \multicolumn{4}{l}{Dep. variable: \textbf{Certain infectious and parasitic diseases}} \\ & \multicolumn{3}{c}{Choice of control group} \\ \cmidrule(lr){2-4}
            &\multicolumn{1}{c}{(1)}&\multicolumn{1}{c}{(2)}&\multicolumn{1}{c}{(3)}\\
            &\multicolumn{1}{c}{C1}&\multicolumn{1}{c}{C3}&\multicolumn{1}{c}{C1+C3}\\
\midrule
 \multicolumn{4}{l}{\emph{Panel A. Average causal effects}} \\ Abs. numbers        &      -1.825         &      -5.858\sym{***}&      -3.842         \\
                    &     (1.613)         &     (1.654)         &     (3.158)         \\
 Ratio fertility     &      0.0120         &       0.107\sym{***}&      0.0596         \\
                    &    (0.0482)         &    (0.0335)         &    (0.0442)         \\
 Ratio population    &      0.0541         &       0.129\sym{***}&      0.0913\sym{**} \\
                    &    (0.0397)         &    (0.0392)         &    (0.0449)         \\
 Cum. numbers        &      -21.78         &      -91.15\sym{***}&      -56.47\sym{*}  \\
                    &     (19.54)         &     (19.76)         &     (29.68)         \\
 Cum. ratio          &      0.0776         &       0.512         &       0.295         \\
                    &     (0.605)         &     (0.377)         &     (0.500)         \\
 \midrule\multicolumn{4}{l}{\emph{Panel B. Treatment effect heterogeneity - Women}} \\ Abs. numbers        &      -0.233         &      -3.017\sym{**} &      -1.625         \\
                    &     (1.662)         &     (1.103)         &     (2.609)         \\
 Ratio fertility     &      0.0389         &       0.127\sym{***}&      0.0828         \\
                    &    (0.0819)         &    (0.0447)         &    (0.0790)         \\
 Ratio population    &       0.102\sym{*}  &       0.215\sym{***}&       0.158\sym{**} \\
                    &    (0.0514)         &    (0.0429)         &    (0.0633)         \\
 Cum. numbers        &      -2.225         &      -51.80\sym{***}&      -27.01         \\
                    &     (22.03)         &     (14.28)         &     (27.70)         \\
 Cum. ratio          &       0.439         &       0.550         &       0.495         \\
                    &     (1.050)         &     (0.567)         &     (0.893)         \\
 \midrule\multicolumn{4}{l}{\emph{Panel C. Treatment effect heterogeneity - Men}} \\ Abs. numbers        &      -1.592         &      -2.842\sym{**} &      -2.217         \\
                    &     (1.159)         &     (1.295)         &     (1.318)         \\
 Ratio fertility     &     -0.0137         &      0.0889\sym{*}  &      0.0376         \\
                    &    (0.0569)         &    (0.0510)         &    (0.0568)         \\
 Ratio population    &     0.00660         &      0.0429         &      0.0247         \\
                    &    (0.0503)         &    (0.0502)         &    (0.0496)         \\
 Cum. numbers        &      -19.56         &      -39.35\sym{**} &      -29.45\sym{*}  \\
                    &     (14.37)         &     (16.41)         &     (15.43)         \\
 Cum. ratio          &      -0.266         &       0.472         &       0.103         \\
                    &     (0.733)         &     (0.616)         &     (0.744)         \\
 
\bottomrule \end{tabular} } \begin{tablenotes} \item \scriptsize \emph{Notes:} Clustered standard errors in parentheses. All regression are run with month-of-birth FEs and control cohort 2 is assigned with the treatment status. All regressions are carried out with a window width of half a year. \end{tablenotes} \end{threeparttable} \end{table} 

 \begin{table}[H] \centering \begin{threeparttable} \caption{Placebo 3 (CONTROL3 ist TREAT) } {\def\sym#1{\ifmmode^{#1}\else\(^{#1}\)\fi} \begin{tabular}{l*{4}{c}} \toprule \multicolumn{4}{l}{Dep. variable: \textbf{Certain infectious and parasitic diseases}} \\ & \multicolumn{3}{c}{Choice of control group} \\ \cmidrule(lr){2-4}
            &\multicolumn{1}{c}{(1)}&\multicolumn{1}{c}{(2)}&\multicolumn{1}{c}{(3)}\\
            &\multicolumn{1}{c}{C1}&\multicolumn{1}{c}{C2}&\multicolumn{1}{c}{C1+C2}\\
\midrule
 \multicolumn{4}{l}{\emph{Panel A. Average causal effects}} \\ Abs. numbers        &       4.033\sym{*}  &       5.858\sym{***}&       4.946\sym{**} \\
                    &     (2.127)         &     (1.654)         &     (2.102)         \\
 Ratio fertility     &     -0.0954\sym{**} &      -0.107\sym{***}&      -0.101\sym{**} \\
                    &    (0.0408)         &    (0.0335)         &    (0.0453)         \\
 Ratio population    &     -0.0745         &      -0.129\sym{***}&      -0.102\sym{**} \\
                    &    (0.0464)         &    (0.0392)         &    (0.0434)         \\
 Cum. numbers        &       69.37\sym{***}&       91.15\sym{***}&       80.26\sym{***}\\
                    &     (21.24)         &     (19.76)         &     (23.02)         \\
 Cum. ratio          &      -0.435         &      -0.512         &      -0.473         \\
                    &     (0.489)         &     (0.377)         &     (0.525)         \\
 \midrule\multicolumn{4}{l}{\emph{Panel B. Treatment effect heterogeneity - Women}} \\ Abs. numbers        &       2.783\sym{**} &       3.017\sym{**} &       2.900\sym{**} \\
                    &     (1.115)         &     (1.103)         &     (1.326)         \\
 Ratio fertility     &     -0.0878         &      -0.127\sym{***}&      -0.107         \\
                    &    (0.0679)         &    (0.0447)         &    (0.0689)         \\
 Ratio population    &      -0.113\sym{**} &      -0.215\sym{***}&      -0.164\sym{***}\\
                    &    (0.0463)         &    (0.0429)         &    (0.0475)         \\
 Cum. numbers        &       49.58\sym{***}&       51.80\sym{***}&       50.69\sym{***}\\
                    &     (15.08)         &     (14.28)         &     (16.92)         \\
 Cum. ratio          &      -0.111         &      -0.550         &      -0.330         \\
                    &     (0.908)         &     (0.567)         &     (0.873)         \\
 \midrule\multicolumn{4}{l}{\emph{Panel C. Treatment effect heterogeneity - Men}} \\ Abs. numbers        &       1.250         &       2.842\sym{**} &       2.046         \\
                    &     (1.742)         &     (1.295)         &     (1.535)         \\
 Ratio fertility     &      -0.103\sym{*}  &     -0.0889\sym{*}  &     -0.0958\sym{*}  \\
                    &    (0.0508)         &    (0.0510)         &    (0.0523)         \\
 Ratio population    &     -0.0363         &     -0.0429         &     -0.0396         \\
                    &    (0.0701)         &    (0.0502)         &    (0.0601)         \\
 Cum. numbers        &       19.79         &       39.35\sym{**} &       29.57         \\
                    &     (18.84)         &     (16.41)         &     (17.69)         \\
 Cum. ratio          &      -0.738         &      -0.472         &      -0.605         \\
                    &     (0.531)         &     (0.616)         &     (0.590)         \\
 
\bottomrule \end{tabular} } \begin{tablenotes} \item \scriptsize \emph{Notes:} Clustered standard errors in parentheses. All regression are run with month-of-birth FEs and control cohort 3 is assigned with the treatment status. All regressions are carried out with a window width of half a year. \end{tablenotes} \end{threeparttable} \end{table} 

%---------------------------------
% CUMMULATIVE APPROACH
\begin{landscape}
 \begin{table}[H] \begin{threeparttable} \centering \caption{Cummulative effects for upt to different points of age} {\def\sym#1{\ifmmode^{#1}\else\(^{#1}\)\fi} \begin{tabular}{l*{13}{c}} \toprule & \multicolumn{12}{c}{Dependent variable: \textbf{Certain infectious and parasitic diseases}} \\ \cmidrule(lr){2-13}
            &\multicolumn{4}{c}{Average Causal Effects}         &\multicolumn{4}{c}{Women}                          &\multicolumn{4}{c}{Men}                            \\\cmidrule(lr){2-5}\cmidrule(lr){6-9}\cmidrule(lr){10-13}
            &\multicolumn{1}{c}{(1)}&\multicolumn{1}{c}{(2)}&\multicolumn{1}{c}{(3)}&\multicolumn{1}{c}{(4)}&\multicolumn{1}{c}{(5)}&\multicolumn{1}{c}{(6)}&\multicolumn{1}{c}{(7)}&\multicolumn{1}{c}{(8)}&\multicolumn{1}{c}{(9)}&\multicolumn{1}{c}{(10)}&\multicolumn{1}{c}{(11)}&\multicolumn{1}{c}{(12)}\\
            &\multicolumn{1}{c}{2M}&\multicolumn{1}{c}{4M}&\multicolumn{1}{c}{6M}&\multicolumn{1}{c}{Donut}&\multicolumn{1}{c}{2M}&\multicolumn{1}{c}{4M}&\multicolumn{1}{c}{6M}&\multicolumn{1}{c}{Donut}&\multicolumn{1}{c}{2M}&\multicolumn{1}{c}{4M}&\multicolumn{1}{c}{6M}&\multicolumn{1}{c}{Donut}\\
\midrule
 \multicolumn{13}{l}{\emph{Panel A. 2 Up to the age of 21}} \\ Cum. numbers        &      -75.00\sym{***}&       9.750         &       45.33         &       67.00\sym{*}  &      -54.00\sym{*}  &       26.75         &       38.67         &       50.00         &         -21         &         -17         &       6.667         &       17.00         \\
                    &     (9.823)         &     (45.41)         &     (33.75)         &     (35.92)         &     (26.24)         &     (38.86)         &     (25.34)         &     (29.20)         &     (20.40)         &     (18.37)         &     (19.12)         &     (21.19)         \\
 Cum. ratio          &      -1.857\sym{**} &      -0.696         &      -0.301         &     -0.0480         &      -2.707\sym{**} &       0.107         &      0.0620         &       0.328         &      -1.064         &      -1.462         &      -0.681         &      -0.440         \\
                    &     (0.537)         &     (0.739)         &     (0.556)         &     (0.627)         &     (1.030)         &     (1.326)         &     (0.859)         &     (1.021)         &     (0.998)         &     (0.896)         &     (0.833)         &     (0.981)         \\
 \midrule\multicolumn{13}{l}{\emph{Panel B. Up to the age of 26}} \\ Cum. numbers        &      -29.00         &       13.75         &       55.17         &       67.60         &      -25.50         &       17.75         &       44.83         &       49.80         &      -3.500         &      -4.000         &       10.33         &       17.80         \\
                    &     (19.24)         &     (51.14)         &     (40.72)         &     (43.77)         &     (44.03)         &     (49.30)         &     (34.29)         &     (32.50)         &     (25.04)         &     (24.20)         &     (23.01)         &     (26.99)         \\
 Cum. ratio          &      -1.182         &      -1.315         &      -1.109         &      -1.221         &      -1.815         &      -1.092         &      -0.901         &      -1.105         &      -0.608         &      -1.531         &      -1.359         &      -1.383         \\
                    &     (1.076)         &     (1.192)         &     (0.931)         &     (1.068)         &     (2.431)         &     (1.800)         &     (1.200)         &     (1.076)         &     (0.446)         &     (1.443)         &     (1.263)         &     (1.535)         \\
 \midrule\multicolumn{13}{l}{\emph{Panel C. Up to the age of 31}} \\ Cum. numbers        &      -66.00\sym{**} &       24.25         &       68.00         &       100.2\sym{**} &      -68.00\sym{**} &       0.250         &       53.33         &       72.60         &       2.000         &       24.00         &       14.67         &       27.60         \\
                    &     (19.21)         &     (57.16)         &     (42.81)         &     (38.72)         &     (21.38)         &     (58.82)         &     (45.00)         &     (47.96)         &     (39.40)         &     (22.00)         &     (17.37)         &     (18.08)         \\
 Cum. ratio          &      -2.156\sym{*}  &      -1.684         &      -1.737\sym{*}  &      -1.599         &      -3.897\sym{***}&      -2.511         &      -1.562         &      -1.318         &      -0.534         &      -0.907         &      -1.959         &      -1.918         \\
                    &     (1.111)         &     (1.076)         &     (0.858)         &     (0.980)         &     (0.947)         &     (1.785)         &     (1.253)         &     (1.334)         &     (1.985)         &     (1.543)         &     (1.323)         &     (1.620)         \\
 \midrule\multicolumn{13}{l}{\emph{Panel D. Up to the age of 34}} \\ Cum. numbers        &      -154.0\sym{***}&       46.00         &       102.8         &       150.2\sym{**} &      -138.0\sym{**} &       15.75         &       71.17         &       100.4         &      -16.00         &       30.25         &       31.67         &       49.80\sym{**} \\
                    &     (34.18)         &     (96.00)         &     (66.50)         &     (68.87)         &     (45.22)         &     (82.57)         &     (59.51)         &     (64.55)         &     (38.33)         &     (32.39)         &     (21.56)         &     (19.63)         \\
 Cum. ratio          &      -4.188\sym{***}&      -1.797         &      -1.816         &      -1.494         &      -7.098\sym{**} &      -2.433         &      -1.713         &      -1.180         &      -1.460         &      -1.203         &      -1.974         &      -1.849         \\
                    &     (0.878)         &     (1.532)         &     (1.073)         &     (1.272)         &     (2.142)         &     (2.581)         &     (1.674)         &     (1.864)         &     (0.915)         &     (1.659)         &     (1.335)         &     (1.595)         \\
 
\bottomrule \end{tabular} } \begin{tablenotes} \item \scriptsize \emph{Notes:} Clustered standard errors in parentheses (MxY). All regressions contain Birthmonth FE. Ratios indicate cases per thousand; original number of births. \end{tablenotes} \end{threeparttable} \end{table} 

\end{landscape}
\begin{landscape}
 \begin{table}[H] \begin{threeparttable} \centering \caption{Cummulative effects for upt to different points of age - BOOTSTRAPPED} {\def\sym#1{\ifmmode^{#1}\else\(^{#1}\)\fi} \begin{tabular}{l*{13}{c}} \toprule & \multicolumn{12}{c}{Dependent variable: \textbf{Certain infectious and parasitic diseases}} \\ \cmidrule(lr){2-13}
            &\multicolumn{4}{c}{Average Causal Effects}         &\multicolumn{4}{c}{Women}                          &\multicolumn{4}{c}{Men}                            \\\cmidrule(lr){2-5}\cmidrule(lr){6-9}\cmidrule(lr){10-13}
            &\multicolumn{1}{c}{(1)}&\multicolumn{1}{c}{(2)}&\multicolumn{1}{c}{(3)}&\multicolumn{1}{c}{(4)}&\multicolumn{1}{c}{(5)}&\multicolumn{1}{c}{(6)}&\multicolumn{1}{c}{(7)}&\multicolumn{1}{c}{(8)}&\multicolumn{1}{c}{(9)}&\multicolumn{1}{c}{(10)}&\multicolumn{1}{c}{(11)}&\multicolumn{1}{c}{(12)}\\
            &\multicolumn{1}{c}{2M}&\multicolumn{1}{c}{4M}&\multicolumn{1}{c}{6M}&\multicolumn{1}{c}{Donut}&\multicolumn{1}{c}{2M}&\multicolumn{1}{c}{4M}&\multicolumn{1}{c}{6M}&\multicolumn{1}{c}{Donut}&\multicolumn{1}{c}{2M}&\multicolumn{1}{c}{4M}&\multicolumn{1}{c}{6M}&\multicolumn{1}{c}{Donut}\\
\midrule
 \multicolumn{13}{l}{\emph{Panel A. 2 Up to the age of 21}} \\ Cum. numbers        &      -75.00\sym{***}&       9.750         &       45.33         &       67.00         &      -54.00\sym{**} &       26.75         &       38.67         &       50.00         &         -21         &         -17         &       6.667         &       17.00         \\
                    &     (8.850)         &     (60.38)         &     (47.23)         &     (44.12)         &     (22.72)         &     (50.30)         &     (38.13)         &     (40.63)         &     (18.28)         &     (23.52)         &     (26.90)         &     (27.46)         \\
 Cum. ratio          &      -1.857\sym{***}&      -0.696         &      -0.301         &     -0.0480         &      -2.707\sym{***}&       0.107         &      0.0620         &       0.328         &      -1.064         &      -1.462         &      -0.681         &      -0.440         \\
                    &     (0.524)         &     (0.963)         &     (0.869)         &     (0.818)         &     (0.892)         &     (1.672)         &     (1.337)         &     (1.398)         &     (0.940)         &     (1.137)         &     (1.262)         &     (1.409)         \\
 \midrule\multicolumn{13}{l}{\emph{Panel B. Up to the age of 26}} \\ Cum. numbers        &      -29.00\sym{*}  &       13.75         &       55.17         &       67.60         &      -25.50         &       17.75         &       44.83         &       49.80         &      -3.500         &      -4.000         &       10.33         &       17.80         \\
                    &     (17.57)         &     (70.79)         &     (60.35)         &     (64.81)         &     (40.90)         &     (67.72)         &     (49.49)         &     (43.61)         &     (23.56)         &     (32.64)         &     (33.46)         &     (40.49)         \\
 Cum. ratio          &      -1.182         &      -1.315         &      -1.109         &      -1.221         &      -1.815         &      -1.092         &      -0.901         &      -1.105         &      -0.608         &      -1.531         &      -1.359         &      -1.383         \\
                    &     (1.046)         &     (1.605)         &     (1.347)         &     (1.594)         &     (2.323)         &     (2.355)         &     (1.759)         &     (1.465)         &     (0.384)         &     (2.035)         &     (1.773)         &     (2.255)         \\
 \midrule\multicolumn{13}{l}{\emph{Panel C. Up to the age of 31}} \\ Cum. numbers        &      -66.00\sym{***}&       24.25         &       68.00         &       100.2\sym{*}  &      -68.00\sym{***}&       0.250         &       53.33         &       72.60         &       2.000         &       24.00         &       14.67         &       27.60         \\
                    &     (16.54)         &     (76.29)         &     (60.04)         &     (58.67)         &     (19.18)         &     (83.92)         &     (64.08)         &     (67.04)         &     (34.43)         &     (31.93)         &     (24.46)         &     (21.81)         \\
 Cum. ratio          &      -2.156\sym{**} &      -1.684         &      -1.737         &      -1.599         &      -3.897\sym{***}&      -2.511         &      -1.562         &      -1.318         &      -0.534         &      -0.907         &      -1.959         &      -1.918         \\
                    &     (1.092)         &     (1.478)         &     (1.266)         &     (1.384)         &     (0.860)         &     (2.466)         &     (1.952)         &     (1.895)         &     (1.869)         &     (2.366)         &     (1.767)         &     (2.129)         \\
 \midrule\multicolumn{13}{l}{\emph{Panel D. Up to the age of 34}} \\ Cum. numbers        &      -154.0\sym{***}&       46.00         &       102.8         &       150.2         &      -138.0\sym{***}&       15.75         &       71.17         &       100.4         &      -16.00         &       30.25         &       31.67         &       49.80\sym{*}  \\
                    &     (32.92)         &     (125.4)         &     (96.04)         &     (100.8)         &     (39.02)         &     (111.9)         &     (86.25)         &     (89.51)         &     (35.13)         &     (40.53)         &     (30.01)         &     (26.12)         \\
 Cum. ratio          &      -4.188\sym{***}&      -1.797         &      -1.816         &      -1.494         &      -7.098\sym{***}&      -2.433         &      -1.713         &      -1.180         &      -1.460\sym{*}  &      -1.203         &      -1.974         &      -1.849         \\
                    &     (0.833)         &     (2.143)         &     (1.749)         &     (1.682)         &     (1.912)         &     (3.455)         &     (2.668)         &     (2.507)         &     (0.810)         &     (2.503)         &     (1.895)         &     (2.047)         \\
 
\bottomrule \end{tabular} } \begin{tablenotes} \item \scriptsize \emph{Notes:} \textbf{BOOTSTRAPPED} standard errors in parentheses (MxY), with 400 replications. All regressions contain Birthmonth FE. Ratios indicate cases per thousand; original number of births. \end{tablenotes} \end{threeparttable} \end{table} 

\end{landscape}
%---------------------------------
\newpage
FEBRUAR CASES:
 \begin{table}[H] \begin{threeparttable} \centering \caption{Dep. variable: \textbf{Certain infectious and parasitic diseases}} {\def\sym#1{\ifmmode^{#1}\else\(^{#1}\)\fi} \begin{tabular}{l*{13}{c}} \toprule year & \multicolumn{12}{c}{Month of birth} \\ \cmidrule(lr){2-13} 
            &          11&          12&           1&           2&           3&           4&           5&           6&           7&           8&           9&          10\\
1995        &         196&         190&         201&         192&         240&         188&         227&         189&         161&         207&         180&         205\\
1996        &         196&         231&         199&         235&         210&         201&         200&         205&         186&         211&         201&         199\\
1997        &         192&         206&         222&         214&         230&         201&         233&         242&         218&         213&         187&         224\\
1998        &         234&         211&         211&         213&         229&         211&         216&         233&         209&         179&         220&         193\\
1999        &         185&         188&         187&         197&         202&         189&         196&         191&         191&         201&         198&         203\\
2000        &         188&         200&         234&         191&         225&         204&         179&         219&         225&         178&         207&         218\\
2001        &         185&         241&         220&         216&         238&         248&         236&         251&         256&         228&         203&         251\\
2002        &         188&         204&         196&         194&         223&         215&         236&         208&         231&         238&         216&         224\\
2003        &         163&         176&         188&         162&         197&         176&         196&         195&         219&         212&         209&         189\\
2004        &         177&         200&         188&         174&         230&         207&         203&         191&         204&         214&         224&         166\\
2005        &         158&         154&         169&         169&         175&         143&         174&         172&         174&         181&         158&         168\\
2006        &         179&         150&         163&         167&         172&         161&         157&         162&         178&         161&         169&         178\\
2007        &         158&         188&         181&         167&         204&         160&         190&         180&         192&         179&         186&         186\\
2008        &         175&         155&         177&         189&         188&         138&         188&         199&         167&         196&         192&         195\\
2009        &         169&         157&         166&         183&         184&         162&         180&         190&         183&         175&         151&         193\\
2010        &         172&         183&         197&         190&         186&         191&         213&         195&         168&         181&         187&         181\\
2011        &         217&         196&         219&         225&         226&         198&         245&         254&         240&         212&         202&         215\\
2012        &         196&         177&         171&         182&         212&         189&         224&         213&         212&         192&         205&         212\\
2013        &         197&         193&         199&         199&         229&         208&         191&         208&         177&         169&         202&         193\\
2014        &         197&         186&         201&         236&         206&         214&         220&         195&         243&         191&         199&         191\\
 \bottomrule \end{tabular} } \begin{tablenotes} \item \scriptsize \emph{Notes:} Number of cases per year and MOB in treatment cohort. \end{tablenotes} \end{threeparttable} \end{table} 

 \begin{table}[H] \begin{threeparttable} \centering \caption{Dep. variable: \textbf{Certain infectious and parasitic diseases}} {\def\sym#1{\ifmmode^{#1}\else\(^{#1}\)\fi} \begin{tabular}{l*{13}{c}} \toprule year & \multicolumn{12}{c}{Month of birth} \\ \cmidrule(lr){2-13} 
            &          11&          12&           1&           2&           3&           4&           5&           6&           7&           8&           9&          10\\
1995        &          42&          25&           1&          -8&          26&           5&          52&           8&         -37&          52&          -5&           9\\
1996        &          11&          40&         -10&          47&          -9&          20&         -35&           9&          -9&          20&         -10&           3\\
1997        &           1&           2&          21&          -4&          10&          17&          53&          46&          18&          32&           0&           8\\
1998        &          39&          11&          20&          30&           6&           5&          13&          25&         -18&         -66&           2&         -28\\
1999        &           7&           5&         -24&          -3&         -15&         -30&         -17&         -14&         -52&         -23&          -3&           5\\
2000        &         -23&           4&          33&         -31&          -7&          -8&         -19&          18&          12&         -68&           3&         -16\\
2001        &         -40&           1&           6&         -32&          13&          44&          -8&          37&         -15&         -20&         -32&           7\\
2002        &         -20&          17&         -35&         -17&          -9&           6&          12&         -32&          13&           6&         -34&         -27\\
2003        &         -17&           7&         -21&         -23&         -36&         -18&           9&          13&           3&           6&          10&         -11\\
2004        &          -6&          -6&          -4&         -38&          35&           3&           4&         -20&         -37&          13&          24&         -56\\
2005        &         -10&         -13&          28&           0&           0&         -24&          -6&         -28&           8&           2&         -25&          -9\\
2006        &           3&         -22&          20&          12&           4&          22&         -23&         -24&         -13&         -28&           6&          11\\
2007        &          -2&          12&         -27&          -7&          -5&         -19&          41&         -18&          -8&         -15&          -9&         -15\\
2008        &           2&         -24&          -8&          33&           4&         -30&           5&          32&         -17&         -19&         -29&          11\\
2009        &         -20&         -30&          -6&           2&           8&           0&          -8&          -1&           5&           3&         -33&           0\\
2010        &           0&          11&           2&          -3&         -31&         -18&          17&          -9&         -33&         -28&         -13&         -35\\
2011        &          18&          -4&          -1&          -5&         -27&         -28&          25&          60&         -14&         -17&         -13&         -33\\
2012        &          -5&         -24&         -14&           1&         -14&         -34&          -5&           6&          18&         -20&          15&          -3\\
2013        &          10&           7&          17&          11&          36&           2&         -46&           8&         -28&         -46&          -8&         -29\\
2014        &          20&          -5&          27&          39&         -60&          -2&         -11&         -24&           8&         -54&         -14&         -26\\
 \bottomrule \end{tabular} } \begin{tablenotes} \item \scriptsize \emph{Notes:} Difference of cases (control - treatment) per year and MOB in treatment cohort. \end{tablenotes} \end{threeparttable} \end{table} 

