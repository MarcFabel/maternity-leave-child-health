%---------------------------------
% INPUT FOR VARIABLE: d10
%---------------------------------
\subsection{d10}
% RD overview
\begin{landscape}
\begin{figure}[H]
	\centering
	\begin{minipage}{.95\linewidth}
	\includegraphics[width=\linewidth]{rd_d10_overview_panel1}
	{\scriptsize \emph{Notes:} The figures show monthly RD plots with averages obtained from a bin width of one month. The solid vertical line divides pre- and post-reform regime. The averages are taken over the period of at most 1995-2014. \par}
\end{minipage}
\end{figure}
\end{landscape}
\begin{landscape}
\begin{figure}[H]
	\centering
\begin{minipage}{.95\linewidth}
	\includegraphics[width=\linewidth]{rd_d10_overview_panel2}
	{\scriptsize \emph{Notes:} The figures show monthly RD plots with a moving average window width of 3 months. The solid vertical line divides pre- and post-reform regime. The averages are taken over the period of at most 1995-2014. \par}
\end{minipage}
\end{figure}
\end{landscape}
%---------------------------------
% TABELLEN
 \begin{table}[H] \begin{threeparttable} \centering \caption{Dep. variable: \textbf{Diseases of the digestive system}} {\def\sym#1{\ifmmode^{#1}\else\(^{#1}\)\fi} \begin{tabular}{l*{8}{c}} \toprule & \multicolumn{7}{c}{Estimation window} \\ \cmidrule(lr){2-8}
            &\multicolumn{1}{c}{(1)}&\multicolumn{1}{c}{(2)}&\multicolumn{1}{c}{(3)}&\multicolumn{1}{c}{(4)}&\multicolumn{1}{c}{(5)}&\multicolumn{1}{c}{(6)}&\multicolumn{1}{c}{(7)}\\
            &\multicolumn{1}{c}{1M}&\multicolumn{1}{c}{2M}&\multicolumn{1}{c}{3M}&\multicolumn{1}{c}{4M}&\multicolumn{1}{c}{5M}&\multicolumn{1}{c}{6M}&\multicolumn{1}{c}{Donut}\\
\midrule
 \multicolumn{8}{l}{\emph{Panel A. Average causal effects}} \\ Abs. numbers        &       22.25\sym{***}&       11.08         &       3.617         &       10.14         &       11.77\sym{**} &       17.37\sym{***}&       16.39\sym{**} \\
                    &  (1.47e-13)         &     (6.461)         &     (6.309)         &     (6.244)         &     (5.074)         &     (5.007)         &     (5.988)         \\
 Ratio fertility     &       0.333\sym{***}&     0.00903         &      -0.404\sym{*}  &      -0.319\sym{*}  &      -0.327\sym{**} &      -0.395\sym{***}&      -0.541\sym{***}\\
                    &  (3.18e-15)         &     (0.134)         &     (0.203)         &     (0.182)         &     (0.149)         &     (0.129)         &     (0.131)         \\
 Ratio population    &       0.473\sym{***}&       0.115         &      -0.277         &      -0.178         &     -0.0647         &     -0.0873         &      -0.199         \\
                    &  (7.07e-16)         &     (0.165)         &     (0.203)         &     (0.189)         &     (0.163)         &     (0.136)         &     (0.138)         \\
 Ratio fert(03-14)   &       0.450\sym{***}&     -0.0845         &      -0.628\sym{**} &      -0.453         &      -0.311         &      -0.373\sym{*}  &      -0.537\sym{**} \\
                    &  (1.78e-15)         &     (0.212)         &     (0.278)         &     (0.262)         &     (0.220)         &     (0.185)         &     (0.197)         \\
 Cum. numbers        &       167.8\sym{***}&       138.7\sym{**} &       109.9\sym{**} &       149.8\sym{**} &       119.8\sym{**} &       174.5\sym{***}&       175.8\sym{**} \\
                    &  (2.03e-12)         &     (46.07)         &     (46.06)         &     (67.44)         &     (55.40)         &     (52.84)         &     (62.81)         \\
 Cum. ratio          &       2.066\sym{***}&       0.504         &      -2.886         &      -2.606\sym{*}  &      -3.706\sym{***}&      -4.571\sym{***}&      -5.899\sym{***}\\
                    &  (6.24e-14)         &     (0.897)         &     (1.621)         &     (1.304)         &     (1.257)         &     (1.140)         &     (1.147)         \\
 \midrule\multicolumn{8}{l}{\emph{Panel B. Treatment effect heterogeneity - Women}} \\ Abs. numbers        &       16.45\sym{***}&       7.950\sym{*}  &       5.033         &       6.288\sym{**} &       4.040         &       6.000\sym{**} &       3.910\sym{*}  \\
                    &  (1.17e-13)         &     (4.018)         &     (3.171)         &     (2.896)         &     (2.546)         &     (2.482)         &     (2.132)         \\
 Ratio fertility     &       0.496\sym{***}&       0.305\sym{**} &      0.0203         &    -0.00534         &      -0.117         &      -0.218\sym{*}  &      -0.361\sym{***}\\
                    &  (1.43e-15)         &     (0.125)         &     (0.148)         &     (0.116)         &     (0.115)         &     (0.107)         &    (0.0995)         \\
 Ratio population    &       0.574\sym{***}&       0.148         &      -0.309         &      -0.237         &      -0.301         &      -0.439\sym{**} &      -0.641\sym{***}\\
                    &  (1.22e-15)         &     (0.285)         &     (0.274)         &     (0.244)         &     (0.198)         &     (0.178)         &     (0.178)         \\
 Ratio fert(03-14)   &       0.767\sym{***}&       0.200         &      -0.415         &      -0.317         &      -0.441         &      -0.652\sym{**} &      -0.936\sym{***}\\
                    &  (2.03e-15)         &     (0.381)         &     (0.367)         &     (0.328)         &     (0.267)         &     (0.246)         &     (0.244)         \\
 Cum. numbers        &       132.2\sym{***}&       71.10\sym{**} &       70.77\sym{***}&       67.17         &       32.83         &       57.13\sym{*}  &       42.11         \\
                    &  (1.37e-12)         &     (23.82)         &     (19.16)         &     (39.18)         &     (35.11)         &     (31.51)         &     (34.45)         \\
 Cum. ratio          &       2.516\sym{***}&      -0.334         &      -3.601         &      -4.302\sym{**} &      -6.776\sym{***}&      -8.669\sym{***}&      -10.91\sym{***}\\
                    &  (1.03e-13)         &     (2.175)         &     (2.021)         &     (1.535)         &     (1.797)         &     (1.803)         &     (1.751)         \\
 \midrule\multicolumn{8}{l}{\emph{Panel C. Treatment effect heterogeneity - Men}} \\ Abs. numbers        &       5.800\sym{***}&       3.125         &      -1.417         &       3.850         &       7.730\sym{*}  &       11.37\sym{***}&       12.48\sym{**} \\
                    &  (5.78e-14)         &     (7.245)         &     (5.540)         &     (4.925)         &     (4.332)         &     (3.996)         &     (4.620)         \\
 Ratio fertility     &       0.215\sym{***}&     -0.0328         &      -0.444\sym{*}  &      -0.282         &      -0.134         &      -0.111         &      -0.176         \\
                    &  (1.90e-15)         &     (0.175)         &     (0.229)         &     (0.211)         &     (0.185)         &     (0.156)         &     (0.174)         \\
 Ratio population    &       0.112\sym{***}&      -0.282         &      -0.656\sym{**} &      -0.459\sym{*}  &      -0.129         &     -0.0522         &     -0.0851         \\
                    &  (3.46e-15)         &     (0.335)         &     (0.288)         &     (0.245)         &     (0.253)         &     (0.218)         &     (0.240)         \\
 Ratio fert(03-14)   &       0.143\sym{***}&      -0.357         &      -0.832\sym{**} &      -0.582\sym{*}  &      -0.194         &      -0.116         &      -0.168         \\
                    &  (1.41e-15)         &     (0.426)         &     (0.366)         &     (0.312)         &     (0.311)         &     (0.265)         &     (0.290)         \\
 Cum. numbers        &       35.60\sym{***}&       67.55         &       39.08         &       82.63\sym{*}  &       87.00\sym{**} &       117.3\sym{***}&       133.7\sym{***}\\
                    &  (1.73e-12)         &     (50.31)         &     (41.01)         &     (40.83)         &     (32.85)         &     (30.79)         &     (35.81)         \\
 Cum. ratio          &       1.670\sym{***}&       1.881\sym{**} &      -0.269         &       0.270         &       0.331         &       0.604         &       0.391         \\
                    &  (3.13e-14)         &     (0.764)         &     (1.091)         &     (1.062)         &     (0.908)         &     (0.780)         &     (0.826)         \\
 
\bottomrule \end{tabular} } \begin{tablenotes} \item \scriptsize \emph{Notes:} Clustered standard errors in parentheses. All regression are run with CG2 (i.e. the cohort prior to the reform) and with month-of-birth FEs. Ratios indicate cases per thousand; either approximated population (with weights coming from the original fertility distribution) or original number of births. \end{tablenotes} \end{threeparttable} \end{table} 

 \begin{table}[H] \begin{threeparttable} \centering \caption{Robustness with respect to the choice of \texttt{control group}} {\def\sym#1{\ifmmode^{#1}\else\(^{#1}\)\fi} \begin{tabular}{l*{10}{c}} \toprule & \multicolumn{9}{c}{Dependent variable: \textbf{Diseases of the digestive system}} \\ \cmidrule(lr){2-10}
            &\multicolumn{3}{c}{Average Causal Effects}&\multicolumn{3}{c}{Women}             &\multicolumn{3}{c}{Men}               \\\cmidrule(lr){2-4}\cmidrule(lr){5-7}\cmidrule(lr){8-10}
            &\multicolumn{1}{c}{(1)}&\multicolumn{1}{c}{(2)}&\multicolumn{1}{c}{(3)}&\multicolumn{1}{c}{(4)}&\multicolumn{1}{c}{(5)}&\multicolumn{1}{c}{(6)}&\multicolumn{1}{c}{(7)}&\multicolumn{1}{c}{(8)}&\multicolumn{1}{c}{(9)}\\
            &\multicolumn{1}{c}{C2}&\multicolumn{1}{c}{C1+C2}&\multicolumn{1}{c}{C1-C3}&\multicolumn{1}{c}{C2}&\multicolumn{1}{c}{C1+C2}&\multicolumn{1}{c}{C1-C3}&\multicolumn{1}{c}{C2}&\multicolumn{1}{c}{C1+C2}&\multicolumn{1}{c}{C1-C3}\\
\midrule
 \multicolumn{10}{l}{\emph{Panel A. 2 Month bandwidth}} \\ Abs. numbers        &       11.08         &       4.738         &       4.633         &       7.950\sym{*}  &       4.588         &       2.367         &       3.125         &       0.150         &       2.267         \\
                    &     (6.461)         &     (10.63)         &     (14.31)         &     (4.018)         &     (5.662)         &     (12.95)         &     (7.245)         &     (8.812)         &     (7.451)         \\
 Ratio population    &      -0.274         &      0.0125         &       0.191         &      -0.400         &      -0.156         &      -0.177         &      -0.111         &       0.203         &       0.555         \\
                    &     (0.152)         &     (0.241)         &     (0.287)         &     (0.423)         &     (0.512)         &     (0.576)         &     (0.497)         &     (0.383)         &     (0.438)         \\
 Ratio population    &       0.115         &       0.187         &       0.248\sym{**} &       0.399\sym{*}  &       0.382\sym{*}  &       0.376\sym{*}  &      -0.167         &    -0.00450         &       0.124         \\
                    &     (0.165)         &     (0.117)         &     (0.106)         &     (0.208)         &     (0.212)         &     (0.202)         &     (0.369)         &     (0.314)         &     (0.318)         \\
 \midrule\multicolumn{10}{l}{\emph{Panel B. 4 Month bandwidth}} \\ Abs. numbers        &       2.444         &       4.903         &       7.250         &       1.083         &       2.389         &       3.019         &       1.361         &       2.514         &       4.231         \\
                    &     (11.43)         &     (9.446)         &     (8.672)         &     (3.849)         &     (3.895)         &     (4.586)         &     (9.418)         &     (7.921)         &     (6.975)         \\
 Ratio fertility     &     -0.0589         &      0.0100         &      0.0380         &    -0.00534         &       0.111         &       0.132         &     -0.0989         &     -0.0717         &     -0.0390         \\
                    &     (0.114)         &     (0.118)         &     (0.106)         &     (0.116)         &     (0.136)         &     (0.144)         &     (0.145)         &     (0.137)         &     (0.161)         \\
 Ratio fertility     &      -0.165         &      0.0109         &      0.0801         &      -0.136         &      0.0635         &       0.114         &      -0.189         &     -0.0336         &      0.0530         \\
                    &     (0.212)         &     (0.193)         &     (0.200)         &     (0.190)         &     (0.194)         &     (0.180)         &     (0.274)         &     (0.239)         &     (0.255)         \\
 \midrule\multicolumn{10}{l}{\emph{Panel C. 6 Month bandwidth}} \\ Abs. numbers        &       16.69\sym{*}  &       15.62\sym{*}  &       13.48\sym{*}  &       0.130         &       2.741         &       0.642         &       16.56\sym{**} &       12.88\sym{*}  &       12.83\sym{**} \\
                    &     (8.919)         &     (7.759)         &     (7.088)         &     (3.909)         &     (3.915)         &     (4.465)         &     (7.935)         &     (6.580)         &     (5.850)         \\
 Ratio fertility     &      -0.395\sym{***}&      -0.388\sym{**} &      -0.228\sym{*}  &      -0.718\sym{***}&      -0.604\sym{***}&      -0.433\sym{*}  &      -0.111         &      -0.205         &     -0.0547         \\
                    &     (0.129)         &     (0.146)         &     (0.134)         &     (0.178)         &     (0.214)         &     (0.223)         &     (0.156)         &     (0.145)         &     (0.169)         \\
 Ratio fertility     &     -0.0629         &      0.0837         &       0.127         &      -0.339\sym{**} &     -0.0829         &     -0.0541         &       0.199         &       0.242         &       0.299         \\
                    &     (0.149)         &     (0.145)         &     (0.151)         &     (0.150)         &     (0.159)         &     (0.144)         &     (0.233)         &     (0.201)         &     (0.217)         \\
 \midrule\multicolumn{10}{l}{\emph{Panel D. Donut specification}} \\ Abs. numbers        &       13.00         &       13.62         &       10.75         &      -3.289         &       0.344         &      -1.659         &       16.29\sym{*}  &       13.28\sym{*}  &       12.41\sym{*}  \\
                    &     (10.48)         &     (8.989)         &     (8.027)         &     (4.078)         &     (4.386)         &     (4.849)         &     (9.247)         &     (7.264)         &     (6.281)         \\
 Ratio fertility     &      -0.165\sym{*}  &     -0.0427         &     -0.0148         &      -0.361\sym{***}&      -0.136         &     -0.0839         &      0.0201         &      0.0485         &      0.0544         \\
                    &    (0.0823)         &    (0.0971)         &    (0.0930)         &    (0.0995)         &     (0.127)         &     (0.123)         &     (0.116)         &     (0.115)         &     (0.149)         \\
 Ratio population    &      -0.360\sym{**} &      -0.311\sym{**} &      -0.187         &      -0.641\sym{***}&      -0.508\sym{***}&      -0.395\sym{**} &     -0.0851         &      -0.119         &      0.0156         \\
                    &     (0.158)         &     (0.137)         &     (0.140)         &     (0.178)         &     (0.167)         &     (0.162)         &     (0.240)         &     (0.194)         &     (0.217)         \\
 
\bottomrule \end{tabular} } \begin{tablenotes} \item \scriptsize \emph{Notes:} Clustered standard errors in parentheses. All regressions contain Birthmonth FE. Ratios indicate cases per thousand; either approximated population (with weights coming from the original fertility distribution) or original number of births. \end{tablenotes} \end{threeparttable} \end{table} 

%---------------------------------
% Life-course figure (Panel1)
\begin{landscape}
\begin{figure}[H]
\centering
\begin{minipage}{.9\linewidth}
\includegraphics[width=\linewidth]{lc_d10_overview_panel1}
{\scriptsize \emph{Notes:} The figures depict DDRD estimates and 90\% confidence intervals over the life-course. The years are harmonized such that the cohorts are in the same age when they are compared. All regressions are carried out with month-of-birth FE and make use of clustered standard errors. Furthermore, we used a bandwidth of half a year and only the control cohort that was born one year prior to the reform. Ratios indicate cases per thousand; using in the denominator the approximated population (with weights coming from the original fertility distribution) or original number of births. \par}
\end{minipage}
\end{figure}
\end{landscape}
%---------------------------------
% Life-course figure (Panel2)
\begin{landscape}
\begin{figure}[H]
\centering
\begin{minipage}{.9\linewidth}
\includegraphics[width=\linewidth]{lc_d10_overview_panel2}
{\scriptsize \emph{Notes:} The figures depict DDRD estimates and 90\% confidence intervals over the life-course. The years are harmonized such that the cohorts are in the same age when they are compared. All regressions are carried out with month-of-birth FE and make use of clustered standard errors. Furthermore, we used a bandwidth of half a year. Ratios indicate cases per thousand; using in the denominator the approximated population (with weights coming from the original fertility distribution) or original number of births. \par}
\end{minipage}
\end{figure}
\end{landscape}
%---------------------------------
% Life-course (panel 3 - 6)
\begin{figure}[H]%\vspace*{-2cm}
	\centering
	\includegraphics[width=.9\linewidth]{lc_d10_overview_panel3}
	\includegraphics[width=.9\linewidth]{lc_d10_overview_panel4}
\end{figure}
\begin{figure}[H]
	\centering	
	\includegraphics[width=.97\linewidth]{lc_d10_overview_panel5}
	\includegraphics[width=.97\linewidth]{lc_d10_overview_panel6}
\end{figure}
% Life-course TABLE Format
 \begin{table}[H] \centering \begin{threeparttable} \caption{Life-course approach - Table format} {\def\sym#1{\ifmmode^{#1}\else\(^{#1}\)\fi} \begin{tabular}{l*{5}{c}} \toprule \multicolumn{5}{l}{Dep. variable: \textbf{Diseases of the digestive system}} \\ & \multicolumn{4}{c}{Estimation window} \\ \cmidrule(lr){2-5}
            &\multicolumn{1}{c}{(1)}&\multicolumn{1}{c}{(2)}&\multicolumn{1}{c}{(3)}&\multicolumn{1}{c}{(4)}\\
            &\multicolumn{1}{c}{Age 17-21}&\multicolumn{1}{c}{Age 22-26}&\multicolumn{1}{c}{Age 27-31}&\multicolumn{1}{c}{Age 32-35}\\
\midrule
 \multicolumn{5}{l}{\emph{Panel A. Average causal effects}} \\ Abs. numbers        &       20.13         &       8.933         &       14.97         &       15.50         \\
                    &     (16.05)         &     (22.72)         &     (27.33)         &     (23.90)         \\
 Ratio fertility     &      -0.421         &      -0.485         &      -0.376         &      -0.494         \\
                    &     (0.337)         &     (0.524)         &     (0.465)         &     (0.646)         \\
 Ratio population    &      -0.351         &      -0.248         &      -0.337         \\
                    &     (0.436)         &     (0.357)         &     (0.505)         \\
 Cum. numbers        &       107.8\sym{**} &       197.0\sym{**} &       260.4\sym{*}  &       307.2         \\
                    &     (54.80)         &     (95.98)         &     (140.2)         &     (194.2)         \\
 Cum. ratio          &      -1.795         &      -3.671         &      -5.630\sym{*}  &      -8.059\sym{*}  \\
                    &     (1.274)         &     (2.405)         &     (3.169)         &     (4.652)         \\
 \midrule\multicolumn{5}{l}{\emph{Panel B. Treatment effect heterogeneity - Women}} \\ Abs. numbers        &       8.667         &       6.133         &      -5.567         &       4.000         \\
                    &     (12.73)         &     (16.49)         &     (11.45)         &     (16.62)         \\
 Ratio fertility     &      -0.796         &      -0.592         &      -1.069\sym{**} &      -0.734         \\
                    &     (0.517)         &     (0.825)         &     (0.470)         &     (0.790)         \\
 Ratio population    &      -0.236         &      -0.753\sym{**} &      -0.497         \\
                    &     (0.790)         &     (0.333)         &     (0.588)         \\
 Cum. numbers        &       44.03         &       94.07         &       103.7         &       86.04         \\
                    &     (39.37)         &     (72.79)         &     (78.81)         &     (92.31)         \\
 Cum. ratio          &      -3.883\sym{**} &      -6.630\sym{*}  &      -10.30\sym{**} &      -14.94\sym{**} \\
                    &     (1.683)         &     (3.384)         &     (5.101)         &     (6.011)         \\
 \midrule\multicolumn{5}{l}{\emph{Panel C. Treatment effect heterogeneity - Men}} \\ Abs. numbers        &       11.47\sym{**} &       2.800         &       20.53         &       11.50         \\
                    &     (5.841)         &     (10.75)         &     (21.07)         &     (13.11)         \\
 Ratio fertility     &      -0.110         &      -0.406         &       0.273         &      -0.266         \\
                    &     (0.312)         &     (0.426)         &     (0.773)         &     (0.675)         \\
 Ratio population    &      -0.481         &       0.251         &      -0.176         \\
                    &     (0.376)         &     (0.616)         &     (0.550)         \\
 Cum. numbers        &       63.77\sym{**} &       103.0\sym{***}&       156.7\sym{*}  &       221.2         \\
                    &     (30.36)         &     (37.23)         &     (90.56)         &     (149.1)         \\
 Cum. ratio          &     -0.0185         &      -1.222         &      -1.625         &      -1.977         \\
                    &     (1.473)         &     (2.454)         &     (3.404)         &     (6.177)         \\
 
\bottomrule \end{tabular} } \begin{tablenotes} \item \scriptsize \emph{Notes:} Clustered standard errors in parentheses. All regression are run with CG2 (i.e. the cohort prior to the reform) and with month-of-birth FEs. Ratios indicate cases per thousand; either approximated population (with weights coming from the original fertility distribution) or original number of births. Raqtio population muss eins nach rechts gerückt werden \end{tablenotes} \end{threeparttable} \end{table} 

%---------------------------------
% PLACEBO EXERCISES
\newpage
\begin{landscape}
\begin{figure}[H]
	\centering
    \begin{minipage}{.9\linewidth}
	\includegraphics[width=\linewidth]{placebo_graph_d10.pdf}
    {\scriptsize \emph{Notes:} The figures depict DDRD estimates and 95\% confidence intervals when the treatment cohort is shifted over time. The date on the abscissa indicates the starting date of the treated.  All regressions are carried out with month-of-birth FE and make use of clustered standard errors. Furthermore, we used a bandwidth of half a year. Ratios indicate cases per thousand; using in the denominator the approximated population (with weights coming from the original fertility distribution) or original number of births. \par}
    \end{minipage}
\end{figure}
\end{landscape}
 \begin{table}[H] \centering \begin{threeparttable} \caption{Placebo 1 (CONTROL1 ist TREAT) } {\def\sym#1{\ifmmode^{#1}\else\(^{#1}\)\fi} \begin{tabular}{l*{4}{c}} \toprule \multicolumn{4}{l}{Dep. variable: \textbf{Diseases of the digestive system}} \\ & \multicolumn{3}{c}{Choice of control group} \\ \cmidrule(lr){2-4}
            &\multicolumn{1}{c}{(1)}&\multicolumn{1}{c}{(2)}&\multicolumn{1}{c}{(3)}\\
            &\multicolumn{1}{c}{C2}&\multicolumn{1}{c}{C3}&\multicolumn{1}{c}{C2+C3}\\
\midrule
 \multicolumn{4}{l}{\emph{Panel A. Average causal effects}} \\ Abs. numbers        &       9.850\sym{*}  &      -12.30\sym{*}  &      -1.225         \\
                    &     (5.121)         &     (6.360)         &     (10.56)         \\
 Ratio fertility     &     -0.0138         &       0.474\sym{***}&       0.230         \\
                    &     (0.121)         &     (0.153)         &     (0.147)         \\
 Ratio population    &     -0.0816         &       0.286\sym{***}&       0.102         \\
                    &     (0.114)         &    (0.0981)         &     (0.126)         \\
 Cum. numbers        &       104.5\sym{*}  &      -191.3\sym{**} &      -43.39         \\
                    &     (54.51)         &     (82.57)         &     (147.9)         \\
 Cum. ratio          &      -0.107         &       4.203\sym{**} &       2.048         \\
                    &     (1.177)         &     (1.833)         &     (1.645)         \\
 \midrule\multicolumn{4}{l}{\emph{Panel B. Treatment effect heterogeneity - Women}} \\ Abs. numbers        &       0.308         &      -11.43\sym{**} &      -5.562         \\
                    &     (2.571)         &     (4.388)         &     (8.540)         \\
 Ratio fertility     &      -0.228\sym{**} &       0.400\sym{*}  &      0.0864         \\
                    &     (0.106)         &     (0.194)         &     (0.188)         \\
 Ratio population    &      -0.204\sym{*}  &       0.171         &     -0.0166         \\
                    &     (0.103)         &     (0.126)         &     (0.121)         \\
 Cum. numbers        &      -16.14         &      -153.7\sym{**} &      -84.93         \\
                    &     (30.34)         &     (59.89)         &     (117.2)         \\
 Cum. ratio          &      -3.271\sym{**} &       3.957         &       0.343         \\
                    &     (1.303)         &     (2.416)         &     (2.708)         \\
 \midrule\multicolumn{4}{l}{\emph{Panel C. Treatment effect heterogeneity - Men}} \\ Abs. numbers        &       9.542\sym{***}&      -0.867         &       4.338         \\
                    &     (3.385)         &     (3.492)         &     (3.681)         \\
 Ratio fertility     &       0.188         &       0.546\sym{***}&       0.367\sym{*}  \\
                    &     (0.163)         &     (0.157)         &     (0.208)         \\
 Ratio population    &      0.0389         &       0.402\sym{***}&       0.221         \\
                    &     (0.174)         &     (0.114)         &     (0.198)         \\
 Cum. numbers        &       120.6\sym{***}&      -37.54         &       41.54         \\
                    &     (32.33)         &     (36.40)         &     (43.09)         \\
 Cum. ratio          &       2.873\sym{*}  &       4.440\sym{**} &       3.656\sym{*}  \\
                    &     (1.536)         &     (1.725)         &     (1.830)         \\
 
\bottomrule \end{tabular} } \begin{tablenotes} \item \scriptsize \emph{Notes:} Clustered standard errors in parentheses. All regression are run with month-of-birth FEs and control cohort 2 is assigned with the treatment status. All regressions are carried out with a window width of half a year. \end{tablenotes} \end{threeparttable} \end{table} 

 \begin{table}[H] \centering \begin{threeparttable} \caption{Placebo 2 (CONTROL2 ist TREAT) } {\def\sym#1{\ifmmode^{#1}\else\(^{#1}\)\fi} \begin{tabular}{l*{4}{c}} \toprule \multicolumn{4}{l}{Dep. variable: \textbf{Diseases of the digestive system}} \\ & \multicolumn{3}{c}{Choice of control group} \\ \cmidrule(lr){2-4}
            &\multicolumn{1}{c}{(1)}&\multicolumn{1}{c}{(2)}&\multicolumn{1}{c}{(3)}\\
            &\multicolumn{1}{c}{C1}&\multicolumn{1}{c}{C3}&\multicolumn{1}{c}{C1+C3}\\
\midrule
 \multicolumn{4}{l}{\emph{Panel A. Average causal effects}} \\ Abs. numbers        &      -9.850\sym{*}  &      -22.15\sym{***}&      -16.00         \\
                    &     (5.121)         &     (5.949)         &     (11.71)         \\
 Ratio fertility     &      0.0138         &       0.488\sym{***}&       0.251\sym{*}  \\
                    &     (0.121)         &     (0.117)         &     (0.133)         \\
 Ratio population    &      0.0816         &       0.368\sym{***}&       0.225         \\
                    &     (0.114)         &     (0.124)         &     (0.148)         \\
 Cum. numbers        &      -104.5\sym{*}  &      -295.7\sym{***}&      -200.1         \\
                    &     (54.51)         &     (61.54)         &     (169.6)         \\
 Cum. ratio          &       0.107         &       4.309\sym{***}&       2.208         \\
                    &     (1.177)         &     (1.242)         &     (1.882)         \\
 \midrule\multicolumn{4}{l}{\emph{Panel B. Treatment effect heterogeneity - Women}} \\ Abs. numbers        &      -0.308         &      -11.74\sym{***}&      -6.025         \\
                    &     (2.571)         &     (3.610)         &    (10.000)         \\
 Ratio fertility     &       0.228\sym{**} &       0.628\sym{***}&       0.428\sym{*}  \\
                    &     (0.106)         &     (0.172)         &     (0.239)         \\
 Ratio population    &       0.204\sym{*}  &       0.374\sym{**} &       0.289\sym{**} \\
                    &     (0.103)         &     (0.144)         &     (0.127)         \\
 Cum. numbers        &       16.14         &      -137.6\sym{***}&      -60.72         \\
                    &     (30.34)         &     (48.55)         &     (146.5)         \\
 Cum. ratio          &       3.271\sym{**} &       7.229\sym{***}&       5.250         \\
                    &     (1.303)         &     (2.158)         &     (3.951)         \\
 \midrule\multicolumn{4}{l}{\emph{Panel C. Treatment effect heterogeneity - Men}} \\ Abs. numbers        &      -9.542\sym{***}&      -10.41\sym{**} &      -9.975\sym{**} \\
                    &     (3.385)         &     (3.898)         &     (3.705)         \\
 Ratio fertility     &      -0.188         &       0.358\sym{***}&      0.0848         \\
                    &     (0.163)         &     (0.112)         &     (0.182)         \\
 Ratio population    &     -0.0389         &       0.363\sym{**} &       0.162         \\
                    &     (0.174)         &     (0.147)         &     (0.221)         \\
 Cum. numbers        &      -120.6\sym{***}&      -158.2\sym{***}&      -139.4\sym{***}\\
                    &     (32.33)         &     (33.67)         &     (37.61)         \\
 Cum. ratio          &      -2.873\sym{*}  &       1.568         &      -0.653         \\
                    &     (1.536)         &     (0.930)         &     (1.492)         \\
 
\bottomrule \end{tabular} } \begin{tablenotes} \item \scriptsize \emph{Notes:} Clustered standard errors in parentheses. All regression are run with month-of-birth FEs and control cohort 2 is assigned with the treatment status. All regressions are carried out with a window width of half a year. \end{tablenotes} \end{threeparttable} \end{table} 

 \begin{table}[H] \centering \begin{threeparttable} \caption{Placebo 3 (CONTROL3 ist TREAT) } {\def\sym#1{\ifmmode^{#1}\else\(^{#1}\)\fi} \begin{tabular}{l*{4}{c}} \toprule \multicolumn{4}{l}{Dep. variable: \textbf{Diseases of the digestive system}} \\ & \multicolumn{3}{c}{Choice of control group} \\ \cmidrule(lr){2-4}
            &\multicolumn{1}{c}{(1)}&\multicolumn{1}{c}{(2)}&\multicolumn{1}{c}{(3)}\\
            &\multicolumn{1}{c}{C1}&\multicolumn{1}{c}{C2}&\multicolumn{1}{c}{C1+C2}\\
\midrule
 \multicolumn{4}{l}{\emph{Panel A. Average causal effects}} \\ Abs. numbers        &       12.30\sym{*}  &       22.15\sym{***}&       17.22\sym{**} \\
                    &     (6.360)         &     (5.949)         &     (6.400)         \\
 Ratio fertility     &      -0.474\sym{***}&      -0.488\sym{***}&      -0.481\sym{***}\\
                    &     (0.153)         &     (0.117)         &     (0.149)         \\
 Ratio population    &      -0.286\sym{***}&      -0.368\sym{***}&      -0.327\sym{***}\\
                    &    (0.0981)         &     (0.124)         &     (0.114)         \\
 Cum. numbers        &       191.3\sym{**} &       295.8\sym{***}&       243.5\sym{***}\\
                    &     (82.57)         &     (61.54)         &     (79.20)         \\
 Cum. ratio          &      -4.203\sym{**} &      -4.309\sym{***}&      -4.256\sym{**} \\
                    &     (1.833)         &     (1.242)         &     (1.817)         \\
 \midrule\multicolumn{4}{l}{\emph{Panel B. Treatment effect heterogeneity - Women}} \\ Abs. numbers        &       11.43\sym{**} &       11.74\sym{***}&       11.59\sym{**} \\
                    &     (4.388)         &     (3.610)         &     (4.345)         \\
 Ratio fertility     &      -0.400\sym{*}  &      -0.628\sym{***}&      -0.514\sym{**} \\
                    &     (0.194)         &     (0.172)         &     (0.210)         \\
 Ratio population    &      -0.171         &      -0.374\sym{**} &      -0.273\sym{*}  \\
                    &     (0.126)         &     (0.144)         &     (0.139)         \\
 Cum. numbers        &       153.7\sym{**} &       137.6\sym{***}&       145.7\sym{**} \\
                    &     (59.89)         &     (48.55)         &     (63.50)         \\
 Cum. ratio          &      -3.957         &      -7.229\sym{***}&      -5.593\sym{*}  \\
                    &     (2.416)         &     (2.158)         &     (2.886)         \\
 \midrule\multicolumn{4}{l}{\emph{Panel C. Treatment effect heterogeneity - Men}} \\ Abs. numbers        &       0.867         &       10.41\sym{**} &       5.638         \\
                    &     (3.492)         &     (3.898)         &     (3.791)         \\
 Ratio fertility     &      -0.546\sym{***}&      -0.358\sym{***}&      -0.452\sym{***}\\
                    &     (0.157)         &     (0.112)         &     (0.138)         \\
 Ratio population    &      -0.402\sym{***}&      -0.363\sym{**} &      -0.383\sym{***}\\
                    &     (0.114)         &     (0.147)         &     (0.131)         \\
 Cum. numbers        &       37.54         &       158.2\sym{***}&       97.85\sym{**} \\
                    &     (36.40)         &     (33.67)         &     (36.94)         \\
 Cum. ratio          &      -4.440\sym{**} &      -1.568         &      -3.004\sym{**} \\
                    &     (1.725)         &     (0.930)         &     (1.414)         \\
 
\bottomrule \end{tabular} } \begin{tablenotes} \item \scriptsize \emph{Notes:} Clustered standard errors in parentheses. All regression are run with month-of-birth FEs and control cohort 3 is assigned with the treatment status. All regressions are carried out with a window width of half a year. \end{tablenotes} \end{threeparttable} \end{table} 

%---------------------------------
% CUMMULATIVE APPROACH
\begin{landscape}
 \begin{table}[H] \begin{threeparttable} \centering \caption{Cummulative effects for upt to different points of age} {\def\sym#1{\ifmmode^{#1}\else\(^{#1}\)\fi} \begin{tabular}{l*{13}{c}} \toprule & \multicolumn{12}{c}{Dependent variable: \textbf{Diseases of the digestive system}} \\ \cmidrule(lr){2-13}
            &\multicolumn{4}{c}{Average Causal Effects}         &\multicolumn{4}{c}{Women}                          &\multicolumn{4}{c}{Men}                            \\\cmidrule(lr){2-5}\cmidrule(lr){6-9}\cmidrule(lr){10-13}
            &\multicolumn{1}{c}{(1)}&\multicolumn{1}{c}{(2)}&\multicolumn{1}{c}{(3)}&\multicolumn{1}{c}{(4)}&\multicolumn{1}{c}{(5)}&\multicolumn{1}{c}{(6)}&\multicolumn{1}{c}{(7)}&\multicolumn{1}{c}{(8)}&\multicolumn{1}{c}{(9)}&\multicolumn{1}{c}{(10)}&\multicolumn{1}{c}{(11)}&\multicolumn{1}{c}{(12)}\\
            &\multicolumn{1}{c}{2M}&\multicolumn{1}{c}{4M}&\multicolumn{1}{c}{6M}&\multicolumn{1}{c}{Donut}&\multicolumn{1}{c}{2M}&\multicolumn{1}{c}{4M}&\multicolumn{1}{c}{6M}&\multicolumn{1}{c}{Donut}&\multicolumn{1}{c}{2M}&\multicolumn{1}{c}{4M}&\multicolumn{1}{c}{6M}&\multicolumn{1}{c}{Donut}\\
\midrule
 \multicolumn{13}{l}{\emph{Panel A. 2 Up to the age of 21}} \\ Cum. numbers        &       119.5         &       203.5\sym{**} &       166.5\sym{**} &       188.4\sym{**} &       25.00         &       82.00         &       83.50         &       98.20         &       94.50\sym{**} &       121.5\sym{***}&       83.00\sym{***}&       90.20\sym{***}\\
                    &     (104.9)         &     (87.85)         &     (64.96)         &     (76.43)         &     (78.50)         &     (74.03)         &     (53.26)         &     (63.36)         &     (37.79)         &     (19.99)         &     (23.07)         &     (27.45)         \\
 Cum. ratio          &       0.862         &       0.194         &      -2.213\sym{*}  &      -2.563\sym{*}  &      -1.300         &      -1.791         &      -4.416\sym{**} &      -4.806\sym{**} &       2.785\sym{*}  &       2.034         &      -0.407         &      -0.716         \\
                    &     (0.943)         &     (0.548)         &     (1.221)         &     (1.447)         &     (1.416)         &     (1.180)         &     (1.689)         &     (1.977)         &     (1.433)         &     (1.215)         &     (1.339)         &     (1.618)         \\
 \midrule\multicolumn{13}{l}{\emph{Panel B. Up to the age of 26}} \\ Cum. numbers        &       111.0         &       205.2\sym{*}  &       211.2\sym{**} &       211.0\sym{*}  &       57.50         &       107.7         &       114.2\sym{*}  &       99.40         &       53.50         &       97.50\sym{*}  &       97.00\sym{**} &       111.6\sym{**} \\
                    &     (76.81)         &     (107.6)         &     (97.58)         &     (116.5)         &     (111.1)         &     (78.95)         &     (64.57)         &     (74.30)         &     (39.54)         &     (46.41)         &     (42.12)         &     (49.95)         \\
 Cum. ratio          &      -0.202         &      -2.099         &      -4.638\sym{**} &      -6.028\sym{**} &      -1.164         &      -3.417         &      -7.378\sym{**} &      -9.658\sym{***}&       0.529         &      -0.936         &      -2.435         &      -2.977         \\
                    &     (4.036)         &     (2.467)         &     (2.208)         &     (2.395)         &     (6.962)         &     (3.371)         &     (3.213)         &     (3.360)         &     (1.326)         &     (2.314)         &     (1.861)         &     (2.228)         \\
 \midrule\multicolumn{13}{l}{\emph{Panel C. Up to the age of 31}} \\ Cum. numbers        &       171.5         &       192.5         &       286.0\sym{**} &       298.6\sym{*}  &       99.00         &       103.5         &       86.33         &       68.20         &       72.50         &       89.00         &       199.7\sym{**} &       230.4\sym{*}  \\
                    &     (179.4)         &     (183.3)         &     (134.9)         &     (163.8)         &     (58.60)         &     (87.04)         &     (65.52)         &     (70.31)         &     (196.4)         &     (124.5)         &     (96.50)         &     (113.8)         \\
 Cum. ratio          &      0.0476         &      -4.785         &      -6.517\sym{**} &      -8.215\sym{***}&      -0.632         &      -6.224         &      -12.72\sym{***}&      -15.78\sym{***}&       0.488         &      -3.509         &      -1.073         &      -1.476         \\
                    &     (1.520)         &     (3.386)         &     (2.442)         &     (2.603)         &     (5.788)         &     (3.568)         &     (3.857)         &     (3.823)         &     (3.895)         &     (4.614)         &     (3.190)         &     (3.815)         \\
 \midrule\multicolumn{13}{l}{\emph{Panel D. Up to the age of 34}} \\ Cum. numbers        &       199.0         &       190.2         &       348.0\sym{**} &       322.6\sym{*}  &         107         &       101.7         &       102.3         &       54.80         &       92.00         &       88.50         &       245.7\sym{**} &       267.8\sym{*}  \\
                    &     (238.1)         &     (174.4)         &     (139.1)         &     (168.1)         &     (190.1)         &     (92.45)         &     (75.44)         &     (66.68)         &     (233.2)         &     (137.8)         &     (113.7)         &     (134.1)         \\
 Cum. ratio          &      -0.331         &      -7.165         &      -8.492\sym{**} &      -11.48\sym{**} &      -1.351         &      -8.708         &      -15.66\sym{***}&      -20.46\sym{***}&       0.438         &      -5.782         &      -2.138         &      -3.395         \\
                    &     (5.458)         &     (5.766)         &     (3.922)         &     (4.048)         &     (11.73)         &     (6.671)         &     (5.357)         &     (4.915)         &     (4.827)         &     (6.487)         &     (4.696)         &     (5.472)         \\
 
\bottomrule \end{tabular} } \begin{tablenotes} \item \scriptsize \emph{Notes:} Clustered standard errors in parentheses (MxY). All regressions contain Birthmonth FE. Ratios indicate cases per thousand; original number of births. \end{tablenotes} \end{threeparttable} \end{table} 

\end{landscape}
\begin{landscape}
 \begin{table}[H] \begin{threeparttable} \centering \caption{Cummulative effects for upt to different points of age - BOOTSTRAPPED} {\def\sym#1{\ifmmode^{#1}\else\(^{#1}\)\fi} \begin{tabular}{l*{13}{c}} \toprule & \multicolumn{12}{c}{Dependent variable: \textbf{Diseases of the digestive system}} \\ \cmidrule(lr){2-13}
            &\multicolumn{4}{c}{Average Causal Effects}         &\multicolumn{4}{c}{Women}                          &\multicolumn{4}{c}{Men}                            \\\cmidrule(lr){2-5}\cmidrule(lr){6-9}\cmidrule(lr){10-13}
            &\multicolumn{1}{c}{(1)}&\multicolumn{1}{c}{(2)}&\multicolumn{1}{c}{(3)}&\multicolumn{1}{c}{(4)}&\multicolumn{1}{c}{(5)}&\multicolumn{1}{c}{(6)}&\multicolumn{1}{c}{(7)}&\multicolumn{1}{c}{(8)}&\multicolumn{1}{c}{(9)}&\multicolumn{1}{c}{(10)}&\multicolumn{1}{c}{(11)}&\multicolumn{1}{c}{(12)}\\
            &\multicolumn{1}{c}{2M}&\multicolumn{1}{c}{4M}&\multicolumn{1}{c}{6M}&\multicolumn{1}{c}{Donut}&\multicolumn{1}{c}{2M}&\multicolumn{1}{c}{4M}&\multicolumn{1}{c}{6M}&\multicolumn{1}{c}{Donut}&\multicolumn{1}{c}{2M}&\multicolumn{1}{c}{4M}&\multicolumn{1}{c}{6M}&\multicolumn{1}{c}{Donut}\\
\midrule
 \multicolumn{13}{l}{\emph{Panel A. 2 Up to the age of 21}} \\ Cum. numbers        &       119.5         &       203.5         &       166.5\sym{**} &       188.4\sym{*}  &       25.00         &       82.00         &       83.50         &       98.20         &       94.50\sym{***}&       121.5\sym{***}&       83.00\sym{***}&       90.20\sym{**} \\
                    &     (101.9)         &     (125.2)         &     (84.37)         &     (106.9)         &     (77.56)         &     (108.9)         &     (69.84)         &     (83.11)         &     (33.82)         &     (24.98)         &     (30.90)         &     (36.59)         \\
 Cum. ratio          &       0.862         &       0.194         &      -2.213         &      -2.563         &      -1.300         &      -1.791         &      -4.416\sym{*}  &      -4.806\sym{*}  &       2.785\sym{**} &       2.034         &      -0.407         &      -0.716         \\
                    &     (0.814)         &     (0.695)         &     (1.672)         &     (1.776)         &     (1.350)         &     (1.759)         &     (2.433)         &     (2.491)         &     (1.290)         &     (1.746)         &     (1.793)         &     (1.896)         \\
 \midrule\multicolumn{13}{l}{\emph{Panel B. Up to the age of 26}} \\ Cum. numbers        &       111.0         &       205.2         &       211.2\sym{*}  &       211.0         &       57.50         &       107.7         &       114.2         &       99.40         &       53.50         &       97.50\sym{*}  &       97.00\sym{*}  &       111.6         \\
                    &     (67.91)         &     (134.8)         &     (123.2)         &     (162.3)         &     (100.9)         &     (99.60)         &     (83.46)         &     (102.7)         &     (37.97)         &     (57.97)         &     (57.18)         &     (73.48)         \\
 Cum. ratio          &      -0.202         &      -2.099         &      -4.638         &      -6.028\sym{*}  &      -1.164         &      -3.417         &      -7.378         &      -9.658\sym{**} &       0.529         &      -0.936         &      -2.435         &      -2.977         \\
                    &     (3.915)         &     (3.494)         &     (3.181)         &     (3.214)         &     (6.686)         &     (4.570)         &     (4.635)         &     (4.715)         &     (1.306)         &     (3.315)         &     (2.882)         &     (2.979)         \\
 \midrule\multicolumn{13}{l}{\emph{Panel C. Up to the age of 31}} \\ Cum. numbers        &       171.5         &       192.5         &       286.0         &       298.6         &       99.00\sym{*}  &       103.5         &       86.33         &       68.20         &       72.50         &       89.00         &       199.7         &       230.4         \\
                    &     (176.1)         &     (252.7)         &     (185.1)         &     (244.6)         &     (51.62)         &     (118.0)         &     (90.07)         &     (102.2)         &     (194.2)         &     (170.1)         &     (129.9)         &     (165.7)         \\
 Cum. ratio          &      0.0476         &      -4.785         &      -6.517\sym{*}  &      -8.215\sym{**} &      -0.632         &      -6.224         &      -12.72\sym{**} &      -15.78\sym{***}&       0.488         &      -3.509         &      -1.073         &      -1.476         \\
                    &     (1.358)         &     (4.468)         &     (3.442)         &     (3.649)         &     (5.641)         &     (4.639)         &     (5.142)         &     (4.701)         &     (3.844)         &     (6.258)         &     (4.708)         &     (5.559)         \\
 \midrule\multicolumn{13}{l}{\emph{Panel D. Up to the age of 34}} \\ Cum. numbers        &       199.0         &       190.2         &       348.0\sym{*}  &       322.6         &         107         &       101.7         &       102.3         &       54.80         &       92.00         &       88.50         &       245.7         &       267.8         \\
                    &     (214.0)         &     (244.8)         &     (201.8)         &     (234.3)         &     (171.1)         &     (119.4)         &     (106.2)         &     (89.05)         &     (229.7)         &     (192.0)         &     (156.2)         &     (186.0)         \\
 Cum. ratio          &      -0.331         &      -7.165         &      -8.492\sym{*}  &      -11.48\sym{**} &      -1.351         &      -8.708         &      -15.66\sym{**} &      -20.46\sym{***}&       0.438         &      -5.782         &      -2.138         &      -3.395         \\
                    &     (4.900)         &     (7.846)         &     (5.161)         &     (5.659)         &     (11.18)         &     (9.017)         &     (6.726)         &     (6.431)         &     (4.571)         &     (8.889)         &     (6.636)         &     (7.798)         \\
 
\bottomrule \end{tabular} } \begin{tablenotes} \item \scriptsize \emph{Notes:} \textbf{BOOTSTRAPPED} standard errors in parentheses (MxY), with 400 replications. All regressions contain Birthmonth FE. Ratios indicate cases per thousand; original number of births. \end{tablenotes} \end{threeparttable} \end{table} 

\end{landscape}
%---------------------------------
\newpage
FEBRUAR CASES:
 \begin{table}[H] \begin{threeparttable} \centering \caption{Dep. variable: \textbf{Diseases of the digestive system}} {\def\sym#1{\ifmmode^{#1}\else\(^{#1}\)\fi} \begin{tabular}{l*{13}{c}} \toprule year & \multicolumn{12}{c}{Month of birth} \\ \cmidrule(lr){2-13} 
            &          11&          12&           1&           2&           3&           4&           5&           6&           7&           8&           9&          10\\
1995        &         886&         916&         862&         994&        1057&         937&         958&         976&        1001&         963&         982&         990\\
1996        &         765&         812&         860&         832&         969&         883&         899&         900&         857&         891&         921&         897\\
1997        &         793&         843&         822&         868&         963&         908&         918&         887&         888&         923&         897&         840\\
1998        &         846&         790&         819&         826&         907&         889&         908&         920&         866&         879&         920&         904\\
1999        &         726&         738&         770&         750&         976&         819&         847&         791&         820&         783&         808&         851\\
2000        &         767&         741&         786&         738&         815&         891&         902&         857&         880&         818&         808&         862\\
2001        &         770&         725&         776&         746&         861&         835&         878&         764&         810&         752&         798&         773\\
2002        &         735&         742&         761&         724&         767&         789&         806&         775&         845&         802&         820&         744\\
2003        &         650&         690&         775&         694&         766&         768&         728&         769&         802&         754&         728&         729\\
2004        &         656&         659&         650&         678&         682&         714&         756&         724&         756&         706&         645&         725\\
2005        &         657&         677&         617&         628&         713&         664&         716&         674&         724&         674&         653&         601\\
2006        &         620&         674&         628&         655&         710&         651&         721&         727&         728&         704&         684&         653\\
2007        &         641&         605&         647&         648&         762&         666&         762&         751&         669&         693&         672&         728\\
2008        &         644&         701&         688&         691&         793&         731&         733&         770&         743&         685&         685&         740\\
2009        &         666&         712&         710&         726&         866&         787&         775&         824&         773&         775&         738&         723\\
2010        &         675&         702&         713&         670&         786&         795&         736&         760&         762&         786&         751&         742\\
2011        &         723&         769&         788&         770&         820&         816&         797&         821&         825&         830&         752&         806\\
2012        &         750&         778&         861&         775&         858&         845&         822&         806&         888&         898&         802&         812\\
2013        &         835&         833&         887&         904&         932&         928&         988&         906&         959&         903&         890&         811\\
2014        &         968&         935&        1034&         982&        1094&         983&        1013&        1019&        1063&         948&         983&         974\\
 \bottomrule \end{tabular} } \begin{tablenotes} \item \scriptsize \emph{Notes:} Number of cases per year and MOB in treatment cohort. \end{tablenotes} \end{threeparttable} \end{table} 

 \begin{table}[H] \begin{threeparttable} \centering \caption{Dep. variable: \textbf{Diseases of the digestive system}} {\def\sym#1{\ifmmode^{#1}\else\(^{#1}\)\fi} \begin{tabular}{l*{13}{c}} \toprule year & \multicolumn{12}{c}{Month of birth} \\ \cmidrule(lr){2-13} 
            &          11&          12&           1&           2&           3&           4&           5&           6&           7&           8&           9&          10\\
1995        &          28&         -12&         -58&          82&         114&         -30&         -54&          26&           7&         -32&           4&          -4\\
1996        &        -114&         -82&          60&         -76&          -7&         -43&         -42&         -32&         -40&         -66&          29&          20\\
1997        &         -45&           9&         -50&          13&          19&         -28&         -32&         -51&         -79&         -28&         -18&        -136\\
1998        &          17&         -38&         -77&           1&         -38&           7&         -49&           7&         -65&         -48&           7&          -6\\
1999        &           9&         -66&         -52&           2&          39&         -70&         -28&         -81&        -126&        -128&        -104&         -43\\
2000        &          31&         -43&         -64&         -31&         -94&          76&          71&          24&          30&         -76&         -25&         -53\\
2001        &          64&         -36&         -37&         -23&         -12&          76&          -9&         -70&         -52&         -85&         -51&         -94\\
2002        &          26&         -11&           2&         -55&         -40&         -16&          -1&         -26&           9&           4&          47&         -63\\
2003        &          26&          12&          44&           2&         -32&          38&         -71&          42&          76&         -58&          12&           2\\
2004        &          -1&          -1&         -24&          31&         -87&         -14&          77&          25&          28&         -35&         -35&         -30\\
2005        &          77&          67&          33&         -69&          38&          -4&         -30&         -13&          17&         -45&          58&        -126\\
2006        &           1&          49&         -83&         -35&           8&         -64&         -24&           0&          32&          -1&         -69&         -32\\
2007        &          13&         -19&          14&         -48&          76&         -19&          78&          44&         -52&         -26&         -32&          -5\\
2008        &         -15&           9&          -1&          27&          65&          32&         -43&          31&         -10&         -78&         -46&          54\\
2009        &          26&           5&           5&         -17&         174&          35&         -71&         118&           8&          20&          -1&         -56\\
2010        &         -32&          39&          30&         -82&          13&          72&          -6&          46&          48&          53&         -24&         -17\\
2011        &          79&          64&          55&         -13&           8&          64&           5&          58&          23&          57&         -10&          75\\
2012        &          -9&         -11&          98&         -49&         -40&          -3&         -45&           0&          62&          96&         -18&         -11\\
2013        &          87&          35&          79&          48&         -15&          89&          22&          54&          30&         -25&         -40&         -48\\
2014        &          86&         125&         124&          77&          71&          34&          39&          60&          65&           5&          45&          20\\
 \bottomrule \end{tabular} } \begin{tablenotes} \item \scriptsize \emph{Notes:} Difference of cases (control - treatment) per year and MOB in treatment cohort. \end{tablenotes} \end{threeparttable} \end{table} 

