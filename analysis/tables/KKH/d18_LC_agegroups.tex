 \begin{table}[H] \centering \begin{threeparttable} \caption{Life-course approach - Table format} {\def\sym#1{\ifmmode^{#1}\else\(^{#1}\)\fi} \begin{tabular}{l*{5}{c}} \toprule \multicolumn{5}{l}{Dep. variable: \textbf{External causes of morbidity and mortality}} \\ & \multicolumn{4}{c}{Estimation window} \\ \cmidrule(lr){2-5}
            &\multicolumn{1}{c}{(1)}&\multicolumn{1}{c}{(2)}&\multicolumn{1}{c}{(3)}&\multicolumn{1}{c}{(4)}\\
            &\multicolumn{1}{c}{Age 17-21}&\multicolumn{1}{c}{Age 22-26}&\multicolumn{1}{c}{Age 27-31}&\multicolumn{1}{c}{Age 32-35}\\
\midrule
 \multicolumn{5}{l}{\emph{Panel A. Average causal effects}} \\ Abs. numbers        &       41.40         &       10.80         &       14.53         &       4.667         \\
                    &     (25.29)         &     (15.10)         &     (20.95)         &     (16.30)         \\
 Ratio fertility     &      -0.460         &      -0.685         &      -0.429         &      -0.612\sym{**} \\
                    &     (0.626)         &     (0.457)         &     (0.436)         &     (0.255)         \\
 Ratio population    &      -0.445         &      -0.286         &      -0.429\sym{**} \\
                    &     (0.463)         &     (0.352)         &     (0.207)         \\
 Cum. numbers        &       189.3\sym{**} &       281.6\sym{*}  &       315.1         &       357.4         \\
                    &     (78.08)         &     (158.9)         &     (215.9)         &     (269.9)         \\
 Cum. ratio          &      -2.334         &      -5.849         &      -8.989         &      -11.36         \\
                    &     (2.236)         &     (4.946)         &     (6.011)         &     (6.943)         \\
 \midrule\multicolumn{5}{l}{\emph{Panel B. Treatment effect heterogeneity - Women}} \\ Abs. numbers        &       8.033         &       7.767         &       7.700         &       1.542         \\
                    &     (8.977)         &     (11.81)         &     (9.072)         &     (11.39)         \\
 Ratio fertility     &      -0.663\sym{**} &      -0.322         &      -0.219         &      -0.498         \\
                    &     (0.260)         &     (0.472)         &     (0.450)         &     (0.342)         \\
 Ratio population    &      -0.269         &      -0.134         &      -0.338         \\
                    &     (0.375)         &     (0.339)         &     (0.252)         \\
 Cum. numbers        &       71.07\sym{*}  &       109.3         &       128.8         &       150.3         \\
                    &     (39.50)         &     (71.45)         &     (95.04)         &     (121.9)         \\
 Cum. ratio          &      -1.884\sym{**} &      -4.216\sym{*}  &      -6.157\sym{*}  &      -7.805\sym{**} \\
                    &     (0.832)         &     (2.315)         &     (3.447)         &     (3.977)         \\
 \midrule\multicolumn{5}{l}{\emph{Panel C. Treatment effect heterogeneity - Men}} \\ Abs. numbers        &       33.37         &       3.033         &       6.833         &       3.125         \\
                    &     (22.35)         &     (14.14)         &     (16.36)         &     (11.07)         \\
 Ratio fertility     &      -0.189         &      -0.965         &      -0.580         &      -0.680         \\
                    &     (1.083)         &     (0.822)         &     (0.618)         &     (0.553)         \\
 Ratio population    &      -0.567         &      -0.401         &      -0.493         \\
                    &     (0.802)         &     (0.513)         &     (0.449)         \\
 Cum. numbers        &       118.2         &       172.3         &       186.3         &       207.1         \\
                    &     (77.32)         &     (124.5)         &     (161.6)         &     (188.9)         \\
 Cum. ratio          &      -2.488         &      -6.764         &      -10.77         &      -13.65         \\
                    &     (4.135)         &     (7.875)         &     (9.586)         &     (11.07)         \\
 
\bottomrule \end{tabular} } \begin{tablenotes} \item \scriptsize \emph{Notes:} Clustered standard errors in parentheses. All regression are run with CG2 (i.e. the cohort prior to the reform) and with month-of-birth FEs. Ratios indicate cases per thousand; either approximated population (with weights coming from the original fertility distribution) or original number of births. Raqtio population muss eins nach rechts gerückt werden \end{tablenotes} \end{threeparttable} \end{table} 
