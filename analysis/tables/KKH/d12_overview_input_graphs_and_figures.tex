%---------------------------------
% INPUT FOR VARIABLE: d12
%---------------------------------
\subsection{d12}
% RD overview
\begin{landscape}
\begin{figure}[H]
	\centering
	\begin{minipage}{.95\linewidth}
	\includegraphics[width=\linewidth]{rd_d12_overview_panel1}
	{\scriptsize \emph{Notes:} The figures show monthly RD plots with averages obtained from a bin width of one month. The solid vertical line divides pre- and post-reform regime. The averages are taken over the period of at most 1995-2014. \par}
\end{minipage}
\end{figure}
\end{landscape}
\begin{landscape}
\begin{figure}[H]
	\centering
\begin{minipage}{.95\linewidth}
	\includegraphics[width=\linewidth]{rd_d12_overview_panel2}
	{\scriptsize \emph{Notes:} The figures show monthly RD plots with a moving average window width of 3 months. The solid vertical line divides pre- and post-reform regime. The averages are taken over the period of at most 1995-2014. \par}
\end{minipage}
\end{figure}
\end{landscape}
%---------------------------------
% TABELLEN
 \begin{table}[H] \begin{threeparttable} \centering \caption{Dep. variable: \textbf{Diseases of the musculoskeletal system and connective tissue}} {\def\sym#1{\ifmmode^{#1}\else\(^{#1}\)\fi} \begin{tabular}{l*{8}{c}} \toprule & \multicolumn{7}{c}{Estimation window} \\ \cmidrule(lr){2-8}
            &\multicolumn{1}{c}{(1)}&\multicolumn{1}{c}{(2)}&\multicolumn{1}{c}{(3)}&\multicolumn{1}{c}{(4)}&\multicolumn{1}{c}{(5)}&\multicolumn{1}{c}{(6)}&\multicolumn{1}{c}{(7)}\\
            &\multicolumn{1}{c}{1M}&\multicolumn{1}{c}{2M}&\multicolumn{1}{c}{3M}&\multicolumn{1}{c}{4M}&\multicolumn{1}{c}{5M}&\multicolumn{1}{c}{6M}&\multicolumn{1}{c}{Donut}\\
\midrule
 \multicolumn{8}{l}{\emph{Panel A. Average causal effects}} \\ Abs. numbers        &       3.350\sym{***}&       15.75\sym{**} &       18.43\sym{***}&       15.31\sym{***}&       13.17\sym{***}&       16.78\sym{***}&       19.47\sym{***}\\
                    &  (7.08e-14)         &     (4.767)         &     (3.624)         &     (4.771)         &     (4.323)         &     (4.079)         &     (4.202)         \\
 Ratio fertility     &      0.0326\sym{***}&       0.212\sym{*}  &       0.147\sym{*}  &      0.0539         &     -0.0123         &     -0.0356         &     -0.0493         \\
                    &  (2.02e-15)         &     (0.106)         &    (0.0750)         &    (0.0700)         &    (0.0642)         &    (0.0544)         &    (0.0593)         \\
 Ratio population    &      0.0337\sym{***}&       0.183\sym{*}  &       0.154\sym{**} &      0.0465         &    0.000772         &     0.00267         &    -0.00353         \\
                    &  (2.15e-15)         &    (0.0798)         &    (0.0542)         &    (0.0659)         &    (0.0576)         &    (0.0480)         &    (0.0549)         \\
 Ratio fert(03-14)   &      0.0455\sym{***}&       0.239\sym{*}  &       0.201\sym{**} &      0.0597         &     -0.0206         &     -0.0323         &     -0.0478         \\
                    &  (1.85e-15)         &     (0.104)         &    (0.0706)         &    (0.0862)         &    (0.0820)         &    (0.0695)         &    (0.0785)         \\
 Cum. numbers        &      -34.95\sym{***}&       131.4\sym{*}  &       152.5\sym{***}&       141.4\sym{**} &       110.8\sym{**} &       142.8\sym{***}&       178.4\sym{***}\\
                    &  (2.86e-12)         &     (64.13)         &     (45.23)         &     (50.45)         &     (42.73)         &     (38.92)         &     (38.61)         \\
 Cum. ratio          &      -0.103\sym{***}&       2.055\sym{*}  &       1.671\sym{**} &       1.142\sym{**} &       0.608         &       0.572         &       0.707         \\
                    &  (2.83e-14)         &     (0.888)         &     (0.604)         &     (0.508)         &     (0.519)         &     (0.434)         &     (0.517)         \\
 \midrule\multicolumn{8}{l}{\emph{Panel B. Treatment effect heterogeneity - Women}} \\ Abs. numbers        &       3.850\sym{***}&       10.25\sym{***}&       13.73\sym{***}&       12.16\sym{***}&       11.65\sym{***}&       12.59\sym{***}&       14.34\sym{***}\\
                    &  (5.52e-14)         &     (2.501)         &     (3.034)         &     (3.274)         &     (2.682)         &     (2.422)         &     (2.226)         \\
 Ratio fertility     &      0.0757\sym{***}&       0.301\sym{**} &       0.339\sym{***}&       0.249\sym{**} &       0.194\sym{**} &       0.124\sym{*}  &       0.134\sym{*}  \\
                    &  (2.49e-15)         &     (0.123)         &    (0.0923)         &    (0.0854)         &    (0.0724)         &    (0.0691)         &    (0.0685)         \\
 Ratio population    &       0.434\sym{***}&       0.571\sym{***}&       0.668\sym{***}&       0.506\sym{***}&       0.415\sym{***}&       0.342\sym{***}&       0.324\sym{***}\\
                    &  (1.89e-15)         &    (0.0535)         &    (0.0916)         &     (0.117)         &     (0.104)         &    (0.0944)         &     (0.107)         \\
 Ratio fert(03-14)   &       0.456\sym{***}&       0.595\sym{***}&       0.714\sym{***}&       0.515\sym{***}&       0.385\sym{***}&       0.274\sym{**} &       0.238\sym{*}  \\
                    &  (1.73e-15)         &    (0.0785)         &     (0.126)         &     (0.148)         &     (0.133)         &     (0.123)         &     (0.133)         \\
 Cum. numbers        &      -52.40\sym{***}&       52.80         &       78.55\sym{**} &       80.30\sym{**} &       80.71\sym{***}&       94.23\sym{***}&       123.6\sym{***}\\
                    &  (1.92e-13)         &     (44.85)         &     (34.99)         &     (30.29)         &     (24.30)         &     (22.48)         &     (13.50)         \\
 Cum. ratio          &      -1.279\sym{***}&       1.893         &       1.918\sym{*}  &       1.677\sym{**} &       1.555\sym{**} &       1.304\sym{**} &       1.821\sym{***}\\
                    &  (3.22e-14)         &     (1.404)         &     (0.930)         &     (0.705)         &     (0.586)         &     (0.506)         &     (0.469)         \\
 \midrule\multicolumn{8}{l}{\emph{Panel C. Treatment effect heterogeneity - Men}} \\ Abs. numbers        &      -0.500\sym{***}&       5.500\sym{**} &       4.700\sym{**} &       3.150         &       1.520         &       4.192\sym{*}  &       5.130\sym{*}  \\
                    &  (5.54e-14)         &     (2.318)         &     (1.659)         &     (1.999)         &     (2.137)         &     (2.178)         &     (2.533)         \\
 Ratio fertility     &      0.0457\sym{***}&       0.158\sym{**} &      0.0640         &     -0.0272         &     -0.0801         &     -0.0432         &     -0.0610         \\
                    &  (2.25e-15)         &    (0.0528)         &    (0.0684)         &    (0.0740)         &    (0.0639)         &    (0.0568)         &    (0.0665)         \\
 Ratio population    &      -0.264\sym{***}&     -0.0729         &      -0.220\sym{*}  &      -0.288\sym{**} &      -0.296\sym{***}&      -0.218\sym{**} &      -0.209\sym{*}  \\
                    &  (1.01e-15)         &    (0.0999)         &     (0.107)         &     (0.122)         &     (0.103)         &    (0.0933)         &     (0.112)         \\
 Ratio fert(03-14)   &      -0.182\sym{***}&     0.00916         &      -0.101         &      -0.189         &      -0.209\sym{*}  &      -0.126         &      -0.114         \\
                    &  (7.07e-16)         &    (0.0820)         &     (0.105)         &     (0.123)         &     (0.103)         &    (0.0946)         &     (0.111)         \\
 Cum. numbers        &       17.45\sym{***}&       78.55\sym{**} &       73.92\sym{***}&       61.07\sym{**} &       30.09         &       48.57\sym{**} &       54.80\sym{*}  \\
                    &  (8.79e-13)         &     (27.18)         &     (17.82)         &     (26.09)         &     (25.26)         &     (22.81)         &     (27.25)         \\
 Cum. ratio          &       1.010\sym{***}&       2.202\sym{***}&       1.431\sym{**} &       0.633         &      -0.289         &      -0.117         &      -0.342         \\
                    &  (3.51e-14)         &     (0.459)         &     (0.522)         &     (0.534)         &     (0.656)         &     (0.579)         &     (0.655)         \\
 
\bottomrule \end{tabular} } \begin{tablenotes} \item \scriptsize \emph{Notes:} Clustered standard errors in parentheses. All regression are run with CG2 (i.e. the cohort prior to the reform) and with month-of-birth FEs. Ratios indicate cases per thousand; either approximated population (with weights coming from the original fertility distribution) or original number of births. \end{tablenotes} \end{threeparttable} \end{table} 

 \begin{table}[H] \begin{threeparttable} \centering \caption{Robustness with respect to the choice of \texttt{control group}} {\def\sym#1{\ifmmode^{#1}\else\(^{#1}\)\fi} \begin{tabular}{l*{10}{c}} \toprule & \multicolumn{9}{c}{Dependent variable: \textbf{Diseases of the musculoskeletal system and connective tissue}} \\ \cmidrule(lr){2-10}
            &\multicolumn{3}{c}{Average Causal Effects}&\multicolumn{3}{c}{Women}             &\multicolumn{3}{c}{Men}               \\\cmidrule(lr){2-4}\cmidrule(lr){5-7}\cmidrule(lr){8-10}
            &\multicolumn{1}{c}{(1)}&\multicolumn{1}{c}{(2)}&\multicolumn{1}{c}{(3)}&\multicolumn{1}{c}{(4)}&\multicolumn{1}{c}{(5)}&\multicolumn{1}{c}{(6)}&\multicolumn{1}{c}{(7)}&\multicolumn{1}{c}{(8)}&\multicolumn{1}{c}{(9)}\\
            &\multicolumn{1}{c}{C2}&\multicolumn{1}{c}{C1+C2}&\multicolumn{1}{c}{C1-C3}&\multicolumn{1}{c}{C2}&\multicolumn{1}{c}{C1+C2}&\multicolumn{1}{c}{C1-C3}&\multicolumn{1}{c}{C2}&\multicolumn{1}{c}{C1+C2}&\multicolumn{1}{c}{C1-C3}\\
\midrule
 \multicolumn{10}{l}{\emph{Panel A. 2 Month bandwidth}} \\ Abs. numbers        &       15.75\sym{**} &       14.18         &       14.38         &       10.25\sym{***}&       7.475\sym{*}  &       5.858         &       5.500\sym{**} &       6.700         &       8.525         \\
                    &     (4.767)         &     (10.31)         &     (14.55)         &     (2.501)         &     (3.790)         &     (5.471)         &     (2.318)         &     (6.864)         &     (9.330)         \\
 Ratio fertility     &       0.244\sym{***}&       0.293\sym{*}  &       0.304         &       0.333\sym{***}&       0.327\sym{**} &       0.283         &       0.158\sym{**} &       0.261         &       0.323         \\
                    &    (0.0648)         &     (0.148)         &     (0.267)         &    (0.0783)         &     (0.117)         &     (0.229)         &    (0.0528)         &     (0.183)         &     (0.307)         \\
 Ratio population    &       0.183\sym{*}  &       0.289         &       0.309         &       0.441\sym{***}&       0.433\sym{**} &       0.372         &     -0.0729         &       0.147         &       0.249         \\
                    &    (0.0798)         &     (0.200)         &     (0.305)         &    (0.0600)         &     (0.148)         &     (0.280)         &    (0.0999)         &     (0.263)         &     (0.351)         \\
 \midrule\multicolumn{10}{l}{\emph{Panel B. 4 Month bandwidth}} \\ Abs. numbers        &       21.78\sym{**} &       13.69         &       8.648         &       21.44\sym{***}&       15.78\sym{**} &       11.54         &       0.333         &      -2.083         &      -2.889         \\
                    &     (7.643)         &     (10.52)         &     (12.71)         &     (6.120)         &     (6.234)         &     (7.043)         &     (4.047)         &     (5.968)         &     (7.380)         \\
 Ratio population    &      0.0165         &     -0.0302         &     -0.0188         &       0.620\sym{*}  &       0.499         &       0.349         &      -0.625\sym{**} &      -0.617\sym{*}  &      -0.431         \\
                    &     (0.141)         &     (0.204)         &     (0.270)         &     (0.341)         &     (0.292)         &     (0.352)         &     (0.286)         &     (0.323)         &     (0.396)         \\
 Ratio fertility     &       0.215\sym{**} &       0.192         &       0.129         &       0.563\sym{***}&       0.494\sym{***}&       0.373         &      -0.114         &     -0.0944         &      -0.102         \\
                    &    (0.0834)         &     (0.159)         &     (0.255)         &     (0.141)         &     (0.175)         &     (0.264)         &     (0.139)         &     (0.194)         &     (0.284)         \\
 \midrule\multicolumn{10}{l}{\emph{Panel C. 6 Month bandwidth}} \\ Abs. numbers        &       16.78\sym{***}&       12.47\sym{*}  &       8.206         &       12.59\sym{***}&       10.26\sym{***}&       6.922\sym{**} &       4.192\sym{*}  &       2.208         &       1.283         \\
                    &     (4.079)         &     (6.175)         &     (7.606)         &     (2.422)         &     (2.933)         &     (2.973)         &     (2.178)         &     (3.770)         &     (5.193)         \\
 Ratio fertility     &      0.0846\sym{**} &       0.121         &      0.0865         &       0.221\sym{***}&       0.247\sym{***}&       0.183         &     -0.0432         &     0.00162         &    -0.00427         \\
                    &    (0.0373)         &    (0.0887)         &     (0.161)         &    (0.0491)         &    (0.0837)         &     (0.132)         &    (0.0568)         &     (0.111)         &     (0.200)         \\
 Ratio fertility     &       0.193\sym{***}&       0.222         &       0.154         &       0.398\sym{***}&       0.433\sym{***}&       0.300         &     0.00241         &      0.0254         &      0.0186         \\
                    &    (0.0681)         &     (0.135)         &     (0.213)         &     (0.106)         &     (0.138)         &     (0.205)         &     (0.121)         &     (0.170)         &     (0.252)         \\
 \midrule\multicolumn{10}{l}{\emph{Panel D. Donut specification}} \\ Abs. numbers        &       29.16\sym{***}&       22.74\sym{**} &       14.30         &       19.04\sym{***}&       17.09\sym{***}&       10.50\sym{*}  &       10.11\sym{*}  &       5.656         &       3.800         \\
                    &     (7.974)         &     (10.76)         &     (12.67)         &     (4.580)         &     (5.486)         &     (6.006)         &     (5.340)         &     (6.437)         &     (8.086)         \\
 Ratio fertility     &     -0.0493         &      -0.129         &     -0.0897         &       0.134\sym{*}  &      0.0465         &      0.0315         &      -0.219\sym{**} &      -0.291\sym{*}  &      -0.201         \\
                    &    (0.0593)         &     (0.111)         &     (0.229)         &    (0.0685)         &     (0.112)         &     (0.184)         &    (0.0815)         &     (0.143)         &     (0.291)         \\
 Ratio population    &      0.0950         &       0.134         &      0.0582         &       0.324\sym{***}&       0.344\sym{**} &       0.210         &      -0.132         &     -0.0755         &     -0.0931         \\
                    &    (0.0643)         &     (0.137)         &     (0.222)         &     (0.107)         &     (0.152)         &     (0.208)         &     (0.122)         &     (0.165)         &     (0.270)         \\
 
\bottomrule \end{tabular} } \begin{tablenotes} \item \scriptsize \emph{Notes:} Clustered standard errors in parentheses. All regressions contain Birthmonth FE. Ratios indicate cases per thousand; either approximated population or original number of births. \end{tablenotes} \end{threeparttable} \end{table} 

%---------------------------------
% Life-course figure (Panel1)
\begin{landscape}
\begin{figure}[H]
\centering
\begin{minipage}{.9\linewidth}
\includegraphics[width=\linewidth]{lc_d12_overview_panel1}
{\scriptsize \emph{Notes:} The figures depict DDRD estimates and 90\% confidence intervals over the life-course. The years are harmonized such that the cohorts are in the same age when they are compared. All regressions are carried out with month-of-birth FE and make use of clustered standard errors. Furthermore, we used a bandwidth of half a year and only the control cohort that was born one year prior to the reform. Ratios indicate cases per thousand; using in the denominator the approximated population (with weights coming from the original fertility distribution) or original number of births. \par}
\end{minipage}
\end{figure}
\end{landscape}
%---------------------------------
% Life-course figure (Panel2)
\begin{landscape}
\begin{figure}[H]
\centering
\begin{minipage}{.9\linewidth}
\includegraphics[width=\linewidth]{lc_d12_overview_panel2}
{\scriptsize \emph{Notes:} The figures depict DDRD estimates and 90\% confidence intervals over the life-course. The years are harmonized such that the cohorts are in the same age when they are compared. All regressions are carried out with month-of-birth FE and make use of clustered standard errors. Furthermore, we used a bandwidth of half a year. Ratios indicate cases per thousand; using in the denominator the approximated population (with weights coming from the original fertility distribution) or original number of births. \par}
\end{minipage}
\end{figure}
\end{landscape}
%---------------------------------
% Life-course (panel 3 - 6)
\begin{figure}[H]%\vspace*{-2cm}
	\centering
	\includegraphics[width=.9\linewidth]{lc_d12_overview_panel3}
	\includegraphics[width=.9\linewidth]{lc_d12_overview_panel4}
\end{figure}
\begin{figure}[H]
	\centering	
	\includegraphics[width=.97\linewidth]{lc_d12_overview_panel5}
	\includegraphics[width=.97\linewidth]{lc_d12_overview_panel6}
\end{figure}
% Life-course TABLE Format
 \begin{table}[H] \centering \begin{threeparttable} \caption{Life-course approach - Table format} {\def\sym#1{\ifmmode^{#1}\else\(^{#1}\)\fi} \begin{tabular}{l*{5}{c}} \toprule \multicolumn{5}{l}{Dep. variable: \textbf{Diseases of the musculoskeletal system and connective tissue}} \\ & \multicolumn{4}{c}{Estimation window} \\ \cmidrule(lr){2-5}
            &\multicolumn{1}{c}{(1)}&\multicolumn{1}{c}{(2)}&\multicolumn{1}{c}{(3)}&\multicolumn{1}{c}{(4)}\\
            &\multicolumn{1}{c}{Age 17-21}&\multicolumn{1}{c}{Age 22-26}&\multicolumn{1}{c}{Age 27-31}&\multicolumn{1}{c}{Age 32-35}\\
\midrule
 \multicolumn{5}{l}{\emph{Panel A. Average causal effects}} \\ Abs. numbers        &       14.23         &       9.100         &       16.07         &       35.37\sym{*}  \\
                    &     (9.261)         &     (12.74)         &     (14.80)         &     (20.22)         \\
 Ratio fertility     &     -0.0260         &      -0.155         &     -0.0562         &       0.152         \\
                    &     (0.179)         &     (0.306)         &     (0.296)         &     (0.302)         \\
 Ratio population    &      -0.215         &     -0.0204         &       0.146         \\
                    &     (0.305)         &     (0.227)         &     (0.220)         \\
 Cum. numbers        &       60.57         &       120.5\sym{**} &       166.4\sym{*}  &       288.0\sym{**} \\
                    &     (38.45)         &     (48.81)         &     (95.49)         &     (137.1)         \\
 Cum. ratio          &      -0.232         &      -0.694         &      -1.497         &      -1.226         \\
                    &     (0.707)         &     (1.312)         &     (2.415)         &     (2.626)         \\
 \midrule\multicolumn{5}{l}{\emph{Panel B. Treatment effect heterogeneity - Women}} \\ Abs. numbers        &       11.13         &       2.300         &       14.43         &       25.83\sym{*}  \\
                    &     (7.741)         &     (8.215)         &     (8.994)         &     (14.10)         \\
 Ratio fertility     &       0.119         &      -0.267         &       0.207         &       0.498         \\
                    &     (0.287)         &     (0.408)         &     (0.385)         &     (0.483)         \\
 Ratio population    &     -0.0865         &       0.174         &       0.399         \\
                    &     (0.390)         &     (0.286)         &     (0.349)         \\
 Cum. numbers        &       40.73         &       70.13\sym{**} &       112.9\sym{*}  &       195.0\sym{**} \\
                    &     (28.81)         &     (32.95)         &     (58.15)         &     (76.06)         \\
 Cum. ratio          &      0.0364         &      -0.518         &      -0.555         &       0.604         \\
                    &     (1.034)         &     (1.569)         &     (2.968)         &     (3.137)         \\
 \midrule\multicolumn{5}{l}{\emph{Panel C. Treatment effect heterogeneity - Men}} \\ Abs. numbers        &       3.100         &       6.800         &       1.633         &       9.542         \\
                    &     (8.842)         &     (7.580)         &     (11.84)         &     (8.424)         \\
 Ratio fertility     &      -0.167         &     -0.0445         &      -0.298         &      -0.168         \\
                    &     (0.338)         &     (0.336)         &     (0.464)         &     (0.281)         \\
 Ratio population    &      -0.338         &      -0.208         &      -0.102         \\
                    &     (0.340)         &     (0.364)         &     (0.211)         \\
 Cum. numbers        &       19.83         &       50.37         &       53.50         &       93.00         \\
                    &     (32.78)         &     (48.25)         &     (55.45)         &     (83.50)         \\
 Cum. ratio          &      -0.508         &      -0.876         &      -2.376         &      -2.902         \\
                    &     (1.233)         &     (1.958)         &     (2.378)         &     (2.958)         \\
 
\bottomrule \end{tabular} } \begin{tablenotes} \item \scriptsize \emph{Notes:} Clustered standard errors in parentheses. All regression are run with CG2 (i.e. the cohort prior to the reform) and with month-of-birth FEs. Ratios indicate cases per thousand; either approximated population (with weights coming from the original fertility distribution) or original number of births. Raqtio population muss eins nach rechts gerückt werden \end{tablenotes} \end{threeparttable} \end{table} 

%---------------------------------
% PLACEBO EXERCISES
\newpage
\begin{landscape}
\begin{figure}[H]
	\centering
    \begin{minipage}{.9\linewidth}
	\includegraphics[width=\linewidth]{placebo_graph_d12.pdf}
    {\scriptsize \emph{Notes:} The figures depict DDRD estimates and 95\% confidence intervals when the treatment cohort is shifted over time. The date on the abscissa indicates the starting date of the treated.  All regressions are carried out with month-of-birth FE and make use of clustered standard errors. Furthermore, we used a bandwidth of half a year. Ratios indicate cases per thousand; using in the denominator the approximated population (with weights coming from the original fertility distribution) or original number of births. \par}
    \end{minipage}
\end{figure}
\end{landscape}
 \begin{table}[H] \centering \begin{threeparttable} \caption{Placebo 1 (CONTROL1 ist TREAT) } {\def\sym#1{\ifmmode^{#1}\else\(^{#1}\)\fi} \begin{tabular}{l*{4}{c}} \toprule \multicolumn{4}{l}{Dep. variable: \textbf{Diseases of the musculoskeletal system and connective tissue}} \\ & \multicolumn{3}{c}{Choice of control group} \\ \cmidrule(lr){2-4}
            &\multicolumn{1}{c}{(1)}&\multicolumn{1}{c}{(2)}&\multicolumn{1}{c}{(3)}\\
            &\multicolumn{1}{c}{C2}&\multicolumn{1}{c}{C3}&\multicolumn{1}{c}{C2+C3}\\
\midrule
 \multicolumn{4}{l}{\emph{Panel A. Average causal effects}} \\ Abs. numbers        &       8.625\sym{**} &      -8.483         &      0.0708         \\
                    &     (3.556)         &     (5.092)         &     (7.047)         \\
 Ratio fertility     &      0.0466         &       0.120         &      0.0832         \\
                    &    (0.0981)         &     (0.116)         &     (0.235)         \\
 Ratio population    &     -0.0587         &      -0.116         &     -0.0875         \\
                    &    (0.0802)         &    (0.0971)         &     (0.190)         \\
 Cum. numbers        &       83.94\sym{**} &      -36.79         &       23.57         \\
                    &     (38.79)         &     (52.27)         &     (64.50)         \\
 Cum. ratio          &       0.551         &       2.057         &       1.304         \\
                    &     (1.067)         &     (1.224)         &     (2.175)         \\
 \midrule\multicolumn{4}{l}{\emph{Panel B. Treatment effect heterogeneity - Women}} \\ Abs. numbers        &       4.658\sym{*}  &      -7.692\sym{**} &      -1.517         \\
                    &     (2.292)         &     (2.938)         &     (3.230)         \\
 Ratio fertility     &      0.0730         &     -0.0142         &      0.0294         \\
                    &     (0.114)         &     (0.144)         &     (0.206)         \\
 Ratio population    &   -0.000730         &      -0.256\sym{**} &      -0.129         \\
                    &     (0.107)         &    (0.0958)         &     (0.180)         \\
 Cum. numbers        &       27.22         &      -59.72\sym{*}  &      -16.25         \\
                    &     (23.86)         &     (33.02)         &     (30.23)         \\
 Cum. ratio          &      0.0560         &       0.643         &       0.350         \\
                    &     (1.271)         &     (1.613)         &     (1.853)         \\
 \midrule\multicolumn{4}{l}{\emph{Panel C. Treatment effect heterogeneity - Men}} \\ Abs. numbers        &       3.967         &      -0.792         &       1.587         \\
                    &     (2.401)         &     (3.363)         &     (4.835)         \\
 Ratio fertility     &      0.0212         &       0.246         &       0.134         \\
                    &     (0.122)         &     (0.148)         &     (0.289)         \\
 Ratio population    &      -0.115         &      0.0248         &     -0.0452         \\
                    &     (0.117)         &     (0.149)         &     (0.232)         \\
 Cum. numbers        &       56.72\sym{**} &       22.92         &       39.82         \\
                    &     (23.44)         &     (32.49)         &     (48.46)         \\
 Cum. ratio          &       1.006         &       3.390\sym{**} &       2.198         \\
                    &     (1.173)         &     (1.479)         &     (2.828)         \\
 
\bottomrule \end{tabular} } \begin{tablenotes} \item \scriptsize \emph{Notes:} Clustered standard errors in parentheses. All regression are run with month-of-birth FEs and control cohort 2 is assigned with the treatment status. All regressions are carried out with a window width of half a year. \end{tablenotes} \end{threeparttable} \end{table} 

 \begin{table}[H] \centering \begin{threeparttable} \caption{Placebo 2 (CONTROL2 ist TREAT) } {\def\sym#1{\ifmmode^{#1}\else\(^{#1}\)\fi} \begin{tabular}{l*{4}{c}} \toprule \multicolumn{4}{l}{Dep. variable: \textbf{Diseases of the musculoskeletal system and connective tissue}} \\ & \multicolumn{3}{c}{Choice of control group} \\ \cmidrule(lr){2-4}
            &\multicolumn{1}{c}{(1)}&\multicolumn{1}{c}{(2)}&\multicolumn{1}{c}{(3)}\\
            &\multicolumn{1}{c}{C1}&\multicolumn{1}{c}{C3}&\multicolumn{1}{c}{C1+C3}\\
\midrule
 \multicolumn{4}{l}{\emph{Panel A. Average causal effects}} \\ Abs. numbers        &      -8.625\sym{**} &      -17.11\sym{***}&      -12.87         \\
                    &     (3.556)         &     (4.058)         &     (10.35)         \\
 Ratio fertility     &     -0.0466         &      0.0731         &      0.0132         \\
                    &    (0.0981)         &    (0.0620)         &     (0.293)         \\
 Ratio population    &      0.0587         &     -0.0576         &    0.000540         \\
                    &    (0.0802)         &    (0.0641)         &     (0.282)         \\
 Cum. numbers        &      -83.94\sym{**} &      -120.7\sym{***}&      -102.3         \\
                    &     (38.79)         &     (38.15)         &     (84.95)         \\
 Cum. ratio          &      -0.551         &       1.507\sym{***}&       0.478         \\
                    &     (1.067)         &     (0.524)         &     (2.500)         \\
 \midrule\multicolumn{4}{l}{\emph{Panel B. Treatment effect heterogeneity - Women}} \\ Abs. numbers        &      -4.658\sym{*}  &      -12.35\sym{***}&      -8.504\sym{**} \\
                    &     (2.292)         &     (2.638)         &     (4.036)         \\
 Ratio fertility     &     -0.0730         &     -0.0871         &     -0.0800         \\
                    &     (0.114)         &    (0.0618)         &     (0.237)         \\
 Ratio population    &    0.000730         &      -0.256\sym{***}&      -0.127         \\
                    &     (0.107)         &    (0.0543)         &     (0.262)         \\
 Cum. numbers        &      -27.22         &      -86.93\sym{***}&      -57.08\sym{*}  \\
                    &     (23.86)         &     (26.55)         &     (28.26)         \\
 Cum. ratio          &     -0.0560         &       0.587         &       0.266         \\
                    &     (1.271)         &     (0.690)         &     (1.678)         \\
 \midrule\multicolumn{4}{l}{\emph{Panel C. Treatment effect heterogeneity - Men}} \\ Abs. numbers        &      -3.967         &      -4.758\sym{*}  &      -4.363         \\
                    &     (2.401)         &     (2.489)         &     (6.925)         \\
 Ratio fertility     &     -0.0212         &       0.225\sym{*}  &       0.102         \\
                    &     (0.122)         &     (0.111)         &     (0.361)         \\
 Ratio population    &       0.115         &       0.140         &       0.128         \\
                    &     (0.117)         &     (0.119)         &     (0.319)         \\
 Cum. numbers        &      -56.73\sym{**} &      -33.80         &      -45.26         \\
                    &     (23.44)         &     (22.33)         &     (68.05)         \\
 Cum. ratio          &      -1.006         &       2.383\sym{**} &       0.688         \\
                    &     (1.173)         &     (0.959)         &     (3.487)         \\
 
\bottomrule \end{tabular} } \begin{tablenotes} \item \scriptsize \emph{Notes:} Clustered standard errors in parentheses. All regression are run with month-of-birth FEs and control cohort 2 is assigned with the treatment status. All regressions are carried out with a window width of half a year. \end{tablenotes} \end{threeparttable} \end{table} 

 \begin{table}[H] \centering \begin{threeparttable} \caption{Placebo 3 (CONTROL3 ist TREAT) } {\def\sym#1{\ifmmode^{#1}\else\(^{#1}\)\fi} \begin{tabular}{l*{4}{c}} \toprule \multicolumn{4}{l}{Dep. variable: \textbf{Diseases of the musculoskeletal system and connective tissue}} \\ & \multicolumn{3}{c}{Choice of control group} \\ \cmidrule(lr){2-4}
            &\multicolumn{1}{c}{(1)}&\multicolumn{1}{c}{(2)}&\multicolumn{1}{c}{(3)}\\
            &\multicolumn{1}{c}{C1}&\multicolumn{1}{c}{C2}&\multicolumn{1}{c}{C1+C2}\\
\midrule
 \multicolumn{4}{l}{\emph{Panel A. Average causal effects}} \\ Abs. numbers        &       8.483         &       17.11\sym{***}&       12.80\sym{*}  \\
                    &     (5.092)         &     (4.058)         &     (6.355)         \\
 Ratio fertility     &      -0.120         &     -0.0731         &     -0.0964         \\
                    &     (0.116)         &    (0.0620)         &     (0.117)         \\
 Ratio population    &       0.116         &      0.0576         &      0.0870         \\
                    &    (0.0971)         &    (0.0641)         &     (0.132)         \\
 Cum. numbers        &       36.79         &       120.7\sym{***}&       78.76         \\
                    &     (52.27)         &     (38.15)         &     (56.03)         \\
 Cum. ratio          &      -2.057         &      -1.507\sym{***}&      -1.782         \\
                    &     (1.224)         &     (0.524)         &     (1.058)         \\
 \midrule\multicolumn{4}{l}{\emph{Panel B. Treatment effect heterogeneity - Women}} \\ Abs. numbers        &       7.692\sym{**} &       12.35\sym{***}&       10.02\sym{***}\\
                    &     (2.938)         &     (2.638)         &     (3.278)         \\
 Ratio fertility     &      0.0142         &      0.0871         &      0.0507         \\
                    &     (0.144)         &    (0.0618)         &     (0.124)         \\
 Ratio population    &       0.256\sym{**} &       0.256\sym{***}&       0.256\sym{*}  \\
                    &    (0.0958)         &    (0.0543)         &     (0.126)         \\
 Cum. numbers        &       59.72\sym{*}  &       86.93\sym{***}&       73.33\sym{**} \\
                    &     (33.02)         &     (26.55)         &     (30.60)         \\
 Cum. ratio          &      -0.643         &      -0.587         &      -0.615         \\
                    &     (1.613)         &     (0.690)         &     (1.237)         \\
 \midrule\multicolumn{4}{l}{\emph{Panel C. Treatment effect heterogeneity - Men}} \\ Abs. numbers        &       0.792         &       4.758\sym{*}  &       2.775         \\
                    &     (3.363)         &     (2.489)         &     (3.963)         \\
 Ratio fertility     &      -0.246         &      -0.225\sym{*}  &      -0.235         \\
                    &     (0.148)         &     (0.111)         &     (0.156)         \\
 Ratio population    &     -0.0248         &      -0.140         &     -0.0823         \\
                    &     (0.149)         &     (0.119)         &     (0.171)         \\
 Cum. numbers        &      -22.92         &       33.80         &       5.438         \\
                    &     (32.49)         &     (22.33)         &     (37.49)         \\
 Cum. ratio          &      -3.390\sym{**} &      -2.383\sym{**} &      -2.886\sym{*}  \\
                    &     (1.479)         &     (0.959)         &     (1.480)         \\
 
\bottomrule \end{tabular} } \begin{tablenotes} \item \scriptsize \emph{Notes:} Clustered standard errors in parentheses. All regression are run with month-of-birth FEs and control cohort 3 is assigned with the treatment status. All regressions are carried out with a window width of half a year. \end{tablenotes} \end{threeparttable} \end{table} 

%---------------------------------
% CUMMULATIVE APPROACH
\begin{landscape}
 \begin{table}[H] \begin{threeparttable} \centering \caption{Cummulative effects for upt to different points of age} {\def\sym#1{\ifmmode^{#1}\else\(^{#1}\)\fi} \begin{tabular}{l*{13}{c}} \toprule & \multicolumn{12}{c}{Dependent variable: \textbf{Diseases of the musculoskeletal system and connective tissue}} \\ \cmidrule(lr){2-13}
            &\multicolumn{4}{c}{Average Causal Effects}         &\multicolumn{4}{c}{Women}                          &\multicolumn{4}{c}{Men}                            \\\cmidrule(lr){2-5}\cmidrule(lr){6-9}\cmidrule(lr){10-13}
            &\multicolumn{1}{c}{(1)}&\multicolumn{1}{c}{(2)}&\multicolumn{1}{c}{(3)}&\multicolumn{1}{c}{(4)}&\multicolumn{1}{c}{(5)}&\multicolumn{1}{c}{(6)}&\multicolumn{1}{c}{(7)}&\multicolumn{1}{c}{(8)}&\multicolumn{1}{c}{(9)}&\multicolumn{1}{c}{(10)}&\multicolumn{1}{c}{(11)}&\multicolumn{1}{c}{(12)}\\
            &\multicolumn{1}{c}{2M}&\multicolumn{1}{c}{4M}&\multicolumn{1}{c}{6M}&\multicolumn{1}{c}{Donut}&\multicolumn{1}{c}{2M}&\multicolumn{1}{c}{4M}&\multicolumn{1}{c}{6M}&\multicolumn{1}{c}{Donut}&\multicolumn{1}{c}{2M}&\multicolumn{1}{c}{4M}&\multicolumn{1}{c}{6M}&\multicolumn{1}{c}{Donut}\\
\midrule
 \multicolumn{13}{l}{\emph{Panel A. 2 Up to the age of 21}} \\ Cum. numbers        &       18.00         &       58.00         &       84.17\sym{*}  &       116.4\sym{**} &      -33.00         &       12.00         &       61.17\sym{**} &       88.80\sym{***}&          51         &       46.00         &       23.00         &       27.60         \\
                    &     (83.86)         &     (59.26)         &     (42.64)         &     (41.92)         &     (31.62)         &     (26.86)         &     (29.36)         &     (26.83)         &     (57.31)         &     (40.44)         &     (32.70)         &     (37.64)         \\
 Cum. ratio          &      -0.232         &      -0.322         &      -0.391         &     -0.0453         &      -2.080         &      -1.082         &       0.220         &       1.068         &       1.511         &       0.386         &      -0.994         &      -1.124         \\
                    &     (1.376)         &     (0.780)         &     (0.658)         &     (0.682)         &     (1.419)         &     (0.954)         &     (0.957)         &     (0.906)         &     (1.656)         &     (1.078)         &     (1.145)         &     (1.309)         \\
 \midrule\multicolumn{13}{l}{\emph{Panel B. Up to the age of 26}} \\ Cum. numbers        &       131.0         &       166.3\sym{**} &       129.7\sym{**} &       150.2\sym{***}&       21.50         &       66.75         &       72.67\sym{**} &       105.8\sym{***}&       109.5\sym{***}&       99.50\sym{**} &       57.00         &       44.40         \\
                    &     (78.05)         &     (61.00)         &     (49.57)         &     (52.34)         &     (90.60)         &     (42.89)         &     (31.05)         &     (15.85)         &     (16.29)         &     (40.35)         &     (39.78)         &     (47.13)         \\
 Cum. ratio          &       1.635         &       0.740         &      -1.165         &      -1.373         &      -0.291         &       0.104         &      -1.117         &      -0.341         &       3.455\sym{**} &       1.329         &      -1.216         &      -2.357         \\
                    &     (2.223)         &     (1.083)         &     (1.187)         &     (1.386)         &     (4.369)         &     (1.924)         &     (1.415)         &     (1.479)         &     (1.161)         &     (1.215)         &     (1.652)         &     (1.752)         \\
 \midrule\multicolumn{13}{l}{\emph{Panel C. Up to the age of 31}} \\ Cum. numbers        &       202.5         &       208.2\sym{*}  &       210.0\sym{**} &       258.2\sym{**} &       103.0         &       154.5\sym{*}  &       144.8\sym{**} &       182.6\sym{***}&       99.50         &       53.75         &       65.17         &       75.60         \\
                    &     (168.4)         &     (99.11)         &     (89.55)         &     (93.80)         &     (126.9)         &     (76.75)         &     (54.35)         &     (39.15)         &     (73.06)         &     (45.36)         &     (50.88)         &     (61.51)         \\
 Cum. ratio          &       2.581         &       0.247         &      -1.446         &      -1.418         &       2.537         &       2.449         &     -0.0836         &       0.580         &       2.630         &      -1.840         &      -2.704         &      -3.277         \\
                    &     (3.973)         &     (2.024)         &     (1.655)         &     (1.936)         &     (6.347)         &     (2.849)         &     (2.150)         &     (2.112)         &     (1.879)         &     (2.199)         &     (1.833)         &     (2.180)         \\
 \midrule\multicolumn{13}{l}{\emph{Panel D. Up to the age of 34}} \\ Cum. numbers        &       340.5\sym{*}  &       341.0\sym{**} &       351.5\sym{***}&       402.0\sym{***}&       220.0\sym{**} &       255.2\sym{**} &       248.2\sym{***}&       276.8\sym{***}&       120.5         &       85.75         &       103.3\sym{*}  &       125.2\sym{*}  \\
                    &     (170.8)         &     (145.2)         &     (115.5)         &     (122.0)         &     (88.67)         &     (99.57)         &     (70.73)         &     (64.33)         &     (95.06)         &     (63.20)         &     (60.17)         &     (71.09)         \\
 Cum. ratio          &       4.693         &       1.402         &      -0.839         &      -1.144         &       6.608         &       5.148\sym{*}  &       1.908         &       1.856         &       2.906         &      -2.141         &      -3.377\sym{*}  &      -3.920         \\
                    &     (4.064)         &     (2.172)         &     (1.738)         &     (1.913)         &     (5.053)         &     (2.795)         &     (2.344)         &     (2.269)         &     (3.184)         &     (2.738)         &     (1.968)         &     (2.362)         \\
 
\bottomrule \end{tabular} } \begin{tablenotes} \item \scriptsize \emph{Notes:} Clustered standard errors in parentheses (MxY). All regressions contain Birthmonth FE. Ratios indicate cases per thousand; original number of births. \end{tablenotes} \end{threeparttable} \end{table} 

\end{landscape}
\begin{landscape}
 \begin{table}[H] \begin{threeparttable} \centering \caption{Cummulative effects for upt to different points of age - BOOTSTRAPPED} {\def\sym#1{\ifmmode^{#1}\else\(^{#1}\)\fi} \begin{tabular}{l*{13}{c}} \toprule & \multicolumn{12}{c}{Dependent variable: \textbf{Diseases of the musculoskeletal system and connective tissue}} \\ \cmidrule(lr){2-13}
            &\multicolumn{4}{c}{Average Causal Effects}         &\multicolumn{4}{c}{Women}                          &\multicolumn{4}{c}{Men}                            \\\cmidrule(lr){2-5}\cmidrule(lr){6-9}\cmidrule(lr){10-13}
            &\multicolumn{1}{c}{(1)}&\multicolumn{1}{c}{(2)}&\multicolumn{1}{c}{(3)}&\multicolumn{1}{c}{(4)}&\multicolumn{1}{c}{(5)}&\multicolumn{1}{c}{(6)}&\multicolumn{1}{c}{(7)}&\multicolumn{1}{c}{(8)}&\multicolumn{1}{c}{(9)}&\multicolumn{1}{c}{(10)}&\multicolumn{1}{c}{(11)}&\multicolumn{1}{c}{(12)}\\
            &\multicolumn{1}{c}{2M}&\multicolumn{1}{c}{4M}&\multicolumn{1}{c}{6M}&\multicolumn{1}{c}{Donut}&\multicolumn{1}{c}{2M}&\multicolumn{1}{c}{4M}&\multicolumn{1}{c}{6M}&\multicolumn{1}{c}{Donut}&\multicolumn{1}{c}{2M}&\multicolumn{1}{c}{4M}&\multicolumn{1}{c}{6M}&\multicolumn{1}{c}{Donut}\\
\midrule
 \multicolumn{13}{l}{\emph{Panel A. 2 Up to the age of 21}} \\ Cum. numbers        &       18.00         &       58.00         &       84.17         &       116.4\sym{**} &      -33.00         &       12.00         &       61.17\sym{*}  &       88.80\sym{***}&          51         &       46.00         &       23.00         &       27.60         \\
                    &     (76.74)         &     (74.96)         &     (51.22)         &     (55.12)         &     (27.44)         &     (32.28)         &     (36.89)         &     (33.40)         &     (54.24)         &     (55.21)         &     (47.29)         &     (50.08)         \\
 Cum. ratio          &      -0.232         &      -0.322         &      -0.391         &     -0.0453         &      -2.080\sym{*}  &      -1.082         &       0.220         &       1.068         &       1.511         &       0.386         &      -0.994         &      -1.124         \\
                    &     (1.200)         &     (1.001)         &     (0.944)         &     (0.930)         &     (1.231)         &     (1.334)         &     (1.290)         &     (1.265)         &     (1.519)         &     (1.435)         &     (1.761)         &     (1.700)         \\
 \midrule\multicolumn{13}{l}{\emph{Panel B. Up to the age of 26}} \\ Cum. numbers        &       131.0\sym{*}  &       166.3\sym{**} &       129.7\sym{**} &       150.2\sym{**} &       21.50         &       66.75         &       72.67\sym{*}  &       105.8\sym{***}&       109.5\sym{***}&       99.50\sym{*}  &       57.00         &       44.40         \\
                    &     (68.07)         &     (79.43)         &     (64.25)         &     (72.86)         &     (79.99)         &     (56.17)         &     (42.33)         &     (20.09)         &     (15.61)         &     (59.32)         &     (54.27)         &     (64.45)         \\
 Cum. ratio          &       1.635         &       0.740         &      -1.165         &      -1.373         &      -0.291         &       0.104         &      -1.117         &      -0.341         &       3.455\sym{***}&       1.329         &      -1.216         &      -2.357         \\
                    &     (2.100)         &     (1.453)         &     (1.712)         &     (1.775)         &     (3.978)         &     (2.834)         &     (2.112)         &     (1.838)         &     (1.099)         &     (1.500)         &     (2.263)         &     (2.304)         \\
 \midrule\multicolumn{13}{l}{\emph{Panel C. Up to the age of 31}} \\ Cum. numbers        &       202.5         &       208.2         &       210.0\sym{*}  &       258.2\sym{**} &       103.0         &       154.5         &       144.8\sym{**} &       182.6\sym{***}&       99.50         &       53.75         &       65.17         &       75.60         \\
                    &     (145.4)         &     (130.4)         &     (119.4)         &     (131.0)         &     (114.7)         &     (99.24)         &     (68.71)         &     (54.97)         &     (66.27)         &     (62.21)         &     (71.37)         &     (84.24)         \\
 Cum. ratio          &       2.581         &       0.247         &      -1.446         &      -1.418         &       2.537         &       2.449         &     -0.0836         &       0.580         &       2.630         &      -1.840         &      -2.704         &      -3.277         \\
                    &     (3.633)         &     (2.647)         &     (2.699)         &     (2.785)         &     (5.893)         &     (3.776)         &     (3.214)         &     (2.767)         &     (1.644)         &     (2.796)         &     (2.888)         &     (3.207)         \\
 \midrule\multicolumn{13}{l}{\emph{Panel D. Up to the age of 34}} \\ Cum. numbers        &       340.5\sym{**} &       341.0\sym{*}  &       351.5\sym{**} &       402.0\sym{**} &       220.0\sym{***}&       255.2\sym{*}  &       248.2\sym{***}&       276.8\sym{***}&       120.5         &       85.75         &       103.3         &       125.2         \\
                    &     (147.0)         &     (200.1)         &     (155.6)         &     (173.8)         &     (77.82)         &     (137.2)         &     (89.55)         &     (94.81)         &     (83.74)         &     (85.60)         &     (84.67)         &     (99.04)         \\
 Cum. ratio          &       4.693         &       1.402         &      -0.839         &      -1.144         &       6.608         &       5.148         &       1.908         &       1.856         &       2.906         &      -2.141         &      -3.377         &      -3.920         \\
                    &     (3.748)         &     (2.814)         &     (2.676)         &     (2.602)         &     (4.726)         &     (3.625)         &     (3.153)         &     (2.874)         &     (2.879)         &     (3.680)         &     (3.097)         &     (3.426)         \\
 
\bottomrule \end{tabular} } \begin{tablenotes} \item \scriptsize \emph{Notes:} \textbf{BOOTSTRAPPED} standard errors in parentheses (MxY), with 400 replications. All regressions contain Birthmonth FE. Ratios indicate cases per thousand; original number of births. \end{tablenotes} \end{threeparttable} \end{table} 

\end{landscape}
%---------------------------------
\newpage
FEBRUAR CASES:
 \begin{table}[H] \begin{threeparttable} \centering \caption{Dep. variable: \textbf{Diseases of the musculoskeletal system and connective tissue}} {\def\sym#1{\ifmmode^{#1}\else\(^{#1}\)\fi} \begin{tabular}{l*{13}{c}} \toprule year & \multicolumn{12}{c}{Month of birth} \\ \cmidrule(lr){2-13} 
            &          11&          12&           1&           2&           3&           4&           5&           6&           7&           8&           9&          10\\
1995        &         372&         407&         361&         377&         401&         346&         396&         351&         413&         338&         408&         344\\
1996        &         370&         309&         333&         376&         372&         364&         381&         326&         377&         341&         333&         361\\
1997        &         355&         374&         337&         351&         375&         378&         362&         331&         358&         387&         376&         344\\
1998        &         342&         351&         373&         354&         411&         354&         380&         380&         392&         368&         333&         413\\
1999        &         357&         367&         323&         393&         393&         343&         381&         361&         395&         352&         345&         349\\
2000        &         353&         337&         351&         340&         388&         396&         371&         349&         392&         358&         338&         347\\
2001        &         360&         360&         366&         385&         415&         414&         426&         410&         412&         412&         382&         379\\
2002        &         383&         368&         381&         398&         452&         414&         421&         409&         372&         409&         389&         386\\
2003        &         324&         370&         374&         397&         412&         399&         420&         401&         382&         356&         408&         401\\
2004        &         323&         313&         359&         361&         383&         362&         394&         364&         349&         357&         387&         334\\
2005        &         328&         346&         353&         334&         391&         333&         384&         368&         351&         343&         343&         350\\
2006        &         366&         345&         317&         397&         350&         389&         397&         346&         349&         395&         345&         365\\
2007        &         414&         396&         381&         362&         457&         396&         429&         393&         416&         392&         421&         347\\
2008        &         441&         386&         395&         442&         466&         388&         454&         427&         440&         445&         392&         412\\
2009        &         468&         462&         447&         514&         508&         522&         504&         504&         517&         470&         466&         468\\
2010        &         498&         511&         513&         580&         564&         542&         589&         514&         507&         514&         479&         485\\
2011        &         511&         541&         530&         569&         652&         573&         592&         571&         576&         536&         514&         583\\
2012        &         578&         526&         595&         634&         670&         591&         601&         613&         642&         570&         573&         609\\
2013        &         612&         635&         647&         684&         772&         680&         730&         649&         658&         609&         624&         579\\
2014        &         730&         695&         694&         710&         719&         719&         751&         738&         723&         769&         682&         699\\
 \bottomrule \end{tabular} } \begin{tablenotes} \item \scriptsize \emph{Notes:} Number of cases per year and MOB in treatment cohort. \end{tablenotes} \end{threeparttable} \end{table} 

 \begin{table}[H] \begin{threeparttable} \centering \caption{Dep. variable: \textbf{Diseases of the musculoskeletal system and connective tissue}} {\def\sym#1{\ifmmode^{#1}\else\(^{#1}\)\fi} \begin{tabular}{l*{13}{c}} \toprule year & \multicolumn{12}{c}{Month of birth} \\ \cmidrule(lr){2-13} 
            &          11&          12&           1&           2&           3&           4&           5&           6&           7&           8&           9&          10\\
1995        &          59&          63&          10&          56&          69&          -1&          36&          20&          65&         -50&          64&          29\\
1996        &          46&         -33&          -5&          21&          -7&         -29&          28&         -29&          27&         -46&          -8&          -1\\
1997        &          23&          41&          12&           4&          14&           6&           2&          31&         -33&          29&          11&         -69\\
1998        &          35&          41&          40&          11&          74&           0&          40&          28&          -8&         -23&         -26&          25\\
1999        &          -6&          29&         -31&          52&          26&          14&         -22&          11&          33&          20&         -19&          -7\\
2000        &          19&           0&         -12&           3&          13&          41&          -1&         -24&          27&         -17&          25&          15\\
2001        &           4&           8&           2&           0&          46&         102&          51&           7&          17&          12&          32&          21\\
2002        &          25&         -32&          35&          38&           1&          49&          30&           2&         -45&          14&         -14&          -2\\
2003        &          -8&          13&          45&           9&          37&          -1&          50&           3&         -33&           3&          54&           5\\
2004        &         -15&         -11&          24&           2&          22&          17&           5&         -14&         -22&          27&          72&         -12\\
2005        &         -10&          18&          18&          16&          36&           8&          11&          38&         -21&         -22&          12&           8\\
2006        &          29&          35&         -33&          39&         -13&          42&          50&         -34&          24&          86&          12&          25\\
2007        &          87&          23&           0&         -22&          93&          27&          21&         -12&           5&          -6&          59&         -73\\
2008        &          62&         -47&           2&          75&          44&           0&          18&         -19&           2&          46&         -20&          -5\\
2009        &          89&          15&           4&          54&          83&          62&          34&          69&          33&          10&          21&           4\\
2010        &          53&         -13&          93&          90&          44&          36&          98&          19&           4&          -6&         -26&         -42\\
2011        &         -13&          32&         -17&          38&          58&          75&          52&          46&           9&           1&           5&          39\\
2012        &          20&         -26&         -11&          65&          56&          25&         -53&          42&          57&         -40&         -28&          -5\\
2013        &          70&          41&          54&         104&         156&          51&          54&           9&          20&          23&         -32&         -58\\
2014        &          60&          91&          24&          34&         -23&          68&          21&          73&          52&          74&           2&          35\\
 \bottomrule \end{tabular} } \begin{tablenotes} \item \scriptsize \emph{Notes:} Difference of cases (control - treatment) per year and MOB in treatment cohort. \end{tablenotes} \end{threeparttable} \end{table} 

