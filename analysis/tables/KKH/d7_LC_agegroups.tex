 \begin{table}[H] \centering \begin{threeparttable} \caption{Life-course approach - Table format} {\def\sym#1{\ifmmode^{#1}\else\(^{#1}\)\fi} \begin{tabular}{l*{5}{c}} \toprule \multicolumn{5}{l}{Dep. variable: \textbf{Diseases of the eye and ear}} \\ & \multicolumn{4}{c}{Estimation window} \\ \cmidrule(lr){2-5}
            &\multicolumn{1}{c}{(1)}&\multicolumn{1}{c}{(2)}&\multicolumn{1}{c}{(3)}&\multicolumn{1}{c}{(4)}\\
            &\multicolumn{1}{c}{Age 17-21}&\multicolumn{1}{c}{Age 22-26}&\multicolumn{1}{c}{Age 27-31}&\multicolumn{1}{c}{Age 32-35}\\
\midrule
 \multicolumn{5}{l}{\emph{Panel A. Average causal effects}} \\ Abs. numbers        &      -2.600         &      -0.500         &      -3.900         &      -7.125         \\
                    &     (5.655)         &     (6.391)         &     (6.157)         &     (8.349)         \\
 Ratio fertility     &      -0.168         &     -0.0990         &      -0.172         &      -0.261         \\
                    &     (0.122)         &     (0.105)         &     (0.118)         &     (0.197)         \\
 Ratio population    &     -0.0509         &      -0.126         &      -0.193         \\
                    &    (0.0933)         &    (0.0930)         &     (0.151)         \\
 Cum. numbers        &       11.17         &       11.63         &      -4.167         &      -29.71         \\
                    &     (18.25)         &     (37.09)         &     (56.29)         &     (58.43)         \\
 Cum. ratio          &      -0.258         &      -0.755         &      -1.516\sym{*}  &      -2.522\sym{**} \\
                    &     (0.410)         &     (0.640)         &     (0.907)         &     (1.026)         \\
 \midrule\multicolumn{5}{l}{\emph{Panel B. Treatment effect heterogeneity - Women}} \\ Abs. numbers        &      -5.000         &       1.200         &      -2.500         &      -4.500         \\
                    &     (3.149)         &     (3.757)         &     (5.437)         &     (6.414)         \\
 Ratio fertility     &      -0.343\sym{**} &     -0.0591         &      -0.217         &      -0.323         \\
                    &     (0.134)         &     (0.152)         &     (0.207)         &     (0.301)         \\
 Ratio population    &     -0.0447         &      -0.155         &      -0.232         \\
                    &     (0.175)         &     (0.156)         &     (0.223)         \\
 Cum. numbers        &      -7.933         &      -8.833         &      -20.87         &      -36.79         \\
                    &     (14.68)         &     (22.29)         &     (33.90)         &     (45.07)         \\
 Cum. ratio          &      -0.881         &      -1.543\sym{*}  &      -2.586\sym{**} &      -3.823\sym{**} \\
                    &     (0.630)         &     (0.901)         &     (1.210)         &     (1.843)         \\
 \midrule\multicolumn{5}{l}{\emph{Panel C. Treatment effect heterogeneity - Men}} \\ Abs. numbers        &       2.400         &      -1.700         &      -1.400         &      -2.625         \\
                    &     (5.276)         &     (5.561)         &     (2.972)         &     (6.696)         \\
 Ratio fertility     &    -0.00121         &      -0.137         &      -0.133         &      -0.203         \\
                    &     (0.217)         &     (0.193)         &     (0.133)         &     (0.282)         \\
 Ratio population    &     -0.0569         &     -0.0983         &      -0.155         \\
                    &     (0.200)         &     (0.106)         &     (0.223)         \\
 Cum. numbers        &       19.10         &       20.47         &       16.70         &       7.083         \\
                    &     (18.18)         &     (35.13)         &     (44.49)         &     (45.96)         \\
 Cum. ratio          &       0.335         &    -0.00714         &      -0.507         &      -1.301         \\
                    &     (0.769)         &     (1.272)         &     (1.577)         &     (1.675)         \\
 
\bottomrule \end{tabular} } \begin{tablenotes} \item \scriptsize \emph{Notes:} Clustered standard errors in parentheses. All regression are run with CG2 (i.e. the cohort prior to the reform) and with month-of-birth FEs. Ratios indicate cases per thousand; either approximated population (with weights coming from the original fertility distribution) or original number of births. Raqtio population muss eins nach rechts gerückt werden \end{tablenotes} \end{threeparttable} \end{table} 
