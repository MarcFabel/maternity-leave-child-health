 \begin{table}[H] \centering \begin{threeparttable} \caption{Life-course approach - Table format} {\def\sym#1{\ifmmode^{#1}\else\(^{#1}\)\fi} \begin{tabular}{l*{5}{c}} \toprule \multicolumn{5}{l}{Dep. variable: \textbf{Diseases of the genitourinary system}} \\ & \multicolumn{4}{c}{Estimation window} \\ \cmidrule(lr){2-5}
            &\multicolumn{1}{c}{(1)}&\multicolumn{1}{c}{(2)}&\multicolumn{1}{c}{(3)}&\multicolumn{1}{c}{(4)}\\
            &\multicolumn{1}{c}{Age 17-21}&\multicolumn{1}{c}{Age 22-26}&\multicolumn{1}{c}{Age 27-31}&\multicolumn{1}{c}{Age 32-35}\\
\midrule
 \multicolumn{5}{l}{\emph{Panel A. Average causal effects}} \\ Abs. numbers        &       34.57\sym{*}  &       31.70         &       13.23         &       23.71         \\
                    &     (17.87)         &     (23.90)         &     (21.17)         &     (17.75)         \\
 Ratio fertility     &       0.241         &       0.180         &      -0.194         &     -0.0356         \\
                    &     (0.313)         &     (0.440)         &     (0.493)         &     (0.294)         \\
 Ratio population    &       0.323         &      -0.122         &     0.00110         \\
                    &     (0.288)         &     (0.369)         &     (0.235)         \\
 Cum. numbers        &       88.67         &       246.4\sym{*}  &       351.7\sym{*}  &       421.8\sym{*}  \\
                    &     (57.49)         &     (131.9)         &     (199.9)         &     (233.8)         \\
 Cum. ratio          &     -0.0571         &       0.704         &       0.615         &      -0.180         \\
                    &     (1.088)         &     (2.326)         &     (3.831)         &     (4.308)         \\
 \midrule\multicolumn{5}{l}{\emph{Panel B. Treatment effect heterogeneity - Women}} \\ Abs. numbers        &       20.03\sym{*}  &       22.83         &       4.300         &       26.21\sym{**} \\
                    &     (10.45)         &     (17.13)         &     (15.48)         &     (13.12)         \\
 Ratio fertility     &      0.0523         &       0.147         &      -0.570         &       0.313         \\
                    &     (0.518)         &     (0.573)         &     (0.756)         &     (0.523)         \\
 Ratio population    &       0.146         &      -0.385         &       0.276         \\
                    &     (0.510)         &     (0.548)         &     (0.412)         \\
 Cum. numbers        &       40.07         &         150\sym{*}  &       202.7         &       265.2\sym{*}  \\
                    &     (38.45)         &     (79.39)         &     (133.2)         &     (143.1)         \\
 Cum. ratio          &      -1.290         &      -0.935         &      -2.443         &      -3.307         \\
                    &     (1.834)         &     (3.450)         &     (5.751)         &     (6.361)         \\
 \midrule\multicolumn{5}{l}{\emph{Panel C. Treatment effect heterogeneity - Men}} \\ Abs. numbers        &       14.53         &       8.867         &       8.933         &      -2.500         \\
                    &     (12.00)         &     (11.00)         &     (9.654)         &     (12.49)         \\
 Ratio fertility     &       0.373         &       0.155         &       0.121         &      -0.406         \\
                    &     (0.430)         &     (0.446)         &     (0.432)         &     (0.449)         \\
 Ratio population    &       0.458\sym{*}  &       0.109         &      -0.303         \\
                    &     (0.266)         &     (0.339)         &     (0.351)         \\
 Cum. numbers        &       48.60         &       96.43         &       149.0         &       156.6         \\
                    &     (33.32)         &     (82.65)         &     (106.8)         &     (135.7)         \\
 Cum. ratio          &       0.979         &       1.843         &       2.868         &       1.951         \\
                    &     (1.145)         &     (3.012)         &     (4.067)         &     (5.034)         \\
 
\bottomrule \end{tabular} } \begin{tablenotes} \item \scriptsize \emph{Notes:} Clustered standard errors in parentheses. All regression are run with CG2 (i.e. the cohort prior to the reform) and with month-of-birth FEs. Ratios indicate cases per thousand; either approximated population (with weights coming from the original fertility distribution) or original number of births. Raqtio population muss eins nach rechts gerückt werden \end{tablenotes} \end{threeparttable} \end{table} 
