 \begin{table}[H] \centering \begin{threeparttable} \caption{Life-course approach - Table format} {\def\sym#1{\ifmmode^{#1}\else\(^{#1}\)\fi} \begin{tabular}{l*{5}{c}} \toprule \multicolumn{5}{l}{Dep. variable: \textbf{Certain infectious and parasitic diseases}} \\ & \multicolumn{4}{c}{Estimation window} \\ \cmidrule(lr){2-5}
            &\multicolumn{1}{c}{(1)}&\multicolumn{1}{c}{(2)}&\multicolumn{1}{c}{(3)}&\multicolumn{1}{c}{(4)}\\
            &\multicolumn{1}{c}{Age 17-21}&\multicolumn{1}{c}{Age 22-26}&\multicolumn{1}{c}{Age 27-31}&\multicolumn{1}{c}{Age 32-35}\\
\midrule
 \multicolumn{5}{l}{\emph{Panel A. Average causal effects}} \\ Abs. numbers        &       9.933         &       1.967         &       2.567         &       8.708         \\
                    &     (7.792)         &     (8.529)         &     (5.289)         &     (12.62)         \\
 Ratio fertility     &      0.0130         &      -0.162         &      -0.126         &     -0.0197         \\
                    &     (0.143)         &     (0.197)         &     (0.102)         &     (0.235)         \\
 Ratio population    &      -0.198         &     -0.0865         &    -0.00711         \\
                    &     (0.221)         &    (0.0761)         &     (0.180)         \\
 Cum. numbers        &       27.97         &       58.70         &       69.13         &       75.87         \\
                    &     (35.11)         &     (58.02)         &     (57.28)         &     (85.31)         \\
 Cum. ratio          &      -0.265         &      -0.676         &      -1.346         &      -2.079         \\
                    &     (0.663)         &     (1.164)         &     (1.294)         &     (1.567)         \\
 \midrule\multicolumn{5}{l}{\emph{Panel B. Treatment effect heterogeneity - Women}} \\ Abs. numbers        &       8.367         &       1.233         &       1.700         &       4.458         \\
                    &     (6.177)         &     (7.051)         &     (5.096)         &     (9.444)         \\
 Ratio fertility     &       0.107         &      -0.193         &      -0.132         &     -0.0378         \\
                    &     (0.230)         &     (0.304)         &     (0.161)         &     (0.373)         \\
 Ratio population    &      -0.343         &     -0.0876         &     -0.0172         \\
                    &     (0.324)         &     (0.121)         &     (0.275)         \\
 Cum. numbers        &       23.87         &       48.03         &       53.30         &       54.12         \\
                    &     (27.34)         &     (46.21)         &     (56.03)         &     (77.80)         \\
 Cum. ratio          &     -0.0738         &      -0.330         &      -1.140         &      -2.109         \\
                    &     (0.975)         &     (1.591)         &     (1.840)         &     (2.333)         \\
 \midrule\multicolumn{5}{l}{\emph{Panel C. Treatment effect heterogeneity - Men}} \\ Abs. numbers        &       1.567         &       0.733         &       0.867         &       4.250         \\
                    &     (3.972)         &     (6.625)         &     (6.356)         &     (5.521)         \\
 Ratio fertility     &     -0.0824         &      -0.135         &      -0.120         &    -0.00380         \\
                    &     (0.181)         &     (0.281)         &     (0.278)         &     (0.202)         \\
 Ratio population    &     -0.0546         &     -0.0860         &     0.00165         \\
                    &     (0.323)         &     (0.216)         &     (0.161)         \\
 Cum. numbers        &       4.100         &       10.67         &       15.83         &       21.75         \\
                    &     (21.22)         &     (26.82)         &     (23.17)         &     (24.83)         \\
 Cum. ratio          &      -0.473         &      -1.051         &      -1.597         &      -2.111         \\
                    &     (0.947)         &     (1.426)         &     (1.607)         &     (1.746)         \\
 
\bottomrule \end{tabular} } \begin{tablenotes} \item \scriptsize \emph{Notes:} Clustered standard errors in parentheses. All regression are run with CG2 (i.e. the cohort prior to the reform) and with month-of-birth FEs. Ratios indicate cases per thousand; either approximated population (with weights coming from the original fertility distribution) or original number of births. Raqtio population muss eins nach rechts gerückt werden \end{tablenotes} \end{threeparttable} \end{table} 
