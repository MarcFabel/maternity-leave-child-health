%---------------------------------
% INPUT FOR VARIABLE: d13
%---------------------------------
\subsection{d13}
% RD overview
\begin{landscape}
\begin{figure}[H]
	\centering
	\begin{minipage}{.95\linewidth}
	\includegraphics[width=\linewidth]{rd_d13_overview_panel1}
	{\scriptsize \emph{Notes:} The figures show monthly RD plots with averages obtained from a bin width of one month. The solid vertical line divides pre- and post-reform regime. The averages are taken over the period of at most 1995-2014. \par}
\end{minipage}
\end{figure}
\end{landscape}
\begin{landscape}
\begin{figure}[H]
	\centering
\begin{minipage}{.95\linewidth}
	\includegraphics[width=\linewidth]{rd_d13_overview_panel2}
	{\scriptsize \emph{Notes:} The figures show monthly RD plots with a moving average window width of 3 months. The solid vertical line divides pre- and post-reform regime. The averages are taken over the period of at most 1995-2014. \par}
\end{minipage}
\end{figure}
\end{landscape}
%---------------------------------
% TABELLEN
 \begin{table}[H] \begin{threeparttable} \centering \caption{Dep. variable: \textbf{Diseases of the genitourinary system}} {\def\sym#1{\ifmmode^{#1}\else\(^{#1}\)\fi} \begin{tabular}{l*{8}{c}} \toprule & \multicolumn{7}{c}{Estimation window} \\ \cmidrule(lr){2-8}
            &\multicolumn{1}{c}{(1)}&\multicolumn{1}{c}{(2)}&\multicolumn{1}{c}{(3)}&\multicolumn{1}{c}{(4)}&\multicolumn{1}{c}{(5)}&\multicolumn{1}{c}{(6)}&\multicolumn{1}{c}{(7)}\\
            &\multicolumn{1}{c}{1M}&\multicolumn{1}{c}{2M}&\multicolumn{1}{c}{3M}&\multicolumn{1}{c}{4M}&\multicolumn{1}{c}{5M}&\multicolumn{1}{c}{6M}&\multicolumn{1}{c}{Donut}\\
\midrule
 \multicolumn{8}{l}{\emph{Panel A. Average causal effects}} \\ Abs. numbers        &       18.75\sym{***}&       9.875\sym{**} &       18.32\sym{*}  &       25.48\sym{**} &       22.78\sym{***}&       22.88\sym{***}&       23.71\sym{***}\\
                    &  (1.18e-13)         &     (3.675)         &     (8.342)         &     (9.178)         &     (7.551)         &     (6.268)         &     (6.263)         \\
 Ratio fertility     &       0.319\sym{***}&       0.161\sym{*}  &       0.191\sym{*}  &       0.251\sym{***}&       0.200\sym{***}&       0.136\sym{**} &      0.0990         \\
                    &  (7.39e-16)         &    (0.0784)         &    (0.0882)         &    (0.0734)         &    (0.0639)         &    (0.0624)         &    (0.0645)         \\
 Ratio population    &       0.364\sym{***}&       0.177         &       0.119         &       0.195\sym{*}  &       0.129         &       0.101         &      0.0485         \\
                    &  (1.16e-15)         &     (0.131)         &     (0.117)         &    (0.0944)         &    (0.0817)         &    (0.0692)         &    (0.0576)         \\
 Ratio fert(03-14)   &       0.365\sym{***}&      0.0730         &     -0.0276         &       0.103         &      0.0116         &     -0.0473         &      -0.130         \\
                    &  (1.58e-15)         &     (0.226)         &     (0.195)         &     (0.157)         &     (0.132)         &     (0.113)         &    (0.0886)         \\
 Cum. numbers        &       187.1\sym{***}&       108.6\sym{**} &       232.5\sym{**} &       305.2\sym{***}&       280.6\sym{***}&       256.7\sym{***}&       270.6\sym{***}\\
                    &  (6.65e-13)         &     (36.91)         &     (95.59)         &     (98.85)         &     (81.66)         &     (68.72)         &     (75.62)         \\
 Cum. ratio          &       3.380\sym{***}&       0.825         &       1.880         &       3.028\sym{**} &       2.296\sym{**} &       0.579         &      0.0193         \\
                    &  (6.82e-14)         &     (1.107)         &     (1.499)         &     (1.253)         &     (1.053)         &     (1.217)         &     (1.344)         \\
 \midrule\multicolumn{8}{l}{\emph{Panel B. Treatment effect heterogeneity - Women}} \\ Abs. numbers        &       12.45\sym{***}&       11.68\sym{**} &       16.77\sym{**} &       17.99\sym{***}&       13.33\sym{**} &       14.91\sym{***}&       15.40\sym{***}\\
                    &  (5.01e-14)         &     (3.383)         &     (5.753)         &     (5.689)         &     (5.035)         &     (4.247)         &     (3.843)         \\
 Ratio fertility     &       0.361\sym{***}&       0.253         &       0.254         &       0.264         &     0.00311         &      -0.133         &      -0.232         \\
                    &  (2.44e-15)         &     (0.270)         &     (0.218)         &     (0.166)         &     (0.180)         &     (0.169)         &     (0.164)         \\
 Ratio population    &       0.443\sym{***}&       0.469\sym{**} &       0.363\sym{**} &       0.314\sym{**} &       0.163         &      0.0957         &      0.0262         \\
                    &  (3.53e-15)         &     (0.146)         &     (0.127)         &     (0.107)         &     (0.111)         &    (0.0981)         &    (0.0875)         \\
 Ratio fert(03-14)   &       0.427\sym{***}&       0.454\sym{**} &       0.351\sym{**} &       0.304\sym{**} &       0.163         &       0.102         &      0.0370         \\
                    &  (1.73e-15)         &     (0.141)         &     (0.122)         &     (0.103)         &     (0.106)         &    (0.0932)         &    (0.0837)         \\
 Cum. numbers        &       121.6\sym{***}&       93.77\sym{***}&       171.0\sym{**} &       196.1\sym{***}&       143.9\sym{**} &       151.3\sym{***}&       157.2\sym{***}\\
                    &  (1.82e-12)         &     (25.80)         &     (69.54)         &     (66.17)         &     (58.50)         &     (48.70)         &     (47.79)         \\
 Cum. ratio          &       3.581\sym{***}&       1.474         &       2.612         &       3.223\sym{*}  &       0.386         &      -1.412         &      -2.411         \\
                    &  (8.50e-14)         &     (2.427)         &     (2.338)         &     (1.801)         &     (1.949)         &     (1.941)         &     (2.021)         \\
 \midrule\multicolumn{8}{l}{\emph{Panel C. Treatment effect heterogeneity - Men}} \\ Abs. numbers        &       6.300\sym{***}&      -1.800         &       1.550         &       7.488         &       9.450\sym{**} &       7.975\sym{**} &       8.310\sym{**} \\
                    &  (4.77e-14)         &     (3.697)         &     (3.612)         &     (4.404)         &     (3.761)         &     (3.197)         &     (3.759)         \\
 Ratio fertility     &       0.209\sym{***}&     -0.0523         &    -0.00791         &       0.134         &       0.189\sym{**} &       0.119         &       0.101         \\
                    &  (4.62e-16)         &     (0.106)         &    (0.0880)         &    (0.0936)         &    (0.0793)         &    (0.0736)         &    (0.0877)         \\
 Ratio population    &       0.232\sym{***}&      -0.148         &      -0.178         &      0.0500         &      0.0702         &      0.0571         &      0.0222         \\
                    &  (6.25e-16)         &     (0.166)         &     (0.146)         &     (0.150)         &     (0.120)         &    (0.0997)         &     (0.109)         \\
 Ratio fert(03-14)   &       0.297\sym{***}&      -0.188         &      -0.229         &      0.0623         &      0.0759         &      0.0511         &     0.00183         \\
                    &  (3.75e-16)         &     (0.213)         &     (0.187)         &     (0.192)         &     (0.153)         &     (0.127)         &     (0.138)         \\
 Cum. numbers        &       65.45\sym{***}&       14.85         &       61.47         &       109.0\sym{**} &       136.7\sym{***}&       105.5\sym{***}&       113.5\sym{**} \\
                    &  (4.31e-13)         &     (48.82)         &     (41.21)         &     (42.71)         &     (37.92)         &     (34.76)         &     (41.64)         \\
 Cum. ratio          &       2.082\sym{***}&       0.333         &       1.224         &       2.322\sym{**} &       3.108\sym{***}&       1.927\sym{**} &       1.896\sym{*}  \\
                    &  (1.39e-14)         &     (1.231)         &     (0.945)         &     (0.873)         &     (0.795)         &     (0.865)         &     (1.030)         \\
 
\bottomrule \end{tabular} } \begin{tablenotes} \item \scriptsize \emph{Notes:} Clustered standard errors in parentheses. All regression are run with CG2 (i.e. the cohort prior to the reform) and with month-of-birth FEs. Ratios indicate cases per thousand; either approximated population (with weights coming from the original fertility distribution) or original number of births. \end{tablenotes} \end{threeparttable} \end{table} 

 \begin{table}[H] \begin{threeparttable} \centering \caption{Robustness with respect to the choice of \texttt{control group}} {\def\sym#1{\ifmmode^{#1}\else\(^{#1}\)\fi} \begin{tabular}{l*{10}{c}} \toprule & \multicolumn{9}{c}{Dependent variable: \textbf{Diseases of the genitourinary system}} \\ \cmidrule(lr){2-10}
            &\multicolumn{3}{c}{Average Causal Effects}&\multicolumn{3}{c}{Women}             &\multicolumn{3}{c}{Men}               \\\cmidrule(lr){2-4}\cmidrule(lr){5-7}\cmidrule(lr){8-10}
            &\multicolumn{1}{c}{(1)}&\multicolumn{1}{c}{(2)}&\multicolumn{1}{c}{(3)}&\multicolumn{1}{c}{(4)}&\multicolumn{1}{c}{(5)}&\multicolumn{1}{c}{(6)}&\multicolumn{1}{c}{(7)}&\multicolumn{1}{c}{(8)}&\multicolumn{1}{c}{(9)}\\
            &\multicolumn{1}{c}{C2}&\multicolumn{1}{c}{C1+C2}&\multicolumn{1}{c}{C1-C3}&\multicolumn{1}{c}{C2}&\multicolumn{1}{c}{C1+C2}&\multicolumn{1}{c}{C1-C3}&\multicolumn{1}{c}{C2}&\multicolumn{1}{c}{C1+C2}&\multicolumn{1}{c}{C1-C3}\\
\midrule
 \multicolumn{10}{l}{\emph{Panel A. 2 Month bandwidth}} \\ Abs. numbers        &       12.61         &       2.333         &       2.630         &       15.50\sym{**} &       4.083         &       4.704         &      -2.889         &      -1.750         &      -2.074         \\
                    &     (7.649)         &     (8.501)         &     (7.271)         &     (4.974)         &     (6.184)         &     (4.728)         &     (4.633)         &     (4.660)         &     (4.214)         \\
 Ratio population    &      -0.156         &     -0.0841         &     0.00438         &      -0.129         &      -0.145         &      -0.178         &      -0.235\sym{***}&     -0.0707         &      0.0257         \\
                    &     (0.151)         &     (0.264)         &     (0.280)         &     (0.418)         &     (0.493)         &     (0.585)         &    (0.0517)         &     (0.174)         &     (0.165)         \\
 Ratio population    &      0.0543         &      0.0828         &       0.112         &       0.236         &       0.170         &       0.189         &      -0.148         &     -0.0262         &      0.0193         \\
                    &     (0.172)         &     (0.211)         &     (0.221)         &     (0.214)         &     (0.249)         &     (0.277)         &     (0.166)         &     (0.206)         &     (0.193)         \\
 \midrule\multicolumn{10}{l}{\emph{Panel B. 4 Month bandwidth}} \\ Abs. numbers        &       17.72\sym{*}  &       13.61         &       9.694         &       13.08\sym{**} &       6.069         &       5.444         &       4.639         &       7.542         &       4.250         \\
                    &     (8.907)         &     (8.552)         &     (7.314)         &     (4.511)         &     (4.465)         &     (3.464)         &     (5.698)         &     (5.547)         &     (4.999)         \\
 Ratio fertility     &       0.202         &       0.171         &       0.195         &       0.264         &       0.187         &       0.231         &       0.132         &       0.152         &       0.157         \\
                    &     (0.130)         &     (0.174)         &     (0.290)         &     (0.166)         &     (0.274)         &     (0.479)         &     (0.143)         &     (0.150)         &     (0.163)         \\
 Ratio population    &      0.0782         &      0.0862         &      0.0868         &      0.0974         &      0.0564         &      0.0930         &      0.0500         &       0.113         &      0.0770         \\
                    &     (0.119)         &     (0.134)         &     (0.149)         &     (0.126)         &     (0.163)         &     (0.171)         &     (0.150)         &     (0.150)         &     (0.159)         \\
 \midrule\multicolumn{10}{l}{\emph{Panel C. 6 Month bandwidth}} \\ Abs. numbers        &       17.04\sym{**} &       17.73\sym{***}&       10.51\sym{*}  &       11.22\sym{***}&       9.000\sym{**} &       5.809\sym{*}  &       5.815         &       8.731\sym{**} &       4.704         \\
                    &     (6.472)         &     (6.389)         &     (5.697)         &     (3.829)         &     (3.801)         &     (3.196)         &     (3.842)         &     (3.756)         &     (3.471)         \\
 Ratio population    &      -0.115         &     -0.0195         &     -0.0441         &     -0.0950         &     -0.0996         &      -0.152         &     -0.0739         &      0.0785         &      0.0304         \\
                    &     (0.166)         &     (0.172)         &     (0.171)         &     (0.303)         &     (0.279)         &     (0.288)         &     (0.218)         &     (0.205)         &     (0.204)         \\
 Ratio population    &    -0.00563         &      0.0719         &      0.0362         &      -0.102         &     -0.0120         &     -0.0206         &      0.0571         &       0.129         &      0.0659         \\
                    &    (0.0834)         &    (0.0988)         &     (0.118)         &     (0.106)         &     (0.133)         &     (0.146)         &    (0.0997)         &     (0.101)         &     (0.116)         \\
 \midrule\multicolumn{10}{l}{\emph{Panel D. Donut specification}} \\ Abs. numbers        &       15.31\sym{**} &       17.52\sym{**} &       9.504         &       10.09\sym{***}&       9.078\sym{**} &       5.230         &       5.222         &       8.444\sym{*}  &       4.274         \\
                    &     (5.602)         &     (6.755)         &     (6.038)         &     (2.958)         &     (3.857)         &     (3.314)         &     (4.231)         &     (4.324)         &     (3.939)         \\
 Ratio fertility     &     -0.0599         &      -0.120         &     -0.0522         &      -0.232         &      -0.322         &      -0.193         &      0.0606         &      0.0339         &      0.0386         \\
                    &     (0.102)         &     (0.117)         &     (0.240)         &     (0.164)         &     (0.214)         &     (0.407)         &     (0.133)         &     (0.121)         &     (0.139)         \\
 Ratio population    &     -0.0623         &    0.000131         &     -0.0247         &      -0.178\sym{*}  &     -0.0815         &     -0.0810         &      0.0222         &      0.0632         &     0.00714         \\
                    &    (0.0680)         &    (0.0958)         &     (0.116)         &    (0.0887)         &     (0.140)         &     (0.149)         &     (0.109)         &     (0.106)         &     (0.122)         \\
 
\bottomrule \end{tabular} } \begin{tablenotes} \item \scriptsize \emph{Notes:} Clustered standard errors in parentheses. All regressions contain Birthmonth FE. Ratios indicate cases per thousand; either approximated population (with weights coming from the original fertility distribution) or original number of births. \end{tablenotes} \end{threeparttable} \end{table} 

%---------------------------------
% Life-course figure (Panel1)
\begin{landscape}
\begin{figure}[H]
\centering
\begin{minipage}{.9\linewidth}
\includegraphics[width=\linewidth]{lc_d13_overview_panel1}
{\scriptsize \emph{Notes:} The figures depict DDRD estimates and 90\% confidence intervals over the life-course. The years are harmonized such that the cohorts are in the same age when they are compared. All regressions are carried out with month-of-birth FE and make use of clustered standard errors. Furthermore, we used a bandwidth of half a year and only the control cohort that was born one year prior to the reform. Ratios indicate cases per thousand; using in the denominator the approximated population (with weights coming from the original fertility distribution) or original number of births. \par}
\end{minipage}
\end{figure}
\end{landscape}
%---------------------------------
% Life-course figure (Panel2)
\begin{landscape}
\begin{figure}[H]
\centering
\begin{minipage}{.9\linewidth}
\includegraphics[width=\linewidth]{lc_d13_overview_panel2}
{\scriptsize \emph{Notes:} The figures depict DDRD estimates and 90\% confidence intervals over the life-course. The years are harmonized such that the cohorts are in the same age when they are compared. All regressions are carried out with month-of-birth FE and make use of clustered standard errors. Furthermore, we used a bandwidth of half a year. Ratios indicate cases per thousand; using in the denominator the approximated population (with weights coming from the original fertility distribution) or original number of births. \par}
\end{minipage}
\end{figure}
\end{landscape}
%---------------------------------
% Life-course (panel 3 - 6)
\begin{figure}[H]%\vspace*{-2cm}
	\centering
	\includegraphics[width=.9\linewidth]{lc_d13_overview_panel3}
	\includegraphics[width=.9\linewidth]{lc_d13_overview_panel4}
\end{figure}
\begin{figure}[H]
	\centering	
	\includegraphics[width=.97\linewidth]{lc_d13_overview_panel5}
	\includegraphics[width=.97\linewidth]{lc_d13_overview_panel6}
\end{figure}
% Life-course TABLE Format
 \begin{table}[H] \centering \begin{threeparttable} \caption{Life-course approach - Table format} {\def\sym#1{\ifmmode^{#1}\else\(^{#1}\)\fi} \begin{tabular}{l*{5}{c}} \toprule \multicolumn{5}{l}{Dep. variable: \textbf{Diseases of the genitourinary system}} \\ & \multicolumn{4}{c}{Estimation window} \\ \cmidrule(lr){2-5}
            &\multicolumn{1}{c}{(1)}&\multicolumn{1}{c}{(2)}&\multicolumn{1}{c}{(3)}&\multicolumn{1}{c}{(4)}\\
            &\multicolumn{1}{c}{Age 17-21}&\multicolumn{1}{c}{Age 22-26}&\multicolumn{1}{c}{Age 27-31}&\multicolumn{1}{c}{Age 32-35}\\
\midrule
 \multicolumn{5}{l}{\emph{Panel A. Average causal effects}} \\ Abs. numbers        &       34.57\sym{*}  &       31.70         &       13.23         &       23.71         \\
                    &     (17.87)         &     (23.90)         &     (21.17)         &     (17.75)         \\
 Ratio fertility     &       0.241         &       0.180         &      -0.194         &     -0.0356         \\
                    &     (0.313)         &     (0.440)         &     (0.493)         &     (0.294)         \\
 Ratio population    &       0.323         &      -0.122         &     0.00110         \\
                    &     (0.288)         &     (0.369)         &     (0.235)         \\
 Cum. numbers        &       88.67         &       246.4\sym{*}  &       351.7\sym{*}  &       421.8\sym{*}  \\
                    &     (57.49)         &     (131.9)         &     (199.9)         &     (233.8)         \\
 Cum. ratio          &     -0.0571         &       0.704         &       0.615         &      -0.180         \\
                    &     (1.088)         &     (2.326)         &     (3.831)         &     (4.308)         \\
 \midrule\multicolumn{5}{l}{\emph{Panel B. Treatment effect heterogeneity - Women}} \\ Abs. numbers        &       20.03\sym{*}  &       22.83         &       4.300         &       26.21\sym{**} \\
                    &     (10.45)         &     (17.13)         &     (15.48)         &     (13.12)         \\
 Ratio fertility     &      0.0523         &       0.147         &      -0.570         &       0.313         \\
                    &     (0.518)         &     (0.573)         &     (0.756)         &     (0.523)         \\
 Ratio population    &       0.146         &      -0.385         &       0.276         \\
                    &     (0.510)         &     (0.548)         &     (0.412)         \\
 Cum. numbers        &       40.07         &         150\sym{*}  &       202.7         &       265.2\sym{*}  \\
                    &     (38.45)         &     (79.39)         &     (133.2)         &     (143.1)         \\
 Cum. ratio          &      -1.290         &      -0.935         &      -2.443         &      -3.307         \\
                    &     (1.834)         &     (3.450)         &     (5.751)         &     (6.361)         \\
 \midrule\multicolumn{5}{l}{\emph{Panel C. Treatment effect heterogeneity - Men}} \\ Abs. numbers        &       14.53         &       8.867         &       8.933         &      -2.500         \\
                    &     (12.00)         &     (11.00)         &     (9.654)         &     (12.49)         \\
 Ratio fertility     &       0.373         &       0.155         &       0.121         &      -0.406         \\
                    &     (0.430)         &     (0.446)         &     (0.432)         &     (0.449)         \\
 Ratio population    &       0.458\sym{*}  &       0.109         &      -0.303         \\
                    &     (0.266)         &     (0.339)         &     (0.351)         \\
 Cum. numbers        &       48.60         &       96.43         &       149.0         &       156.6         \\
                    &     (33.32)         &     (82.65)         &     (106.8)         &     (135.7)         \\
 Cum. ratio          &       0.979         &       1.843         &       2.868         &       1.951         \\
                    &     (1.145)         &     (3.012)         &     (4.067)         &     (5.034)         \\
 
\bottomrule \end{tabular} } \begin{tablenotes} \item \scriptsize \emph{Notes:} Clustered standard errors in parentheses. All regression are run with CG2 (i.e. the cohort prior to the reform) and with month-of-birth FEs. Ratios indicate cases per thousand; either approximated population (with weights coming from the original fertility distribution) or original number of births. Raqtio population muss eins nach rechts gerückt werden \end{tablenotes} \end{threeparttable} \end{table} 

%---------------------------------
% PLACEBO EXERCISES
\newpage
\begin{landscape}
\begin{figure}[H]
	\centering
    \begin{minipage}{.9\linewidth}
	\includegraphics[width=\linewidth]{placebo_graph_d13.pdf}
    {\scriptsize \emph{Notes:} The figures depict DDRD estimates and 95\% confidence intervals when the treatment cohort is shifted over time. The date on the abscissa indicates the starting date of the treated.  All regressions are carried out with month-of-birth FE and make use of clustered standard errors. Furthermore, we used a bandwidth of half a year. Ratios indicate cases per thousand; using in the denominator the approximated population (with weights coming from the original fertility distribution) or original number of births. \par}
    \end{minipage}
\end{figure}
\end{landscape}
 \begin{table}[H] \centering \begin{threeparttable} \caption{Placebo 1 (CONTROL1 ist TREAT) } {\def\sym#1{\ifmmode^{#1}\else\(^{#1}\)\fi} \begin{tabular}{l*{4}{c}} \toprule \multicolumn{4}{l}{Dep. variable: \textbf{Diseases of the genitourinary system}} \\ & \multicolumn{3}{c}{Choice of control group} \\ \cmidrule(lr){2-4}
            &\multicolumn{1}{c}{(1)}&\multicolumn{1}{c}{(2)}&\multicolumn{1}{c}{(3)}\\
            &\multicolumn{1}{c}{C2}&\multicolumn{1}{c}{C3}&\multicolumn{1}{c}{C2+C3}\\
\midrule
 \multicolumn{4}{l}{\emph{Panel A. Average causal effects}} \\ Abs. numbers        &       8.392\sym{**} &      -10.78\sym{***}&      -1.196         \\
                    &     (3.018)         &     (3.721)         &     (6.425)         \\
 Ratio fertility     &      0.0145         &       0.128         &      0.0712         \\
                    &    (0.0840)         &    (0.0956)         &     (0.242)         \\
 Ratio population    &      -0.155\sym{*}  &      -0.184\sym{*}  &      -0.170         \\
                    &    (0.0766)         &     (0.107)         &     (0.119)         \\
 Cum. numbers        &       156.6\sym{***}&      -28.33         &       64.12         \\
                    &     (34.76)         &     (37.53)         &     (101.7)         \\
 Cum. ratio          &       1.544         &       2.627\sym{**} &       2.085         \\
                    &     (1.141)         &     (1.160)         &     (3.310)         \\
 \midrule\multicolumn{4}{l}{\emph{Panel B. Treatment effect heterogeneity - Women}} \\ Abs. numbers        &       7.383\sym{**} &      -4.658         &       1.362         \\
                    &     (2.783)         &     (2.805)         &     (5.343)         \\
 Ratio fertility     &      0.0674         &       0.316\sym{**} &       0.192         \\
                    &     (0.133)         &     (0.123)         &     (0.391)         \\
 Ratio population    &      -0.180         &      -0.116         &      -0.148         \\
                    &     (0.127)         &     (0.131)         &     (0.150)         \\
 Cum. numbers        &       115.9\sym{***}&       3.358         &       59.65         \\
                    &     (32.24)         &     (35.19)         &     (94.55)         \\
 Cum. ratio          &       2.238         &       4.733\sym{**} &       3.486         \\
                    &     (1.779)         &     (1.822)         &     (5.831)         \\
 \midrule\multicolumn{4}{l}{\emph{Panel C. Treatment effect heterogeneity - Men}} \\ Abs. numbers        &       1.008         &      -6.125\sym{***}&      -2.558         \\
                    &     (1.138)         &     (2.100)         &     (1.959)         \\
 Ratio fertility     &     -0.0336         &     -0.0461         &     -0.0399         \\
                    &    (0.0557)         &    (0.0899)         &     (0.117)         \\
 Ratio population    &      -0.144\sym{***}&      -0.261\sym{**} &      -0.202\sym{*}  \\
                    &    (0.0394)         &    (0.0966)         &     (0.104)         \\
 Cum. numbers        &       40.64\sym{***}&      -31.69         &       4.475         \\
                    &     (13.61)         &     (20.06)         &     (19.36)         \\
 Cum. ratio          &       0.915         &       0.685         &       0.800         \\
                    &     (0.773)         &     (0.873)         &     (1.165)         \\
 
\bottomrule \end{tabular} } \begin{tablenotes} \item \scriptsize \emph{Notes:} Clustered standard errors in parentheses. All regression are run with month-of-birth FEs and control cohort 2 is assigned with the treatment status. All regressions are carried out with a window width of half a year. \end{tablenotes} \end{threeparttable} \end{table} 

 \begin{table}[H] \centering \begin{threeparttable} \caption{Placebo 2 (CONTROL2 ist TREAT) } {\def\sym#1{\ifmmode^{#1}\else\(^{#1}\)\fi} \begin{tabular}{l*{4}{c}} \toprule \multicolumn{4}{l}{Dep. variable: \textbf{Diseases of the genitourinary system}} \\ & \multicolumn{3}{c}{Choice of control group} \\ \cmidrule(lr){2-4}
            &\multicolumn{1}{c}{(1)}&\multicolumn{1}{c}{(2)}&\multicolumn{1}{c}{(3)}\\
            &\multicolumn{1}{c}{C1}&\multicolumn{1}{c}{C3}&\multicolumn{1}{c}{C1+C3}\\
\midrule
 \multicolumn{4}{l}{\emph{Panel A. Average causal effects}} \\ Abs. numbers        &      -8.392\sym{**} &      -19.17\sym{***}&      -13.78         \\
                    &     (3.018)         &     (4.482)         &     (10.04)         \\
 Ratio fertility     &     -0.0145         &       0.114\sym{*}  &      0.0495         \\
                    &    (0.0840)         &    (0.0576)         &     (0.299)         \\
 Ratio population    &       0.155\sym{*}  &     -0.0294         &      0.0628         \\
                    &    (0.0766)         &    (0.0518)         &     (0.134)         \\
 Cum. numbers        &      -156.6\sym{***}&      -184.9\sym{***}&      -170.8         \\
                    &     (34.76)         &     (44.08)         &     (158.3)         \\
 Cum. ratio          &      -1.544         &       1.083         &      -0.230         \\
                    &     (1.141)         &     (0.778)         &     (4.260)         \\
 \midrule\multicolumn{4}{l}{\emph{Panel B. Treatment effect heterogeneity - Women}} \\ Abs. numbers        &      -7.383\sym{**} &      -12.04\sym{***}&      -9.713         \\
                    &     (2.783)         &     (4.006)         &     (9.023)         \\
 Ratio fertility     &     -0.0674         &       0.249\sym{***}&      0.0906         \\
                    &     (0.133)         &    (0.0837)         &     (0.523)         \\
 Ratio population    &       0.180         &      0.0639         &       0.122         \\
                    &     (0.127)         &    (0.0576)         &     (0.180)         \\
 Cum. numbers        &      -115.9\sym{***}&      -112.6\sym{**} &      -114.3         \\
                    &     (32.24)         &     (45.65)         &     (152.5)         \\
 Cum. ratio          &      -2.238         &       2.495\sym{*}  &       0.129         \\
                    &     (1.779)         &     (1.250)         &     (7.955)         \\
 \midrule\multicolumn{4}{l}{\emph{Panel C. Treatment effect heterogeneity - Men}} \\ Abs. numbers        &      -1.008         &      -7.133\sym{***}&      -4.071\sym{**} \\
                    &     (1.138)         &     (2.069)         &     (1.985)         \\
 Ratio fertility     &      0.0336         &     -0.0125         &      0.0106         \\
                    &    (0.0557)         &    (0.0841)         &     (0.113)         \\
 Ratio population    &       0.144\sym{***}&      -0.117         &      0.0133         \\
                    &    (0.0394)         &    (0.0842)         &     (0.110)         \\
 Cum. numbers        &      -40.64\sym{***}&      -72.33\sym{***}&      -56.49\sym{***}\\
                    &     (13.61)         &     (18.86)         &     (17.64)         \\
 Cum. ratio          &      -0.915         &      -0.230         &      -0.573         \\
                    &     (0.773)         &     (0.901)         &     (1.114)         \\
 
\bottomrule \end{tabular} } \begin{tablenotes} \item \scriptsize \emph{Notes:} Clustered standard errors in parentheses. All regression are run with month-of-birth FEs and control cohort 2 is assigned with the treatment status. All regressions are carried out with a window width of half a year. \end{tablenotes} \end{threeparttable} \end{table} 

 \begin{table}[H] \centering \begin{threeparttable} \caption{Placebo 3 (CONTROL3 ist TREAT) } {\def\sym#1{\ifmmode^{#1}\else\(^{#1}\)\fi} \begin{tabular}{l*{4}{c}} \toprule \multicolumn{4}{l}{Dep. variable: \textbf{Diseases of the genitourinary system}} \\ & \multicolumn{3}{c}{Choice of control group} \\ \cmidrule(lr){2-4}
            &\multicolumn{1}{c}{(1)}&\multicolumn{1}{c}{(2)}&\multicolumn{1}{c}{(3)}\\
            &\multicolumn{1}{c}{C1}&\multicolumn{1}{c}{C2}&\multicolumn{1}{c}{C1+C2}\\
\midrule
 \multicolumn{4}{l}{\emph{Panel A. Average causal effects}} \\ Abs. numbers        &       10.78\sym{***}&       19.17\sym{***}&       14.98\sym{**} \\
                    &     (3.721)         &     (4.482)         &     (5.868)         \\
 Ratio fertility     &      -0.128         &      -0.114\sym{*}  &      -0.121         \\
                    &    (0.0956)         &    (0.0576)         &     (0.102)         \\
 Ratio population    &       0.184\sym{*}  &      0.0294         &       0.107         \\
                    &     (0.107)         &    (0.0518)         &    (0.0934)         \\
 Cum. numbers        &       28.33         &       184.9\sym{***}&       106.6         \\
                    &     (37.53)         &     (44.08)         &     (74.29)         \\
 Cum. ratio          &      -2.627\sym{**} &      -1.083         &      -1.855         \\
                    &     (1.160)         &     (0.778)         &     (1.431)         \\
 \midrule\multicolumn{4}{l}{\emph{Panel B. Treatment effect heterogeneity - Women}} \\ Abs. numbers        &       4.658         &       12.04\sym{***}&       8.350         \\
                    &     (2.805)         &     (4.006)         &     (5.273)         \\
 Ratio fertility     &      -0.316\sym{**} &      -0.249\sym{***}&      -0.282         \\
                    &     (0.123)         &    (0.0837)         &     (0.175)         \\
 Ratio population    &       0.116         &     -0.0639         &      0.0259         \\
                    &     (0.131)         &    (0.0576)         &     (0.123)         \\
 Cum. numbers        &      -3.358         &       112.6\sym{**} &       54.61         \\
                    &     (35.19)         &     (45.65)         &     (73.00)         \\
 Cum. ratio          &      -4.733\sym{**} &      -2.495\sym{*}  &      -3.614         \\
                    &     (1.822)         &     (1.250)         &     (2.736)         \\
 \midrule\multicolumn{4}{l}{\emph{Panel C. Treatment effect heterogeneity - Men}} \\ Abs. numbers        &       6.125\sym{***}&       7.133\sym{***}&       6.629\sym{***}\\
                    &     (2.100)         &     (2.069)         &     (2.072)         \\
 Ratio fertility     &      0.0461         &      0.0125         &      0.0293         \\
                    &    (0.0899)         &    (0.0841)         &    (0.0862)         \\
 Ratio population    &       0.261\sym{**} &       0.117         &       0.189\sym{**} \\
                    &    (0.0966)         &    (0.0842)         &    (0.0911)         \\
 Cum. numbers        &       31.69         &       72.33\sym{***}&       52.01\sym{**} \\
                    &     (20.06)         &     (18.86)         &     (19.71)         \\
 Cum. ratio          &      -0.685         &       0.230         &      -0.228         \\
                    &     (0.873)         &     (0.901)         &     (0.886)         \\
 
\bottomrule \end{tabular} } \begin{tablenotes} \item \scriptsize \emph{Notes:} Clustered standard errors in parentheses. All regression are run with month-of-birth FEs and control cohort 3 is assigned with the treatment status. All regressions are carried out with a window width of half a year. \end{tablenotes} \end{threeparttable} \end{table} 

%---------------------------------
% CUMMULATIVE APPROACH
\begin{landscape}
 \begin{table}[H] \begin{threeparttable} \centering \caption{Cummulative effects for upt to different points of age} {\def\sym#1{\ifmmode^{#1}\else\(^{#1}\)\fi} \begin{tabular}{l*{13}{c}} \toprule & \multicolumn{12}{c}{Dependent variable: \textbf{Diseases of the genitourinary system}} \\ \cmidrule(lr){2-13}
            &\multicolumn{4}{c}{Average Causal Effects}         &\multicolumn{4}{c}{Women}                          &\multicolumn{4}{c}{Men}                            \\\cmidrule(lr){2-5}\cmidrule(lr){6-9}\cmidrule(lr){10-13}
            &\multicolumn{1}{c}{(1)}&\multicolumn{1}{c}{(2)}&\multicolumn{1}{c}{(3)}&\multicolumn{1}{c}{(4)}&\multicolumn{1}{c}{(5)}&\multicolumn{1}{c}{(6)}&\multicolumn{1}{c}{(7)}&\multicolumn{1}{c}{(8)}&\multicolumn{1}{c}{(9)}&\multicolumn{1}{c}{(10)}&\multicolumn{1}{c}{(11)}&\multicolumn{1}{c}{(12)}\\
            &\multicolumn{1}{c}{2M}&\multicolumn{1}{c}{4M}&\multicolumn{1}{c}{6M}&\multicolumn{1}{c}{Donut}&\multicolumn{1}{c}{2M}&\multicolumn{1}{c}{4M}&\multicolumn{1}{c}{6M}&\multicolumn{1}{c}{Donut}&\multicolumn{1}{c}{2M}&\multicolumn{1}{c}{4M}&\multicolumn{1}{c}{6M}&\multicolumn{1}{c}{Donut}\\
\midrule
 \multicolumn{13}{l}{\emph{Panel A. 2 Up to the age of 21}} \\ Cum. numbers        &       179.0         &       215.0\sym{**} &       156.8\sym{**} &       144.2\sym{*}  &       113.5\sym{**} &       132.8\sym{***}&       80.67\sym{**} &       69.60         &       65.50         &       82.25\sym{*}  &       76.17\sym{**} &       74.60\sym{*}  \\
                    &     (119.5)         &     (79.24)         &     (57.43)         &     (69.21)         &     (32.76)         &     (40.94)         &     (37.14)         &     (44.43)         &     (88.30)         &     (46.00)         &     (35.49)         &     (42.71)         \\
 Cum. ratio          &       2.750         &       2.375\sym{**} &       0.342         &      -0.342         &       3.257\sym{***}&       2.463\sym{**} &      -1.295         &      -2.396         &       2.128         &       2.244         &       1.651         &       1.380         \\
                    &     (1.520)         &     (0.854)         &     (1.150)         &     (1.270)         &     (0.700)         &     (0.933)         &     (1.932)         &     (2.165)         &     (2.962)         &     (1.397)         &     (1.290)         &     (1.538)         \\
 \midrule\multicolumn{13}{l}{\emph{Panel B. Up to the age of 26}} \\ Cum. numbers        &       66.00         &       341.2\sym{*}  &       315.3\sym{**} &       351.8\sym{**} &       45.50         &       215.5         &       194.8\sym{**} &       218.2\sym{**} &       20.50         &       125.8\sym{*}  &       120.5\sym{*}  &       133.6\sym{*}  \\
                    &     (99.66)         &     (179.4)         &     (118.4)         &     (136.1)         &     (27.55)         &     (125.2)         &     (83.21)         &     (87.30)         &     (105.7)         &     (70.92)         &     (62.57)         &     (75.17)         \\
 Cum. ratio          &      -0.179         &       3.352         &       1.241         &       1.120         &      -0.787         &       3.302         &      -0.558         &      -0.888         &       0.114         &       3.283         &       2.428         &       2.538         \\
                    &     (1.433)         &     (2.201)         &     (1.989)         &     (2.354)         &     (3.466)         &     (3.371)         &     (2.906)         &     (3.187)         &     (3.325)         &     (1.998)         &     (2.387)         &     (2.895)         \\
 \midrule\multicolumn{13}{l}{\emph{Panel C. Up to the age of 31}} \\ Cum. numbers        &       141.0\sym{*}  &       469.0\sym{*}  &       381.5\sym{**} &       411.8\sym{**} &       113.0         &       298.7\sym{*}  &       216.3\sym{*}  &       235.4\sym{*}  &       28.00         &       170.3\sym{*}  &       165.2\sym{**} &       176.4\sym{**} \\
                    &     (69.59)         &     (234.1)         &     (157.5)         &     (167.5)         &     (106.9)         &     (162.0)         &     (119.1)         &     (116.0)         &     (96.03)         &     (90.13)         &     (68.00)         &     (82.57)         \\
 Cum. ratio          &       0.776         &       4.362         &       0.272         &      -0.391         &       0.987         &       4.440         &      -3.409         &      -4.511         &       0.174         &       4.114         &       3.030         &       2.829         \\
                    &     (3.821)         &     (3.524)         &     (3.020)         &     (3.278)         &     (8.148)         &     (4.981)         &     (4.957)         &     (5.208)         &     (3.160)         &     (2.943)         &     (2.681)         &     (3.275)         \\
 \midrule\multicolumn{13}{l}{\emph{Panel D. Up to the age of 34}} \\ Cum. numbers        &       220.5         &       516.0\sym{*}  &       476.3\sym{**} &       481.2\sym{**} &       225.5         &       368.7\sym{**} &       321.2\sym{**} &       326.6\sym{***}&      -5.000         &       147.3         &       155.2         &       154.6         \\
                    &     (165.3)         &     (274.8)         &     (178.5)         &     (188.0)         &     (133.2)         &     (171.5)         &     (121.4)         &     (111.9)         &     (153.3)         &     (130.3)         &     (90.56)         &     (110.0)         \\
 Cum. ratio          &       1.793         &       3.820         &       0.130         &      -1.390         &       4.674         &       5.237         &      -2.158         &      -4.270         &      -1.432         &       2.261         &       1.406         &       0.501         \\
                    &     (5.714)         &     (4.086)         &     (3.191)         &     (3.128)         &     (10.16)         &     (5.262)         &     (4.886)         &     (4.689)         &     (5.604)         &     (4.297)         &     (3.314)         &     (3.949)         \\
 
\bottomrule \end{tabular} } \begin{tablenotes} \item \scriptsize \emph{Notes:} Clustered standard errors in parentheses (MxY). All regressions contain Birthmonth FE. Ratios indicate cases per thousand; original number of births. \end{tablenotes} \end{threeparttable} \end{table} 

\end{landscape}
\begin{landscape}
 \begin{table}[H] \begin{threeparttable} \centering \caption{Cummulative effects for upt to different points of age - BOOTSTRAPPED} {\def\sym#1{\ifmmode^{#1}\else\(^{#1}\)\fi} \begin{tabular}{l*{13}{c}} \toprule & \multicolumn{12}{c}{Dependent variable: \textbf{Diseases of the genitourinary system}} \\ \cmidrule(lr){2-13}
            &\multicolumn{4}{c}{Average Causal Effects}         &\multicolumn{4}{c}{Women}                          &\multicolumn{4}{c}{Men}                            \\\cmidrule(lr){2-5}\cmidrule(lr){6-9}\cmidrule(lr){10-13}
            &\multicolumn{1}{c}{(1)}&\multicolumn{1}{c}{(2)}&\multicolumn{1}{c}{(3)}&\multicolumn{1}{c}{(4)}&\multicolumn{1}{c}{(5)}&\multicolumn{1}{c}{(6)}&\multicolumn{1}{c}{(7)}&\multicolumn{1}{c}{(8)}&\multicolumn{1}{c}{(9)}&\multicolumn{1}{c}{(10)}&\multicolumn{1}{c}{(11)}&\multicolumn{1}{c}{(12)}\\
            &\multicolumn{1}{c}{2M}&\multicolumn{1}{c}{4M}&\multicolumn{1}{c}{6M}&\multicolumn{1}{c}{Donut}&\multicolumn{1}{c}{2M}&\multicolumn{1}{c}{4M}&\multicolumn{1}{c}{6M}&\multicolumn{1}{c}{Donut}&\multicolumn{1}{c}{2M}&\multicolumn{1}{c}{4M}&\multicolumn{1}{c}{6M}&\multicolumn{1}{c}{Donut}\\
\midrule
 \multicolumn{13}{l}{\emph{Panel A. 2 Up to the age of 21}} \\ Cum. numbers        &       179.0         &       215.0\sym{*}  &       156.8\sym{*}  &       144.2         &       113.5\sym{***}&       132.8\sym{**} &       80.67         &       69.60         &       65.50         &       82.25         &       76.17         &       74.60         \\
                    &     (117.0)         &     (123.0)         &     (86.24)         &     (105.6)         &     (31.28)         &     (62.93)         &     (51.01)         &     (59.02)         &     (86.91)         &     (68.02)         &     (56.78)         &     (67.75)         \\
 Cum. ratio          &       2.750\sym{*}  &       2.375\sym{*}  &       0.342         &      -0.342         &       3.257\sym{***}&       2.463\sym{**} &      -1.295         &      -2.396         &       2.128         &       2.244         &       1.651         &       1.380         \\
                    &     (1.454)         &     (1.343)         &     (1.534)         &     (1.670)         &     (0.613)         &     (1.218)         &     (2.602)         &     (2.581)         &     (2.905)         &     (2.085)         &     (2.001)         &     (2.396)         \\
 \midrule\multicolumn{13}{l}{\emph{Panel B. Up to the age of 26}} \\ Cum. numbers        &       66.00         &       341.2         &       315.3\sym{*}  &       351.8\sym{*}  &       45.50\sym{*}  &       215.5         &       194.8\sym{*}  &       218.2\sym{*}  &       20.50         &       125.8         &       120.5         &       133.6         \\
                    &     (95.29)         &     (248.9)         &     (170.0)         &     (208.6)         &     (25.01)         &     (172.2)         &     (104.7)         &     (127.5)         &     (103.6)         &     (96.33)         &     (97.59)         &     (111.6)         \\
 Cum. ratio          &      -0.179         &       3.352         &       1.241         &       1.120         &      -0.787         &       3.302         &      -0.558         &      -0.888         &       0.114         &       3.283         &       2.428         &       2.538         \\
                    &     (1.235)         &     (3.023)         &     (2.923)         &     (3.058)         &     (3.380)         &     (4.497)         &     (3.954)         &     (3.752)         &     (3.216)         &     (2.695)         &     (3.606)         &     (4.103)         \\
 \midrule\multicolumn{13}{l}{\emph{Panel C. Up to the age of 31}} \\ Cum. numbers        &       141.0\sym{**} &       469.0         &       381.5\sym{*}  &       411.8         &       113.0         &       298.7         &       216.3         &       235.4         &       28.00         &       170.3         &       165.2         &       176.4         \\
                    &     (61.99)         &     (311.1)         &     (219.3)         &     (251.3)         &     (105.7)         &     (212.0)         &     (149.1)         &     (168.5)         &     (90.19)         &     (122.4)         &     (111.2)         &     (124.8)         \\
 Cum. ratio          &       0.776         &       4.362         &       0.272         &      -0.391         &       0.987         &       4.440         &      -3.409         &      -4.511         &       0.174         &       4.114         &       3.030         &       2.829         \\
                    &     (3.673)         &     (4.773)         &     (4.369)         &     (3.752)         &     (8.055)         &     (6.558)         &     (6.790)         &     (5.874)         &     (2.810)         &     (4.105)         &     (4.275)         &     (4.599)         \\
 \midrule\multicolumn{13}{l}{\emph{Panel D. Up to the age of 34}} \\ Cum. numbers        &       220.5         &       516.0         &       476.3\sym{*}  &       481.2\sym{*}  &       225.5\sym{*}  &       368.7         &       321.2\sym{**} &       326.6\sym{**} &      -5.000         &       147.3         &       155.2         &       154.6         \\
                    &     (143.1)         &     (377.8)         &     (249.7)         &     (290.2)         &     (130.1)         &     (231.7)         &     (148.3)         &     (161.9)         &     (139.9)         &     (180.0)         &     (149.8)         &     (168.4)         \\
 Cum. ratio          &       1.793         &       3.820         &       0.130         &      -1.390         &       4.674         &       5.237         &      -2.158         &      -4.270         &      -1.432         &       2.261         &       1.406         &       0.501         \\
                    &     (5.304)         &     (5.525)         &     (4.600)         &     (3.792)         &     (9.997)         &     (6.808)         &     (6.607)         &     (5.484)         &     (4.904)         &     (6.035)         &     (5.532)         &     (5.690)         \\
 
\bottomrule \end{tabular} } \begin{tablenotes} \item \scriptsize \emph{Notes:} \textbf{BOOTSTRAPPED} standard errors in parentheses (MxY), with 400 replications. All regressions contain Birthmonth FE. Ratios indicate cases per thousand; original number of births. \end{tablenotes} \end{threeparttable} \end{table} 

\end{landscape}
%---------------------------------
\newpage
FEBRUAR CASES:
 \begin{table}[H] \begin{threeparttable} \centering \caption{Dep. variable: \textbf{Diseases of the genitourinary system}} {\def\sym#1{\ifmmode^{#1}\else\(^{#1}\)\fi} \begin{tabular}{l*{13}{c}} \toprule year & \multicolumn{12}{c}{Month of birth} \\ \cmidrule(lr){2-13} 
            &          11&          12&           1&           2&           3&           4&           5&           6&           7&           8&           9&          10\\
1995        &         378&         455&         435&         412&         471&         423&         416&         385&         394&         394&         364&         413\\
1996        &         489&         482&         458&         502&         507&         472&         440&         478&         444&         432&         413&         453\\
1997        &         493&         523&         530&         555&         602&         523&         542&         553&         470&         517&         494&         509\\
1998        &         534&         488&         518&         568&         649&         571&         552&         553&         568&         533&         516&         529\\
1999        &         553&         516&         540&         521&         667&         563&         599&         564&         598&         553&         520&         577\\
2000        &         513&         529&         530&         563&         549&         552&         599&         565&         517&         534&         534&         544\\
2001        &         537&         548&         545&         517&         542&         579&         587&         589&         542&         540&         539&         580\\
2002        &         589&         517&         570&         550&         618&         531&         592&         592&         575&         541&         521&         556\\
2003        &         527&         514&         562&         584&         634&         568&         575&         616&         533&         515&         520&         575\\
2004        &         472&         460&         504&         477&         553&         494&         540&         493&         476&         450&         461&         443\\
2005        &         435&         415&         463&         436&         483&         489&         473&         454&         458&         440&         459&         434\\
2006        &         477&         402&         468&         442&         484&         482&         507&         472&         464&         439&         443&         446\\
2007        &         473&         436&         412&         440&         498&         428&         487&         473&         518&         455&         473&         493\\
2008        &         416&         451&         505&         535&         514&         482&         481&         465&         478&         466&         504&         480\\
2009        &         488&         469&         485&         475&         559&         522&         588&         515&         519&         486&         518&         493\\
2010        &         507&         473&         484&         485&         559&         535&         548&         532&         514&         498&         489&         519\\
2011        &         485&         526&         497&         523&         564&         539&         548&         550&         517&         545&         544&         485\\
2012        &         507&         533&         545&         553&         592&         585&         586&         604&         559&         548&         514&         529\\
2013        &         559&         553&         601&         598&         575&         646&         584&         563&         573&         565&         550&         567\\
2014        &         551&         563&         590&         631&         629&         614&         691&         598&         614&         587&         625&         575\\
 \bottomrule \end{tabular} } \begin{tablenotes} \item \scriptsize \emph{Notes:} Number of cases per year and MOB in treatment cohort. \end{tablenotes} \end{threeparttable} \end{table} 

 \begin{table}[H] \begin{threeparttable} \centering \caption{Dep. variable: \textbf{Diseases of the genitourinary system}} {\def\sym#1{\ifmmode^{#1}\else\(^{#1}\)\fi} \begin{tabular}{l*{13}{c}} \toprule year & \multicolumn{12}{c}{Month of birth} \\ \cmidrule(lr){2-13} 
            &          11&          12&           1&           2&           3&           4&           5&           6&           7&           8&           9&          10\\
1995        &          36&         142&         107&          55&          95&          90&          37&          86&          31&          81&          96&          82\\
1996        &         105&          57&          47&         110&          48&          90&          11&         113&          15&          54&          55&          60\\
1997        &          31&          63&          75&         113&          99&          86&          31&          87&           3&          22&          38&          82\\
1998        &          30&         -35&         -37&          70&         111&          36&          18&           2&         -20&         -23&          -7&          -5\\
1999        &          32&         -51&           1&          -9&         104&          43&          15&          28&          30&         -28&         -24&          39\\
2000        &          -9&          59&          17&          78&           5&          -3&          57&          24&         -25&         -57&           7&          -8\\
2001        &          17&           8&          44&         -12&         -53&          67&          44&          42&         -63&        -104&         -26&          36\\
2002        &          44&          34&          18&          -1&          26&          -7&          59&          26&         -34&         -74&         -37&         -49\\
2003        &          21&          15&          20&          94&          27&          -9&         -16&          59&         -17&         -55&         -54&         -13\\
2004        &          62&          18&          19&           1&          27&          23&          50&          13&         -35&         -53&         -54&         -12\\
2005        &         -39&          -3&          36&         -22&          25&          47&           8&           6&         -38&         -20&         -14&         -39\\
2006        &          54&         -54&          29&         -41&         -24&          72&          64&          14&         -35&         -63&         -40&          18\\
2007        &          31&          -6&          11&         -42&           0&         -16&          22&          12&          18&         -29&          14&          60\\
2008        &         -41&          -1&          48&          67&           7&          47&          -2&         -42&          24&         -55&          47&         -45\\
2009        &          29&         -31&          45&         -28&          54&          39&          56&           1&          25&         -10&          28&           1\\
2010        &          20&         -16&           4&         -16&          22&          75&           9&          23&         -53&         -62&           9&          -8\\
2011        &         -30&          50&         -34&          18&          36&          19&         -20&          35&         -30&          49&           6&         -38\\
2012        &          72&          24&         -47&         -18&          26&          31&          34&          85&          38&         -21&         -28&         -84\\
2013        &          78&          29&          34&          58&         -20&          60&         -28&          -4&         -19&         -28&           2&          -5\\
2014        &         -19&           8&          -9&          32&           7&          81&          47&          -8&         -12&         -35&          52&         -16\\
 \bottomrule \end{tabular} } \begin{tablenotes} \item \scriptsize \emph{Notes:} Difference of cases (control - treatment) per year and MOB in treatment cohort. \end{tablenotes} \end{threeparttable} \end{table} 

