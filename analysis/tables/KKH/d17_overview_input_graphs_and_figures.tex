%---------------------------------
% INPUT FOR VARIABLE: d17
%---------------------------------
\subsection{d17}
% RD overview
\begin{landscape}
\begin{figure}[H]
	\centering
	\begin{minipage}{.95\linewidth}
	\includegraphics[width=\linewidth]{rd_d17_overview_panel1}
	{\scriptsize \emph{Notes:} The figures show monthly RD plots with averages obtained from a bin width of one month. The solid vertical line divides pre- and post-reform regime. The averages are taken over the period of at most 1995-2014. \par}
\end{minipage}
\end{figure}
\end{landscape}
\begin{landscape}
\begin{figure}[H]
	\centering
\begin{minipage}{.95\linewidth}
	\includegraphics[width=\linewidth]{rd_d17_overview_panel2}
	{\scriptsize \emph{Notes:} The figures show monthly RD plots with a moving average window width of 3 months. The solid vertical line divides pre- and post-reform regime. The averages are taken over the period of at most 1995-2014. \par}
\end{minipage}
\end{figure}
\end{landscape}
%---------------------------------
% TABELLEN
 \begin{table}[H] \begin{threeparttable} \centering \caption{Dep. variable: \textbf{Symptoms, signs and abnormal clinical and laboratory findings, not elsewhere classified}} {\def\sym#1{\ifmmode^{#1}\else\(^{#1}\)\fi} \begin{tabular}{l*{8}{c}} \toprule & \multicolumn{7}{c}{Estimation window} \\ \cmidrule(lr){2-8}
            &\multicolumn{1}{c}{(1)}&\multicolumn{1}{c}{(2)}&\multicolumn{1}{c}{(3)}&\multicolumn{1}{c}{(4)}&\multicolumn{1}{c}{(5)}&\multicolumn{1}{c}{(6)}&\multicolumn{1}{c}{(7)}\\
            &\multicolumn{1}{c}{1M}&\multicolumn{1}{c}{2M}&\multicolumn{1}{c}{3M}&\multicolumn{1}{c}{4M}&\multicolumn{1}{c}{5M}&\multicolumn{1}{c}{6M}&\multicolumn{1}{c}{Donut}\\
\midrule
 \multicolumn{8}{l}{\emph{Panel A. Average causal effects}} \\ Abs. numbers        &       8.600\sym{***}&       10.85\sym{***}&       7.867\sym{*}  &       8.100\sym{**} &       7.960\sym{***}&       10.53\sym{***}&       10.92\sym{***}\\
                    &  (4.77e-14)         &     (1.292)         &     (4.135)         &     (3.152)         &     (2.524)         &     (2.512)         &     (2.851)         \\
 Ratio fertility     &       0.144\sym{***}&       0.169\sym{***}&      0.0550         &      0.0316         &      0.0219         &      0.0177         &    -0.00756         \\
                    &  (7.81e-16)         &    (0.0118)         &    (0.0640)         &    (0.0585)         &    (0.0534)         &    (0.0443)         &    (0.0515)         \\
 Ratio population    &      0.0988\sym{***}&      0.0737         &   -0.000706         &     -0.0465         &     -0.0174         &     -0.0368         &     -0.0640         \\
                    &  (1.27e-15)         &    (0.0523)         &    (0.0601)         &    (0.0517)         &    (0.0450)         &    (0.0395)         &    (0.0421)         \\
 Ratio fert(03-14)   &       0.144\sym{***}&       0.142\sym{***}&      0.0830\sym{*}  &      0.0267         &      0.0466         &      0.0332         &      0.0110         \\
                    &  (7.28e-16)         &    (0.0242)         &    (0.0399)         &    (0.0417)         &    (0.0367)         &    (0.0317)         &    (0.0355)         \\
 Cum. numbers        &       115.0\sym{***}&       144.8\sym{***}&       91.55         &       104.5\sym{**} &       90.86\sym{**} &       122.1\sym{***}&       123.6\sym{***}\\
                    &  (2.02e-12)         &     (13.74)         &     (52.31)         &     (39.24)         &     (32.95)         &     (32.28)         &     (37.72)         \\
 Cum. ratio          &       1.899\sym{***}&       2.083\sym{***}&      -0.138         &     -0.0329         &      -0.456         &      -0.541         &      -1.029         \\
                    &  (2.94e-14)         &     (0.459)         &     (1.318)         &     (1.170)         &     (1.029)         &     (0.858)         &     (0.989)         \\
 \midrule\multicolumn{8}{l}{\emph{Panel B. Treatment effect heterogeneity - Women}} \\ Abs. numbers        &       5.200\sym{***}&       8.925\sym{***}&       7.750\sym{*}  &       8.350\sym{***}&       7.340\sym{***}&       7.075\sym{***}&       7.450\sym{***}\\
                    &  (2.61e-14)         &     (2.106)         &     (3.576)         &     (2.666)         &     (2.206)         &     (2.003)         &     (2.349)         \\
 Ratio fertility     &       0.162\sym{***}&       0.292\sym{***}&       0.165         &       0.158\sym{*}  &       0.113         &      0.0339         &     0.00825         \\
                    &  (6.27e-16)         &    (0.0583)         &     (0.101)         &    (0.0832)         &    (0.0790)         &    (0.0765)         &    (0.0910)         \\
 Ratio population    &       0.202\sym{***}&       0.331\sym{***}&       0.155         &       0.146         &       0.131\sym{*}  &      0.0395         &     0.00707         \\
                    &  (7.90e-16)         &    (0.0763)         &     (0.110)         &    (0.0834)         &    (0.0706)         &    (0.0739)         &    (0.0834)         \\
 Ratio fert(03-14)   &       0.193\sym{***}&       0.317\sym{***}&       0.148         &       0.140\sym{*}  &       0.129\sym{*}  &      0.0423         &      0.0122         \\
                    &  (7.90e-16)         &    (0.0726)         &     (0.105)         &    (0.0798)         &    (0.0681)         &    (0.0707)         &    (0.0801)         \\
 Cum. numbers        &       68.15\sym{***}&       96.82\sym{**} &       89.55\sym{*}  &       97.74\sym{***}&       79.01\sym{**} &       84.17\sym{***}&       87.37\sym{**} \\
                    &  (9.22e-13)         &     (29.65)         &     (44.40)         &     (33.13)         &     (28.52)         &     (26.15)         &     (31.27)         \\
 Cum. ratio          &       2.094\sym{***}&       3.135\sym{***}&       1.965         &       1.958\sym{*}  &       1.246         &       0.666         &       0.381         \\
                    &  (2.45e-14)         &     (0.758)         &     (1.191)         &     (0.959)         &     (0.955)         &     (0.862)         &     (1.021)         \\
 \midrule\multicolumn{8}{l}{\emph{Panel C. Treatment effect heterogeneity - Men}} \\ Abs. numbers        &       3.400\sym{***}&       1.925\sym{*}  &       0.117         &      -0.250         &       0.620         &       3.458\sym{**} &       3.470\sym{*}  \\
                    &  (1.41e-14)         &     (0.815)         &     (1.106)         &     (1.057)         &     (0.937)         &     (1.557)         &     (1.815)         \\
 Ratio fertility     &       0.121\sym{***}&      0.0567         &     -0.0446         &     -0.0806         &     -0.0591         &     0.00268         &     -0.0210         \\
                    &  (7.84e-16)         &    (0.0450)         &    (0.0530)         &    (0.0486)         &    (0.0424)         &    (0.0469)         &    (0.0549)         \\
 Ratio population    &       0.106\sym{***}&     -0.0209         &      0.0288         &     -0.0818         &     -0.0304         &      0.0262         &      0.0102         \\
                    &  (6.87e-16)         &    (0.0499)         &    (0.0395)         &    (0.0591)         &    (0.0528)         &    (0.0565)         &    (0.0665)         \\
 Ratio fert(03-14)   &      0.0954\sym{***}&     -0.0200         &      0.0248         &     -0.0751         &     -0.0277         &      0.0254         &      0.0114         \\
                    &  (9.02e-16)         &    (0.0453)         &    (0.0358)         &    (0.0534)         &    (0.0479)         &    (0.0517)         &    (0.0609)         \\
 Cum. numbers        &       46.85\sym{***}&       47.98\sym{**} &       2.000         &       6.762         &       11.85         &       37.97\sym{**} &       36.19\sym{*}  \\
                    &  (3.28e-14)         &     (19.93)         &     (24.55)         &     (18.47)         &     (14.99)         &     (17.65)         &     (20.40)         \\
 Cum. ratio          &       1.536\sym{***}&       1.382         &      -0.403         &      -0.487         &      -0.363         &       0.222         &     -0.0409         \\
                    &  (6.95e-15)         &     (0.763)         &     (0.924)         &     (0.819)         &     (0.687)         &     (0.646)         &     (0.762)         \\
 
\bottomrule \end{tabular} } \begin{tablenotes} \item \scriptsize \emph{Notes:} Clustered standard errors in parentheses. All regression are run with CG2 (i.e. the cohort prior to the reform) and with month-of-birth FEs. Ratios indicate cases per thousand; either approximated population (with weights coming from the original fertility distribution) or original number of births. \end{tablenotes} \end{threeparttable} \end{table} 

 \begin{table}[H] \begin{threeparttable} \centering \caption{Robustness with respect to the choice of \texttt{control group}} {\def\sym#1{\ifmmode^{#1}\else\(^{#1}\)\fi} \begin{tabular}{l*{10}{c}} \toprule & \multicolumn{9}{c}{Dependent variable: \textbf{Symptoms, signs and abnormal clinical and laboratory findings, not elsewhere classified}} \\ \cmidrule(lr){2-10}
            &\multicolumn{3}{c}{Average Causal Effects}&\multicolumn{3}{c}{Women}             &\multicolumn{3}{c}{Men}               \\\cmidrule(lr){2-4}\cmidrule(lr){5-7}\cmidrule(lr){8-10}
            &\multicolumn{1}{c}{(1)}&\multicolumn{1}{c}{(2)}&\multicolumn{1}{c}{(3)}&\multicolumn{1}{c}{(4)}&\multicolumn{1}{c}{(5)}&\multicolumn{1}{c}{(6)}&\multicolumn{1}{c}{(7)}&\multicolumn{1}{c}{(8)}&\multicolumn{1}{c}{(9)}\\
            &\multicolumn{1}{c}{C2}&\multicolumn{1}{c}{C1+C2}&\multicolumn{1}{c}{C1-C3}&\multicolumn{1}{c}{C2}&\multicolumn{1}{c}{C1+C2}&\multicolumn{1}{c}{C1-C3}&\multicolumn{1}{c}{C2}&\multicolumn{1}{c}{C1+C2}&\multicolumn{1}{c}{C1-C3}\\
\midrule
 \multicolumn{10}{l}{\emph{Panel A. 2 Month bandwidth}} \\ Abs. numbers        &       10.85\sym{***}&       8.400\sym{**} &       7.083         &       8.925\sym{***}&       8.300\sym{**} &       6.675         &       1.925\sym{*}  &       0.100         &       0.408         \\
                    &     (1.292)         &     (3.550)         &     (4.458)         &     (2.106)         &     (2.928)         &     (4.301)         &     (0.815)         &     (1.186)         &     (0.891)         \\
 Ratio population    &     -0.0625         &      0.0718         &      0.0869         &    -0.00162         &       0.140         &      0.0222         &     -0.0888         &      0.0340         &       0.148         \\
                    &    (0.0403)         &     (0.168)         &     (0.144)         &     (0.167)         &     (0.309)         &     (0.317)         &     (0.126)         &     (0.134)         &     (0.144)         \\
 Ratio population    &      0.0737         &       0.146         &       0.164         &       0.213\sym{*}  &       0.275         &       0.270         &     -0.0708         &      0.0112         &      0.0550         \\
                    &    (0.0523)         &     (0.141)         &     (0.119)         &    (0.0981)         &     (0.216)         &     (0.188)         &    (0.0557)         &    (0.0845)         &    (0.0791)         \\
 \midrule\multicolumn{10}{l}{\emph{Panel B. 4 Month bandwidth}} \\ Abs. numbers        &       6.278\sym{**} &       6.500         &       4.796         &       6.639\sym{**} &       6.917\sym{*}  &       4.972         &      -0.361         &      -0.417         &      -0.176         \\
                    &     (2.911)         &     (4.287)         &     (4.891)         &     (2.303)         &     (3.810)         &     (4.477)         &     (2.470)         &     (2.839)         &     (2.610)         \\
 Ratio fertility     &      0.0316         &      0.0627         &      0.0663         &       0.158\sym{*}  &       0.215\sym{**} &       0.196\sym{**} &     -0.0806         &     -0.0708         &     -0.0480         \\
                    &    (0.0585)         &    (0.0589)         &    (0.0540)         &    (0.0832)         &    (0.0901)         &    (0.0829)         &    (0.0486)         &    (0.0634)         &    (0.0663)         \\
 Ratio population    &      0.0295         &      0.0863         &      0.0939         &       0.146         &       0.208\sym{*}  &       0.202\sym{*}  &     -0.0818         &     -0.0282         &    -0.00823         \\
                    &    (0.0449)         &    (0.0672)         &    (0.0569)         &    (0.0834)         &     (0.117)         &     (0.104)         &    (0.0591)         &    (0.0858)         &    (0.0821)         \\
 \midrule\multicolumn{10}{l}{\emph{Panel C. 6 Month bandwidth}} \\ Abs. numbers        &       10.20\sym{***}&       6.602\sym{*}  &       3.907         &       4.148\sym{**} &       2.546         &       0.481         &       6.056\sym{**} &       4.056         &       3.426         \\
                    &     (2.633)         &     (3.543)         &     (3.954)         &     (1.833)         &     (3.050)         &     (3.577)         &     (2.902)         &     (2.750)         &     (2.462)         \\
 Ratio fertility     &     -0.0922         &      -0.130         &     -0.0606         &      -0.136         &      -0.181         &      -0.103         &     -0.0642         &     -0.0949         &     -0.0339         \\
                    &    (0.0657)         &    (0.0889)         &    (0.0804)         &     (0.109)         &     (0.154)         &     (0.143)         &    (0.0625)         &    (0.0725)         &    (0.0735)         \\
 Ratio population    &      0.0328         &      0.0267         &      0.0342         &      0.0395         &      0.0321         &      0.0307         &      0.0262         &      0.0228         &      0.0403         \\
                    &    (0.0336)         &    (0.0579)         &    (0.0508)         &    (0.0739)         &     (0.106)         &    (0.0989)         &    (0.0565)         &    (0.0657)         &    (0.0624)         \\
 \midrule\multicolumn{10}{l}{\emph{Panel D. Donut specification}} \\ Abs. numbers        &       10.51\sym{***}&       5.056         &       2.526         &       3.822\sym{**} &       0.978         &      -0.941         &       6.689\sym{*}  &       4.078         &       3.467         \\
                    &     (2.952)         &     (3.842)         &     (4.460)         &     (1.704)         &     (2.803)         &     (3.725)         &     (3.472)         &     (3.247)         &     (2.883)         \\
 Ratio fertility     &      -0.137\sym{*}  &      -0.226\sym{***}&      -0.138\sym{*}  &      -0.185         &      -0.314\sym{*}  &      -0.221         &      -0.104         &      -0.155\sym{*}  &     -0.0732         \\
                    &    (0.0740)         &    (0.0806)         &    (0.0748)         &     (0.129)         &     (0.155)         &     (0.145)         &    (0.0704)         &    (0.0772)         &    (0.0802)         \\
 Ratio fertility     &     -0.0192         &     -0.0282         &     -0.0171         &     -0.0953         &     -0.0955         &     -0.0912         &      0.0547         &      0.0375         &      0.0559         \\
                    &    (0.0453)         &    (0.0528)         &    (0.0475)         &    (0.0921)         &     (0.102)         &    (0.0876)         &    (0.0844)         &    (0.0840)         &    (0.0876)         \\
 
\bottomrule \end{tabular} } \begin{tablenotes} \item \scriptsize \emph{Notes:} Clustered standard errors in parentheses. All regressions contain Birthmonth FE. Ratios indicate cases per thousand; either approximated population (with weights coming from the original fertility distribution) or original number of births. \end{tablenotes} \end{threeparttable} \end{table} 

%---------------------------------
% Life-course figure (Panel1)
\begin{landscape}
\begin{figure}[H]
\centering
\begin{minipage}{.9\linewidth}
\includegraphics[width=\linewidth]{lc_d17_overview_panel1}
{\scriptsize \emph{Notes:} The figures depict DDRD estimates and 90\% confidence intervals over the life-course. The years are harmonized such that the cohorts are in the same age when they are compared. All regressions are carried out with month-of-birth FE and make use of clustered standard errors. Furthermore, we used a bandwidth of half a year and only the control cohort that was born one year prior to the reform. Ratios indicate cases per thousand; using in the denominator the approximated population (with weights coming from the original fertility distribution) or original number of births. \par}
\end{minipage}
\end{figure}
\end{landscape}
%---------------------------------
% Life-course figure (Panel2)
\begin{landscape}
\begin{figure}[H]
\centering
\begin{minipage}{.9\linewidth}
\includegraphics[width=\linewidth]{lc_d17_overview_panel2}
{\scriptsize \emph{Notes:} The figures depict DDRD estimates and 90\% confidence intervals over the life-course. The years are harmonized such that the cohorts are in the same age when they are compared. All regressions are carried out with month-of-birth FE and make use of clustered standard errors. Furthermore, we used a bandwidth of half a year. Ratios indicate cases per thousand; using in the denominator the approximated population (with weights coming from the original fertility distribution) or original number of births. \par}
\end{minipage}
\end{figure}
\end{landscape}
%---------------------------------
% Life-course (panel 3 - 6)
\begin{figure}[H]%\vspace*{-2cm}
	\centering
	\includegraphics[width=.9\linewidth]{lc_d17_overview_panel3}
	\includegraphics[width=.9\linewidth]{lc_d17_overview_panel4}
\end{figure}
\begin{figure}[H]
	\centering	
	\includegraphics[width=.97\linewidth]{lc_d17_overview_panel5}
	\includegraphics[width=.97\linewidth]{lc_d17_overview_panel6}
\end{figure}
% Life-course TABLE Format
 \begin{table}[H] \centering \begin{threeparttable} \caption{Life-course approach - Table format} {\def\sym#1{\ifmmode^{#1}\else\(^{#1}\)\fi} \begin{tabular}{l*{5}{c}} \toprule \multicolumn{5}{l}{Dep. variable: \textbf{Symptoms, signs and abnormal clinical and laboratory findings, not elsewhere classified}} \\ & \multicolumn{4}{c}{Estimation window} \\ \cmidrule(lr){2-5}
            &\multicolumn{1}{c}{(1)}&\multicolumn{1}{c}{(2)}&\multicolumn{1}{c}{(3)}&\multicolumn{1}{c}{(4)}\\
            &\multicolumn{1}{c}{Age 17-21}&\multicolumn{1}{c}{Age 22-26}&\multicolumn{1}{c}{Age 27-31}&\multicolumn{1}{c}{Age 32-35}\\
\midrule
 \multicolumn{5}{l}{\emph{Panel A. Average causal effects}} \\ Abs. numbers        &       6.767         &       17.87         &       7.233         &       14.21\sym{*}  \\
                    &     (11.31)         &     (11.51)         &     (6.400)         &     (8.333)         \\
 Ratio fertility     &      -0.175         &       0.112         &      -0.164         &      -0.122         \\
                    &     (0.303)         &     (0.216)         &     (0.113)         &     (0.195)         \\
 Ratio population    &       0.263\sym{*}  &      -0.109         &     -0.0730         \\
                    &     (0.160)         &    (0.0844)         &     (0.158)         \\
 Cum. numbers        &       49.30         &       100.6         &       196.2\sym{**} &       221.8\sym{**} \\
                    &     (35.89)         &     (77.19)         &     (95.66)         &     (101.6)         \\
 Cum. ratio          &      -0.354         &      -0.694         &      -0.100         &      -1.228         \\
                    &     (1.065)         &     (2.026)         &     (2.453)         &     (2.845)         \\
 \midrule\multicolumn{5}{l}{\emph{Panel B. Treatment effect heterogeneity - Women}} \\ Abs. numbers        &       6.333         &       12.23         &       0.433         &       9.583         \\
                    &     (6.609)         &     (9.107)         &     (7.630)         &     (7.903)         \\
 Ratio fertility     &      -0.214         &       0.139         &      -0.396         &      -0.105         \\
                    &     (0.322)         &     (0.342)         &     (0.301)         &     (0.464)         \\
 Ratio population    &       0.323         &      -0.274         &     -0.0519         \\
                    &     (0.248)         &     (0.214)         &     (0.349)         \\
 Cum. numbers        &       46.00\sym{**} &       73.47         &       127.3         &       136.0         \\
                    &     (22.08)         &     (56.18)         &     (81.73)         &     (91.27)         \\
 Cum. ratio          &      -0.153         &      -1.127         &      -0.722         &      -2.433         \\
                    &     (1.185)         &     (2.444)         &     (3.371)         &     (4.297)         \\
 \midrule\multicolumn{5}{l}{\emph{Panel C. Treatment effect heterogeneity - Men}} \\ Abs. numbers        &       0.433         &       5.633         &       6.800         &       4.625         \\
                    &     (8.105)         &     (4.497)         &     (6.678)         &     (8.842)         \\
 Ratio fertility     &      -0.158         &      0.0689         &      0.0486         &      -0.143         \\
                    &     (0.385)         &     (0.183)         &     (0.271)         &     (0.256)         \\
 Ratio population    &       0.191         &      0.0516         &     -0.0973         \\
                    &     (0.193)         &     (0.218)         &     (0.206)         \\
 Cum. numbers        &       3.300         &       27.13         &       68.87         &       85.87\sym{*}  \\
                    &     (23.10)         &     (46.24)         &     (47.35)         &     (49.00)         \\
 Cum. ratio          &      -0.629         &      -0.464         &       0.255         &      -0.348         \\
                    &     (1.183)         &     (2.307)         &     (2.582)         &     (2.589)         \\
 
\bottomrule \end{tabular} } \begin{tablenotes} \item \scriptsize \emph{Notes:} Clustered standard errors in parentheses. All regression are run with CG2 (i.e. the cohort prior to the reform) and with month-of-birth FEs. Ratios indicate cases per thousand; either approximated population (with weights coming from the original fertility distribution) or original number of births. Raqtio population muss eins nach rechts gerückt werden \end{tablenotes} \end{threeparttable} \end{table} 

%---------------------------------
% PLACEBO EXERCISES
\newpage
\begin{landscape}
\begin{figure}[H]
	\centering
    \begin{minipage}{.9\linewidth}
	\includegraphics[width=\linewidth]{placebo_graph_d17.pdf}
    {\scriptsize \emph{Notes:} The figures depict DDRD estimates and 95\% confidence intervals when the treatment cohort is shifted over time. The date on the abscissa indicates the starting date of the treated.  All regressions are carried out with month-of-birth FE and make use of clustered standard errors. Furthermore, we used a bandwidth of half a year. Ratios indicate cases per thousand; using in the denominator the approximated population (with weights coming from the original fertility distribution) or original number of births. \par}
    \end{minipage}
\end{figure}
\end{landscape}
 \begin{table}[H] \centering \begin{threeparttable} \caption{Placebo 1 (CONTROL1 ist TREAT) } {\def\sym#1{\ifmmode^{#1}\else\(^{#1}\)\fi} \begin{tabular}{l*{4}{c}} \toprule \multicolumn{4}{l}{Dep. variable: \textbf{Symptoms, signs and abnormal clinical and laboratory findings, not elsewhere classified}} \\ & \multicolumn{3}{c}{Choice of control group} \\ \cmidrule(lr){2-4}
            &\multicolumn{1}{c}{(1)}&\multicolumn{1}{c}{(2)}&\multicolumn{1}{c}{(3)}\\
            &\multicolumn{1}{c}{C2}&\multicolumn{1}{c}{C3}&\multicolumn{1}{c}{C2+C3}\\
\midrule
 \multicolumn{4}{l}{\emph{Panel A. Average causal effects}} \\ Abs. numbers        &       7.892\sym{**} &      -2.408         &       2.742         \\
                    &     (3.240)         &     (3.596)         &     (5.073)         \\
 Ratio fertility     &      0.0749         &       0.245\sym{**} &       0.160\sym{*}  \\
                    &    (0.0947)         &    (0.0878)         &    (0.0922)         \\
 Ratio population    &      0.0383         &       0.124         &      0.0813         \\
                    &    (0.0812)         &    (0.0769)         &    (0.0784)         \\
 Cum. numbers        &       103.5\sym{**} &      -18.82         &       42.36         \\
                    &     (38.13)         &     (39.13)         &     (55.54)         \\
 Cum. ratio          &       1.297         &       2.575\sym{***}&       1.936\sym{**} \\
                    &     (0.974)         &     (0.826)         &     (0.902)         \\
 \midrule\multicolumn{4}{l}{\emph{Panel B. Treatment effect heterogeneity - Women}} \\ Abs. numbers        &       4.583         &      -2.742         &       0.921         \\
                    &     (2.856)         &     (3.379)         &     (4.616)         \\
 Ratio fertility     &      0.0906         &       0.280\sym{**} &       0.185         \\
                    &     (0.139)         &     (0.132)         &     (0.140)         \\
 Ratio population    &      0.0478         &       0.109         &      0.0786         \\
                    &     (0.118)         &     (0.114)         &     (0.118)         \\
 Cum. numbers        &       61.32\sym{*}  &      -20.83         &       20.25         \\
                    &     (31.69)         &     (35.49)         &     (50.65)         \\
 Cum. ratio          &       1.565         &       3.355\sym{**} &       2.460         \\
                    &     (1.453)         &     (1.354)         &     (1.474)         \\
 \midrule\multicolumn{4}{l}{\emph{Panel C. Treatment effect heterogeneity - Men}} \\ Abs. numbers        &       3.308\sym{**} &       0.333         &       1.821         \\
                    &     (1.533)         &     (1.061)         &     (1.369)         \\
 Ratio fertility     &      0.0616         &       0.214\sym{**} &       0.138         \\
                    &    (0.0837)         &    (0.0871)         &    (0.0982)         \\
 Ratio population    &      0.0273         &       0.140         &      0.0839         \\
                    &    (0.0889)         &    (0.0868)         &    (0.0931)         \\
 Cum. numbers        &       42.22\sym{***}&       2.017         &       22.12         \\
                    &     (13.45)         &     (14.39)         &     (14.43)         \\
 Cum. ratio          &       1.055         &       1.852\sym{**} &       1.454         \\
                    &     (0.693)         &     (0.780)         &     (0.880)         \\
 
\bottomrule \end{tabular} } \begin{tablenotes} \item \scriptsize \emph{Notes:} Clustered standard errors in parentheses. All regression are run with month-of-birth FEs and control cohort 2 is assigned with the treatment status. All regressions are carried out with a window width of half a year. \end{tablenotes} \end{threeparttable} \end{table} 

 \begin{table}[H] \centering \begin{threeparttable} \caption{Placebo 2 (CONTROL2 ist TREAT) } {\def\sym#1{\ifmmode^{#1}\else\(^{#1}\)\fi} \begin{tabular}{l*{4}{c}} \toprule \multicolumn{4}{l}{Dep. variable: \textbf{Symptoms, signs and abnormal clinical and laboratory findings, not elsewhere classified}} \\ & \multicolumn{3}{c}{Choice of control group} \\ \cmidrule(lr){2-4}
            &\multicolumn{1}{c}{(1)}&\multicolumn{1}{c}{(2)}&\multicolumn{1}{c}{(3)}\\
            &\multicolumn{1}{c}{C1}&\multicolumn{1}{c}{C3}&\multicolumn{1}{c}{C1+C3}\\
\midrule
 \multicolumn{4}{l}{\emph{Panel A. Average causal effects}} \\ Abs. numbers        &      -7.892\sym{**} &      -10.30\sym{***}&      -9.096         \\
                    &     (3.240)         &     (3.663)         &     (5.508)         \\
 Ratio fertility     &     -0.0749         &       0.170\sym{***}&      0.0474         \\
                    &    (0.0947)         &    (0.0403)         &    (0.0787)         \\
 Ratio population    &     -0.0383         &      0.0861         &      0.0239         \\
                    &    (0.0812)         &    (0.0553)         &    (0.0701)         \\
 Cum. numbers        &      -103.5\sym{**} &      -122.4\sym{***}&      -113.0\sym{*}  \\
                    &     (38.13)         &     (41.49)         &     (64.94)         \\
 Cum. ratio          &      -1.297         &       1.279\sym{***}&    -0.00896         \\
                    &     (0.974)         &     (0.430)         &     (0.874)         \\
 \midrule\multicolumn{4}{l}{\emph{Panel B. Treatment effect heterogeneity - Women}} \\ Abs. numbers        &      -4.583         &      -7.325\sym{**} &      -5.954         \\
                    &     (2.856)         &     (2.733)         &     (4.970)         \\
 Ratio fertility     &     -0.0906         &       0.189\sym{**} &      0.0492         \\
                    &     (0.139)         &    (0.0852)         &     (0.143)         \\
 Ratio population    &     -0.0478         &      0.0615         &     0.00688         \\
                    &     (0.118)         &    (0.0856)         &     (0.105)         \\
 Cum. numbers        &      -61.33\sym{*}  &      -82.16\sym{**} &      -71.74         \\
                    &     (31.69)         &     (34.88)         &     (60.28)         \\
 Cum. ratio          &      -1.565         &       1.790\sym{*}  &       0.113         \\
                    &     (1.453)         &     (1.005)         &     (1.727)         \\
 \midrule\multicolumn{4}{l}{\emph{Panel C. Treatment effect heterogeneity - Men}} \\ Abs. numbers        &      -3.308\sym{**} &      -2.975\sym{*}  &      -3.142\sym{**} \\
                    &     (1.533)         &     (1.446)         &     (1.485)         \\
 Ratio fertility     &     -0.0616         &       0.152\sym{***}&      0.0454         \\
                    &    (0.0837)         &    (0.0386)         &    (0.0796)         \\
 Ratio population    &     -0.0273         &       0.113\sym{**} &      0.0429         \\
                    &    (0.0889)         &    (0.0499)         &    (0.0834)         \\
 Cum. numbers        &      -42.22\sym{***}&      -40.20\sym{**} &      -41.21\sym{**} \\
                    &     (13.45)         &     (16.97)         &     (15.18)         \\
 Cum. ratio          &      -1.055         &       0.797         &      -0.129         \\
                    &     (0.693)         &     (0.586)         &     (0.770)         \\
 
\bottomrule \end{tabular} } \begin{tablenotes} \item \scriptsize \emph{Notes:} Clustered standard errors in parentheses. All regression are run with month-of-birth FEs and control cohort 2 is assigned with the treatment status. All regressions are carried out with a window width of half a year. \end{tablenotes} \end{threeparttable} \end{table} 

 \begin{table}[H] \centering \begin{threeparttable} \caption{Placebo 3 (CONTROL3 ist TREAT) } {\def\sym#1{\ifmmode^{#1}\else\(^{#1}\)\fi} \begin{tabular}{l*{4}{c}} \toprule \multicolumn{4}{l}{Dep. variable: \textbf{Symptoms, signs and abnormal clinical and laboratory findings, not elsewhere classified}} \\ & \multicolumn{3}{c}{Choice of control group} \\ \cmidrule(lr){2-4}
            &\multicolumn{1}{c}{(1)}&\multicolumn{1}{c}{(2)}&\multicolumn{1}{c}{(3)}\\
            &\multicolumn{1}{c}{C1}&\multicolumn{1}{c}{C2}&\multicolumn{1}{c}{C1+C2}\\
\midrule
 \multicolumn{4}{l}{\emph{Panel A. Average causal effects}} \\ Abs. numbers        &       2.408         &       10.30\sym{***}&       6.354\sym{*}  \\
                    &     (3.596)         &     (3.663)         &     (3.744)         \\
 Ratio fertility     &      -0.245\sym{**} &      -0.170\sym{***}&      -0.207\sym{***}\\
                    &    (0.0878)         &    (0.0403)         &    (0.0728)         \\
 Ratio population    &      -0.124         &     -0.0861         &      -0.105         \\
                    &    (0.0769)         &    (0.0553)         &    (0.0665)         \\
 Cum. numbers        &       18.82         &       122.4\sym{***}&       70.59         \\
                    &     (39.13)         &     (41.49)         &     (43.39)         \\
 Cum. ratio          &      -2.575\sym{***}&      -1.279\sym{***}&      -1.927\sym{**} \\
                    &     (0.826)         &     (0.430)         &     (0.771)         \\
 \midrule\multicolumn{4}{l}{\emph{Panel B. Treatment effect heterogeneity - Women}} \\ Abs. numbers        &       2.742         &       7.325\sym{**} &       5.033         \\
                    &     (3.379)         &     (2.733)         &     (3.172)         \\
 Ratio fertility     &      -0.280\sym{**} &      -0.189\sym{**} &      -0.234\sym{*}  \\
                    &     (0.132)         &    (0.0852)         &     (0.120)         \\
 Ratio population    &      -0.109         &     -0.0615         &     -0.0854         \\
                    &     (0.114)         &    (0.0856)         &    (0.0993)         \\
 Cum. numbers        &       20.83         &       82.16\sym{**} &       51.50         \\
                    &     (35.49)         &     (34.88)         &     (37.75)         \\
 Cum. ratio          &      -3.355\sym{**} &      -1.790\sym{*}  &      -2.572\sym{*}  \\
                    &     (1.354)         &     (1.005)         &     (1.378)         \\
 \midrule\multicolumn{4}{l}{\emph{Panel C. Treatment effect heterogeneity - Men}} \\ Abs. numbers        &      -0.333         &       2.975\sym{*}  &       1.321         \\
                    &     (1.061)         &     (1.446)         &     (1.303)         \\
 Ratio fertility     &      -0.214\sym{**} &      -0.152\sym{***}&      -0.183\sym{***}\\
                    &    (0.0871)         &    (0.0386)         &    (0.0672)         \\
 Ratio population    &      -0.140         &      -0.113\sym{**} &      -0.127\sym{*}  \\
                    &    (0.0868)         &    (0.0499)         &    (0.0704)         \\
 Cum. numbers        &      -2.017         &       40.20\sym{**} &       19.09         \\
                    &     (14.39)         &     (16.97)         &     (16.16)         \\
 Cum. ratio          &      -1.852\sym{**} &      -0.797         &      -1.325\sym{*}  \\
                    &     (0.780)         &     (0.586)         &     (0.697)         \\
 
\bottomrule \end{tabular} } \begin{tablenotes} \item \scriptsize \emph{Notes:} Clustered standard errors in parentheses. All regression are run with month-of-birth FEs and control cohort 3 is assigned with the treatment status. All regressions are carried out with a window width of half a year. \end{tablenotes} \end{threeparttable} \end{table} 

%---------------------------------
% CUMMULATIVE APPROACH
\begin{landscape}
 \begin{table}[H] \begin{threeparttable} \centering \caption{Cummulative effects for upt to different points of age} {\def\sym#1{\ifmmode^{#1}\else\(^{#1}\)\fi} \begin{tabular}{l*{13}{c}} \toprule & \multicolumn{12}{c}{Dependent variable: \textbf{Symptoms, signs and abnormal clinical and laboratory findings, not elsewhere classified}} \\ \cmidrule(lr){2-13}
            &\multicolumn{4}{c}{Average Causal Effects}         &\multicolumn{4}{c}{Women}                          &\multicolumn{4}{c}{Men}                            \\\cmidrule(lr){2-5}\cmidrule(lr){6-9}\cmidrule(lr){10-13}
            &\multicolumn{1}{c}{(1)}&\multicolumn{1}{c}{(2)}&\multicolumn{1}{c}{(3)}&\multicolumn{1}{c}{(4)}&\multicolumn{1}{c}{(5)}&\multicolumn{1}{c}{(6)}&\multicolumn{1}{c}{(7)}&\multicolumn{1}{c}{(8)}&\multicolumn{1}{c}{(9)}&\multicolumn{1}{c}{(10)}&\multicolumn{1}{c}{(11)}&\multicolumn{1}{c}{(12)}\\
            &\multicolumn{1}{c}{2M}&\multicolumn{1}{c}{4M}&\multicolumn{1}{c}{6M}&\multicolumn{1}{c}{Donut}&\multicolumn{1}{c}{2M}&\multicolumn{1}{c}{4M}&\multicolumn{1}{c}{6M}&\multicolumn{1}{c}{Donut}&\multicolumn{1}{c}{2M}&\multicolumn{1}{c}{4M}&\multicolumn{1}{c}{6M}&\multicolumn{1}{c}{Donut}\\
\midrule
 \multicolumn{13}{l}{\emph{Panel A. 2 Up to the age of 21}} \\ Cum. numbers        &          73         &       60.25         &       62.33         &       52.20         &       52.50         &          65\sym{**} &       49.00\sym{*}  &       38.40         &       20.50         &      -4.750         &       13.33         &       13.80         \\
                    &     (40.50)         &     (41.95)         &     (37.95)         &     (45.53)         &     (37.37)         &     (25.84)         &     (24.47)         &     (28.52)         &     (42.61)         &     (33.35)         &     (24.98)         &     (30.30)         \\
 Cum. ratio          &       0.957         &      -0.194         &      -0.676         &      -1.167         &       1.290         &       0.692         &      -0.939         &      -1.766         &       0.578         &      -1.055         &      -0.550         &      -0.718         \\
                    &     (1.389)         &     (1.495)         &     (1.141)         &     (1.328)         &     (1.384)         &     (1.351)         &     (1.338)         &     (1.472)         &     (2.128)         &     (1.819)         &     (1.262)         &     (1.528)         \\
 \midrule\multicolumn{13}{l}{\emph{Panel B. Up to the age of 26}} \\ Cum. numbers        &       161.5\sym{***}&       130.8         &       151.7\sym{**} &       154.8\sym{**} &       111.5         &       114.7\sym{*}  &       110.2\sym{**} &       116.4\sym{**} &       50.00         &       16.00         &       41.50         &       38.40         \\
                    &     (29.77)         &     (74.96)         &     (57.75)         &     (69.64)         &     (76.03)         &     (57.74)         &     (41.34)         &     (50.32)         &     (49.04)         &     (40.97)         &     (32.74)         &     (37.60)         \\
 Cum. ratio          &       2.390\sym{**} &       0.326         &      -0.116         &      -0.559         &       3.179         &       1.538         &      -0.242         &      -0.660         &       1.529         &      -0.866         &      -0.205         &      -0.661         \\
                    &     (0.695)         &     (2.137)         &     (1.523)         &     (1.815)         &     (1.849)         &     (2.381)         &     (1.917)         &     (2.309)         &     (2.758)         &     (2.340)         &     (1.637)         &     (1.887)         \\
 \midrule\multicolumn{13}{l}{\emph{Panel C. Up to the age of 31}} \\ Cum. numbers        &       181.5\sym{***}&       147.5         &       187.8\sym{**} &       194.2\sym{**} &       109.5         &       127.8         &       112.3\sym{*}  &       116.2         &       72.00         &       19.75         &       75.50\sym{*}  &       78.00         \\
                    &     (24.52)         &     (88.14)         &     (69.83)         &     (80.94)         &     (76.21)         &     (81.08)         &     (59.88)         &     (72.17)         &     (59.03)         &     (36.07)         &     (40.96)         &     (48.00)         \\
 Cum. ratio          &       2.395         &      -0.414         &      -0.934         &      -1.569         &       2.523\sym{*}  &       0.742         &      -2.220         &      -3.051         &       2.143         &      -1.557         &      0.0375         &      -0.396         \\
                    &     (1.333)         &     (2.624)         &     (1.756)         &     (2.027)         &     (1.232)         &     (3.245)         &     (2.634)         &     (3.130)         &     (3.605)         &     (2.680)         &     (1.977)         &     (2.325)         \\
 \midrule\multicolumn{13}{l}{\emph{Panel D. Up to the age of 34}} \\ Cum. numbers        &       254.5\sym{***}&       184.2\sym{*}  &       244.7\sym{***}&       249.4\sym{**} &       190.5\sym{**} &       171.5\sym{*}  &       150.7\sym{**} &       156.4\sym{**} &       64.00         &       12.75         &       94.00\sym{*}  &       93.00         \\
                    &     (38.82)         &     (95.96)         &     (74.96)         &     (87.35)         &     (73.67)         &     (84.68)         &     (60.70)         &     (72.40)         &        (35)         &     (34.13)         &     (49.55)         &     (59.53)         \\
 Cum. ratio          &       3.416\sym{*}  &      -0.837         &      -1.422         &      -2.347         &       5.337\sym{**} &       1.227         &      -2.639         &      -3.676         &       1.442         &      -2.843         &      -0.536         &      -1.338         \\
                    &     (1.613)         &     (2.978)         &     (2.034)         &     (2.317)         &     (1.989)         &     (3.996)         &     (3.356)         &     (3.994)         &     (3.162)         &     (2.449)         &     (2.012)         &     (2.321)         \\
 
\bottomrule \end{tabular} } \begin{tablenotes} \item \scriptsize \emph{Notes:} Clustered standard errors in parentheses (MxY). All regressions contain Birthmonth FE. Ratios indicate cases per thousand; original number of births. \end{tablenotes} \end{threeparttable} \end{table} 

\end{landscape}
\begin{landscape}
 \begin{table}[H] \begin{threeparttable} \centering \caption{Cummulative effects for upt to different points of age - BOOTSTRAPPED} {\def\sym#1{\ifmmode^{#1}\else\(^{#1}\)\fi} \begin{tabular}{l*{13}{c}} \toprule & \multicolumn{12}{c}{Dependent variable: \textbf{Symptoms, signs and abnormal clinical and laboratory findings, not elsewhere classified}} \\ \cmidrule(lr){2-13}
            &\multicolumn{4}{c}{Average Causal Effects}         &\multicolumn{4}{c}{Women}                          &\multicolumn{4}{c}{Men}                            \\\cmidrule(lr){2-5}\cmidrule(lr){6-9}\cmidrule(lr){10-13}
            &\multicolumn{1}{c}{(1)}&\multicolumn{1}{c}{(2)}&\multicolumn{1}{c}{(3)}&\multicolumn{1}{c}{(4)}&\multicolumn{1}{c}{(5)}&\multicolumn{1}{c}{(6)}&\multicolumn{1}{c}{(7)}&\multicolumn{1}{c}{(8)}&\multicolumn{1}{c}{(9)}&\multicolumn{1}{c}{(10)}&\multicolumn{1}{c}{(11)}&\multicolumn{1}{c}{(12)}\\
            &\multicolumn{1}{c}{2M}&\multicolumn{1}{c}{4M}&\multicolumn{1}{c}{6M}&\multicolumn{1}{c}{Donut}&\multicolumn{1}{c}{2M}&\multicolumn{1}{c}{4M}&\multicolumn{1}{c}{6M}&\multicolumn{1}{c}{Donut}&\multicolumn{1}{c}{2M}&\multicolumn{1}{c}{4M}&\multicolumn{1}{c}{6M}&\multicolumn{1}{c}{Donut}\\
\midrule
 \multicolumn{13}{l}{\emph{Panel A. 2 Up to the age of 21}} \\ Cum. numbers        &          73\sym{*}  &       60.25         &       62.33         &       52.20         &       52.50         &          65\sym{*}  &       49.00         &       38.40         &       20.50         &      -4.750         &       13.33         &       13.80         \\
                    &     (37.87)         &     (59.45)         &     (56.87)         &     (61.80)         &     (32.63)         &     (36.89)         &     (35.01)         &     (38.70)         &     (41.91)         &     (49.69)         &     (38.02)         &     (43.14)         \\
 Cum. ratio          &       0.957         &      -0.194         &      -0.676         &      -1.167         &       1.290         &       0.692         &      -0.939         &      -1.766         &       0.578         &      -1.055         &      -0.550         &      -0.718         \\
                    &     (1.350)         &     (2.103)         &     (1.620)         &     (1.867)         &     (1.205)         &     (1.836)         &     (1.741)         &     (1.935)         &     (2.099)         &     (2.661)         &     (1.896)         &     (2.219)         \\
 \midrule\multicolumn{13}{l}{\emph{Panel B. Up to the age of 26}} \\ Cum. numbers        &       161.5\sym{***}&       130.8         &       151.7\sym{*}  &       154.8         &       111.5         &       114.7         &       110.2\sym{*}  &       116.4         &       50.00         &       16.00         &       41.50         &       38.40         \\
                    &     (28.28)         &     (100.8)         &     (89.05)         &     (105.6)         &     (74.60)         &     (81.42)         &     (64.27)         &     (82.69)         &     (48.48)         &     (55.42)         &     (47.34)         &     (50.83)         \\
 Cum. ratio          &       2.390\sym{***}&       0.326         &      -0.116         &      -0.559         &       3.179\sym{*}  &       1.538         &      -0.242         &      -0.660         &       1.529         &      -0.866         &      -0.205         &      -0.661         \\
                    &     (0.677)         &     (2.848)         &     (2.295)         &     (2.681)         &     (1.780)         &     (3.138)         &     (2.789)         &     (3.473)         &     (2.725)         &     (3.230)         &     (2.434)         &     (2.684)         \\
 \midrule\multicolumn{13}{l}{\emph{Panel C. Up to the age of 31}} \\ Cum. numbers        &       181.5\sym{***}&       147.5         &       187.8\sym{*}  &       194.2         &       109.5         &       127.8         &       112.3         &       116.2         &       72.00         &       19.75         &       75.50         &       78.00         \\
                    &     (22.76)         &     (118.5)         &     (104.1)         &     (121.0)         &     (75.26)         &     (115.2)         &     (93.49)         &     (120.3)         &     (58.20)         &     (47.00)         &     (52.32)         &     (63.65)         \\
 Cum. ratio          &       2.395\sym{*}  &      -0.414         &      -0.934         &      -1.569         &       2.523\sym{**} &       0.742         &      -2.220         &      -3.051         &       2.143         &      -1.557         &      0.0375         &      -0.396         \\
                    &     (1.315)         &     (3.449)         &     (2.640)         &     (2.983)         &     (1.206)         &     (4.373)         &     (3.864)         &     (4.669)         &     (3.565)         &     (3.608)         &     (2.809)         &     (3.444)         \\
 \midrule\multicolumn{13}{l}{\emph{Panel D. Up to the age of 34}} \\ Cum. numbers        &       254.5\sym{***}&       184.2         &       244.7\sym{**} &       249.4\sym{*}  &       190.5\sym{***}&       171.5         &       150.7         &       156.4         &       64.00\sym{*}  &       12.75         &       94.00         &       93.00         \\
                    &     (36.85)         &     (129.9)         &     (107.0)         &     (128.1)         &     (69.39)         &     (115.1)         &     (91.82)         &     (120.1)         &     (32.67)         &     (43.07)         &     (59.04)         &     (68.63)         \\
 Cum. ratio          &       3.416\sym{**} &      -0.837         &      -1.422         &      -2.347         &       5.337\sym{***}&       1.227         &      -2.639         &      -3.676         &       1.442         &      -2.843         &      -0.536         &      -1.338         \\
                    &     (1.591)         &     (3.782)         &     (2.901)         &     (3.502)         &     (1.713)         &     (5.109)         &     (4.628)         &     (5.739)         &     (3.063)         &     (3.109)         &     (2.683)         &     (3.221)         \\
 
\bottomrule \end{tabular} } \begin{tablenotes} \item \scriptsize \emph{Notes:} \textbf{BOOTSTRAPPED} standard errors in parentheses (MxY), with 400 replications. All regressions contain Birthmonth FE. Ratios indicate cases per thousand; original number of births. \end{tablenotes} \end{threeparttable} \end{table} 

\end{landscape}
%---------------------------------
\newpage
FEBRUAR CASES:
 \begin{table}[H] \begin{threeparttable} \centering \caption{Dep. variable: \textbf{Symptoms, signs and abnormal clinical and laboratory findings, not elsewhere classified}} {\def\sym#1{\ifmmode^{#1}\else\(^{#1}\)\fi} \begin{tabular}{l*{13}{c}} \toprule year & \multicolumn{12}{c}{Month of birth} \\ \cmidrule(lr){2-13} 
            &          11&          12&           1&           2&           3&           4&           5&           6&           7&           8&           9&          10\\
1995        &         261&         257&         241&         242&         318&         281&         284&         256&         261&         266&         285&         250\\
1996        &         303&         312&         330&         339&         360&         323&         335&         331&         308&         290&         324&         324\\
1997        &         334&         324&         327&         336&         381&         347&         382&         355&         393&         380&         374&         335\\
1998        &         321&         332&         369&         313&         355&         327&         319&         348&         344&         351&         339&         326\\
1999        &         260&         289&         298&         312&         329&         331&         321&         332&         323&         328&         330&         325\\
2000        &         250&         238&         262&         260&         340&         259&         268&         285&         305&         285&         306&         272\\
2001        &         268&         289&         251&         258&         339&         275&         294&         303&         303&         263&         289&         272\\
2002        &         291&         265&         307&         253&         325&         294&         317&         300&         287&         284&         295&         307\\
2003        &         266&         238&         253&         241&         283&         251&         268&         286&         260&         281&         238&         257\\
2004        &         236&         265&         242&         240&         264&         256&         287&         237&         235&         239&         244&         221\\
2005        &         228&         257&         246&         243&         277&         280&         237&         252&         235&         266&         270&         230\\
2006        &         289&         274&         268&         282&         276&         273&         287&         282&         296&         276&         273&         255\\
2007        &         287&         272&         280&         303&         314&         289&         316&         277&         285&         303&         287&         299\\
2008        &         314&         282&         320&         311&         360&         290&         355&         351&         343&         315&         300&         356\\
2009        &         319&         299&         308&         315&         333&         316&         385&         324&         317&         342&         319&         317\\
2010        &         330&         302&         352&         306&         371&         332&         362&         327&         337&         351&         343&         349\\
2011        &         371&         368&         353&         373&         402&         388&         380&         344&         357&         374&         368&         356\\
2012        &         360&         382&         384&         399&         419&         396&         465&         411&         445&         423&         392&         408\\
2013        &         394&         403&         412&         437&         467&         463&         441&         441&         455&         447&         430&         435\\
2014        &         422&         429&         480&         460&         469&         451&         500&         495&         478&         476&         487&         479\\
 \bottomrule \end{tabular} } \begin{tablenotes} \item \scriptsize \emph{Notes:} Number of cases per year and MOB in treatment cohort. \end{tablenotes} \end{threeparttable} \end{table} 

 \begin{table}[H] \begin{threeparttable} \centering \caption{Dep. variable: \textbf{Symptoms, signs and abnormal clinical and laboratory findings, not elsewhere classified}} {\def\sym#1{\ifmmode^{#1}\else\(^{#1}\)\fi} \begin{tabular}{l*{13}{c}} \toprule year & \multicolumn{12}{c}{Month of birth} \\ \cmidrule(lr){2-13} 
            &          11&          12&           1&           2&           3&           4&           5&           6&           7&           8&           9&          10\\
1995        &           5&          28&          21&          13&          76&          43&          23&         -18&           2&          11&          15&         -16\\
1996        &          30&          16&          31&          37&          30&          36&          19&          14&         -40&         -41&          14&          36\\
1997        &          21&         -13&         -25&         -91&         -77&          19&          11&           0&          37&          35&          46&         -19\\
1998        &          -4&          11&          31&         -21&         -26&         -28&         -50&         -17&          -2&          -2&         -15&         -78\\
1999        &         -42&          -6&          22&         -41&           1&          -9&         -37&         -22&         -26&         -10&          17&         -23\\
2000        &         -11&          -1&           1&          -3&          37&         -20&         -29&          -4&          -1&         -15&          29&         -15\\
2001        &           0&          17&         -18&         -20&          66&         -14&           2&          25&         -27&         -37&         -10&         -47\\
2002        &          27&         -29&          26&         -47&          14&          -4&          13&          -8&         -22&          -1&         -11&         -15\\
2003        &          32&         -14&          13&         -13&           1&           4&           1&          34&           0&           8&         -44&           2\\
2004        &           3&          67&           2&          28&          22&          50&          13&         -44&         -40&         -33&          -4&         -26\\
2005        &           5&          24&         -17&         -22&          22&          33&         -35&          -6&         -48&         -29&          -1&         -41\\
2006        &          29&         -14&          -2&           5&         -36&         -17&         -29&         -15&         -16&         -20&         -22&         -58\\
2007        &          19&          30&          -5&          28&          33&          -1&          -2&          -6&         -19&          24&          10&         -21\\
2008        &          10&         -22&         -13&          10&          26&          -4&          35&          37&          45&         -13&         -51&          40\\
2009        &          -7&         -63&          -5&         -23&         -16&          10&          31&         -22&         -29&         -33&         -17&         -21\\
2010        &           7&         -46&           9&         -68&          18&           5&          11&          -8&         -15&           3&          -8&         -14\\
2011        &          36&          17&         -40&          16&           8&          24&          15&         -35&         -37&           1&         -11&         -10\\
2012        &         -28&          18&         -38&          23&           8&         -14&           2&          -5&          34&          -1&         -58&          17\\
2013        &          13&          22&          10&          14&          13&          11&         -59&          -3&          17&           9&           7&         -16\\
2014        &         -32&          22&          -6&         -10&          -2&          -1&          16&          59&         -36&         -35&          30&         -30\\
 \bottomrule \end{tabular} } \begin{tablenotes} \item \scriptsize \emph{Notes:} Difference of cases (control - treatment) per year and MOB in treatment cohort. \end{tablenotes} \end{threeparttable} \end{table} 

