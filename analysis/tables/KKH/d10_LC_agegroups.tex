 \begin{table}[H] \centering \begin{threeparttable} \caption{Life-course approach - Table format} {\def\sym#1{\ifmmode^{#1}\else\(^{#1}\)\fi} \begin{tabular}{l*{5}{c}} \toprule \multicolumn{5}{l}{Dep. variable: \textbf{Diseases of the digestive system}} \\ & \multicolumn{4}{c}{Estimation window} \\ \cmidrule(lr){2-5}
            &\multicolumn{1}{c}{(1)}&\multicolumn{1}{c}{(2)}&\multicolumn{1}{c}{(3)}&\multicolumn{1}{c}{(4)}\\
            &\multicolumn{1}{c}{Age 17-21}&\multicolumn{1}{c}{Age 22-26}&\multicolumn{1}{c}{Age 27-31}&\multicolumn{1}{c}{Age 32-35}\\
\midrule
 \multicolumn{5}{l}{\emph{Panel A. Average causal effects}} \\ Abs. numbers        &       20.13         &       8.933         &       14.97         &       15.50         \\
                    &     (16.05)         &     (22.72)         &     (27.33)         &     (23.90)         \\
 Ratio fertility     &      -0.421         &      -0.485         &      -0.376         &      -0.494         \\
                    &     (0.337)         &     (0.524)         &     (0.465)         &     (0.646)         \\
 Ratio population    &      -0.351         &      -0.248         &      -0.337         \\
                    &     (0.436)         &     (0.357)         &     (0.505)         \\
 Cum. numbers        &       107.8\sym{**} &       197.0\sym{**} &       260.4\sym{*}  &       307.2         \\
                    &     (54.80)         &     (95.98)         &     (140.2)         &     (194.2)         \\
 Cum. ratio          &      -1.795         &      -3.671         &      -5.630\sym{*}  &      -8.059\sym{*}  \\
                    &     (1.274)         &     (2.405)         &     (3.169)         &     (4.652)         \\
 \midrule\multicolumn{5}{l}{\emph{Panel B. Treatment effect heterogeneity - Women}} \\ Abs. numbers        &       8.667         &       6.133         &      -5.567         &       4.000         \\
                    &     (12.73)         &     (16.49)         &     (11.45)         &     (16.62)         \\
 Ratio fertility     &      -0.796         &      -0.592         &      -1.069\sym{**} &      -0.734         \\
                    &     (0.517)         &     (0.825)         &     (0.470)         &     (0.790)         \\
 Ratio population    &      -0.236         &      -0.753\sym{**} &      -0.497         \\
                    &     (0.790)         &     (0.333)         &     (0.588)         \\
 Cum. numbers        &       44.03         &       94.07         &       103.7         &       86.04         \\
                    &     (39.37)         &     (72.79)         &     (78.81)         &     (92.31)         \\
 Cum. ratio          &      -3.883\sym{**} &      -6.630\sym{*}  &      -10.30\sym{**} &      -14.94\sym{**} \\
                    &     (1.683)         &     (3.384)         &     (5.101)         &     (6.011)         \\
 \midrule\multicolumn{5}{l}{\emph{Panel C. Treatment effect heterogeneity - Men}} \\ Abs. numbers        &       11.47\sym{**} &       2.800         &       20.53         &       11.50         \\
                    &     (5.841)         &     (10.75)         &     (21.07)         &     (13.11)         \\
 Ratio fertility     &      -0.110         &      -0.406         &       0.273         &      -0.266         \\
                    &     (0.312)         &     (0.426)         &     (0.773)         &     (0.675)         \\
 Ratio population    &      -0.481         &       0.251         &      -0.176         \\
                    &     (0.376)         &     (0.616)         &     (0.550)         \\
 Cum. numbers        &       63.77\sym{**} &       103.0\sym{***}&       156.7\sym{*}  &       221.2         \\
                    &     (30.36)         &     (37.23)         &     (90.56)         &     (149.1)         \\
 Cum. ratio          &     -0.0185         &      -1.222         &      -1.625         &      -1.977         \\
                    &     (1.473)         &     (2.454)         &     (3.404)         &     (6.177)         \\
 
\bottomrule \end{tabular} } \begin{tablenotes} \item \scriptsize \emph{Notes:} Clustered standard errors in parentheses. All regression are run with CG2 (i.e. the cohort prior to the reform) and with month-of-birth FEs. Ratios indicate cases per thousand; either approximated population (with weights coming from the original fertility distribution) or original number of births. Raqtio population muss eins nach rechts gerückt werden \end{tablenotes} \end{threeparttable} \end{table} 
