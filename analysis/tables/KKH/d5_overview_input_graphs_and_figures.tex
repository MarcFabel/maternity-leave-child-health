%---------------------------------
% INPUT FOR VARIABLE: d5
%---------------------------------
\subsection{d5}
% RD overview
\begin{landscape}
\begin{figure}[H]
	\centering
	\begin{minipage}{.95\linewidth}
	\includegraphics[width=\linewidth]{rd_d5_overview_panel1}
	{\scriptsize \emph{Notes:} The figures show monthly RD plots with averages obtained from a bin width of one month. The solid vertical line divides pre- and post-reform regime. The averages are taken over the period of at most 1995-2014. \par}
\end{minipage}
\end{figure}
\end{landscape}
\begin{landscape}
\begin{figure}[H]
	\centering
\begin{minipage}{.95\linewidth}
	\includegraphics[width=\linewidth]{rd_d5_overview_panel2}
	{\scriptsize \emph{Notes:} The figures show monthly RD plots with a moving average window width of 3 months. The solid vertical line divides pre- and post-reform regime. The averages are taken over the period of at most 1995-2014. \par}
\end{minipage}
\end{figure}
\end{landscape}
%---------------------------------
% TABELLEN
 \begin{table}[H] \begin{threeparttable} \centering \caption{Dep. variable: \textbf{Mental and behavioral disorders}} {\def\sym#1{\ifmmode^{#1}\else\(^{#1}\)\fi} \begin{tabular}{l*{8}{c}} \toprule & \multicolumn{7}{c}{Estimation window} \\ \cmidrule(lr){2-8}
            &\multicolumn{1}{c}{(1)}&\multicolumn{1}{c}{(2)}&\multicolumn{1}{c}{(3)}&\multicolumn{1}{c}{(4)}&\multicolumn{1}{c}{(5)}&\multicolumn{1}{c}{(6)}&\multicolumn{1}{c}{(7)}\\
            &\multicolumn{1}{c}{1M}&\multicolumn{1}{c}{2M}&\multicolumn{1}{c}{3M}&\multicolumn{1}{c}{4M}&\multicolumn{1}{c}{5M}&\multicolumn{1}{c}{6M}&\multicolumn{1}{c}{Donut}\\
\midrule
 \multicolumn{8}{l}{\emph{Panel A. Average causal effects}} \\ Abs. numbers        &       17.10\sym{***}&       15.28         &      -5.733         &      -12.25         &      -5.230         &       10.22         &       8.840         \\
                    &  (3.15e-13)         &     (9.196)         &     (11.97)         &     (9.553)         &     (8.827)         &     (10.47)         &     (11.62)         \\
 Ratio fertility     &       0.243\sym{***}&      0.0829         &      -0.656         &      -0.852\sym{**} &      -0.756\sym{**} &      -0.634\sym{**} &      -0.809\sym{***}\\
                    &  (1.86e-15)         &     (0.353)         &     (0.402)         &     (0.340)         &     (0.274)         &     (0.244)         &     (0.267)         \\
 Ratio population    &      -0.692\sym{***}&      -0.551\sym{*}  &      -1.047\sym{***}&      -1.246\sym{***}&      -1.059\sym{***}&      -0.832\sym{***}&      -0.860\sym{***}\\
                    &  (2.54e-15)         &     (0.287)         &     (0.293)         &     (0.259)         &     (0.224)         &     (0.234)         &     (0.277)         \\
 Ratio fert(03-14)   &      -0.477\sym{***}&      -0.267         &      -0.701\sym{**} &      -0.933\sym{***}&      -0.782\sym{***}&      -0.556\sym{**} &      -0.572\sym{**} \\
                    &  (1.41e-15)         &     (0.203)         &     (0.233)         &     (0.225)         &     (0.193)         &     (0.208)         &     (0.250)         \\
 Cum. numbers        &       528.8\sym{***}&       419.5\sym{***}&       151.4         &       100.9         &       133.1         &       245.3\sym{**} &       188.6\sym{*}  \\
                    &  (5.96e-12)         &     (109.8)         &     (142.1)         &     (108.3)         &     (93.93)         &     (94.27)         &     (95.62)         \\
 Cum. ratio          &       9.879\sym{***}&       6.461         &      -1.640         &      -3.221         &      -2.951         &      -2.415         &      -4.873\sym{*}  \\
                    &  (1.04e-13)         &     (3.778)         &     (4.322)         &     (3.555)         &     (2.832)         &     (2.380)         &     (2.360)         \\
 \midrule\multicolumn{8}{l}{\emph{Panel B. Treatment effect heterogeneity - Women}} \\ Abs. numbers        &       6.250\sym{***}&       13.23         &       7.067         &       8.762         &       13.16\sym{**} &       20.18\sym{***}&       22.97\sym{***}\\
                    &  (7.95e-14)         &     (10.55)         &     (8.009)         &     (5.990)         &     (5.221)         &     (5.566)         &     (5.369)         \\
 Ratio fertility     &      0.0796\sym{***}&       0.333         &      -0.192         &      -0.163         &     -0.0424         &      0.0599         &      0.0560         \\
                    &  (2.31e-15)         &     (0.558)         &     (0.436)         &     (0.350)         &     (0.284)         &     (0.260)         &     (0.289)         \\
 Ratio population    &      -0.831\sym{***}&      0.0589         &      -0.122         &      -0.199         &     0.00203         &       0.147         &       0.342         \\
                    &  (1.58e-15)         &     (0.511)         &     (0.348)         &     (0.267)         &     (0.234)         &     (0.237)         &     (0.250)         \\
 Ratio fert(03-14)   &      -0.796\sym{***}&      0.0584         &      -0.115         &      -0.188         &      0.0123         &       0.156         &       0.347         \\
                    &  (1.41e-15)         &     (0.489)         &     (0.333)         &     (0.256)         &     (0.226)         &     (0.230)         &     (0.244)         \\
 Cum. numbers        &       270.1\sym{***}&       223.3         &       107.9         &       139.0\sym{*}  &       159.2\sym{**} &       213.8\sym{***}&       202.6\sym{***}\\
                    &  (1.71e-12)         &     (121.3)         &     (98.89)         &     (75.22)         &     (60.76)         &     (56.67)         &     (52.97)         \\
 Cum. ratio          &       9.567\sym{***}&       7.275         &     -0.0810         &       0.865         &       1.130         &       1.575         &     -0.0233         \\
                    &  (3.67e-14)         &     (6.247)         &     (5.246)         &     (4.210)         &     (3.350)         &     (2.844)         &     (2.938)         \\
 \midrule\multicolumn{8}{l}{\emph{Panel C. Treatment effect heterogeneity - Men}} \\ Abs. numbers        &       10.85\sym{***}&       2.050         &      -12.80\sym{*}  &      -21.01\sym{***}&      -18.39\sym{***}&      -9.967         &      -14.13\sym{*}  \\
                    &  (8.50e-14)         &     (3.535)         &     (7.070)         &     (6.783)         &     (6.044)         &     (6.428)         &     (7.281)         \\
 Ratio fertility     &       0.434\sym{***}&      -0.140         &      -1.086\sym{**} &      -1.498\sym{***}&      -1.415\sym{***}&      -1.267\sym{***}&      -1.607\sym{***}\\
                    &  (6.65e-15)         &     (0.273)         &     (0.455)         &     (0.413)         &     (0.331)         &     (0.286)         &     (0.283)         \\
 Ratio population    &      -0.170\sym{***}&      -0.647\sym{**} &      -1.395\sym{***}&      -1.826\sym{***}&      -1.710\sym{***}&      -1.364\sym{***}&      -1.603\sym{***}\\
                    &  (1.75e-15)         &     (0.189)         &     (0.349)         &     (0.364)         &     (0.295)         &     (0.298)         &     (0.330)         \\
 Ratio fert(03-14)   &      -0.161\sym{***}&      -0.589\sym{**} &      -1.265\sym{***}&      -1.656\sym{***}&      -1.545\sym{***}&      -1.228\sym{***}&      -1.442\sym{***}\\
                    &  (1.50e-15)         &     (0.169)         &     (0.316)         &     (0.330)         &     (0.267)         &     (0.272)         &     (0.301)         \\
 Cum. numbers        &       258.6\sym{***}&       196.1\sym{***}&       43.53         &      -38.11         &      -26.10         &       31.50         &      -13.93         \\
                    &  (1.95e-12)         &     (30.37)         &     (69.41)         &     (66.38)         &     (60.36)         &     (57.13)         &     (63.10)         \\
 Cum. ratio          &       10.44\sym{***}&       5.780\sym{**} &      -3.035         &      -7.041\sym{*}  &      -6.707\sym{**} &      -6.036\sym{**} &      -9.332\sym{***}\\
                    &  (1.54e-13)         &     (2.082)         &     (4.066)         &     (3.735)         &     (3.007)         &     (2.514)         &     (2.369)         \\
 
\bottomrule \end{tabular} } \begin{tablenotes} \item \scriptsize \emph{Notes:} Clustered standard errors in parentheses. All regression are run with CG2 (i.e. the cohort prior to the reform) and with month-of-birth FEs. Ratios indicate cases per thousand; either approximated population (with weights coming from the original fertility distribution) or original number of births. \end{tablenotes} \end{threeparttable} \end{table} 

 \begin{table}[H] \begin{threeparttable} \centering \caption{Robustness with respect to the choice of \texttt{control group}} {\def\sym#1{\ifmmode^{#1}\else\(^{#1}\)\fi} \begin{tabular}{l*{10}{c}} \toprule & \multicolumn{9}{c}{Dependent variable: \textbf{Mental and behavioral disorders}} \\ \cmidrule(lr){2-10}
            &\multicolumn{3}{c}{Average Causal Effects}&\multicolumn{3}{c}{Women}             &\multicolumn{3}{c}{Men}               \\\cmidrule(lr){2-4}\cmidrule(lr){5-7}\cmidrule(lr){8-10}
            &\multicolumn{1}{c}{(1)}&\multicolumn{1}{c}{(2)}&\multicolumn{1}{c}{(3)}&\multicolumn{1}{c}{(4)}&\multicolumn{1}{c}{(5)}&\multicolumn{1}{c}{(6)}&\multicolumn{1}{c}{(7)}&\multicolumn{1}{c}{(8)}&\multicolumn{1}{c}{(9)}\\
            &\multicolumn{1}{c}{C2}&\multicolumn{1}{c}{C1+C2}&\multicolumn{1}{c}{C1-C3}&\multicolumn{1}{c}{C2}&\multicolumn{1}{c}{C1+C2}&\multicolumn{1}{c}{C1-C3}&\multicolumn{1}{c}{C2}&\multicolumn{1}{c}{C1+C2}&\multicolumn{1}{c}{C1-C3}\\
\midrule
 \multicolumn{10}{l}{\emph{Panel A. 2 Month bandwidth}} \\ Abs. numbers        &      -36.06\sym{***}&      -34.22\sym{**} &      -44.50\sym{**} &      -6.944         &       3.222         &      -10.22         &      -29.11\sym{**} &      -37.44\sym{**} &      -34.28\sym{**} \\
                    &     (6.516)         &     (13.75)         &     (19.79)         &     (16.38)         &     (15.20)         &     (21.13)         &     (10.14)         &     (12.58)         &     (13.81)         \\
 Ratio fertility     &      0.0829         &       0.138         &      0.0205         &       0.333         &       0.633         &       0.387         &      -0.140         &      -0.312         &      -0.312         \\
                    &     (0.353)         &     (0.509)         &     (0.478)         &     (0.558)         &     (0.554)         &     (0.608)         &     (0.273)         &     (0.628)         &     (0.665)         \\
 Ratio population    &      -0.286         &      -0.181         &      -0.375         &      0.0589         &       0.423         &      0.0468         &      -0.647\sym{**} &      -0.801\sym{*}  &      -0.808\sym{**} \\
                    &     (0.219)         &     (0.351)         &     (0.318)         &     (0.511)         &     (0.529)         &     (0.619)         &     (0.189)         &     (0.424)         &     (0.339)         \\
 \midrule\multicolumn{10}{l}{\emph{Panel B. 4 Month bandwidth}} \\ Abs. numbers        &      -53.50\sym{***}&      -31.22\sym{*}  &      -27.82         &      -4.972         &       10.75         &       4.787         &      -48.53\sym{***}&      -41.97\sym{***}&      -32.61\sym{***}\\
                    &     (9.326)         &     (16.87)         &     (19.68)         &     (9.401)         &     (11.79)         &     (15.46)         &     (7.820)         &     (9.695)         &     (9.696)         \\
 Ratio fertility     &      -0.852\sym{**} &      -0.490         &      -0.291         &      -0.163         &       0.440         &       0.467         &      -1.498\sym{***}&      -1.368\sym{***}&      -1.005\sym{*}  \\
                    &     (0.340)         &     (0.368)         &     (0.369)         &     (0.350)         &     (0.382)         &     (0.389)         &     (0.413)         &     (0.455)         &     (0.529)         \\
 Ratio population    &      -1.001\sym{***}&      -0.533\sym{*}  &      -0.453\sym{*}  &      -0.199         &       0.374         &       0.269         &      -1.826\sym{***}&      -1.471\sym{***}&      -1.199\sym{***}\\
                    &     (0.242)         &     (0.282)         &     (0.237)         &     (0.267)         &     (0.330)         &     (0.362)         &     (0.364)         &     (0.359)         &     (0.309)         \\
 \midrule\multicolumn{10}{l}{\emph{Panel C. 6 Month bandwidth}} \\ Abs. numbers        &      -16.11         &      -10.21         &      -9.667         &       10.19         &       13.32         &       5.049         &      -26.30\sym{**} &      -23.54\sym{**} &      -14.72         \\
                    &     (14.92)         &     (15.27)         &     (17.02)         &     (8.336)         &     (9.111)         &     (11.38)         &     (9.967)         &     (9.501)         &     (10.10)         \\
 Ratio fertility     &      -0.634\sym{**} &      -0.553\sym{**} &      -0.281         &      0.0599         &       0.289         &       0.307         &      -1.267\sym{***}&      -1.323\sym{***}&      -0.811\sym{*}  \\
                    &     (0.244)         &     (0.266)         &     (0.272)         &     (0.260)         &     (0.283)         &     (0.278)         &     (0.286)         &     (0.318)         &     (0.409)         \\
 Ratio population    &      -0.609\sym{**} &      -0.400\sym{*}  &      -0.297         &       0.147         &       0.362         &       0.230         &      -1.364\sym{***}&      -1.170\sym{***}&      -0.830\sym{***}\\
                    &     (0.220)         &     (0.222)         &     (0.196)         &     (0.237)         &     (0.253)         &     (0.265)         &     (0.298)         &     (0.282)         &     (0.276)         \\
 \midrule\multicolumn{10}{l}{\emph{Panel D. Donut specification}} \\ Abs. numbers        &       8.840         &       9.305         &       10.05         &       22.97\sym{***}&       26.81\sym{***}&       20.93\sym{***}&      -14.13\sym{*}  &      -17.51\sym{**} &      -10.88         \\
                    &     (11.62)         &     (11.15)         &     (10.19)         &     (5.369)         &     (5.717)         &     (6.919)         &     (7.281)         &     (7.014)         &     (6.737)         \\
 Ratio population    &      -0.933\sym{**} &      -0.826\sym{*}  &      -0.623         &     0.00732         &      0.0376         &      -0.234         &      -1.850\sym{*}  &      -1.682\sym{*}  &      -0.930         \\
                    &     (0.435)         &     (0.417)         &     (0.377)         &     (0.426)         &     (0.466)         &     (0.503)         &     (1.005)         &     (0.895)         &     (0.869)         \\
 Ratio fertility     &      -0.685\sym{***}&      -0.380         &      -0.205         &       0.171         &       0.385         &       0.249         &      -1.494\sym{***}&      -1.107\sym{***}&      -0.639\sym{*}  \\
                    &     (0.233)         &     (0.251)         &     (0.230)         &     (0.235)         &     (0.284)         &     (0.251)         &     (0.299)         &     (0.335)         &     (0.352)         \\
 
\bottomrule \end{tabular} } \begin{tablenotes} \item \scriptsize \emph{Notes:} Clustered standard errors in parentheses. All regressions contain Birthmonth FE. Ratios indicate cases per thousand; either approximated population or original number of births. \end{tablenotes} \end{threeparttable} \end{table} 

%---------------------------------
% Life-course figure (Panel1)
\begin{landscape}
\begin{figure}[H]
\centering
\begin{minipage}{.9\linewidth}
\includegraphics[width=\linewidth]{lc_d5_overview_panel1}
{\scriptsize \emph{Notes:} The figures depict DDRD estimates and 90\% confidence intervals over the life-course. The years are harmonized such that the cohorts are in the same age when they are compared. All regressions are carried out with month-of-birth FE and make use of clustered standard errors. Furthermore, we used a bandwidth of half a year and only the control cohort that was born one year prior to the reform. Ratios indicate cases per thousand; using in the denominator the approximated population (with weights coming from the original fertility distribution) or original number of births. \par}
\end{minipage}
\end{figure}
\end{landscape}
%---------------------------------
% Life-course figure (Panel2)
\begin{landscape}
\begin{figure}[H]
\centering
\begin{minipage}{.9\linewidth}
\includegraphics[width=\linewidth]{lc_d5_overview_panel2}
{\scriptsize \emph{Notes:} The figures depict DDRD estimates and 90\% confidence intervals over the life-course. The years are harmonized such that the cohorts are in the same age when they are compared. All regressions are carried out with month-of-birth FE and make use of clustered standard errors. Furthermore, we used a bandwidth of half a year. Ratios indicate cases per thousand; using in the denominator the approximated population (with weights coming from the original fertility distribution) or original number of births. \par}
\end{minipage}
\end{figure}
\end{landscape}
%---------------------------------
% Life-course (panel 3 - 6)
\begin{figure}[H]%\vspace*{-2cm}
	\centering
	\includegraphics[width=.9\linewidth]{lc_d5_overview_panel3}
	\includegraphics[width=.9\linewidth]{lc_d5_overview_panel4}
\end{figure}
\begin{figure}[H]
	\centering	
	\includegraphics[width=.97\linewidth]{lc_d5_overview_panel5}
	\includegraphics[width=.97\linewidth]{lc_d5_overview_panel6}
\end{figure}
% Life-course TABLE Format
 \begin{table}[H] \centering \begin{threeparttable} \caption{Life-course approach - Table format} {\def\sym#1{\ifmmode^{#1}\else\(^{#1}\)\fi} \begin{tabular}{l*{5}{c}} \toprule \multicolumn{5}{l}{Dep. variable: \textbf{Mental and behavioral disorders}} \\ & \multicolumn{4}{c}{Estimation window} \\ \cmidrule(lr){2-5}
            &\multicolumn{1}{c}{(1)}&\multicolumn{1}{c}{(2)}&\multicolumn{1}{c}{(3)}&\multicolumn{1}{c}{(4)}\\
            &\multicolumn{1}{c}{Age 17-21}&\multicolumn{1}{c}{Age 22-26}&\multicolumn{1}{c}{Age 27-31}&\multicolumn{1}{c}{Age 32-35}\\
\midrule
 \multicolumn{5}{l}{\emph{Panel A. Average causal effects}} \\ Abs. numbers        &       38.17\sym{*}  &       41.83         &       0.133         &      -35.79         \\
                    &     (22.26)         &     (32.82)         &     (31.26)         &     (37.63)         \\
 Ratio fertility     &       0.174         &    -0.00769         &      -1.000         &      -1.906\sym{***}\\
                    &     (0.527)         &     (0.772)         &     (0.708)         &     (0.712)         \\
 Ratio population    &      -0.153         &      -0.708         &      -1.399\sym{**} \\
                    &     (0.636)         &     (0.556)         &     (0.572)         \\
 Cum. numbers        &       115.1\sym{*}  &       340.7\sym{**} &       418.8\sym{*}  &       326.2         \\
                    &     (68.65)         &     (149.5)         &     (245.3)         &     (349.2)         \\
 Cum. ratio          &       0.179         &       0.820         &      -2.277         &      -9.097         \\
                    &     (1.785)         &     (4.105)         &     (6.548)         &     (8.627)         \\
 \midrule\multicolumn{5}{l}{\emph{Panel B. Treatment effect heterogeneity - Women}} \\ Abs. numbers        &       24.50\sym{*}  &       24.07         &       11.00         &       19.96         \\
                    &     (14.62)         &     (18.85)         &     (19.42)         &     (16.40)         \\
 Ratio fertility     &       0.388         &       0.205         &      -0.426         &      -0.163         \\
                    &     (0.671)         &     (0.910)         &     (0.911)         &     (0.714)         \\
 Ratio population    &      -0.112         &      -0.270         &     -0.0690         \\
                    &     (0.773)         &     (0.690)         &     (0.563)         \\
 Cum. numbers        &       94.27\sym{*}  &       232.1\sym{**} &       297.8\sym{*}  &       337.8         \\
                    &     (54.13)         &     (112.9)         &     (170.2)         &     (207.9)         \\
 Cum. ratio          &       1.448         &       3.449         &       1.956         &      -0.613         \\
                    &     (2.739)         &     (5.449)         &     (8.843)         &     (10.86)         \\
 \midrule\multicolumn{5}{l}{\emph{Panel C. Treatment effect heterogeneity - Men}} \\ Abs. numbers        &       13.67         &       17.77         &      -10.87         &      -55.75\sym{*}  \\
                    &     (11.34)         &     (23.16)         &     (22.17)         &     (29.10)         \\
 Ratio fertility     &     -0.0319         &      -0.180         &      -1.504         &      -3.518\sym{***}\\
                    &     (0.497)         &     (0.909)         &     (0.937)         &     (1.017)         \\
 Ratio population    &      -0.164         &      -1.116         &      -2.712\sym{***}\\
                    &     (0.893)         &     (0.744)         &     (0.817)         \\
 Cum. numbers        &       20.80         &       108.6         &       121.0         &      -11.58         \\
                    &     (30.12)         &     (80.62)         &     (156.0)         &     (239.2)         \\
 Cum. ratio          &      -1.054         &      -1.614         &      -6.046         &      -16.71\sym{*}  \\
                    &     (1.254)         &     (3.846)         &     (6.504)         &     (9.228)         \\
 
\bottomrule \end{tabular} } \begin{tablenotes} \item \scriptsize \emph{Notes:} Clustered standard errors in parentheses. All regression are run with CG2 (i.e. the cohort prior to the reform) and with month-of-birth FEs. Ratios indicate cases per thousand; either approximated population (with weights coming from the original fertility distribution) or original number of births. Raqtio population muss eins nach rechts gerückt werden \end{tablenotes} \end{threeparttable} \end{table} 

%---------------------------------
% PLACEBO EXERCISES
\newpage
\begin{landscape}
\begin{figure}[H]
	\centering
    \begin{minipage}{.9\linewidth}
	\includegraphics[width=\linewidth]{placebo_graph_d5.pdf}
    {\scriptsize \emph{Notes:} The figures depict DDRD estimates and 95\% confidence intervals when the treatment cohort is shifted over time. The date on the abscissa indicates the starting date of the treated.  All regressions are carried out with month-of-birth FE and make use of clustered standard errors. Furthermore, we used a bandwidth of half a year. Ratios indicate cases per thousand; using in the denominator the approximated population (with weights coming from the original fertility distribution) or original number of births. \par}
    \end{minipage}
\end{figure}
\end{landscape}
 \begin{table}[H] \centering \begin{threeparttable} \caption{Placebo 1 (CONTROL1 ist TREAT) } {\def\sym#1{\ifmmode^{#1}\else\(^{#1}\)\fi} \begin{tabular}{l*{4}{c}} \toprule \multicolumn{4}{l}{Dep. variable: \textbf{Mental and behavioral disorders}} \\ & \multicolumn{3}{c}{Choice of control group} \\ \cmidrule(lr){2-4}
            &\multicolumn{1}{c}{(1)}&\multicolumn{1}{c}{(2)}&\multicolumn{1}{c}{(3)}\\
            &\multicolumn{1}{c}{C2}&\multicolumn{1}{c}{C3}&\multicolumn{1}{c}{C2+C3}\\
\midrule
 \multicolumn{4}{l}{\emph{Panel A. Average causal effects}} \\ Abs. numbers        &       4.108         &      -0.650         &       1.729         \\
                    &     (13.27)         &     (8.986)         &     (11.42)         \\
 Ratio fertility     &      -0.162         &       0.736\sym{**} &       0.287         \\
                    &     (0.298)         &     (0.282)         &     (0.348)         \\
 Ratio population    &      -0.329         &       0.407\sym{*}  &      0.0390         \\
                    &     (0.270)         &     (0.230)         &     (0.259)         \\
 Cum. numbers        &       132.9         &       41.77         &       87.32         \\
                    &     (134.7)         &     (107.3)         &     (126.5)         \\
 Cum. ratio          &       0.397         &       6.968\sym{**} &       3.682         \\
                    &     (2.887)         &     (2.771)         &     (3.949)         \\
 \midrule\multicolumn{4}{l}{\emph{Panel B. Treatment effect heterogeneity - Women}} \\ Abs. numbers        &      -6.392         &      -20.86\sym{***}&      -13.63         \\
                    &     (7.757)         &     (4.542)         &     (8.429)         \\
 Ratio fertility     &      -0.457         &      -0.172         &      -0.315         \\
                    &     (0.362)         &     (0.211)         &     (0.296)         \\
 Ratio population    &      -0.355         &      -0.377\sym{*}  &      -0.366         \\
                    &     (0.295)         &     (0.191)         &     (0.260)         \\
 Cum. numbers        &      -56.85         &      -165.3\sym{***}&      -111.1         \\
                    &     (81.94)         &     (48.05)         &     (81.19)         \\
 Cum. ratio          &      -4.265         &      -0.641         &      -2.453         \\
                    &     (3.682)         &     (2.082)         &     (2.982)         \\
 \midrule\multicolumn{4}{l}{\emph{Panel C. Treatment effect heterogeneity - Men}} \\ Abs. numbers        &       10.50         &       20.21\sym{**} &       15.35\sym{*}  \\
                    &     (7.832)         &     (8.474)         &     (8.747)         \\
 Ratio fertility     &       0.112         &       1.591\sym{***}&       0.851         \\
                    &     (0.325)         &     (0.463)         &     (0.561)         \\
 Ratio population    &      -0.294         &       1.200\sym{***}&       0.453         \\
                    &     (0.320)         &     (0.398)         &     (0.398)         \\
 Cum. numbers        &       189.7\sym{**} &       207.1\sym{**} &       198.4         \\
                    &     (83.38)         &     (93.46)         &     (120.9)         \\
 Cum. ratio          &       4.749         &       14.12\sym{***}&       9.435         \\
                    &     (3.326)         &     (4.539)         &     (6.708)         \\
 
\bottomrule \end{tabular} } \begin{tablenotes} \item \scriptsize \emph{Notes:} Clustered standard errors in parentheses. All regression are run with month-of-birth FEs and control cohort 2 is assigned with the treatment status. All regressions are carried out with a window width of half a year. \end{tablenotes} \end{threeparttable} \end{table} 

 \begin{table}[H] \centering \begin{threeparttable} \caption{Placebo 2 (CONTROL2 ist TREAT) } {\def\sym#1{\ifmmode^{#1}\else\(^{#1}\)\fi} \begin{tabular}{l*{4}{c}} \toprule \multicolumn{4}{l}{Dep. variable: \textbf{Mental and behavioral disorders}} \\ & \multicolumn{3}{c}{Choice of control group} \\ \cmidrule(lr){2-4}
            &\multicolumn{1}{c}{(1)}&\multicolumn{1}{c}{(2)}&\multicolumn{1}{c}{(3)}\\
            &\multicolumn{1}{c}{C1}&\multicolumn{1}{c}{C3}&\multicolumn{1}{c}{C1+C3}\\
\midrule
 \multicolumn{4}{l}{\emph{Panel A. Average causal effects}} \\ Abs. numbers        &      -4.108         &      -4.758         &      -4.433         \\
                    &     (13.27)         &     (13.07)         &     (13.27)         \\
 Ratio fertility     &       0.162         &       0.898\sym{***}&       0.530         \\
                    &     (0.298)         &     (0.288)         &     (0.327)         \\
 Ratio population    &       0.329         &       0.736\sym{**} &       0.532\sym{*}  \\
                    &     (0.270)         &     (0.272)         &     (0.273)         \\
 Cum. numbers        &      -132.9         &      -91.10         &      -112.0         \\
                    &     (134.7)         &     (126.5)         &     (141.6)         \\
 Cum. ratio          &      -0.397         &       6.571\sym{**} &       3.087         \\
                    &     (2.887)         &     (2.692)         &     (4.005)         \\
 \midrule\multicolumn{4}{l}{\emph{Panel B. Treatment effect heterogeneity - Women}} \\ Abs. numbers        &       6.392         &      -14.47\sym{*}  &      -4.038         \\
                    &     (7.757)         &     (7.175)         &     (9.413)         \\
 Ratio fertility     &       0.457         &       0.285         &       0.371         \\
                    &     (0.362)         &     (0.318)         &     (0.342)         \\
 Ratio population    &       0.355         &     -0.0219         &       0.167         \\
                    &     (0.295)         &     (0.323)         &     (0.323)         \\
 Cum. numbers        &       56.85         &      -108.4         &      -25.80         \\
                    &     (81.94)         &     (64.91)         &     (83.55)         \\
 Cum. ratio          &       4.265         &       3.625         &       3.945         \\
                    &     (3.682)         &     (2.896)         &     (3.274)         \\
 \midrule\multicolumn{4}{l}{\emph{Panel C. Treatment effect heterogeneity - Men}} \\ Abs. numbers        &      -10.50         &       9.708         &      -0.396         \\
                    &     (7.832)         &     (8.916)         &     (9.015)         \\
 Ratio fertility     &      -0.112         &       1.479\sym{***}&       0.684         \\
                    &     (0.325)         &     (0.376)         &     (0.494)         \\
 Ratio population    &       0.294         &       1.494\sym{***}&       0.894\sym{**} \\
                    &     (0.320)         &     (0.346)         &     (0.353)         \\
 Cum. numbers        &      -189.7\sym{**} &       17.34         &      -86.19         \\
                    &     (83.38)         &     (84.90)         &     (128.0)         \\
 Cum. ratio          &      -4.749         &       9.372\sym{**} &       2.311         \\
                    &     (3.326)         &     (3.409)         &     (6.535)         \\
 
\bottomrule \end{tabular} } \begin{tablenotes} \item \scriptsize \emph{Notes:} Clustered standard errors in parentheses. All regression are run with month-of-birth FEs and control cohort 2 is assigned with the treatment status. All regressions are carried out with a window width of half a year. \end{tablenotes} \end{threeparttable} \end{table} 

 \begin{table}[H] \centering \begin{threeparttable} \caption{Placebo 3 (CONTROL3 ist TREAT) } {\def\sym#1{\ifmmode^{#1}\else\(^{#1}\)\fi} \begin{tabular}{l*{4}{c}} \toprule \multicolumn{4}{l}{Dep. variable: \textbf{Mental and behavioral disorders}} \\ & \multicolumn{3}{c}{Choice of control group} \\ \cmidrule(lr){2-4}
            &\multicolumn{1}{c}{(1)}&\multicolumn{1}{c}{(2)}&\multicolumn{1}{c}{(3)}\\
            &\multicolumn{1}{c}{C1}&\multicolumn{1}{c}{C2}&\multicolumn{1}{c}{C1+C2}\\
\midrule
 \multicolumn{4}{l}{\emph{Panel A. Average causal effects}} \\ Abs. numbers        &       0.650         &       4.758         &       2.704         \\
                    &     (8.986)         &     (13.07)         &     (11.10)         \\
 Ratio fertility     &      -0.736\sym{**} &      -0.898\sym{***}&      -0.817\sym{***}\\
                    &     (0.282)         &     (0.288)         &     (0.285)         \\
 Ratio population    &      -0.407\sym{*}  &      -0.736\sym{**} &      -0.571\sym{**} \\
                    &     (0.230)         &     (0.272)         &     (0.252)         \\
 Cum. numbers        &      -41.77         &       91.10         &       24.67         \\
                    &     (107.3)         &     (126.5)         &     (118.5)         \\
 Cum. ratio          &      -6.968\sym{**} &      -6.571\sym{**} &      -6.769\sym{**} \\
                    &     (2.771)         &     (2.692)         &     (2.702)         \\
 \midrule\multicolumn{4}{l}{\emph{Panel B. Treatment effect heterogeneity - Women}} \\ Abs. numbers        &       20.86\sym{***}&       14.47\sym{*}  &       17.66\sym{***}\\
                    &     (4.542)         &     (7.175)         &     (5.971)         \\
 Ratio fertility     &       0.172         &      -0.285         &     -0.0565         \\
                    &     (0.211)         &     (0.318)         &     (0.272)         \\
 Ratio population    &       0.377\sym{*}  &      0.0219         &       0.200         \\
                    &     (0.191)         &     (0.323)         &     (0.264)         \\
 Cum. numbers        &       165.3\sym{***}&       108.4         &       136.9\sym{**} \\
                    &     (48.05)         &     (64.91)         &     (57.40)         \\
 Cum. ratio          &       0.641         &      -3.625         &      -1.492         \\
                    &     (2.082)         &     (2.896)         &     (2.535)         \\
 \midrule\multicolumn{4}{l}{\emph{Panel C. Treatment effect heterogeneity - Men}} \\ Abs. numbers        &      -20.21\sym{**} &      -9.708         &      -14.96\sym{*}  \\
                    &     (8.474)         &     (8.916)         &     (8.674)         \\
 Ratio fertility     &      -1.591\sym{***}&      -1.479\sym{***}&      -1.535\sym{***}\\
                    &     (0.463)         &     (0.376)         &     (0.421)         \\
 Ratio population    &      -1.200\sym{***}&      -1.494\sym{***}&      -1.347\sym{***}\\
                    &     (0.398)         &     (0.346)         &     (0.372)         \\
 Cum. numbers        &      -207.1\sym{**} &      -17.34         &      -112.2         \\
                    &     (93.46)         &     (84.90)         &     (91.16)         \\
 Cum. ratio          &      -14.12\sym{***}&      -9.372\sym{**} &      -11.75\sym{***}\\
                    &     (4.539)         &     (3.409)         &     (3.999)         \\
 
\bottomrule \end{tabular} } \begin{tablenotes} \item \scriptsize \emph{Notes:} Clustered standard errors in parentheses. All regression are run with month-of-birth FEs and control cohort 3 is assigned with the treatment status. All regressions are carried out with a window width of half a year. \end{tablenotes} \end{threeparttable} \end{table} 

%---------------------------------
% CUMMULATIVE APPROACH
\begin{landscape}
 \begin{table}[H] \begin{threeparttable} \centering \caption{Cummulative effects for upt to different points of age} {\def\sym#1{\ifmmode^{#1}\else\(^{#1}\)\fi} \begin{tabular}{l*{13}{c}} \toprule & \multicolumn{12}{c}{Dependent variable: \textbf{Mental and behavioral disorders}} \\ \cmidrule(lr){2-13}
            &\multicolumn{4}{c}{Average Causal Effects}         &\multicolumn{4}{c}{Women}                          &\multicolumn{4}{c}{Men}                            \\\cmidrule(lr){2-5}\cmidrule(lr){6-9}\cmidrule(lr){10-13}
            &\multicolumn{1}{c}{(1)}&\multicolumn{1}{c}{(2)}&\multicolumn{1}{c}{(3)}&\multicolumn{1}{c}{(4)}&\multicolumn{1}{c}{(5)}&\multicolumn{1}{c}{(6)}&\multicolumn{1}{c}{(7)}&\multicolumn{1}{c}{(8)}&\multicolumn{1}{c}{(9)}&\multicolumn{1}{c}{(10)}&\multicolumn{1}{c}{(11)}&\multicolumn{1}{c}{(12)}\\
            &\multicolumn{1}{c}{2M}&\multicolumn{1}{c}{4M}&\multicolumn{1}{c}{6M}&\multicolumn{1}{c}{Donut}&\multicolumn{1}{c}{2M}&\multicolumn{1}{c}{4M}&\multicolumn{1}{c}{6M}&\multicolumn{1}{c}{Donut}&\multicolumn{1}{c}{2M}&\multicolumn{1}{c}{4M}&\multicolumn{1}{c}{6M}&\multicolumn{1}{c}{Donut}\\
\midrule
 \multicolumn{13}{l}{\emph{Panel A. 2 Up to the age of 21}} \\ Cum. numbers        &       228.5         &       180.0         &       193.7\sym{**} &       155.6\sym{**} &         132         &       124.8         &       136.3\sym{**} &       122.6\sym{**} &       96.50         &       55.25         &       57.33         &       33.00         \\
                    &     (152.5)         &     (125.2)         &     (81.37)         &     (73.51)         &     (83.05)         &     (74.15)         &     (50.31)         &     (53.34)         &     (71.11)         &     (62.70)         &     (40.07)         &     (32.44)         \\
 Cum. ratio          &       3.740         &       1.058         &       0.394         &      -0.915         &       4.510         &       2.560         &       1.840         &       0.763         &       3.005         &      -0.372         &      -0.998         &      -2.535         \\
                    &     (4.004)         &     (3.082)         &     (2.041)         &     (1.865)         &     (4.176)         &     (3.681)         &     (2.430)         &     (2.575)         &     (3.917)         &     (2.882)         &     (1.939)         &     (1.502)         \\
 \midrule\multicolumn{13}{l}{\emph{Panel B. Up to the age of 26}} \\ Cum. numbers        &       589.0         &       287.5         &       402.8\sym{**} &       312.6\sym{**} &       248.0         &       194.5         &       256.7\sym{**} &       223.6\sym{**} &       341.0\sym{***}&       93.00         &       146.2         &       89.00         \\
                    &     (340.5)         &     (192.0)         &     (145.8)         &     (131.1)         &     (281.5)         &     (138.2)         &     (95.04)         &     (90.84)         &     (71.47)         &     (113.0)         &     (86.34)         &     (89.84)         \\
 Cum. ratio          &       9.845         &       0.198         &       0.355         &      -2.714         &       8.237         &       2.802         &       2.864         &       0.387         &       11.45\sym{*}  &      -2.235         &      -1.896         &      -5.545         \\
                    &     (9.484)         &     (6.169)         &     (3.925)         &     (3.589)         &     (13.69)         &     (7.560)         &     (4.794)         &     (4.813)         &     (5.740)         &     (6.216)         &     (4.008)         &     (3.609)         \\
 \midrule\multicolumn{13}{l}{\emph{Panel C. Up to the age of 31}} \\ Cum. numbers        &       596.0\sym{*}  &       95.75         &       403.5         &       349.6         &       291.5         &       168.5         &       311.7\sym{**} &       313.4\sym{**} &       304.5\sym{**} &      -72.75         &       91.83         &       36.20         \\
                    &     (310.3)         &     (250.2)         &     (237.8)         &     (266.5)         &     (402.8)         &     (189.5)         &     (141.6)         &     (128.9)         &     (107.7)         &     (160.9)         &     (146.4)         &     (172.5)         \\
 Cum. ratio          &       8.631         &      -7.340         &      -4.644         &      -7.813         &       8.863         &      -1.277         &       0.735         &      -0.980         &       8.595\sym{*}  &      -13.00         &      -9.416         &      -13.99\sym{*}  \\
                    &     (10.86)         &     (9.117)         &     (5.946)         &     (6.465)         &     (20.26)         &     (10.58)         &     (6.860)         &     (6.990)         &     (3.873)         &     (10.14)         &     (6.738)         &     (7.229)         \\
 \midrule\multicolumn{13}{l}{\emph{Panel D. Up to the age of 34}} \\ Cum. numbers        &       283.0         &      -200.0         &       260.3         &       237.8         &       240.0         &       170.2         &       391.5\sym{**} &       446.4\sym{***}&       43.00         &      -370.3\sym{*}  &      -131.2         &      -208.6         \\
                    &     (303.0)         &     (279.6)         &     (307.3)         &     (347.2)         &     (397.1)         &     (193.6)         &     (166.5)         &     (155.4)         &     (176.1)         &     (205.3)         &     (194.7)         &     (227.9)         \\
 Cum. ratio          &       1.086         &      -16.75         &      -12.27\sym{*}  &      -15.61\sym{*}  &       5.589         &      -3.961         &      0.0829         &     -0.0107         &      -2.922         &      -28.75\sym{**} &      -23.49\sym{***}&      -29.95\sym{***}\\
                    &     (12.23)         &     (9.874)         &     (6.843)         &     (7.464)         &     (21.02)         &     (10.92)         &     (7.608)         &     (8.184)         &     (9.170)         &     (11.85)         &     (8.034)         &     (8.062)         \\
 
\bottomrule \end{tabular} } \begin{tablenotes} \item \scriptsize \emph{Notes:} Clustered standard errors in parentheses (MxY). All regressions contain Birthmonth FE. Ratios indicate cases per thousand; original number of births. \end{tablenotes} \end{threeparttable} \end{table} 

\end{landscape}
\begin{landscape}
 \begin{table}[H] \begin{threeparttable} \centering \caption{Cummulative effects for upt to different points of age - BOOTSTRAPPED} {\def\sym#1{\ifmmode^{#1}\else\(^{#1}\)\fi} \begin{tabular}{l*{13}{c}} \toprule & \multicolumn{12}{c}{Dependent variable: \textbf{Mental and behavioral disorders}} \\ \cmidrule(lr){2-13}
            &\multicolumn{4}{c}{Average Causal Effects}         &\multicolumn{4}{c}{Women}                          &\multicolumn{4}{c}{Men}                            \\\cmidrule(lr){2-5}\cmidrule(lr){6-9}\cmidrule(lr){10-13}
            &\multicolumn{1}{c}{(1)}&\multicolumn{1}{c}{(2)}&\multicolumn{1}{c}{(3)}&\multicolumn{1}{c}{(4)}&\multicolumn{1}{c}{(5)}&\multicolumn{1}{c}{(6)}&\multicolumn{1}{c}{(7)}&\multicolumn{1}{c}{(8)}&\multicolumn{1}{c}{(9)}&\multicolumn{1}{c}{(10)}&\multicolumn{1}{c}{(11)}&\multicolumn{1}{c}{(12)}\\
            &\multicolumn{1}{c}{2M}&\multicolumn{1}{c}{4M}&\multicolumn{1}{c}{6M}&\multicolumn{1}{c}{Donut}&\multicolumn{1}{c}{2M}&\multicolumn{1}{c}{4M}&\multicolumn{1}{c}{6M}&\multicolumn{1}{c}{Donut}&\multicolumn{1}{c}{2M}&\multicolumn{1}{c}{4M}&\multicolumn{1}{c}{6M}&\multicolumn{1}{c}{Donut}\\
\midrule
 \multicolumn{13}{l}{\emph{Panel A. 2 Up to the age of 21}} \\ Cum. numbers        &       228.5         &       180.0         &       193.7\sym{*}  &       155.6         &         132\sym{*}  &       124.8         &       136.3\sym{*}  &       122.6\sym{*}  &       96.50         &       55.25         &       57.33         &       33.00         \\
                    &     (142.0)         &     (164.0)         &     (117.1)         &     (106.5)         &     (78.68)         &     (99.80)         &     (77.23)         &     (73.80)         &     (64.77)         &     (84.06)         &     (55.96)         &     (46.45)         \\
 Cum. ratio          &       3.740         &       1.058         &       0.394         &      -0.915         &       4.510         &       2.560         &       1.840         &       0.763         &       3.005         &      -0.372         &      -0.998         &      -2.535         \\
                    &     (3.826)         &     (4.045)         &     (2.907)         &     (2.666)         &     (4.043)         &     (5.049)         &     (3.710)         &     (3.479)         &     (3.684)         &     (3.649)         &     (2.583)         &     (2.274)         \\
 \midrule\multicolumn{13}{l}{\emph{Panel B. Up to the age of 26}} \\ Cum. numbers        &       589.0\sym{*}  &       287.5         &       402.8\sym{**} &       312.6\sym{*}  &       248.0         &       194.5         &       256.7\sym{*}  &       223.6\sym{*}  &       341.0\sym{***}&       93.00         &       146.2         &       89.00         \\
                    &     (326.0)         &     (238.7)         &     (190.8)         &     (170.6)         &     (273.1)         &     (190.4)         &     (136.3)         &     (125.5)         &     (63.74)         &     (145.0)         &     (122.1)         &     (111.5)         \\
 Cum. ratio          &       9.845         &       0.198         &       0.355         &      -2.714         &       8.237         &       2.802         &       2.864         &       0.387         &       11.45\sym{**} &      -2.235         &      -1.896         &      -5.545         \\
                    &     (9.203)         &     (8.008)         &     (5.094)         &     (5.393)         &     (13.37)         &     (10.61)         &     (6.823)         &     (6.985)         &     (5.453)         &     (7.562)         &     (5.094)         &     (5.223)         \\
 \midrule\multicolumn{13}{l}{\emph{Panel C. Up to the age of 31}} \\ Cum. numbers        &       596.0\sym{*}  &       95.75         &       403.5         &       349.6         &       291.5         &       168.5         &       311.7         &       313.4\sym{*}  &       304.5\sym{***}&      -72.75         &       91.83         &       36.20         \\
                    &     (306.3)         &     (308.5)         &     (311.3)         &     (334.6)         &     (398.3)         &     (256.2)         &     (198.7)         &     (180.9)         &     (103.6)         &     (206.1)         &     (197.1)         &     (210.6)         \\
 Cum. ratio          &       8.631         &      -7.340         &      -4.644         &      -7.813         &       8.863         &      -1.277         &       0.735         &      -0.980         &       8.595\sym{**} &      -13.00         &      -9.416         &      -13.99         \\
                    &     (10.72)         &     (11.89)         &     (8.020)         &     (9.585)         &     (20.03)         &     (14.72)         &     (10.15)         &     (10.66)         &     (3.464)         &     (12.56)         &     (8.419)         &     (10.13)         \\
 \midrule\multicolumn{13}{l}{\emph{Panel D. Up to the age of 34}} \\ Cum. numbers        &       283.0         &      -200.0         &       260.3         &       237.8         &       240.0         &       170.2         &       391.5\sym{*}  &       446.4\sym{**} &       43.00         &      -370.3         &      -131.2         &      -208.6         \\
                    &     (298.7)         &     (341.8)         &     (381.1)         &     (411.4)         &     (389.3)         &     (255.8)         &     (221.2)         &     (204.3)         &     (156.9)         &     (275.9)         &     (258.4)         &     (270.7)         \\
 Cum. ratio          &       1.086         &      -16.75         &      -12.27         &      -15.61         &       5.589         &      -3.961         &      0.0829         &     -0.0107         &      -2.922         &      -28.75\sym{**} &      -23.49\sym{**} &      -29.95\sym{***}\\
                    &     (12.07)         &     (12.56)         &     (9.314)         &     (10.86)         &     (20.66)         &     (15.14)         &     (11.48)         &     (12.36)         &     (8.168)         &     (14.48)         &     (9.990)         &     (11.03)         \\
 
\bottomrule \end{tabular} } \begin{tablenotes} \item \scriptsize \emph{Notes:} \textbf{BOOTSTRAPPED} standard errors in parentheses (MxY), with 400 replications. All regressions contain Birthmonth FE. Ratios indicate cases per thousand; original number of births. \end{tablenotes} \end{threeparttable} \end{table} 

\end{landscape}
%---------------------------------
\newpage
FEBRUAR CASES:
 \begin{table}[H] \begin{threeparttable} \centering \caption{Dep. variable: \textbf{Mental and behavioral disorders}} {\def\sym#1{\ifmmode^{#1}\else\(^{#1}\)\fi} \begin{tabular}{l*{13}{c}} \toprule year & \multicolumn{12}{c}{Month of birth} \\ \cmidrule(lr){2-13} 
            &          11&          12&           1&           2&           3&           4&           5&           6&           7&           8&           9&          10\\
1995        &         373&         369&         426&         373&         430&         417&         401&         350&         382&         371&         368&         374\\
1996        &         479&         493&         463&         459&         489&         501&         466&         479&         456&         429&         419&         394\\
1997        &         588&         551&         593&         577&         617&         605&         582&         560&         637&         588&         572&         528\\
1998        &         683&         614&         653&         643&         721&         691&         725&         663&         718&         692&         644&         690\\
1999        &         726&         780&         742&         639&         753&         756&         743&         719&         792&         746&         768&         717\\
2000        &         738&         670&         819&         818&         883&         852&         769&         782&         884&         900&         753&         775\\
2001        &         873&         839&         843&         847&         902&         865&         861&         837&         958&         986&         863&         903\\
2002        &         823&         833&         922&         842&         920&         917&         904&         805&         972&        1013&         853&         870\\
2003        &         817&         862&         859&         851&         920&         854&         920&         821&         976&         954&         840&         885\\
2004        &         922&         874&         850&         928&        1032&         935&         921&         869&         979&        1005&         875&         880\\
2005        &         910&         887&         910&         873&        1002&         869&         963&         895&         984&        1005&         994&         877\\
2006        &         933&         880&         934&         940&        1040&         928&         985&         950&        1051&        1003&         915&         943\\
2007        &         999&         932&         938&        1015&        1130&        1002&        1063&        1012&        1134&        1146&        1010&         958\\
2008        &         995&         982&        1027&         974&        1084&        1067&        1069&        1099&        1134&        1078&        1032&        1003\\
2009        &         978&         969&        1053&        1040&        1151&        1076&        1178&        1096&        1173&        1181&        1041&        1046\\
2010        &        1041&        1035&        1006&        1056&        1149&        1057&        1085&        1184&        1187&        1134&        1072&        1139\\
2011        &        1055&         987&        1029&        1080&        1176&        1114&        1258&        1132&        1192&        1236&        1174&        1160\\
2012        &        1128&        1110&        1106&        1151&        1271&        1126&        1284&        1209&        1264&        1241&        1182&        1221\\
2013        &        1176&        1156&        1125&        1122&        1144&        1161&        1282&        1228&        1259&        1346&        1222&        1208\\
2014        &        1227&        1195&        1266&        1139&        1238&        1243&        1274&        1162&        1215&        1415&        1359&        1219\\
 \bottomrule \end{tabular} } \begin{tablenotes} \item \scriptsize \emph{Notes:} Number of cases per year and MOB in treatment cohort. \end{tablenotes} \end{threeparttable} \end{table} 

 \begin{table}[H] \begin{threeparttable} \centering \caption{Dep. variable: \textbf{Mental and behavioral disorders}} {\def\sym#1{\ifmmode^{#1}\else\(^{#1}\)\fi} \begin{tabular}{l*{13}{c}} \toprule year & \multicolumn{12}{c}{Month of birth} \\ \cmidrule(lr){2-13} 
            &          11&          12&           1&           2&           3&           4&           5&           6&           7&           8&           9&          10\\
1995        &          42&          51&          94&          19&          39&          94&          51&          14&          33&          30&          37&          15\\
1996        &          50&          95&          54&          68&          81&         140&          41&          54&          12&           9&          17&          16\\
1997        &         125&          92&          67&         -11&          88&          98&          69&          85&          96&          29&          80&          65\\
1998        &          39&         -19&          52&           9&          92&         115&          68&          50&          21&           5&         103&          73\\
1999        &          33&         109&          60&         -58&          -8&          77&         -31&         -21&          28&         -82&          19&         -46\\
2000        &          16&         -75&          20&          55&          92&         134&          -7&          28&          44&          88&         -90&         -53\\
2001        &          80&          45&         -28&          12&          22&          92&         -90&        -101&         -17&          10&         -12&         -22\\
2002        &          56&           3&          78&         -12&         -14&          77&         -75&         -64&          84&          84&         -82&          -8\\
2003        &          -3&         -23&         -22&         -63&         -60&          55&          37&         -35&           6&          42&         -87&         -56\\
2004        &          82&         -84&         -34&          38&          35&          56&           8&         -96&          11&          68&        -120&        -126\\
2005        &          94&           8&         -67&         -13&         -71&        -108&          -1&         -49&         -12&         -37&          63&        -163\\
2006        &          36&         -70&          -5&         -17&           1&         -72&          -5&          -6&          55&         -43&        -107&         -56\\
2007        &          62&         -14&         -61&          -7&          49&           4&           0&         -32&          33&         103&         -33&         -64\\
2008        &          51&          27&          20&        -112&           0&          18&         -33&         103&          59&          -2&         -43&        -103\\
2009        &          43&          -4&         -56&         -60&          12&          28&         116&        -111&          17&          31&         -71&         -66\\
2010        &          34&         -11&        -147&         -63&         -16&         -49&         -34&          39&          47&          26&         -21&           8\\
2011        &          52&         -83&        -131&         -59&         -33&          41&          56&         -16&          50&          29&          41&         -24\\
2012        &          81&         -43&         -70&          17&          68&         -20&         120&          91&           5&         -77&         -82&          58\\
2013        &          90&         -25&        -104&        -107&        -123&         -24&          58&          58&         -57&           8&         -65&           0\\
2014        &          71&          31&         -51&        -186&         -85&         -73&          -7&         -91&        -110&         -16&           6&         -63\\
 \bottomrule \end{tabular} } \begin{tablenotes} \item \scriptsize \emph{Notes:} Difference of cases (control - treatment) per year and MOB in treatment cohort. \end{tablenotes} \end{threeparttable} \end{table} 

