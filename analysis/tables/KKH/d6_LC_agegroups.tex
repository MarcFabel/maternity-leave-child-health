 \begin{table}[H] \centering \begin{threeparttable} \caption{Life-course approach - Table format} {\def\sym#1{\ifmmode^{#1}\else\(^{#1}\)\fi} \begin{tabular}{l*{5}{c}} \toprule \multicolumn{5}{l}{Dep. variable: \textbf{Diseases of the nervous system}} \\ & \multicolumn{4}{c}{Estimation window} \\ \cmidrule(lr){2-5}
            &\multicolumn{1}{c}{(1)}&\multicolumn{1}{c}{(2)}&\multicolumn{1}{c}{(3)}&\multicolumn{1}{c}{(4)}\\
            &\multicolumn{1}{c}{Age 17-21}&\multicolumn{1}{c}{Age 22-26}&\multicolumn{1}{c}{Age 27-31}&\multicolumn{1}{c}{Age 32-35}\\
\midrule
 \multicolumn{5}{l}{\emph{Panel A. Average causal effects}} \\ Abs. numbers        &       2.533         &       20.53\sym{***}&       13.07         &       13.12         \\
                    &     (5.342)         &     (5.446)         &     (10.20)         &     (12.58)         \\
 Ratio fertility     &     -0.0919         &       0.238         &      0.0374         &     -0.0190         \\
                    &     (0.129)         &     (0.149)         &     (0.191)         &     (0.256)         \\
 Ratio population    &       0.116         &      0.0420         &     0.00134         \\
                    &     (0.150)         &     (0.150)         &     (0.203)         \\
 Cum. numbers        &      -4.333         &       74.03\sym{*}  &       169.5\sym{***}&       215.1\sym{**} \\
                    &     (30.43)         &     (38.50)         &     (50.38)         &     (92.63)         \\
 Cum. ratio          &      -0.722         &      0.0343         &       0.957         &       0.682         \\
                    &     (0.617)         &     (1.055)         &     (1.312)         &     (2.020)         \\
 \midrule\multicolumn{5}{l}{\emph{Panel B. Treatment effect heterogeneity - Women}} \\ Abs. numbers        &      -2.367         &       10.60\sym{*}  &       8.833         &       10.71\sym{*}  \\
                    &     (3.358)         &     (5.934)         &     (9.153)         &     (5.943)         \\
 Ratio fertility     &      -0.274         &       0.221         &       0.106         &       0.115         \\
                    &     (0.178)         &     (0.277)         &     (0.336)         &     (0.275)         \\
 Ratio population    &       0.183         &      0.0931         &       0.102         \\
                    &     (0.181)         &     (0.252)         &     (0.203)         \\
 Cum. numbers        &      -13.13         &       15.23         &       72.60\sym{*}  &       102.3         \\
                    &     (14.38)         &     (35.11)         &     (42.80)         &     (73.16)         \\
 Cum. ratio          &      -1.285\sym{*}  &      -1.158         &      0.0326         &      -0.107         \\
                    &     (0.724)         &     (1.828)         &     (1.972)         &     (3.086)         \\
 \midrule\multicolumn{5}{l}{\emph{Panel C. Treatment effect heterogeneity - Men}} \\ Abs. numbers        &       4.900         &       9.933\sym{*}  &       4.233         &       2.417         \\
                    &     (6.026)         &     (5.411)         &     (6.688)         &     (10.74)         \\
 Ratio fertility     &      0.0787         &       0.250         &     -0.0319         &      -0.150         \\
                    &     (0.247)         &     (0.250)         &     (0.291)         &     (0.421)         \\
 Ratio population    &      0.0455         &     -0.0122         &      -0.102         \\
                    &     (0.254)         &     (0.231)         &     (0.342)         \\
 Cum. numbers        &       8.800         &       58.80\sym{*}  &       96.87\sym{**} &       112.7         \\
                    &     (30.53)         &     (33.17)         &     (47.94)         &     (83.29)         \\
 Cum. ratio          &      -0.196         &       1.142         &       1.791         &       1.369         \\
                    &     (1.183)         &     (1.454)         &     (2.259)         &     (3.625)         \\
 
\bottomrule \end{tabular} } \begin{tablenotes} \item \scriptsize \emph{Notes:} Clustered standard errors in parentheses. All regression are run with CG2 (i.e. the cohort prior to the reform) and with month-of-birth FEs. Ratios indicate cases per thousand; either approximated population (with weights coming from the original fertility distribution) or original number of births. Raqtio population muss eins nach rechts gerückt werden \end{tablenotes} \end{threeparttable} \end{table} 
