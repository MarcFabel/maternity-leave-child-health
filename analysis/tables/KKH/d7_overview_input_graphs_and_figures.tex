%---------------------------------
% INPUT FOR VARIABLE: d7
%---------------------------------
\subsection{d7}
% RD overview
\begin{landscape}
\begin{figure}[H]
	\centering
	\begin{minipage}{.95\linewidth}
	\includegraphics[width=\linewidth]{rd_d7_overview_panel1}
	{\scriptsize \emph{Notes:} The figures show monthly RD plots with averages obtained from a bin width of one month. The solid vertical line divides pre- and post-reform regime. The averages are taken over the period of at most 1995-2014. \par}
\end{minipage}
\end{figure}
\end{landscape}
\begin{landscape}
\begin{figure}[H]
	\centering
\begin{minipage}{.95\linewidth}
	\includegraphics[width=\linewidth]{rd_d7_overview_panel2}
	{\scriptsize \emph{Notes:} The figures show monthly RD plots with a moving average window width of 3 months. The solid vertical line divides pre- and post-reform regime. The averages are taken over the period of at most 1995-2014. \par}
\end{minipage}
\end{figure}
\end{landscape}
%---------------------------------
% TABELLEN
 \begin{table}[H] \begin{threeparttable} \centering \caption{Dep. variable: \textbf{Diseases of the eye and ear}} {\def\sym#1{\ifmmode^{#1}\else\(^{#1}\)\fi} \begin{tabular}{l*{8}{c}} \toprule & \multicolumn{7}{c}{Estimation window} \\ \cmidrule(lr){2-8}
            &\multicolumn{1}{c}{(1)}&\multicolumn{1}{c}{(2)}&\multicolumn{1}{c}{(3)}&\multicolumn{1}{c}{(4)}&\multicolumn{1}{c}{(5)}&\multicolumn{1}{c}{(6)}&\multicolumn{1}{c}{(7)}\\
            &\multicolumn{1}{c}{1M}&\multicolumn{1}{c}{2M}&\multicolumn{1}{c}{3M}&\multicolumn{1}{c}{4M}&\multicolumn{1}{c}{5M}&\multicolumn{1}{c}{6M}&\multicolumn{1}{c}{Donut}\\
\midrule
 \multicolumn{8}{l}{\emph{Panel A. Average causal effects}} \\ Abs. numbers        &      -6.300\sym{***}&      -6.575\sym{***}&      -5.450\sym{***}&      -5.238\sym{***}&      -5.100\sym{***}&      -2.142         &      -1.310         \\
                    &  (8.61e-15)         &     (1.205)         &     (1.283)         &     (0.956)         &     (0.762)         &     (1.535)         &     (1.686)         \\
 Ratio fertility     &      -0.143\sym{***}&      -0.164\sym{***}&      -0.173\sym{***}&      -0.177\sym{***}&      -0.181\sym{***}&      -0.144\sym{***}&      -0.145\sym{***}\\
                    &  (2.19e-16)         &   (0.00947)         &    (0.0244)         &    (0.0284)         &    (0.0237)         &    (0.0266)         &    (0.0279)         \\
 Ratio population    &      -0.145\sym{***}&      -0.166\sym{***}&      -0.132\sym{***}&      -0.147\sym{***}&      -0.140\sym{***}&      -0.128\sym{***}&      -0.125\sym{***}\\
                    &  (2.81e-16)         &    (0.0265)         &    (0.0380)         &    (0.0456)         &    (0.0381)         &    (0.0322)         &    (0.0272)         \\
 Ratio fert(03-14)   &      -0.116\sym{***}&      -0.128\sym{***}&     -0.0941\sym{**} &      -0.114\sym{**} &      -0.110\sym{***}&     -0.0972\sym{***}&     -0.0934\sym{***}\\
                    &  (4.33e-16)         &    (0.0319)         &    (0.0402)         &    (0.0450)         &    (0.0375)         &    (0.0317)         &    (0.0250)         \\
 Cum. numbers        &      -42.80\sym{***}&      -44.38\sym{***}&      -55.83\sym{***}&      -51.03\sym{***}&      -47.30\sym{***}&      -5.000         &       2.560         \\
                    &  (6.59e-14)         &     (3.235)         &     (6.663)         &     (9.887)         &     (8.882)         &     (21.35)         &     (25.34)         \\
 Cum. ratio          &      -1.011\sym{***}&      -1.213\sym{***}&      -1.792\sym{***}&      -1.779\sym{***}&      -1.777\sym{***}&      -1.150\sym{***}&      -1.178\sym{***}\\
                    &  (3.73e-15)         &     (0.163)         &     (0.300)         &     (0.230)         &     (0.183)         &     (0.333)         &     (0.401)         \\
 \midrule\multicolumn{8}{l}{\emph{Panel B. Treatment effect heterogeneity - Women}} \\ Abs. numbers        &      -2.900\sym{***}&      -3.850\sym{***}&      -2.217\sym{**} &      -4.113\sym{***}&      -4.080\sym{***}&      -2.508\sym{**} &      -2.430\sym{*}  \\
                    &  (9.09e-15)         &     (0.516)         &     (0.894)         &     (1.092)         &     (1.020)         &     (1.127)         &     (1.316)         \\
 Ratio fertility     &     -0.0838\sym{***}&      -0.105\sym{***}&     -0.0793\sym{***}&      -0.145\sym{***}&      -0.150\sym{***}&      -0.122\sym{***}&      -0.129\sym{***}\\
                    &  (1.06e-16)         &    (0.0113)         &    (0.0243)         &    (0.0357)         &    (0.0372)         &    (0.0341)         &    (0.0373)         \\
 Ratio population    &     -0.0693\sym{***}&     -0.0755\sym{***}&     -0.0444         &      -0.143\sym{**} &      -0.146\sym{**} &      -0.144\sym{***}&      -0.159\sym{***}\\
                    &  (2.77e-16)         &    (0.0147)         &    (0.0455)         &    (0.0570)         &    (0.0529)         &    (0.0441)         &    (0.0471)         \\
 Ratio fert(03-14)   &      -0.129\sym{***}&      -0.154\sym{***}&      -0.112\sym{*}  &      -0.238\sym{***}&      -0.244\sym{***}&      -0.244\sym{***}&      -0.267\sym{***}\\
                    &  (3.80e-16)         &    (0.0107)         &    (0.0608)         &    (0.0747)         &    (0.0681)         &    (0.0566)         &    (0.0611)         \\
 Cum. numbers        &      -37.30\sym{***}&      -43.98\sym{***}&      -30.50\sym{***}&      -45.04\sym{***}&      -41.05\sym{***}&      -19.53         &      -15.98         \\
                    &  (1.50e-14)         &     (2.663)         &     (7.353)         &     (8.937)         &     (8.866)         &     (12.60)         &     (15.05)         \\
 Cum. ratio          &      -1.072\sym{***}&      -1.204\sym{***}&      -1.074\sym{***}&      -1.602\sym{***}&      -1.534\sym{***}&      -1.072\sym{***}&      -1.072\sym{**} \\
                    &  (6.39e-15)         &    (0.0514)         &    (0.0856)         &     (0.251)         &     (0.306)         &     (0.345)         &     (0.410)         \\
 \midrule\multicolumn{8}{l}{\emph{Panel C. Treatment effect heterogeneity - Men}} \\ Abs. numbers        &      -3.400\sym{***}&      -2.725\sym{**} &      -3.233\sym{***}&      -1.125         &      -1.020         &       0.367         &       1.120         \\
                    &  (7.69e-15)         &     (0.875)         &     (0.763)         &     (1.119)         &     (1.047)         &     (1.088)         &     (1.211)         \\
 Ratio fertility     &     -0.0819\sym{***}&     -0.0762\sym{***}&      -0.108\sym{***}&     -0.0604         &     -0.0595\sym{*}  &     -0.0281         &     -0.0174         \\
                    &  (2.77e-16)         &    (0.0155)         &    (0.0250)         &    (0.0363)         &    (0.0301)         &    (0.0292)         &    (0.0319)         \\
 Ratio population    &      -0.196\sym{***}&      -0.218\sym{***}&      -0.181\sym{***}&      -0.118         &      -0.104\sym{*}  &     -0.0827         &     -0.0600         \\
                    &  (1.53e-16)         &    (0.0483)         &    (0.0400)         &    (0.0720)         &    (0.0577)         &    (0.0494)         &    (0.0468)         \\
 Ratio fert(03-14)   &      -0.163\sym{***}&      -0.180\sym{***}&      -0.142\sym{***}&     -0.0911         &     -0.0816         &     -0.0599         &     -0.0392         \\
                    &  (2.83e-16)         &    (0.0491)         &    (0.0419)         &    (0.0671)         &    (0.0536)         &    (0.0462)         &    (0.0420)         \\
 Cum. numbers        &      -5.500\sym{***}&      -0.400         &      -25.33\sym{*}  &      -5.987         &      -6.250         &       14.53         &       18.54         \\
                    &  (1.22e-13)         &     (3.143)         &     (13.62)         &     (14.54)         &     (14.49)         &     (15.70)         &     (18.74)         \\
 Cum. ratio          &      -0.192\sym{***}&      -0.242\sym{*}  &      -1.526\sym{**} &      -0.862         &      -0.921\sym{*}  &      -0.279         &      -0.296         \\
                    &  (7.33e-15)         &     (0.118)         &     (0.646)         &     (0.567)         &     (0.526)         &     (0.537)         &     (0.647)         \\
 
\bottomrule \end{tabular} } \begin{tablenotes} \item \scriptsize \emph{Notes:} Clustered standard errors in parentheses. All regression are run with CG2 (i.e. the cohort prior to the reform) and with month-of-birth FEs. Ratios indicate cases per thousand; either approximated population (with weights coming from the original fertility distribution) or original number of births. \end{tablenotes} \end{threeparttable} \end{table} 

 \begin{table}[H] \begin{threeparttable} \centering \caption{Robustness with respect to the choice of \texttt{control group}} {\def\sym#1{\ifmmode^{#1}\else\(^{#1}\)\fi} \begin{tabular}{l*{10}{c}} \toprule & \multicolumn{9}{c}{Dependent variable: \textbf{Diseases of the eye and ear}} \\ \cmidrule(lr){2-10}
            &\multicolumn{3}{c}{Average Causal Effects}&\multicolumn{3}{c}{Women}             &\multicolumn{3}{c}{Men}               \\\cmidrule(lr){2-4}\cmidrule(lr){5-7}\cmidrule(lr){8-10}
            &\multicolumn{1}{c}{(1)}&\multicolumn{1}{c}{(2)}&\multicolumn{1}{c}{(3)}&\multicolumn{1}{c}{(4)}&\multicolumn{1}{c}{(5)}&\multicolumn{1}{c}{(6)}&\multicolumn{1}{c}{(7)}&\multicolumn{1}{c}{(8)}&\multicolumn{1}{c}{(9)}\\
            &\multicolumn{1}{c}{C2}&\multicolumn{1}{c}{C1+C2}&\multicolumn{1}{c}{C1-C3}&\multicolumn{1}{c}{C2}&\multicolumn{1}{c}{C1+C2}&\multicolumn{1}{c}{C1-C3}&\multicolumn{1}{c}{C2}&\multicolumn{1}{c}{C1+C2}&\multicolumn{1}{c}{C1-C3}\\
\midrule
 \multicolumn{10}{l}{\emph{Panel A. 2 Month bandwidth}} \\ Abs. numbers        &      -14.11\sym{***}&      -5.444         &      -4.185         &      -4.500\sym{***}&      -2.806\sym{*}  &      -0.759         &      -9.611\sym{***}&      -2.639         &      -3.426         \\
                    &     (2.409)         &     (4.293)         &     (3.514)         &     (0.441)         &     (1.461)         &     (1.824)         &     (1.973)         &     (3.562)         &     (2.614)         \\
 Ratio fertility     &     -0.0905\sym{***}&     -0.0107         &     -0.0208         &      -0.105\sym{***}&     -0.0911\sym{***}&     -0.0795         &     -0.0762\sym{***}&      0.0658         &      0.0351         \\
                    &    (0.0118)         &    (0.0353)         &    (0.0483)         &    (0.0113)         &    (0.0137)         &    (0.0511)         &    (0.0155)         &    (0.0590)         &    (0.0563)         \\
 Ratio population    &      -0.138\sym{***}&     -0.0191         &     -0.0107         &     -0.0755\sym{***}&     -0.0389         &      0.0117         &      -0.200\sym{***}&    0.000845         &     -0.0333         \\
                    &    (0.0343)         &    (0.0597)         &    (0.0601)         &    (0.0147)         &    (0.0264)         &    (0.0565)         &    (0.0545)         &    (0.0956)         &    (0.0802)         \\
 \midrule\multicolumn{10}{l}{\emph{Panel B. 4 Month bandwidth}} \\ Abs. numbers        &      -9.194\sym{**} &      -1.653         &      -1.852         &      -5.944\sym{***}&      -3.056         &      -2.306         &      -3.250         &       1.403         &       0.454         \\
                    &     (3.382)         &     (3.231)         &     (2.915)         &     (1.941)         &     (2.118)         &     (2.212)         &     (2.793)         &     (2.526)         &     (2.097)         \\
 Ratio fertility     &      -0.102\sym{***}&    -0.00798         &     -0.0229         &      -0.145\sym{***}&     -0.0843\sym{*}  &     -0.0825         &     -0.0604         &      0.0649         &      0.0339         \\
                    &    (0.0221)         &    (0.0344)         &    (0.0408)         &    (0.0357)         &    (0.0458)         &    (0.0535)         &    (0.0363)         &    (0.0421)         &    (0.0420)         \\
 Ratio population    &      -0.147\sym{***}&     -0.0505         &     -0.0403         &      -0.177\sym{***}&      -0.105\sym{*}  &     -0.0730         &      -0.118         &     0.00403         &    -0.00792         \\
                    &    (0.0456)         &    (0.0448)         &    (0.0480)         &    (0.0554)         &    (0.0558)         &    (0.0654)         &    (0.0720)         &    (0.0636)         &    (0.0573)         \\
 \midrule\multicolumn{10}{l}{\emph{Panel C. 6 Month bandwidth}} \\ Abs. numbers        &      -5.944\sym{**} &      -0.306         &      -1.605         &      -3.815\sym{**} &      -1.259         &      -1.241         &      -2.130         &       0.954         &      -0.364         \\
                    &     (2.459)         &     (2.331)         &     (2.241)         &     (1.471)         &     (1.550)         &     (1.568)         &     (1.915)         &     (1.799)         &     (1.641)         \\
 Ratio fertility     &     -0.0736\sym{***}&    -0.00713         &     -0.0231         &      -0.122\sym{***}&     -0.0644         &     -0.0624         &     -0.0281         &      0.0471         &      0.0142         \\
                    &    (0.0216)         &    (0.0292)         &    (0.0335)         &    (0.0341)         &    (0.0389)         &    (0.0442)         &    (0.0292)         &    (0.0341)         &    (0.0354)         \\
 Ratio population    &      -0.106\sym{***}&     -0.0301         &     -0.0395         &      -0.144\sym{***}&     -0.0633         &     -0.0524         &     -0.0674         &     0.00363         &     -0.0262         \\
                    &    (0.0339)         &    (0.0366)         &    (0.0421)         &    (0.0441)         &    (0.0442)         &    (0.0500)         &    (0.0512)         &    (0.0506)         &    (0.0516)         \\
 \midrule\multicolumn{10}{l}{\emph{Panel D. Donut specification}} \\ Abs. numbers        &      -3.511         &       1.144         &      -1.148         &      -3.556\sym{**} &      -1.300         &      -1.785         &      0.0444         &       2.444         &       0.637         \\
                    &     (2.035)         &     (2.044)         &     (2.248)         &     (1.606)         &     (1.669)         &     (1.641)         &     (1.716)         &     (1.748)         &     (1.758)         \\
 Ratio fertility     &     -0.0719\sym{***}&    -0.00690         &     -0.0231         &      -0.129\sym{***}&     -0.0609         &     -0.0598         &     -0.0174         &      0.0446         &      0.0120         \\
                    &    (0.0212)         &    (0.0303)         &    (0.0332)         &    (0.0373)         &    (0.0426)         &    (0.0453)         &    (0.0319)         &    (0.0360)         &    (0.0376)         \\
 Ratio fertility     &      -0.103\sym{**} &    -0.00846         &     -0.0300         &      -0.161\sym{***}&     -0.0626         &     -0.0643         &     -0.0473         &      0.0430         &     0.00268         \\
                    &    (0.0364)         &    (0.0398)         &    (0.0491)         &    (0.0508)         &    (0.0571)         &    (0.0620)         &    (0.0552)         &    (0.0549)         &    (0.0612)         \\
 
\bottomrule \end{tabular} } \begin{tablenotes} \item \scriptsize \emph{Notes:} Clustered standard errors in parentheses. All regressions contain Birthmonth FE. Ratios indicate cases per thousand; either approximated population (with weights coming from the original fertility distribution) or original number of births. \end{tablenotes} \end{threeparttable} \end{table} 

%---------------------------------
% Life-course figure (Panel1)
\begin{landscape}
\begin{figure}[H]
\centering
\begin{minipage}{.9\linewidth}
\includegraphics[width=\linewidth]{lc_d7_overview_panel1}
{\scriptsize \emph{Notes:} The figures depict DDRD estimates and 90\% confidence intervals over the life-course. The years are harmonized such that the cohorts are in the same age when they are compared. All regressions are carried out with month-of-birth FE and make use of clustered standard errors. Furthermore, we used a bandwidth of half a year and only the control cohort that was born one year prior to the reform. Ratios indicate cases per thousand; using in the denominator the approximated population (with weights coming from the original fertility distribution) or original number of births. \par}
\end{minipage}
\end{figure}
\end{landscape}
%---------------------------------
% Life-course figure (Panel2)
\begin{landscape}
\begin{figure}[H]
\centering
\begin{minipage}{.9\linewidth}
\includegraphics[width=\linewidth]{lc_d7_overview_panel2}
{\scriptsize \emph{Notes:} The figures depict DDRD estimates and 90\% confidence intervals over the life-course. The years are harmonized such that the cohorts are in the same age when they are compared. All regressions are carried out with month-of-birth FE and make use of clustered standard errors. Furthermore, we used a bandwidth of half a year. Ratios indicate cases per thousand; using in the denominator the approximated population (with weights coming from the original fertility distribution) or original number of births. \par}
\end{minipage}
\end{figure}
\end{landscape}
%---------------------------------
% Life-course (panel 3 - 6)
\begin{figure}[H]%\vspace*{-2cm}
	\centering
	\includegraphics[width=.9\linewidth]{lc_d7_overview_panel3}
	\includegraphics[width=.9\linewidth]{lc_d7_overview_panel4}
\end{figure}
\begin{figure}[H]
	\centering	
	\includegraphics[width=.97\linewidth]{lc_d7_overview_panel5}
	\includegraphics[width=.97\linewidth]{lc_d7_overview_panel6}
\end{figure}
% Life-course TABLE Format
 \begin{table}[H] \centering \begin{threeparttable} \caption{Life-course approach - Table format} {\def\sym#1{\ifmmode^{#1}\else\(^{#1}\)\fi} \begin{tabular}{l*{5}{c}} \toprule \multicolumn{5}{l}{Dep. variable: \textbf{Diseases of the eye and ear}} \\ & \multicolumn{4}{c}{Estimation window} \\ \cmidrule(lr){2-5}
            &\multicolumn{1}{c}{(1)}&\multicolumn{1}{c}{(2)}&\multicolumn{1}{c}{(3)}&\multicolumn{1}{c}{(4)}\\
            &\multicolumn{1}{c}{Age 17-21}&\multicolumn{1}{c}{Age 22-26}&\multicolumn{1}{c}{Age 27-31}&\multicolumn{1}{c}{Age 32-35}\\
\midrule
 \multicolumn{5}{l}{\emph{Panel A. Average causal effects}} \\ Abs. numbers        &      -2.600         &      -0.500         &      -3.900         &      -7.125         \\
                    &     (5.655)         &     (6.391)         &     (6.157)         &     (8.349)         \\
 Ratio fertility     &      -0.168         &     -0.0990         &      -0.172         &      -0.261         \\
                    &     (0.122)         &     (0.105)         &     (0.118)         &     (0.197)         \\
 Ratio population    &     -0.0509         &      -0.126         &      -0.193         \\
                    &    (0.0933)         &    (0.0930)         &     (0.151)         \\
 Cum. numbers        &       11.17         &       11.63         &      -4.167         &      -29.71         \\
                    &     (18.25)         &     (37.09)         &     (56.29)         &     (58.43)         \\
 Cum. ratio          &      -0.258         &      -0.755         &      -1.516\sym{*}  &      -2.522\sym{**} \\
                    &     (0.410)         &     (0.640)         &     (0.907)         &     (1.026)         \\
 \midrule\multicolumn{5}{l}{\emph{Panel B. Treatment effect heterogeneity - Women}} \\ Abs. numbers        &      -5.000         &       1.200         &      -2.500         &      -4.500         \\
                    &     (3.149)         &     (3.757)         &     (5.437)         &     (6.414)         \\
 Ratio fertility     &      -0.343\sym{**} &     -0.0591         &      -0.217         &      -0.323         \\
                    &     (0.134)         &     (0.152)         &     (0.207)         &     (0.301)         \\
 Ratio population    &     -0.0447         &      -0.155         &      -0.232         \\
                    &     (0.175)         &     (0.156)         &     (0.223)         \\
 Cum. numbers        &      -7.933         &      -8.833         &      -20.87         &      -36.79         \\
                    &     (14.68)         &     (22.29)         &     (33.90)         &     (45.07)         \\
 Cum. ratio          &      -0.881         &      -1.543\sym{*}  &      -2.586\sym{**} &      -3.823\sym{**} \\
                    &     (0.630)         &     (0.901)         &     (1.210)         &     (1.843)         \\
 \midrule\multicolumn{5}{l}{\emph{Panel C. Treatment effect heterogeneity - Men}} \\ Abs. numbers        &       2.400         &      -1.700         &      -1.400         &      -2.625         \\
                    &     (5.276)         &     (5.561)         &     (2.972)         &     (6.696)         \\
 Ratio fertility     &    -0.00121         &      -0.137         &      -0.133         &      -0.203         \\
                    &     (0.217)         &     (0.193)         &     (0.133)         &     (0.282)         \\
 Ratio population    &     -0.0569         &     -0.0983         &      -0.155         \\
                    &     (0.200)         &     (0.106)         &     (0.223)         \\
 Cum. numbers        &       19.10         &       20.47         &       16.70         &       7.083         \\
                    &     (18.18)         &     (35.13)         &     (44.49)         &     (45.96)         \\
 Cum. ratio          &       0.335         &    -0.00714         &      -0.507         &      -1.301         \\
                    &     (0.769)         &     (1.272)         &     (1.577)         &     (1.675)         \\
 
\bottomrule \end{tabular} } \begin{tablenotes} \item \scriptsize \emph{Notes:} Clustered standard errors in parentheses. All regression are run with CG2 (i.e. the cohort prior to the reform) and with month-of-birth FEs. Ratios indicate cases per thousand; either approximated population (with weights coming from the original fertility distribution) or original number of births. Raqtio population muss eins nach rechts gerückt werden \end{tablenotes} \end{threeparttable} \end{table} 

%---------------------------------
% PLACEBO EXERCISES
\newpage
\begin{landscape}
\begin{figure}[H]
	\centering
    \begin{minipage}{.9\linewidth}
	\includegraphics[width=\linewidth]{placebo_graph_d7.pdf}
    {\scriptsize \emph{Notes:} The figures depict DDRD estimates and 95\% confidence intervals when the treatment cohort is shifted over time. The date on the abscissa indicates the starting date of the treated.  All regressions are carried out with month-of-birth FE and make use of clustered standard errors. Furthermore, we used a bandwidth of half a year. Ratios indicate cases per thousand; using in the denominator the approximated population (with weights coming from the original fertility distribution) or original number of births. \par}
    \end{minipage}
\end{figure}
\end{landscape}
 \begin{table}[H] \centering \begin{threeparttable} \caption{Placebo 1 (CONTROL1 ist TREAT) } {\def\sym#1{\ifmmode^{#1}\else\(^{#1}\)\fi} \begin{tabular}{l*{4}{c}} \toprule \multicolumn{4}{l}{Dep. variable: \textbf{Diseases of the eye and ear}} \\ & \multicolumn{3}{c}{Choice of control group} \\ \cmidrule(lr){2-4}
            &\multicolumn{1}{c}{(1)}&\multicolumn{1}{c}{(2)}&\multicolumn{1}{c}{(3)}\\
            &\multicolumn{1}{c}{C2}&\multicolumn{1}{c}{C3}&\multicolumn{1}{c}{C2+C3}\\
\midrule
 \multicolumn{4}{l}{\emph{Panel A. Average causal effects}} \\ Abs. numbers        &      -5.667\sym{**} &      -8.192\sym{***}&      -6.929\sym{***}\\
                    &     (2.239)         &     (1.553)         &     (1.941)         \\
 Ratio fertility     &      -0.147\sym{***}&     -0.0757\sym{**} &      -0.112\sym{**} \\
                    &    (0.0477)         &    (0.0274)         &    (0.0513)         \\
 Ratio population    &      -0.142\sym{***}&     -0.0730\sym{***}&      -0.108\sym{**} \\
                    &    (0.0396)         &    (0.0215)         &    (0.0445)         \\
 Cum. numbers        &      -39.32         &      -70.91\sym{***}&      -55.11\sym{**} \\
                    &     (28.40)         &     (16.22)         &     (23.11)         \\
 Cum. ratio          &      -1.120\sym{*}  &      -0.465         &      -0.793         \\
                    &     (0.566)         &     (0.313)         &     (0.529)         \\
 \midrule\multicolumn{4}{l}{\emph{Panel B. Treatment effect heterogeneity - Women}} \\ Abs. numbers        &      -1.917\sym{*}  &      -1.933\sym{*}  &      -1.925\sym{*}  \\
                    &     (1.014)         &     (1.033)         &     (1.012)         \\
 Ratio fertility     &      -0.117\sym{**} &      0.0140         &     -0.0514         \\
                    &    (0.0454)         &    (0.0338)         &    (0.0526)         \\
 Ratio population    &      -0.149\sym{***}&     -0.0150         &     -0.0820\sym{*}  \\
                    &    (0.0314)         &    (0.0316)         &    (0.0439)         \\
 Cum. numbers        &      -6.150         &      -11.53         &      -8.842         \\
                    &     (13.51)         &     (11.29)         &     (12.37)         \\
 Cum. ratio          &      -0.627         &       0.531         &     -0.0481         \\
                    &     (0.577)         &     (0.361)         &     (0.555)         \\
 \midrule\multicolumn{4}{l}{\emph{Panel C. Treatment effect heterogeneity - Men}} \\ Abs. numbers        &      -3.750\sym{**} &      -6.258\sym{***}&      -5.004\sym{***}\\
                    &     (1.735)         &     (1.128)         &     (1.486)         \\
 Ratio fertility     &      -0.176\sym{**} &      -0.161\sym{***}&      -0.169\sym{**} \\
                    &    (0.0692)         &    (0.0436)         &    (0.0668)         \\
 Ratio population    &      -0.136\sym{*}  &      -0.132\sym{**} &      -0.134\sym{*}  \\
                    &    (0.0704)         &    (0.0494)         &    (0.0688)         \\
 Cum. numbers        &      -33.17         &      -59.38\sym{***}&      -46.27\sym{***}\\
                    &     (20.83)         &     (10.66)         &     (16.58)         \\
 Cum. ratio          &      -1.585\sym{*}  &      -1.414\sym{***}&      -1.499\sym{**} \\
                    &     (0.786)         &     (0.456)         &     (0.692)         \\
 
\bottomrule \end{tabular} } \begin{tablenotes} \item \scriptsize \emph{Notes:} Clustered standard errors in parentheses. All regression are run with month-of-birth FEs and control cohort 2 is assigned with the treatment status. All regressions are carried out with a window width of half a year. \end{tablenotes} \end{threeparttable} \end{table} 

 \begin{table}[H] \centering \begin{threeparttable} \caption{Placebo 2 (CONTROL2 ist TREAT) } {\def\sym#1{\ifmmode^{#1}\else\(^{#1}\)\fi} \begin{tabular}{l*{4}{c}} \toprule \multicolumn{4}{l}{Dep. variable: \textbf{Diseases of the eye and ear}} \\ & \multicolumn{3}{c}{Choice of control group} \\ \cmidrule(lr){2-4}
            &\multicolumn{1}{c}{(1)}&\multicolumn{1}{c}{(2)}&\multicolumn{1}{c}{(3)}\\
            &\multicolumn{1}{c}{C1}&\multicolumn{1}{c}{C3}&\multicolumn{1}{c}{C1+C3}\\
\midrule
 \multicolumn{4}{l}{\emph{Panel A. Average causal effects}} \\ Abs. numbers        &       5.667\sym{**} &      -2.525         &       1.571         \\
                    &     (2.239)         &     (1.863)         &     (2.307)         \\
 Ratio fertility     &       0.147\sym{***}&      0.0717\sym{**} &       0.110\sym{*}  \\
                    &    (0.0477)         &    (0.0269)         &    (0.0541)         \\
 Ratio population    &       0.142\sym{***}&      0.0693\sym{**} &       0.106\sym{*}  \\
                    &    (0.0396)         &    (0.0302)         &    (0.0535)         \\
 Cum. numbers        &       39.32         &      -31.59         &       3.863         \\
                    &     (28.40)         &     (26.25)         &     (28.10)         \\
 Cum. ratio          &       1.120\sym{*}  &       0.656         &       0.888         \\
                    &     (0.566)         &     (0.434)         &     (0.580)         \\
 \midrule\multicolumn{4}{l}{\emph{Panel B. Treatment effect heterogeneity - Women}} \\ Abs. numbers        &       1.917\sym{*}  &     -0.0167         &       0.950         \\
                    &     (1.014)         &     (1.328)         &     (1.209)         \\
 Ratio fertility     &       0.117\sym{**} &       0.131\sym{***}&       0.124\sym{**} \\
                    &    (0.0454)         &    (0.0353)         &    (0.0541)         \\
 Ratio population    &       0.149\sym{***}&       0.134\sym{***}&       0.142\sym{***}\\
                    &    (0.0314)         &    (0.0345)         &    (0.0469)         \\
 Cum. numbers        &       6.150         &      -5.383         &       0.383         \\
                    &     (13.51)         &     (17.16)         &     (15.35)         \\
 Cum. ratio          &       0.627         &       1.157\sym{**} &       0.892         \\
                    &     (0.577)         &     (0.514)         &     (0.631)         \\
 \midrule\multicolumn{4}{l}{\emph{Panel C. Treatment effect heterogeneity - Men}} \\ Abs. numbers        &       3.750\sym{**} &      -2.508\sym{*}  &       0.621         \\
                    &     (1.735)         &     (1.299)         &     (1.704)         \\
 Ratio fertility     &       0.176\sym{**} &      0.0152         &      0.0957         \\
                    &    (0.0692)         &    (0.0501)         &    (0.0731)         \\
 Ratio population    &       0.136\sym{*}  &     0.00441         &      0.0703         \\
                    &    (0.0704)         &    (0.0493)         &    (0.0770)         \\
 Cum. numbers        &       33.17         &      -26.21         &       3.479         \\
                    &     (20.83)         &     (17.91)         &     (20.20)         \\
 Cum. ratio          &       1.585\sym{*}  &       0.171         &       0.878         \\
                    &     (0.786)         &     (0.684)         &     (0.790)         \\
 
\bottomrule \end{tabular} } \begin{tablenotes} \item \scriptsize \emph{Notes:} Clustered standard errors in parentheses. All regression are run with month-of-birth FEs and control cohort 2 is assigned with the treatment status. All regressions are carried out with a window width of half a year. \end{tablenotes} \end{threeparttable} \end{table} 

 \begin{table}[H] \centering \begin{threeparttable} \caption{Placebo 3 (CONTROL3 ist TREAT) } {\def\sym#1{\ifmmode^{#1}\else\(^{#1}\)\fi} \begin{tabular}{l*{4}{c}} \toprule \multicolumn{4}{l}{Dep. variable: \textbf{Diseases of the eye and ear}} \\ & \multicolumn{3}{c}{Choice of control group} \\ \cmidrule(lr){2-4}
            &\multicolumn{1}{c}{(1)}&\multicolumn{1}{c}{(2)}&\multicolumn{1}{c}{(3)}\\
            &\multicolumn{1}{c}{C1}&\multicolumn{1}{c}{C2}&\multicolumn{1}{c}{C1+C2}\\
\midrule
 \multicolumn{4}{l}{\emph{Panel A. Average causal effects}} \\ Abs. numbers        &       8.192\sym{***}&       2.525         &       5.358\sym{***}\\
                    &     (1.553)         &     (1.863)         &     (1.840)         \\
 Ratio fertility     &      0.0757\sym{**} &     -0.0717\sym{**} &     0.00201         \\
                    &    (0.0274)         &    (0.0269)         &    (0.0312)         \\
 Ratio population    &      0.0730\sym{***}&     -0.0693\sym{**} &     0.00186         \\
                    &    (0.0215)         &    (0.0302)         &    (0.0313)         \\
 Cum. numbers        &       70.91\sym{***}&       31.59         &       51.25\sym{**} \\
                    &     (16.22)         &     (26.25)         &     (22.19)         \\
 Cum. ratio          &       0.465         &      -0.656         &     -0.0956         \\
                    &     (0.313)         &     (0.434)         &     (0.392)         \\
 \midrule\multicolumn{4}{l}{\emph{Panel B. Treatment effect heterogeneity - Women}} \\ Abs. numbers        &       1.933\sym{*}  &      0.0167         &       0.975         \\
                    &     (1.033)         &     (1.328)         &     (1.208)         \\
 Ratio fertility     &     -0.0140         &      -0.131\sym{***}&     -0.0723\sym{*}  \\
                    &    (0.0338)         &    (0.0353)         &    (0.0365)         \\
 Ratio population    &      0.0150         &      -0.134\sym{***}&     -0.0595         \\
                    &    (0.0316)         &    (0.0345)         &    (0.0367)         \\
 Cum. numbers        &       11.53         &       5.383         &       8.458         \\
                    &     (11.29)         &     (17.16)         &     (14.56)         \\
 Cum. ratio          &      -0.531         &      -1.157\sym{**} &      -0.844\sym{*}  \\
                    &     (0.361)         &     (0.514)         &     (0.448)         \\
 \midrule\multicolumn{4}{l}{\emph{Panel C. Treatment effect heterogeneity - Men}} \\ Abs. numbers        &       6.258\sym{***}&       2.508\sym{*}  &       4.383\sym{***}\\
                    &     (1.128)         &     (1.299)         &     (1.282)         \\
 Ratio fertility     &       0.161\sym{***}&     -0.0152         &      0.0729         \\
                    &    (0.0436)         &    (0.0501)         &    (0.0501)         \\
 Ratio population    &       0.132\sym{**} &    -0.00441         &      0.0637         \\
                    &    (0.0494)         &    (0.0493)         &    (0.0522)         \\
 Cum. numbers        &       59.37\sym{***}&       26.21         &       42.79\sym{***}\\
                    &     (10.66)         &     (17.91)         &     (14.99)         \\
 Cum. ratio          &       1.414\sym{***}&      -0.171         &       0.621         \\
                    &     (0.456)         &     (0.684)         &     (0.598)         \\
 
\bottomrule \end{tabular} } \begin{tablenotes} \item \scriptsize \emph{Notes:} Clustered standard errors in parentheses. All regression are run with month-of-birth FEs and control cohort 3 is assigned with the treatment status. All regressions are carried out with a window width of half a year. \end{tablenotes} \end{threeparttable} \end{table} 

%---------------------------------
% CUMMULATIVE APPROACH
\begin{landscape}
 \begin{table}[H] \begin{threeparttable} \centering \caption{Cummulative effects for upt to different points of age} {\def\sym#1{\ifmmode^{#1}\else\(^{#1}\)\fi} \begin{tabular}{l*{13}{c}} \toprule & \multicolumn{12}{c}{Dependent variable: \textbf{Diseases of the eye and ear}} \\ \cmidrule(lr){2-13}
            &\multicolumn{4}{c}{Average Causal Effects}         &\multicolumn{4}{c}{Women}                          &\multicolumn{4}{c}{Men}                            \\\cmidrule(lr){2-5}\cmidrule(lr){6-9}\cmidrule(lr){10-13}
            &\multicolumn{1}{c}{(1)}&\multicolumn{1}{c}{(2)}&\multicolumn{1}{c}{(3)}&\multicolumn{1}{c}{(4)}&\multicolumn{1}{c}{(5)}&\multicolumn{1}{c}{(6)}&\multicolumn{1}{c}{(7)}&\multicolumn{1}{c}{(8)}&\multicolumn{1}{c}{(9)}&\multicolumn{1}{c}{(10)}&\multicolumn{1}{c}{(11)}&\multicolumn{1}{c}{(12)}\\
            &\multicolumn{1}{c}{2M}&\multicolumn{1}{c}{4M}&\multicolumn{1}{c}{6M}&\multicolumn{1}{c}{Donut}&\multicolumn{1}{c}{2M}&\multicolumn{1}{c}{4M}&\multicolumn{1}{c}{6M}&\multicolumn{1}{c}{Donut}&\multicolumn{1}{c}{2M}&\multicolumn{1}{c}{4M}&\multicolumn{1}{c}{6M}&\multicolumn{1}{c}{Donut}\\
\midrule
 \multicolumn{13}{l}{\emph{Panel A. 2 Up to the age of 21}} \\ Cum. numbers        &       4.000         &      -18.50         &       8.000         &       9.600         &      -26.00\sym{***}&         -38\sym{***}&      -17.67         &      -17.80         &       30.00         &       19.50         &       25.67         &       27.40         \\
                    &     (24.21)         &     (21.09)         &     (21.16)         &     (24.76)         &     (6.708)         &     (6.655)         &     (12.59)         &     (15.41)         &     (23.77)         &     (19.95)         &     (15.82)         &     (17.62)         \\
 Cum. ratio          &      -0.105         &      -0.905         &      -0.559         &      -0.632         &      -1.336\sym{***}&      -2.188\sym{***}&      -1.574\sym{***}&      -1.684\sym{**} &       1.062         &       0.318         &       0.406         &       0.370         \\
                    &     (0.726)         &     (0.582)         &     (0.436)         &     (0.507)         &     (0.372)         &     (0.392)         &     (0.497)         &     (0.601)         &     (1.173)         &     (0.912)         &     (0.659)         &     (0.737)         \\
 \midrule\multicolumn{13}{l}{\emph{Panel B. Up to the age of 26}} \\ Cum. numbers        &      -16.50         &      -31.25         &       5.500         &       5.000         &      -31.50\sym{***}&      -23.50         &      -11.67         &      -10.00         &       15.00         &      -7.750         &       17.17         &          15         \\
                    &     (17.50)         &     (28.38)         &     (34.57)         &     (41.77)         &     (8.139)         &     (19.10)         &     (19.51)         &     (23.77)         &     (9.434)         &     (29.09)         &     (28.51)         &     (34.53)         \\
 Cum. ratio          &      -0.676         &      -1.465\sym{***}&      -1.054\sym{*}  &      -1.249\sym{*}  &      -1.750\sym{**} &      -1.933\sym{***}&      -1.869\sym{**} &      -1.990\sym{**} &       0.337         &      -1.013         &      -0.279         &      -0.541         \\
                    &     (0.554)         &     (0.458)         &     (0.520)         &     (0.605)         &     (0.502)         &     (0.610)         &     (0.718)         &     (0.873)         &     (0.599)         &     (1.064)         &     (0.981)         &     (1.165)         \\
 \midrule\multicolumn{13}{l}{\emph{Panel C. Up to the age of 31}} \\ Cum. numbers        &      -95.00\sym{*}  &      -92.25\sym{**} &      -14.00         &      -9.000         &      -84.00\sym{**} &      -71.75\sym{**} &      -24.17         &      -20.40         &      -11.00         &      -20.50         &       10.17         &       11.40         \\
                    &     (42.54)         &     (32.61)         &     (50.66)         &     (61.85)         &     (28.99)         &     (27.68)         &     (31.81)         &     (38.81)         &     (18.83)         &     (30.89)         &     (32.17)         &     (39.25)         \\
 Cum. ratio          &      -2.415\sym{**} &      -3.040\sym{***}&      -1.916\sym{**} &      -2.076\sym{**} &      -4.128\sym{**} &      -4.320\sym{***}&      -2.953\sym{**} &      -3.066\sym{**} &      -0.800         &      -1.821         &      -0.944         &      -1.147         \\
                    &     (0.957)         &     (0.611)         &     (0.789)         &     (0.950)         &     (1.345)         &     (0.948)         &     (1.067)         &     (1.301)         &     (0.617)         &     (1.248)         &     (1.118)         &     (1.349)         \\
 \midrule\multicolumn{13}{l}{\emph{Panel D. Up to the age of 34}} \\ Cum. numbers        &      -140.5\sym{***}&      -107.3\sym{***}&      -42.50         &      -24.20         &      -72.00\sym{***}&      -75.50\sym{**} &      -42.17         &      -38.00         &      -68.50\sym{**} &      -31.75         &      -0.333         &       13.80         \\
                    &     (33.20)         &     (30.90)         &     (46.43)         &     (51.86)         &     (10.61)         &     (32.86)         &     (32.22)         &     (37.89)         &     (24.46)         &     (37.42)         &     (34.44)         &     (39.80)         \\
 Cum. ratio          &      -3.471\sym{***}&      -3.653\sym{***}&      -2.959\sym{***}&      -2.927\sym{***}&      -3.815\sym{***}&      -4.806\sym{***}&      -4.244\sym{***}&      -4.410\sym{***}&      -3.158\sym{***}&      -2.555         &      -1.758         &      -1.534         \\
                    &     (0.255)         &     (0.943)         &     (0.811)         &     (0.895)         &     (0.460)         &     (1.512)         &     (1.288)         &     (1.466)         &     (0.311)         &     (1.655)         &     (1.250)         &     (1.488)         \\
 
\bottomrule \end{tabular} } \begin{tablenotes} \item \scriptsize \emph{Notes:} Clustered standard errors in parentheses (MxY). All regressions contain Birthmonth FE. Ratios indicate cases per thousand; original number of births. \end{tablenotes} \end{threeparttable} \end{table} 

\end{landscape}
\begin{landscape}
 \begin{table}[H] \begin{threeparttable} \centering \caption{Cummulative effects for upt to different points of age - BOOTSTRAPPED} {\def\sym#1{\ifmmode^{#1}\else\(^{#1}\)\fi} \begin{tabular}{l*{13}{c}} \toprule & \multicolumn{12}{c}{Dependent variable: \textbf{Diseases of the eye and ear}} \\ \cmidrule(lr){2-13}
            &\multicolumn{4}{c}{Average Causal Effects}         &\multicolumn{4}{c}{Women}                          &\multicolumn{4}{c}{Men}                            \\\cmidrule(lr){2-5}\cmidrule(lr){6-9}\cmidrule(lr){10-13}
            &\multicolumn{1}{c}{(1)}&\multicolumn{1}{c}{(2)}&\multicolumn{1}{c}{(3)}&\multicolumn{1}{c}{(4)}&\multicolumn{1}{c}{(5)}&\multicolumn{1}{c}{(6)}&\multicolumn{1}{c}{(7)}&\multicolumn{1}{c}{(8)}&\multicolumn{1}{c}{(9)}&\multicolumn{1}{c}{(10)}&\multicolumn{1}{c}{(11)}&\multicolumn{1}{c}{(12)}\\
            &\multicolumn{1}{c}{2M}&\multicolumn{1}{c}{4M}&\multicolumn{1}{c}{6M}&\multicolumn{1}{c}{Donut}&\multicolumn{1}{c}{2M}&\multicolumn{1}{c}{4M}&\multicolumn{1}{c}{6M}&\multicolumn{1}{c}{Donut}&\multicolumn{1}{c}{2M}&\multicolumn{1}{c}{4M}&\multicolumn{1}{c}{6M}&\multicolumn{1}{c}{Donut}\\
\midrule
 \multicolumn{13}{l}{\emph{Panel A. 2 Up to the age of 21}} \\ Cum. numbers        &       4.000         &      -18.50         &       8.000         &       9.600         &      -26.00\sym{***}&         -38\sym{***}&      -17.67         &      -17.80         &       30.00         &       19.50         &       25.67         &       27.40         \\
                    &     (23.86)         &     (27.01)         &     (27.47)         &     (31.41)         &     (5.894)         &     (8.821)         &     (15.84)         &     (17.90)         &     (22.97)         &     (26.88)         &     (24.41)         &     (22.52)         \\
 Cum. ratio          &      -0.105         &      -0.905         &      -0.559         &      -0.632         &      -1.336\sym{***}&      -2.188\sym{***}&      -1.574\sym{**} &      -1.684\sym{**} &       1.062         &       0.318         &       0.406         &       0.370         \\
                    &     (0.716)         &     (0.791)         &     (0.622)         &     (0.688)         &     (0.354)         &     (0.473)         &     (0.673)         &     (0.800)         &     (1.144)         &     (1.289)         &     (1.043)         &     (0.924)         \\
 \midrule\multicolumn{13}{l}{\emph{Panel B. Up to the age of 26}} \\ Cum. numbers        &      -16.50         &      -31.25         &       5.500         &       5.000         &      -31.50\sym{***}&      -23.50         &      -11.67         &      -10.00         &       15.00\sym{*}  &      -7.750         &       17.17         &          15         \\
                    &     (15.08)         &     (38.38)         &     (44.04)         &     (51.66)         &     (7.006)         &     (27.92)         &     (26.35)         &     (26.64)         &     (8.160)         &     (35.72)         &     (41.30)         &     (45.30)         \\
 Cum. ratio          &      -0.676         &      -1.465\sym{**} &      -1.054         &      -1.249         &      -1.750\sym{***}&      -1.933\sym{**} &      -1.869\sym{*}  &      -1.990\sym{*}  &       0.337         &      -1.013         &      -0.279         &      -0.541         \\
                    &     (0.511)         &     (0.569)         &     (0.696)         &     (0.763)         &     (0.470)         &     (0.799)         &     (1.064)         &     (1.083)         &     (0.546)         &     (1.343)         &     (1.446)         &     (1.492)         \\
 \midrule\multicolumn{13}{l}{\emph{Panel C. Up to the age of 31}} \\ Cum. numbers        &      -95.00\sym{**} &      -92.25\sym{**} &      -14.00         &      -9.000         &      -84.00\sym{***}&      -71.75\sym{*}  &      -24.17         &      -20.40         &      -11.00         &      -20.50         &       10.17         &       11.40         \\
                    &     (37.21)         &     (42.47)         &     (62.54)         &     (74.83)         &     (24.71)         &     (39.62)         &     (43.25)         &     (41.18)         &     (17.78)         &     (38.37)         &     (43.03)         &     (51.94)         \\
 Cum. ratio          &      -2.415\sym{***}&      -3.040\sym{***}&      -1.916\sym{*}  &      -2.076\sym{*}  &      -4.128\sym{***}&      -4.320\sym{***}&      -2.953\sym{*}  &      -3.066\sym{**} &      -0.800         &      -1.821         &      -0.944         &      -1.147         \\
                    &     (0.837)         &     (0.786)         &     (0.990)         &     (1.226)         &     (1.195)         &     (1.234)         &     (1.551)         &     (1.536)         &     (0.531)         &     (1.699)         &     (1.511)         &     (1.756)         \\
 \midrule\multicolumn{13}{l}{\emph{Panel D. Up to the age of 34}} \\ Cum. numbers        &      -140.5\sym{***}&      -107.3\sym{***}&      -42.50         &      -24.20         &      -72.00\sym{***}&      -75.50\sym{*}  &      -42.17         &      -38.00         &      -68.50\sym{***}&      -31.75         &      -0.333         &       13.80         \\
                    &     (32.69)         &     (38.29)         &     (57.70)         &     (65.93)         &     (9.956)         &     (44.37)         &     (48.66)         &     (43.98)         &     (24.19)         &     (49.03)         &     (47.80)         &     (52.47)         \\
 Cum. ratio          &      -3.471\sym{***}&      -3.653\sym{***}&      -2.959\sym{***}&      -2.927\sym{**} &      -3.815\sym{***}&      -4.806\sym{**} &      -4.244\sym{**} &      -4.410\sym{**} &      -3.158\sym{***}&      -2.555         &      -1.758         &      -1.534         \\
                    &     (0.222)         &     (1.336)         &     (1.106)         &     (1.338)         &     (0.436)         &     (1.941)         &     (2.067)         &     (2.129)         &     (0.290)         &     (2.424)         &     (1.842)         &     (1.859)         \\
 
\bottomrule \end{tabular} } \begin{tablenotes} \item \scriptsize \emph{Notes:} \textbf{BOOTSTRAPPED} standard errors in parentheses (MxY), with 400 replications. All regressions contain Birthmonth FE. Ratios indicate cases per thousand; original number of births. \end{tablenotes} \end{threeparttable} \end{table} 

\end{landscape}
%---------------------------------
\newpage
FEBRUAR CASES:
 \begin{table}[H] \begin{threeparttable} \centering \caption{Dep. variable: \textbf{Diseases of the eye and ear}} {\def\sym#1{\ifmmode^{#1}\else\(^{#1}\)\fi} \begin{tabular}{l*{13}{c}} \toprule year & \multicolumn{12}{c}{Month of birth} \\ \cmidrule(lr){2-13} 
            &          11&          12&           1&           2&           3&           4&           5&           6&           7&           8&           9&          10\\
1995        &          93&         103&         107&          84&         120&         113&         131&         124&         109&         122&         101&          92\\
1996        &         120&         120&         115&         107&         108&         128&         136&          93&         111&         105&         106&          85\\
1997        &         104&         104&         142&          93&         104&         119&         118&         122&         133&         119&         112&         110\\
1998        &         141&         120&         119&         102&         151&         139&         143&         119&         125&         125&         127&         122\\
1999        &         124&         114&         124&         122&         142&         141&         132&         140&         147&         134&         131&         124\\
2000        &         129&         113&         124&         116&         140&         146&         121&         139&         124&         123&          92&         123\\
2001        &         124&         106&         122&         136&         134&         131&         136&         149&         121&         119&         144&         125\\
2002        &         114&          79&         107&         112&         129&         122&         102&         128&         128&         113&         137&         115\\
2003        &         103&         124&         103&         126&         122&         105&         120&         116&         115&         106&         118&          94\\
2004        &          94&          84&          83&          82&         102&         114&         101&          91&          95&          96&         106&          91\\
2005        &          99&          86&          98&          84&         106&          72&         113&         107&         114&         107&          97&          81\\
2006        &          99&          95&         102&         102&         107&          88&          99&          93&         103&         109&         106&          79\\
2007        &         111&          99&         101&          86&         103&         102&          98&         108&         114&         105&         112&          96\\
2008        &         105&         109&         105&         113&         105&          90&         115&         119&          95&         116&         105&          99\\
2009        &         109&         109&          98&         113&         111&         108&         118&         122&         131&         133&         107&         126\\
2010        &         105&          89&         127&         120&         150&         105&         128&         104&         115&         137&         119&         116\\
2011        &         113&         123&         125&         105&         125&         103&         140&         137&         132&         132&         126&         119\\
2012        &         120&         127&         124&         118&         110&         123&         133&         121&         125&         123&         129&         114\\
2013        &         113&         113&         143&         131&         132&         120&         147&         130&         144&         137&         143&         134\\
2014        &         132&         136&         119&         136&         154&         137&         129&         144&         126&         137&         147&         133\\
 \bottomrule \end{tabular} } \begin{tablenotes} \item \scriptsize \emph{Notes:} Number of cases per year and MOB in treatment cohort. \end{tablenotes} \end{threeparttable} \end{table} 

 \begin{table}[H] \begin{threeparttable} \centering \caption{Dep. variable: \textbf{Diseases of the eye and ear}} {\def\sym#1{\ifmmode^{#1}\else\(^{#1}\)\fi} \begin{tabular}{l*{13}{c}} \toprule year & \multicolumn{12}{c}{Month of birth} \\ \cmidrule(lr){2-13} 
            &          11&          12&           1&           2&           3&           4&           5&           6&           7&           8&           9&          10\\
1995        &           7&           5&          -2&           1&          29&           9&          12&           0&          -2&           0&          -6&         -22\\
1996        &          12&          17&          -1&          -3&         -14&          20&          25&         -23&          12&         -11&         -21&         -23\\
1997        &         -16&         -25&          31&         -26&         -27&         -19&          -8&          13&          30&           3&           6&          -5\\
1998        &          42&          -3&         -18&          -6&           6&           2&           8&          -2&          -4&          -3&          -6&          -7\\
1999        &          13&          -8&           1&          -2&          11&           8&           3&           9&           9&          -5&          15&         -12\\
2000        &          10&         -10&         -23&         -14&          -1&          20&         -25&          -2&           5&          -7&         -34&          10\\
2001        &          31&          11&         -15&          19&          10&         -16&          36&          18&           9&         -14&          28&         -23\\
2002        &          10&         -21&         -11&           8&          -8&          -2&          -6&          14&           1&          -6&          31&           0\\
2003        &          11&          21&           5&          44&          18&          -1&           9&          22&          17&          16&          16&          -5\\
2004        &          10&          -6&         -22&         -15&           5&          28&         -17&          11&          -4&          -1&           6&         -11\\
2005        &          -8&          -5&           6&           2&           5&         -28&          17&          16&          18&          27&           6&         -10\\
2006        &          29&           5&          23&          12&           2&          -4&          -6&           1&          17&          15&           8&         -12\\
2007        &          11&          13&           3&         -23&           1&           4&           3&          13&           8&          12&           0&         -11\\
2008        &          17&           1&         -20&          13&          -5&          -5&          -3&          25&           1&           9&           2&          -8\\
2009        &           3&           1&         -17&          24&         -28&          -1&          19&           1&          28&          15&         -14&          26\\
2010        &           2&         -16&          22&          15&           8&         -14&          -4&           1&         -10&          10&          -7&           3\\
2011        &          15&          17&          14&          -7&          -4&         -36&           7&          12&           7&          32&          27&          -3\\
2012        &          -6&           2&          18&          -4&           2&           5&          13&          -4&           0&           4&          15&           5\\
2013        &         -16&          10&          37&          17&          -8&         -26&          12&          -1&           2&          17&           7&          31\\
2014        &          16&           8&         -12&          18&           4&           5&         -20&          19&          -7&          -2&          29&          17\\
 \bottomrule \end{tabular} } \begin{tablenotes} \item \scriptsize \emph{Notes:} Difference of cases (control - treatment) per year and MOB in treatment cohort. \end{tablenotes} \end{threeparttable} \end{table} 

