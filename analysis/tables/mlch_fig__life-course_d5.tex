%\newgeometry{left=2cm,right=3cm,top=3cm,bottom=3cm} 
\begin{landscape}
	\vspace*{\fill}
	\begin{figure}[H]\centering
		\begin{subfigure}[h]{0.31\linewidth}\centering\caption{Total}
			\includegraphics[width=\linewidth]{paper/mlch_lc_trends_d5.pdf}
		\end{subfigure}
		\begin{subfigure}[h]{0.31\linewidth}\centering\caption{Women}
			\includegraphics[width=\linewidth]{paper/mlch_lc_trends_d5_f.pdf}
		\end{subfigure}
		\begin{subfigure}[h]{0.31\linewidth}\centering\caption{Men}
			\includegraphics[width=\linewidth]{paper/mlch_lc_trends_d5_m.pdf}
		\end{subfigure}
		\scriptsize
		\begin{minipage}{\linewidth}
			\caption{Life-course approach for mental and behavioral disorders}\label{fig_mlch: lc_d5_frg_DD}
			\emph{Notes:} The top panels show pre-threshold (born November-April) means for the treatment (1978/79, solid line) and control cohort (1977/78, dashed line) across the years. The bottom panels plot DiD estimates (along with 90\% and 95\% confidence intervals) for the impact of the reform on mental and behavioral disorders over the life-course. For a given reporting year ($N=24$), I estimate the model in equation \ref{eq_mlch:DD_basline} (without $\rho_t$) and plot the DiD estimate and the corresponding confidence interval for that year. The outcomes are defined as the number of cases per 1,000 individuals. Column a shows the results for all admissions, whereas columns b and c show the estimates for females and males, respectively.
		\end{minipage}
	\end{figure}
	\vspace*{\fill}\clearpage
\end{landscape}
%\restoregeometry