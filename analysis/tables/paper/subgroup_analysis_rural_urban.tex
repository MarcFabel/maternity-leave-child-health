
\begin{table}[t] \centering 
	\begin{threeparttable} \centering 
		\caption{Subgroup analysis} \label{tab_mlch: heterogeneity analysis} 
		{\def\sym#1{\ifmmode^{#1}\else\(^{#1}\)\fi} 
			\begin{tabular}{l*{2}{c}} \toprule 
				
				&  \multicolumn{2}{c}{Heterogeneity}\\
				\cmidrule(lr){2-3} 
				&\multicolumn{1}{c}{(1)}&\multicolumn{1}{c}{(2)}\\
				&\multicolumn{1}{c}{\clb{c}{rural}}&\multicolumn{1}{c}{\clb{c}{urban}}\\
				\midrule
				\\

				\textit{Panel A: Hospital admissions}\\
				DiD estimate 		&	-1.654		 &	-1.799\sym{***} \\
									&	(1.096)		 &	(0.598)			\\

				Dependent mean 		&	101.3		 &	96.50			\\
				Effect in SDs [\%] 	&	3.880		 &	5.600			\\
				$N$ 				&	24,287		 &	29,568			\\
				\\ \\


				\textit{Panel B: Mental and behavioral disorders}\\
				DiD estimate 		&	-0.241		&	-0.986\sym{***} 	\\
									&	(0.564)		&	(0.196)				\\							 
				Dependent mean 		&	17.00  		&	18.61				\\
				Effect in SDs [\%] 	&	1.310		&	7.100				\\
				$N$ 				&	24,287		&	29,568				\\

				\\
				\midrule
				MOB fixed effects 	&	\checkmark	&	\checkmark		    \\ 
				Year fixed effects  &	\checkmark	&	\checkmark		    \\
				Region fixed effects& 	\checkmark	&	\checkmark		    \\
				\bottomrule
		\end{tabular}}
	\end{threeparttable} 
	\begin{minipage}{0.7\linewidth}
		\scriptsize \emph{Notes:} The table shows DiD estimates of the 1979 maternity leave reform on hospital admission for rural and urban areas. The DiD estimates stem from weighted regression (by population) over the entire pooled time frame, and a bandwidth of half a year around the cutoff. The effects in rural and urban areas are not significantly different from each other. The level of analysis is on Labor Market Regions. Figure \ref{fig_mlch: AMR_regions_Germany} shows a map of Germany with the regions marked as rural/urban. \newline Significance levels: * p < 0.10, ** p < 0.05, *** p < 0.01. \newline
	\end{minipage}
\end{table} 

