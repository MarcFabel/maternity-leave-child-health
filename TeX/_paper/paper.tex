%--------------------------------------------------------------------
%	DOCUMENT CLASS
%--------------------------------------------------------------------
\documentclass{scrartcl} % type of document (paper, presentation, book,...); scrartcl class with sans serif titles, European layout 
\usepackage{fullpage} % leaves less space at margins of page
\usepackage[onehalfspacing]{setspace} % determine line pitch to 1.5

%--------------------------------------------------------------------
%	INPUT
%--------------------------------------------------------------------
\usepackage[T1]{fontenc} % Use 8-bit encoding that has 256 glyphs
\usepackage[utf8]{inputenc} % Required for including letters with accents, Umlaute,...
\usepackage{float} % better control over placement of tables and figures in the text
\usepackage{graphicx} % input of graphics
\usepackage{xcolor} % advanced color package
\usepackage{url, hyperref} % include (clickable) URLs

%--------------------------------------------------------------------
%	TABLES, FIGURES, LISTS
%--------------------------------------------------------------------
\usepackage{booktabs} % better tables
\usepackage{threeparttable} % add notes below tables

\renewcommand\TPTrlap{}		% add margins on the side of the notes
\renewcommand\TPTnoteSettings{%
\setlength\leftmargin{5 pt}%
\setlength\rightmargin{5 pt}%
}
\usepackage[center, format=plain, font=normalsize, nooneline, labelfont={bf}]{caption} % change format of captions of tables and graphs 

%--------------------------------------------------------------------
%	MATH
%--------------------------------------------------------------------
\usepackage{amsmath,amssymb,amsfonts} % more math symbols and commands

%--------------------------------------------------------------------
%	LANGUAGE SPECIFICS
%--------------------------------------------------------------------
\usepackage[american]{babel} % man­ages cul­tur­ally-de­ter­mined ty­po­graph­i­cal (and other) rules, and hy­phen­ation pat­terns
\usepackage{csquotes} % language specific quotations



\usepackage[left=3cm,right=3cm,top=2cm,bottom=2cm]{geometry}
%--------------------------------------------------------------------
%	AUTHOR & TITLE
%--------------------------------------------------------------------
\author{Natalia Danzer \& Marc Fabel}
\title{Maternity Leave Coverage and the Long-Term Health Outcomes of Children in Germany}

%--------------------------------------------------------------------
%	BEGIN DOCUMENT
%--------------------------------------------------------------------
\begin{document}
\maketitle
\tableofcontents
\newpage



\section{Useful links for the paper}

Which item shall I inlcude in the metabolic syndrome: 
\begin{enumerate}
\item obesity
\item hypertension
\end{enumerate}

\begin{itemize}
\item \textbf{breastfeeding \& metabolism:}

\begin{itemize}
\item[-]\href{http://www.who.int/elena/titles/bbc/breastfeeding_childhood_obesity/en/}{WHO link})\\ 
\textit{"The positive impact of breastfeeding on lowering the risk of death from infectious diseases in the first two years of life is now well-established (1). A mounting body of evidence suggests that breastfeeding may also play a role in programming noncommunicable disease risk later in life (2-13) including protection against overweight and obesity in childhood (2-6)."}

\item[-]\href{http://articles.latimes.com/2011/may/02/news/la-heb-infant-feeding-20110502}{LA Times article}\\
\textit{"The study showed that children who received breast milk for the first four months had a specific pattern of growth and metabolic profile that differed from the formula-fed babies. Even at 15 days of life, the breast-fed infants had blood insulin levels that were lower than the formula-fed infants.\newline
By 3 years of age, many of the metabolic and growth differences between the breast-fed and formula-fed infants had disappeared. However, blood pressure readings were higher in the infants who had been fed the high-protein formula compared with breast-fed infants. The blood pressure rates were still within the normal range."}
\end{itemize}
\end{itemize}

















\newpage
\section{Appendix}
\subsection{Outcomes Hospital registry data}


\begin{table}[h] % table environment for caption and label
\begin{threeparttable}
\centering % center the tabular
\caption{Overview of outcome variables} % caption
\label{tab:outcomes_coding_main_chapters} 
\begin{tabular}{lrrr} % alignment and number of columns of actual table
\toprule % top thicker horizontal line (" rule ")
        &\multicolumn{1}{c}{(1)}& &\multicolumn{1}{c}{(2)}\\
&\multicolumn{1}{c}{ICD-9} & & \multicolumn{1}{c}{ICD-10} \\ 
\midrule
%-------------------------------------------------------------------------
\textit{Hospital admission - main diagnosis chapters:}\\
 \hspace{4pt} Infectious and parasitic diseases                           	&	001-139		& &		A00-B99 \\
 \hspace{4pt} Neoplasms                                                   	&	140-239		& &		C00-D48 \\
 %\hspace{4pt} Diseases of the blood and blood-forming organs              	&	280-289		& &		D50-D90 \\
 \hspace{4pt} Endocrine, nutritional and metabolic diseases					&	240-278		& &		E00-E90 \\
 \hspace{4pt} Mental \& behavioral  disorders                             	&	290-319		& &		F00-F99 \\
 \hspace{4pt} Diseases of the nervous system                              	&	320-359		& &		G00-G99 \\
 \hspace{4pt} Diseases of the sense organs                                	&	360-389		& &		H00-H95 \\
 \hspace{4pt} Diseases of the circulatory system                          	&	390-459		& &		I00-I99 \\
 \hspace{4pt} Diseases of the respiratory system                          	&	460-519		& &		J00-J99 \\
 \hspace{4pt} Diseases of the digestive system                            	&	520-579		& &		K00-K93 \\
 \hspace{4pt} Diseases of the skin and subcutaneous tissue                	&	680-709		& &		L00-L99 \\
 \hspace{4pt} Diseases of the musculoskeletal system						&	710-739		& &		M00-M99 \\
 \hspace{4pt} Diseases of the genitourinary system                        	&	580-629		& &		N00-N99 \\
 %\hspace{4pt} Pregnancy, childbirth, and the puerperium  					&	630-676		& &		O00-O99 \\
 %\hspace{4pt} Certain conditions originating in the perinatal period      	&	760-779		& &		P00-P96 \\
 %\hspace{4pt} Congenital anomalies                                        	&	740-759		& &		Q00-Q99 \\
 \hspace{4pt} Symptoms, signs, and ill-defined conditions                 	&	780-799		& &		R00-R99 \\
 \hspace{4pt} Injury, poisoning and certain other                          	&	800-999		& &		S00-T98 \\
 \hspace{8pt} consequences of external causes  \\
\\

\textit{Subcategories (mental \& behavioral disorders)}\\
 \hspace{12pt} Organic, including symptomatic, mental disorders				& 290,293,294,310		& & F00-F09 \\
 \hspace{12pt} MBD due to psychoactive substance use$^1$						& 291,292,303,304,305	& & F10-F19 \\
 \hspace{12pt} Schizophrenia, schizotypal and delusional disorders			& 295,297,298			& & F20-F29 \\
 \hspace{12pt} Mood [affective] disorders									& 296,311				& & F30-F39 \\
 \hspace{12pt} Neurotic, stress-related and somatoform disorders			& 300,306,308,309		& & F40-F48 \\
 \hspace{12pt} Behavioural syndromes associated with 						& 316					& & F50-F59 \\
 \hspace{16pt} physiological disturbances and physical factors \\
 \hspace{12pt} Disorders of adult personality and behavior					& 301,302				& & F60-F69 \\
 \hspace{12pt} Mental retardation 											& 317,318,319			& &	F70-F79 \\
 \hspace{12pt} Disorders of psychological development						& 299,315				& & F80-F89 \\
 \hspace{12pt} Behavioural and emotional disorders with 					& 312,313,314,307		& & F90-F98 \\
 \hspace{16pt} onset usually occurring in childhood and adolescence \\

%  \hspace{4pt} Metabolic Syndrome\\
%  \hspace{12pt} Type II Diabetes												& 250 				& & E11-E12 \\
%  \hspace{12pt} Overweight \& Obesity										& 278			    & & E65-E68 \\
%  \hspace{12pt} Hypertension													& 401,402,404,405	& & I10,I11,I13,I15	\\
%  \hspace{12pt} Ischemic heart disease										& 410-414	 		& & I20-I25	\\
% \vspace{-10pt}\\
%  \hspace{4pt} Index respiratory system										&					& &		    \\
%  \hspace{12pt} Pneumonia													& 480-486			& & J12-J18	\\
%  \hspace{12pt} Acute bronchitis												& 460-466			& & J20-J22	\\
%  \hspace{12pt} Chronic lower respiratory diseases							& 490-494,496		& & J40-47	\\
% \vspace{-10pt}\\
%  \hspace{4pt} Index drug abuse        										& 291,303-305,980   & &	F10-F19	\\	

%-------------------------------------------------------------------------
\bottomrule % bottom thicker horizontal line (" rule ")
\end{tabular}
\begin{tablenotes}
      \scriptsize{ \item \textit{Notes:} Classification of diseases according to the "International Statistical Classification of Diseases and Related Health Problems (ICD)", a medical classification list provided by the World Health Organisation. The index for drug abuse indicates mental and behavioral disorders due to psychoactive substances. The list of psychoactive substances include alcohol, opioids, cannabinoids, sedatives or hypnotics, cocaine, other stimulants (including caffeine), hallucinogens, tobacco, volatile solvents,  multiple drug use and use of other psychoactive substances.\newline \textit{Source:} World Health Organisation (WHO), see for example: \href{http://www.who.int/classifications/icd/en/}{http://www.who.int/classifications/icd/en/} }
    \end{tablenotes}
  \end{threeparttable}
\end{table}


































\end{document}