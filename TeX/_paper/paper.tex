\documentclass[11pt,a4paper]{article}
\usepackage[utf8]{inputenc}
\usepackage[english]{babel}
\usepackage{amsmath}
\usepackage{amsfonts}
\usepackage{amssymb}
\usepackage{graphicx}
\usepackage{hyperref}
\usepackage[left=3cm,right=3cm,top=2cm,bottom=2cm]{geometry}
\author{Natalia Danzer \& Marc Fabel}
\title{Maternity Leave Coverage and the Long-Term Health Outcomes of Children in Germany}
\begin{document}
\maketitle
\tableofcontents
\newpage



\section{Useful links for the paper}

\begin{itemize}
\item \textbf{breastfeeding \& metabolism:}

\begin{itemize}
\item[-]\href{http://www.who.int/elena/titles/bbc/breastfeeding_childhood_obesity/en/}{WHO link})\\ 
\textit{"The positive impact of breastfeeding on lowering the risk of death from infectious diseases in the first two years of life is now well-established (1). A mounting body of evidence suggests that breastfeeding may also play a role in programming noncommunicable disease risk later in life (2-13) including protection against overweight and obesity in childhood (2-6)."}

\item[-]\href{http://articles.latimes.com/2011/may/02/news/la-heb-infant-feeding-20110502}{LA Times article}\\
\textit{"The study showed that children who received breast milk for the first four months had a specific pattern of growth and metabolic profile that differed from the formula-fed babies. Even at 15 days of life, the breast-fed infants had blood insulin levels that were lower than the formula-fed infants.\newline
By 3 years of age, many of the metabolic and growth differences between the breast-fed and formula-fed infants had disappeared. However, blood pressure readings were higher in the infants who had been fed the high-protein formula compared with breast-fed infants. The blood pressure rates were still within the normal range."}
\end{itemize}
\end{itemize}




























\end{document}