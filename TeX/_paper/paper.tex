%--------------------------------------------------------------------
%	DOCUMENT CLASS
%--------------------------------------------------------------------
\documentclass[11pt, a4paper]{article} % type of document (paper, presentation, book,...); scrartcl class with sans serif titles, European layout 
\usepackage{fullpage} % leaves less space at margins of page
\usepackage[onehalfspacing]{setspace} % determine line pitch to 1.5

%--------------------------------------------------------------------
%	INPUT
%--------------------------------------------------------------------
\usepackage[T1]{fontenc} 	% Use 8-bit encoding that has 256 glyphs
\usepackage[utf8]{inputenc} % Required for including letters with accents, Umlaute,...
\usepackage{float} 			% better control over placement of tables and figures in the text
\usepackage{graphicx} 		% input of graphics
\usepackage{xcolor} 		% advanced color package
\usepackage{url, hyperref} 	% include (clickable) URLs
\usepackage{pdfpages}		% insert pages of external pdf documents
\setlength{\parskip}{0em}	% vertical spacing for paragraphs
\setlength{\parindent}{0em}	% horizonzal spacing for paragraphs

%--------------------------------------------------------------------
%	TABLES, FIGURES, LISTS
%--------------------------------------------------------------------
\usepackage{booktabs} 		% better tables
\usepackage{longtable}		% tables that may be continued on the next page
\usepackage{threeparttable} % add notes below tables
\renewcommand\TPTrlap{}		% add margins on the side of the notes
	\renewcommand\TPTnoteSettings{%
	\setlength\leftmargin{5 pt}%
	\setlength\rightmargin{5 pt}%
}
\usepackage[
center, format=plain,
font=normalsize,
nooneline,
labelfont={bf}
]{caption} 				% change format of captions of tables and graphs 
%USED IN MPHIL: \usepackage[labelfont=bf,labelsep = period, singlelinecheck=off,justification=raggedright]{caption}, other specifications which are nice: labelformat = parens -> number in paranthesis 


%\usepackage{threeparttablex} % for "ThreePartTable" environment, helps to combine threepart and longtable

% Allow line breaks with \\ in column headings of tables
\newcommand{\clb}[3][c]{%
	\begin{tabular}[#1]{@{}#2@{}}#3\end{tabular}}

% allow line breaks with \\ in row titles
\usepackage{multirow}

\newcommand{\rlb}[3][c]{%
\multirow{2}{*}{\begin{tabular}[#1]{@{}#2@{}}#3\end{tabular}}}% optional argument: b = bottom or t= top alignment


\usepackage[singlelinecheck=on]{subcaption}%both together help to have subfigures
\usepackage{wrapfig}				% wrap text around figure


\usepackage{rotating}				% rotating figures & tables
\usepackage{enumerate}				% change appearance of the enumerator
\usepackage{paralist, enumitem}		% better enumerations
\setlist{noitemsep}					% no additional vertical spacing for enurations
%--------------------------------------------------------------------
%	MATH
%--------------------------------------------------------------------
\usepackage{amsmath,amssymb,amsfonts} % more math symbols and commands
\let\vec\mathbf				 % make vector bold, with no arrow and not in italic

%--------------------------------------------------------------------
%	LANGUAGE SPECIFICS
%--------------------------------------------------------------------
\usepackage[american]{babel} % man­ages cul­tur­ally-de­ter­mined ty­po­graph­i­cal (and other) rules, and hy­phen­ation pat­terns
\usepackage{csquotes} % language specific quotations

%--------------------------------------------------------------------
%	BIBLIOGRAPHY & CITATIONS
%--------------------------------------------------------------------
\usepackage{csquotes} % language specific quotations
\usepackage{etex}		% some more Tex functionality
\usepackage[nottoc]{tocbibind} %add bibliography to TOC
\usepackage[authoryear, round, comma]{natbib} %biblatex

%--------------------------------------------------------------------
%	PATHS
%--------------------------------------------------------------------
\makeatletter
\def\input@path{{../../analysis/tables/}}	%PATH TO TABLES
%or: \def\input@path{{/path/to/folder/}{/path/to/other/folder/}}
\makeatother
\graphicspath{{../../analysis/graphs/}}		% PATH TO GRAPHS

%--------------------------------------------------------------------
%	LAYOUT
%--------------------------------------------------------------------
\usepackage[left=3cm,right=3cm,top=2cm,bottom=3cm]{geometry}
\usepackage{pdflscape} % lscape.sty Produce landscape pages in a (mainly) portrait document.

\definecolor{darkblue}{rgb}{0.0,0.0,0.6}

% CAPTIAL LETTERS FOR SECTION CAPTIONS
%\usepackage{sectsty}
%\sectionfont{\normalfont\scshape\centering}
%\renewcommand{\thesection}{\Roman{section}.}
%\renewcommand{\thesubsection}{\thesection\Alph{subsection}.}
%\subsectionfont{\itshape}
%\subsubsectionfont{\scshape}
%\newcommand\relphantom[1]{\mathrel{\phantom{#1}}}
%\setlength\topmargin{0.1in} \setlength\headheight{0.1in}
%\setlength\headsep{0in} \setlength\textheight{9.2in}
%\setlength\textwidth{6.3in} \setlength\oddsidemargin{0.1in}
%\setlength\evensidemargin{0.1in}

\hypersetup{
  colorlinks  = true,
  citecolor   = darkblue,
 	linkcolor   = darkblue,
  urlcolor    = darkblue 
} % macht die URLS blau   
     
\usepackage{lettrine}	% First letter capitalized

% have date in month year format (i.e. omit the day in dates)
\usepackage{datetime}
\newdateformat{monthyeardate}{%
  \monthname[\THEMONTH], \THEYEAR}
%--------------------------------------------------------------------
%	AUTHOR & TITLE
%--------------------------------------------------------------------
\title{Maternity Leave and Long-Term Health Outcomes of Children\footnote{We are grateful to Maarten Lindeboom, Erik Plug, Helmut Rainer and participants at several conferences for helpful comments and suggestions. All errors are our own.}}
\author{{\color{red}Preliminary and incomplete draft\newline Please do not cite or circulate without the authors' permission.}\\ \\
Natalia Danzer \& Marc Fabel \thanks{Marc Fabel (corresponding author): Munich Graduate School of Economics (MGSE) and ifo Institute for Economic Research (email: \href{mailto:fabel@ifo.de}{fabel@ifo.de}).\newline Natalia Danzer: ifo Institute for Economic Research, University of Munich and IZA (email: \href{mailto:danzer@ifo.de }{danzer@ifo.de}).}}

\date{\monthyeardate\today}










%--------------------------------------------------------------------
%	BEGIN DOCUMENT
%--------------------------------------------------------------------
\begin{document}

\tableofcontents
\newpage
\maketitle
\renewcommand{\abstractname}{\vspace{-\baselineskip}} % GET RID OF ABSTRACT TITLE

  \begin{abstract}\noindent 
   \footnotesize{\begin{center}\textbf{Abstract}\end{center} text here
   \\\newline \textbf{Keywords:} Early childhood development, health, paid maternity leave, life-cycle approach \newline \textbf{JEL codes:} I10, J13, J18}
    \end{abstract}

\newpage


%--------------------------------------------------------------------
% INTRODUCTION
%--------------------------------------------------------------------
\section{Introduction}\label{sec:introduction}
Intro –
Health paper or PL paper?
Why relevant?
What do we (not) know?
What do we do?
What do we find?
 
To which literature do we contribute to?
\begin{itemize}
  \item Parental leave literature
  \item Literature on the role of early childhood interventions on long-run child development
  \begin{itemize}
    \item The role of type of nurture at the beginning of life on later health outcomes
    \item Fetal origin hypotheses extended
  \end{itemize}
  \item Spill-over of labor market policy on health outcomes
\end{itemize}

% Nice way of framing taken from ACD (2017) NBER 
%  -AC(2011b) initial effects of something fade out in the beginning and reappear in adulthood 
%  - "Broadly considered, there are two types of resources that can be expected to benefit children: Material resources (Y) and time inputs (It), which might be an argument in the production of child investment" 
%  - Maternity Leave: "If childhood investments are an increasing function of parental time, then maternity leave policies may increase investments at key developmental stages. Such policies appear to be predicated on the belief that the elasticity of child investments in (1) with respect to parental time is large in very early childhood. The key policy question is when specifically maternal (or paternal) time is most important?" 
% heterogenity in the effect (according to parental SES) not all parents can make use of their resouirfces in the same efficient way; implies different production functions 


%--------------------------------------------------------------------
% BACKGROUND
%--------------------------------------------------------------------
\section{Background on the 1979 maternity leave reform}\label{sec:background}
\subsection{reform}
  \begin{itemize}
    \item description 
    \item eligibility
    \item take-up
    \item first stage
    \begin{itemize}
      \item D+S 2012
      \item L+S
      \item own evidence?
    \end{itemize}
  \end{itemize}
\subsection{Counterfactual mode of care}
\subsection{breastfeeding 1979} 
\subsection{any prior on health effects given literature}


%--------------------------------------------------------------------
% IDENTIFICATION
%--------------------------------------------------------------------
\newpage
\section{Empirical strategy}\label{sec:empirical_strategy}
\subsection{Design}

\begin{itemize}
	\item combination RD and DD
	\item local estimation 
	\item control cohort: one year prior to reform year
	\item intention-to-treat effect
\end{itemize}
We use a difference-in-difference specification to estimate the effect of the length of maternity leave on children's health outcomes.

In order to be \\
%same strategy:  \textcite{Dustmann2012}, \textcite{Lalive2014}, \textcite{RafaelLaliveandJosefZweimuller2009}, \textcite{Ekberg2013}

\cite{Dustmann2012}
\cite{Lalive2014}
\cite{RafaelLaliveandJosefZweimuller2009}, and \cite{Ekberg2013}
In particular, our baseline approach to estimating the effect of the length of maternity leave on children's health outcomes corresponds to the following equation
\begin{align}
Y_{mrt} = \gamma_0 + \gamma_1 Treat_{mr} + \gamma_2 After_{mr} + \gamma_3 Treat_{mr} * After_{mr} + \psi_m + \phi_r + \rho_t + \varepsilon_{mrt}
\end{align}
where $Y_{mrt}$ is the number of diagnoses per thousand individuals of the cohort born in month $m$, who live in region $r$ at time period $t$. $Treat_{mr}$c is a dichotomous variable equal to one for groups that are born shortly before or after the legislation took place (i.e. the treatment cohort).\footnote{In the widest specification this involves children, who are born between November 1978 and October 1979, implying a bandwidth of half a year around the cutoff.} $After_{mr}$ is a dummy variable that equals to one if the group [XXX other word for group, repetitive] is born after the threshold month May (i.e. born in May-October in the widest specification, for both treatment and control cohort). $\psi_m$, $\phi_r$, $\rho_t$ are month-of-birth, region, and survey year fixed effects. The interaction term $Treat_{mr} * After_{mr}$ equals one for the group of interest (the children born between May and October 1979,i.e. the post-reform children in the treatment group).


%clustering
Sandwiched standard error estimates, allowing errors to be correlated over time within a month-of-birth cohort, and across 
Diagnosis rates are serially correlated, cluster on month-of-birth and state level.


$\varepsilon_{mrt}$


\subsection{Threat}
\subsection{validity}

%--------------------------------------------------------------------
% DATA & VARIABLES
%--------------------------------------------------------------------
\section{Hospital registry data}\label{sec:data}

%--------------------------------------------------------------------
% RESULTS
%--------------------------------------------------------------------
\section{Results}\label{sec:results}
\subsection{pooled}
\subsection{life-course approach}
\subsection{subcategories}
\subsection{further results}
show results for other outcomes, eventually include results from German Micro Census
\subsection{Robustness}

%--------------------------------------------------------------------
% CONCLUSION
%-------------------------------------------------------------------
\section{Concluding remarks}\label{sec:conclusion}








%--------------------------------------------------------------------
% BIBLIOGRAPHY
%--------------------------------------------------------------------
\newpage


\bibliographystyle{chicago}
\bibliography{mlch_bibliography}

%\printbibliography


%--------------------------------------------------------------------
% FIGURES AND TABLES
%--------------------------------------------------------------------
\newpage
\section{Figures and tables}


%--------------------------------------------------------------------
% APPENDIX
%--------------------------------------------------------------------
\newpage
\section{Appendix}
\subsection{Outcomes Hospital registry data}


\begin{table}[h] % table environment for caption and label
\begin{threeparttable}
\centering % center the tabular
\caption{Overview of outcome variables} % caption
\label{tab:outcomes_coding_main_chapters} 
\begin{tabular}{lrrr} % alignment and number of columns of actual table
\toprule % top thicker horizontal line (" rule ")
        &\multicolumn{1}{c}{(1)}& &\multicolumn{1}{c}{(2)}\\
&\multicolumn{1}{c}{ICD-9} & & \multicolumn{1}{c}{ICD-10} \\ 
\midrule
%-------------------------------------------------------------------------
\textit{Main diagnosis chapters}\\
 \hspace{4pt} Infectious and parasitic diseases                           	&	001-139		& &		A00-B99 \\
 \hspace{4pt} Neoplasms                                                   	&	140-239		& &		C00-D48 \\
%\hspace{4pt} Diseases of the blood and blood-forming organs              	&	280-289		& &		D50-D90 \\
 \hspace{4pt} Endocrine, nutritional and metabolic diseases					&	240-278		& &		E00-E90 \\
 \hspace{4pt} Mental \& behavioral  disorders                             	&	290-319		& &		F00-F99 \\
 \hspace{4pt} Diseases of the nervous system                              	&	320-359		& &		G00-G99 \\
 \hspace{4pt} Diseases of the sense organs                                	&	360-389		& &		H00-H95 \\
 \hspace{4pt} Diseases of the circulatory system                          	&	390-459		& &		I00-I99 \\
 \hspace{4pt} Diseases of the respiratory system                          	&	460-519		& &		J00-J99 \\
 \hspace{4pt} Diseases of the digestive system                            	&	520-579		& &		K00-K93 \\
 \hspace{4pt} Diseases of the skin and subcutaneous tissue                	&	680-709		& &		L00-L99 \\
 \hspace{4pt} Diseases of the musculoskeletal system and connective tissue	&	710-739		& &		M00-M99 \\
 \hspace{4pt} Diseases of the genitourinary system                        	&	580-629		& &		N00-N99 \\
 \hspace{4pt} Complications of pregnancy, childbirth, and the puerperium  	&	630-676		& &		O00-O99 \\
%\hspace{4pt} Certain conditions originating in the perinatal period      	&	760-779		& &		P00-P96 \\
%\hspace{4pt} Congenital anomalies                                        	&	740-759		& &		Q00-Q99 \\
 \hspace{4pt} Symptoms, signs, and ill-defined conditions                 	&	780-799		& &		R00-R99 \\
 \hspace{4pt} Injury and poisoning                                        	&	800-999		& &		S00-T98 \\
 \\
 \textit{}

\
%-------------------------------------------------------------------------
\bottomrule % bottom thicker horizontal line (" rule ")
\end{tabular}
\begin{tablenotes}
      \scriptsize{ \item \textit{Notes:} Classification of diseases according to the "International Statistical Classification of Diseases and Related Health Problems (ICD)", a medical classification list provided by the World Health Organization. The index for drug abuse indicates mental and behavioral disorders due to psychoactive substances. The list of psychoactive substances include alcohol, opioids, cannabinoids, sedatives or hypnotics, cocaine, other stimulants (including caffeine), hallucinogens, tobacco, volatile solvents,  multiple drug use and use of other psychoactive substances.\newline \textit{Source:} World Health Organization (WHO), see for example: \href{http://www.who.int/classifications/icd/en/}{http://www.who.int/classifications/icd/en/} }
    \end{tablenotes}
  \end{threeparttable}
\end{table}

%--------------------------------------------------------------------
% Useful links for the paper
%-------------------------------------------------------------------
\newpage
\subsection{Useful links for the paper}

Which item shall I include in the metabolic syndrome: 
\begin{enumerate}
\item obesity
\item hypertension
\end{enumerate}

\begin{itemize}
\item \textbf{breastfeeding \& metabolism:}

\begin{itemize}
\item[-]\href{http://www.who.int/elena/titles/bbc/breastfeeding_childhood_obesity/en/}{WHO link})\\ 
\textit{"The positive impact of breastfeeding on lowering the risk of death from infectious diseases in the first two years of life is now well-established (1). A mounting body of evidence suggests that breastfeeding may also play a role in programming noncommunicable disease risk later in life (2-13) including protection against overweight and obesity in childhood (2-6)."}

\item[-]\href{http://articles.latimes.com/2011/may/02/news/la-heb-infant-feeding-20110502}{LA Times article}\\
\textit{"The study showed that children who received breast milk for the first four months had a specific pattern of growth and metabolic profile that differed from the formula-fed babies. Even at 15 days of life, the breast-fed infants had blood insulin levels that were lower than the formula-fed infants.\newline
By 3 years of age, many of the metabolic and growth differences between the breast-fed and formula-fed infants had disappeared. However, blood pressure readings were higher in the infants who had been fed the high-protein formula compared with breast-fed infants. The blood pressure rates were still within the normal range."}
\end{itemize}
\end{itemize}















\end{document}