%--------------------------------------------------------------------
%	DOCUMENT CLASS
%--------------------------------------------------------------------
\documentclass[11pt, a4paper,draft]{article} % type of document (paper, presentation, book,...); scrartcl class with sans serif titles, European layout 
\usepackage{fullpage} % leaves less space at margins of page
\usepackage[onehalfspacing]{setspace} % determine line pitch to 1.5

%--------------------------------------------------------------------
%	INPUT
%--------------------------------------------------------------------
\usepackage[T1]{fontenc} 	% Use 8-bit encoding that has 256 glyphs
\usepackage[utf8]{inputenc} % Required for including letters with accents, Umlaute,...
\usepackage{float} 			% better control over placement of tables and figures in the text
\usepackage{graphicx} 		% input of graphics
\usepackage{xcolor} 		% advanced color package
\usepackage{url, hyperref} 	% include (clickable) URLs
\usepackage{pdfpages}		% insert pages of external pdf documents
\setlength{\parskip}{0em}	% vertical spacing for paragraphs
\setlength{\parindent}{0em}	% horizonzal spacing for paragraphs
%
%--------------------------------------------------------------------
%	TABLES, FIGURES, LISTS
%--------------------------------------------------------------------
\usepackage{booktabs} 		% better tables
\usepackage{longtable}		% tables that may be continued on the next page
\usepackage{threeparttable} % add notes below tables
\renewcommand\TPTrlap{}		% add margins on the side of the notes
	\renewcommand\TPTnoteSettings{%
	\setlength\leftmargin{5 pt}%
	\setlength\rightmargin{5 pt}%
}
\usepackage[
center, format=plain,
font=normalsize,
nooneline,
labelfont={bf}
]{caption} 				% change format of captions of tables and graphs 
%USED IN MPHIL: \usepackage[labelfont=bf,labelsep = period, singlelinecheck=off,justification=raggedright]{caption}, other specifications which are nice: labelformat = parens -> number in paranthesis 


%\usepackage{threeparttablex} % for "ThreePartTable" environment, helps to combine threepart and longtable

% Allow line breaks with \\ in column headings of tables
\newcommand{\clb}[3][c]{%
	\begin{tabular}[#1]{@{}#2@{}}#3\end{tabular}}

% allow line breaks with \\ in row titles
\usepackage{multirow}

\newcommand{\rlb}[3][c]{%
\multirow{2}{*}{\begin{tabular}[#1]{@{}#2@{}}#3\end{tabular}}}% optional argument: b = bottom or t= top alignment


\usepackage[singlelinecheck=on]{subcaption}%both together help to have subfigures
\usepackage{wrapfig}				% wrap text around figure


\usepackage{rotating}				% rotating figures & tables
\usepackage{enumerate}				% change appearance of the enumerator
\usepackage{paralist, enumitem}		% better enumerations
\setlist{noitemsep}					% no additional vertical spacing for enurations
%--------------------------------------------------------------------
%	MATH
%--------------------------------------------------------------------
\usepackage{amsmath,amssymb,amsfonts} % more math symbols and commands
\let\vec\mathbf				 % make vector bold, with no arrow and not in italic

%--------------------------------------------------------------------
%	LANGUAGE SPECIFICS
%--------------------------------------------------------------------
\usepackage[american]{babel} % man­ages cul­tur­ally-de­ter­mined ty­po­graph­i­cal (and other) rules, and hy­phen­ation pat­terns
\usepackage{csquotes} % language specific quotations

%--------------------------------------------------------------------
%	BIBLIOGRAPHY & CITATIONS
%--------------------------------------------------------------------
\usepackage{csquotes} % language specific quotations
\usepackage{etex}		% some more Tex functionality
\usepackage[nottoc]{tocbibind} %add bibliography to TOC
\usepackage[authoryear, round, comma]{natbib} %biblatex

%--------------------------------------------------------------------
%	PATHS
%--------------------------------------------------------------------
\makeatletter
\def\input@path{{../../analysis/tables/}}	%PATH TO TABLES
%or: \def\input@path{{/path/to/folder/}{/path/to/other/folder/}}
\makeatother
\graphicspath{{../../analysis/graphs/}}		% PATH TO GRAPHS

%--------------------------------------------------------------------
%	LAYOUT
%--------------------------------------------------------------------
\usepackage[left=3cm,right=3cm,top=2cm,bottom=3cm]{geometry}
\usepackage{pdflscape} % lscape.sty Produce landscape pages in a (mainly) portrait document.

\definecolor{darkblue}{rgb}{0.0,0.0,0.6}

% CAPTIAL LETTERS FOR SECTION CAPTIONS
%\usepackage{sectsty}
%\sectionfont{\normalfont\scshape\centering}
%\renewcommand{\thesection}{\Roman{section}.}
%\renewcommand{\thesubsection}{\thesection\Alph{subsection}.}
%\subsectionfont{\itshape}
%\subsubsectionfont{\scshape}
%\newcommand\relphantom[1]{\mathrel{\phantom{#1}}}
%\setlength\topmargin{0.1in} \setlength\headheight{0.1in}
%\setlength\headsep{0in} \setlength\textheight{9.2in}
%\setlength\textwidth{6.3in} \setlength\oddsidemargin{0.1in}
%\setlength\evensidemargin{0.1in}

\hypersetup{
  colorlinks  = true,
  citecolor   = darkblue,
 	linkcolor   = darkblue,
  urlcolor    = darkblue 
} % macht die URLS blau   
     
\usepackage{lettrine}	% First letter capitalized

% have date in month year format (i.e. omit the day in dates)
\usepackage{datetime}
\newdateformat{monthyeardate}{%
  \monthname[\THEMONTH], \THEYEAR}
%--------------------------------------------------------------------
%	AUTHOR & TITLE
%--------------------------------------------------------------------
\title{Maternity Leave and Long-Term Health Outcomes of Children\footnote{We are grateful to Maarten Lindeboom, Erik Plug, Helmut Rainer and participants at several conferences for helpful comments and suggestions. All errors are our own.
[Q to ND:  other people which we would like to thank: Daniel Kühnle, Anna Raute, Mathias Hübener,]}}
\author{Natalia Danzer \& Marc Fabel \thanks{Marc Fabel (corresponding author): Munich Graduate School of Economics (MGSE) and ifo Institute for Economic Research (email: \href{mailto:fabel@ifo.de}{fabel@ifo.de}).\newline Natalia Danzer: Free University of Berlin, ifo Institute for Economic Research, CESifo and IZA (email: \href{mailto:natalia.danzer@fu-berlin.de}{natalia.danzer@fu-berlin.de}).}}

\date{\monthyeardate\today}










%--------------------------------------------------------------------
%	BEGIN DOCUMENT
%--------------------------------------------------------------------
\begin{document}

\tableofcontents
\newpage
\maketitle

\textbf{\color{red} Preliminary and incomplete draft\newline Please do not cite or circulate without the authors' permission}
\renewcommand{\abstractname}{\vspace{-\baselineskip}} % GET RID OF ABSTRACT TITLE

  \begin{abstract}\noindent 
   \footnotesize{\begin{center}\textbf{Abstract}\end{center} text here
   \\\newline \textbf{Keywords:} Early childhood development, health, paid maternity leave, life-cycle approach \newline \textbf{JEL codes:} I10, J13, J18}
    \end{abstract}

\newpage


%--------------------------------------------------------------------
% INTRODUCTION
%--------------------------------------------------------------------
\section{Introduction (evtl NATALIA)}\label{sec:introduction}
Intro –
Health paper or PL paper?
Why relevant?
What do we (not) know?
What do we do?
What do we find?
 
To which literature do we contribute to?
\begin{itemize}
  \item Parental leave literature
  \item Literature on the role of early childhood interventions on long-run child development
  \begin{itemize}
    \item The role of type of nurture at the beginning of life on later health outcomes
    \item Fetal origin hypotheses extended
  \end{itemize}
  \item Spill-over of labor market policy on health outcomes
\end{itemize}

What is long-term health??( evtl auch in die background section?) 
\begin{itemize}
	\item Hospital admission
	\item why relevant? costs...
	\item what are frequencies (compare to ND's picture for the IZA presentation)
\end{itemize}

% Nice way of framing taken from ACD (2017) NBER 
%  -AC(2011b) initial effects of something fade out in the beginning and reappear in adulthood 
%  - "Broadly considered, there are two types of resources that can be expected to benefit children: Material resources (Y) and time inputs (It), which might be an argument in the production of child investment" 
%  - Maternity Leave: "If childhood investments are an increasing function of parental time, then maternity leave policies may increase investments at key developmental stages. Such policies appear to be predicated on the belief that the elasticity of child investments in (1) with respect to parental time is large in very early childhood. The key policy question is when specifically maternal (or paternal) time is most important?" 
% heterogenity in the effect (according to parental SES) not all parents can make use of their resouirfces in the same efficient way; implies different production functions 
[Q ND: Motivation - shall we have a cross-country (maybe OECD data) scatter with linear fit, Y: health outcomes and X lenght of Maternity/parental leave]

%--------------------------------------------------------------------
% BACKGROUND
%--------------------------------------------------------------------
\newpage
\section{Background on the 1979 maternity leave reform}\label{sec:background}
\subsection{reform}
  \begin{itemize}
    \item description of the reform \footnote{Es gab noch andere Reformen, dies ist nicht der aktuelle Stand, weitere Reformen 1986,..., 2007 current system was installed}\newline reform has already been investigated in two seminal papars by \cite{Dustmann2012} and \cite{schonberg2014expansions}....
    \item eligibility
    \item take-up

    
\newpage
\textbf{First stage}\newline 
%[XXX Ich habe jetzt pro Aspekt die beiden Zusammengeworfen und nicht jedes paper für sich behandelt, notwendig da SL zB mehr in die Tiefe geht bei maternal outcomes]
The effect of the 1979 maternity leave reform has been investigated by the two seminal papers of \cite{Dustmann2012} and \cite{schonberg2014expansions}.\footnote{Both papers analyze the impact of the expansion in maternity leave on maternal labor market outcomes. While the former study augments this aspect by evaluating potential changes in child outcomes due to the reform, the latter study focuses on maternal labor market outcomes but elicits more in depth results.} Both studies show that the reform had a large impact on mothers' short-run labor market outcomes. \newline First, many mothers strongly adjust their labor supply downwards during the four months of extra leave, and return to the labor market as soon as the leave period terminates. Yet, it seems that there is only a small effect on long-run maternal labor supply (i.e. for the time period beyond six months after childbirth). For instance, while the reform decreased the share of mothers who had returned to the labor market by the third month after childbirth by 30.5 percentage points, the reduction in the share of returned mothers by the 52nd/76th month after childbirth is only around one percentage point.\footnote{When looking at long-run maternal labor force participation rates (i.e. at the child's sixth birthday), one can see that the reform leads to a modest increase in the probability a mother is working. The implication is that the mothers who abstain from the labor market in the long-run, would have only returned to work temporarily in absence of the reform.}
In total, the change of postnatal maternity leave from two to six months caused mothers to postpone their return to work by, on average, 0.835 months.\footnote{This number corresponds to the number of months away from work in the first 40 months since childbirth.} Furthermore, approximately two-third of the decline in short-run female labor force participation is the result from a contraction in full-time work. Most of the mothers who postponed their labor market return, would have returned to their previous employers in absence of the reform. \newline
Second, the expansion in maternity leave lead to changes in mothers' income. When focusing on mere labor market income there is no effect of the reform that is significantly different from zero. However, when maternity benefit payments are included in the income measure for the eligible women, there is an overall increase in cumulative total income by, on average, 1,700 Deutschmarks (DM) as a result of the reform.\footnote{Maternal cumulative total  income is defined as the accumulated total income up to the point when the child is 40 years old. It consists of monthly earnings when the mother is working, equals to the benefits when she is on leave or is zero otherwise.\newline \cite{Dustmann2012} deflate monthly income such that everything is in 1992 prices. The benefit of 750 DM (1,190 DM in 1992 prices) resembles approximately one-third (55\%) of the mother's pre-birth (post-birth) earnings.} This is due to the fact that the effect of unexpected extra income for mothers who would have stayed at home even without the reform prevails over the decrease in available income stemming from the reduction in maternal employment. In contrast to the impact on labor supply, there is a strong effect heterogeneity in cumulative income across wages (an increase of 2,850 DM for mothers in the lowest tercile of the wage distribution, while the additional income amounts to 1,050 DM for women in the highest tercile). \newline
% vielleicht nochmal wrap up wie in der Discussion von SL, short-run effet on maternal employment yes - LR nichts verbessert, no depreciation of HC... 
Although there are strong effects on maternal labor market outcomes, in particular in the short-run, \cite{Dustmann2012} do not find evidence that the reform had an impact on children's educational attainment and labor market outcomes. They do not find any changes in the levels or years of education, log wages, or the share of individuals in full-time employment in response to the reform. 

% so strong effect on mother's return to work behavior given but so far in the domains that were looked at, there are no effects on children's long-run outcomes.




\textbf{The effects of other reforms on health or where you find something in other LR domains}
Even though there was no effect of this particular reform on children's long-term outcomes, there are examples which do find differentials in child outcomes.



    
[Q ND: SHALL WE INCLUDE DS(2012) FIGURES, SUCH AS FIG 2A \& 4A-F, AT LEAST IN THE APPENDIX OR DO WE EXPECT THE READER TO LOOK THIS UP HER-/HIMSELF?]    
  \end{itemize}





\subsection{Counterfactual mode of care}
describe female labor force participation and childcare
\subsection{potential channels}
\begin{itemize}
	\item breastfeeding, what was situation in 1979
	\item attachment theory/Fetal origin hypothesis/neurobiological literature (talk to Prof. Sulz)
	\item income (depreciation of human capital, change selection of mothers into work, change in labor market attachment see SL2014)
	\item other outcomes (fertility)
\end{itemize} 
\subsection{any prior on health effects given literature}


%--------------------------------------------------------------------
% IDENTIFICATION
%--------------------------------------------------------------------
\newpage
\section{Empirical strategy}\label{sec:empirical_strategy}
\subsection{Design}
In order to estimate the causal effect of maternity leave at the intensive margin, we exploit the 1979 reform's eligibility rule, which is contingent on childrens' birth date (see section \ref{sec:background}). Children born on/after a specified birth cutoff date (May 01, 1979) fall under the new regime with six months of maternity leave after childbirth, whereas children born before the threshold are exposed to two months. Assignment to treatment is a deterministic function of the birth date of the child and thus "sharp" as in the terminology of \cite{hahn2001identification}. A regression discontinuity design (RDD) might constitute a first potential identification strategy, in which one compares health outcomes of children which are quite similar with the only notable exception that their mothers were entitled to different lengths of maternity leave.\newline 
We augment this idea of a local identification by combining the RDD with a difference-in-differences (DD) approach. A large body of literature suggests that there is a strong relationship between season of birth, health and other socioeconomic outcomes. The seasonality may come about due to reasons that are associated with either pre- or postnatal factors. First, the seasonality might arise due to selective conception, i.e. the socioeconomic composition of mothers varies over time \citep{buckles2013season}. Second, \cite{currie2013within} argue that the issue of selection is not as large as previously thought and put forward the idea that any season of birth effects might be due to seasonal patterns of in-utero disease prevalence and nutrition.\footnote{In particular, they find seasonal influenza as a potential mechanism between month of birth and later outcomes.} Last, the seasonality aspect may also be the result of social postnatal factors such as age at school-entry \citep{black2011too}. \newline If one would not take these season-of-birth effects into consideration, it could be the case that the estimated effect is partly driven by the difference of health outcomes from that seasonality component and not due to the expansion in maternity leave coverage. Yet, by matching up the difference in health outcomes of children born within a distance to the birth cutoff date in which the legislation change took place (henceforth referred to as the treatment cohort) with differences in outcomes of children born around the same threshold, but in a year in which no reform occurred (control group) we can eliminate the seasonality component while preserving the local identification aspect.\footnote{In our widest specification we use a bandwidth of half a year on either side, i.e. the treatment groups consists of people that were born between November 1978 and October 1979.}
The implicit identifying assumption is that seasonality is time-invariant, in other words, one has to assume that the season-of-birth effects are the same for treatment and control group.\newline

Our main specification to estimating the effect of the length of maternity leave on children's health outcomes corresponds to the following equation: \footnote{The estimation procedure can also be found in similar contexts in \cite{RafaelLaliveandJosefZweimuller2009}, \cite{Dustmann2012}, \cite{Ekberg2013}, \cite{schonberg2014expansions}, \cite{Lalive2014}, \cite{Huebener2017}, and \cite{danzer2017}.[XXX: neues paper (Chile), benutzt das auch RDDD?]}
\begin{align}
Y_{mrt} = \gamma_0 + \gamma_1 Treat_{mr} + \gamma_2 After_{mr} + \gamma_3 Treat_{mr} * After_{mr} + \psi_m + \phi_r + \rho_t + \varepsilon_{mrt}
\end{align}
where $Y_{mrt}$ is the number of diagnoses per thousand individuals of the cohort born in month $m$, who reside in region $r$ at time period $t$. $Treat_{mr}$ is a dichotomous variable equal to one for groups that are born shortly before or after the legislation took place (i.e. the treatment cohort).\footnote{In the widest specification this involves children, who are born between November 1978 and October 1979, implying a bandwidth of half a year around the cutoff.} $After_{mr}$ is a dummy variable that equals to one if the individuals are born after the threshold month May (i.e. born in May-October in the widest specification, for both treatment and control cohort). $\psi_m$, $\phi_r$, $\rho_t$ are month-of-birth, region, and survey year fixed effects respectively. At first, $Y_{mrt}$ correspond to outcomes observed over the period 1995-2014. In section \ref{sec:results-lifecourse}, we apply a life-course approach by running the regression for each year individually.\footnote{text} 


The parameter of interest is $\gamma_3$ which captures the effect of the policy change on health outcomes. As we don't have any information on the fact whether the individuals' mothers were on leave, the identified parameter is an intention-to-treat effect. \newline
 %The interaction term $Treat_{mr} * After_{mr}$ equals one for the group of interest (the children born between May and October 1979,i.e. the post-reform children in the treatment group).


%control group
Unlike \cite{Dustmann2012} we only use the cohort born in the year prior to the reform as control group.\footnote{\cite{Dustmann2012} use in total three birth cohorts as control group, two cohorts before and one cohort after the treatment cohort: group 1 born 11/1976-10/1976; group 2 born 11/1977-10/1978; and group 3 born 11/1979-10/1980 compromise the control group.} The more cohorts are used as potential control groups, the less likely it is that the identifying assumption is met. Additionally, taking a birth cohort in the year after the policy change as control group might invalidate the comparability between the treatment and control group as parents might have enough time to react to the reform and adjust fertility patterns. For these reasons, we have just the one cohort prior to the treatment cohort as control group in our main specifications and in the robustness section we show the results of estimations which are in line with the approach chosen by \cite{Dustmann2012}.\newline


%clustering
Standard errors are clustered on the interaction of birth month and state level in order to account for likely correlation of the error $\varepsilon_{mrt}$ over time for a given month of birth cohort and across cohorts for a given state.
%We use sandwiched standard error estimates
%, allowing errors to be correlated over time within a month-of-birth cohort, and across 
%Diagnosis rates are serially correlated, cluster on month-of-birth and state level.







\newpage
\subsection{Threat}
timing of birth potentially possible, yet hardly the case literature search carried out by \cite{Dustmann2012},
\subsection{validity}
\begin{itemize}
	\item MZ (problem of selected sample, but large (enough?) number of obs) -> balancing table of parental predetermined covariates
	\item Fertility distribution  -> histograms
	\item potentially for later: SOEP, Zensus2011 (is there a question about parental background?)
\end{itemize}

%--------------------------------------------------------------------
% DATA & VARIABLES
%--------------------------------------------------------------------
\section{Data}\label{sec:data}

We use for this research the hospital register, provided by the Research Data Centers of the Federal Statistical Office and the statistical offices of the Länder, which contains information on the universe of German in-patient cases ($\sim$ 18 million cases per year). 
 spanning the period from 1995 to 2014.

-Migration problem: where we observe individuals at time $t$ does not imply that they were also born there (general analysis on federal level) $\rightarrow$ only robustness and heterogeneity on smaller aggregation.

\begin{itemize}
	\item Hospital registry data
	
	- Region: We use labor market regions as defined  \footnote{The regions are defined by the \href{https://www.bbsr.bund.de/BBSR/DE/Raumbeobachtung/Raumabgrenzungen/AMR/amr_node.html}{Federal Institute for Research on Building, Urban Affairs and Spatial Development}.}
		- aggregation of districts (kreise)
		- not each distract has its own hospital 
		%how many regions 
	-only from regions of former FRG (discussion migration)
	% denominator per 1000 individuals
	% we know number of people who live in certain region from The Regional Database Germany, we use either weights coming from German Micro Census or from the original birth statistic to approximate the number of people per month.

	
	\item Regional Database Germany
	\item German Micro Census
	\item Region classification (Federal Institute for Research on Building-BBSR)
\end{itemize}
%--------------------------------------------------------------------
% RESULTS
%--------------------------------------------------------------------
\section{Results (evtl NATALIA)}\label{sec:results}

\begin{enumerate}
	\item Hospital admission: effects for different age brackets and various bandwidths, RD plot (also for different age brackets) + graphical representation of the life-course
	\item main diagnosis chapter: effect for different age brackets + appendix: matrix of LC graphs\newline $\rightarrow$ results from hospital admission is triggered by mental illnesses
	\item look at mental and behavioral disorders: subcategories (also in life-course representation) 
	\item (further results: e.g. MZ)
	\item robustness (see for instance slides of IZA WL conference)
\end{enumerate}



%OLD:
%\subsection{pooled}\label{sec:results-pooled}
%\subsection{life-course approach}\label{sec:results-lifecourse}
%correction for age difference -> create pseudo years
%\subsection{subcategories}\label{sec:results-subcategories}
%\subsection{further results}
%show results for other outcomes, eventually include results from German Micro Census
%\subsection{Robustness}

%--------------------------------------------------------------------
% CONCLUSION
%-------------------------------------------------------------------
\section{Concluding remarks (evtl NATALIA)}\label{sec:conclusion}








%--------------------------------------------------------------------
% BIBLIOGRAPHY
%--------------------------------------------------------------------
\newpage


\bibliographystyle{chicago}
\bibliography{mlch_bibliography}

%\printbibliography


%--------------------------------------------------------------------
% FIGURES AND TABLES
%--------------------------------------------------------------------
\newpage
\section{Figures and tables}
\begin{figure}[H]\centering
	\caption{The regions in Germany}\label{fig: AMR_regions_Germany}
	\includegraphics[width=0.8\linewidth]{paper/AMR_germany.png}
	\scriptsize
	\begin{minipage}{0.9 \linewidth}
		\emph{Notes:} This map shows the regions used in the analysis. The areas with the red background depict the area of the former Federal Republic of Germany ("West Germany"), while the white areas indicate the area of the former German Democratic Republic ("East Germany"). The districts of West Germany are used throughout the paper, the regions of East Germany only in a robustness check (triple-differences model). The black outlines indicate federal state boundaries and the red dots represent the corresponding state capitals.\newline \emph{Source:} Own representation with data from the Federal Institute for Research on Building, Urban Affairs and Spatial Development (BBSR).
	\end{minipage}
\end{figure}



\newpage
\begin{figure}[H]\centering
	\caption{Region-level population density}\label{fig: AMR_regions_population_density}
	\includegraphics[width=0.8 \linewidth]{paper/AMR_popdensity.png}
	\scriptsize
	\begin{minipage}{0.9\linewidth}
		\emph{Notes:} This map shows the regional variation of population density across German regions. \emph{Source:} Own representation with data from the Federal Institute for Research on Building, Urban Affairs and Spatial Development (BBSR) and the Regional Database Germany.
	\end{minipage}
\end{figure}






%--------------------------------------------------------------------
% APPENDIX
%--------------------------------------------------------------------
\newpage
\section{Appendix}
\subsection{Outcomes Hospital registry data}


\begin{table}[h] % table environment for caption and label
\begin{threeparttable}
\centering % center the tabular
\caption{Overview of outcome variables} % caption
\label{tab:outcomes_coding_main_chapters} 
\begin{tabular}{lrrr} % alignment and number of columns of actual table
\toprule % top thicker horizontal line (" rule ")
        &\multicolumn{1}{c}{(1)}& &\multicolumn{1}{c}{(2)}\\
&\multicolumn{1}{c}{ICD-9} & & \multicolumn{1}{c}{ICD-10} \\ 
\midrule
%-------------------------------------------------------------------------
\textit{Main diagnosis chapters}\\
 \hspace{4pt} Infectious and parasitic diseases                           	&	001-139		& &		A00-B99 \\
 \hspace{4pt} Neoplasms                                                   	&	140-239		& &		C00-D48 \\
%\hspace{4pt} Diseases of the blood and blood-forming organs              	&	280-289		& &		D50-D90 \\
 \hspace{4pt} Endocrine, nutritional and metabolic diseases					&	240-278		& &		E00-E90 \\
 \hspace{4pt} Mental \& behavioral  disorders                             	&	290-319		& &		F00-F99 \\
 \hspace{4pt} Diseases of the nervous system                              	&	320-359		& &		G00-G99 \\
 \hspace{4pt} Diseases of the sense organs                                	&	360-389		& &		H00-H95 \\
 \hspace{4pt} Diseases of the circulatory system                          	&	390-459		& &		I00-I99 \\
 \hspace{4pt} Diseases of the respiratory system                          	&	460-519		& &		J00-J99 \\
 \hspace{4pt} Diseases of the digestive system                            	&	520-579		& &		K00-K93 \\
 \hspace{4pt} Diseases of the skin and subcutaneous tissue                	&	680-709		& &		L00-L99 \\
 \hspace{4pt} Diseases of the musculoskeletal system and connective tissue	&	710-739		& &		M00-M99 \\
 \hspace{4pt} Diseases of the genitourinary system                        	&	580-629		& &		N00-N99 \\
 \hspace{4pt} Complications of pregnancy, childbirth, and the puerperium  	&	630-676		& &		O00-O99 \\
%\hspace{4pt} Certain conditions originating in the perinatal period      	&	760-779		& &		P00-P96 \\
%\hspace{4pt} Congenital anomalies                                        	&	740-759		& &		Q00-Q99 \\
 \hspace{4pt} Symptoms, signs, and ill-defined conditions                 	&	780-799		& &		R00-R99 \\
 \hspace{4pt} Injury and poisoning                                        	&	800-999		& &		S00-T98 \\
 \\
 \textit{}

\
%-------------------------------------------------------------------------
\bottomrule % bottom thicker horizontal line (" rule ")
\end{tabular}
\begin{tablenotes}
      \scriptsize{ \item \textit{Notes:} Classification of diseases according to the "International Statistical Classification of Diseases and Related Health Problems (ICD)", a medical classification list provided by the World Health Organization. The index for drug abuse indicates mental and behavioral disorders due to psychoactive substances. The list of psychoactive substances include alcohol, opioids, cannabinoids, sedatives or hypnotics, cocaine, other stimulants (including caffeine), hallucinogens, tobacco, volatile solvents,  multiple drug use and use of other psychoactive substances.\newline \textit{Source:} World Health Organization (WHO), see for example: \href{http://www.who.int/classifications/icd/en/}{http://www.who.int/classifications/icd/en/} }
    \end{tablenotes}
  \end{threeparttable}
\end{table}

%--------------------------------------------------------------------
% Useful links for the paper
%-------------------------------------------------------------------
\newpage
\subsection{Useful links for the paper}

Which item shall I include in the metabolic syndrome: 
\begin{enumerate}
\item obesity
\item hypertension
\end{enumerate}

\begin{itemize}
\item \textbf{breastfeeding \& metabolism:}

\begin{itemize}
\item[-]\href{http://www.who.int/elena/titles/bbc/breastfeeding_childhood_obesity/en/}{WHO link})\\ 
\textit{"The positive impact of breastfeeding on lowering the risk of death from infectious diseases in the first two years of life is now well-established (1). A mounting body of evidence suggests that breastfeeding may also play a role in programming noncommunicable disease risk later in life (2-13) including protection against overweight and obesity in childhood (2-6)."}

\item[-]\href{http://articles.latimes.com/2011/may/02/news/la-heb-infant-feeding-20110502}{LA Times article}\\
\textit{"The study showed that children who received breast milk for the first four months had a specific pattern of growth and metabolic profile that differed from the formula-fed babies. Even at 15 days of life, the breast-fed infants had blood insulin levels that were lower than the formula-fed infants.\newline
By 3 years of age, many of the metabolic and growth differences between the breast-fed and formula-fed infants had disappeared. However, blood pressure readings were higher in the infants who had been fed the high-protein formula compared with breast-fed infants. The blood pressure rates were still within the normal range."}
\end{itemize}
\end{itemize}















\end{document}