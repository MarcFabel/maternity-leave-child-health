%--------------------------------------------------------------------
%	DOCUMENT CLASS
%--------------------------------------------------------------------
\documentclass{scrartcl} % type of document (paper, presentation, book,...); scrartcl class with sans serif titles, European layout 
\usepackage{fullpage} % leaves less space at margins of page
\usepackage[onehalfspacing]{setspace} % determine line pitch to 1.5

%--------------------------------------------------------------------
%	INPUT
%--------------------------------------------------------------------
\usepackage[T1]{fontenc} % Use 8-bit encoding that has 256 glyphs
\usepackage[utf8]{inputenc} % Required for including letters with accents, Umlaute,...
\usepackage{float} % better control over placement of tables and figures in the text
\usepackage{graphicx} % input of graphics
\usepackage{xcolor} % advanced color package
\usepackage{url, hyperref} % include (clickable) URLs

%--------------------------------------------------------------------
%	TABLES, FIGURES, LISTS
%--------------------------------------------------------------------
\usepackage{booktabs} % better tables
\usepackage{threeparttable} % add notes below tables

\renewcommand\TPTrlap{}		% add margins on the side of the notes
\renewcommand\TPTnoteSettings{%
\setlength\leftmargin{5 pt}%
\setlength\rightmargin{5 pt}%
}
\usepackage[center, format=plain, font=normalsize, nooneline, labelfont={bf}]{caption} % change format of captions of tables and graphs 

%--------------------------------------------------------------------
%	MATH
%--------------------------------------------------------------------
\usepackage{amsmath,amssymb,amsfonts} % more math symbols and commands

%--------------------------------------------------------------------
%	LANGUAGE SPECIFICS
%--------------------------------------------------------------------
\usepackage[american]{babel} % man­ages cul­tur­ally-de­ter­mined ty­po­graph­i­cal (and other) rules, and hy­phen­ation pat­terns
\usepackage{csquotes} % language specific quotations



\usepackage{graphicx}
\usepackage[left=3cm,right=3cm,top=2cm,bottom=2cm]{geometry}
%--------------------------------------------------------------------
%	AUTHOR & TITLE
%--------------------------------------------------------------------
\author{Natalia Danzer \& Marc Fabel}
\title{Maternity Leave Coverage and the Long-Term Health Outcomes of Children in Germany}

%--------------------------------------------------------------------
%	BEGIN DOCUMENT
%--------------------------------------------------------------------
\begin{document}
\maketitle
\tableofcontents
\newpage



\section{Useful links for the paper}

Which item shall I inlcude in the metabolic syndrome: 
\begin{enumerate}
\item obesity
\item hypertension
\end{enumerate}

\begin{itemize}
\item \textbf{breastfeeding \& metabolism:}

\begin{itemize}
\item[-]\href{http://www.who.int/elena/titles/bbc/breastfeeding_childhood_obesity/en/}{WHO link})\\ 
\textit{"The positive impact of breastfeeding on lowering the risk of death from infectious diseases in the first two years of life is now well-established (1). A mounting body of evidence suggests that breastfeeding may also play a role in programming noncommunicable disease risk later in life (2-13) including protection against overweight and obesity in childhood (2-6)."}

\item[-]\href{http://articles.latimes.com/2011/may/02/news/la-heb-infant-feeding-20110502}{LA Times article}\\
\textit{"The study showed that children who received breast milk for the first four months had a specific pattern of growth and metabolic profile that differed from the formula-fed babies. Even at 15 days of life, the breast-fed infants had blood insulin levels that were lower than the formula-fed infants.\newline
By 3 years of age, many of the metabolic and growth differences between the breast-fed and formula-fed infants had disappeared. However, blood pressure readings were higher in the infants who had been fed the high-protein formula compared with breast-fed infants. The blood pressure rates were still within the normal range."}
\end{itemize}
\end{itemize}

















\newpage
\section{Appendix}
\subsection{Outcomes Hospital registry data}


\begin{table}[h] % table environment for caption and label
\begin{threeparttable}
\centering % center the tabular
\caption{Main diagnosis chapters} % caption
\label{tab:outcomes_coding_main_chapters} 
\begin{tabular}{lrrr} % alignment and number of columns of actual table
\toprule % top thicker horizontal line (" rule ")
&\multicolumn{2}{c}{ICD-9} & \multicolumn{1}{c}{ICD-10} \\ 
\cmidrule(lr){2 -3} \cmidrule(lr){4 -4}
%	qui gen Diag_1	= 1 if (diag000 >= 001 & diag000 <= 139)	// Infektiöse Krankheiten
%	qui gen Diag_2	= 1 if (diag000 >= 140 & diag000 <= 239)	// Neubildungen
%	qui gen Diag_4 	= 1 if (diag000 >= 240 & diag000 <= 278)	// Stoffwechselkrankheiten
%	qui gen Diag_5	= 1 if (diag000 >= 290 & diag000 <= 319) 	//*mental and behavioral disorders
%	qui gen Diag_6  = 1 if (diag000 >= 320 & diag000 <= 359)	// Krankheiten des Nervensystems
%	qui gen Diag_7  = 1 if (diag000 >= 360 & diag000 <= 389)	// Auge & Ohr
%	qui gen Diag_8	= 1 if (diag000 >= 390 & diag000 <= 459)	//*Krankheiten des Kreislaufsystems
%	qui gen Diag_9	= 1 if (diag000 >= 460 & diag000 <= 519)	//*Atmungssystem
%	qui gen Diag_10 = 1 if (diag000 >= 520 & diag000 <= 579)	//* Verdauungssystem
%	qui gen Diag_11 = 1 if (diag000 >= 680 & diag000 <= 709)	// Haut
%	qui gen Diag_12 = 1 if (diag000 >= 710 & diag000 <= 739)	// Muskel_Skelett_Bindegewebe
%	qui gen Diag_13 = 1 if (diag000 >= 580 & diag000 <= 629)	// Urogenitalsystem
%	qui gen Diag_14 = 1 if (diag000 >= 630 & diag000 <= 676)	// Schwangerschaft
%	qui gen Diag_17 = 1 if (diag000 >= 780 & diag000 <= 799)	// Symptome & abnorme klinische Befunde
%	qui gen Diag_18 = 1 if (diag000 >= 800 & diag000 <= 999)	// Verletzungen und co
	
 001–139: infectious and parasitic diseases
 140–239: neoplasms
 240–279: endocrine, nutritional and metabolic diseases, and immunity disorders
 280–289: diseases of the blood and blood-forming organs
 290–319: mental disorders
 320–359: diseases of the nervous system
 360–389: diseases of the sense organs
 390–459: diseases of the circulatory system
 460–519: diseases of the respiratory system
 520–579: diseases of the digestive system
 580–629: diseases of the genitourinary system
 630–679: complications of pregnancy, childbirth, and the puerperium
 680–709: diseases of the skin and subcutaneous tissue
 710–739: diseases of the musculoskeletal system and connective tissue
 740–759: congenital anomalies
 760–779: certain conditions originating in the perinatal period
 780–799: symptoms, signs, and ill-defined conditions
 800–999: injury and poisoning
 
 * Indices:
	* 1) Metabolic Syndorme
	qui gen Diag_metabolic_syndrome = 1 if (diag000 == 250) // diabetes
	qui replace Diag_metabolic_syndrome = 1 if (diag000 == 401 | diag000 == 402 | diag000 == 404 | diag000 == 405) //bluthochdruck
	qui replace Diag_metabolic_syndrome = 1 if (diag000 == 278) // Adipositas und übergewicht
	qui replace Diag_metabolic_syndrome = 1 if (diag000 >= 410 & diag000 <= 414) //Ischämische Herzkrankheiten
	* 2) Respiratory Sytem
	qui gen Diag_Index_respiratory = 1 if (diag000 >= 460 & diag000 <= 466)	// Akute Infektionen der Atmungsorgane
	qui replace Diag_Index_respira = 1 if (diag000 >= 480 & diag000 <= 486)	// Lungenentzündung
	qui replace Diag_Index_respira = 1 if (diag000 >= 490 & diag000 <= 496 & diag000 != 495)	// Chron Entzündung untere Atemweg (inkl Asthma)
	
	*Einzeldiagnosen
	qui gen Diag_symp_circ_resp = 1 if (diag000 == 785 | diag000 == 786)
	qui gen Diag_symp_verdauung = 1 if diag000 == 787	// Syptmoe des Verdauungssytems
	qui gen Diag_sonst_herzkrank = 1 if (diag000 >= 420 & diag000 <= 429 & diag000 != 424)
	qui gen Diag_psych_drogen = 1 if (diag000 == 291 | diag000 ==303 | diag000 == 980 | diag000 == 304 | diag000 == 305)
	
%--------------------------------------------------------------------
%	ICD-10
%--------------------------------------------------------------------	
	qui replace diagOvrvw=1  if diagX == "A" | diagX == "B"						
	qui replace diagOvrvw=2  if diagX == "C" | (diagX == "D" & diag00<=48)		
	*qui replace diagOvrvw=3  if diagX == "D" & (diag00 >=50 & diag00<=90)		
	qui replace diagOvrvw=4  if diagX == "E" & (diag00 >=00 & diag00 <=90)		
	qui replace diagOvrvw=5  if diagX == "F"									
	qui replace diagOvrvw=6  if diagX == "G"									
	qui replace diagOvrvw=7  if diagX == "H" & diag00<=95						
	qui replace diagOvrvw=8  if diagX == "I"									
	qui replace diagOvrvw=9  if diagX == "J"									
	qui replace diagOvrvw=10 if diagX == "K" & diag00 <=93						
	qui replace diagOvrvw=11 if diagX == "L" 									
	qui replace diagOvrvw=12 if diagX == "M"									
	qui replace diagOvrvw=13 if diagX == "N"									
	qui replace diagOvrvw=14 if diagX == "O"									
	qui replace diagOvrvw=15 if diagX == "P" & diag00 <=96						
	qui replace diagOvrvw=16 if diagX == "Q"									
	qui replace diagOvrvw=17 if diagX == "R"									
	qui replace diagOvrvw=18 if diagX == "S" | diagX == "T"						
	qui replace diagOvrvw=19 if diagX == "Z"
	
	
#delimit;
	label define DIAGNOSEN
	1  "[A00-B99] Infektiöse und parasitäre Krankheiten"
	2  "[C00-D48] Neubildungen"
	3  "[D50-D90] Krankheiten des Blutes und blutbildenden Organe"
	4  "[E00-E90] Endokrine, Ernährungs- u Stoffwechselkrankh."
	5  "[F00-F99] Psychische und Verhaltensstörungen"
	6  "[G00-G99] Krankheiten des Nervensystems"
	7  "[H00-H95] Krankheiten des Auges und Ohres"
	8  "[I00-I99] Krankheiten des Kreislaufsystems"	
	9  "[J00-J99] Krankheiten des Atmungssystems"
	10 "[K00-K93] Krankheiten des Verdauungssystems"
	11 "[L00-L99] Krankheiten der Haut"
	12 "[M00-M99] Muskel, Skelett u Bindegewebe"
	13 "[N00-N99] Urogenitalsystem"
	14 "[O00-O99] Schwangerschaft, Geburt, Wochenbett"
	15 "[P00-P96] Perinatalperiode"
	16 "[Q00-Q99] Angeborene Fehlbildungen, Deformitäten, Chromosomenabnomalien"
	17 "[R00-R99] Symptome, aborm. Laborbefunde"
	18 "[S00-T98] Verletzungen, Vergiftungen, äußere Ursachen"
	19 "[Z00-Z99] anderes, z.B Neugeborene"	;
#delimit cr
	
//Indices 
	* 1) Metabolic Syndorme
	qui gen Diag_metabolic_syndrome = 1 if (diagX == "E" & (diag00 ==11 | diag00 ==12)) // diabetes [HIER NUR TYP 2!!!]
	qui replace Diag_metabolic_syndrome = 1 if (diagX == "I" & (diag00 == 10 | diag00 == 11 | diag00 == 13 | diag00 == 15)) //bluthochdruck
	qui replace Diag_metabolic_syndrome = 1 if (diagX == "E" & (diag00 >= 65 & diag00 <= 68)) // Adipositas und übergewicht
	qui replace Diag_metabolic_syndrome = 1 if (diagX == "I" & (diag00 >= 20 & diag00 <= 25)) //Ischämische Herzkrankheiten
	* 2) Respiratory system
	qui gen Diag_Index_respiratory = 1 if (diagX == "J" & (diag00 >= 40 & diag00 <= 47)) // Chron Entzündung untere Atemweg (inkl Asthma)
	qui replace Diag_Index_respira = 1 if (diagX == "J" & (diag00 >= 20 & diag00 <= 22)) // akute Bronchitis
	qui replace Diag_Index_respira = 1 if (diagX == "J" & (diag00 >= 12 & diag00 <= 18)) // Lungenentzündung
		
	*Einzeldiagnosen
	qui gen Diag_symp_circ_resp = 1 if (diagX == "R" & (diag00 >= 00 & diag00 <= 09))
	qui gen Diag_symp_verdauung = 1 if (diagX == "R" & (diag00 >= 10 & diag00 <= 19))
	qui gen Diag_sonst_herzkrank = 1 if diagX == "I" & ((diag00>=30 & diag00<=33) | (diag00>=39 & diag00<=52))  
	qui gen Diag_psych_drogen = 1 if (diagX == "F" & diag00==10) | (diagX == "T" & diag00==51) 
	qui replace Diag_psych_drogen = 1 if (diagX == "F" & (diag00>= 11 & diag00<=19 & diag00 !=17) )	
	
%List of ICD-9 codes E and V codes: external causes of injury and supplemental classification
\bottomrule % bottom thicker horizontal line (" rule ")
\end{tabular}
\begin{tablenotes}
      \scriptsize{ \item \textit{Notes:} From European shortlist
      International statistical classification of diseases and related health problems (ICD), medical classification list provided by the World Health Organisation.}
    \end{tablenotes}
  \end{threeparttable}
\end{table}




\end{document}