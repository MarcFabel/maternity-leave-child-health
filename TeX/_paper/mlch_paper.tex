% 	TO DO: 
%	MZ Tabelle nicht einfügen: DDRD estimate von Regression mit 3 CG


%--------------------------------------------------------------------
%	DOCUMENT CLASS
%--------------------------------------------------------------------
\documentclass[11pt, a4paper]{article} % type of document (paper, presentation, book,...); scrartcl class with sans serif titles, European layout 
\usepackage{fullpage} % leaves less space at margins of page
\usepackage[onehalfspacing]{setspace} % determine line pitch to 1.5

%--------------------------------------------------------------------
%	INPUT
%--------------------------------------------------------------------
\usepackage[T1]{fontenc} 	% Use 8-bit encoding that has 256 glyphs
\usepackage[utf8]{inputenc} % Required for including letters with accents, Umlaute,...
\usepackage{float} 			% better control over placement of tables and figures in the text
\usepackage{graphicx} 		% input of graphics
\usepackage{xcolor} 		% advanced color package
\usepackage{url, hyperref} 	% include (clickable) URLs
\usepackage{pdfpages}		% insert pages of external pdf documents
\setlength{\parskip}{0em}	% vertical spacing for paragraphs
\setlength{\parindent}{0em}	% horizonzal spacing for paragraphs
\usepackage{tikz}
\usepackage{tikzscale}		% helps to adjust tikz pictures to textwidth/linewidth
\usetikzlibrary{decorations.pathreplacing}
\usetikzlibrary{patterns}
\usetikzlibrary{arrows}
\usepackage{eurosym}		% Eurosymbol

% Have sections in TOC, but not in text
\usepackage{xparse}% for easier management of optional arguments
\ExplSyntaxOn
\NewDocumentCommand{\TODO}{msom}
{
	\IfBooleanF{#1}% do nothing if it's starred
	{
		\cs_if_eq:NNT #1 \chapter { \cleardoublepage\mbox{} }
		\refstepcounter{\cs_to_str:N #1}
		\IfNoValueTF{#3}
		{
			\addcontentsline{toc}{\cs_to_str:N #1}{\protect\numberline{\use:c{the\cs_to_str:N #1}}#4}
		}
		{
			\addcontentsline{toc}{\cs_to_str:N #1}{\protect\numberline{\use:c{the\cs_to_str:N #1}}#3}
		}
	}
	\cs_if_eq:NNF #1 \chapter { \mbox{} }% allow page breaks after sections
}
\ExplSyntaxOff

%--------------------------------------------------------------------
%	TABLES, FIGURES, LISTS
%--------------------------------------------------------------------
\usepackage{booktabs} 		% better tables
\usepackage{longtable}		% tables that may be continued on the next page
\usepackage{threeparttable} % add notes below tables
\renewcommand\TPTrlap{}		% add margins on the side of the notes
	\renewcommand\TPTnoteSettings{%
	\setlength\leftmargin{5 pt}%
	\setlength\rightmargin{5 pt}%
}
\usepackage[
center, format=plain,
font=normalsize,
nooneline,
labelfont={bf}
]{caption} 				% change format of captions of tables and graphs 
%USED IN MPHIL: \usepackage[labelfont=bf,labelsep = period, singlelinecheck=off,justification=raggedright]{caption}, other specifications which are nice: labelformat = parens -> number in paranthesis 


%\usepackage{threeparttablex} % for "ThreePartTable" environment, helps to combine threepart and longtable

% Allow line breaks with \\ in column headings of tables
\newcommand{\clb}[3][c]{%
	\begin{tabular}[#1]{@{}#2@{}}#3\end{tabular}}

% allow line breaks with \\ in row titles
\usepackage{multirow}

\newcommand{\rlb}[3][c]{%
\multirow{2}{*}{\begin{tabular}[#1]{@{}#2@{}}#3\end{tabular}}}% optional argument: b = bottom or t= top alignment


\usepackage[singlelinecheck=on]{subcaption}%both together help to have subfigures
\usepackage{wrapfig}				% wrap text around figure


\usepackage{rotating}				% rotating figures & tables
\usepackage{enumerate}				% change appearance of the enumerator
\usepackage{paralist, enumitem}		% better enumerations
\setlist{noitemsep}					% no additional vertical spacing for enurations
%--------------------------------------------------------------------
%	MATH
%--------------------------------------------------------------------
\usepackage{amsmath,amssymb,amsfonts} % more math symbols and commands
\let\vec\mathbf				 % make vector bold, with no arrow and not in italic

%--------------------------------------------------------------------
%	LANGUAGE SPECIFICS
%--------------------------------------------------------------------
\usepackage[american]{babel} % man­ages cul­tur­ally-de­ter­mined ty­po­graph­i­cal (and other) rules, and hy­phen­ation pat­terns
\usepackage{csquotes} % language specific quotations

%--------------------------------------------------------------------
%	BIBLIOGRAPHY & CITATIONS
%--------------------------------------------------------------------
\usepackage{csquotes} % language specific quotations
\usepackage{etex}		% some more Tex functionality
\usepackage[nottoc]{tocbibind} %add bibliography to TOC
\usepackage[authoryear, round, comma]{natbib} %biblatex

%--------------------------------------------------------------------
%	PATHS
%--------------------------------------------------------------------
\makeatletter
\def\input@path{{../../analysis/tables/}}	%PATH TO TABLES
%or: \def\input@path{{/path/to/folder/}{/path/to/other/folder/}}
\makeatother
\graphicspath{{../../analysis/graphs/}}		% PATH TO GRAPHS

%--------------------------------------------------------------------
%	LAYOUT
%--------------------------------------------------------------------
\usepackage[left=3cm,right=3cm,top=2cm,bottom=3cm]{geometry}
\usepackage{pdflscape} % lscape.sty Produce landscape pages in a (mainly) portrait document.

\definecolor{darkblue}{rgb}{0.0,0.0,0.6}
\newcommand\natalia[1]{\textcolor{orange}{#1}}

% CAPTIAL LETTERS FOR SECTION CAPTIONS
%\usepackage{sectsty}
%\sectionfont{\normalfont\scshape\centering\textbf}
%\renewcommand{\thesection}{\Roman{section}.}
%\renewcommand{\thesubsection}{\Alph{subsection}.}%\thesection\Alph{subsection}.
%\subsectionfont{\itshape}
%\subsubsectionfont{\scshape}
%\newcommand\relphantom[1]{\mathrel{\phantom{#1}}}
%\setlength\topmargin{0.1in} \setlength\headheight{0.1in}
%\setlength\headsep{0in} \setlength\textheight{9.2in}
%\setlength\textwidth{6.3in} \setlength\oddsidemargin{0.1in}
%\setlength\evensidemargin{0.1in}

\hypersetup{
  colorlinks  = true,
  citecolor   = darkblue,
 	linkcolor   = darkblue,
  urlcolor    = darkblue 
} % macht die URLS blau   
     
\usepackage{lettrine}	% First letter capitalized

% have date in month year format (i.e. omit the day in dates)
\usepackage{datetime}
\newdateformat{monthyeardate}{%
  \monthname[\THEMONTH], \THEYEAR}
%--------------------------------------------------------------------
%	AUTHOR & TITLE
%--------------------------------------------------------------------
\title{Maternity Leave and Long-Term Health Outcomes of Children\footnote{I am very grateful to Natalia Danzer and Helmut Rainer for valuable comments and discussions. I also benefited from inspiring feedback from Birgitta Rabe, Daniel Kühnle, Maarten Lindeboom, Erik Plug, Anna Raute, Tanya Wilson and participants at several conferences. Furthermore, I thank Heiko Bergmann for assistance in accessing data. The author gratefully acknowledges financial support from the Leibniz association. Maximilian Grasser and Paulina Hofmann provided excellent research assistance. All errors and omissions are my own.
}}
\author{
	Marc Fabel 
		\thanks{Munich Graduate School of Economics (MGSE) and ifo Institute for Economic Research, ifo Center for Labor and Demographic Economics (email: \href{mailto:fabel@ifo.de}{fabel@ifo.de}).
		}
}

\date{\monthyeardate\today}








%--------------------------------------------------------------------
%	BEGIN DOCUMENT
%--------------------------------------------------------------------
\begin{document}
\setcounter{page}{0}  
% \tableofcontents
\newpage
\setcounter{page}{1}    
\maketitle

\textbf{\color{red} Preliminary and incomplete draft\newline Please do not cite or circulate without the authors' permission}
\renewcommand{\abstractname}{\vspace{-\baselineskip}} % GET RID OF ABSTRACT TITLE

  \begin{abstract}\noindent 
   \footnotesize{\begin{center}\textbf{Abstract}\end{center} This paper assesses the impact of the length of maternity leave on children’s long-run health outcomes. Our quasi-experimental design evaluates an expansion in maternity leave coverage from two to six months, which occurred in the Federal Republic of Germany in 1979. The expansion came into effect after a sharp cutoff date and significantly increased the time working mothers stayed at home with their newborns during the first six months after childbirth. In our analysis, we exploit German hospital registry data, containing detailed information about the universe of inpatients' diagnoses for the years 1995 to 2014. 
   By tracking the health of treated and control children from age 16 up to age 35, we provide new insights into the trajectory of health differentials over the life-cycle.
   We find a positive effect of the legislative change on several measures of long-term child health. Our intention-to-treat estimates suggest that children who were born shortly after the implementation of the reform experience fewer hospital admissions and are less likely to be diagnosed with mental and behavioral disorders.
   	\\\newline \textbf{Keywords:} Early childhood development, health, paid maternity leave, life-cycle approach \newline \textbf{JEL codes:} I10, J13, J18}
    \end{abstract}

\newpage


%--------------------------------------------------------------------
% INTRODUCTION
%--------------------------------------------------------------------
\section{Introduction}\label{sec:introduction}


\newpage
\textbf{Notes what we could potentially include in the introduction}\newline
Intro –
Health paper or PL paper?
Why relevant?
What do we (not) know?
What do we do?
What do we find?

To which literature do we contribute to?
\begin{itemize}
	\item Parental leave literature
	\item Literature on the role of early childhood interventions on long-run child development
	\begin{itemize}
		\item The role of type of nurture at the beginning of life on later health outcomes
		\item Fetal origin hypotheses extended
	\end{itemize}
	\item Spill-over of labor market policy on health outcomes
\end{itemize}

What is long-term health??( evtl auch in die background section?) 
\begin{itemize}
	\item Hospital admission
	\item why relevant? costs...
	\item what are frequencies (compare to ND's picture for the IZA presentation)
\end{itemize}


\textbf{RELEVANT LITERATURE, either other LR domains or SR health} 
\begin{itemize}
	\item long-run domain
	\begin{itemize}
		\item intro \cite{currie2011human} \& \cite{almond2017childhood}
		\item find positive effect: \cite{carneiro2015flying} Norway; \cite{albagli2018}: reform 2011 in chile, pos effect on child cognitive skills (0.2 std), stronger effect for children of mothers with low education levels persistence in adulthood (mention effect on breastfeeding)  \cite{danzer2017} parental leave Austria (change 12->24 months job protection and benefit period) no effect on PISA scores in pooled sample, but positive effects for children with a high SES background , in particular for boys, detrimental effects for children from lower educated mothers $\rightarrow$ "when no formal  child care system: early maternal employment of highly educated women might have detrimental effects for their offspring"
		\item find null effect \cite{Dahl2016Case}, \cite{rasmussen2010increasing} (parental access to birth-related leave)
	\end{itemize}
	\item health domain
	\begin{itemize}
		\item \cite{stearns2015effects} and \cite{rossin2011effects} (backup birth-weight \cite{almond2005costs} and \cite{currie2007biology})
		\item \cite{beuchert2016}
		\item \cite{adhvaryu2018early}: effects of changes in real producer price of cocoa in early life on mental health outcomes in Ghana
	\end{itemize}
	\item beides combined: \cite{danzer2017parental} disability status, fit for military service, finding positive effects on children's health status but none for educational attainment and labor market outcomes. No heterogeneity by gender or SES, but by counterfactual mode of care (pos effects in regions with no nursery).
\end{itemize}

\newpage 
\textbf{From the abstract of our submission to the dggö}

This paper assesses the impact of the length of maternity leave on children’s long-run health outcomes. Our quasi-experimental design evaluates an expansion in maternity leave coverage from two to six months, which occurred in the Federal Republic of Germany in 1979. The expansion came into effect after a sharp cutoff date and significantly increased the time working mothers stayed at home with their newborns during the first six months after childbirth. In our analysis, we exploit German hospital registry data, containing detailed information about the universe of inpatients' diagnoses for the years 1995 to 2014. \newline

In order to estimate the causal effect of the length of maternity leave on child health outcomes we exploit the exogenous variation stemming from the reform, which provides a treatment assignment that is as good as random. In order to eliminate potential season of birth effects, we combine a regression discontinuity with a difference-in-difference approach: health outcomes of children, who were born shortly before and after the implementation of the policy reform, are compared to the outcomes of children who are born in the same calendar months but in the previous year in which no legislative change took place.\newline

By tracking the health of treated and control children from age 16 up to age 35, we provide new insights into the trajectory of health differentials over the life-cycle.
We find a positive effect of the legislative change on several measures of long-term child health. Our intention-to-treat estimates suggest that children who were born shortly after the implementation of the reform experience fewer hospital admissions.
In particular, we see that this decline in hospital admissions is due to fewer diagnoses of mental and behavioral disorders (the most common diagnoses for individuals aged 15-35) and that the effect is mostly driven by males and occurs towards the end of the observed time period (at the end of 20s up to 35). Last, the largest effect is observed for mental and behavioral disorders due to psychoactive substance use and schizophrenia. 
Investigating other health outcomes, we see either positive (diseases of the respiratory and digestive system, and injuries) or null effects. \newline



We contribute to the growing literature on parental leave and child development by adding new and unique insights into long-run health effects. Furthermore, it is crucial to understand the implications of maternity leave on various aspects of child development and outcomes for a comprehensive assessment of the cost and benefits of such leave schemes. These results illustrate the vast impact that early childhood conditions can have on later life (health) outcomes. Our intention-to-treat estimates suggest that the expansion in maternity leave from two to six months has a positive impact on children’s health in the long-run.
% Due to the fact that (long-run) health outcomes were not so much in the center of the discussion and the health effects under investigation only materialize later in life, we hope to add to the current debate about the costs and benefits of such leave schemes.






% Nice way of framing taken from ACD (2017) NBER 
%  -AC(2011b) initial effects of something fade out in the beginning and reappear in adulthood 
%  - "Broadly considered, there are two types of resources that can be expected to benefit children: Material resources (Y) and time inputs (It), which might be an argument in the production of child investment" 
%  - Maternity Leave: "If childhood investments are an increasing function of parental time, then maternity leave policies may increase investments at key developmental stages. Such policies appear to be predicated on the belief that the elasticity of child investments in (1) with respect to parental time is large in very early childhood. The key policy question is when specifically maternal (or paternal) time is most important?" 
% heterogenity in the effect (according to parental SES) not all parents can make use of their resouirfces in the same efficient way; implies different production functions 
% early childhood environmetn in the context of intergenerational mobility
% [Q ND: Motivation - shall we have a cross-country (maybe OECD data) scatter with linear fit, Y: health outcomes and X lenght of Maternity/parental leave]





%--------------------------------------------------------------------
% BACKGROUND
%--------------------------------------------------------------------
\newpage
\section{Background}\label{sec:background}
\subsection[Reform]{Institutional set-up}
%reform
In contrast to the United States, maternity leave laws have been established much longer in Germany.\footnote{The following facts about maternity leave and benefit legislation are based on information in \cite{DIW2002}, \cite{schonberg2014expansions}, \cite{Dustmann2012}, and  \cite{zmarzlik1999mutterschutzgesetz}.} Before another reform took place in 1986, only mothers were eligible for job-protected leave.\footnote{The leave scheme described here does not correspond to the system, which is in place at the moment. After a series of reforms in the 1980s and 1990s that prolonged the job protection and/or benefit period, the current system was installed in 2007. \cite{Kluve2013} offer a good overview about the current parental leave regulation in place.} Since the mid 1950s employed mothers held the right to a paid protection period of six weeks before and eight weeks after child birth.\footnote{Compare with: "Gesetz zum Schutze der erwerbstätigen Mutter" (Mother-protection law), Bundesgesetzblatt (Federal law gazette), Part I, Nr. 5, p. 69-74, 30.01.1952.} 
 \newline During that 'mother protection period' women must not work, but they are protected from being dismissed and upon their return to work they hold the right to be placed to a job, which is comparable with their prior assignment. The benefits in this period correspond to a 100\% replacement rate and equal women's average income over the three months before childbirth. The payment is co-funded by public health insurance funds (750 DM per month), the federal government (400 DM, one-time payment) and employers (the remainder). This pre-reform setting is to some extent comparable to the current maximum of 12 weeks of unpaid, job-protected leave in the US (under the FMLA) and the minimum of 14 weeks of paid, job-protected leave in the EU \citep{guertzgen2018}.\footnote{Since the Family and Medical Leave Act of 1993 (FMLA), mothers in the US are entitled to leave if they have been working for at least one year with their employer, have accumulated a minimum of 1250 working hours during that year, and if they have been working for an employer with at least 50 employees \citep{baum2003effect}.} \newline

The socio-liberal coalition of chancellor Helmut Schmidt passed a reform bill in 1979, which introduced four extra months after the mother protection period ended up to the point when the child is six months old. In other words, the total length of maternity leave after childbirth (job protection and benefits) was increased from eight weeks to six months (see Figure \ref{fig: MLreform}).\footnote{Compare with: "Gesetz zur Einführung eines Mutterschutzurlaubes" (Maternity leave law), Bundesgesetzblatt (Federal law gazette), Part I, Nr. 32, p.797-802, 30.06.1979 (see Figure \ref{fig: bundesgesetzblatt_antjehuber}).} The federal government primarily wanted to safeguard maternal health after childbirth with this reform. However, positive spill-over effects on the child were pleasantly acknowledged.\footnote{Gesetzesentwurf der Bundesregierung (Draft bill), Drucksache 8/2613.} While the initial benefits of the period from six weeks before and eight weeks after childbirth were not changed, the payments were equal to 750 DM from the third month after delivery.\footnote{This amount corresponds to approximately 44\% of average prebirth earnings in 1979 \citep{schonberg2014expansions}.} Although eligibility for maternity leave was universal among working women, take-up rates were not. The approximated maternity leave take-up rate is slightly below 40\% in 1979 \citep{Dustmann2012}. Yet, it is very likely that the true share of mothers who was on leave, is considerably higher. This is due to the fact that \cite{Dustmann2012} approximate this share as the ratio of the number of women on leave in their data set ($\sim$ 80\% of the workforce) divided by the number of births in that year.
%It should last until the 1986 reform until all mothers (irrespective of their employment status) and fathers became eligibile for parental leave.
\newline

The reform was initiated by a draft bill on January 05, 1979. The final law was ratified by the German Bundesrat (the Upper House of the German Parliament) on May 19 and by the German Bundestag (the Lower House) on June 22, 1979.  All previously employed women, who gave birth on/after May 01, 1979, were eligible for six months of maternity leave after childbirth, whereas mothers, who delivered their baby before the cutoff date were only entitled to the 'common' 2 months of job-protected maternity leave. Note -- in relation to behavioral responses -- that the conception period for births when the reform took effect was way earlier than when the draft bill was proposed. This implies that families were not able to anticipate the legislation change and the 1979 reform in maternity leave can be seen as a quasi-experiment. This issue is discussed in more detail in Section \ref{sec:empirical_strategy_threats+validity}.\newline



%--------------------------------------------------------------------    
% FEMALE LABOR FORCE PARTICIPATION AND CHILDCARE SITUATION
\subsection{Female labor force participation and childcare situation}
In April 1979, the average female labor force participation rate was at 49.7\% for women in the age group between 15 and 65 years and women comprised around 38\% of the labor force \citep{federalstatisticaloffice1981yearbook}.\footnote{At the same time, as a comparison, 50.6\% of all women in the US participated in the labor market (see US Bureau of Labor Statistics).} Yet, the average female labor force participation rate masks pronounced heterogeneity. For instance, female labor market attachment was at 62.4\% for singles, 45.2\% for married, 32.5\% for widowed, and 76.5\% for divorced women. Additionally, there was a strong gradient with respect to age. While 69.2\% of all women aged between 20 and 25 years were participating in the labor force, the share is 55.0\% for women in the age bracket between 30 and 35 years.\footnote{Women aged 20 to 35 are the relevant age group in this context, as 84\% of all children were delivered by these women \citep{federalstatisticaloffice1981yearbook}.} The high numbers for younger women and singles indicate that a high share of mothers-to-be were an active part of the labor force and thus eligible for maternity leave. \newline 
%between year1 and year 2, maternal labor market attachment increased from..x to y 

Nevertheless, next to the female labor force participation rate, it is the counterfactual mode of care that has an impact on how the reform can alter children's outcomes. \cite{danzer2017parental} show that the 1990 parental leave expansion in Austria had a positive effect on children in regions where there was no formal childcare.\footnote{The Austrian reform increased paid and job-protected parental leave from one year to two years.} In other words, there were only positive effects of the reform in cases when informal child care was substituted with parental care. The provision of childcare in West Germany in the late 1970s was characterized by a situation that would allow a shift from informal to maternal care. \cite{hank2001childcare} describe the situation at that time as a \textit{'patchwork [of] childcare arrangements'}, meaning that parents had to rely on a broad range of care types, such as parental care, day care centers (nurseries), social networks, and private child minders. The different forms vary greatly in cost and quality. In 1980, only 1.5\% of children attended a public \textit{Krippe} (nursery, for children aged 0 to 3) \citep[p.~34]{bildungsbericht2006}.\footnote{In the 1970s the provision of part-time care for pre-schoolers (4-6 years) was established. However, only since 1996, children (aged 3 to school age) were legally entitled to a slot in a public \textit{Kindergarten}. For toddlers (1-3 years), parents have a legal claim to a care slot in a \textit{Krippe} since 2013.} Thus, due to the fact that public daycare was basically nonexistent, parents had to rely almost exclusively on informal care, apart from parental care. This makes the situation in the Federal Republic of Germany in the late 1970s quite comparable to the situation of \cite{danzer2017parental} in which informal arrangements were substituted with maternal care, allowing a positive impact on child outcomes.
%[XXX Shall we look for information about quality of care? ]



%--------------------------------------------------------------------    
\bigskip
\subsection[Effects of 1979 reform on outcomes]{Effects of the 1979 maternity leave reform on maternal and child labor market outcomes}
%[XXX Ich habe jetzt pro Aspekt die beiden Zusammengeworfen und nicht jedes paper für sich behandelt, notwendig da SL zB mehr in die Tiefe geht bei maternal outcomes]
The effects of the 1979 maternity leave reform has been investigated by the three studies of \cite{Dustmann2012}, \cite{schonberg2014expansions}, and \cite{guertzgen2018}.\footnote{Both, \cite{Dustmann2012} and \cite{schonberg2014expansions} analyze the impact of the expansion in maternity leave on maternal labor market outcomes. While the former study augments this aspect by evaluating changes in child outcomes due to the reform, the latter study focuses on maternal labor market outcomes but elicits more in-depth results.} \newline


Both, \cite{Dustmann2012} as well as \cite{schonberg2014expansions}, show that the reform has a large impact on mothers' labor market outcomes, particularly in the short-run. \newline First, many mothers strongly adjust their labor supply downwards during the four months of extra leave, and return to the labor market as soon as the leave period terminates. Yet, it seems that there is only a small effect on long-run maternal labor supply (i.e. for the time period beyond six months after childbirth). For instance, while the reform decreases the share of mothers who returned to the labor market by the third month after childbirth by 30.5 percentage points, the reduction in the share of returned mothers by the 52nd/76th month after childbirth is only around one percentage point.\footnote{When looking at long-run maternal labor force participation rates (at the child's sixth birthday), one can see that the reform leads to a modest increase in the probability a mother is working. The implication is that the mothers who abstain from the labor market in the long-run, would have only returned to work temporarily in absence of the reform.}
In total, the change of postnatal maternity leave from two to six months causes mothers to postpone their return to work by, on average, 0.835 months.\footnote{This number corresponds to the number of months away from work in the first 40 months since childbirth.} Furthermore, approximately two-third of the decline in short-run female labor force participation is the result from a contraction in full-time work. Most of the mothers who postpone their labor market return, would have returned to their previous employers in absence of the reform. \newline
Second, the expansion in maternity leave lead to changes in mothers' income. When focusing on mere labor market income there is no effect of the reform that is significantly different from zero. However, when maternity benefit payments are included in the income measure for the eligible women, there is an overall increase in cumulative total income by, on average, 1,700 Deutschmarks (DM) as a result of the reform.\footnote{Maternal cumulative total  income is defined as the accumulated total income up to the point when the child is 40 years old. It consists of monthly earnings when the mother is working, equals to the benefits when she is on leave or is zero otherwise.\newline \cite{Dustmann2012} deflate monthly income such that everything is in 1992 prices. The benefit of 750 DM (1,190 DM in 1992 prices) resembles approximately one-third (55\%) of the mother's pre-birth (post-birth) earnings.} This is due to the fact that the effect of unexpected extra income for mothers who would have stayed at home even without the reform prevails over the decrease in available income stemming from the reduction in maternal employment. In contrast to the impact on labor supply, there is a strong effect heterogeneity in cumulative income across wages (an increase of 2,850 DM for mothers in the lowest tercile of the wage distribution, while the additional income amounts to 1,050 DM for women in the highest tercile). \newline
% vielleicht nochmal wrap up wie in der Discussion von SL, short-run effet on maternal employment yes - LR nichts verbessert, no depreciation of HC... 
Although there are strong effects on maternal labor market outcomes, in particular in the short-run, \cite{Dustmann2012} do not find evidence that the reform has an impact on children's educational attainment and labor market outcomes. They do not find any changes in the levels or years of education, log wages, or the share of individuals in full-time employment in response to the reform. 

% so strong effect on mother's return to work behavior given but so far in the domains that were looked at, there are no effects on children's long-run outcomes.

\cite{guertzgen2018} investigate the impact of the 1979 reform on mothers long-term ($>$ 6 weeks) sickness absence after childbirth. They find positive differentials for mothers who give birth under the more generous maternity leave regime. In other words, mothers who give birth after the threshold exhibit more long-term sickness spells compared to mothers who give birth before the cutoff date. For instance, post-reform mothers have a 3.1 percentage points higher probability of having ever experienced a long-term illness spell by the third year after childbirth. This result remains the same even after controlling for observable health differences. Furthermore, they suggest a selection story in which mothers with worse health status are more likely to return to the labor market and that this group is by a large extent driving the adverse health effects in response to the reform.\newline




    
%[XXXX Q ND: SHALL WE INCLUDE DS(2012) FIGURES, SUCH AS FIG 2A \& 4A-F, AT LEAST IN THE APPENDIX OR DO WE EXPECT THE READER TO LOOK THIS UP HER-/HIMSELF?]    





%--------------------------------------------------------------------
\subsection[Channels]{Potential channels and prior on health effects}
How might the length of maternity leave affect long-run health outcomes of children? The notion that later-life outcomes have their origins in early-childhood is not new. Already in 1990, \citeauthor{Barker1990origins} postulated the idea that conditions in-utero and during infancy have long-lasting effects on later life health. Adult physical and mental wellbeing is influenced by early experiences - whether they are good or bad - in at least two ways \citep{shonkoff2009neuroscience}. On the one hand, this can be a \emph{cumulative} process, in which physically and psychologically stressful events are experienced again and again until chronic health impairments eventually materialize. The persistent experience of stressful events causes a constant provocation of neurobiological responses, which are supposed to help you cope with stress and are under normal circumstances healthy and protective, may become pathogenic.\footnote{The neurobiological reactions can include, for instance, the release of stress hormones, higher blood pressure and heart rate, protective mobilization of nutrients, among other. See for example \cite{mcewen1998stress,shonkoff2009neuroscience}.} On the other hand, the environment at key developmental stages is 'programmed' into regulatory physiological systems such that it impacts adult disease and risk factors latently. This idea of \emph{biological embedding} of adversities is consistent with \citeauthor{Barker1990origins}'s hypothesis. In these sensitive periods, the developing brain's architecture is modified considerably and is thus particularly sensitive to environmental stimuli. The programming process starts in the embryonic state and culminates in the first years of life \citep{raikkonen2012early}. However, as not all brain circuits develop at the same time, it depends on the timing of the experience/adversity.\footnote{See the 'life-cycle' model of stress \citep{lupien2009effects}.} In other words, the adversity has the highest impact on the brain region that undergoes the most change at the time when the adversity is experienced. During infancy, the hippocampus is the area of the brain that is maturing most expeditiously and is consequently more vulnerable than during other stages. This region is tied to the regulation of emotions, social behavior, stress responsiveness, and ultimately mental health \citep{center2016best,shonkoff2009neuroscience}. Yet, the effects on later life health impairment are not only determined by the timing, but also by the type of the experience. In both cases - the cumulative exposure and the programming of the adversity - the effects of experiences made in early life may be latent at first until the onset of a particular condition \citep{almond2011fetalorigins}. The time lag can be many years, even decades \citep{shonkoff2009neuroscience}.\newline


% how the 1979 maternity leave reform might affect early life conditions of the newborns
Before proceeding with the analysis, we provide a short discussion how the child's environment may be altered by the 1979 maternity leave reform, i.e. potential mechanisms through which the reform could affect child health outcomes. For this purpose, we consider differences in care quality, parental health differentials, and changes in family income as potential pathways how the reform may affect long-run child health outcomes.  \newline %disoriented attachment patterns

% more maternal time - higher quality of care
First, the reform induced mothers to postpone their return to the labor market and allowed more maternal time during a crucial time period for child development. With more time at home, mothers are more likely to breastfeed and do so for an extended period \citep{baker2008maternal,berger2005earlymaternal}.\footnote{The German Health Interview and Examination Survey for Children and Adolescents (KiGGS) presents representative data on breastfeeding rates from 1986 onwards \citep{lange2007breastfeeding}. From the West-German 1986 cohort born, around 75\% of children were breastfed at least once and the share of children who was breastfed exclusively for half a year is roughly 38\%.} Breastfeeding is known to have a wide array of medical benefits.\footnote{The World Health Organization recommends that \textit{"infants should be exclusively breastfed for the first six months of life to achieve optimal growth, development and health"} (\href{http://www.who.int/features/factfiles/breastfeeding/en/}{http://www.who.int/features/factfiles/breastfeeding/en/}).} The advantages for children that were breastfed range from reduced incidence or severity of asthma, allergies, diarrhea, mortality, morbidity and chronic conditions in the short run, to lower prevalence rates of obesity and overweight, and type II diabetes in adulthood \citep{ruhm2000parental, victora2016breastfeeding}. Moreover, there is correlational evidence that the length of breastfeeding is negatively associated with mental health problems and adverse health behavior (drinking) \citep{oddy2010longterm,falk2016early}. Yet, there are not only direct health effects, but also indirect effects via the effect of breastfeeding on third outcomes that in turn affect health. For example, breastfeeding exhibits a positive effect on cognitive development \citep{albagli2018}, educational attainment and income \citep{victoria2015association}, the formation of preferences \citep{falk2016early}, and the quality of mother-child interactions \citep{papp2014longitudinal}. \newline 
In addition to reduced breastfeeding, early maternal employment impedes the monitoring of children's health status. \cite{berger2005earlymaternal} present associations of maternal employment and a decrease in the use of preventive health care services (immunizations and 'well-baby' visits), while at the same time problems with externalizing behavior exacerbate. \cite{morrill2011} presents instrumental variable estimates, which suggest that maternal employment leads to a higher likelihood that the child suffers from an adverse health event (overnight hospitalization, asthma episode, or injury/poisoning).\newline 





% attachment theory and neurobiological changes effect of seperation
% Second, in absence of the reform, mothers might have opted for an earlier return to the labor market causing a separation from the primary caregiver. 
% he effect that this separation from the primary caregiver can have on a very young child depends on the quality
% of care and stimulation the child receives from the alternative caregiver
% (Belsky et al. 2007: Are There Long-Term Effects of Early Child Care)\newline




%links to papers that are important for the attachment tehory section
%https://onlinelibrary.wiley.com/doi/abs/10.1002/1097-0355%28200101/04%2922%3A1%3C201%3A%3AAID-IMHJ8%3E3.0.CO%3B2-9 (The effects of early relational trauma on right brain development, affect regulation, and infant mental health)
% https://www.tandfonline.com/doi/abs/10.1080/03004430701292988 (The socio‐emotional effects of non‐maternal childcare on children in the USA: a critical review of recent studies)
% https://developingchild.harvard.edu/science/deep-dives (Havard group on developing children)
% https://www.ncbi.nlm.nih.gov/pubmed/19401723 (Effects of stress throughout the lifespan on the brain, behaviour and cognition)
% http://psycnet.apa.org/buy/2003-01660-002 (Trajectories leading to school-age conduct problems.)
%https://www-sciencedirect-com.emedien.ub.uni-muenchen.de/science/article/pii/S0014292118300953?via%3Dihub (Gender differences in the benefits of an influential early childhood program)






% Parental/ maternal health outcomes
Second, changes in maternal health outcomes, which in turn might affect the ability to nurture, are other mechanisms of how the reform might impact child health outcomes.\footnote{See for instance \cite{patel2004} or \cite{frech2011maternal}.} There is correlational evidence indicating that more maternal employment is related to lower levels of mental well-being as well as self-rated overall health, and a higher frequency of depressive symptoms and problems with parenting stress \citep{chatterji2005does,Chatterji2013}. In addition to the correlational evidence, there exists a large body of quasi-experimental literature. \cite{beuchert2016} exploit a reform of the parental leave scheme in Denmark and find effects on maternal and siblings health outcomes.\footnote{The Danish reform increased post-birth maternity leave by, on average, 32 days. The length of paid leave was increased from 24 weeks (14 weeks maternity and 10 weeks of joint leave) to 46 weeks (14 ML + 32 joint).} Additionally they detect larger gains for low-resource families. \cite{butikofer2018impact} exploit the 1977 maternity leave reform in Norway in order to demonstrate how the legislation change enhanced a battery of mid- and long-term maternal health outcomes, such as BMI, blood pressure, pain and mental health and it leads to more favorable health behavior (physical exercise and smoking abstinence).\footnote{The pre-reform scheme of 12 weeks of unpaid leave was changed to 4 months of paid and 12 months of unpaid leave.} \cite{albagli2018} show in the Chilean context that mothers who gave birth under a more generous leave regime have lower stress indices as compared to mothers who were shorter on leave.\footnote{The length of paid leave was raised from 12 to 24 weeks.} \newline 
Summing up, if there are effects of extending maternity leave on parental health outcomes, they are positive, which in turn would enhance parents' ability to nurture.\footnote{ For that one has to assume that more time is devoted to child rearing due to less medical complications.}\newline 
%  \cite{avendano2015long}



% Income & other outcomes (fertility)
Third, changes in household income are potential pathways for how the reform could affect child health outcomes. A considerable amount of research has been concerned with the positive association of family income with child cognitive achievement and health outcomes. Some studies have exploited expansions in the earned income tax credit (EITC) in order to investigate effects of household income on child outcomes. \cite{dahl2012impact} find that an increase in family income goes hand in hand with a boost in math and reading test scores, with the highest gains made among children from a low SES background. \cite{hoynes2015income} identify that an increase in household income concurs with a reduced likelihood of low birth weight, which is induced by more prenatal care and less adverse health behavior such as smoking. Recently, the literature was focusing on the associations associations between SES and functional brain development \citep{tomalski2013}. In order to evaluate the causal effect of economic resources in early childhood on cognitive, socio-emotional and brain development, an RCT was established and is still running until 2022.\footnote{The experiment with the title "Household Income and Child Development in the First Three Years of Life" is led by Greg Duncan.} The effect of parental income must not be underestimated as \cite{case2002economic} mention that the impact of parental income on children's health status is partly responsible for the intergenerational transmission mechanisms of socioeconomic status, due to the fact that adverse health effects of lower income are accumulated over children's life. \newline
Another aspect is that women's share of household income may decrease when they stop working to take care of their children. This is problematic, for the reason that expenditures for child goods are often correlated with women's relative income share. The basis for this intra-household conflict may be rooted in asymmetric preferences for household goods across men and women \citep{anderson2002economics}.\footnote{The problem with intra-household allocation of resources and child well-being is not only found in developing countries, but was also a rationale for changing child benefit structures in the UK in the late 1970s \citep{lundberg1996bargaining}. Prior to the reform the benefit was basically a tax reduction for fathers, which was replaced by a cash payment to mothers in the hope that "kids do better".}\newline


As the reform 'improved' the children's environment in the context of the previously raised points, we expect a-priori an enhancement of child health status.




%  income (depreciation of human capital, change selection of mothers into work, change in labor market attachment see SL2014)
% transitional changes might not matter; 


% hardly the case for families at the bottom of the SES distribution $\rightarrow$	 
% experiment led by Greg Duncan: "RCT to evaluate the role of economic resources in early development. Is there a causal effect of unconditional cash transfers on cognitive, socio-emotional and brain development of infants in low-income US families. Brain circuitry may be sensitive to the effects of early experience even before early behavioral differences can be detected. In order to understand the impacts of added income on children's brain functioning at age 3, we will assess, during a lab visit, treatment/control group differences in measures of brain activity" (official title: Household Income and Child Development in the First Three Years of Life)

% https://onlinelibrary-wiley-com.emedien.ub.uni-muenchen.de/doi/epdf/10.1111/desc.12079 (Socioeconomic status and functional braindevelopment )



%other outcomes (fertility) 

% \begin{itemize}
% 	\item attachment theory/Fetal origin hypothesis/neurobiology literature (talk to Prof. Sulz)

% 	\item LR labor market effect: pos effect on probability of being employed one year after birth; \cite{albagli2018}
% \end{itemize} 


%\cite{grossman1972healthcapital} individuals come with initial stock of capital, that depreciates over age, which can be increased by investment Heckmann 2007: parental investment not in the Grossman model. 
%(here: investment at time after birth by mother - who might supply better quality of care, depending on the counterfactual - increase stock of health in critical period) 




%--------------------------------------------------------------------
% IDENTIFICATION
%--------------------------------------------------------------------
\bigskip
\section{Empirical strategy}\label{sec:empirical_strategy}
\subsection{Design}\label{sec:empirical_strategy_1design}
In order to estimate the causal effect of the length of maternity leave, we exploit the 1979 reform's eligibility rule, which is contingent on childrens' birth date (see section \ref{sec:background}). Children born on/after the specified birth cutoff date (May 01, 1979) fall under the new regime, in which their mothers are eligible for six months of maternity leave after childbirth. Mothers of children born before the threshold are entitled to two months. Assignment to one of the two schemes is a deterministic function of the birth date of the child. A regression discontinuity design (RDD) might constitute a first potential identification strategy, in which one compares health outcomes of children which are quite similar with the only notable exception that their mothers were entitled to different lengths of maternity leave.\newline 
We augment this idea of a local identification by combining the RDD with a difference-in-differences (DD) approach. A large body of literature suggests that there is a strong relationship between season of birth, health and other socioeconomic outcomes. The seasonality may come about due to reasons that are associated with either pre- or postnatal factors. First, the seasonality might arise due to selective conception, i.e. the socioeconomic composition of mothers varies over time \citep{buckles2013season}. Second, \cite{currie2013within} argue that season of birth effects might arise due to seasonal patterns of in-utero disease prevalence and nutrition.\footnote{In particular, they find seasonal influenza as a potential mechanism between month of birth and later outcomes.} Last, the seasonality aspect may also be the result of social postnatal factors such as age at school-entry \citep{black2011too}. \newline Not accounting for these season-of-birth effects, the estimated health differentials of children born before and after the reform date might be biased. The bias emanates from the fact that the estimated effect might be partly driven by the difference of heath outcomes from the seasonality component and not so much by the expansion in maternity leave coverage itself. Yet, by matching up the difference in health outcomes of children born within a distance to the birth cutoff date in which the legislation change took place (henceforth referred to as the treatment cohort) with differences in outcomes of children born around the same threshold one year prior the reform (control group), we can eliminate the seasonality component while preserving the local identification aspect. The implicit identifying assumption is that seasonality is time-invariant, in other words, one has to assume that the season-of-birth effects are the same for treatment and control group.\newline

Our main specification to estimating the effect of the length of maternity leave on children's health outcomes corresponds to the following equation: \footnote{The estimation procedure can also be found in similar contexts in \cite{RafaelLaliveandJosefZweimuller2009}, \cite{Dustmann2012}, \cite{Ekberg2013parental}, \cite{schonberg2014expansions}, \cite{Lalive2014}, \cite{Huebener2017}, \cite{danzer2017}, \cite{guertzgen2018}, and \cite{avdic2018modern}.}
\begin{align}
Y_{mt} = \gamma_0 + \gamma_1 Treat_{m} + \gamma_2 After_{m} + \gamma_3 (Treat_{m} \times After_{m}) + \psi_m + \rho_t + \varepsilon_{mt} \label{eq:DD_basline}
\end{align}
where $Y_{mt}$ is the number of diagnoses per thousand individuals of the cohort born in month $m$, at time period $t$. $Treat_{m}$ is a dichotomous variable equal to one for groups that are born shortly before or after the legislation took place (i.e. the treatment cohort).\footnote{In the widest specification this involves children, who are born between November 1978 and October 1979, implying a bandwidth of half a year around the cutoff.} As control group we choose the cohort born in the same months, but in the year prior to the reform as control group.\footnote{\cite{Dustmann2012} use in total three birth cohorts as control group, two cohorts before and one cohort after the treatment cohort: group 1 born 11/1976-10/1976; group 2 born 11/1977-10/1978; and group 3 born 11/1979-10/1980 compromise the control group. Our control cohort is identical to the one used by \cite{guertzgen2018}. \newline We choose the cohort prior to the treatment cohort as control group in our main specifications for two reasons. First, with more cohorts as control groups, it may be less likely that the identifying assumption (time-invariance of seasonality) is met. Second, taking a birth cohort in the year after the policy change as control group might invalidate the comparability between the treatment and control group as parents might have enough time to react to the reform and adjust fertility patterns. Nevertheless, in the robustness section we present results when adding more control cohorts.}


$After_{m}$ is a dummy variable that is equal to one if the individuals are born after the threshold month May (i.e. born in May-October in the widest specification, for both treatment and control cohort). $\psi_m$, $\rho_t$ are month-of-birth and survey year fixed effects, respectively. Initially, $Y_{mt}$ corresponds to outcomes observed over the pooled period 1995-2014. We successively break up the entire time frame in different age groups until we apply a life-course approach by running the regression for each year of life separately.\newline 
The parameter of interest is $\gamma_3$, which captures the effect of the policy change on health outcomes. As we do not have any information on the fact whether the individuals' mothers were on leave, the identified parameter is an intention-to-treat effect. \newline
 %The interaction term $Treat_{mr} * After_{mr}$ equals one for the group of interest (the children born between May and October 1979,i.e. the post-reform children in the treatment group).

%clustering
Standard errors are clustered on the birth month$\times$birth year level in order to account for likely correlation of the error $\varepsilon_{mt}$ over time for a given month of birth cohort.
%We use sandwiched standard error estimates
%, allowing errors to be correlated over time within a month-of-birth cohort, and across 
%Diagnosis rates are serially correlated, cluster on month-of-birth and state level.



\bigskip
\subsection{Potential threats \& validity of the identification}\label{sec:empirical_strategy_threats+validity}
Behavioral responses with respect to the forcing variable (the birth date of the child) would jeopardize the validity of the identification strategy. Usually, the occurrence of birth can be considered as a random event, given the date of conception.\footnote{Gestation length is a normally distributed random variable with mean of 40 weeks and 2 weeks standard deviation \citep{Ekberg2013parental}.} However, women could influence the time when their baby is born by strategic conception and postponing the time they give birth by delaying induced births \& cesarean sections.  

\bigskip\bigskip
\textbf{A. Threats to identification}
\begin{enumerate}
	\item \underline{Strategic conception}\\ As already mentioned in Section \ref{sec:background}, it was impossible to conceive a child as a reaction to the draft bill, because the draft was only proposed four months before the threshold birth date.\footnote{The earliest possible birth date as a reaction to the draft bill would be October 1979. In the widest specification for the RD design, the sample contains children born between November 1978 and October 1979.} However, it could be possible that the topic aroused public awareness due to media coverage before the draft bill was initiated. For that reason \cite{Dustmann2012} conduct a literature search for any articles about the reform in the two leading national newspapers.\footnote{For their newspaper search \cite{Dustmann2012} use the daily national papers \emph{Frankfurter Allgemeine} and \emph{Süddeutsche Zeitung}.} They find that the earliest articles were only published two months prior to when the reform was put into practice.
	
	\item \underline{Induced births and cesarean sections}\\ Mothers with due dates close to the cutoff date on May 1, 1979 could time the birth date of their child by postponing induced births and cesarean sections.\footnote{Bringing the birth date forward around the threshold does not constitute a threat to the identification strategy, because this behavioral response would destroy their children's eligibility. Hence, it can be safely assumed that women would like to postpone the date of birth, if any.} \cite{gans2009born} find that during the introduction of a \$3.000 "Baby Bonus" in Australia, parents postponed the birth by as much as a week in order to be eligible for the benefits. Consequently the data shows that sharply before the cutoff birth date there is a dip in the birth rate and a huge increase on the first day after the threshold.
\end{enumerate}
It is possible that there are similar distortionary "introduction effects" at work in the maternity leave expansion in Germany in 1979. In other words, even though the announcement period does not allow for a strategic conception, parents may be incentivized to delay the timing of their child due to the reform.\footnote{Yet, we believe that this effect is rather unlikely as the C-section/induced birth rate in the FRG around 1980 was significantly lower than its counterpart in Australia in 2004. This gives less scope for shifting births around the threshold. } %{\color{red} Quelle?}


\bigskip\bigskip
\textbf{B. Validity of the empirical approach}\newline 
In order to check the existence of potential behavioral responses, we investigate the fertility distribution around the cutoff. We follow the example of \cite{gans2009born} very closely to show that there is no tampering with the fertility distribution. In our analysis we use daily data on the number of births from the federal states of Baden-Württemberg and North Rhine-Westphalia over the window 1977-1990.\footnote{Appendix Figure \ref{fig: fertility_hist} shows the monthly number of births per day for both treatment and control cohort. Although the figure utilizes all childbirths from the former region of the Federal Republic of Germany, it is difficult to see whether parents moved births from April to May. For that reason, we use daily births from two federal states.} Both states together act as a good proxy for the entire area of the former FRG as they account for almost 36\% of all births in 1979.\footnote{In 1979, 8.50\%/27.4\% of all births occurred in Baden-Württemberg/NRW.} In our analysis, we limit the sample to a time window of one month before and after the cutoff date (just April and May). Panel A of Figure \ref{fig: fertilitydistr} shows the (unadjusted) daily number of births for April and May 1979. Over the time window, one can observe a strong weekly pattern with more births born on weekdays and less births on weekends. On May 01, which was a Tuesday, one can see a drop in the number of births, which seems unexpected. If parents want to move the time of birth, one would expect that there is a rise in the number of births right after the cutoff date. However, May 01 is a national public holiday (Labor Day) and birth rates are generally lower on public holidays. \newline 

%Panel B
Panel B of Figure \ref{fig: fertilitydistr} shows the time series when taking out any variation in the timing of births stemming from day of week, public holiday, year, and day of year. To do so, we estimate the following equation on all years, \emph{except} April and May 1979:
\begin{align}
\text{Births}_i = I^{\text{Year}}_i\times I^{\text{Day of Week}}_i + I^{\text{Day of Year}}_i + I^{\text{Public Holiday}}_i + \varepsilon_i \label{eq: validity_fig}
\end{align}
where $\text{Births}_i$ is the number of children born on day $i$. This is a linear combination of dummies for the year interacted with the day of the week (allowing a different effect for Tuesdays in 1979), for the day of the year (permits a different effect for May 01), and a dummy for public national holidays.\footnote{For public holidays we use Good Friday, Holy Saturday, Easter Sunday, Easter Monday, Labor Day, Ascension Day, Whit Sunday, Whit Monday, and Corpus Christi.} Parameters are left out for better readability. \newline We use the calibrated model to predict the expected number of births for April and May 1979 $(\widehat{\text{Births}})$. In panel B, we plot the residuals $(\text{Births}-\widehat{\text{Births}})$. In contrast to \cite{gans2009born}, we do not detect systematically less births than expected in the month prior to the policy reform, nor more births than expected right after the policy change came into effect. They observe a trough piling up right before and a peak just after the cut-off. Instead, we notice a series that exhibits a mean reverting tendency across all days. Thus, graphically there is no evidence that parents moved the births to after the cutoff in order to become eligible for th extended leave. \newline 

To formalize this, we estimate the following models: 
\begin{align}
\text{Births}_i &= I^{\text{Reform}}_i + I^{\text{Year}}_i\times I^{\text{Day of Week}}_i + I^{\text{Day of Year}}_i + I^{\text{Public Holiday}}_i + \varepsilon_i  \label{eq: validity_tab_birth} \\
ln(\text{Births}_i) &= I^{\text{Reform}}_i + I^{\text{Year}}_i\times I^{\text{Day of Week}}_i + I^{\text{Day of Year}}_i + I^{\text{Public Holiday}}_i + \varepsilon_i \label{eq: validity_tab_lnbirth}
\end{align}
with the new dummy $I^{\text{Reform}}_i$, which equals one for days after the cutoff in 1979. Table \ref{tab: validity_birth_rate} presents the estimates of the expansion in maternity leave on fertility for different estimation windows. Irrespective of whether we use the number of births or the log of the number of births, there is no evidence that suggests that parents postponed births to after the threshold. On the contrary, the point estimates are throughout negative and become significantly different from zero with increasing estimation window. This indicates rather a reduction in fertility after the policy came into effect, if any. Furthermore, the magnitude of the estimates are constant across estimation windows, which would seem counterintuitive if parents shifted the births from the week prior to the reform to the week after the reform. If such behavior were present, one would expect point estimates to decrease in absolute value as the estimation window is enlarged. \newline 

%wrap up
By investigating the fertility distribution around the threshold we do not find evidence that parents strategically delayed births in the reform year. For this reason, the 1979 maternity leave reform can be considered as a valid natural experiment. Nonetheless, to scrutinize the possibility that the identification strategy is not jeopardized by selective fertility changes, we analyze the robustness of the results by applying a Donut specification, in which we exclude those children from the analysis that were born in the month before and after the policy was implemented (i.e. exclude the children, who were born either in April or May).




% Was noch reinkommen könnte:
% 1) Parental covariate balance
	% and the balance of (parental) predetermined characteristics\footnote{If there are behavioral responses due to the reform, one would expect that there is sorting around the threshold. In other words, it may be that the fertility pattern depends on parental family background characteristics.}
	%\item MZ (problem of selected sample, but large (enough?) number of obs) -> balancing table of parental predetermined covariates
	%\item potentially for later: SOEP, Zensus2011 (is there a question about parental background?)
	
% 2) Migration 
	% \item general: if people migrate at random (across MOB) it's not a problem, effect is diluted and we estimate a lower bound of the true effect
	%\item David Neumarks comment: differential migration patterns across month of birth would pose a threat









%--------------------------------------------------------------------
% DATA & VARIABLES
%--------------------------------------------------------------------
\bigskip
\section{Data}\label{sec:data} 
% Description data
For this research we use hospital register data spanning the period from 1995 to 2014, provided by the Research Data Centers of the Federal Statistical Office and the statistical offices of the Länder.\footnote{Due to data confidentiality regulations data access was provided via on-site and remote access at the research data center.} The register contains information on the universe of German in-patient cases (in 2014: 19.6 million observations). It covers \textit{all} patients that were discharged from \textit{any} hospital or medical prevention/rehabilitation facility in Germany in the reporting year.\footnote{The data does not cover police hospitals, hospitals of the penal system, and military hospitals. Military hospitals are included to the extent in which they offer services to civilians.\newline Medical prevention/rehabilitation centers are included since 2003 if they have more than 100 beds. This criterion might potentially exclude establishments with a specialized range of treatments. Among the ten most frequent diagnoses made in prevention/rehabilitation facilities, one can find diseases of the musculoskeletal system, of the circulatory system, and mental \& behavioral disorders.} The data includes the patient's main diagnosis, socio-demographic characteristics (year and month of birth, gender, and postal code of the place of residence), the length of stay, whether the patient died or underwent a surgery, and in which medicating specialist department the patient stayed the longest.\newline 
%https://www.destatis.de/EN/FactsFigures/SocietyState/Health/PreventionRehabilitationFacilities/PreventionRehabilitationFacilities.html

%ICD Classification
The main diagnosis indicates the major reason for the patient's in-patient hospitalization. It is coded according to the guidelines of the "International Statistical Classification of Diseases and Related Health Problems" (ICD), which is maintained by the World Health Organization (WHO).\footnote{Please note that the data exploits a German modification that is issued by the German Institute of Medical Documentation (DIMDI) - a subordinate authority of the Federal Ministry of Health.} Up to and including 1999, the coding follows the ICD-9 classification, since 2000 the ICD-10 system is in place \footnote{The numbers of the ICD classification refer to the revision, which is in place at the year of reporting. The ICD-9 classification is a 3 digit numeric code, whereas the ICD-10 is a 3 digit alphanumeric code.} Table \ref{tab:outcomes_coding_main_chapters} gives an overview of the transition between the two systems. The table lists the mapping provided by the European shortlist and some extensions (for subcategories) that were made by us.\footnote{The European shortlist covers the main chapters and important single diagnoses \citep[p. 76]{statistisches2012diagnosedaten}.} Furthermore, the table lists all the diagnosis chapters that constitute our hospitalization variable. We exclude most notably diagnoses related to pregnancy, childbirth and the puerperium, among others that occur very infrequently. In our sample, consequences of external causes are the most frequent diagnosis type, followed closely by mental and behavioral disorders, and diseases of the digestive system. This pattern holds also in the cross-section:  Figure \ref{fig: top5diagnosis_in_2014_across_agegroups} gives an overview of the five most common diseases and health problems that were diagnosed for individuals aged between 0 and 35 years in 2014. For the age group that we observe in our sample, i.e. people between 15 and 35 years, mental and behavioral disorders are the most common diagnoses (357 thousand diagnoses), followed by injuries (310 thousand diagnoses), and diseases of the digestive system (260 thousand diagnoses). Thus, the same three health problems constitute the top 3 diagnoses. \newline


% Descriptives hospital admission
Figure \ref{fig: descriptive_hospital_admission} illustrates trends in hospital admissions for the treatment and control cohort (only for the cohorts born before the threshold) from 1996 to 2014. In panel A, we observe a S-shaped line for the number of admissions. The rate of hospitalization is increasing until age 19 (35,000 cases per year), then decreasing up to the age of 26 after which the number of admissions are growing again (in 2014, there are 40,500 admissions). In panel B, one can see that the share of women is decreasing from 55\% in 1995 to below 48\% in 2014. Panel C shows a reduction in the average length of stay from 7.7 days to 6 days from the beginning to the end of the observation period. Panel D shows a hump-shaped evolution of surgeries that are related with hospitalization. The share of surgeries for in-patient cases is increasing up to the age of 22 (more than every other in-patient undergoes surgery), then declining until the age of 26. After that, the share of in-patients with a surgery stays constant at around 35\%.\newline %Panel E displays the fraction of people that died during hospitalization, which is around 2 deaths per 1,000 hospitalizations over the entire time frame. 



% Outcomes and aggregation
We aggregate the number of hospitalizations (or for each diagnosis chapter) by birth month (per cohort) and survey year. For our baseline specification, we use all hospitalizations of individuals who reside in the area of the former Federal Republic of Germany.\footnote{Due to the fact that Berlin cannot be assigned to either FRG or GRD unambiguously, it is dropped from the analysis.} Thus, for our DiD baseline specification, we have a pseudo-panel with $2\times12\times20=480$ (cohorts$\times$month-of-birth$\times$years) observations. Yet, in order to avoid any confounding that might be triggered by differential maturity effects between treatment and control group, we want to compare the two birth cohorts at the same age. To achieve this, we shift a control observation from period $t$ to period $t+1$, which decreases the number of effective observations to $456$.\footnote{The shifting only reduces the number of observations at the beginning and the end of the observed time frame. For instance, in 1995, when the treatment group is aged 16, there is no control group with which we can compare them. So we omit these 12 months of the treatment group from the sample. The control group that is already 17 years old is shifted to 1996. Analogously, we drop the control group in 2014.} \newline % popf - seit 2003 macht die 58,752 observationen; popmz sogar erst seit 2005 48,960
In the next step, we define the outcomes as the number of diagnoses per 1,000 individuals. We refrain from using the absolute numbers as the month-of-birth cohorts vary substantially in size (due to e.g. number of days per month or seasonality of births over one year). In our baseline specification, we use the number of births in the denominator.\footnote{In the robustness section, we present results with the approximated number of current inhabitants on different regional levels. The advantage of using the original number of births is twofold. First, we can trace out differentials for a longer period, because the population data is only available from 2003 onwards. Second, we avoid to induce measurement error in our dependent variable due to the fact that there is only information on the number of individuals aged $x$ years. In order to obtain an approximated number of persons per birth-month, we assign them using either weights from the German Micro Census or the original fertility distribution.} \newline
 

% Vorteile data
The data set has three main advantages. First, compared to survey data, the hospital registry data covers the universe of German in-patient cases and is consequently not prone to sampling errors or problems associated with attrition. Additionally, the large number of observations come in handy when identifying local effects. Second, due to the longitudinal character of the data set we are able to trace out the trajectory of children's health differentials over 20 years of their adulthood. Third, there is almost surely no measurement error. On the one hand, this might affect the birth date, which is accurate to the month in our context. This is particularly relevant as the time of birth is the deciding factor whether an individual's mother is eligible for the more generous leave scheme or not. \cite{Dustmann2012} do not have information about the exact birth date and have to impute the month-of-birth.\footnote{\cite{Dustmann2012} estimate that in roughly 70\% of the cases they obtain the correct birth month.} On the other hand, the lack of measurement error holds for the outcome variables, as well. As the diagnosis code matters for remuneration, the ICD code is of a very high quality. Moreover, in comparison with self-reported survey data, administrative data does not suffer from issues related to social desirability bias \citep{marcus2015}.\newline

%Nachteile
Nevertheless, the data set has some drawbacks, too. First, the source of data limits us to analyze rather severe health events. A priori we expect to find an effect only for the types of health outcomes, which are particularly salient in the hospital registry data. Yet, some health conditions, which are diagnosed elsewhere (e.g. practitioner), might be even more affected than the diagnoses that we observe.\footnote{One could supplement health insurance data for more common diagnosis types that are more likely made by practitioners, but are probably affected by the extension of maternity leave (e.g. such as the incidence of metabolic syndrome).} This implies that our estimates represent a lower-bound of the overall effect of the maternity leave on health outcomes. \newline%In a way, we just scratch the tip of the iceberg  
%{\color{red} Max: Hausarztbesuche}
 Second, although the registry data is rich in both, the numbers of observations and the quality of its entries, it contains only few socio-economic variables. This implies that we can only demonstrate average effects of the 1979 maternity leave reform, but cannot present any results by subgroups. Last, connected to the previous point, we do not have information about the place of birth, implying that the region $r$ in which we observe a patient in year $t$ does not necessarily have to coincide with the patient's place of birth. Ideally, we would like to exclude individuals from the analysis whose mothers were not affected by the reform, such as foreign-born children and children born in the German Democratic Republic (GDR). Yet, as we are unable to do so, the results get diluted such that our ITT estimates are pushed towards zero. The implicit assumption is that migration to the area of the former FRG occurred at random (with respect to the month-of-birth). In order to refute this concern to some extent, our baseline specification is aggregated to the level of the former FRG and GRD and we rely on more disaggregated data only for the investigation of effect heterogeneity and robustness tests. In this way, we hope to limit the impact of within region migration of the area of the former FRG or GDR. \footnote{Undoubtedly, if unaffected individuals migrated differently by the month-of-birth, this would cause some trouble for identification.}\newline









%--------------------------------------------------------------------
% RESULTS
%--------------------------------------------------------------------
\section{Results}\label{sec:results}

\subsection[The effect on hospital admission]{The effect of the 1979 maternity leave reform on hospitalization}


% Hospital - Total
Table \ref{tab: DD_hopsital2_total} contains difference-in-difference estimates of the impact of the expansion in maternity leave on hospital admission, based on equation \ref{eq:DD_basline}. The outcome is defined as the annual number of hospital admissions per 1,000 individuals per month of birth. In panel A, we present our ITT results based on the pooled data covering the years from 1995 to 2014, i.e. from age 17 to 35. The maternity leave reform significantly reduced the fraction of annual inpatient treatments in hospitals by 2.1 cases per 1,000 individuals (column 1, baseline specification). This corresponds to a reduction by 1.7\% from the pre-treatment mean.  The point estimate is very robust to using narrower estimation windows (columns 2 - 4) and to excluding children born close to the cut-off (children born in April and May).\footnote{The results for the narrower bandwidths are less precisely estimated due to the smaller sample sizes; however, the point estimates do not differ significantly across the specifications.} 
\newline     
%alt: With our baseline specification that uses a bandwidth of half a year (column 4), we find that there are on average 2.076 fewer hospitalizations per 1,000 individuals for children whose mothers were eligible for prolonged maternity leave. In the next step, we vary the width of the estimation window from three months to half a year and run a "Donut" specification in which we omit children who are born in the months directly around the threshold (i.e. exclude children born in April and May). The point estimate is fairly robust to different estimation windows.  

To investigate whether these effects are stable over the life-cycle or vary by age, we refine our analysis and consider how the effects of the maternity leave reform evolve with age.  \natalia{If parents are aware of certain health problems of their offspring, it is well possible that they engage in additional health treatments to compensate for this disadvantage and any negative health differential might dissipate over time.} Panel B contains our results for the reform's impact on the hospitalization ratio measured at different ages, 17-21, 22-26, 27-31 and 32-35. Interestingly, the reform effect on hospitalization rates appears to increase over time. Even though all estimates  carry a negative sign, the point estimates become larger and significantly different from zero only at older ages (-2.7 and -3.9 hospital admissions for the age group 27-31 and 32-35, respectively). The average impact on hospitalizations at younger ages (17-26 years of age) is much smaller and generally not statistically different from zero.\newline

%ADD LATER: In principle one could expect that the effect diminishes with age - in particular in a country like Germany with a strong health system. CITE ALMOND/CURRIE Development under age 5!

%On average, we find not significant effects of the extended maternity leave on hospitalizations at younger ages (17-26 year-olds). However, with increasing age as the children grow older, the differentials are opening up for the cohorts aged 27-31 and 32-35. The magnitude of the effects are larger and statistically significant. 
%level of analysis is on MOBxYOBXyear level
 
% Hospital -  Distinction Women and Men (tabelle)
We next explore whether these general findings and the age pattern hold similarly for men and women. Tables \ref{tab: DD_hopsital2_female} and \ref{tab: DD_hopsital2_male} show the effect of the 1979 maternity leave reform on hospital admissions for women and men, respectively. In general, the point estimates for men are mostly larger than the corresponding ones for women.\footnote{Yet, the baseline means are not significantly different from each other.} Furthermore, the effects for women are less robust with respect to the choice of bandwidth than the effects for men, which mirror the overall effects from Table \ref{tab: DD_hopsital2_total} very closely. On the one hand, we find for men in the pooled sample an average reduction of 2.410 fewer hospitalizations for MOB cohorts that were born under the more generous regime, irrespective of the estimation window. On the other hand, the effects are growing in size the longer we follow individuals across their life-cycle. \newline

%  Hospital - life-course graph
Figure \ref{fig: lc_hospital2_frg_DD} illustrates the trajectories of the hospitalization differentials between treatment and control group on a yearly basis (ranging from age 17 up to 35).\footnote{We still control for any maturity effects and compare outcomes of treatment and control cohorts measured at the same age.} We estimate the specification as in equation \ref{eq:DD_basline} year by year and plot the DD estimates (along with 90\%/95\% confidence intervals) across time.\footnote{Survey year fixed effects $\rho_t$ have to be omitted from the specification due to collinearity.} Panel a shows effect on hospital admissions over the life-course for the pooled sample of men and women. Overall, the figure reveals that the average reform effect is close to zero at younger ages, but grows in magnitude and becomes significantly different from zero at older ages. This negative trend (which reflects a growing positive health effect of the reform with increasing age) is even more pronounced for men (panel c). For men, the positive health impact of the reform seems to  materialize more and more with increasing age. In contrast, the figure for women (panel b) does not reveal a comparably similar and clear structure: While the figure reveals several significant reductions in hospitalization rates at older ages, there is no significant reform effect at age 35.\newline
%The gray bold line in the background, specification of 6M BW

\subsection{Robustness}\label{sec: robustness}

We performed several sensitivity and placebo test to assess the robustness of our findings. The result of these additional tests are reported in Table \ref{tab: robustness_hospital}. 

\textit{1. Alternative Specifications}\newline
First, we test whether our results are sensitive to the respective denominator used in calculating the fraction of diagnoses by 1000 individuals born in a specific month and year. Ideally, we would use as denominator the actual number of individuals living in Germany in a respective year (1995-2014) and born in a particular year and month. This data is not available. In the main analysis, we thus employ as denominator the number of births in a particular year and month (based on federal vital statistics). However, over time, this denominator might reflect the actual population size born in a particular month and year only imperfectly (due to migration or death). Therefore, we construct an alternative denominator using actual annual population figures by year of birth, weighted by the relative frequency of births across birth months in the year of birth. \footnote{The number of observations is smaller since the statistics on annual population size by birth month and year are only available since 2003.}
Column 2 in Table \ref{tab: robustness_hospital} shows that our result is robust to using this alternative denominator.\newline
Second, we exploit information on regional location of hospitals to dis-aggregate our data spatially. We create a regional panel for the years 2003 - 2014 for the 204 labor market regions in West Germany containing information on number of diagnoses per 1000 individuals in a respective MOB cohort. Actual population size by MOB cohort on the LMR level is approximated by data on current population by year of birth - i.e. the number of inhabitants in year $t$ who were born in 1979 in region $r$ - weighted by the relative frequency of birth across months (as given by the fertility distribution in the national vital statistics). An advantage of this regional-level analysis is that it allows including region fixed effects. We can thus control for potentially confounding effects on the regional level. The corresponding regression specification is as follows: 
\bigskip
%\textbf{Analysis carried out on the LMR level}
\begin{align}
Y_{mrt} =\ \gamma_0 + \gamma_1 Treat_{m} + \gamma_2 After_{m} + \gamma_3 (Treat_{m} \times After_{m}) + \psi_m + \phi_r + \rho_t + \varepsilon_{mrt} \label{eq:DD_LMR}
\end{align}
%additionally include region fxied effects $\phi_r$
%new: on regional level $r$
%in total there are 245 LMR, 204 in the area of the former FRG, 41 in the area of the former GDR. 
The estimated coefficients based on the regional panel data are very similar to the main results (column 3). \newline

\textit{2. Alternative Estimations}\newline
Next, we performed several alternative estimations using additional or alternative control groups. In Table \ref{tab: robustness_hospital}, column (4), we report estimates based on a Difference-in-difference-in-differences approach. The additional control group is comprised of individuals living on the territory of the former GDR (East Germany). The underlying idea is that children (and their mothers) born in East Germany in 1979 were not affected by the West German 1979 maternity leave reform and should thus resemble a valid comparison group (conditional the common trend assumption). Importantly, this specification allows us netting out any general time trend in the outcome variables that might affect the results. The corresponding Triple-difference-estimation specification becomes: \newline


%\textbf{Triple difference specification (DDD)}
\begin{align}
Y_{mt} =\ &\beta_0 + \beta_1 Treat_{m} + \beta_2 After_{m} + \beta_3 FRG_m \nonumber\\&+ \beta_4 (Treat_{m} \times After_{m}) + \beta_5 (Treat_m \times FRG_m) + \beta_6 (After_m \times FRG_m) \nonumber\\ &+ \beta_7 (Treat_m\times After_m\times FRG_m) + \psi_m + \rho_t + \varepsilon_{mt} \label{eq:DDD}
\end{align}

which now contains an additional dummy variable $FRG$ (West Germany) as well as interactions of this control group with the treatment cohort 1979 (Treat), children born after April (After) and with the interaction term Treat X After. Once again, the general pattern of our main results remains robust to this alternative estimation. While the estimates become less precise, the point estimates increase slightly (but not significantly).  This general pattern also holds true if only use the 1979 birth cohort born in East Germany (GDR) as control group in an alternative difference-in-difference specification (see column 5). \footnote{
% Robustness - alternative DD 
%\textbf{Alternative DD}
\begin{align}
Y_{mt} =\ &\delta_0 +  \delta_1 After_{m} + \delta_2 FRG_m + \delta_3 (After_m \times FRG_m) + \psi_m + \rho_t + \varepsilon_{mt} \label{eq:alt_DD}
\end{align}
}

As a further sensitivity test with respect to our chosen estimation specification and included control groups, we re-estimate our main regression equation including an additional control group, that is children born in West Germany in 1977. The advantage of including another pre-reform West German birth cohort is that might better help to account for systematic month-of-birth patterns in hospitalization rates (due to a larger sample size). \footnote{However, it also reduces the age span for which we can perform our analysis.}
Once again, the results and pattern of our original main results hold when adding this additional control cohort (column 6).\newline

Overall, the sensitivity tests demonstrate that our main results are robust to alternative specifications and estimations, indicating that the maternity leave reform significantly reduced hospitalization rates between the ages 16 to 34, and in particular so for men.\newline

\textit{3. Placebo Tests}\newline
We performed two placebo analyses to test the validity of our identifying assumptions: First, to test whether our estimated effects are caused by children eligible to the more generous maternity leave regime and not by an underlying general time trend affecting children born post-April, we ran a placebo analysis with two pre-reform birth cohorts 1978 and 1977, pretending that May 1, 1978 was the reform date. As a second placebo analysis we re-estimated our main DiD specification using the birth cohorts 1979 and 1978 and May 1, 1979 as reform cut-off date, but this time using the sample of East German individuals (who were not affected by the reform).  
Reassuringly, both placebo tests yield insignificant effects supporting the notion that our main findings are indeed caused by the West German maternity leave reform.\newline

\textit{4. Subgroup analysis: Rural and Urban Areas}\newline
Last, we used our regional panel on inpatient cases to assess whether reform had a different impact on children's health in rural and urban areas. Theoretically, there are three reasons for why we might estimate a differential impact of the maternity leave reform. First, (female) labor force participation rates are generally higher in urban areas. Hence, it is likely that maternity leave eligibility rates in urban areas were higher than in rural areas. If that is true, we would expect our ITT estimates to be larger in urban areas. Second, apart from general differences in labor force participation rates, mothers in urban and rural areas might differ in their return to work behavior. If urban pre-reform mothers were more likely to return to the labor market quickly (after 2 months of giving birth) than rural mothers, the reform impact for urban mothers would comprise a partly compensated reduction in their reduced labor supply in the first 6 months after childbirth; in contrast, if rural mothers would have stayed at home in the absence of the reform anyways, the reform implied a windfall profit. To be clear, this argumentation is very hypothetical, since we lack information on the return-to-work behavior for mothers in urban and rural regions. This hypothesis is grounded on the observation that the rural population tends to be more traditional and conservative. Third, differential reform effects in urban and rural regions might also result from differences in the available counterfactual mode of care (see Danzer et al.). 

\bigskip

%Table \ref{tab: robustness_hospital}
%ORDER OF COLUMNS:
%1) BL 
%2) current population 
%3) LMR 
%4) DDD
%5) alt. DD 
%6) ad CG 
%7) Placebo: temporal 
%8) Placebo spatial  
%9 + 10) rural/urban heterogeneity (non-homogeneous first stage)

%11) Do not shift age observations


\subsection[The effect on main diagnosis chapters]{Results on main diagnosis chapters}

% Main Diagnosis Chapters
What is driving the significant reductions in hospitalization rates? We exploit the detailed reporting on main diagnosis related to each hospitalization case in the data and assess the effect of extended maternity leave on specific diagnoses, grouped in 13 so-called main diagnosis chapters. The outcome variables now refer to the number of hospitalizations due to particular diagnoses per 1000 individuals (per MOB cohort). An overview of these main diagnosis chapters is given in Panel A of Table \ref{tab:outcomes_coding_main_chapters}. \newline
%To further examine the source for the reductions in hospitalizations, we look at the impact of the maternity leave reform on the reasons of hospitalizations. For that reason, we use the main diagnoses chapters as dependent variables.
%\footnote{Panel A of Table \ref{tab:outcomes_coding_main_chapters} gives an overview of the diagnosis chapters that generate the hospital admission variable.} 
Figure \ref{fig: DD_across_main chapters} plots ITT estimates of the effect of the policy change on these 13 main diagnoses chapters for all patients (panel a) and separately for women (panel b) and men (panel c). Each panel provides the respective point estimates (left hand side) and the overall incidence of inpatient cases in each diagnosis chapter from 1995-2004 (on the right hand side). The estimates are based on the pooled sample and a bandwidth of six months. The results in panel a (men and women) reveal that almost all point estimates are either negative or not significantly different from zero. This implies that we can reject the hypothesis that the expansion has detrimental effect on long-run child health. Furthermore, the largest effect of the maternity leave extension can be found with respect to diagnoses of mental and behavioral disorders, followed by consequences of external causes, diseases of the digestive system, and respiratory maladies. Assessing the results against the background of the overall frequency of these diagnoses chapters, the strong effect on mental and behavioral disorders clearly stands out as this diagnosis chapters has such a high prevalance in this age group.
When looking at effect heterogeneity across gender, wee see that the reduction in hospitalization for women is driven by fewer diagnoses of digestive diseases, whereas for males the reduction mostly stems from less diagnoses of mental and behavioral disorders. sout{Women as well as men exhibit the largest impact in that diagnosis chapter, which is the particularly salient one.} \newline
Breaking up the analysis by age group generates additional insights. Table \ref{tab: ITT_across_chapters_per_age_group_total} contains DD estimates of the impact of the expansion in maternity leave on the main diagnosis chapters (for all inpatients).\footnote{Appendix Tables \ref{tab: ITT_across_chapters_per_age_group_women} and \ref{tab: ITT_across_chapters_per_age_group_men} show the effects for women and men, respectively.} Column 1 reports the effect on the pooled sample and confirms the graphical evidence. Moreover, columns 2 to 5 list the impact of the expansion on the main diagnosis chapter per age bracket. For some chapters we see health differentials that are increasing with age, and this is particularly the case for mental and behavioral disorders. These are congruent with the overall effect on hospital admission. Other chapters display differentials that either fade out (disease of the digestive system and respiratory maladies), or are constant (e.g. illnesses affecting sense organs) across age groups.\footnote{Appendix Figure \ref{fig: appendix_lc_matrix_chapters} shows the respective life-course figures for each diagnosis chapter separately and summarizes the previous findings.}\newline 

% Transition to d5 - why focus on d5
Or results provide evidence that the expansion in maternity leave caused a reduction in hospitalization, in particular for men and towards the end of the observed time window. Furthermore, when considering the main diagnosis chapters, we see that the largest driver for the decline in hospital admissions is coming from mental and behavioral disorders.\footnote{This is also true when we consider the effect in percent of the baseline mean as an effective measure. The largest reduction is coming from diseases of the sense organs (7.03\%), but is followed by mental and behavioral disorders (3.17\%).} 

\bigskip
% EFFECT ON MENTAL AND BEHAVIORAL DISORDERS
\subsection{The effect on mental \& behavioral disorders}
Since the estimated effects of the maternity leave reform on mental and behavioral disorders stand out in terms of effect size and frequency of diagnoses, we present more refined evidence on this diagnosis chapter in the following. To recapitulate, first, in the overall sample, MBD accounts for one third of the reduction in hospitalizations, and in the last age bracket the importance of this diagnosis chapter for the drop in hospitalizations rises to almost 50\%. Second, as shown in Figure \ref{fig: top5diagnosis_in_2014_across_agegroups}, MBD are the most frequent diagnosis type for individuals aged between 0 and 35 years in 2014. Another important aspect is that MBD caused the highest costs per diagnosis among all diagnoses chapters for the age group 15-45 years in 2015.\footnote{The top three chapters with the highest costs per diagnosis (for the age group 15-45 in 2015) were: mental and behavioral disorders (9,800{\euro}), diseases of the digestive system (8,900{\euro}), and illnesses of the musculoskeletal system (4,900{\euro}).} \newline

Tables \ref{tab: DD_d5_total}, \ref{tab: DD_d5_female} and \ref{tab: DD_d5_male} show the ITT estimates for the effect of the expansion in maternity leave on the diagnosis of mental and behavioral disorders for all in-patients, women and men, respectively. While the estimates for women are close to zero and not statistically significant from zero, the magnitude of the effects for men are large. Males who are born after the expansion of maternity leave experience 1.192 fewer diagnoses per 1,000 individuals compared to individuals whose mothers were not eligible for the extended leave period. When considering the estimates per age bracket one can see the same pattern as with hospital admission - namely that the health differentials are opening up at the age of 26 and are further increasing towards the end of the observed time span. Figure \ref{fig: lc_d5_frg_DD} shows the corresponding life-course graphs, which confirm the impression obtained from the tables. We run the same set of sensitivity test as we did for hospital admission in Table \ref{tab: robustness_hospital} and find that the results are robust to different specifications and estimation procedures.\footnote{Robustness table for mental and behavioral disorders can be found in Appendix Table \ref{tab: robustness_d5}.} \newline

In the next step, we investigate what are the drivers for the reduction in mental and behavioral disorders in response to the 1979 maternity leave reform. To do so we utilize the number of subdiagnosis per 1,000 individuals as dependent outcome as shown in equation \ref{eq:DD_basline}. The subcategories are defined in Panel B of Table \ref{tab:outcomes_coding_main_chapters} and Figure \ref{fig: d5partition} shows the incidence distribution of the top 5 subcategories over time. Table \ref{tab: ITT_across_d5subcategories_per_age_group_total} and Figure \ref{fig: ITT_d5_subcategories} show that the reduction in mental and behavioral disorders is coming from fewer diagnoses of mental disorders due to psychoactive substances and incidences of Schizophrenia. \footnote{The table for women and men can be found in Table \ref{tab: ITT_across_d5subcategories_per_age_group_women} and \ref{tab: ITT_across_d5subcategories_per_age_group_men}.}







%RF
%pooled
%pooled \ref{fig: rf_d5_pooled}
%age group \ref{fig: rf_d5_agegroup}

% \subsection{Subsections of mental \& behavioral disorders}














%--------------------------------------------------------------------
% CONCLUSION
%-------------------------------------------------------------------
\section{Concluding remarks}\label{sec:conclusion}

Our results show that the 1979 maternity leave reform had positive health effects as indicated by fewer hospitalizations. Importantly, some of these effects materialize only later in life. Hence, our results imply that the assessment of maternity leave, which focuses only on short-run outcomes, might miss out certain aspects that develop only in the long-run.

 
% Why is it interestign to look at F-chapter
% Mental \& behavioral disorders are the diagnosis types with the longest average length of stay: 20.1 days (in comparisson to a general average of 7.6 days) [Data source: \cite[p. 5]{statistisches2012diagnosedaten} ]

% Back-of-the-envelope calculation

% Financial returns to society through health promotion and disease prevention



%--------------------------------------------------------------------
% BIBLIOGRAPHY
%--------------------------------------------------------------------
\newpage


\bibliographystyle{ecca_edited}%previous style-chicago
\bibliography{mlch_bibliography}

%\printbibliography


%--------------------------------------------------------------------
% FIGURES AND TABLES
%--------------------------------------------------------------------
%\newpage
%\section{Figures and tables}
\newpage
\TODO\section{Figures}
\vspace*{\fill}
{\Huge \begin{center}\textbf{FIGURES}\end{center}}
\vspace*{\fill}\clearpage
%--------------------------------------------

%WMWMWMWMWMWMWMWMWMWMWMWMWMWMWMWMWMWMWMWM
% BACKGROUND
%WMWMWMWMWMWMWMWMWMWMWMWMWMWMWMWMWMWMWMWM
% figure: reform 
\begin{figure}[H]\centering
	\caption{1979 reform in maternity leave legislation in the Federal Republic of Germany}\label{fig: MLreform}
	%\input{../../analysis/graphs/paper/tikz_reform_extended_pre_and_post_birth.tex}
	%\includegraphics[width=\linewidth]{../../analysis/graphs/paper/tikz_reform_extended_pre_and_post_birth.tikz}
	%\includegraphics[width=0.9\linewidth]{../../analysis/graphs/paper/tikz_reform_short_post_birth.tikz}
	\includegraphics[width=0.8\linewidth]{SOEP/Reform_shortened.pdf}
	\begin{minipage}{\linewidth}
		\scriptsize{\emph{Notes:} The figure describes the legislative change in the length of job protection and maternity leave, which took place in the Federal Republic of Germany in 1979. The reform increased post-birth maternity leave from eight weeks to six months, while keeping the initial structure of the period from six weeks before until eight weeks after childbirth unchanged (mother protection period).\newline \textit{Source: }The figure is based on information from \cite{Dustmann2012}, \cite{DIW2002}, \cite{schonberg2014expansions} as well as \cite{zmarzlik1999mutterschutzgesetz}.}
	\end{minipage}
\end{figure}
%--------------------------------------------
% figure: Bundesgesetzblatt und Familienministerin
\begin{figure}[H]\centering
	\caption{Introduction of the maternity leave law}\label{fig: bundesgesetzblatt_antjehuber}
	\begin{subfigure}[h]{0.48\linewidth}\centering%\caption{Federal law gazette}
		\includegraphics[width=\linewidth]{paper/bundesgesetzblatt_coverpage.png}
	\end{subfigure}
	\begin{subfigure}[h]{0.48\linewidth}\centering%\caption{Federal Minister of Family Affairs Antje Huber} 
	\includegraphics[width=\linewidth]{paper/antje_huber.jpg}
	\end{subfigure}
	\begin{minipage}{\linewidth}
		\scriptsize{\emph{Notes:} On the left, the figure shows the maternity leave law as published in the Federal law gazette. On the right, one can see the Federal Minister of Family affairs Antje Huber at the introduction of the maternity leave law, which was published by the Ministry on the occasion of 30 years of the Federal Ministry of Women.\newline \emph{Source:} Bundesanzeiger Verlag and Federal Ministry of Justice and Consumer Protection (\hyperlink{http://www.bgbl.de/xaver/bgbl/start.xav?startbk=Bundesanzeiger_BGBl&jumpTo=bgbl179s0797.pdf}{BMJV}) and Federal Ministry for Family Affairs, Senior Citizens, Women and Youth (\hyperlink{https://twitter.com/bmfsfj/status/745513281989677058}{BMFSFJ}).}
	\end{minipage}
\end{figure}

%--------------------------------------------


%WMWMWMWMWMWMWMWMWMWMWMWMWMWMWMWMWMWMWMWM
% VALIDITY
%WMWMWMWMWMWMWMWMWMWMWMWMWMWMWMWMWMWMWMWM

%--------------------------------------------
% Fertility distribuition
	\vspace*{\fill}
\begin{figure}[H]\centering
	\caption{Daily number of births around the expansion in maternity leave}\label{fig: fertilitydistr}
	\includegraphics[width=0.9\linewidth]{paper/fertility_raw_regression_adjusted.pdf}
	\scriptsize
	\begin{minipage}{0.9 \linewidth}
		\emph{Notes:} The figure plots the number of births around the cutoff date May 01 1979 for the expansion in maternity leave from two to six months after childbirth. Panel A shows the raw data, i.e. the actual number of births per day (unadjusted). Panel B, however, plots the difference between the raw and expected number of births when accounting for day of year, public holiday, and year$\times$day of week fixed effects. For the expected number of births we use data in the same time window (one month before and after the threshold) for the years 1977-1990, except for the year in which the reform took place. \newline\emph{Source:} Birth registry data from North Rhine-Westphalia and Baden-Württemberg. Taken together, both states account for almost 36\% of all births in the former Federal Republic of Germany in 1979.
		% {\color{red}how do I get the gray shading from the original figure in the bottom part?}
	\end{minipage}
\end{figure}
\vspace*{\fill}\clearpage

%WMWMWMWMWMWMWMWMWMWMWMWMWMWMWMWMWMWMWMWM
% DATA
%WMWMWMWMWMWMWMWMWMWMWMWMWMWMWMWMWMWMWMWM

% figure: matrix descriptive hospital admission
\newpage
\vspace*{\fill}
\begin{figure}[H]\centering
	\caption{Hospital admissions}\label{fig: descriptive_hospital_admission}
	\includegraphics[width=\linewidth]{paper/descriptive_admission_TCG.pdf}
		\begin{minipage}{\linewidth}
		\scriptsize{\emph{Notes:} The figure depicts the evolution of key variables for the treatment and control cohort (only pre-threshold months, i.e. for individuals born between November 1977 and April 1978 as well as November 1978 and April 1979) over the period from 1996 to 2014. The dark lines correspond to the treatment cohorts, whereas control units are marked by light dashed lines. Hospital admissions are defined as the sum of all diagnosis chapters listed in Panel A of Table \ref{tab:outcomes_coding_main_chapters}. This excludes diagnoses of the "O" chapter (pregnancy, childbirth, and the puerperium), among others that occur very infrequently. The x-axis shows in addition to the year the age of the treatment cohort in brackets.} 
	\end{minipage}
\end{figure}
\vspace*{\fill}\clearpage
%--------------------------------------------
%figure TOP 5 Diagnoses across age groups
\vspace*{\fill}
\begin{figure}[H]\centering
	\caption{Five main diagnoses of inpatients aged 0 to 35 in 2014}\label{fig: top5diagnosis_in_2014_across_agegroups}
	\includegraphics[width=0.8\linewidth]{paper/top5diagnoses_across_agegroups.pdf}
	\begin{minipage}{\linewidth}
	\scriptsize{\emph{Notes:} The figure shows the incidence distribution across different age brackets of the top five diagnoses for inpatients aged 0 to 35 in 2014. Diagnoses associated with pregnancy, childbirth, and the puerperium are not taken into account in this representation. The large remainder in the age bracket '0-5' consists mostly of conditions that originate in the perinatal period.}
	\end{minipage}
\end{figure}
\vspace*{\fill}\clearpage
%--------------------------------------------



%WMWMWMWMWMWMWMWMWMWMWMWMWMWMWMWMWMWMWMWM
% RESULTS
%WMWMWMWMWMWMWMWMWMWMWMWMWMWMWMWMWMWMWMWM
%--------------------------------------------
\newpage
%Life-course Hospital admission
\begin{landscape}
	\vspace*{\fill}
	\begin{figure}[H]\centering
		\caption{Life-course approach for \textbf{hospital admission}}\label{fig: lc_hospital2_frg_DD}
		\begin{subfigure}[h]{0.31\linewidth}\centering\caption{Total}
			\includegraphics[width=\linewidth]{paper/lc_hospital2_total_gdr.pdf}
		\end{subfigure}
		\begin{subfigure}[h]{0.31\linewidth}\centering\caption{Women}
			\includegraphics[width=\linewidth]{paper/lc_hospital2_female_gdr.pdf}
		\end{subfigure}
		\begin{subfigure}[h]{0.31\linewidth}\centering\caption{Men}
			\includegraphics[width=\linewidth]{paper/lc_hospital2_male_gdr.pdf}
		\end{subfigure}
		\scriptsize
		\begin{minipage}{\linewidth}
			\emph{Notes:} The figures plot DD estimates (along with 90\% and 95\% confidence intervals) for the impact of the reform on hospital admission over the life-course. The light gray line in the background represents the baseline mean of the pre-reform treated cohort. The outcomes are defined as the number of cases per 1,000 individuals (births). Panel a shows the results for all admissions, whereas panel b and c show the estimates for females and males respectively. The control group is comprised of children	that are born in the same months but one year before (i.e. children born between November 1977 and October 1978).
		\end{minipage}
	\end{figure}
	\vspace*{\fill}\clearpage
\end{landscape}
%--------------------------------------------
% HOPSITAL2 - RD plots
%
% Hospital - Reduced form pooled
\newgeometry{left=1cm,right=1cm,top=3cm,bottom=3cm} 
\begin{landscape}
	\vspace*{\fill}
	\begin{figure}
		[H]\centering
		\caption{RD plots for hospital admission (pooled)}\label{fig: rf_hospital2_pooled}
		\begin{subfigure}[h]{0.31\linewidth}\centering\caption{Total}
			\includegraphics[width=\linewidth]{paper/rd_hospital2_total_pooled.pdf}
		\end{subfigure}
		\begin{subfigure}[h]{0.31\linewidth}\centering\caption{Women}
			\includegraphics[width=\linewidth]{paper/rd_hospital2_female_pooled.pdf}
		\end{subfigure}
		\begin{subfigure}[h]{0.31\linewidth}\centering\caption{Men}
			\includegraphics[width=\linewidth]{paper/rd_hospital2_male_pooled.pdf}
		\end{subfigure}
		\scriptsize
		\begin{minipage}{0.95\linewidth}
			\emph{Notes:} The figure plots the average number of diagnoses per 1,000 individuals for month-of-birth cohorts born half a year around the cut-off date of the 1979 maternity leave expansion. The monthly averages are taken over the entire sample length from 1995 to 2014. The dashed lines represent linear fitted values along with 90\%/95\% confidence intervals. The solid vertical red line divides pre- and post-reform schemes (two vs. six months of leave).\newline
			\emph{Source:} Hospital registry data for individuals born between November 1978 and October 1979.
		\end{minipage}
	\end{figure}
	\vspace*{\fill}\clearpage
\end{landscape}
\restoregeometry  
%--------------------------------------------
% Hospital -Reduced form AGE groups 
% original 3 3 2 3
\newgeometry{left=1cm,right=1cm,top=3cm,bottom=3cm} 
\begin{landscape}
	\vspace*{\fill}
	\begin{figure}
		[H]\centering
		\caption{RD plots for hospital admission across age groups}\label{fig: rf_hospital2_agegroup}
		\includegraphics[width=0.85\linewidth]{paper/rd_r_fert_hospital2_overview_agegroups_CIfits.pdf}
		\scriptsize
		\begin{minipage}{0.9\linewidth}
			\emph{Notes:} The figure plots the number of diagnoses per 1,000 individuals for month-of-birth cohorts born half a year around the cut-off date of the 1979 maternity leave expansion across gender and different age groups. The first column reports the ratios for all patients, and the second and third column do so for women and men, respectively. The rows show the ratios across different age groups. The dashed lines represent linear fitted values along with 90\%/95\% confidence intervals. The solid vertical red line divides pre- and post-reform schemes (two vs. six months of leave).\newline
			\emph{Source:} Hospital registry data for individuals born between November 1978 and October 1979.
		\end{minipage}
	\end{figure}
	\vspace*{\fill}\clearpage
\end{landscape}
\restoregeometry

%--------------------------------------------
% Figure: effect sizes and frequency across chapters

\begin{landscape}
	\vspace*{\fill}
	\begin{figure}[H]\centering
		\caption{Intention-to-treat effects across \textbf{main diagnosis chapters}}\label{fig: DD_across_main chapters}
		\begin{subfigure}[h]{0.31\linewidth}\centering\caption{Total}
			\includegraphics[width=\linewidth]{paper/effect_chapters_frequency.pdf}
		\end{subfigure}
		\begin{subfigure}[h]{0.31\linewidth}\centering\caption{Women}
			\includegraphics[width=\linewidth]{paper/effect_chapters_frequency_f.pdf}
		\end{subfigure}
		\begin{subfigure}[h]{0.31\linewidth}\centering\caption{Men}
			\includegraphics[width=\linewidth]{paper/effect_chapters_frequency_m.pdf}
		\end{subfigure}
		\scriptsize
		\begin{minipage}{\linewidth}
			\emph{Notes:} The figures plots intention-to-treat estimates (along with 90\%/95\% confidence intervals) across the main diagnosis chapters. Furthermore, they indicate how often each chapter is diagnosed over the entire time frame (1995-2014). The outcomes are defined as the number of cases per 1,000 individuals (births). The point estimates are coming from a DiD regression as described in section \ref{sec:empirical_strategy}, with a bandwidth of six months, month-of-birth and year fixed effects, and clustered standard errors on the month-of-birth level. The control group is comprised of children	that are born in the same months but one year before the reform (i.e. children born between November 1977 and October 1978). \newline
			\emph{Legend:} Infectious and parasitic diseases (IPD), neoplasms (Neo), mental \& behavioral disorders (MBD), diseases of the nervous system (Ner), diseases of the sense organs (Sen), diseases of the circulatory system (Cir), diseases of the respiratory system (Res), diseases of the digestive system (Dig), diseases of the skin and subcutaneous tissue (SST), diseases of the musculoskeletal system (Mus), diseases of the genitourinary system (Gen), "symptoms, signs, and ill-defined conditions" (Sym), "injury, poisoning and certain other consequences of external causes" (Ext).
			
		\end{minipage}
	\end{figure}
	\vspace*{\fill}\clearpage
\end{landscape}
%--------------------------------------------

% D5 - LC Approach (Mental and behavioral disod)
\begin{landscape}
	\vspace*{\fill}
	\begin{figure}[H]\centering
		\caption{Life-course approach for \textbf{mental and behavioral disorders}}\label{fig: lc_d5_frg_DD}r	\begin{subfigure}[h]{0.31\linewidth}\centering\caption{Total}
			\includegraphics[width=\linewidth]{paper/lc_d5_total_gdr.pdf}
		\end{subfigure}
		\begin{subfigure}[h]{0.31\linewidth}\centering\caption{Women}
			\includegraphics[width=\linewidth]{paper/lc_d5_female_gdr.pdf}
		\end{subfigure}
		\quad
		\begin{subfigure}[h]{0.31\linewidth}\centering\caption{Men}
			\includegraphics[width=\linewidth]{paper/lc_d5_male_gdr.pdf}
		\end{subfigure}
		\scriptsize
		\begin{minipage}{\linewidth}
			\emph{Notes:} The figures plot DD estimates (along with 90\% and 95\% confidence intervals) for the impact of the reform on mental and behavioral disorders over the life-course. The light gray line in the background represents the baseline mean of the pre-reform treated cohort. The outcomes are defined as the number of cases per 1,000 individuals (births). Panel a shows the results for all admissions, whereas panel b and c show the estimates for females and males respectively. The control group is comprised of children	that are born in the same months but one year before (i.e. children born between November 1977 and October 1978).
		\end{minipage}
	\end{figure}
	\vspace*{\fill}\clearpage
\end{landscape}

%--------------------------------------------
% D5 RD plots
%
% d5 - RF pooled
\newgeometry{left=1cm,right=1cm,top=3cm,bottom=3cm} 
\begin{landscape}
	\vspace*{\fill}
	\begin{figure}
		[H]\centering
		\caption{RD plots for mental \& behavioral disorders (pooled)}\label{fig: rf_d5_pooled}
		\begin{subfigure}[h]{0.31\linewidth}\centering\caption{Total}
			\includegraphics[width=\linewidth]{paper/rd_d5_total_pooled.pdf}
		\end{subfigure}
		\begin{subfigure}[h]{0.31\linewidth}\centering\caption{Women}
			\includegraphics[width=\linewidth]{paper/rd_d5_female_pooled.pdf}
		\end{subfigure}
		\begin{subfigure}[h]{0.31\linewidth}\centering\caption{Men}
			\includegraphics[width=\linewidth]{paper/rd_d5_male_pooled.pdf}
		\end{subfigure}
		\scriptsize
		\begin{minipage}{0.95\linewidth}
			\emph{Notes:} The figure plots the average number of diagnoses per 1,000 individuals for month-of-birth cohorts born half a year around the cut-off date of the 1979 maternity leave expansion. The monthly averages are taken over the entire sample length from 1995 to 2014. The dashed lines represent linear fitted values along with 90\%/95\% confidence intervals. The solid vertical red line divides pre- and post-reform schemes (two vs. six months of leave).\newline
			\emph{Source:} Hospital registry data for individuals born between November 1978 and October 1979.
		\end{minipage}
	\end{figure}
	\vspace*{\fill}\clearpage
\end{landscape}
\restoregeometry 




%--------------------------------------------
% D5 - RF (age group)
\newgeometry{left=1cm,right=1cm,top=3cm,bottom=3cm} 
\begin{landscape}
	\vspace*{\fill}
	\begin{figure}
		[H]\centering
		\caption{RD plots for mental \& behavioral disorders across age groups}\label{fig: rf_d5_agegroup}
		\includegraphics[width=0.85\linewidth]{paper/rd_r_fert_d5_overview_agegroups_CIfits.pdf}
		\scriptsize
		\begin{minipage}{0.9\linewidth}
			\emph{Notes:} The figure plots the number of diagnoses per 1,000 individuals for month-of-birth cohorts born half a year around the cut-off date of the 1979 maternity leave expansion across gender and different age groups. The first column reports the ratios for all patients, and the second and third column do so for women and men, respectively. The rows show the ratios across different age groups. The dashed lines represent linear fitted values along with 90\%/95\% confidence intervals. The solid vertical red line divides pre- and post-reform schemes (two vs. six months of leave).\newline
			\emph{Source:} Hospital registry data for individuals born between November 1978 and October 1979.
		\end{minipage}
	\end{figure}
	\vspace*{\fill}\clearpage
\end{landscape}
\restoregeometry

  
%--------------------------------------------
% Graph D5 partition - Subcategories
\vspace*{\fill}
\begin{figure}[H]\centering
	\caption{The top five subcategories of mental \& behavioral diagnoses}\label{fig: d5partition}
	\includegraphics[width=0.8\linewidth]{../../analysis/graphs/paper/d5partition_lfstat.pdf}
	\scriptsize
	\begin{minipage}{0.9\linewidth}
	\emph{Notes:} This figure plots the yearly number of diagnoses for the treatment cohort (i.e. the individuals born between November 1978 and October 1979). The subcategories are ordered by their occurrence in 2014 (from the most to the least frequent diagnosis), which also coincides by chance with the ordering in the ICD-10 classification system. The five most frequent subcategories - as shown here - comprise more than 95\% of all mental and behavioral disorders. 
	\end{minipage}
\end{figure}
\vspace*{\fill}\clearpage%--------------------------------------------
% figure: iverview of d5 subcategories (effects + frequency)
\newgeometry{left=1cm,right=1cm,top=3cm,bottom=3cm} 
\begin{landscape}
	\vspace*{\fill}
	\begin{figure}
		[H]\centering
		\caption{ITT effect for \textbf{subcategories of mental \& behavioral disorders (pooled)}}\label{fig: ITT_d5_subcategories}
		\begin{subfigure}[h]{0.31\linewidth}\centering\caption{Total}
			\includegraphics[width=\linewidth]{paper/effect_d5_frequency.pdf}
		\end{subfigure}
		\begin{subfigure}[h]{0.31\linewidth}\centering\caption{Women}
			\includegraphics[width=\linewidth]{paper/effect_d5_frequency_f.pdf}
		\end{subfigure}
		\begin{subfigure}[h]{0.31\linewidth}\centering\caption{Men}
			\includegraphics[width=\linewidth]{paper/effect_d5_frequency_m.pdf}
		\end{subfigure}
		\scriptsize
		\begin{minipage}{0.95\linewidth}
			\emph{Notes:} The figure plots ITT estimates (along with 90\%/95\% confidence intervals) across the five most common subcategories of mental and behavioral disorders. Moreover, they indicate how often each subcategory is diagnosed over the time window of 1995-2014. The outcomes are defined as the number of cases per 1,000 individuals (births). The point estimates are coming from a DiD regression as described in section \ref{sec:empirical_strategy}, with a bandwidth of six months, month-of-birth and year fixed effects, and standard errors clustered at the month-of-birth level. The control group is comprised of children that are born in the same months but one year before the reform (i.e. children born between November 1977 and October 1978).\newline
			\emph{Source:} Hospital registry data.
		\end{minipage}
	\end{figure}
	\vspace*{\fill}\clearpage
\end{landscape}
\restoregeometry 






%--------------------------------------------------------------------
% TABLES
%--------------------------------------------------------------------
\newpage
\TODO\section{Tables}
\vspace*{\fill}
{\Huge \begin{center}\textbf{TABLES}\end{center}}
\vspace*{\fill}\clearpage

%--------------------------------------------
% Overview ICD 9 and ICD 10 
%\begin{small}
%\vspace*{\fill}
%\begin{table}[h] % table environment for caption and label
%	\begin{threeparttable}
%		\centering % center the tabular
%		\caption{Overview of diagnoses} % caption
%		\label{tab:outcomes_coding_main_chapters} 
%		\begin{tabular}{lrrr} % alignment and number of columns of actual table
%			\toprule % top thicker horizontal line (" rule ")
%			&\multicolumn{1}{r}{(1)}& &\multicolumn{1}{r}{(2)}\\
%			&\multicolumn{1}{r}{ICD-9} & & \multicolumn{1}{r}{ICD-10} \\ 
%			\midrule
%			%-------------------------------------------------------------------------
%			\textit{Hospital admission - main diagnosis chapters:}\\
 \hspace{4pt} Infectious and parasitic diseases                           	&	001-139		& &		A00-B99 \\
 \hspace{4pt} Neoplasms                                                   	&	140-239		& &		C00-D48 \\
 %\hspace{4pt} Diseases of the blood and blood-forming organs              	&	280-289		& &		D50-D90 \\
 \hspace{4pt} Endocrine, nutritional and metabolic diseases					&	240-278		& &		E00-E90 \\
 \hspace{4pt} Mental \& behavioral  disorders                             	&	290-319		& &		F00-F99 \\
 \hspace{4pt} Diseases of the nervous system                              	&	320-359		& &		G00-G99 \\
 \hspace{4pt} Diseases of the sense organs                                	&	360-389		& &		H00-H95 \\
 \hspace{4pt} Diseases of the circulatory system                          	&	390-459		& &		I00-I99 \\
 \hspace{4pt} Diseases of the respiratory system                          	&	460-519		& &		J00-J99 \\
 \hspace{4pt} Diseases of the digestive system                            	&	520-579		& &		K00-K93 \\
 \hspace{4pt} Diseases of the skin and subcutaneous tissue                	&	680-709		& &		L00-L99 \\
 \hspace{4pt} Diseases of the musculoskeletal system						&	710-739		& &		M00-M99 \\
 \hspace{4pt} Diseases of the genitourinary system                        	&	580-629		& &		N00-N99 \\
 %\hspace{4pt} Pregnancy, childbirth, and the puerperium  					&	630-676		& &		O00-O99 \\
 %\hspace{4pt} Certain conditions originating in the perinatal period      	&	760-779		& &		P00-P96 \\
 %\hspace{4pt} Congenital anomalies                                        	&	740-759		& &		Q00-Q99 \\
 \hspace{4pt} Symptoms, signs, and ill-defined conditions                 	&	780-799		& &		R00-R99 \\
 \hspace{4pt} Injury, poisoning and certain other                          	&	800-999		& &		S00-T98 \\
 \hspace{8pt} consequences of external causes  \\
\\

\textit{Subcategories (mental \& behavioral disorders)}\\
 \hspace{12pt} Organic, including symptomatic, mental disorders				& 290,293,294,310		& & F00-F09 \\
 \hspace{12pt} MBD due to psychoactive substance use$^1$						& 291,292,303,304,305	& & F10-F19 \\
 \hspace{12pt} Schizophrenia, schizotypal and delusional disorders			& 295,297,298			& & F20-F29 \\
 \hspace{12pt} Mood [affective] disorders									& 296,311				& & F30-F39 \\
 \hspace{12pt} Neurotic, stress-related and somatoform disorders			& 300,306,308,309		& & F40-F48 \\
 \hspace{12pt} Behavioural syndromes associated with 						& 316					& & F50-F59 \\
 \hspace{16pt} physiological disturbances and physical factors \\
 \hspace{12pt} Disorders of adult personality and behavior					& 301,302				& & F60-F69 \\
 \hspace{12pt} Mental retardation 											& 317,318,319			& &	F70-F79 \\
 \hspace{12pt} Disorders of psychological development						& 299,315				& & F80-F89 \\
 \hspace{12pt} Behavioural and emotional disorders with 					& 312,313,314,307		& & F90-F98 \\
 \hspace{16pt} onset usually occurring in childhood and adolescence \\

%  \hspace{4pt} Metabolic Syndrome\\
%  \hspace{12pt} Type II Diabetes												& 250 				& & E11-E12 \\
%  \hspace{12pt} Overweight \& Obesity										& 278			    & & E65-E68 \\
%  \hspace{12pt} Hypertension													& 401,402,404,405	& & I10,I11,I13,I15	\\
%  \hspace{12pt} Ischemic heart disease										& 410-414	 		& & I20-I25	\\
% \vspace{-10pt}\\
%  \hspace{4pt} Index respiratory system										&					& &		    \\
%  \hspace{12pt} Pneumonia													& 480-486			& & J12-J18	\\
%  \hspace{12pt} Acute bronchitis												& 460-466			& & J20-J22	\\
%  \hspace{12pt} Chronic lower respiratory diseases							& 490-494,496		& & J40-47	\\
% \vspace{-10pt}\\
%  \hspace{4pt} Index drug abuse        										& 291,303-305,980   & &	F10-F19	\\	

%			%-------------------------------------------------------------------------
%			\bottomrule % bottom thicker horizontal line (" rule ")
%		\end{tabular}
%		\begin{tablenotes}
%			\scriptsize{ \item \textit{Notes:} The table shows the classification of diseases according to the "International Statistical Classification of Diseases and Related Health Problems (ICD)", a medical classification list provided by the World Health Organization. \newline \textit{Source:} World Health Organization (WHO), see for example: \href{http://www.who.int/classifications/icd/en/}{http://www.who.int/classifications/icd/en/}\newline$^1$ Psychoactive substances include alcohol, opioids, cannabinoids, sedatives or hypnotics, cocaine, other stimulants (including caffeine), hallucinogens, tobacco, volatile solvents,  multiple drug use and use of other psychoactive substances. }
%		\end{tablenotes}
%	\end{threeparttable}
%\end{table}
%\vspace*{\fill}\clearpage 
%%\end{small}
%%\normalsize


%--------------------------------------------
% Overview Hopsital admission
\newgeometry{left=0cm,right=0cm,top=0cm,bottom=3cm} 

\begin{landscape}
	\vspace*{\fill}
	\begin{table}[h] \centering % table environment for caption and label
		\begin{threeparttable}
			\centering % center the tabular
			\caption{Summary statistics for different diagnoses} % caption
			\label{tab:outcomes_coding_main_chapters} 
			\begin{tabular}{lrrrrrr} % alignment and number of columns of actual table
				\toprule % top thicker horizontal line (" rule ")
				&\multicolumn{1}{r}{(1)}& &\multicolumn{1}{r}{(2)}&\multicolumn{1}{r}{(3)} &\multicolumn{1}{r}{(4)}\\
				&\multicolumn{1}{r}{ICD-9} & & \multicolumn{1}{r}{ICD-10}&\multicolumn{1}{r}{Mean}&\multicolumn{1}{r}{SD} \\ 
				\midrule
				%-------------------------------------------------------------------------
				\\
\underline{\textit{A. Hospital admission}}									& 						& & 			 & 	 120.625     &  10.961\\ 
 \hspace{10pt} Infectious and parasitic diseases                           	&	001-139				& &		A00-B99  & 	   4.210     &   0.493\\
 \hspace{10pt} Neoplasms                                                   	&	140-239				& &		C00-D48  & 	   5.155     &   1.282\\
 \hspace{10pt} Mental \& behavioral  disorders                             	&	290-319				& &		F00-F99  & 	  18.956     &   5.548\\
 \hspace{10pt} Diseases of the nervous system                              	&	320-359				& &		G00-G99  & 	   4.500     &   1.264\\
 \hspace{10pt} Diseases of the sense organs                                	&	360-389				& &		H00-H95  & 	   2.404     &   0.348\\
 \hspace{10pt} Diseases of the circulatory system                          	&	390-459				& &		I00-I99  & 	   4.108     &   1.380\\
 \hspace{10pt} Diseases of the respiratory system                          	&	460-519				& &		J00-J99  & 	  10.994     &   1.939\\
 \hspace{10pt} Diseases of the digestive system                            	&	520-579				& &		K00-K93  & 	  16.746     &   2.079\\
 \hspace{10pt} Diseases of the skin and subcutaneous tissue                	&	680-709				& &		L00-L99  & 	   3.849     &   0.536\\
 \hspace{10pt} Diseases of the musculoskeletal system						&	710-739				& &		M00-M99  & 	   8.897     &   2.228\\
 \hspace{10pt} Diseases of the genitourinary system                        	&	580-629				& &		N00-N99  & 	  10.621     &   1.362\\
 \hspace{10pt} Symptoms, signs, and ill-defined conditions                 	&	780-799				& &		R00-R99  & 	   6.794     &   1.410\\
 \hspace{10pt} Injury, poisoning and certain other                          &	800-999				& &		S00-T98  & 	  21.196     &   5.978\\
 \hspace{18pt} consequences of external causes  \\
 
 %\hspace{4pt} Diseases of the blood and blood-forming organs              	&	280-289		& &		D50-D90  & \\
 %\hspace{4pt} Endocrine, nutritional and metabolic diseases				&	240-278		& &		E00-E90  & \\
 %\hspace{4pt} Pregnancy, childbirth, and the puerperium  					&	630-676		& &		O00-O99  & \\
 %\hspace{4pt} Certain conditions originating in the perinatal period      	&	760-779		& &		P00-P96  & \\
 %\hspace{4pt} Congenital anomalies                                        	&	740-759		& &		Q00-Q99  & \\
 
 
 % ------------------------------------------------------------------------
 % 							Mean           SD          Min          Max
 % ------------------------------------------------------------------------
 % Ratio using number~e      120.625       10.961        99.27       156.35
 % Ratio using number~e        4.210        0.493         2.87         5.65
 % Ratio using number~e        5.155        1.282         3.29         8.97
 % Ratio using number~e       18.956        5.548         6.68        30.17
 % Ratio using number~e        4.500        1.264         2.60         7.50
 % Ratio using number~e        2.404        0.348         1.62         3.07
 % Ratio using number~e        4.108        1.380         1.41         7.66
 % Ratio using number~e       10.994        1.939         8.30        15.55
 % Ratio using number~e       16.746        2.079        12.09        20.97
 % Ratio using number~e        3.849        0.536         2.63         5.12
 % Ratio using number~e        8.897        2.228         6.45        15.46
 % Ratio using number~e       10.621        1.362         6.79        13.64
 % Ratio using number~e        6.794        1.410         4.26        10.70
 % Ratio using number~e       21.196        5.978        14.41        34.06
 % ------------------------------------------------------------------------



				%-------------------------------------------------------------------------
				\bottomrule % bottom thicker horizontal line (" rule ")
			\end{tabular}
			\begin{tablenotes}
				\scriptsize{ \item \textit{Continued on next page}}
			\end{tablenotes}
		\end{threeparttable}
	\end{table}
	\vspace*{\fill}\clearpage 
	
	
	\vspace*{\fill}
	\begin{table}[h] \centering % table environment for caption and label
		\begin{threeparttable}
			\centering % center the tabular
			\caption*{\textit{Summary statistics for different diagnoses (continued)}} % caption

			\begin{tabular}{lrrrrrr} % alignment and number of columns of actual table
				\toprule % top thicker horizontal line (" rule ")
				&\multicolumn{1}{r}{(1)}& &\multicolumn{1}{r}{(2)}&\multicolumn{1}{r}{(3)} &\multicolumn{1}{r}{(4)}\\
				&\multicolumn{1}{r}{ICD-9} & & \multicolumn{1}{r}{ICD-10}&\multicolumn{1}{r}{Mean}&\multicolumn{1}{r}{SD} \\ 
				\midrule
				%-------------------------------------------------------------------------
				

\\
\underline{\textit{B. Mental \& behavioral disorders}}						&						& & 			 & 18.956		& 5.548		\\
 \hspace{10pt} Organic, including symptomatic, mental disorders				& 290,293,294,310		& & 	F00-F09	 &	0.115		& 0.056		\\
 \hspace{10pt} MBD due to psychoactive substance use$^1$					& 291,292,303,304,305	& & 	F10-F19	 &	6.366		& 2.232		\\
 \hspace{10pt} Schizophrenia, schizotypal and delusional disorders			& 295,297,298			& & 	F20-F29	 &	5.140		& 2.246		\\
 \hspace{10pt} Mood [affective] disorders									& 296,311				& & 	F30-F39	 &	2.339		& 1.673		\\
 \hspace{10pt} Neurotic, stress-related and somatoform disorders			& 300,306,308,309		& & 	F40-F48	 &	2.799		& 0.356		\\
 \hspace{10pt} Behavioural syndromes associated with 						& 316					& & 	F50-F59	 &	0.308		& 0.225		\\
 \hspace{18pt} physiological disturbances and physical factors 														 &				& 			\\
 \hspace{10pt} Disorders of adult personality and behavior					& 301,302				& & 	F60-F69	 &	1.375		& 0.511		\\
 \hspace{10pt} Mental retardation 											& 317,318,319			& &		F70-F79	 &	0.121		& 0.075		\\
 \hspace{10pt} Disorders of psychological development						& 299,315				& & 	F80-F89	 &	0.026		& 0.029		\\
 \hspace{10pt} Behavioural and emotional disorders with 					& 312,313,314,307		& & 	F90-F98	 &	0.320		& 0.535		\\
  \hspace{18pt} onset usually occurring in childhood and adolescence 																			\\


% var			mean	SD
% d5						18.956	5.548
% organic					0.115	0.056
% psychoactive substances	6.366	2.232
% schizophrenia			5.140	2.246
% affective				2.339	1.673
% neurosis				2.799	0.356
% physical factors		0.308	0.225
% personality				1.375	0.511
% retardation				0.121	0.075
% development				0.026	0.029
% childhood				0.320	0.535
		


				%-------------------------------------------------------------------------
				\bottomrule % bottom thicker horizontal line (" rule ")
			\end{tabular}
			\begin{tablenotes}
				\scriptsize{ \item \textit{Notes:} The table shows the classification of diseases according to the "International Statistical Classification of Diseases and Related Health Problems (ICD)", a medical classification list provided by the World Health Organization.  The table reports next to the ICD codes summary statistics for the different diagnosis types. Column 3 and 4 show mean and standard deviation of the number of diagnosis per 1,000 individuals for the pre-reform treatment cohort. \newline \textit{Source:} World Health Organization (WHO), see for example: \href{http://www.who.int/classifications/icd/en/}{http://www.who.int/classifications/icd/en/} for the ICD coding, hospital registry data for the mean and standard deviation. \newline\hspace*{15 pt}$^1$ Psychoactive substances include alcohol, opioids, cannabinoids, sedatives or hypnotics, cocaine, other stimulants (including caffeine), hallucinogens, tobacco, volatile solvents,  multiple drug use and use of other psychoactive substances. }
			\end{tablenotes}
		\end{threeparttable}
	\end{table}
	\vspace*{\fill}\clearpage 
\end{landscape}
\restoregeometry
%--------------------------------------------
%VALIDITY: Birth rate effects
 \vspace*{\fill}
 \begin{table}[H] \centering
 \caption{Birth rate effects of the 1979 maternity leave reform}\label{tab: validity_birth_rate}
 {\def\sym#1{\ifmmode^{#1}\else\(^{#1}\)\fi} 
 \begin{tabular}{l*{5}{c}}
 	\toprule
 	& \multicolumn{4}{c}{Estimation window} \\
 	\cmidrule{2-5}
 	&\multicolumn{1}{c}{(1)}&\multicolumn{1}{c}{(2)}&\multicolumn{1}{c}{(3)}&\multicolumn{1}{c}{(4)}\\
	 & \multicolumn{1}{c}{$\pm$7 days} & \multicolumn{1}{c}{$\pm$14 days} & \multicolumn{1}{c}{$\pm$21 days} & \multicolumn{1}{c}{$\pm$28 days}\\ 
 	\midrule
 	\multicolumn{5}{l}{\emph{Panel A. Dependent variable is number of births}}\\
 	ML reform           &      -30.46         &      -30.23\sym{*}  &      -33.32\sym{**} &      -32.78\sym{***}\\
                    &     (30.31)         &     (17.73)         &     (14.08)         &     (12.37)         \\
Observations        &         196         &         392         &         588         &         784         \\
$R^2$               &       0.856         &       0.842         &       0.832         &       0.817         \\

 	\\ \\
 	\multicolumn{5}{l}{\emph{Panel B. Dependent variable is ln(number of births)}}\\
 	ML reform           &     -0.0448         &     -0.0440\sym{*}  &     -0.0477\sym{**} &     -0.0476\sym{***}\\
                    &    (0.0425)         &    (0.0247)         &    (0.0197)         &    (0.0173)         \\
Observations        &         196         &         392         &         588         &         784         \\
$R^2$               &       0.855         &       0.844         &       0.833         &       0.819         \\

 	\bottomrule
 \end{tabular}}
 \begin{minipage}{0.7\linewidth}
 		\scriptsize \emph{Notes:} The table shows regression estimates of the impact of the 1979 maternity leave reform on fertility. The sample comprises daily births within the relevant estimation window in the federal states of Baden-Württemberg and North Rhine-Westphalia from 1977-1990. Panel A uses the number of births as dependent variable, whereas panel B shows results with the log of the number of births as dependent variable. All specifications control for day of year, public holiday, and year$\times$day of week fixed effects. The estimation window is referring to the number of days before and after May 01. For instance, the $\pm$ 7 day window includes the last week in April and the first week in May across all years. Standard errors are reported in parentheses. \newline Significance levels: * p < 0.10, ** p < 0.05, *** p < 0.01. \newline\emph{Source:} Birth registry data from North Rhine-Westphalia and Baden-Württemberg. Taken together, both states account for almost 36\% of all births in the former Federal Republic of Germany in 1979.
 	\end{minipage}
 \end{table}
 \vspace*{\fill}\clearpage 

%--------------------------------------------
% HOSPITAL - Difference in Difference tables for 
\newpage
\input{../../analysis/tables/paper/DD_hospital2_overall_agegroup_total}
\input{../../analysis/tables/paper/DD_hospital2_overall_agegroup_women}
\input{../../analysis/tables/paper/DD_hospital2_overall_agegroup_men}

%--------------------------------------------
% ITT effects nach chaptern
% original 3 3 2 3
\newpage
\newgeometry{left=3cm,right=3cm,top=1cm,bottom=2.5cm} 
\vspace*{\fill}
\begin{table}[H] \centering 
	\begin{threeparttable} \centering \caption{ITT effects on \textbf{hospital admission and main diagnoses chapters (total)}}\label{tab: ITT_across_chapters_per_age_group_total}
		{\def\sym#1{\ifmmode^{#1}\else\(^{#1}\)\fi} 
			\begin{tabular}{l*{5}{c}}
				\toprule 
				&\multicolumn{1}{c}{(1)}&\multicolumn{1}{c}{(2)}&\multicolumn{1}{c}{(3)}&\multicolumn{1}{c}{(4)}&\multicolumn{1}{c}{(5)}\\
				\midrule
				&\multirow{2}{*}{Overall} & \multicolumn{4}{c}{Age brackets [years]} \\ 
				\cmidrule(lr){3-6}
				&&\multicolumn{1}{c}{17-21}&\multicolumn{1}{c}{22-26}&\multicolumn{1}{c}{27-31}&\multicolumn{1}{c}{32-35}\\
				
				\midrule
				
				\input{paper/paper_ITTacrosschapters.tex}
				
				\bottomrule 
		\end{tabular}}
		% \begin{tablenotes} 
		% 	\item 
		% \end{tablenotes} 
	\end{threeparttable} 
	\begin{minipage}{0.95\linewidth}
		\scriptsize \emph{Notes:} This table reports intention-to-treat estimates across the main diagnosis chapters for the entire life-course or per age bracket. The outcomes are defined as the number of cases per 1,000 individuals (births). The point estimates are coming from a DiD regression as described in section \ref{sec:empirical_strategy}, with a bandwidth of six months, month-of-birth and year fixed effects, and clustered standard errors on the month-of-birth level. The control group is comprised of children that are born in the same months but one year before (i.e. children born between November 1977 and October 1978).\newline
		\emph{Legend:} Infectious and parasitic diseases (IPD), neoplasms (Neo), mental \& behavioral disorders (MBD), diseases of the nervous system (Ner), diseases of the sense organs (Sen), diseases of the circulatory system (Cir), diseases of the respiratory system (Res), diseases of the digestive system (Dig), diseases of the skin and subcutaneous tissue (SST), diseases of the musculoskeletal system (Mus), diseases of the genitourinary system (Gen), "symptoms, signs, and ill-defined conditions" (Sym), "injury, poisoning and certain other consequences of external causes" (Ext).
	\end{minipage}
\end{table} 
\vspace*{\fill}\clearpage 
\restoregeometry
%--------------------------------------------
% d5 - Difference in Difference table 
\vspace*{\fill}
\begin{table}[H] \centering 
 \begin{threeparttable} \centering \caption{ITT effects on mental \& behavioral disorders}\label{tab_mlch: DD_d5_total}
  {\def\sym#1{\ifmmode^{#1}\else\(^{#1}\)\fi} 
 	\begin{tabular}{l*{6}{c}}
 		\toprule 
 		%\multicolumn{5}{l}{Dependant variable: \textbf{Hospital admission (total)}}\\ \\ 
 		& \multicolumn{5}{c}{Estimation window} \\ 
 		\cmidrule(lr){2-6}
 		&\multicolumn{1}{c}{(1)}&\multicolumn{1}{c}{(2)}&\multicolumn{1}{c}{(3)}&\multicolumn{1}{c}{(4)}&\multicolumn{1}{c}{(5)}\\
 		&\multicolumn{1}{c}{6M}&\multicolumn{1}{c}{5M}&\multicolumn{1}{c}{4M}&\multicolumn{1}{c}{3M}&\multicolumn{1}{c}{Donut}\\
 		\midrule
 		\multicolumn{5}{l}{\emph{Panel A. Over entire length of the life-course}} \\
 		\hspace*{10pt}Overall&      0.0829         &      -0.852\sym{**} &      -0.634\sym{**} &      -0.809\sym{***}\\
                    &     (0.377)         &     (0.350)         &     (0.249)         &     (0.274)         \\
\midrule Dependent mean&       18.65         &       19.05         &       18.96         &       19.16         \\
Effect in SDs [\%]  &       1.470         &       14.87         &       11.43         &       14.64         \\
Observations        &         160         &         320         &         480         &         400         \\
 \\ \\
 		\multicolumn{5}{l}{\emph{Panel B. Age brackets}} \\
 		\hspace*{10pt}Age 17-21&       0.135         &       0.318         &       0.268         &       0.174         &     -0.0603         \\
                    &     (0.516)         &     (0.387)         &     (0.314)         &     (0.263)         &     (0.239)         \\
 \hspace*{10pt}Age 22-26&       1.221         &      -0.172         &    -0.00769         &      -0.360         \\
                    &     (0.689)         &     (0.607)         &     (0.420)         &     (0.454)         \\
 \hspace*{10pt}Age 27-31&      -0.243         &      -1.508\sym{***}&      -1.000\sym{**} &      -1.020\sym{**} \\
                    &     (0.395)         &     (0.478)         &     (0.357)         &     (0.433)         \\
 \hspace*{10pt}Age 32-35&      -2.293\sym{***}&      -2.352\sym{***}&      -2.015\sym{***}&      -1.906\sym{***}&      -1.949\sym{***}\\
                    &     (0.365)         &     (0.305)         &     (0.295)         &     (0.372)         &     (0.439)         \\
 
 		\bottomrule 
 	\end{tabular}}
 	\begin{tablenotes} 
 		\item \scriptsize \emph{Notes:} The table shows DiD estimates of the 1979 maternity leave reform on mental and behavioral disorders for different estimation windows around the cutoff. The \textit{`Donut'} specification uses a bandwidth of half a year and excludes children born in April and May. Panel A shows the effect for the entire pooled time frame and panel B breaks the life-course up in age brackets. The outcome variables are defined as the number of cases per thousand individuals. All regressions control for year and month-of-birth fixed effects. The control group is comprised of children that are born in the same months but one year before the reform (i.e. children born between November 1977 and October 1978). In order to compare the two birth cohorts at the same age, I shift the control cohort from wave $t$ to wave $t+1$. The dependent mean and the effect size in standard deviation units correspond to pre-reform values of the treated group. Table \ref{tab_mlch: observations_age_brackets} contains the number of observations for the estimations per age bracket. Clustered standard errors are reported in parentheses. \newline Significance levels: * p < 0.10, ** p < 0.05, *** p < 0.01. \newline 	%\emph{Source:} Hospital registry data.
 	\end{tablenotes} 
 \end{threeparttable} 
 \end{table}
\vspace*{\fill}\clearpage 
\input{../../analysis/tables/paper/DD_d5_overall_agegroup_women}
 \vspace*{\fill}
 \begin{table}[H] \centering 
 	\begin{threeparttable} \centering %\caption{ITT effects on \textbf{mental \& behavioral disorders (men)}}\label{tab: DD_d5_male} 
 	{\def\sym#1{\ifmmode^{#1}\else\(^{#1}\)\fi} 
 			\begin{tabular}{l*{6}{c}}
 				\toprule 
 				%\multicolumn{5}{l}{Dependant variable: \textbf{Hospital admission (total)}}\\ \\ 
 				& \multicolumn{5}{c}{Estimation window} \\ 
 				\cmidrule(lr){2-6}
 				&\multicolumn{1}{c}{(1)}&\multicolumn{1}{c}{(2)}&\multicolumn{1}{c}{(3)}&\multicolumn{1}{c}{(4)}&\multicolumn{1}{c}{(5)}\\
 				&\multicolumn{1}{c}{6M}&\multicolumn{1}{c}{5M}&\multicolumn{1}{c}{4M}&\multicolumn{1}{c}{3M}&\multicolumn{1}{c}{Donut}\\
 				\midrule
 				\multicolumn{5}{l}{\emph{Panel A. Over entire length of the life-course}} \\
 				\hspace*{10pt}Overall&      -0.140         &      -1.498\sym{***}&      -1.267\sym{***}&      -1.607\sym{***}\\
                    &     (0.292)         &     (0.426)         &     (0.292)         &     (0.290)         \\
\midrule Dependent mean&       21.60         &       22.22         &          22         &       22.21         \\
Effect in SDs [\%]  &       1.840         &       19.24         &       16.78         &       21.28         \\
Observations        &         160         &         320         &         480         &         400         \\
 \\ \\
 				\multicolumn{5}{l}{\emph{Panel B. Age brackets}} \\
 				\hspace*{10pt}Age 17-21&       0.192         &       0.119         &       0.129         &     -0.0319         &      -0.344\sym{+}  \\
                    &     (0.507)         &     (0.373)         &     (0.300)         &     (0.262)         &     (0.217)         \\
 \hspace*{10pt}Age 22-26&       1.688\sym{***}&      -0.373         &      -0.180         &      -0.602         \\
                    &     (0.231)         &     (0.683)         &     (0.485)         &     (0.526)         \\
 \hspace*{10pt}Age 27-31&      -1.854\sym{**} &      -2.152\sym{***}&      -1.943\sym{***}&      -1.504\sym{***}&      -1.690\sym{***}\\
                    &     (0.652)         &     (0.675)         &     (0.542)         &     (0.507)         &     (0.570)         \\
 \hspace*{10pt}Age 32-35&      -2.879\sym{**} &      -3.938\sym{***}&      -3.518\sym{***}&      -3.989\sym{***}\\
                    &     (0.841)         &     (0.596)         &     (0.515)         &     (0.515)         \\
 
 				\bottomrule 
 		\end{tabular}}
 		\begin{tablenotes} 
 				\item \tiny \emph{Notes:} Clustered standard errors are reported in parentheses. \newline Significance levels: $+$ p $<$ 0.15, * p $<$ 0.10, ** p $<$ 0.05, *** p $<$ 0.01.
 		\end{tablenotes} 
 	\end{threeparttable} 
 \end{table} 
\vspace*{\fill}\clearpage 











%--------------------------------------------
% D5 SUBCATEGORIES
\newpage
\newgeometry{left=3cm,right=3cm,top=1cm,bottom=2.5cm} 
\vspace*{\fill}
\begin{table}[H] \centering 
	\begin{threeparttable} \centering \caption{ITT effects on the \textbf{subcategories of mental \& behavioral disorders (total)}}\label{tab: ITT_across_d5subcategories_per_age_group_total}
		{\def\sym#1{\ifmmode^{#1}\else\(^{#1}\)\fi} 
			\begin{tabular}{l*{5}{c}}
				\toprule 
				&\multicolumn{1}{c}{(1)}&\multicolumn{1}{c}{(2)}&\multicolumn{1}{c}{(3)}&\multicolumn{1}{c}{(4)}&\multicolumn{1}{c}{(5)}\\
				\midrule
				&\multirow{2}{*}{Overall} & \multicolumn{4}{c}{Age brackets [years]} \\ 
				\cmidrule(lr){3-6}
				&&\multicolumn{1}{c}{17-21}&\multicolumn{1}{c}{22-26}&\multicolumn{1}{c}{27-31}&\multicolumn{1}{c}{32-35}\\
				
				\midrule
				
				MBD & -0.621\sym{**} & 0.174 & -0.008 & -1.000\sym{***} & -1.906\sym{***} \\
& (0.242) & (0.257) & (0.410) & (0.349) & (0.362) \\
Psychoactive substances & -0.483\sym{***} & -0.074 & -0.071 & -0.549\sym{***} & -1.428\sym{***
> } \\
& (0.110) & (0.123) & (0.136) & (0.156) & (0.270) \\
Schizophrenia & -0.272\sym{**} & 0.061 & 0.069 & -0.707\sym{***} & -0.572\sym{***} \\
& (0.119) & (0.087) & (0.230) & (0.155) & (0.170) \\
Affective & 0.093\sym{**} & -0.035 & 0.004 & 0.198\sym{***} & 0.235\sym{*} \\
& (0.035) & (0.042) & (0.054) & (0.068) & (0.128) \\
Neurosis & 0.066 & 0.001 & 0.108 & 0.213\sym{***} & -0.088 \\
& (0.040) & (0.086) & (0.102) & (0.066) & (0.054) \\
Personality & 0.013 & 0.172\sym{***} & 0.005 & -0.158\sym{*} & 0.039 \\
& (0.036) & (0.055) & (0.072) & (0.086) & (0.091) \\
				
				\bottomrule 
		\end{tabular}}
		% \begin{tablenotes} 
		% 	\item 
		% \end{tablenotes} 
	\end{threeparttable} 
	\begin{minipage}{0.9\linewidth}
		\scriptsize \emph{Notes:} The table shows DD estimates of the 1979 maternity leave reform on subcategories of mental and behavioral disorders. The first column shows the effect for the entire pooled time frame, whereas columns 2 to 5 display the impact per age group. The outcome variables are defined as the number of cases per thousand individuals (births). The point estimates are coming from a DiD regression as described in section \ref{sec:empirical_strategy}, with a bandwidth of six months, and month-of-birth and year fixed effects. The control group is comprised of children that are born in the same months but one year before the reform (i.e. children born between November 1977 and October 1978). Clustered standard errors are reported in parentheses. \newline Significance levels: * p < 0.10, ** p < 0.05, *** p < 0.01. \newline 	\emph{Source:} Hospital registry data.
	\end{minipage}
\end{table} 
\vspace*{\fill}\clearpage 
\restoregeometry
%--------------------------------------------
% HOSPITAL 2 -ROBUSTNESS TABLE
\newpage
\newgeometry{left=1cm,right=1cm,top=1cm,bottom=2.5cm} 
\begin{landscape}
	\vspace*{\fill}
	\begin{table}[htbp] \centering 
		\begin{threeparttable} \centering 
			\caption{Robustness for \textbf{hospital admission}}\label{tab: robustness_hospital} 
			{\def\sym#1{\ifmmode^{#1}\else\(^{#1}\)\fi} 
				\begin{tabular}{l*{10}{c}} \toprule 
					
					& & \multicolumn{2}{c}{Alternative specifications} & \multicolumn{3}{c}{\clb{c}{Alternative\\estimation}} & \multicolumn{2}{c}{Placebos}& \multicolumn{2}{c}{Heterogeneity}\\
					\cmidrule(lr){3-4} \cmidrule(lr){5-7} \cmidrule(lr){8-9} \cmidrule(lr){10-11}
					&\multicolumn{1}{c}{(1)}&\multicolumn{1}{c}{(2)}&\multicolumn{1}{c}{(3)}&\multicolumn{1}{c}{(4)}&\multicolumn{1}{c}{(5)}&\multicolumn{1}{c}{(6)}&\multicolumn{1}{c}{(7)}&\multicolumn{1}{c}{(8)}&\multicolumn{1}{c}{(9)}&\multicolumn{1}{c}{(10)}\\
					&\multicolumn{1}{c}{Baseline}&\multicolumn{1}{c}{\clb{c}{current\\population}}&\multicolumn{1}{c}{\clb{c}{LMR\\level$^a$}}&\multicolumn{1}{c}{\clb{c}{DDD$^b$}}&\multicolumn{1}{c}{\clb{c}{alt. DD$^b$}}&\multicolumn{1}{c}{add. CG}&\multicolumn{1}{c}{\clb{c}{temporal:\\cohort}}&\multicolumn{1}{c}{\clb{c}{spatial:\\ GDR}}&\multicolumn{1}{c}{\clb{c}{rural$^a$}}&\multicolumn{1}{c}{\clb{c}{urban$^a$}}\\
					\midrule
					\\
					%							1					2					3					4					5					6					7					8				9				10				
					(1) {total} 		&   -2.168\sym{**}	&	-1.581\sym{**}	&   -1.627\sym{**} 	&	-2.226\sym{*}	& 	-2.449\sym{***} & -2.327\sym{**}	&	-0.318			&	-0.0268		&	-0.989		&	-1.779\sym{***} \\
										&	(0.782)			&	(0.675)			&   (0.658)     	&	(1.115)			& 	(0.738)			& (1.003)			&	(0.946)			&	(0.453)		&	(1.143)		&	(0.626)			\\
					(2) {female}		&   -1.815			&	-0.694			& 	-0.558      	&	-1.418			& 	-2.493\sym{***}	& -1.573		    &	0.483			&	-0.319		&	1.248		&	-0.988			\\
										&	(0.807)			&	(0.633)			&   (0.618)     	&	(1.210)			& 	(0.776)			& (1.114)			&	(0.942)			&	(0.457)		&	(1.736)		&	(0.635)			\\
					(3) {male} 			&   -2.525\sym{**}	&	-2.462\sym{**}	&   -2.723\sym{***} &	-2.941\sym{**}	& 	-2.362\sym{**}	& -3.063\sym{**}	&	-1.076			&	0.179		&-3.120\sym{**}	&	-2.628\sym{**}  \\
										&	(0.997)			&	(0.981)			&   (0.957)     	&	(1.271)			& 	(0.774)			& (1.140)			&	 (1.059) 		&	(0.699)		&	(1.180)		&	(1.023)			\\
					\midrule            																																																					
					For total: 																																																			\\							 
					Dependent mean 		&   120.6			&	92.22			&   98.31     		&	121.4			& 	121.4			& 120.6				&	120.2			&	67.4		&	101.0		&	96.11			\\
					Effect in SDs [\%] 	&   19.78			&	16.21			&   4.40      		&	20.25			& 	22.29			& 21.23				&	3.060			&	0.21		&	2.34		&	5.590			\\
					Observations 		&   480				&	288				&   58,751    		&	960				& 	480				& 720				&	480				&	480			&	26,495		&	32,256			\\
					%Federal level		&   \checkmark		&	\checkmark		&   $\times$		& \checkmark		&	\checkmark		& \checkmark		&	\checkmark		&  \checkmark	&	$\times$	&	$\times$		\\ 
					\\
					MOB fixed effects 	&   \checkmark		&	\checkmark		&   \checkmark		& \checkmark		&	\checkmark		& \checkmark		&	\checkmark		&  \checkmark	&	\checkmark	&	\checkmark		\\ 
					Year fixed effects  &   \checkmark		&	\checkmark		&   \checkmark		& \checkmark		&	\checkmark		& \checkmark		&	\checkmark		&  \checkmark	&	\checkmark	&	\checkmark		\\ 
					\bottomrule
			\end{tabular}}
	\end{threeparttable} 
		\begin{minipage}{0.87\linewidth}
		\scriptsize \emph{Notes:} This table %reports intention-to-treat estimates across the main diagnosis chapters for the entire life-course or per age bracket. The outcomes are defined as the number of cases per 1,000 individuals (births). The point estimates are coming from a DiD regression as described in section \ref{sec:empirical_strategy}, with a bandwidth of six months, month-of-birth and year fixed effects, and clustered standard errors on the month-of-birth level. The control group is comprised of children that are born in the same months but one year before (i.e. children born between November 1977 and October 1978).  regressions on LMR include LMR FE
		\newline \emph{Legend:} (1) baseline specification that was used in previous parts of the paper, (2) number of diagnoses divided by the current number of individuals (approximated), (3) the analysis is carried out on the level of labor market regions, (4) triple difference model (the third differences stems from the former region of the GDR), (5) alternative difference-in-difference model which compares pre and post of the treatment cohort in FRG with the respective values in GDR, (6) uses next to the cohort before the reform also the cohort 2 years prior to the policy change as control group, (7) first placebo, in which the entire analysis set-up is pushed back by one year, i.e. the placebo TG is the cohort prior to the real TG and the placebo CG is the cohort born 2 years before the reform took place, (8) second placebo, in which we run the normal DD set-up in the area of the former GDR, (9) + (10)  DD carried out in rural and urban regions (compare with figure \ref{fig: AMR_regions_population_density}). \newline
		\hspace{10 pt}$^a$: level of analysis on Labor Market Regions: weighted regressions (by population), includes region fixed effects.\newline
		\hspace{10 pt}$^b$: standard errors clustered on the month-of-birth$\times$birth-cohort$\times$East-West cell level.
	\end{minipage}
\end{table} 
	\vspace*{\fill}\clearpage
\end{landscape}

% Welche Columns sind geupdated: 1 2 3 4 5 6 7 8 9 10






\restoregeometry
%--------------------------------------------








%--------------------------------------------------------------------
% APPENDIX
%--------------------------------------------------------------------
\newpage
\TODO\section{Appendix}
\vspace*{\fill}
{\Huge \begin{center}\textbf{APPENDIX}\end{center}}
\vspace*{\fill}\clearpage


\renewcommand\thefigure{A\arabic{figure}}
\setcounter{figure}{0} 
\captionsetup[subfigure]{labelformat=parens}
%--------------------------------------------
% Validity: fertility histograms for TG & CG
\begin{landscape}
	\vspace*{\fill}
	\begin{figure}
		[H]\centering
		\caption{Fertility distribution for different years}\label{fig: fertility_hist}
		\begin{subfigure}[h]{0.40\linewidth}\centering
			\caption{Control: Nov 1977-Oct 1978}
			\includegraphics[width=\linewidth]{paper/fertility_per_day_histogram_CG.pdf}
		\end{subfigure}
		\begin{subfigure}[h]{0.40\linewidth}\centering
			\caption{Treatment: Nov 1978-Oct 1979}
			\includegraphics[width=\linewidth]{paper/fertility_per_day_histogram_TG.pdf}
		\end{subfigure}
		\scriptsize
		\begin{minipage}{0.95\linewidth}
			\emph{Notes:} The figure shows the number of births per day across birth-months for the former region of the Federal Republic of Germany. The solid vertical red line divides pre- and post-reform time span for the treatment group, i.e. two or six months of job-protected leave after childbirth. The dashed line illustrates the same cutoff value for the control group born in other years.\newline
			\emph{Source:} German Federal Statistical Office (Destatis).
		\end{minipage}
	\end{figure}
	\vspace*{\fill}\clearpage
\end{landscape}
%--------------------------------------------
% map: AMR of Germany
\vspace*{\fill}
\begin{figure}[H]\centering
	\caption{Labor market regions in Germany}\label{fig: AMR_regions_Germany}
	\includegraphics[width=0.8\linewidth]{paper/AMR_germany.png}
	\scriptsize
	\begin{minipage}{0.9 \linewidth}
		\emph{Notes:} This map shows the labor market regions (LMR) used in the analysis. The areas with the red background depict the area of the former Federal Republic of Germany ("West Germany"), while the white areas indicate the area of the former German Democratic Republic ("East Germany"). The area of West Germany is used throughout the paper, the regions of East Germany only in a robustness check (triple-differences model). The baseline specification aggregates to level of West and East Germany, yet there are some specifications that aggregate to the regional level (red borderlines). There are in total 245 LMR, with 204 in the area of the FRG and 41 in the area of the former GDR. The black outlines indicate federal state boundaries and the red dots represent the corresponding state capitals. The regions \newline \emph{Source:} Own representation with data from the Federal Institute for Research on Building, Urban Affairs and Spatial Development (BBSR).
	\end{minipage}
\end{figure}
\vspace*{\fill}\clearpage
%--------------------------------------------
% map: population density per AMR in Germany
\newpage

\vspace*{\fill}
\begin{figure}[H]\centering
	\caption{Region-level population density}\label{fig: AMR_regions_population_density}
	\includegraphics[width=0.8 \linewidth]{paper/AMR_popdensity.png}
	\scriptsize
	\begin{minipage}{0.9\linewidth}
		\emph{Notes:} This map shows the regional variation of population density across German regions. \newline\emph{Source:} Own representation with data from the Federal Institute for Research on Building, Urban Affairs and Spatial Development (BBSR) and the Regional Database Germany.
	\end{minipage}
\end{figure}
\vspace*{\fill}\clearpage
%--------------------------------------------

% LC MATRIX FOR ALL CHAPTERS

% Part 1 of LC matrix
\begin{figure}[H]\centering
	\caption{Life-course approach for all chapters}\label{fig: appendix_lc_matrix_chapters}
	\includegraphics[width=1.1\linewidth]{paper/lc_matrix_chapters_1.pdf}
		\scriptsize
		\begin{minipage}{\linewidth}
			\emph{Continued on next page}
		\end{minipage}
\end{figure}

%Part 2 of LC matrix
\begin{figure}[H]\centering
		\begin{minipage}{\linewidth}\scriptsize
		\begin{center} \emph{Life-course approach for all chapters (continued)}\end{center}
	\end{minipage}
	\includegraphics[width=1.1\linewidth]{paper/lc_matrix_chapters_2.pdf}
		\begin{minipage}{\linewidth}
		\scriptsize \emph{Notes:} This figure plots intention-to-treat estimates (along with confidence intervals) across the main diagnosis chapters for the entire life-course. The outcomes are defined as the number of cases per 1,000 individuals (births). The point estimates are coming from a DiD regression as described in section \ref{sec:empirical_strategy}, with a bandwidth of six months, month-of-birth fixed effects, and clustered standard errors on the month-of-birth level. The control group is comprised of children that are born in the same months but one year before (i.e. children born between November 1977 and October 1978). On the right axis, one can see the dependent mean for the pre-reform treatment children.
	\end{minipage}
\end{figure}
%--------------------------------------------
% LC MATRIX D5 SUBCATGEORIES
\begin{figure}[H]\centering
	\caption{Life-course approach for the subcategories of mental and behavioral disorders.}\label{fig: appendix_lc_matrix_d5_subcateg}
	\includegraphics[width=1.1\linewidth]{paper/lc_matrix_d5subcategories.pdf}
		\begin{minipage}{\linewidth}
		\scriptsize \emph{Notes:} This figure plots intention-to-treat estimates (along with confidence intervals) across the subcategories of mental and behavioral disorders. The outcomes are defined as the number of cases per 1,000 individuals (births). The point estimates are coming from a DiD regression as described in section \ref{sec:empirical_strategy}, with a bandwidth of six months, month-of-birth fixed effects, and clustered standard errors on the month-of-birth level. The control group is comprised of children that are born in the same months but one year before (i.e. children born between November 1977 and October 1978).
	\end{minipage}
\end{figure}
%--------------------------------------------

% TABLES
\renewcommand\thetable{A\arabic{table}}
\setcounter{table}{0} 
%--------------------------------------------
% ITT effects nach chaptern (WOMEN)

\newpage
\newgeometry{left=3cm,right=3cm,top=1cm,bottom=2.5cm} 
\vspace*{\fill}
\begin{table}[H] \centering 
	\begin{threeparttable} \centering \caption{ITT effects on \textbf{hospital admission and main diagnoses chapters (women)}}\label{tab: ITT_across_chapters_per_age_group_women}
		{\def\sym#1{\ifmmode^{#1}\else\(^{#1}\)\fi} 
			\begin{tabular}{l*{5}{c}}
				\toprule 
				&\multicolumn{1}{c}{(1)}&\multicolumn{1}{c}{(2)}&\multicolumn{1}{c}{(3)}&\multicolumn{1}{c}{(4)}&\multicolumn{1}{c}{(5)}\\
				\midrule
				&\multirow{2}{*}{Overall} & \multicolumn{4}{c}{Age brackets [years]} \\ 
				\cmidrule(lr){3-6}
				&&\multicolumn{1}{c}{17-21}&\multicolumn{1}{c}{22-26}&\multicolumn{1}{c}{27-31}&\multicolumn{1}{c}{32-35}\\
				
				\midrule
				
				Hospital            &      -1.742\sym{**} &      -2.916\sym{***}&      0.0274         &      -2.762\sym{**} &      -1.212         \\
                    &     (0.816)         &     (0.935)         &     (1.267)         &     (1.004)         &     (0.866)         \\
IPD                 &     -0.0652         &       0.107         &      -0.193         &      -0.132         &     -0.0378         \\
                    &    (0.0548)         &     (0.111)         &     (0.158)         &    (0.0889)         &     (0.177)         \\
Neo                 &     -0.0235         &      -0.789\sym{***}&       0.387\sym{**} &       0.162         &       0.188         \\
                    &     (0.102)         &     (0.229)         &     (0.162)         &     (0.126)         &     (0.234)         \\
MBD                 &     0.00972         &       0.388         &       0.205         &      -0.426         &      -0.163         \\
                    &     (0.271)         &     (0.313)         &     (0.466)         &     (0.418)         &     (0.388)         \\
Ner                 &      0.0382         &      -0.274\sym{***}&       0.221\sym{*}  &       0.106         &       0.115         \\
                    &    (0.0758)         &    (0.0887)         &     (0.124)         &     (0.179)         &     (0.139)         \\
Sen                 &      -0.231\sym{***}&      -0.343\sym{***}&     -0.0591         &      -0.217\sym{**} &      -0.323\sym{**} \\
                    &    (0.0464)         &    (0.0705)         &    (0.0802)         &     (0.103)         &     (0.148)         \\
Cir                 &     -0.0638         &      -0.133         &       0.120         &     -0.0920         &      -0.172\sym{*}  \\
                    &    (0.0632)         &    (0.0843)         &     (0.147)         &    (0.0866)         &    (0.0886)         \\
Res                 &      -0.169         &      -0.354         &     -0.0438         &      -0.149         &      -0.118         \\
                    &     (0.108)         &     (0.283)         &     (0.165)         &     (0.202)         &     (0.216)         \\
Dig                 &      -0.801\sym{***}&      -0.796\sym{***}&      -0.592         &      -1.069\sym{***}&      -0.734         \\
                    &     (0.191)         &     (0.258)         &     (0.408)         &     (0.282)         &     (0.438)         \\
SST                 &      0.0373         &     -0.0342         &       0.158         &      0.0960         &     -0.0972         \\
                    &    (0.0435)         &     (0.118)         &     (0.133)         &     (0.106)         &     (0.124)         \\
Mus                 &       0.120         &       0.119         &      -0.267         &       0.207         &       0.498\sym{**} \\
                    &    (0.0849)         &     (0.145)         &     (0.216)         &     (0.198)         &     (0.239)         \\
Gen                 &     -0.0317         &      0.0523         &       0.147         &      -0.570         &       0.313         \\
                    &     (0.172)         &     (0.262)         &     (0.297)         &     (0.369)         &     (0.273)         \\
Sym                 &      -0.146         &      -0.214         &       0.139         &      -0.396\sym{**} &      -0.105         \\
                    &     (0.114)         &     (0.171)         &     (0.158)         &     (0.142)         &     (0.235)         \\
Ext                 &      -0.422\sym{***}&      -0.663\sym{***}&      -0.322         &      -0.219         &      -0.498\sym{**} \\
                    &    (0.0893)         &     (0.124)         &     (0.220)         &     (0.190)         &     (0.185)         \\

				
				\bottomrule 
		\end{tabular}}
		% \begin{tablenotes} 
		% 	\item 
		% \end{tablenotes} 
	\end{threeparttable} 
	\begin{minipage}{0.9\linewidth}
		\scriptsize \emph{Notes:} This table reports intention-to-treat estimates across the main diagnosis chapters for the entire life-course or per age bracket. The outcomes are defined as the number of cases per 1,000 individuals (births). The point estimates are coming from a DiD regression as described in section \ref{sec:empirical_strategy}, with a bandwidth of six months, month-of-birth and year fixed effects, and clustered standard errors on the month-of-birth level. The control group is comprised of children that are born in the same months but one year before (i.e. children born between November 1977 and October 1978).\newline
		\emph{Legend:} Infectious and parasitic diseases (IPD), neoplasms (Neo), mental \& behavioral disorders (MBD), diseases of the nervous system (Ner), diseases of the sense organs (Sen), diseases of the circulatory system (Cir), diseases of the respiratory system (Res), diseases of the digestive system (Dig), diseases of the skin and subcutaneous tissue (SST), diseases of the musculoskeletal system (Mus), diseases of the genitourinary system (Gen), "symptoms, signs, and ill-defined conditions" (Sym), "injury, poisoning and certain other consequences of external causes" (Ext).
	\end{minipage}
\end{table} 
\vspace*{\fill}\clearpage 
\restoregeometry



%--------------------------------------------
% ITT effects nach chaptern (MEN)
% original 3 3 2 3
\newpage
\newgeometry{left=3cm,right=3cm,top=1cm,bottom=2.5cm} 
\vspace*{\fill}
\begin{table}[H] \centering 
	\begin{threeparttable} \centering \caption{ITT effects on \textbf{hospital admission and main diagnoses chapters (men)}}\label{tab: ITT_across_chapters_per_age_group_men}
		{\def\sym#1{\ifmmode^{#1}\else\(^{#1}\)\fi} 
			\begin{tabular}{l*{5}{c}}
				\toprule 
				&\multicolumn{1}{c}{(1)}&\multicolumn{1}{c}{(2)}&\multicolumn{1}{c}{(3)}&\multicolumn{1}{c}{(4)}&\multicolumn{1}{c}{(5)}\\
				\midrule
				&\multirow{2}{*}{Overall} & \multicolumn{4}{c}{Age brackets [years]} \\ 
				\cmidrule(lr){3-6}
				&&\multicolumn{1}{c}{17-21}&\multicolumn{1}{c}{22-26}&\multicolumn{1}{c}{27-31}&\multicolumn{1}{c}{32-35}\\
				
				\midrule
				
				Hospital            &      -2.410\sym{**} &      -0.273         &      -1.230         &      -2.558\sym{*}  &      -6.373\sym{***}\\
                    &     (1.015)         &     (1.201)         &     (1.048)         &     (1.294)         &     (1.526)         \\
IPD                 &     -0.0897\sym{*}  &     -0.0824         &      -0.135         &      -0.120         &    -0.00380         \\
                    &    (0.0472)         &    (0.0887)         &     (0.150)         &     (0.142)         &     (0.104)         \\
Neo                 &      0.0527         &       0.352\sym{*}  &       0.282         &      0.0677         &      -0.627\sym{***}\\
                    &     (0.123)         &     (0.203)         &     (0.176)         &     (0.172)         &     (0.162)         \\
MBD                 &      -1.192\sym{***}&     -0.0319         &      -0.180         &      -1.504\sym{***}&      -3.518\sym{***}\\
                    &     (0.288)         &     (0.262)         &     (0.485)         &     (0.507)         &     (0.515)         \\
Ner                 &      0.0467         &      0.0787         &       0.250\sym{*}  &     -0.0319         &      -0.150         \\
                    &     (0.101)         &     (0.122)         &     (0.125)         &     (0.136)         &     (0.209)         \\
Sen                 &      -0.114\sym{**} &    -0.00121         &      -0.137         &      -0.133\sym{*}  &      -0.203         \\
                    &    (0.0478)         &    (0.0995)         &    (0.0944)         &    (0.0726)         &     (0.126)         \\
Cir                 &      -0.129         &      0.0370         &     -0.0520         &      -0.301\sym{*}  &      -0.218         \\
                    &    (0.0876)         &     (0.122)         &    (0.0922)         &     (0.162)         &     (0.261)         \\
Res                 &      -0.313\sym{***}&      -0.406\sym{**} &      -0.351\sym{**} &      -0.391\sym{***}&     -0.0488         \\
                    &     (0.103)         &     (0.186)         &     (0.152)         &     (0.128)         &     (0.181)         \\
Dig                 &      -0.120         &      -0.110         &      -0.406\sym{*}  &       0.273         &      -0.266         \\
                    &     (0.164)         &     (0.172)         &     (0.208)         &     (0.396)         &     (0.369)         \\
SST                 &       0.166\sym{**} &       0.144         &       0.219\sym{**} &       0.213         &      0.0690         \\
                    &    (0.0735)         &     (0.102)         &    (0.0843)         &     (0.134)         &     (0.141)         \\
Mus                 &      -0.169\sym{**} &      -0.167         &     -0.0445         &      -0.298         &      -0.168         \\
                    &    (0.0732)         &     (0.151)         &     (0.189)         &     (0.240)         &     (0.150)         \\
Gen                 &      0.0854         &       0.373\sym{*}  &       0.155         &       0.121         &      -0.406\sym{*}  \\
                    &     (0.120)         &     (0.200)         &     (0.213)         &     (0.192)         &     (0.211)         \\
Sym                 &     -0.0409         &      -0.158         &      0.0689         &      0.0486         &      -0.143         \\
                    &    (0.0741)         &     (0.184)         &    (0.0951)         &     (0.134)         &     (0.128)         \\
Ext                 &      -0.600\sym{**} &      -0.189         &      -0.965\sym{**} &      -0.580\sym{*}  &      -0.680\sym{**} \\
                    &     (0.280)         &     (0.570)         &     (0.411)         &     (0.294)         &     (0.265)         \\

				
				\bottomrule 
		\end{tabular}}
		% \begin{tablenotes} 
		% 	\item 
		% \end{tablenotes} 
	\end{threeparttable} 
	\begin{minipage}{0.9\linewidth}
		\scriptsize \emph{Notes:} This table reports intention-to-treat estimates across the main diagnosis chapters for the entire life-course or per age bracket. The outcomes are defined as the number of cases per 1,000 individuals (births). The point estimates are coming from a DiD regression as described in section \ref{sec:empirical_strategy}, with a bandwidth of six months, month-of-birth and year fixed effects, and clustered standard errors on the month-of-birth level. The control group is comprised of children that are born in the same months but one year before (i.e. children born between November 1977 and October 1978).\newline
		\emph{Legend:} Infectious and parasitic diseases (IPD), neoplasms (Neo), mental \& behavioral disorders (MBD), diseases of the nervous system (Ner), diseases of the sense organs (Sen), diseases of the circulatory system (Cir), diseases of the respiratory system (Res), diseases of the digestive system (Dig), diseases of the skin and subcutaneous tissue (SST), diseases of the musculoskeletal system (Mus), diseases of the genitourinary system (Gen), "symptoms, signs, and ill-defined conditions" (Sym), "injury, poisoning and certain other consequences of external causes" (Ext).
	\end{minipage}
\end{table} 
\vspace*{\fill}\clearpage 
\restoregeometry



%--------------------------------------------
% D5 SUBCATEGORIES (WOMEN)
\newpage
\newgeometry{left=3cm,right=3cm,top=1cm,bottom=2.5cm} 
\vspace*{\fill}
\begin{table}[H] \centering 
	\begin{threeparttable} \centering \caption{ITT effects on the \textbf{subcategories of mental \& behavioral disorders (women)}}\label{tab: ITT_across_d5subcategories_per_age_group_women}
		{\def\sym#1{\ifmmode^{#1}\else\(^{#1}\)\fi} 
			\begin{tabular}{l*{5}{c}}
				\toprule 
				&\multicolumn{1}{c}{(1)}&\multicolumn{1}{c}{(2)}&\multicolumn{1}{c}{(3)}&\multicolumn{1}{c}{(4)}&\multicolumn{1}{c}{(5)}\\
				\midrule
				&\multirow{2}{*}{Overall} & \multicolumn{4}{c}{Age brackets [years]} \\ 
				\cmidrule(lr){3-6}
				&&\multicolumn{1}{c}{17-21}&\multicolumn{1}{c}{22-26}&\multicolumn{1}{c}{27-31}&\multicolumn{1}{c}{32-35}\\
				
				\midrule
				
				MBD & 0.060 & 0.388 & 0.205 & -0.426 & -0.163 \\
& (0.266) & (0.305) & (0.455) & (0.408) & (0.378) \\
Psychoactive substances & -0.091 & -0.155 & 0.152 & 0.177 & -0.641\sym{***} \\
& (0.124) & (0.190) & (0.157) & (0.163) & (0.228) \\
Schizophrenia & -0.033 & 0.063 & 0.030 & -0.594\sym{***} & 0.280 \\
& (0.101) & (0.063) & (0.230) & (0.117) & (0.204) \\
Affective & 0.155\sym{**} & 0.031 & 0.030 & 0.189 & 0.276 \\
& (0.064) & (0.040) & (0.101) & (0.124) & (0.173) \\
Neurosis & 0.046 & 0.090 & 0.099 & 0.096 & -0.122 \\
& (0.080) & (0.180) & (0.130) & (0.132) & (0.128) \\
Personality & 0.032 & 0.272\sym{***} & 0.071 & -0.286\sym{*} & 0.102 \\
& (0.057) & (0.089) & (0.104) & (0.170) & (0.166) \\
				
				\bottomrule 
		\end{tabular}}
		% \begin{tablenotes} 
		% 	\item 
		% \end{tablenotes} 
	\end{threeparttable} 
	\begin{minipage}{0.9\linewidth}
		\scriptsize \emph{Notes:} The table shows DD estimates of the 1979 maternity leave reform on subcategories of mental and behavioral disorders. The first column shows the effect for the entire pooled time frame, whereas columns 2 to 5 display the impact per age group. The outcome variables are defined as the number of cases per thousand individuals (births). The point estimates are coming from a DiD regression as described in section \ref{sec:empirical_strategy}, with a bandwidth of six months, and month-of-birth and year fixed effects. The control group is comprised of children that are born in the same months but one year before the reform (i.e. children born between November 1977 and October 1978). Clustered standard errors are reported in parentheses. \newline Significance levels: * p < 0.10, ** p < 0.05, *** p < 0.01. \newline 	\emph{Source:} Hospital registry data.
	\end{minipage}
\end{table} 
\vspace*{\fill}\clearpage 
\restoregeometry

%--------------------------------------------
% D5 SUBCATEGORIES
\newpage
\newgeometry{left=3cm,right=3cm,top=1cm,bottom=2.5cm} 
\vspace*{\fill}
\begin{table}[H] \centering 
	\begin{threeparttable} \centering \caption{ITT effects on the \textbf{subcategories of mental \& behavioral disorders (men)}}\label{tab: ITT_across_d5subcategories_per_age_group_men}
		{\def\sym#1{\ifmmode^{#1}\else\(^{#1}\)\fi} 
			\begin{tabular}{l*{5}{c}}
				\toprule 
				&\multicolumn{1}{c}{(1)}&\multicolumn{1}{c}{(2)}&\multicolumn{1}{c}{(3)}&\multicolumn{1}{c}{(4)}&\multicolumn{1}{c}{(5)}\\
				\midrule
				&\multirow{2}{*}{Overall} & \multicolumn{4}{c}{Age brackets [years]} \\ 
				\cmidrule(lr){3-6}
				&&\multicolumn{1}{c}{17-21}&\multicolumn{1}{c}{22-26}&\multicolumn{1}{c}{27-31}&\multicolumn{1}{c}{32-35}\\
				
				\midrule
				
				MBD & -1.267\sym{***} & -0.032 & -0.180 & -1.504\sym{***} & -3.518\sym{***} \\
& (0.292) & (0.255) & (0.473) & (0.495) & (0.502) \\
Psychoactive substances & -0.886\sym{***} & 0.012 & -0.262 & -1.205\sym{***} & -2.133\sym{***} \\
& (0.182) & (0.205) & (0.219) & (0.248) & (0.380) \\
Schizophrenia & -0.462\sym{**} & 0.067 & 0.135 & -0.787\sym{**} & -1.355\sym{***} \\
& (0.203) & (0.143) & (0.330) & (0.311) & (0.270) \\
Affective & 0.078\sym{***} & -0.099\sym{*} & -0.026 & 0.198\sym{***} & 0.184 \\
& (0.020) & (0.057) & (0.061) & (0.066) & (0.131) \\
Neurosis & 0.049 & -0.094 & 0.113 & 0.321\sym{***} & -0.060 \\
& (0.054) & (0.098) & (0.146) & (0.059) & (0.072) \\
Personality & -0.012 & 0.075\sym{*} & -0.064 & -0.042 & -0.025 \\
& (0.034) & (0.043) & (0.086) & (0.037) & (0.084) \\
				
				\bottomrule 
		\end{tabular}}
		% \begin{tablenotes} 
		% 	\item 
		% \end{tablenotes} 
	\end{threeparttable} 
	\begin{minipage}{0.9\linewidth}
		\scriptsize \emph{Notes:} The table shows DD estimates of the 1979 maternity leave reform on subcategories of mental and behavioral disorders. The first column shows the effect for the entire pooled time frame, whereas columns 2 to 5 display the impact per age group. The outcome variables are defined as the number of cases per thousand individuals (births). The point estimates are coming from a DiD regression as described in section \ref{sec:empirical_strategy}, with a bandwidth of six months, and month-of-birth and year fixed effects. The control group is comprised of children that are born in the same months but one year before the reform (i.e. children born between November 1977 and October 1978). Clustered standard errors are reported in parentheses. \newline Significance levels: * p < 0.10, ** p < 0.05, *** p < 0.01. \newline 	\emph{Source:} Hospital registry data.
	\end{minipage}
\end{table} 
\vspace*{\fill}\clearpage 
\restoregeometry


%--------------------------------------------
% d5 - robustness in one table - blueprint for paper
\newpage
\newgeometry{left=1cm,right=1cm,top=1cm,bottom=2.5cm} 
\begin{landscape}
	\vspace*{\fill}
	\begin{table}[htbp] \centering 
		\begin{threeparttable} \centering 
			\caption{Robustness for \textbf{mental and behavioral disorders}} \label{tab: robustness_d5} 
			{\def\sym#1{\ifmmode^{#1}\else\(^{#1}\)\fi} 
				\begin{tabular}{l*{10}{c}} \toprule 
					
					& & \multicolumn{2}{c}{Alternative specifications} & \multicolumn{3}{c}{\clb{c}{Alternative\\estimation}} & \multicolumn{2}{c}{Placebos}& \multicolumn{2}{c}{Heterogeneity}\\
					\cmidrule(lr){3-4} \cmidrule(lr){5-7} \cmidrule(lr){8-9} \cmidrule(lr){10-11}
					&\multicolumn{1}{c}{(1)}&\multicolumn{1}{c}{(2)}&\multicolumn{1}{c}{(3)}&\multicolumn{1}{c}{(4)}&\multicolumn{1}{c}{(5)}&\multicolumn{1}{c}{(6)}&\multicolumn{1}{c}{(7)}&\multicolumn{1}{c}{(8)}&\multicolumn{1}{c}{(9)}&\multicolumn{1}{c}{(10)}\\
					&\multicolumn{1}{c}{Baseline}&\multicolumn{1}{c}{\clb{c}{current\\population}}&\multicolumn{1}{c}{\clb{c}{LMR\\level$^a$}}&\multicolumn{1}{c}{\clb{c}{DDD}}&\multicolumn{1}{c}{\clb{c}{alt. DD}}&\multicolumn{1}{c}{add. CG}&\multicolumn{1}{c}{\clb{c}{temporal:\\cohort}}&\multicolumn{1}{c}{\clb{c}{spatial:\\ GDR}}&\multicolumn{1}{c}{\clb{c}{rural$^a$}}&\multicolumn{1}{c}{\clb{c}{urban$^a$}}\\
					\midrule
					\\
					(1) {total} 		&   -0.634\sym{**}	&	-0.832\sym{***}	&   -0.831\sym{**}  &	-0.834\sym{**}  & 	-0.558\sym{**}  & -0.553\sym{**}	&	0.162			&	0.200		&	-0.0987		&	-1.006\sym{***} 	\\
										&	(0.249)			&	(0.239)			&   (0.229)     	&	(0.321)			& 	(0.157)			& (0.269)			&	(0.304)			&	(0.141)		&	(0.588)		&	(0.202)				\\
					(2) {female}		&   0.0599			&	-0.0853			& 	-0.0724     	&	-0.0385			& 	-0.217			& 0.289			    &	0.457			&	0.0984		&	0.299		&	-0.161				\\
										&	(0.266)			&	(0.266)			&   (0.258)     	&	(0.320)			& 	(0.144)			& (0.287)			&	(0.369)			&	(0.196)		&	(0.816)		&	(0.234)				\\
					(3) {male} 			&   -1.267\sym{***}	&	-1.554\sym{***}	&   -1.598\sym{***} &	-1.533\sym{***} & 	-0.834\sym{***} & -1.323\sym{***}	&	-0.112			&	0.266		&	-0.414		&	-1.877\sym{***} 	\\
										&	(0.292)			&	(0.299)			&   (0.286)     	&	(0.388)			& 	(0.190)			& (0.322)			&	 (0.331) 		&	(0.198)		&	(0.487)		&	(0.333)				\\
					\midrule            																																																						
					For total: 																																																				\\							 
					Dependent mean 		&   18.96			&	17.28			&   17.65     		&	13.80			& 	13.80			& 18.96				&	18.67			&	8.640		&	16.76		&	18.38				\\
					Effect in SDs [\%] 	&   11.43			&	41.92			&   5.20      		&	12.53			& 	8.380			& 9.960				&	3.490			&	9.440		&	1.69		&	7.27				\\
					Observations 		&   480				&	288				&   58,751    		&	960				& 	480				& 720				&	480				&	480			&	26,495		&	32,256				\\
					Federal level		&   \checkmark		&	\checkmark		&   $\times$		& \checkmark		&	\checkmark		& \checkmark		&	\checkmark		&  \checkmark	&	$\times$	&	$\times$			\\ 
					\bottomrule
			\end{tabular}}
	\end{threeparttable} 
		\begin{minipage}{0.87\linewidth}
		\scriptsize \emph{Notes:} This table %reports intention-to-treat estimates across the main diagnosis chapters for the entire life-course or per age bracket. The outcomes are defined as the number of cases per 1,000 individuals (births). The point estimates are coming from a DiD regression as described in section \ref{sec:empirical_strategy}, with a bandwidth of six months, month-of-birth and year fixed effects, and clustered standard errors on the month-of-birth level. The control group is comprised of children that are born in the same months but one year before (i.e. children born between November 1977 and October 1978).  regressions on LMR include LMR FE
		\newline \emph{Legend:} (1) baseline specification that was used in previous parts of the paper, (2) number of diagnoses divided by the current number of individuals (approximated), (3) the analysis is carried out on the level of labor market regions, (4) triple difference model (the third differences stems from the former region of the GDR), (5) alternative difference-in-difference model which compares pre and post of the treatment cohort in FRG with the respective values in GDR, (6) uses next to the cohort before the reform also the cohort 2 years prior to the policy change as control group, (7) first placebo, in which the entire analysis set-up is pushed back by one year, i.e. the placebo TG is the cohort prior to the real TG and the placebo CG is the cohort born 2 years before the reform took place, (8) second placebo, in which we run the normal DD set-up in the area of the former GDR, (9) + (10)  DD carried out in rural and urban regions (compare with figure \ref{fig: AMR_regions_population_density}). \newline
		\hspace{10 pt}$^a$ population weighted regression, includes region fixed effects.
	\end{minipage}
\end{table} 
	\vspace*{\fill}\clearpage
\end{landscape}

% Columns 1, 2 8 (lokal) ok 
% COLUMN 8 uses the r_popf specification
% sind DDD, alt DD, spatial placebo mit r_pop


\restoregeometry
%--------------------------------------------


%t-tests
%%hospital total
\begin{table}[H] \centering 
	\begin{threeparttable} \centering \caption{t-test for \textbf{hospital admission (total)}}\label{tab:t-test_d5total}
		\begin{footnotesize}
			{\def\sym#1{\ifmmode^{#1}\else\(^{#1}\)\fi} 
				\begin{tabular}{l*{3}{c}}
					\toprule 
					& \multicolumn{1}{c}{Before} & \multicolumn{1}{c}{After} & \multicolumn{1}{c}{Difference} \\
					&\multicolumn{1}{c}{(1)}&\multicolumn{1}{c}{(2)}&\multicolumn{1}{c}{(1)-(2)}\\
					\midrule
					Overall (pooled)    &       120.6&       118.4&        2.22   \\
                    &      [11.0]&      [9.85]&      (1.35)   \\
\hspace{12pt}1995   &       111.1&       106.7&        4.48** \\
                    &      [3.65]&      [2.73]&      (1.86)   \\
\hspace{12pt}1996   &       117.6&       112.5&        5.08** \\
                    &      [5.13]&      [1.92]&      (2.24)   \\
\hspace{12pt}1997   &       125.9&       120.1&        5.76** \\
                    &      [5.12]&      [2.07]&      (2.25)   \\
\hspace{12pt}1998   &       126.1&       125.1&        1.00   \\
                    &      [3.99]&      [2.48]&      (1.92)   \\
\hspace{12pt}1999   &       124.7&       124.7&     -0.0025   \\
                    &      [4.76]&      [2.48]&      (2.19)   \\
\hspace{12pt}2000   &       120.1&       119.1&        0.92   \\
                    &      [4.39]&      [3.69]&      (2.34)   \\
\hspace{12pt}2001   &       119.4&       121.9&       -2.48   \\
                    &      [4.33]&      [3.47]&      (2.27)   \\
\hspace{12pt}2002   &       120.6&       118.6&        1.99   \\
                    &      [5.13]&      [1.76]&      (2.21)   \\
\hspace{12pt}2003   &       113.6&       112.5&        1.12   \\
                    &      [4.75]&      [0.77]&      (1.96)   \\
\hspace{12pt}2004   &       109.2&       109.0&        0.18   \\
                    &      [3.49]&      [2.61]&      (1.78)   \\
\hspace{12pt}2005   &       105.6&       105.5&       0.045   \\
                    &      [4.36]&      [3.64]&      (2.32)   \\
\hspace{12pt}2006   &       108.9&       106.0&        2.84   \\
                    &      [3.78]&      [1.64]&      (1.68)   \\
\hspace{12pt}2007   &       109.8&       108.2&        1.67   \\
                    &      [4.46]&      [2.47]&      (2.08)   \\
\hspace{12pt}2008   &       113.9&       112.0&        1.91   \\
                    &      [4.47]&      [1.37]&      (1.91)   \\
\hspace{12pt}2009   &       119.2&       117.9&        1.34   \\
                    &      [5.47]&      [2.10]&      (2.39)   \\
\hspace{12pt}2010   &       122.5&       118.8&        3.67   \\
                    &      [5.48]&      [2.92]&      (2.53)   \\
\hspace{12pt}2011   &       128.6&       123.7&        4.84*  \\
                    &      [5.69]&      [1.51]&      (2.40)   \\
\hspace{12pt}2012   &       133.8&       129.1&        4.67** \\
                    &      [4.72]&      [1.97]&      (2.09)   \\
\hspace{12pt}2013   &       136.4&       134.6&        1.80   \\
                    &      [4.85]&      [1.59]&      (2.09)   \\
\hspace{12pt}2014   &       145.5&       142.0&        3.51   \\
                    &      [7.17]&      [2.97]&      (3.17)   \\

					\bottomrule
			\end{tabular}}
		\end{footnotesize}
	\end{threeparttable} 
	\begin{minipage}{0.9\linewidth}
		\scriptsize \emph{Notes:} This table shows descriptive statistics for two samples: (i) cohorts born before May 1979; (ii) cohorts born after May 1979. Columns 1 and 2 show means with standard deviations in brackets. Column 3 report the difference in means between columns 1 and 2 with standard errors in parenthesis. For each sample we take a bandwidth of half a year around the cutoff date, i.e. individuals born between Nov78-Ap79 (May79-Oct79) in the 'before' ('after') sample.
	\end{minipage}
\end{table} 
%hospital women
\begin{table}[H] \centering 
	\begin{threeparttable} \centering \caption{t-test for \textbf{hospital admission (women)}}\label{tab:t-test_d5female}
		\begin{footnotesize}
			{\def\sym#1{\ifmmode^{#1}\else\(^{#1}\)\fi} 
				\begin{tabular}{l*{3}{c}}
					\toprule 
					& \multicolumn{1}{c}{Before} & \multicolumn{1}{c}{After} & \multicolumn{1}{c}{Difference} \\
					&\multicolumn{1}{c}{(1)}&\multicolumn{1}{c}{(2)}&\multicolumn{1}{c}{(1)-(2)}\\
					\midrule
					Overall (pooled)    &       122.4&       120.6&        1.85   \\
                    &      [11.1]&      [10.8]&      (1.42)   \\
\hspace{12pt}1995   &       124.5&       119.4&        5.14** \\
                    &      [4.21]&      [2.06]&      (1.91)   \\
\hspace{12pt}1996   &       129.7&       125.1&        4.61   \\
                    &      [6.62]&      [1.80]&      (2.80)   \\
\hspace{12pt}1997   &       135.0&       131.1&        3.90   \\
                    &      [5.25]&      [3.80]&      (2.65)   \\
\hspace{12pt}1998   &       131.9&       132.8&       -0.89   \\
                    &      [5.50]&      [4.08]&      (2.80)   \\
\hspace{12pt}1999   &       128.2&       129.0&       -0.81   \\
                    &      [4.89]&      [3.66]&      (2.49)   \\
\hspace{12pt}2000   &       123.9&       122.5&        1.37   \\
                    &      [4.76]&      [4.68]&      (2.72)   \\
\hspace{12pt}2001   &       122.6&       126.1&       -3.52   \\
                    &      [4.55]&      [4.36]&      (2.57)   \\
\hspace{12pt}2002   &       124.1&       121.1&        3.04   \\
                    &      [6.53]&      [2.35]&      (2.83)   \\
\hspace{12pt}2003   &       115.8&       115.1&        0.77   \\
                    &      [4.75]&      [1.55]&      (2.04)   \\
\hspace{12pt}2004   &       109.5&       109.8&       -0.37   \\
                    &      [4.92]&      [3.32]&      (2.42)   \\
\hspace{12pt}2005   &       103.5&       104.1&       -0.60   \\
                    &      [4.02]&      [3.50]&      (2.18)   \\
\hspace{12pt}2006   &       107.6&       104.6&        3.01   \\
                    &      [3.91]&      [3.43]&      (2.12)   \\
\hspace{12pt}2007   &       108.3&       105.8&        2.55   \\
                    &      [4.61]&      [2.82]&      (2.21)   \\
\hspace{12pt}2008   &       112.3&       109.4&        2.90   \\
                    &      [4.30]&      [2.60]&      (2.05)   \\
\hspace{12pt}2009   &       118.2&       114.7&        3.54   \\
                    &      [5.05]&      [2.28]&      (2.26)   \\
\hspace{12pt}2010   &       120.7&       116.6&        4.12   \\
                    &      [5.52]&      [3.49]&      (2.67)   \\
\hspace{12pt}2011   &       126.4&       121.8&        4.53   \\
                    &      [5.58]&      [3.14]&      (2.61)   \\
\hspace{12pt}2012   &       131.5&       126.4&        5.15** \\
                    &      [4.92]&      [2.12]&      (2.19)   \\
\hspace{12pt}2013   &       133.1&       133.8&       -0.69   \\
                    &      [4.11]&      [3.97]&      (2.33)   \\
\hspace{12pt}2014   &       141.5&       142.3&       -0.76   \\
                    &      [6.44]&      [2.95]&      (2.89)   \\

					\bottomrule
			\end{tabular}}
		\end{footnotesize}
	\end{threeparttable} 
	\begin{minipage}{0.9\linewidth}
		\scriptsize \emph{Notes:} This table shows descriptive statistics for two samples: (i) cohorts born before May 1979; (ii) cohorts born after May 1979. Columns 1 and 2 show means with standard deviations in brackets. Column 3 report the difference in means between columns 1 and 2 with standard errors in parenthesis. For each sample we take a bandwidth of half a year around the cutoff date, i.e. individuals born between Nov78-Ap79 (May79-Oct79) in the 'before' ('after') sample.
	\end{minipage}
\end{table} 
%hospital men
\begin{table}[H] \centering 
	\begin{threeparttable} \centering \caption{t-test for \textbf{hospital admission (men)}}\label{tab:t-test_d5male}
		\begin{footnotesize}
			{\def\sym#1{\ifmmode^{#1}\else\(^{#1}\)\fi} 
				\begin{tabular}{l*{3}{c}}
					\toprule 
					& \multicolumn{1}{c}{Before} & \multicolumn{1}{c}{After} & \multicolumn{1}{c}{Difference} \\
					&\multicolumn{1}{c}{(1)}&\multicolumn{1}{c}{(2)}&\multicolumn{1}{c}{(1)-(2)}\\
					\midrule
					Overall (pooled)    &       118.9&       116.4&        2.59   \\
                    &      [13.1]&      [11.5]&      (1.59)   \\
\hspace{12pt}1995   &        98.5&        94.5&        3.96*  \\
                    &      [3.22]&      [3.71]&      (2.01)   \\
\hspace{12pt}1996   &       106.1&       100.5&        5.61** \\
                    &      [4.06]&      [2.32]&      (1.91)   \\
\hspace{12pt}1997   &       117.3&       109.7&        7.62** \\
                    &      [5.86]&      [1.59]&      (2.48)   \\
\hspace{12pt}1998   &       120.6&       117.8&        2.84   \\
                    &      [5.58]&      [2.09]&      (2.43)   \\
\hspace{12pt}1999   &       121.4&       120.6&        0.80   \\
                    &      [4.82]&      [3.25]&      (2.37)   \\
\hspace{12pt}2000   &       116.4&       115.9&        0.52   \\
                    &      [4.88]&      [3.55]&      (2.46)   \\
\hspace{12pt}2001   &       116.4&       117.9&       -1.46   \\
                    &      [4.64]&      [4.33]&      (2.59)   \\
\hspace{12pt}2002   &       117.3&       116.3&        1.02   \\
                    &      [5.87]&      [1.40]&      (2.46)   \\
\hspace{12pt}2003   &       111.6&       110.1&        1.47   \\
                    &      [5.50]&      [1.82]&      (2.36)   \\
\hspace{12pt}2004   &       108.9&       108.2&        0.70   \\
                    &      [3.18]&      [3.56]&      (1.95)   \\
\hspace{12pt}2005   &       107.5&       106.9&        0.64   \\
                    &      [5.92]&      [5.48]&      (3.29)   \\
\hspace{12pt}2006   &       110.1&       107.4&        2.68   \\
                    &      [4.95]&      [2.67]&      (2.30)   \\
\hspace{12pt}2007   &       111.3&       110.5&        0.83   \\
                    &      [7.50]&      [3.44]&      (3.37)   \\
\hspace{12pt}2008   &       115.4&       114.4&        0.96   \\
                    &      [6.04]&      [2.28]&      (2.63)   \\
\hspace{12pt}2009   &       120.2&       121.0&       -0.76   \\
                    &      [6.70]&      [2.58]&      (2.93)   \\
\hspace{12pt}2010   &       124.2&       121.0&        3.22   \\
                    &      [6.45]&      [2.50]&      (2.83)   \\
\hspace{12pt}2011   &       130.7&       125.6&        5.13*  \\
                    &      [5.85]&      [1.83]&      (2.50)   \\
\hspace{12pt}2012   &       136.0&       131.8&        4.21   \\
                    &      [4.93]&      [2.85]&      (2.32)   \\
\hspace{12pt}2013   &       139.5&       135.4&        4.15   \\
                    &      [6.11]&      [2.77]&      (2.74)   \\
\hspace{12pt}2014   &       149.3&       141.8&        7.57*  \\
                    &      [8.14]&      [3.18]&      (3.57)   \\

					\bottomrule
			\end{tabular}}
		\end{footnotesize}
	\end{threeparttable} 
	\begin{minipage}{0.9\linewidth}
		\scriptsize \emph{Notes:} This table shows descriptive statistics for two samples: (i) cohorts born before May 1979; (ii) cohorts born after May 1979. Columns 1 and 2 show means with standard deviations in brackets. Column 3 report the difference in means between columns 1 and 2 with standard errors in parenthesis. For each sample we take a bandwidth of half a year around the cutoff date, i.e. individuals born between Nov78-Ap79 (May79-Oct79) in the 'before' ('after') sample.
	\end{minipage}
\end{table} 



%d5 total
\begin{table}[H] \centering 
	\begin{threeparttable} \centering \caption{t-test for \textbf{mental \& behavioral disorders (total)}}\label{tab:t-test_d5total}
		\begin{footnotesize}
			{\def\sym#1{\ifmmode^{#1}\else\(^{#1}\)\fi} 
			\begin{tabular}{l*{3}{c}}
				\toprule 
				& \multicolumn{1}{c}{Before} & \multicolumn{1}{c}{After} & \multicolumn{1}{c}{Difference} \\
				&\multicolumn{1}{c}{(1)}&\multicolumn{1}{c}{(2)}&\multicolumn{1}{c}{(1)-(2)}\\
				\midrule
				Overall (pooled)    &        19.0&        18.6&        0.38   \\
                    &      [5.55]&      [5.44]&      (0.71)   \\
\hspace{12pt}1995   &        7.38&        6.90&        0.48   \\
                    &      [0.61]&      [0.24]&      (0.27)   \\
\hspace{12pt}1996   &        8.64&        8.33&        0.31   \\
                    &      [0.79]&      [0.43]&      (0.37)   \\
\hspace{12pt}1997   &        11.1&        10.2&        0.91   \\
                    &      [1.18]&      [0.60]&      (0.54)   \\
\hspace{12pt}1998   &        13.4&        12.7&        0.69   \\
                    &      [1.17]&      [1.05]&      (0.64)   \\
\hspace{12pt}1999   &        15.1&        15.4&       -0.36   \\
                    &      [0.88]&      [0.55]&      (0.42)   \\
\hspace{12pt}2000   &        16.3&        16.2&        0.13   \\
                    &      [0.83]&      [0.62]&      (0.42)   \\
\hspace{12pt}2001   &        17.8&        18.8&       -1.03*  \\
                    &      [1.03]&      [0.74]&      (0.52)   \\
\hspace{12pt}2002   &        18.2&        18.3&      -0.059   \\
                    &      [0.88]&      [0.66]&      (0.45)   \\
\hspace{12pt}2003   &        19.0&        18.3&        0.68   \\
                    &      [1.51]&      [0.68]&      (0.68)   \\
\hspace{12pt}2004   &        19.6&        19.3&        0.28   \\
                    &      [1.18]&      [0.96]&      (0.62)   \\
\hspace{12pt}2005   &        20.1&        19.8&        0.36   \\
                    &      [1.08]&      [0.99]&      (0.60)   \\
\hspace{12pt}2006   &        20.8&        20.1&        0.73*  \\
                    &      [0.79]&      [0.51]&      (0.38)   \\
\hspace{12pt}2007   &        21.5&        21.1&        0.44   \\
                    &      [1.05]&      [0.44]&      (0.47)   \\
\hspace{12pt}2008   &        22.0&        21.5&        0.56   \\
                    &      [1.41]&      [0.61]&      (0.63)   \\
\hspace{12pt}2009   &        22.7&        22.7&      -0.058   \\
                    &      [1.42]&      [1.37]&      (0.80)   \\
\hspace{12pt}2010   &        23.7&        22.5&        1.22** \\
                    &      [0.96]&      [0.78]&      (0.51)   \\
\hspace{12pt}2011   &        23.9&        23.4&        0.50   \\
                    &      [1.31]&      [0.75]&      (0.62)   \\
\hspace{12pt}2012   &        24.7&        24.3&        0.35   \\
                    &      [0.78]&      [1.24]&      (0.60)   \\
\hspace{12pt}2013   &        25.8&        25.2&        0.64   \\
                    &      [1.20]&      [0.96]&      (0.63)   \\
\hspace{12pt}2014   &        27.3&        26.5&        0.88   \\
                    &      [1.53]&      [1.10]&      (0.77)   \\

				\bottomrule
			\end{tabular}}
		\end{footnotesize}
	\end{threeparttable} 
	\begin{minipage}{0.9\linewidth}
		\scriptsize \emph{Notes:} This table shows descriptive statistics for two samples: (i) cohorts born before May 1979; (ii) cohorts born after May 1979. Columns 1 and 2 show means with standard deviations in brackets. Column 3 report the difference in means between columns 1 and 2 with standard errors in parenthesis. For each sample we take a bandwidth of half a year around the cutoff date, i.e. individuals born between Nov78-Ap79 (May79-Oct79) in the 'before' ('after') sample.
	\end{minipage}
\end{table} 
%d5 women
\begin{table}[H] \centering 
	\begin{threeparttable} \centering \caption{t-test for \textbf{mental \& behavioral disorders (women)}}\label{tab:t-test_d5female}
		\begin{footnotesize}
			{\def\sym#1{\ifmmode^{#1}\else\(^{#1}\)\fi} 
				\begin{tabular}{l*{3}{c}}
					\toprule 
					& \multicolumn{1}{c}{Before} & \multicolumn{1}{c}{After} & \multicolumn{1}{c}{Difference} \\
					&\multicolumn{1}{c}{(1)}&\multicolumn{1}{c}{(2)}&\multicolumn{1}{c}{(1)-(2)}\\
					\midrule
					Overall (pooled)    &        15.7&        15.8&      -0.070   \\
                    &      [3.59]&      [3.56]&      (0.46)   \\
\hspace{12pt}1995   &        8.73&        8.62&        0.10   \\
                    &      [0.78]&      [0.58]&      (0.40)   \\
\hspace{12pt}1996   &        9.30&        9.79&       -0.49   \\
                    &      [1.16]&      [0.82]&      (0.58)   \\
\hspace{12pt}1997   &        11.6&        11.1&        0.48   \\
                    &      [1.07]&      [0.52]&      (0.49)   \\
\hspace{12pt}1998   &        13.0&        13.1&       -0.15   \\
                    &      [1.73]&      [1.62]&      (0.97)   \\
\hspace{12pt}1999   &        12.8&        13.6&       -0.75   \\
                    &      [0.61]&      [0.98]&      (0.47)   \\
\hspace{12pt}2000   &        13.1&        13.6&       -0.45   \\
                    &      [0.25]&      [1.10]&      (0.46)   \\
\hspace{12pt}2001   &        14.2&        16.5&       -2.32** \\
                    &      [1.36]&      [1.29]&      (0.76)   \\
\hspace{12pt}2002   &        15.3&        15.3&     -0.0090   \\
                    &      [1.28]&      [0.65]&      (0.59)   \\
\hspace{12pt}2003   &        16.1&        15.6&        0.55   \\
                    &      [1.85]&      [0.74]&      (0.81)   \\
\hspace{12pt}2004   &        15.7&        16.3&       -0.62   \\
                    &      [1.45]&      [1.15]&      (0.75)   \\
\hspace{12pt}2005   &        16.2&        15.9&        0.35   \\
                    &      [0.87]&      [1.62]&      (0.75)   \\
\hspace{12pt}2006   &        16.7&        16.3&        0.40   \\
                    &      [0.78]&      [1.20]&      (0.58)   \\
\hspace{12pt}2007   &        17.7&        16.9&        0.77   \\
                    &      [1.31]&      [0.98]&      (0.67)   \\
\hspace{12pt}2008   &        18.0&        16.9&        1.16   \\
                    &      [1.66]&      [1.23]&      (0.84)   \\
\hspace{12pt}2009   &        18.2&        17.4&        0.76   \\
                    &      [1.47]&      [1.58]&      (0.88)   \\
\hspace{12pt}2010   &        18.6&        18.1&        0.53   \\
                    &      [0.69]&      [1.24]&      (0.58)   \\
\hspace{12pt}2011   &        18.2&        18.3&       -0.13   \\
                    &      [1.05]&      [1.44]&      (0.73)   \\
\hspace{12pt}2012   &        20.0&        20.0&     0.00093   \\
                    &      [0.98]&      [1.26]&      (0.65)   \\
\hspace{12pt}2013   &        20.2&        20.8&       -0.56   \\
                    &      [1.02]&      [1.48]&      (0.73)   \\
\hspace{12pt}2014   &        21.2&        22.3&       -1.03   \\
                    &      [1.25]&      [1.04]&      (0.66)   \\

					\bottomrule
			\end{tabular}}
		\end{footnotesize}
	\end{threeparttable} 
	\begin{minipage}{0.9\linewidth}
		\scriptsize \emph{Notes:} This table shows descriptive statistics for two samples: (i) cohorts born before May 1979; (ii) cohorts born after May 1979. Columns 1 and 2 show means with standard deviations in brackets. Column 3 report the difference in means between columns 1 and 2 with standard errors in parenthesis. For each sample we take a bandwidth of half a year around the cutoff date, i.e. individuals born between Nov78-Ap79 (May79-Oct79) in the 'before' ('after') sample.
	\end{minipage}
\end{table} 
%d5 men
\begin{table}[H] \centering 
	\begin{threeparttable} \centering \caption{t-test for \textbf{mental \& behavioral disorders (men)}}\label{tab:t-test_d5male}
		\begin{footnotesize}
			{\def\sym#1{\ifmmode^{#1}\else\(^{#1}\)\fi} 
				\begin{tabular}{l*{3}{c}}
					\toprule 
					& \multicolumn{1}{c}{Before} & \multicolumn{1}{c}{After} & \multicolumn{1}{c}{Difference} \\
					&\multicolumn{1}{c}{(1)}&\multicolumn{1}{c}{(2)}&\multicolumn{1}{c}{(1)-(2)}\\
					\midrule
					Overall (pooled)    &        22.0&        21.2&        0.79   \\
                    &      [7.55]&      [7.45]&      (0.97)   \\
\hspace{12pt}1995   &        6.09&        5.26&        0.84** \\
                    &      [0.62]&      [0.67]&      (0.37)   \\
\hspace{12pt}1996   &        8.01&        6.93&        1.08***\\
                    &      [0.52]&      [0.45]&      (0.28)   \\
\hspace{12pt}1997   &        10.5&        9.20&        1.32*  \\
                    &      [1.31]&      [0.82]&      (0.63)   \\
\hspace{12pt}1998   &        13.8&        12.3&        1.48** \\
                    &      [0.95]&      [1.08]&      (0.59)   \\
\hspace{12pt}1999   &        17.2&        17.2&      0.0075   \\
                    &      [1.36]&      [1.05]&      (0.70)   \\
\hspace{12pt}2000   &        19.4&        18.7&        0.67   \\
                    &      [1.57]&      [0.71]&      (0.70)   \\
\hspace{12pt}2001   &        21.2&        21.0&        0.18   \\
                    &      [1.41]&      [1.35]&      (0.80)   \\
\hspace{12pt}2002   &        21.0&        21.2&       -0.13   \\
                    &      [0.93]&      [1.37]&      (0.68)   \\
\hspace{12pt}2003   &        21.7&        21.0&        0.78   \\
                    &      [1.48]&      [0.86]&      (0.70)   \\
\hspace{12pt}2004   &        23.3&        22.2&        1.10   \\
                    &      [1.33]&      [1.11]&      (0.71)   \\
\hspace{12pt}2005   &        23.8&        23.5&        0.34   \\
                    &      [1.48]&      [1.63]&      (0.90)   \\
\hspace{12pt}2006   &        24.7&        23.7&        1.02*  \\
                    &      [0.93]&      [0.92]&      (0.53)   \\
\hspace{12pt}2007   &        25.2&        25.1&       0.092   \\
                    &      [1.86]&      [1.22]&      (0.91)   \\
\hspace{12pt}2008   &        25.9&        25.9&      -0.033   \\
                    &      [1.96]&      [1.35]&      (0.97)   \\
\hspace{12pt}2009   &        26.9&        27.8&       -0.87   \\
                    &      [1.78]&      [1.36]&      (0.91)   \\
\hspace{12pt}2010   &        28.5&        26.7&        1.84*  \\
                    &      [2.02]&      [0.68]&      (0.87)   \\
\hspace{12pt}2011   &        29.3&        28.3&        1.06   \\
                    &      [1.79]&      [0.70]&      (0.78)   \\
\hspace{12pt}2012   &        29.1&        28.4&        0.66   \\
                    &      [0.66]&      [1.36]&      (0.62)   \\
\hspace{12pt}2013   &        31.2&        29.4&        1.75*  \\
                    &      [2.04]&      [1.16]&      (0.96)   \\
\hspace{12pt}2014   &        33.1&        30.5&        2.66** \\
                    &      [1.99]&      [1.44]&      (1.00)   \\

					\bottomrule
			\end{tabular}}
		\end{footnotesize}
	\end{threeparttable} 
	\begin{minipage}{0.9\linewidth}
		\scriptsize \emph{Notes:} This table shows descriptive statistics for two samples: (i) cohorts born before May 1979; (ii) cohorts born after May 1979. Columns 1 and 2 show means with standard deviations in brackets. Column 3 report the difference in means between columns 1 and 2 with standard errors in parenthesis. For each sample we take a bandwidth of half a year around the cutoff date, i.e. individuals born between Nov78-Ap79 (May79-Oct79) in the 'before' ('after') sample.
	\end{minipage}
\end{table} 





%--------------------------------------------------------------------
% Useful links for the paper
%-------------------------------------------------------------------
\newpage
\subsection*{Useful links for the paper}
\begin{itemize}
\item \textbf{breastfeeding \& metabolism:}

\begin{itemize}
\item[-]\href{http://www.who.int/elena/titles/bbc/breastfeeding_childhood_obesity/en/}{WHO link})\\ 
\textit{"The positive impact of breastfeeding on lowering the risk of death from infectious diseases in the first two years of life is now well-established (1). A mounting body of evidence suggests that breastfeeding may also play a role in programming noncommunicable disease risk later in life (2-13) including protection against overweight and obesity in childhood (2-6)."}

\item[-]\href{http://articles.latimes.com/2011/may/02/news/la-heb-infant-feeding-20110502}{LA Times article}\\
\textit{"The study showed that children who received breast milk for the first four months had a specific pattern of growth and metabolic profile that differed from the formula-fed babies. Even at 15 days of life, the breast-fed infants had blood insulin levels that were lower than the formula-fed infants.\newline
By 3 years of age, many of the metabolic and growth differences between the breast-fed and formula-fed infants had disappeared. However, blood pressure readings were higher in the infants who had been fed the high-protein formula compared with breast-fed infants. The blood pressure rates were still within the normal range."}
\end{itemize}
\end{itemize}






% NOT USED ANY MORE
%-----------------------------------------------------
% Earlier version of descriptive, zum Teil 2 lines in einem Graph zusammengefasst
%\begin{figure}[H]\centering
%	\caption{Evolution of key variables}\label{fig: hospital_admissions_lengthstay_surgery}
%	\begin{subfigure}[h]{0.48\linewidth}\centering
%		\caption{Hospital admissions}
%		\includegraphics[width=\linewidth]{paper/hospitaladmission_genderpercent_total}
%	\end{subfigure}
%	\quad
%	\begin{subfigure}[h]{0.48\linewidth}\centering
%		\caption{Length of stay \& surgery}
%		\includegraphics[width=\linewidth]{paper/surgery_lengthofstay}
%	\end{subfigure}
%	\begin{minipage}{\linewidth}
%		\scriptsize{\emph{Notes:} The figure depicts the evolution of key variables for the treatment cohort (i.e. for individuals born between November 1978 and October 1979). In the left figure, the bold gray line shows the total number of hospital admissions (in thousand, right axis), whereas the red (blue) line indicates the relative frequency of female (male) hospital admissions (on the left axis). Hospital admissions are defined as the sum of all diagnoses, except for diagnoses of the "O" chapter (pregnancy, childbirth, and the puerperium). The right figure reports the relative frequency of surgeries (green line, left axis) and the average number of days the patient stayed in the hospital (yellow line, right axis).} 
%	\end{minipage}
%\end{figure}
%-----------------------------------------------------
% Figure: effect sizes and frequency across chapters
% \vspace*{\fill}
% \begin{figure}[H]\centering
% 	\includegraphics[width = 0.9 \linewidth]{paper/effect_chapters_frequency.pdf}
% 	\begin{minipage}
% 		{\linewidth}
% 		\emph{Notes:} %The figures plot DD estimates (along with 90\% and 95\% confidence intervals) for the impact of the reform on hospital admission over the life-course. The light gray line in the background represents the number of diagnoses that were made for both the treatment and control year. Panel a shows the results for all admissions, whereas panel b and c show the estimates for females and males respectively. The control group is comprised of children	that are born in the same months but one year before (i.e. children born between November 1977 and October 1978.)
% 	\end{minipage}
% \end{figure}
% \vspace*{\fill}\clearpage
%-----------------------------------------------------
%--------------------------------------------
% VALIDITY: RD TABEL 
% \vspace*{\fill}
% \begin{table}[H] \centering
% \caption{Validity.... RD estimates for births}
% {\def\sym#1{\ifmmode^{#1}\else\(^{#1}\)\fi} 
% \begin{tabular}{l*{5}{c}}
% 	\toprule
% 	& \multicolumn{4}{c}{Estimation window} \\
% 	\cmidrule{2-5}
% 	&\multicolumn{1}{c}{(1)}&\multicolumn{1}{c}{(2)}&\multicolumn{1}{c}{(3)}&\multicolumn{1}{c}{(4)}\\
% 	\midrule
% 	\multirow{1}{*}{Variable} & \multicolumn{1}{c}{$\pm$5 days} & \multicolumn{1}{c}{$\pm$10 days} & \multicolumn{1}{c}{$\pm$15 days} & \multicolumn{1}{c}{$\pm$30 days}\\ 
% 	\midrule
% 	\emph{Panel A.}\\
% 	Number of births    &     -1374.3         &      -25.75         &      -71.71         &      -21.86         \\
                    &     (904.6)         &     (53.40)         &     (48.10)         &     (29.62)         \\
Observations        &          10         &          20         &          30         &          60         \\
$R^2$               &       0.874         &       0.813         &       0.712         &       0.719         \\

% 	\\ \\
% 	\emph{Panel B.}\\
% 	ln(number of births)    &      -2.182         &     -0.0392         &      -0.106         &     -0.0336         \\
                    &     (1.266)         &    (0.0736)         &    (0.0697)         &    (0.0433)         \\
Observations        &          10         &          20         &          30         &          60         \\
$R^2$               &       0.889         &       0.845         &       0.735         &       0.741         \\

% 	\bottomrule
% \end{tabular}}
% \begin{minipage}{0.9\linewidth}
% 		\scriptsize \emph{Notes:} estimates come from a regression \newline 		$y_t = \beta_1 After_t + \beta_2 \tilde{X}_t + \beta_3 After_t*\tilde{X}_t + I^{DOW}_t + I^{Public Holiday}_t  +\varepsilon_t $, where $\tilde{X}_t$ is the number of days from the threshold\newline Motivated by \cite{gans2009born} $\rightarrow$ let's discuss that tomorrow!!!Probably bullshit, RDD does not make sense here, does it??
% 	\end{minipage}
% \end{table}
% \vspace*{\fill}\clearpage 



\end{document}