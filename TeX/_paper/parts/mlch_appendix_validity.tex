




%WMWMWMWMWMWMWMWMWMWMWMWMWMWMWMWMWMWMWMWM
% VALIDITY
%WMWMWMWMWMWMWMWMWMWMWMWMWMWMWMWMWMWMWMWM

%--------------------------------------------------------------------
% THREATS
\bigskip
\section*{Potential Threats to Identification and Validity of the Design}\label{sec_mlch:empirical_strategy_threats+validity}
Behavioral responses with respect to the forcing variable, the birth date of the child, could potentially jeopardize the validity of the identification strategy. Typically, gestation length is a normally distributed random variable with mean of 40 weeks and 2 weeks standard deviation, from the beginning of the last menstrual cycle \citep{Ekberg2013parental}. However, parents may influence the time when their child is born in two ways. 

First, parents may be incentivized to get strategically pregnant in response to the more attractive ML leave scheme. However, it is unlikely that children in my sample were conceived due to this reason. The draft bill was only proposed four months before the threshold birth date and the widest specification includes children who were born on October 1979, the month in which the earliest possible birth date could be. However, it is plausible that the topic increased public awareness due to media coverage even before the draft bill was initiated. For that reason, \cite{Dustmann2012} conduct a literature search for articles regarding the reform. Subsequently, their results show that the earliest articles were published two months prior to the cutoff date.

Second, mothers with due dates close to the cutoff date on May 1, 1979 could have timed their child's date of birth by postponing induced births and cesarean sections.\footnote{Shifting a planned birth before the threshold is unlikely because this behavior potentially destroys mothers' eligibility for a more generous leave scheme.} \cite{gans2009born} find that during the introduction of a \$3.000 `Baby Bonus' in Australia, parents postponed the birth by as much as a week in order to be eligible for the benefits: The data shows a sharp decline in the birth rate before the threshold, followed by a huge increase on the first day after the cutoff.

It is possible that there are similar distortionary `introduction effects' in the context of the 1979 ML expansion. In other words, even though the announcement period does not allow for a strategic conception, parents may have been incentivized to delay the delivery of their child due to the reform.\footnote{The scope of this effect is much smaller because the C-section or induced birth rates in the FRG around 1980 were significantly lower than in Australia in 2004. \revision{To be precise, \cite{gans2009born} report a C-section rate of 30\% in Australia in 2004 while \cite{aerzteblatt1988} document a frequency of C-sections of 12.4\% in Bavaria in 1982 (second largest state is used as a proxy for the national level).}\label{rev_mlch: footnote_csections_germany}}


% validity
To check for such potential behavioral responses, I investigate \revision{the number of births} around the cutoff. I follow the example of \cite{gans2009born} closely to show that there are no peculiarities in the fertility distribution. In the analysis, I use daily births from the federal states of Baden-Württemberg and North Rhine-Westphalia over the window of 1977-1990. These two states are representative of the former FRG, as they accounted for almost 36\% of all births in 1979. In the analysis, I limit the sample to a time window of one month before and after the cutoff date. Panel A of Figure \ref{fig_mlch: fertilitydistr} shows the (unadjusted) daily number of births for April and May 1979. Over the time window, a strong weekly pattern with more births on weekdays and fewer births on weekends is observed. On May 1, which fell on a Tuesday, there was an unexpected drop in the number of births. If parents delayed the time of birth, one would expect a rise in the number of births right after the cutoff date. However, May 1 was a national public holiday (Labor Day), during which birth rates have been documented to be lower \citep{neugart2013economic}.

\begin{figure}[H]\centering
	\caption{Daily number of births around the ML expansion}\label{fig_mlch: fertilitydistr}
	\includegraphics[width=0.9\linewidth]{paper/fertility_raw_regression_adjusted.pdf}
	\scriptsize
	\begin{minipage}{0.9 \linewidth}
		\emph{Notes:} The figure plots the number of births around the cutoff date May 01, 1979 for the ML expansion from two to six months after childbirth. Panel A shows the raw data, i.e. the actual number of births per day (unadjusted). Panel B, however, plots the difference between the raw and expected number of births when accounting for day of year, public holiday, and year$\times$day of week fixed effects. For the expected number of births I use data in the same time window (one month before and after the threshold) for the years 1977-1990, except for the year in which the reform took place. \newline\emph{Source:} Birth registry data from North Rhine-Westphalia and Baden-Württemberg. Taken together, both states account for almost 36\% of all births in the former Federal Republic of Germany in 1979.
		% {\color{red}how do I get the gray shading from the original figure in the bottom part?}
	\end{minipage}
\end{figure}

%Panel B
Panel B of Figure \ref{fig_mlch: fertilitydistr} shows the time series after removing any variation in the timing of births stemming from year, day of week, day of year, and public holidays. To do so, I estimate the following equation on all years, \emph{except} April and May 1979:
\begin{align}
\text{Births}_i = I^{\text{Year}}_i\times I^{\text{Day of Week}}_i + I^{\text{Day of Year}}_i + I^{\text{Public Holiday}}_i + \varepsilon_i \label{eq: validity_fig}
\end{align}
I regress the number of children born on day $i$, $\text{Births}_i$, on a series of dummies: year interacted with day of the week, day of the year, and a dummy for public national holidays.\footnote{For public holidays I use Good Friday, Holy Saturday, Easter Sunday, Easter Monday, Labor Day, Ascension Day, Whit Sunday, Whit Monday, and Corpus Christi.} Parameters are left out for better readability. Panel B plots the residuals $(\text{Births}-\widehat{\text{Births}})$ from the calibrated model. In contrast to \cite{gans2009born}, I do not observe systematically fewer births prior to the reform, or more births right after the policy change came into effect. These results suggest that there is no evidence that parents delayed births to after the cutoff date.


Table \ref{tab_mlch: validity_birth_rate} presents estimates of the effect of the ML expansion on fertility outcomes, for different estimation windows.\footnote{I estimate: $f(\text{Births}_i) = I^{\text{Reform}}_i + I^{\text{Year}}_i\times I^{\text{Day of Week}}_i + I^{\text{Day of Year}}_i + I^{\text{Public Holiday}}_i + \varepsilon_i$, with the dummy $I^{\text{Reform}}_i$, which equals one for days after the cutoff in 1979. The results remain qualitatively the same if $I^{\text{Public Holiday}}_i$ is interacted with $I^{\text{Day of Week}}_i$.} Irrespective of whether the number of births or the log of the number of births is used, there is no evidence suggesting that parents deliberately postponed births until after the threshold date. On the contrary, the point estimates are negative throughout all specifications and become significantly different from zero with increasing estimation windows. Furthermore, the magnitude of the estimates is constant across estimation windows, which is a counterintuitive result if parents shifted births from the week prior to the reform to the week after the reform. If such behavior were present, one would expect point estimates to decrease in absolute value as the estimation window is enlarged. 


%--------------------------------------------
% Fertility distribuition






%--------------------------------------------
%VALIDITY: Birth rate effects
 \begin{table}[H] \centering
 \caption{Birth rate effects of the 1979 ML reform}\label{tab_mlch: validity_birth_rate}
 {\def\sym#1{\ifmmode^{#1}\else\(^{#1}\)\fi} 
 \begin{tabular}{l*{5}{c}}
 	\toprule
 	& \multicolumn{4}{c}{Estimation window} \\
 	\cmidrule{2-5}
 	&\multicolumn{1}{c}{(1)}&\multicolumn{1}{c}{(2)}&\multicolumn{1}{c}{(3)}&\multicolumn{1}{c}{(4)}\\
	 & \multicolumn{1}{c}{$\pm$7 days} & \multicolumn{1}{c}{$\pm$14 days} & \multicolumn{1}{c}{$\pm$21 days} & \multicolumn{1}{c}{$\pm$28 days}\\ 
 	\midrule
 	\multicolumn{5}{l}{\emph{Panel A. Dependent variable is number of births}}\\
 	ML reform           &      -30.46         &      -30.23\sym{*}  &      -33.32\sym{**} &      -32.78\sym{***}\\
                    &     (30.31)         &     (17.73)         &     (14.08)         &     (12.37)         \\
Observations        &         196         &         392         &         588         &         784         \\
$R^2$               &       0.856         &       0.842         &       0.832         &       0.817         \\

 	\\ \\
 	\multicolumn{5}{l}{\emph{Panel B. Dependent variable is ln(number of births)}}\\
 	ML reform           &     -0.0448         &     -0.0440\sym{*}  &     -0.0477\sym{**} &     -0.0476\sym{***}\\
                    &    (0.0425)         &    (0.0247)         &    (0.0197)         &    (0.0173)         \\
Observations        &         196         &         392         &         588         &         784         \\
$R^2$               &       0.855         &       0.844         &       0.833         &       0.819         \\

 	\bottomrule
 \end{tabular}}
 \begin{minipage}{0.7\linewidth}
 		\scriptsize \emph{Notes:} The table shows regression estimates of the impact of the 1979 ML reform on fertility. The sample comprises daily births within the relevant estimation window in the federal states of Baden-Württemberg and North Rhine-Westphalia from 1977-1990. Panel A uses the number of births as dependent variable, whereas panel B shows results with the log of the number of births as dependent variable. All specifications control for day of year, public holiday, and year$\times$day of week fixed effects. The estimation window is referring to the number of days before and after May 01. For instance, the $\pm$ 7 day window includes the last week in April and the first week in May across all years. Standard errors are reported in parentheses. \newline Significance levels: * p < 0.10, ** p < 0.05, *** p < 0.01. \newline\emph{Source:} Birth registry data from North Rhine-Westphalia and Baden-Württemberg. Taken together, both states account for almost 36\% of all births in the former Federal Republic of Germany in 1979.
 	\end{minipage}
 \end{table}
\clearpage


