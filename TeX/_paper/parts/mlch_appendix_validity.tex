




%WMWMWMWMWMWMWMWMWMWMWMWMWMWMWMWMWMWMWMWM
% VALIDITY
%WMWMWMWMWMWMWMWMWMWMWMWMWMWMWMWMWMWMWMWM

%--------------------------------------------------------------------
% THREATS
\bigskip
\subsection{Potential Threats to Identification and Validity of the Design}\label{sec_mlch:empirical_strategy_threats+validity}

In the following, I discuss two threats to the empirical strategy. First, I examine potential self-selection into the treatment group by postponing delivery or strategic conception. Second, I consider threats along the distribution of birth months, which may result from seasonal and age of school entry effects.

\subsubsection{Threats at the Threshold}
Behavioral responses with respect to the forcing variable, the birth date of the child, could potentially jeopardize the validity of the identification strategy. Typically, gestation length is a normally distributed random variable with mean of 40 weeks and 2 weeks standard deviation, from the beginning of the last menstrual cycle \citep{Ekberg2013parental}. However, parents may influence the time when their child is born in two ways. 

First, parents may be incentivized to get strategically pregnant in response to the more attractive ML leave scheme. However, it is unlikely that children in my sample were conceived due to this reason. The draft bill was only proposed four months before the threshold birth date and the widest specification includes children who were born on October 1979, the month in which the earliest possible birth date could be. However, it is plausible that the topic increased public awareness due to media coverage even before the draft bill was initiated. For that reason, \cite{Dustmann2012} conduct a literature search for articles regarding the reform. Subsequently, their results show that the earliest articles were published two months prior to the cutoff date.

Second, mothers with due dates close to the cutoff date on May 1, 1979 could have timed their child's date of birth by postponing induced births and cesarean sections.\footnote{Shifting a planned birth before the threshold is unlikely because this behavior potentially destroys mothers' eligibility for a more generous leave scheme.} \cite{gans2009born} find that during the introduction of a \$3.000 `Baby Bonus' in Australia, parents postponed the birth by as much as a week in order to be eligible for the benefits: The data shows a sharp decline in the birth rate before the threshold, followed by a huge increase on the first day after the cutoff.

It is possible that there are similar distortionary `introduction effects' in the context of the 1979 ML expansion. In other words, even though the announcement period does not allow for a strategic conception, parents may have been incentivized to delay the delivery of their child due to the reform.\footnote{The scope of this effect is much smaller because the C-section or induced birth rates in the FRG around 1980 were significantly lower than in Australia in 2004. To be precise, \cite{gans2009born} report a C-section rate of 30\% in Australia in 2004 while \cite{aerzteblatt1988} document a frequency of C-sections of 12.4\% in Bavaria in 1982. As national numbers are not available for that time period, I use the second-largest state as a proxy for the federal level.}


% validity
To check for such potential behavioral responses, I investigate the number of births around the cutoff. I follow the example of \cite{gans2009born} closely to show that there are no peculiarities in the fertility distribution. In the analysis, I use daily births from the federal states of Baden-Württemberg and North Rhine-Westphalia over the window of 1977-1990. These two states are representative of the former FRG, as they accounted for almost 36\% of all births in 1979. In the analysis, I limit the sample to a time window of one month before and after the cutoff date. Panel A of Figure \ref{fig_mlch: fertilitydistr} shows the (unadjusted) daily number of births for April and May 1979. Over the time window, a strong weekly pattern with more births on weekdays and fewer births on weekends is observed. On May 1, which fell on a Tuesday, there was an unexpected drop in the number of births. If parents delayed the time of birth, one would expect a rise in the number of births right after the cutoff date. However, May 1 was a national public holiday (Labor Day), during which birth rates have been documented to be lower \citep{neugart2013economic}.

\begin{figure}[H]\centering
	\includegraphics[width=0.9\linewidth]{paper/fertility_raw_regression_adjusted.pdf}
	\scriptsize
	\begin{minipage}{0.9 \linewidth}
		\caption{Daily number of births around the ML expansion}\label{fig_mlch: fertilitydistr}
		\emph{Notes:} The figure plots the number of births around the cutoff date May 01, 1979 for the ML expansion from two to six months after childbirth. Panel A shows the raw data, i.e. the actual number of births per day (unadjusted). Panel B, however, plots the difference between the raw and expected number of births when accounting for day of year, public holiday, and year$\times$day of week fixed effects. For the expected number of births I use data in the same time window (one month before and after the threshold) for the years 1977-1990, except for the year in which the reform took place. \newline\emph{Source:} Birth registry data from North Rhine-Westphalia and Baden-Württemberg. Taken together, both states account for almost 36\% of all births in the former Federal Republic of Germany in 1979.
		% {\color{red}how do I get the gray shading from the original figure in the bottom part?}
	\end{minipage}
\end{figure}

%Panel B
Panel B of Figure \ref{fig_mlch: fertilitydistr} shows the time series after removing any variation in the timing of births stemming from year, day of week, day of year, and public holidays. To do so, I estimate the following equation on all years, \emph{except} April and May 1979:
\begin{align}
\text{Births}_i = I^{\text{Year}}_i\times I^{\text{Day of Week}}_i + I^{\text{Day of Year}}_i + I^{\text{Public Holiday}}_i + \varepsilon_i \label{eq: validity_fig}
\end{align}
I regress the number of children born on day $i$, $\text{Births}_i$, on a series of dummies: year interacted with day of the week, day of the year, and a dummy for public national holidays.\footnote{For public holidays I use Good Friday, Holy Saturday, Easter Sunday, Easter Monday, Labor Day, Ascension Day, Whit Sunday, Whit Monday, and Corpus Christi.} Parameters are left out for better readability. Panel B plots the residuals $(\text{Births}-\widehat{\text{Births}})$ from the calibrated model. In contrast to \cite{gans2009born}, I do not observe systematically fewer births prior to the reform, or more births right after the policy change came into effect. These results suggest that there is no evidence that parents delayed births to after the cutoff date.


%--------------------------------------------
%VALIDITY: Birth rate effects
\begin{table}[t] \centering
	\begin{threeparttable} \centering 
	\caption{Birth rate effects of the 1979 ML reform}\label{tab_mlch: validity_birth_rate}
	{\def\sym#1{\ifmmode^{#1}\else\(^{#1}\)\fi} 
		\begin{tabular}{l*{5}{c}}
			\toprule
			& \multicolumn{4}{c}{Estimation window} \\
			\cmidrule{2-5}
			&\multicolumn{1}{c}{(1)}&\multicolumn{1}{c}{(2)}&\multicolumn{1}{c}{(3)}&\multicolumn{1}{c}{(4)}\\
			& \multicolumn{1}{c}{$\pm$7 days} & \multicolumn{1}{c}{$\pm$14 days} & \multicolumn{1}{c}{$\pm$21 days} & \multicolumn{1}{c}{$\pm$28 days}\\ 
			\midrule
			\multicolumn{5}{l}{\emph{Panel A. Dependent variable is number of births}}\\
			ML reform           &      -30.46         &      -30.23\sym{*}  &      -33.32\sym{**} &      -32.78\sym{***}\\
                    &     (30.31)         &     (17.73)         &     (14.08)         &     (12.37)         \\
Observations        &         196         &         392         &         588         &         784         \\
$R^2$               &       0.856         &       0.842         &       0.832         &       0.817         \\

			\\ \\
			\multicolumn{5}{l}{\emph{Panel B. Dependent variable is ln(number of births)}}\\
			ML reform           &     -0.0448         &     -0.0440\sym{*}  &     -0.0477\sym{**} &     -0.0476\sym{***}\\
                    &    (0.0425)         &    (0.0247)         &    (0.0197)         &    (0.0173)         \\
Observations        &         196         &         392         &         588         &         784         \\
$R^2$               &       0.855         &       0.844         &       0.833         &       0.819         \\

			\bottomrule
	\end{tabular}}
	\begin{tablenotes}
		\item \scriptsize \emph{Notes:} The table shows regression estimates of the impact of the 1979 ML reform on fertility. The sample comprises daily births within the relevant estimation window in the federal states of Baden-Württemberg and North Rhine-Westphalia from 1977-1990. Panel A uses the number of births as dependent variable, whereas panel B shows results with the log of the number of births as dependent variable. All specifications control for day of year, public holiday, and year$\times$day of week fixed effects. The estimation window is referring to the number of days before and after May 01. For instance, the $\pm$ 7 day window includes the last week in April and the first week in May across all years. Standard errors are reported in parentheses. \newline Significance levels: * p < 0.10, ** p < 0.05, *** p < 0.01. \newline\emph{Source:} Birth registry data from North Rhine-Westphalia and Baden-Württemberg. Taken together, both states account for almost 36\% of all births in the former Federal Republic of Germany in 1979.
	\end{tablenotes}
	\end{threeparttable}
\end{table}



Table \ref{tab_mlch: validity_birth_rate} presents estimates of the effect of the ML expansion on fertility outcomes, for different estimation windows.\footnote{I estimate: $f(\text{Births}_i) = I^{\text{Reform}}_i + I^{\text{Year}}_i\times I^{\text{Day of Week}}_i + I^{\text{Day of Year}}_i + I^{\text{Public Holiday}}_i + \varepsilon_i$, with the dummy $I^{\text{Reform}}_i$, which equals one for days after the cutoff in 1979. The results remain qualitatively the same if $I^{\text{Public Holiday}}_i$ is interacted with $I^{\text{Day of Week}}_i$.} Irrespective of whether the number of births or the log of the number of births is used, there is no evidence suggesting that parents deliberately postponed births until after the threshold date. On the contrary, the point estimates are negative throughout all specifications and become significantly different from zero with increasing estimation windows. Furthermore, the magnitude of the estimates is constant across estimation windows, which is a counterintuitive result if parents shifted births from the week prior to the reform to the week after the reform. If such behavior were present, one would expect point estimates to decrease in absolute value as the estimation window is enlarged. 





% Threats along distribution of birth months
\subsubsection{Threats Along the Whole Distribution of Birth Months}
\label{rev_mlch:threats_birth_months}
As discussed in section \ref{sec_mlch:empirical_strategy}, I cannot directly compare children born before and after the 1979 ML reform cutoff as birth months may affect outcomes directly \citep{buckles2013season,currie2013within}. Age-based cutoff rules at school entry are an important example in this respect \citep{black2011too}. As the data is only available on the birth month level, I exploit a DiD design with different estimation windows around the cutoff, with a maximum bandwidth of six months on both sides of the threshold. Under the assumption that the treatment and control cohort share the same seasonality effects, the DiD design can isolate the reform impact from seasonality and age of school entry effects. There is no formal test to investigate whether the assumption of time-invariance of seasonality is met. However, the parallel trends for pre-threshold-born children in treatment and control group (Figures \ref{fig_mlch: lc_hospital2_frg_DD} and \ref{fig_mlch: lc_d5_frg_DD}) provide evidence in favor of the assumption, suggesting that the cohort born in the year prior to the reform may be used as counterfactual.

The parallel trends cannot indicate whether post-threshold season of birth effects, such as discontinuities caused by school entry rules, may jeopardize the study design's validity. In West Germany, the cutoff date for school entry was June 30 and, consequently, falls into the post-threshold group of the 1979 ML reform \citep{juerges2011}.\footnote{The school entry rule dictates that children who turn six before to cutoff date, June 30, are admitted to primary school in that year. Children born after the threshold are admitted one year later.} If there were time-varying unobserved differences between children born around the school entry cutoff, the resulting DiD estimates would be biased. This may happen when treated children are affected differently by the school-based entry rules than control children. To check whether school entry rules or other forms of season of birth effects pose a threat to the identification, I run two robustness checks.
%\footnote{This would imply an interaction of the 1979 ML reform with the school entry cutoff.}

\textbf{Exclude School Entry Discontinuity.---}First, I limit the post-threshold group to children born before July (i.e. May and June). This way, school entry discontinuities are not part of the estimation. Figure \ref{fig_mlch: hospital2_school_cutoff} shows estimates with varying bandwidths for the pre-cutoff side by gender and age group. The first estimate per panel, shown in green, represents the baseline estimate as shown in Tables \ref{tab_mlch: DD_hopsital2_total} and \ref{tab_mlch: DD_hospital2_female_male}. The pre-threshold estimation window contains half a year before the cutoff (Nov-Apr) in the widest specification and shrinks to two months before the threshold (Mar-Apr). The overall conclusions from the results section remain unaffected. The largest effects are found for the oldest age bracket (32-35 years) and males. The DiD estimates are of similar magnitude and mostly significant across the estimation windows. The only difference to the main results is that the effects for the age bracket of 27-31 years disappear.

\textbf{Additional Control Cohort.---}Second, I show estimates when using the additional control group born two years before the reform (i.e. born around May 1977, see column 5 of Table \ref{tab_mlch: robustness_hospital}) for different bandwidths. If systematic month-of-birth patterns (including school entry discontinuities) varied over the cohorts, the estimates from this robustness check should differ from the baseline results. Figure \ref{fig_mlch: hospital2_addcg_bws_age-group_gender} presents estimates from this robustness check for various bandwidths. Once again, the estimates are of similar magnitude and significance compared to the baseline results. The reductions in hospitalizations are more pronounced in the older age brackets and for men. Related to this robustness check, it is worthwhile to point to the discussion on a potential violation of the \textit{stable unit treatment value assumption} due to spillover effects on older siblings (see section \ref{sec_mlch: robustness}). To refute this concern, I present estimates when using control cohorts from different birth years. The fact that the estimates have roughly the same magnitude irrespective of the control group does not support the idea that month-of-birth patterns are different through the years.

Overall, the additional robustness tests demonstrate that the main results are driven by the 1979 ML discontinuity and not by season of birth effects, such as school entry effects.




 




% figure: school_entry cutoffs different bandwidths, per gender and age-group
%\newgeometry{left=2.0cm,right=2.0cm,top=3cm,bottom=3cm} 
\begin{landscape}
	\vspace*{\fill}
	\begin{figure}[H]\centering
		\includegraphics[width=0.91\linewidth]{paper/hospital2_school_entry_small_bw_overview.pdf}
		\scriptsize
		\begin{minipage}{0.95\linewidth}
			\caption{Robustness: Account for School Cutoff Rules, by gender and age-group}\label{fig_mlch: hospital2_school_cutoff}
			\emph{Notes:} The figure displays DiD estimates (along with 95\% confidence intervals) of the 1979 ML reform on hospitalization. To check the robustness of the findings when accounting for school-entry rules, the post-threshold group is restricted to only include the months May and June. Each panel shows the effect of the reform when reducing the estimation window for the people born before the threshold (shrinking from half a year to two months). The first estimate per panel indicates the baseline estimate as shown in Table \ref{tab_mlch: DD_hopsital2_total} and \ref{tab_mlch: DD_hospital2_female_male}. It uses a bandwidth of half a year.
		\end{minipage}
	\end{figure}
	\vspace*{\fill}\clearpage
\end{landscape}
%\restoregeometry 




% figure: additional control group, different bandwidths for t,f,m
%\newgeometry{left=2.0cm,right=2.0cm,top=3cm,bottom=3cm} 
\begin{landscape}
	\vspace*{\fill}
	\begin{figure}[H]\centering
		\includegraphics[width=0.91\linewidth]{paper/hospital2_addcg_different_bws_overview.pdf}
		\scriptsize
		\begin{minipage}{0.95\linewidth}
			\caption{Robustness: Using the Additional Control Group, by gender and age-group}\label{fig_mlch: hospital2_addcg_bws_age-group_gender}
			\emph{Notes:} The Table shows DiD estimates (along with 95\% confidence intervals) of the 1979 ML reform on hospital admissions (pooled). To check the robustness of the findings when accounting for school-entry rules, it presents estimates when including the additional control group born two years before the reform (i.e. born around May 1977, see column 5 of Table \ref{tab_mlch: robustness_hospital}). The DiD estimates are reported for different estimation windows around the cutoff. The \textit{`Donut'} specification uses a bandwidth of half a year and excludes children born in April and May. The first estimate per panel indicates the baseline estimate as shown in Table \ref{tab_mlch: DD_hopsital2_total} and \ref{tab_mlch: DD_hospital2_female_male}. It uses a bandwidth of half a year.
		\end{minipage}
	\end{figure}
	\vspace*{\fill}\clearpage
\end{landscape}
%\restoregeometry 








%-------------------------------------------------------------------
%old robustness:
% Figure: School cutoff, only 2 months in post period Hospital (total) with age-groups
%\begin{figure}[H]\centering\caption{Robustness: Account for School Cutoff Rules for Hospitalization}\label{fig_mlch: school_cutoff_hospital}
%	\begin{subfigure}[h]{0.48\linewidth}\centering
%		\includegraphics[width=\linewidth]{paper/hospital2_school_entry_small_bw_pooled.pdf}
%	\end{subfigure}
%	
%	\begin{subfigure}[h]{0.48\linewidth}\centering
%		\includegraphics[width=\linewidth]{paper/hospital2_school_entry_small_bw_17-21.pdf}
%	\end{subfigure}
%	\begin{subfigure}[h]{0.48\linewidth}\centering
%		\includegraphics[width=\linewidth]{paper/hospital2_school_entry_small_bw_22-26.pdf}
%	\end{subfigure}
%	\begin{subfigure}[h]{0.48\linewidth}\centering
%		\includegraphics[width=\linewidth]{paper/hospital2_school_entry_small_bw_27-31.pdf}
%	\end{subfigure}
%	\begin{subfigure}[h]{0.48\linewidth}\centering
%		\includegraphics[width=\linewidth]{paper/hospital2_school_entry_small_bw_32-35.pdf}
%	\end{subfigure}
%	\scriptsize
%	\begin{minipage}{\linewidth}
%		\emph{Notes:} The figure displays DiD estimates of the 1979 ML reform on hospitalization. To check the robustness of the findings when accounting for school-entry rules, the post-threshold group is restricted to only include the months May and June. Each panel shows the effect of the reform when reducing the estimation window for the people born before the threshold (shrinking from half a year to two months). The first estimate indicates the baseline estimate as shown in Table \ref{tab_mlch: DD_hopsital2_total}.
%	\end{minipage}
%\end{figure}


%-------------------------------------------------------------------

%% Table: School cutoff, only 2 months in post period
%\begin{table}[t] \centering 
%	\begin{threeparttable} \centering \caption{Robustness: Account for School Cutoff Rules}\label{tab_mlch: DD_hospital2_total_school_cutoff}
%		{\def\sym#1{\ifmmode^{#1}\else\(^{#1}\)\fi} 
%			\begin{tabular}{l*{6}{c}}
%				\toprule 
%				%\multicolumn{5}{l}{Dependant variable: \textbf{Hospital admission (total)}}\\ \\ 
%				& \multicolumn{5}{c}{Estimation window pre-threshold} \\ 
%				\cmidrule(lr){2-6}
%				&\multicolumn{1}{c}{(1)}&\multicolumn{1}{c}{(2)}&\multicolumn{1}{c}{(3)}&\multicolumn{1}{c}{(4)}&\multicolumn{1}{c}{(5)}\\
%				&\multicolumn{1}{c}{Nov-Apr}&\multicolumn{1}{c}{Dec-Apr}&\multicolumn{1}{c}{Jan-Apr}&\multicolumn{1}{c}{Feb-Apr}&\multicolumn{1}{c}{Mar-Apr}\\
%				\midrule
%				\multicolumn{5}{l}{\emph{Panel A. Total}} \\
%				Overall             &      -0.892         &      -0.964         &      -1.210         &      -1.726         &      -0.181         \\
                    &     (1.037)         &     (1.138)         &     (1.289)         &     (1.520)         &     (1.476)         \\
 
%				\\ 
%				\multicolumn{5}{l}{\emph{Panel B. Women}} \\
%				Overall             &      -0.960         &      -0.852         &      -0.696         &      -0.959         &      0.0870         \\
                    &     (1.450)         &     (1.518)         &     (1.632)         &     (1.842)         &     (2.119)         \\
 
%				\\
%				\multicolumn{5}{l}{\emph{Panel C. Men}} \\
%				Overall             &      -0.839         &      -1.088         &      -1.706         &      -2.464         &      -0.440         \\
                    &     (0.909)         &     (1.062)         &     (1.211)         &     (1.464)         &     (0.906)         \\
 
%				\midrule
%				Observations & 304      &       266       &      228      &       190       &      152 \\
%				\bottomrule 
%		\end{tabular}}
%		\begin{tablenotes} 
%			\item \scriptsize \emph{Notes:} The table contains DiD estimates of the 1979 ML reform on hospitalization (pooled). To check the robustness of the findings when accounting for school-entry rules, the post-threshold group is restricted to only include the months May and June. Each panel shows the effect of the reform when reducing the estimation window for the people born before the threshold (shrinking from half a year to two months). \newline Significance levels: * p < 0.10, ** p < 0.05, *** p < 0.01.
%		\end{tablenotes} 
%	\end{threeparttable} 
%\end{table}

%-------------------------------------------------------------------

%% table: additional control group - pooled
%\begin{table}[H] \centering 
%	\begin{threeparttable} \centering \caption{Robustness: Using the Additional Control Group}\label{tab_mlch: hospital2_addcg_bws_pooled}
%		{\def\sym#1{\ifmmode^{#1}\else\(^{#1}\)\fi} 
%			\begin{tabular}{l*{6}{c}}
%				\toprule 
%				%\multicolumn{5}{l}{Dependant variable: \textbf{Hospital admission (total)}}\\ \\ 
%				& \multicolumn{5}{c}{Estimation window} \\ 
%				\cmidrule(lr){2-6}
%				&\multicolumn{1}{c}{(1)}&\multicolumn{1}{c}{(2)}&\multicolumn{1}{c}{(3)}&\multicolumn{1}{c}{(4)}&\multicolumn{1}{c}{(5)}\\
%				&\multicolumn{1}{c}{6M}&\multicolumn{1}{c}{5M}&\multicolumn{1}{c}{4M}&\multicolumn{1}{c}{3M}&\multicolumn{1}{c}{Donut}\\
%				\midrule
%				\multicolumn{6}{l}{\textit{Panel A. Total}}\\
%				total               &      -2.293\sym{**} &      -1.734         &      -1.708         &      -1.621         &      -3.346\sym{***}\\
                    &     (0.987)         &     (1.109)         &     (1.372)         &     (1.759)         &     (0.817)         \\
 
%				\\
%				\multicolumn{6}{l}{\textit{Panel B. Women}}\\
%				female              &      -1.559         &      -0.490         &       0.323         &       0.237         &      -2.470\sym{***}\\
                    &     (1.112)         &     (1.155)         &     (1.344)         &     (1.766)         &     (0.884)         \\

%				\\
%				\multicolumn{6}{l}{\textit{Panel C. Men}}\\
%				Overall             &      -3.007\sym{**} &      -2.925\sym{**} &      -3.637\sym{**} &      -3.390         &      -4.192\sym{***}\\
                    &     (1.135)         &     (1.332)         &     (1.596)         &     (1.978)         &     (1.118)         \\

%				\midrule
%				Observations &   672     &        560      &       448        &     336        &     560 \\
%				\bottomrule 
%		\end{tabular}}
%		\begin{tablenotes} 
%			\item \scriptsize \emph{Notes:} The Table shows DiD estimates of the 1979 ML reform on hospital admissions (pooled). To check the robustness of the findings when accounting for school-entry rules, it presents estimates when including the additional control group born two years before the reform (i.e. born around May 1977, see column 5 of Table \ref{tab_mlch: robustness_hospital}). The DiD estimates are reported for different estimation windows around the cutoff. The \textit{`Donut'} specification uses a bandwidth of half a year and excludes children born in April and May. \newline Significance levels: * p < 0.10, ** p < 0.05, *** p < 0.01.
%		\end{tablenotes} 
%	\end{threeparttable} 
%\end{table}

