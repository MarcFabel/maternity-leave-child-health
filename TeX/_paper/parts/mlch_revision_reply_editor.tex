
\clearpage
\section{Comments made by the editor}

%--------------------------------------------------------------------
% IDENTIFICATION
%--------------------------------------------------------------------
\subsection{Identification}

%-----------------------------------------
% Parallel Trends
\phantomsection
\addcontentsline{toc}{subsubsection}{Ed.1: Parallel Trends}
\begin{quote}
\textit{Ed.1: "One of the main problems of the paper (as reviewer 1 also raises) is that you do not end up running a regression discontinuity design, but a difference in differences. For the difference in differences to be credible, you need to provide evidence that the parallel trends assumption holds, which you do not. For those born in 1978-1979 prior to May 1st 1979, you should expect the differences in hospitalization rates 20 years later (relative to those born in the same months of 1977-1978) to be parallel. You should test that, ideally, using the same timeframes you use for the results."}
\end{quote}

\underline{Response:} I agree that evidence for the parallel trends assumption was missing. For this reason, I have extended the figures presenting the year-by-year trajectories of the differentials (`life-course') to also show the pre-threshold means of the treatment and control cohort. The figures show that hospitalization rates for the cohort born in 1978-1979 prior to the threshold in May are parallel to the hospitalization rates for the cohort born in the same month of 1977-1978. See Figure \ref{fig_mlch: lc_hospital2_frg_DD} for hospitalizations (page \pageref{fig_mlch: lc_hospital2_frg_DD}) and Figure \ref{fig_mlch: lc_d5_frg_DD} for MBDs (page \pageref{fig_mlch: lc_d5_frg_DD}). The text in the results section has been adjusted accordingly. Please go to page \pageref{rev_mlch: editor_parallel_trends_text} for changes affecting hospitalization and page \pageref{rev_mlch: editor_parallel_trends_text_d5} for changes concerning MBDs.



%positive side effect: better readability
%adjusted figures in life course approach such that the dependent mean is not shown anymore, the effect size can n


%-----------------------------------------
% Regression Discontinuity
\phantomsection
\addcontentsline{toc}{subsubsection}{Ed.2: Regression Discontinuity Designs}
\bigskip
\begin{quote}
	\textit{Ed.2: "You could also run a difference-in-regression discontinuity design for robustness, using kernel weighting, automatic bandwidth choice, and a polynomial on the running variable. Results would be substantially stronger if you were able to show similar estimates under both approaches (at least for the main specifications)."}
\end{quote}
\underline{Response:}
Your suggestion to run discontinuity designs for robustness is a great idea. Unfortunately, the data does not allow identification with any type of discontinuity approach as standard errors become too large. In order to exemplify this point, I have added Appendix Table \ref{tab_mlch: revision_RDD_hopsital2_total_app}, which contains estimates from a classic Regression Discontinuity Design (RDD). The estimates are described in Footnote \ref{rev_mlch: ed_rdd_empsection} (page \pageref{rev_mlch: ed_rdd_empsection}). Reassuringly, the direction and magnitude of the RDD estimates match the DiD estimates from Table \ref{tab_mlch: DD_hopsital2_total} very closely.

The lack of precision to identify discontinuities of local averages around the threshold is deeply rooted in the level of aggregation in the data set. If I had the birth date on the daily level, I could use an RDD with a small window around the cutoff. The corresponding RD plots shown in Figure \ref{fig: rf_hospital2_pooled_reply} visualize this well.







\begin{figure}
	[H]\centering
	\caption{Discontinuity plots for hospital admission (pooled)}\label{fig: rf_hospital2_pooled_reply}
	\begin{subfigure}[h]{0.48\linewidth}\centering\caption{Regression Discontinuity Design}
		\includegraphics[width=\linewidth]{paper/rd_hospital2_total_pooled.pdf}
	\end{subfigure}
	\begin{subfigure}[h]{0.48\linewidth}\centering\caption{Difference-in-Discontinuities}
		\includegraphics[width=\linewidth]{temp/diff_disc_hospital2.pdf}
	\end{subfigure}
	\scriptsize
	\begin{minipage}{0.95\linewidth}
		\emph{Notes:} %The figure plots the average number of diagnoses per 1,000 individuals for month-of-birth cohorts born half a year around the cut-off date of the 1979 maternity leave expansion. The monthly averages are taken over the entire sample length from 1995 to 2014. The dashed lines represent linear fitted values along with 90\%/95\% confidence intervals. The solid vertical red line divides pre- and post-reform schemes (two vs. six months of leave).\newline
		\emph{Source:} Hospital registry data for individuals born between November 1978 and October 1979.
	\end{minipage}
\end{figure}



The variation between the birth months is larger than the potential discontinuity itself. Because of the birth date's availability on the monthly level and the resulting precision problems associated with discontinuity designs, I have to exploit the DiD setup explained in the manuscript. The same reasoning also applies to a difference-in-discontinuities approach (see Panel B of Figure \ref{fig: rf_hospital2_pooled_reply}). 






To not entertain wrong expectations with the reader, I have reformulated the statement that the identification is build on a \textit{`combination of a DiD and an RDD'} (see page \pageref{rev_mlch: r1_rdd+did_intro} in the introduction and page \pageref{rev_mlch: r1_rdd+did_em_section} in the section on the empirical strategy). Now, I label the empirical approach as a difference-in-differences design. 










%-----------------------------------------
% SUTVA
\phantomsection
\bigskip
\addcontentsline{toc}{subsubsection}{Ed.3: Potential Violation of SUTVA}
\begin{quote}
	\textit{Ed.3: "Extended maternal leave could also have affected older siblings in the family, in particular those close in age to the newborn (by making the mother more available, improving her health or increasing household income). If this were the case, SUTVA would not hold and your effects would represent a lower bound. You conduct a robustness analysis using the cohort just below the one in the core control group. You could try with several cohorts older to avoid this problem."}
\end{quote}
\underline{Response:}
I fully agree that the paper has not sufficiently covered a potential violation of the \textit{stable unit treatment value assumption} (SUTVA). To address potential spillover effects on older siblings, I have added a paragraph in the robustness section after discussing the robustness check using the cohort in the core control group (page \pageref{rev_mlch: editor_SUTVA_older_cohorts}). In the paragraph I discuss results when exchanging the core control group with other older cohorts (see Appendix Figure \ref{fig_mlch: SUTVA_older_controls_hospital2}, page \pageref{fig_mlch: SUTVA_older_controls_hospital2}). As the estimates remain stable in terms of magnitude and significance, I do expect SUTVA to hold.



%-----------------------------------------
% Interactions
\phantomsection
\bigskip
\addcontentsline{toc}{subsubsection}{Ed.4: Interactions with Age Brackets}
\begin{quote}
	\textit{Ed.4: "It is not quite clear whether you have a different regression for each age-bracket (I tend to believe this is the case) or if you include (Treat x After x age-range) interactions in the same regression. This latter approach would address correlation of the errors over time for a given month of birth cohort. In a robustness check, I would like to see results of such a regression, and how they compare with the results you present when running individual regressions."}
\end{quote}
\underline{Response:}
So far, I have used different regressions for each age bracket. Due to the fact that this has not been clear, I have added a phrase in the results section (page \pageref{rev_mlch: editor_comprehension_interaction}) to make it explicit. I have incorporated your comment (the interactions with the age brackets) by adding Table \ref{tab_mlch: interaction_TxA_agegroups_hospital2} (page \pageref{tab_mlch: interaction_TxA_agegroups_hospital2}) in the Appendix and discussing the findings in Footnote \ref{rev_mlch: editor_interaction_TxA_agegroups} (page \pageref{rev_mlch: editor_interaction_TxA_agegroups}).






%-----------------------------------------
% not affected by school entry and others
\phantomsection
\bigskip
\addcontentsline{toc}{subsubsection}{Ed.5: Expansion of the Appendix}
\begin{quote}
	\textit{Ed.5: "You should also provide some evidence that your treated cohort was not affected differentially than the control cohort throughout the years by other policies (school-based or health care-based) that could have affected health outcomes during young adulthood"}
\end{quote}
\underline{Response:} As suggested by reviewer 1, I have expanded the appendix such that it consists of two parts. The first part describes the threats at the threshold (self-selection due to postponed delivery or strategic conception). I have added the second part (section V.2 on page \pageref{rev_mlch:threats_birth_months}), which discusses threats along the distribution of birth months (with an emphasis on discontinuities caused by age-based cutoff rules at school entry). The section contains a theoretical explanation of which threats are encountered in this context, a brief discussion on the parallel trends for the pre-threshold cohorts, and covers the two robustness checks suggested by the reviewer (see Figures \ref{fig_mlch: hospital2_school_cutoff} and \ref{fig_mlch: hospital2_addcg_bws_age-group_gender}). 







%--------------------------------------------------------------------
% PAPER STRUCTURE
%--------------------------------------------------------------------

\subsection{Paper structure and other features}


% Formulation selective fertility
\phantomsection
\bigskip
\addcontentsline{toc}{subsubsection}{Ed.6: Reformulation in the Empirical Strategy Section}
\begin{quote}
	\textit{Ed.6: "In the last paragraph in section 5, “An examination of the fertility…”, you refer to selection fertility as a possible threat, but the real threat is selective delivery. In the appendix you clarify that, but you should make that clearer in that paragraph."}
\end{quote}
\underline{Response:}
I agree that the formulation was not unambiguous. In section \ref{sec_mlch:empirical_strategy} (page \pageref{rev_mlch: editor_selective_delivery}), I have changed the formulation to include both, changes in fertility (strategic conception) and delivery (postponing induced births and C-sections). 




% Restructure of Discussion
\phantomsection
\bigskip
\addcontentsline{toc}{subsubsection}{Ed.7: Restructure Discussion}
\begin{quote}
	\textit{Ed.7: "Last paragraph in subsection 6.3 “Why are the effects of the 1979…”. I would shift this to the discussion of mechanisms section at the end. I would also shift the discussion of long term mechanisms in section 3 to this last section (as suggested by Reviewer 2)."}
\end{quote}
\underline{Response:} I have split up the discussion into two parts to have two independent topics. The first part (page \pageref{rev_mlch: restructure_discussion_framework}, formerly section 3) presents the conceptual framework of how changes in the mediating outcomes affect the children. The second part (page \pageref{ref_mlch: discussion_mechanisms}, formerly section 7) covers mechanisms: suggestive evidence for maternal time using the urban/rural sample split and the theoretic discussion about other mechanisms. I have accommodated the potential explanation of why the 1979 ML reform effects are so strong for MBDs in the subsection on changes in maternal health outcomes.


% expand literature
\phantomsection
\bigskip
\addcontentsline{toc}{subsubsection}{Ed.8: Expansion of the Literature}
\begin{quote}
	\textit{Ed.8: "When discussing mechanisms, please refer to the findings of the other papers that have already studied maternity leave policies with respect to potential mediating outcomes. The literature on the effects of income on children’s health is much broader, and you could try expanding it a little. Same with the literature on parental attachment, parental investment and children’s outcomes."}
\end{quote}
\underline{Response:} 

When discussing mechanisms, I cite papers that investigate the impact of ML expansions on mediating outcomes. For instance, concerning household income, I have added \cite{carneiro2015flying}, \cite{Dahl2016Case}, and \cite{butikofer2018impact} (page \pageref{rev_mlch: r2_maternal_income_missing_link}). Furthermore, I have expanded the literature on household income and parental investment (page \pageref{rev_mlch: ed_literature_income_childhealth}) by the studies of \cite{milligan2011taxbenefits}, \cite{aizer2016cash-transfer}, and \cite{akee2018income}. Moreover, I have added the studies of \cite{baker2016} and \cite{bharadwaj2013discrimination} to report on sex differences in parental investments, which may explain the heterogeneity of the results with respect to gender. Parental attachment/bonding is covered with the explanation of the strong effects for MBDs. Please also see my response to the previous comment (ED.7).

%somewhat related: \cite{cesarini2016lottery}



% missing study in the intro
\phantomsection
\bigskip
\addcontentsline{toc}{subsubsection}{Ed.9: Missing Study in the Intro}
\begin{quote}
	\textit{Ed.9: "You don’t cite Beuchert et al (2016) in the introduction when describing papers that studied the effects of maternal leave on children’s health. Unlike the other papers, Beuchert et al. do not seem to find evidence that maternal leave extensions increase children’s health."}
\end{quote}
\underline{Response:} So far, I have been citing \cite{beuchert2016} in the discussion about mechanisms (paragraph on maternal health outcomes). Yet, I forgot to include their null results in the introduction. In order to account for this issue, I have included the information on \cite{beuchert2016} in footnote \ref{rev_mlch: editor_beuchert_omission} (page \pageref{rev_mlch: editor_beuchert_omission}). Thank you for spotting the omission.
