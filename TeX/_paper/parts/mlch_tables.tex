
%--------------------------------------------
% Overview ICD 9 and ICD 10 
%\begin{small}
%\vspace*{\fill}
%\begin{table}[h] % table environment for caption and label
%	\begin{threeparttable}
%		\centering % center the tabular
%		\caption{Overview of diagnoses} % caption
%		\label{tab_mlch:outcomes_coding_main_chapters} 
%		\begin{tabular}{lrrr} % alignment and number of columns of actual table
%			\toprule % top thicker horizontal line (" rule ")
%			&\multicolumn{1}{r}{(1)}& &\multicolumn{1}{r}{(2)}\\
%			&\multicolumn{1}{r}{ICD-9} & & \multicolumn{1}{r}{ICD-10} \\ 
%			\midrule
%			%-------------------------------------------------------------------------
%			\textit{Main diagnosis chapters}\\
 \hspace{4pt} Infectious and parasitic diseases                           	&	001-139		& &		A00-B99 \\
 \hspace{4pt} Neoplasms                                                   	&	140-239		& &		C00-D48 \\
%\hspace{4pt} Diseases of the blood and blood-forming organs              	&	280-289		& &		D50-D90 \\
 \hspace{4pt} Endocrine, nutritional and metabolic diseases					&	240-278		& &		E00-E90 \\
 \hspace{4pt} Mental \& behavioral  disorders                             	&	290-319		& &		F00-F99 \\
 \hspace{4pt} Diseases of the nervous system                              	&	320-359		& &		G00-G99 \\
 \hspace{4pt} Diseases of the sense organs                                	&	360-389		& &		H00-H95 \\
 \hspace{4pt} Diseases of the circulatory system                          	&	390-459		& &		I00-I99 \\
 \hspace{4pt} Diseases of the respiratory system                          	&	460-519		& &		J00-J99 \\
 \hspace{4pt} Diseases of the digestive system                            	&	520-579		& &		K00-K93 \\
 \hspace{4pt} Diseases of the skin and subcutaneous tissue                	&	680-709		& &		L00-L99 \\
 \hspace{4pt} Diseases of the musculoskeletal system and connective tissue	&	710-739		& &		M00-M99 \\
 \hspace{4pt} Diseases of the genitourinary system                        	&	580-629		& &		N00-N99 \\
 \hspace{4pt} Complications of pregnancy, childbirth, and the puerperium  	&	630-676		& &		O00-O99 \\
%\hspace{4pt} Certain conditions originating in the perinatal period      	&	760-779		& &		P00-P96 \\
%\hspace{4pt} Congenital anomalies                                        	&	740-759		& &		Q00-Q99 \\
 \hspace{4pt} Symptoms, signs, and ill-defined conditions                 	&	780-799		& &		R00-R99 \\
 \hspace{4pt} Injury and poisoning                                        	&	800-999		& &		S00-T98 \\
 \\
 \textit{}

\
%			%-------------------------------------------------------------------------
%			\bottomrule % bottom thicker horizontal line (" rule ")
%		\end{tabular}
%		\begin{tablenotes}
%			\scriptsize{ \item \textit{Notes:} The table shows the classification of diseases according to the "International Statistical Classification of Diseases and Related Health Problems (ICD)", a medical classification list provided by the World Health Organization. \newline \textit{Source:} World Health Organization (WHO), see for example: \href{http://www.who.int/classifications/icd/en/}{http://www.who.int/classifications/icd/en/}\newline$^1$ Psychoactive substances include alcohol, opioids, cannabinoids, sedatives or hypnotics, cocaine, other stimulants (including caffeine), hallucinogens, tobacco, volatile solvents, multiple drug use and use of other psychoactive substances. }
%		\end{tablenotes}
%	\end{threeparttable}
%\end{table}
%\vspace*{\fill}\clearpage 
%%\end{small}
%%\normalsize



%--------------------------------------------
% HOSPITAL - Difference in Difference tables for 
\newpage

\begin{table}[t] \centering 
 \begin{threeparttable} \centering \caption{ITT effects on hospital admission}\label{tab_mlch: DD_hopsital2_total}
  {\def\sym#1{\ifmmode^{#1}\else\(^{#1}\)\fi} 
 	\begin{tabular}{l*{6}{c}}
 		\toprule 
 		%\multicolumn{5}{l}{Dependant variable: \textbf{Hospital admission (total)}}\\ \\ 
 		& \multicolumn{5}{c}{Estimation window} \\ 
 		\cmidrule(lr){2-6}
 		&\multicolumn{1}{c}{(1)}&\multicolumn{1}{c}{(2)}&\multicolumn{1}{c}{(3)}&\multicolumn{1}{c}{(4)}&\multicolumn{1}{c}{(5)}\\
 		&\multicolumn{1}{c}{6M}&\multicolumn{1}{c}{5M}&\multicolumn{1}{c}{4M}&\multicolumn{1}{c}{3M}&\multicolumn{1}{c}{Donut}\\
 		\midrule
 		\multicolumn{5}{l}{\emph{Panel A. Over entire length of the life-course}} \\
 		\hspace*{10pt}Overall&      -2.218         &      -2.181\sym{*}  &      -1.944\sym{**} &      -2.168\sym{**} &      -2.713\sym{***}\\
                    &     (1.401)         &     (1.143)         &     (0.917)         &     (0.782)         &     (0.826)         \\
\midrule Dependent mean&       122.7         &       121.0         &       120.5         &       120.6         &       121.4         \\
Effect in SDs [\%]  &       19.22         &       18.84         &       17.35         &       19.78         &       24.62         \\
Observations        &         240         &         320         &         400         &         480         &         400         \\
 \\ \\
 		\multicolumn{5}{l}{\emph{Panel B. Age brackets}} \\
 		\hspace*{10pt}Age 17-21&      -1.095         &      -0.735         &      -0.590         &      -1.517\sym{+}  &      -1.963\sym{**} \\
                    &     (1.603)         &     (1.198)         &     (0.995)         &     (0.946)         &     (0.931)         \\
 \hspace*{10pt}Age 22-26&      -0.735         &      -0.667         &      -0.613         &      -0.611         &      -1.080         \\
                    &     (1.672)         &     (1.412)         &     (1.113)         &     (0.937)         &     (1.012)         \\
 \hspace*{10pt}Age 27-31&      -2.546\sym{*}  &      -3.209\sym{**} &      -3.015\sym{***}&      -2.665\sym{***}&      -2.974\sym{***}\\
                    &     (1.375)         &     (1.132)         &     (0.909)         &     (0.826)         &     (0.949)         \\
 \hspace*{10pt}Age 32-35&      -3.869\sym{***}&      -3.619\sym{**} &      -4.572\sym{***}&      -5.045\sym{**} &      -4.717\sym{***}\\
                    &     (1.083)         &     (1.277)         &     (1.460)         &     (1.721)         &     (1.191)         \\
 
 		\bottomrule 
 	\end{tabular}}
 	\begin{tablenotes} 
 		\item \scriptsize \emph{Notes:} The table shows DiD estimates of the 1979 maternity leave reform on hospital admission for different estimation windows around the cutoff. The \textit{`Donut'} specification uses a bandwidth of half a year and excludes children born in April and May. Panel A shows the effect for the entire pooled time frame and panel B reports estimates per age bracket. The outcome variables are defined as the number of cases per thousand individuals. All regressions control for year and month-of-birth fixed effects. The control group is comprised of children that are born in the same months but one year before the reform (i.e. children born between November 1977 and October 1978). In order to compare the two birth cohorts at the same age, I shift the control cohort from wave $t$ to wave $t+1$. The dependent mean and the effect size in standard deviation units correspond to pre-reform values of the treated group. Table \ref{tab_mlch: observations_age_brackets} contains the number of observations for the estimations per age bracket. Clustered standard errors are reported in parentheses. \newline Significance levels: * p < 0.10, ** p < 0.05, *** p < 0.01. \newline 	%\emph{Source:} Hospital registry data.
 	\end{tablenotes} 
 \end{threeparttable} 
 \end{table}
% \vspace*{\fill}
 \begin{table}[H] \centering 
 	\begin{threeparttable} \centering \caption{ITT effects on \textbf{hospital admission (women)}}\label{tab: DD_hopsital2_female} {\def\sym#1{\ifmmode^{#1}\else\(^{#1}\)\fi} 
 			\begin{tabular}{l*{6}{c}}
 				\toprule 
 				%\multicolumn{5}{l}{Dependant variable: \textbf{Hospital admission (total)}}\\ \\ 
 				& \multicolumn{5}{c}{Estimation window} \\ 
 				\cmidrule(lr){2-6}
 				&\multicolumn{1}{c}{(1)}&\multicolumn{1}{c}{(2)}&\multicolumn{1}{c}{(3)}&\multicolumn{1}{c}{(4)}&\multicolumn{1}{c}{(5)}\\
 				&\multicolumn{1}{c}{6M}&\multicolumn{1}{c}{5M}&\multicolumn{1}{c}{4M}&\multicolumn{1}{c}{3M}&\multicolumn{1}{c}{Donut}\\
 				\midrule
 				\multicolumn{5}{l}{\emph{Panel A.Over entire length of the life-course}} \\
 				\hspace*{10pt}Overall&       0.149         &      -0.766         &      -1.815\sym{**} &      -2.278\sym{***}\\
                    &     (2.038)         &     (1.088)         &     (0.807)         &     (0.751)         \\
\midrule Dependent mean&       122.3         &       122.0         &       122.4         &       123.3         \\
Effect in SDs [\%]  &       1.350         &       6.530         &       16.29         &       20.28         \\
Observations        &         160         &         320         &         480         &         400         \\
 \\ \\
 				\multicolumn{5}{l}{\emph{Panel B. Age brackets}} \\
 				\hspace*{10pt}Age 17-21&      -2.121\sym{+}  &      -1.274         &      -1.931\sym{*}  &      -2.916\sym{***}&      -3.322\sym{***}\\
                    &     (1.269)         &     (1.018)         &     (0.959)         &     (0.935)         &     (0.985)         \\
 \hspace*{10pt}Age 22-26&       1.661         &       1.126         &      0.0274         &      -0.510         \\
                    &     (3.675)         &     (1.806)         &     (1.267)         &     (1.117)         \\
 \hspace*{10pt}Age 27-31&      -1.379         &      -1.669         &      -2.605\sym{**} &      -2.762\sym{**} &      -2.944\sym{***}\\
                    &     (1.765)         &     (1.336)         &     (1.163)         &     (1.004)         &     (0.917)         \\
 \hspace*{10pt}Age 32-35&      -0.605         &      -1.004         &      -0.841         &      -1.212         &      -1.810\sym{*}  \\
                    &     (1.300)         &     (1.165)         &     (1.024)         &     (0.866)         &     (0.941)         \\
 
 				\bottomrule 
 		\end{tabular}}
 		\begin{tablenotes} 
 			\item \scriptsize \emph{Notes:} The table shows DiD estimates of the 1979 maternity leave reform on hospital admission for different estimation windows around the cutoff. The \textit{`Donut'} specification uses a bandwidth of half a year and excludes children born in April and May. Panel A shows the effect for the entire pooled time frame and panel B breaks the life-course up in age brackets. The outcome variables are defined as the number of cases per thousand individuals (births). All regressions control for year and month-of-birth fixed effects. The control group is comprised of children that are born in the same months but one year before the reform (i.e. children born between November 1977 and October 1978). In order to compare the two birth cohorts at the same age, I shift the control cohort from wave $t$ to wave $t+1$. The dependent mean and the effect size in standard deviation units correspond to pre-reform values of the treated group. Clustered standard errors are reported in parentheses. \newline Significance levels: + p < 0.15, * p < 0.10, ** p < 0.05, *** p < 0.01. \newline 	\emph{Source:} Hospital registry data.
 		\end{tablenotes} 
 	\end{threeparttable} 
 \end{table}
\vspace*{\fill}\clearpage 
%  \vspace*{\fill}
 \begin{table}[H] \centering 
 	\begin{threeparttable} \centering \caption{ITT effects on \textbf{hospital admission (men)}}\label{tab: DD_hopsital2_male} {\def\sym#1{\ifmmode^{#1}\else\(^{#1}\)\fi} 
 		\begin{tabular}{l*{6}{c}}
 			\toprule 
 			%\multicolumn{5}{l}{Dependant variable: \textbf{Hospital admission (total)}}\\ \\ 
 			& \multicolumn{5}{c}{Estimation window} \\ 
 			\cmidrule(lr){2-6}
 			&\multicolumn{1}{c}{(1)}&\multicolumn{1}{c}{(2)}&\multicolumn{1}{c}{(3)}&\multicolumn{1}{c}{(4)}&\multicolumn{1}{c}{(5)}\\
 			&\multicolumn{1}{c}{6M}&\multicolumn{1}{c}{5M}&\multicolumn{1}{c}{4M}&\multicolumn{1}{c}{3M}&\multicolumn{1}{c}{Donut}\\
 				\midrule
 				\multicolumn{5}{l}{\emph{Panel A. Over entire length of the life-course}} \\
 				\hspace*{10pt}Overall&      -0.445         &      -3.528\sym{**} &      -2.525\sym{**} &      -3.148\sym{**} \\
                    &     (1.017)         &     (1.387)         &     (0.997)         &     (1.144)         \\
\midrule Dependent mean&       118.3         &       120.0         &       118.9         &       119.7         \\
Effect in SDs [\%]  &       3.440         &       25.82         &       19.34         &       23.95         \\
Observations        &         160         &         320         &         480         &         400         \\
 \\ \\
 				\multicolumn{5}{l}{\emph{Panel B. Age brackets}} \\
 				\hspace*{10pt}Age 17-21&      -0.157         &      -0.246         &       0.634         &      -0.273         &      -0.757         \\
                    &     (2.147)         &     (1.592)         &     (1.344)         &     (1.201)         &     (1.241)         \\
 \hspace*{10pt}Age 22-26&       0.918         &      -2.373         &      -1.230         &      -1.633         \\
                    &     (0.719)         &     (1.497)         &     (1.048)         &     (1.241)         \\
 \hspace*{10pt}Age 27-31&      -0.652\sym{***}&      -4.669\sym{**} &      -2.558\sym{*}  &      -2.987\sym{*}  \\
                    &     (0.106)         &     (1.625)         &     (1.294)         &     (1.528)         \\
 \hspace*{10pt}Age 32-35&      -5.852\sym{**} &      -7.955\sym{***}&      -6.373\sym{***}&      -7.461\sym{***}\\
                    &     (2.312)         &     (1.969)         &     (1.526)         &     (1.722)         \\
 
 				\bottomrule 
 		\end{tabular}}
 		\begin{tablenotes} 
 			\item \scriptsize \emph{Notes:} The table shows DD estimates of the 1979 maternity leave reform on hospital admission for different bandwidths around the cutoff. Panel A shows the effect for the entire pooled time frame and panel B breaks the life-course up in age brackets. The outcome variables are defined as the number of cases per thousand individuals (births). All regressions control for year and month-of-birth fixed effects. The control group is comprised of children that are born in the same months but one year before the reform (i.e. children born between November 1977 and October 1978). In order to compare the two birth cohorts at the same age, we shift the control cohort from wave $t$ to wave $t+1$. The dependent mean and the effect size in standard deviation units correspond to pre-reform values of the treated group. Clustered standard errors are reported in parentheses. \newline Significance levels: + p < 0.15, * p < 0.10, ** p < 0.05, *** p < 0.01. \newline 	\emph{Source:} Hospital registry data.
 		\end{tablenotes} 
 	\end{threeparttable} 
 \end{table} 
\vspace*{\fill}\clearpage 


\newgeometry{left=1cm,right=1cm,top=1cm,bottom=2.5cm} 
\begin{landscape}
\vspace*{\fill}
 \begin{table}[H] \centering 
 	\begin{threeparttable} \centering \caption{ITT effects on \textbf{hospital admission, by gender}}\label{tab_mlch: DD_hospital2_female_male} {\def\sym#1{\ifmmode^{#1}\else\(^{#1}\)\fi} 
 			\begin{tabular}{l*{12}{c}}
 				\toprule 
 				% \multicolumn{5}{l}{Dependant variable: \textbf{Hospital admission (total)}}\\ \\ 
 				& \multicolumn{5}{c}{Women} && \multicolumn{5}{c}{Men} \\ 
 				\cmidrule(lr){2-6} \cmidrule(lr){8-12}
 				&\multicolumn{1}{c}{(1)}&\multicolumn{1}{c}{(2)}&\multicolumn{1}{c}{(3)}&\multicolumn{1}{c}{(4)}&\multicolumn{1}{c}{(5)}&\multicolumn{1}{c}{        }&\multicolumn{1}{c}{(6)}&\multicolumn{1}{c}{(7)}&\multicolumn{1}{c}{(8)}&\multicolumn{1}{c}{(9)}&\multicolumn{1}{c}{(10)}\\
 				&\multicolumn{1}{c}{6M}&\multicolumn{1}{c}{5M}&\multicolumn{1}{c}{4M}&\multicolumn{1}{c}{3M}&\multicolumn{1}{c}{Donut}&&\multicolumn{1}{c}{6M}&\multicolumn{1}{c}{5M}&\multicolumn{1}{c}{4M}&\multicolumn{1}{c}{3M}&\multicolumn{1}{c}{Donut}\\
 				\midrule
 				\multicolumn{5}{l}{\emph{Panel A.Over entire length of the life-course}} \\

 				\hspace*{10pt}Overall		&      -1.742\sym{**} &      -1.224         &      -0.689         &      -0.862         &      -2.164\sym{***} &&      -2.410\sym{**} &      -2.502\sym{*}  &      -3.593\sym{**} &      -3.506\sym{**} &      -2.986\sym{**} \\
				                    		&     (0.816)         &     (0.924)         &     (1.117)         &     (1.504)         &     (0.718)          &&     (1.015)         &     (1.204)         &     (1.373)         &     (1.568)         &     (1.178)         \\
				\midrule Dependent mean		&       122.3         &       121.9         &       121.9         &       123.8         &       123.2          &&       120.0         &       120.2         &       121.2         &       122.7         &       120.7         \\
				Effect in SDs [\%]  		&       15.30         &       10.61         &       5.750         &       7.350         &       18.84          &&       19.31         &       19.68         &       27.68         &       26.59         &       23.72         \\
				\(N\) (MOB $\times$ year)	&         456         &         380         &         304         &         228         &         380          &&         456         &         380         &         304         &         228         &         380         \\
 				\\

 				\multicolumn{5}{l}{\emph{Panel B. Age brackets}} \\
 				\hspace*{10pt}Age 17-21	&      -2.916\sym{***}&      -1.931\sym{*}  &      -1.274         &      -2.121		    &      -3.322\sym{***} &&      -0.273         &       0.634         &      -0.246         &      -0.157         &      -0.757         \\
				                    	&     (0.935)         &     (0.959)         &     (1.018)         &     (1.269)         &     (0.985)          &&     (1.201)         &     (1.344)         &     (1.592)         &     (2.147)         &     (1.241)         \\
				\hspace*{10pt}Age 22-26	&      0.0274         &       0.557         &       1.126         &       0.707         &      -0.510          &&      -1.230         &      -1.738         &      -2.373         &      -2.113         &      -1.633         \\
				                    	&     (1.267)         &     (1.461)         &     (1.806)         &     (2.395)         &     (1.117)          &&     (1.048)         &     (1.226)         &     (1.497)         &     (1.519)         &     (1.241)         \\
				\hspace*{10pt}Age 27-31	&      -2.762\sym{**} &      -2.605\sym{**} &      -1.669         &      -1.379         &      -2.944\sym{***} &&      -2.558\sym{*}  &      -3.408\sym{**} &      -4.669\sym{**} &      -3.650\sym{**} &      -2.987\sym{*}  \\
				                    	&     (1.004)         &     (1.163)         &     (1.336)         &     (1.765)         &     (0.917)          &&     (1.294)         &     (1.433)         &     (1.625)         &     (1.467)         &     (1.528)         \\
				\hspace*{10pt}Age 32-35	&      -1.212         &      -0.841         &      -1.004         &      -0.605         &      -1.810\sym{*}   &&      -6.373\sym{***}&      -6.244\sym{***}&      -7.955\sym{***}&      -9.253\sym{***}&      -7.461\sym{***}\\
				                   		&     (0.866)         &     (1.024)         &     (1.165)         &     (1.300)         &     (0.941)          &&     (1.526)         &     (1.781)         &     (1.969)         &     (2.318)         &     (1.722)         \\
 				\bottomrule 
 		\end{tabular}}
 		\begin{tablenotes} 
 			\item \scriptsize \emph{Notes:} The table shows DiD estimates of the 1979 maternity leave reform on hospital admission by gender. The \textit{`Donut'} specification uses a bandwidth of half a year and excludes children born in April and May. Panel A shows the effect for the entire pooled time frame and panel B breaks the life-course up in age brackets. The outcome variables are defined as the number of cases per thousand individuals. All regressions control for year and month-of-birth fixed effects. The control group is comprised of children that are born in the same months but one year before the reform (i.e. children born between November 1977 and October 1978). In order to compare the two birth cohorts at the same age, I shift the control cohort from wave $t$ to wave $t+1$. The dependent mean and the effect size in standard deviation units correspond to pre-reform values of the treated group. \revision{Table \ref{tab_mlch: observations_age_brackets} contains the number of observations for the estimations per age bracket.} Clustered standard errors are reported in parentheses. \newline Significance levels: * p < 0.10, ** p < 0.05, *** p < 0.01. \newline 	%\emph{Source:} Hospital registry data.
 		\end{tablenotes} 
 	\end{threeparttable} 
 \end{table}
\vspace*{\fill}\clearpage 
\end{landscape}

% 

\restoregeometry





%--------------------------------------------
% ITT effects nach chaptern
% original 3 3 2 3
\newpage
\newgeometry{left=3cm,right=3cm,top=0.5cm,bottom=2.0cm} 
\vspace*{\fill}
\begin{table}[H] \centering 
	\begin{threeparttable} \centering \caption{ITT effects on \textbf{hospital admission and main diagnoses chapters}}\label{tab_mlch: ITT_across_chapters_per_age_group_total}
		{\def\sym#1{\ifmmode^{#1}\else\(^{#1}\)\fi} 
			\begin{tabular}{l*{5}{c}}
				\toprule 
				&\multicolumn{1}{c}{(1)}&\multicolumn{1}{c}{(2)}&\multicolumn{1}{c}{(3)}&\multicolumn{1}{c}{(4)}&\multicolumn{1}{c}{(5)}\\
				\midrule
				&\multirow{2}{*}{Overall} & \multicolumn{4}{c}{Age brackets [years]} \\ 
				\cmidrule(lr){3-6}
				&&\multicolumn{1}{c}{17-21}&\multicolumn{1}{c}{22-26}&\multicolumn{1}{c}{27-31}&\multicolumn{1}{c}{32-35}\\
				
				\midrule
				
				Hospital            &      -2.168\sym{**} &      -1.517         &      -0.611         &      -2.665\sym{***}&      -3.869\sym{***}\\
                    &     (0.782)         &     (0.946)         &     (0.937)         &     (0.826)         &     (1.083)         \\
IPD                 &     -0.0838\sym{**} &      0.0130         &      -0.162         &      -0.126\sym{**} &     -0.0197         \\
                    &    (0.0334)         &    (0.0680)         &     (0.108)         &    (0.0548)         &     (0.111)         \\
Neo                 &      0.0269         &      -0.198         &       0.336\sym{**} &       0.121         &      -0.217         \\
                    &    (0.0821)         &     (0.155)         &     (0.139)         &     (0.102)         &     (0.164)         \\
MBD                 &      -0.634\sym{**} &       0.174         &    -0.00769         &      -1.000\sym{**} &      -1.906\sym{***}\\
                    &     (0.249)         &     (0.263)         &     (0.420)         &     (0.357)         &     (0.372)         \\
Ner                 &     0.00791         &     -0.0919         &       0.238\sym{***}&      0.0374         &     -0.0190         \\
                    &    (0.0561)         &    (0.0558)         &    (0.0751)         &    (0.0963)         &     (0.126)         \\
Sen                 &      -0.144\sym{***}&      -0.168\sym{**} &     -0.0990\sym{*}  &      -0.172\sym{**} &      -0.261\sym{**} \\
                    &    (0.0272)         &    (0.0626)         &    (0.0518)         &    (0.0640)         &    (0.0998)         \\
Cir                 &     -0.0453         &     -0.0466         &      0.0311         &      -0.199\sym{*}  &      -0.198         \\
                    &    (0.0678)         &    (0.0782)         &    (0.0812)         &    (0.0994)         &     (0.162)         \\
Res                 &      -0.287\sym{***}&      -0.369\sym{**} &      -0.199\sym{***}&      -0.273\sym{**} &     -0.0842         \\
                    &    (0.0787)         &     (0.173)         &    (0.0694)         &     (0.107)         &     (0.167)         \\
Dig                 &      -0.395\sym{***}&      -0.421\sym{**} &      -0.485\sym{*}  &      -0.376         &      -0.494         \\
                    &     (0.131)         &     (0.177)         &     (0.257)         &     (0.227)         &     (0.361)         \\
SST                 &      0.0936\sym{**} &      0.0567         &       0.186\sym{**} &       0.152\sym{*}  &     -0.0149         \\
                    &    (0.0422)         &    (0.0947)         &    (0.0721)         &    (0.0882)         &     (0.120)         \\
Mus                 &     -0.0356         &     -0.0260         &      -0.155         &     -0.0562         &       0.152         \\
                    &    (0.0556)         &    (0.0884)         &     (0.172)         &     (0.141)         &     (0.158)         \\
Gen                 &     0.00619         &       0.241         &       0.180         &      -0.194         &     -0.0356         \\
                    &     (0.109)         &     (0.159)         &     (0.207)         &     (0.235)         &     (0.159)         \\
Sym                 &     -0.0922         &      -0.175         &       0.112         &      -0.164\sym{**} &      -0.122         \\
                    &    (0.0671)         &     (0.151)         &    (0.0988)         &    (0.0604)         &     (0.103)         \\
Ext                 &      -0.597\sym{***}&      -0.460         &      -0.685\sym{***}&      -0.429\sym{**} &      -0.612\sym{***}\\
                    &     (0.177)         &     (0.328)         &     (0.230)         &     (0.193)         &     (0.135)         \\

				
				\bottomrule 
		\end{tabular}}
		% \begin{tablenotes} 
		% 	\item 
		% \end{tablenotes} 
	\end{threeparttable} 
	\begin{minipage}{0.95\linewidth}
		\scriptsize \emph{Notes:} This table reports intention-to-treat estimates across the main diagnosis chapters for the entire life-course or per age bracket. The outcomes are defined as the number of cases per 1,000 individualsbeim . The point estimates are coming from a DiD regression as described in section \ref{sec_mlch:empirical_strategy}, with a bandwidth of six months, month-of-birth and year fixed effects, and clustered standard errors on the month-of-birth level. The control group is comprised of children that are born in the same months but one year before (i.e. children born between November 1977 and October 1978). In order to compare the two birth cohorts at the same age, I shift the control cohort from wave $t$ to wave $t+1$.\newline Significance levels: * p < 0.10, ** p < 0.05, *** p < 0.01.\newline
		\emph{Legend:} Infectious and parasitic diseases (IPD), neoplasms (Neo), mental and behavioral disorders (MBD), diseases of the nervous system (Ner), diseases of the sense organs (Sen), diseases of the circulatory system (Cir), diseases of the respiratory system (Res), diseases of the digestive system (Dig), diseases of the skin and subcutaneous tissue (SST), diseases of the musculoskeletal system (Mus), diseases of the genitourinary system (Gen), symptoms, signs, and ill-defined conditions (Sym), injury, poisoning and certain other consequences of external causes (Ext).\newline %\emph{Source:} Hospital registry data.
	\end{minipage}
\end{table} 
\vspace*{\fill}\clearpage 
\restoregeometry
%--------------------------------------------




%--------------------------------------------
% d5 - Difference in Difference table 
\vspace*{\fill}
\begin{table}[H] \centering 
 \begin{threeparttable} \centering \caption{ITT effects on \textbf{mental \& behavioral disorders (total)}}\label{tab: DD_d5_total}
  {\def\sym#1{\ifmmode^{#1}\else\(^{#1}\)\fi} 
 	\begin{tabular}{l*{6}{c}}
 		\toprule 
 		%\multicolumn{5}{l}{Dependant variable: \textbf{Hospital admission (total)}}\\ \\ 
 		& \multicolumn{5}{c}{Estimation window} \\ 
 		\cmidrule(lr){2-6}
 		&\multicolumn{1}{c}{(1)}&\multicolumn{1}{c}{(2)}&\multicolumn{1}{c}{(3)}&\multicolumn{1}{c}{(4)}&\multicolumn{1}{c}{(5)}\\
 		&\multicolumn{1}{c}{3M}&\multicolumn{1}{c}{4M}&\multicolumn{1}{c}{5M}&\multicolumn{1}{c}{6M}&\multicolumn{1}{c}{Donut}\\
 		\midrule
 		\multicolumn{5}{l}{\emph{Panel A. Over entire length of the life-course}} \\
 		\hspace*{10pt}Overall&      -0.656         &      -0.852\sym{**} &      -0.756\sym{**} &      -0.634\sym{**} &      -0.809\sym{***}\\
                    &     (0.420)         &     (0.350)         &     (0.280)         &     (0.249)         &     (0.274)         \\
\midrule Dependent mean&       19.23         &       19.05         &       18.98         &       18.96         &       19.16         \\
Effect in SDs [\%]  &       11.33         &       14.87         &       13.36         &       11.43         &       14.64         \\
Observations        &         240         &         320         &         400         &         480         &         400         \\
 \\ \\
 		\multicolumn{5}{l}{\emph{Panel B. Age brackets}} \\
 		\hspace*{10pt}Age 17-21&       0.135         &       0.318         &       0.268         &       0.174         &     -0.0603         \\
                    &     (0.516)         &     (0.387)         &     (0.314)         &     (0.263)         &     (0.239)         \\
 \hspace*{10pt}Age 22-26&       0.343         &      -0.172         &      -0.146         &    -0.00769         &      -0.360         \\
                    &     (0.640)         &     (0.607)         &     (0.500)         &     (0.420)         &     (0.454)         \\
 \hspace*{10pt}Age 27-31&      -1.258\sym{**} &      -1.508\sym{***}&      -1.301\sym{***}&      -1.000\sym{**} &      -1.020\sym{**} \\
                    &     (0.546)         &     (0.478)         &     (0.391)         &     (0.357)         &     (0.433)         \\
 \hspace*{10pt}Age 32-35&      -1.886\sym{***}&      -2.352\sym{***}&      -1.906\sym{***}&      -1.949\sym{***}\\
                    &     (0.224)         &     (0.305)         &     (0.372)         &     (0.439)         \\
 
 		\bottomrule 
 	\end{tabular}}
 	\begin{tablenotes} 
 		\item \scriptsize \emph{Notes:} Clustered standard errors in parentheses. All regression are run with CG2 (i.e. the cohort prior to the reform) and with month-of-birth FEs. The 'overall' specification includes year fixed effects, as well. The outcome variables are defined as the number of cases per thousand individuals (births). Dependent mean and the effect size correspond to pre-reform values of the treated group.
 	\end{tablenotes} 
 \end{threeparttable} 
 \end{table}
\vspace*{\fill}\clearpage 
% \vspace*{\fill}
 \begin{table}[H] \centering 
 	\begin{threeparttable} \centering \caption{ITT effects on \textbf{mental \& behavioral disorders (women)}}\label{tab: DD_d5_female} {\def\sym#1{\ifmmode^{#1}\else\(^{#1}\)\fi} 
 			\begin{tabular}{l*{6}{c}}
 				\toprule 
 				%\multicolumn{5}{l}{Dependant variable: \textbf{Hospital admission (total)}}\\ \\ 
 				& \multicolumn{5}{c}{Estimation window} \\ 
 				\cmidrule(lr){2-6}
 				&\multicolumn{1}{c}{(1)}&\multicolumn{1}{c}{(2)}&\multicolumn{1}{c}{(3)}&\multicolumn{1}{c}{(4)}&\multicolumn{1}{c}{(5)}\\
 				&\multicolumn{1}{c}{6M}&\multicolumn{1}{c}{5M}&\multicolumn{1}{c}{4M}&\multicolumn{1}{c}{3M}&\multicolumn{1}{c}{Donut}\\
 				\midrule
 				\multicolumn{5}{l}{\emph{Panel A.Over entire length of the life-course}} \\
 				\hspace*{10pt}Overall&      -0.192         &      -0.163         &     -0.0424         &      0.0599         &      0.0560         \\
                    &     (0.455)         &     (0.362)         &     (0.292)         &     (0.266)         &     (0.296)         \\
\midrule Dependent mean&       15.97         &       15.72         &       15.72         &       15.74         &       15.95         \\
Effect in SDs [\%]  &       5.130         &       4.390         &       1.150         &       1.670         &       1.580         \\
Observations        &         240         &         320         &         400         &         480         &         400         \\
 \\ \\
 				\multicolumn{5}{l}{\emph{Panel B. Age brackets}} \\
 				\hspace*{10pt}Age 17-21&      0.0745         &       0.527         &       0.416         &       0.388         &       0.235         \\
                    &     (0.555)         &     (0.463)         &     (0.378)         &     (0.313)         &     (0.318)         \\
 \hspace*{10pt}Age 22-26&       0.205         &       0.119         &      0.0485         &       0.217         &     -0.0753         \\
                    &     (0.466)         &     (0.558)         &     (0.700)         &     (0.791)         &     (0.499)         \\
 \hspace*{10pt}Age 27-31&       0.125         &      -0.816         &      -0.426         &      -0.273         \\
                    &     (1.093)         &     (0.579)         &     (0.418)         &     (0.426)         \\
 \hspace*{10pt}Age 32-35&      -0.818         &      -0.671\sym{*}  &      -0.163         &       0.242         \\
                    &     (0.619)         &     (0.344)         &     (0.388)         &     (0.396)         \\
 
 				\bottomrule 
 		\end{tabular}}
 		\begin{tablenotes} 
 			\item \scriptsize \emph{Notes:} TThe table shows DD estimates of the 1979 maternity leave reform on mental and behavioral disorders for different bandwidths around the cutoff. Panel A shows the effect for the entire pooled time frame and panel B breaks the life-course up in age brackets. The outcome variables are defined as the number of cases per thousand individuals (births). All regressions control for year and month-of-birth fixed effects. The control group is comprised of children that are born in the same months but one year before the reform (i.e. children born between November 1977 and October 1978). In order to compare the two birth cohorts at the same age, we shift the control cohort from wave $t$ to wave $t+1$. The dependent mean and the effect size in standard deviation units correspond to pre-reform values of the treated group. Clustered standard errors are reported in parentheses. \newline Significance levels: + p < 0.15, * p < 0.10, ** p < 0.05, *** p < 0.01. \newline 	\emph{Source:} Hospital registry data.
 		\end{tablenotes} 
 	\end{threeparttable} 
 \end{table}
\vspace*{\fill}\clearpage 
%  \vspace*{\fill}
 \begin{table}[H] \centering 
 	\begin{threeparttable} \centering \caption{ITT effects on \textbf{mental \& behavioral disorders (men)}}\label{tab: DD_d5_male} {\def\sym#1{\ifmmode^{#1}\else\(^{#1}\)\fi} 
 			\begin{tabular}{l*{6}{c}}
 				\toprule 
 				%\multicolumn{5}{l}{Dependant variable: \textbf{Hospital admission (total)}}\\ \\ 
 				& \multicolumn{5}{c}{Estimation window} \\ 
 				\cmidrule(lr){2-6}
 				&\multicolumn{1}{c}{(1)}&\multicolumn{1}{c}{(2)}&\multicolumn{1}{c}{(3)}&\multicolumn{1}{c}{(4)}&\multicolumn{1}{c}{(5)}\\
 				&\multicolumn{1}{c}{3M}&\multicolumn{1}{c}{4M}&\multicolumn{1}{c}{5M}&\multicolumn{1}{c}{6M}&\multicolumn{1}{c}{Donut}\\
 				\midrule
 				\multicolumn{5}{l}{\emph{Panel A. Over entire length of the life-course}} \\
 				\hspace*{10pt}Overall&      -1.192\sym{***}&      -1.328\sym{***}&      -1.462\sym{***}&      -1.098\sym{**} &      -1.533\sym{***}\\
                    &     (0.288)         &     (0.336)         &     (0.412)         &     (0.486)         &     (0.286)         \\
\midrule Dependent mean&       22.84         &       22.91         &       23.07         &       23.19         &       23.05         \\
%Effect in SDs [\%]  &       17.60         &       19.29         &       20.84         &       15.43         &       22.64         \\
\(N\) (MOB $\times$ year)&         456         &         380         &         304         &         228         &         380         \\
 \\ \\
 				\multicolumn{5}{l}{\emph{Panel B. Age brackets}} \\
 				\hspace*{10pt}Age 17-21&       0.803\sym{*}  &       0.119         &     -0.0319         &      -0.344         \\
                    &     (0.382)         &     (0.373)         &     (0.262)         &     (0.217)         \\
 \hspace*{10pt}Age 22-26&       1.688\sym{***}&      -0.373         &      -0.180         &      -0.602         \\
                    &     (0.231)         &     (0.683)         &     (0.485)         &     (0.526)         \\
 \hspace*{10pt}Age 27-31&      -1.854\sym{**} &      -2.152\sym{***}&      -1.943\sym{***}&      -1.504\sym{***}&      -1.690\sym{***}\\
                    &     (0.652)         &     (0.675)         &     (0.542)         &     (0.507)         &     (0.570)         \\
 \hspace*{10pt}Age 32-35&      -2.879\sym{**} &      -3.938\sym{***}&      -3.518\sym{***}&      -3.989\sym{***}\\
                    &     (0.841)         &     (0.596)         &     (0.515)         &     (0.515)         \\
 
 				\bottomrule 
 		\end{tabular}}
 		\begin{tablenotes} 
 				\item \scriptsize \emph{Notes:} The table shows DD estimates of the 1979 maternity leave reform on mental and behavioral disorders for different bandwidths around the cutoff. Panel A shows the effect for the entire pooled time frame and panel B breaks the life-course up in age brackets. The outcome variables are defined as the number of cases per thousand individuals (births). All regressions control for year and month-of-birth fixed effects. The control group is comprised of children that are born in the same months but one year before the reform (i.e. children born between November 1977 and October 1978). In order to compare the two birth cohorts at the same age, we shift the control cohort from wave $t$ to wave $t+1$. The dependent mean and the effect size in standard deviation units correspond to pre-reform values of the treated group. Clustered standard errors are reported in parentheses. \newline Significance levels: + p < 0.15, * p < 0.10, ** p < 0.05, *** p < 0.01. \newline 	\emph{Source:} Hospital registry data.
 		\end{tablenotes} 
 	\end{threeparttable} 
 \end{table} 
\vspace*{\fill}\clearpage 


%--------------------------------------------
% d5 - Difference in Difference table - by GENDER
\newgeometry{left=1cm,right=1cm,top=1cm,bottom=2.5cm} 
\begin{landscape}
\vspace*{\fill}
 \begin{table}[H] \centering 
 	\begin{threeparttable} \centering \caption{ITT effects on \textbf{mental \& behavioral disorders, by gender}}\label{tab_mlch: DD_d5_female_male} {\def\sym#1{\ifmmode^{#1}\else\(^{#1}\)\fi} 
 			\begin{tabular}{l*{12}{c}}
 				\toprule 
 				% \multicolumn{5}{l}{Dependant variable: \textbf{Hospital admission (total)}}\\ \\ 
 				& \multicolumn{5}{c}{Women} && \multicolumn{5}{c}{Men} \\ 
 				\cmidrule(lr){2-6} \cmidrule(lr){8-12}
 				&\multicolumn{1}{c}{(1)}&\multicolumn{1}{c}{(2)}&\multicolumn{1}{c}{(3)}&\multicolumn{1}{c}{(4)}&\multicolumn{1}{c}{(5)}&\multicolumn{1}{c}{        }&\multicolumn{1}{c}{(6)}&\multicolumn{1}{c}{(7)}&\multicolumn{1}{c}{(8)}&\multicolumn{1}{c}{(9)}&\multicolumn{1}{c}{(10)}\\
 				&\multicolumn{1}{c}{6M}&\multicolumn{1}{c}{5M}&\multicolumn{1}{c}{4M}&\multicolumn{1}{c}{3M}&\multicolumn{1}{c}{Donut}&&\multicolumn{1}{c}{6M}&\multicolumn{1}{c}{5M}&\multicolumn{1}{c}{4M}&\multicolumn{1}{c}{3M}&\multicolumn{1}{c}{Donut}\\
 				\midrule
 				\multicolumn{5}{l}{\emph{Panel A.Over entire length of the life-course}} \\


 				\hspace*{10pt}Overall&     0.00972         &     -0.0900         &      -0.205         &      -0.244         &      0.0211       & &      -1.192\sym{***}&      -1.328\sym{***}&      -1.462\sym{***}&      -1.098\sym{**} &      -1.533\sym{***}\\
				                    &     (0.271)         &     (0.303)         &     (0.377)         &     (0.496)         &     (0.285)        & &     (0.288)         &     (0.336)         &     (0.412)         &     (0.486)         &     (0.286)         \\
				\midrule Dependent mean&       16.11         &       16.09         &       16.09         &       16.32         &       16.32     & &       22.84         &       22.91         &       23.07         &       23.19         &       23.05         \\
				Effect in SDs [\%]  &       0.300         &       2.650         &       5.940         &       7.010         &       0.650        & &       17.60         &       19.29         &       20.84         &       15.43         &       22.64         \\
				\(N\) (MOB $\times$ year)&         456         &         380         &         304         &         228         &         380   & &         456         &         380         &         304         &         228         &         380         \\
				\\

 				\multicolumn{5}{l}{\emph{Panel B. Age brackets}} \\
 				\hspace*{10pt}Age 17-21	&       0.388         &       0.416         &       0.527         &      0.0745         &       0.235   & &    -0.0319         &       0.129         &       0.119         &       0.192         &      -0.344		   \\ 
				                    	&     (0.313)         &     (0.378)         &     (0.463)         &     (0.555)         &     (0.318)   & &    (0.262)         &     (0.300)         &     (0.373)         &     (0.507)         &     (0.217)         \\ 
				\hspace*{10pt}Age 22-26	&       0.205         &       0.119         &      0.0485         &       0.217         &     -0.0753   & &     -0.180         &      -0.379         &      -0.373         &       0.475         &      -0.602         \\ 
				                    	&     (0.466)         &     (0.558)         &     (0.700)         &     (0.791)         &     (0.499)   & &    (0.485)         &     (0.575)         &     (0.683)         &     (0.680)         &     (0.526)         \\ 
				\hspace*{10pt}Age 27-31	&      -0.426         &      -0.598         &      -0.816         &      -0.612         &      -0.273   & &     -1.504\sym{***}&      -1.943\sym{***}&      -2.152\sym{***}&      -1.854\sym{**} &      -1.690\sym{***}\\ 
				                    	&     (0.418)         &     (0.469)         &     (0.579)         &     (0.781)         &     (0.426)   & &    (0.507)         &     (0.542)         &     (0.675)         &     (0.652)         &     (0.570)         \\ 
				\hspace*{10pt}Age 32-35	&      -0.163         &      -0.349         &      -0.671\sym{*}  &      -0.760			&       0.242   & &     -3.518\sym{***}&      -3.568\sym{***}&      -3.938\sym{***}&      -3.733\sym{***}&      -3.989\sym{***}\\ 
				                    	&     (0.388)         &     (0.335)         &     (0.344)         &     (0.466)         &     (0.396)   & &    (0.515)         &     (0.522)         &     (0.596)         &     (0.741)         &     (0.515)         \\ 
 				\bottomrule 
 		\end{tabular}}
 		\begin{tablenotes} 
 			\item \scriptsize \emph{Notes:} The table shows DiD estimates of the effect of the 1979 maternity leave reform on mental and behavioral disorders for different estimation windows around the cutoff. The \textit{`Donut'} specification uses a bandwidth of half a year and excludes children born in April and May. Panel A shows the effect for the entire pooled time frame and panel B breaks the life-course up in age brackets. The outcome variables are defined as the number of cases per thousand individuals. All regressions control for year and month-of-birth fixed effects. The control group is comprised of children that are born in the same months but one year before the reform (i.e. children born between November 1977 and October 1978). In order to compare the two birth cohorts at the same age, I shift the control cohort from wave $t$ to wave $t+1$. The dependent mean and the effect size in standard deviation units correspond to pre-reform values of the treated group. \revision{Table \ref{tab_mlch: observations_age_brackets} contains the number of observations for the estimations per age bracket.} Clustered standard errors are reported in parentheses. \newline Significance levels: * p < 0.10, ** p < 0.05, *** p < 0.01. \newline 	%\emph{Source:} Hospital registry data.
 		\end{tablenotes} 
 	\end{threeparttable} 
 \end{table}
\vspace*{\fill}\clearpage 
\end{landscape}

\restoregeometry






%--------------------------------------------
% HOSPITAL 2 -ROBUSTNESS TABLE
\newpage
\newgeometry{left=1cm,right=1cm,top=1cm,bottom=2.5cm} 
% \begin{landscape}
	\vspace*{\fill}
	\begin{table}[htbp] \centering 
		\begin{threeparttable} \centering 
			\caption{Robustness checks for \textbf{hospital admission}}\label{tab: robustness_hospital} 
			{\def\sym#1{\ifmmode^{#1}\else\(^{#1}\)\fi} 
				\begin{tabular}{l*{10}{c}} \toprule 
					
					& & \multicolumn{2}{c}{Alternative specifications} & \multicolumn{3}{c}{\clb{c}{Alternative\\estimation}} & \multicolumn{2}{c}{Placebos}& \multicolumn{2}{c}{Heterogeneity}\\
					\cmidrule(lr){3-4} \cmidrule(lr){5-7} \cmidrule(lr){8-9} \cmidrule(lr){10-11}
					&\multicolumn{1}{c}{(1)}&\multicolumn{1}{c}{(2)}&\multicolumn{1}{c}{(3)}&\multicolumn{1}{c}{(4)}&\multicolumn{1}{c}{(5)}&\multicolumn{1}{c}{(6)}&\multicolumn{1}{c}{(7)}&\multicolumn{1}{c}{(8)}&\multicolumn{1}{c}{(9)}&\multicolumn{1}{c}{(10)}\\
					&\multicolumn{1}{c}{Baseline}&\multicolumn{1}{c}{\clb{c}{current\\population}}&\multicolumn{1}{c}{\clb{c}{LMR\\level$^a$}}&\multicolumn{1}{c}{\clb{c}{DDD$^b$}}&\multicolumn{1}{c}{\clb{c}{alt. DD$^b$}}&\multicolumn{1}{c}{add. CG}&\multicolumn{1}{c}{\clb{c}{temporal:\\cohort}}&\multicolumn{1}{c}{\clb{c}{spatial:\\ GDR}}&\multicolumn{1}{c}{\clb{c}{rural$^a$}}&\multicolumn{1}{c}{\clb{c}{urban$^a$}}\\
					\midrule
					\\
					%							1					2					3					4					5					6					7					8				9				10				
					(1) {total} 		&   -2.168\sym{**}	&	-1.581\sym{**}	&   -1.627\sym{**} 	&	-2.226\sym{*}	& 	-2.449\sym{***} & -2.327\sym{**}	&	-0.318			&	-0.0268		&	-0.989		&	-1.779\sym{***} \\
										&	(0.782)			&	(0.675)			&   (0.658)     	&	(1.115)			& 	(0.738)			& (1.003)			&	(0.946)			&	(0.453)		&	(1.143)		&	(0.626)			\\
					(2) {female}		&   -1.815			&	-0.694			& 	-0.558      	&	-1.418			& 	-2.493\sym{***}	& -1.573		    &	0.483			&	-0.319		&	1.248		&	-0.988			\\
										&	(0.807)			&	(0.633)			&   (0.618)     	&	(1.210)			& 	(0.776)			& (1.114)			&	(0.942)			&	(0.457)		&	(1.736)		&	(0.635)			\\
					(3) {male} 			&   -2.525\sym{**}	&	-2.462\sym{**}	&   -2.723\sym{***} &	-2.941\sym{**}	& 	-2.362\sym{**}	& -3.063\sym{**}	&	-1.076			&	0.179		&-3.120\sym{**}	&	-2.628\sym{**}  \\
										&	(0.997)			&	(0.981)			&   (0.957)     	&	(1.271)			& 	(0.774)			& (1.140)			&	 (1.059) 		&	(0.699)		&	(1.180)		&	(1.023)			\\
					\midrule            																																																					
					For total: 																																																			\\							 
					Dependent mean 		&   120.6			&	92.22			&   98.31     		&	121.4			& 	121.4			& 120.6				&	120.2			&	67.4		&	101.0		&	96.11			\\
					Effect in SDs [\%] 	&   19.78			&	16.21			&   4.40      		&	20.25			& 	22.29			& 21.23				&	3.060			&	0.21		&	2.34		&	5.590			\\
					$N$ (MOB $\times$ year) 		&   480				&	288				&   58,751    		&	960				& 	480				& 720				&	480				&	480			&	26,495		&	32,256			\\
					%Federal level		&   \checkmark		&	\checkmark		&   $\times$		& \checkmark		&	\checkmark		& \checkmark		&	\checkmark		&  \checkmark	&	$\times$	&	$\times$		\\ 
					\\
					MOB fixed effects 	&   \checkmark		&	\checkmark		&   \checkmark		& \checkmark		&	\checkmark		& \checkmark		&	\checkmark		&  \checkmark	&	\checkmark	&	\checkmark		\\ 
					Year fixed effects  &   \checkmark		&	\checkmark		&   \checkmark		& \checkmark		&	\checkmark		& \checkmark		&	\checkmark		&  \checkmark	&	\checkmark	&	\checkmark		\\ 
					\bottomrule
			\end{tabular}}
	\end{threeparttable} 
		\begin{minipage}{0.87\linewidth}
		\scriptsize \emph{Notes:} This table displays robustness check for the effect of the 1979 maternity leave reform on hospital admissions. We perform the following checks (with reference to the column): (1) baseline specification that was used in previous parts of the paper, (2) for the outcome we use the number of diagnoses divided by the current number of individuals (approximation), (3) the analysis is carried out on the level of labor market regions, (4) triple difference model (the third difference stems from the former region of the GDR), (5) alternative difference-in-difference model which compares pre and post of the treatment cohort in West Germany with the respective values in East Germany, (6) we use as control cohort not only the cohort before the reform, but also the cohort 2 years prior to the policy change, (7) first placebo, in which the entire analysis set-up is pushed back by one year, i.e. the placebo TG is the cohort prior to the real TG and the placebo CG is the cohort born 2 years before the reform took place, (8) second placebo, in which we run the normal DD set-up in the area of the former GDR, (9) + (10)  DD carried out in rural and urban regions (compare with figure \ref{fig: AMR_regions_population_density} to see which regions are marked as rural/urban). \newline Significance levels: * p < 0.10, ** p < 0.05, *** p < 0.01. \newline
		\hspace*{15 pt}$^a$: level of analysis on Labor Market Regions: weighted regressions (by population), includes region fixed effects.\newline
		\hspace*{15 pt}$^b$: standard errors clustered on the month-of-birth$\times$birth-cohort$\times$East-West cell level.
	\end{minipage}
\end{table} 
	\vspace*{\fill}\clearpage
\end{landscape}

% Welche Columns sind geupdated: 1 2 3 4 5 6 7 8 9 10






\newgeometry{left=2.5cm,right=2.5cm,top=2.9cm,bottom=2.9cm} 
\begin{landscape}
	\vspace*{\fill}
	\begin{table}[H] \centering 
		\begin{threeparttable} \centering 
			\caption{Robustness checks for hospital admission}\label{tab_mlch: robustness_hospital} 
			{\def\sym#1{\ifmmode^{#1}\else\(^{#1}\)\fi} 
				\begin{tabular}{l*{8}{c}} \toprule 
					
					& & \multicolumn{2}{c}{Alternative specifications} & \multicolumn{2}{c}{\clb{c}{Alternative\\estimation}} & \multicolumn{2}{c}{Placebos}\\
					\cmidrule(lr){3-4} \cmidrule(lr){5-6} \cmidrule(lr){7-8} 
					&\multicolumn{1}{c}{(1)}&\multicolumn{1}{c}{(2)}&\multicolumn{1}{c}{(3)}&\multicolumn{1}{c}{(4)}&\multicolumn{1}{c}{(5)}&\multicolumn{1}{c}{(6)}&\multicolumn{1}{c}{(7)}\\
					&\multicolumn{1}{c}{Baseline}&\multicolumn{1}{c}{\clb{c}{Current\\population}}&\multicolumn{1}{c}{\clb{c}{LMR\\level$^a$}}&\multicolumn{1}{c}{\clb{c}{DDD$^b$}}&\multicolumn{1}{c}{Add. CG}&\multicolumn{1}{c}{\clb{c}{Temporal:\\cohort}}&\multicolumn{1}{c}{\clb{c}{Spatial:\\ GDR}}\\
					\midrule
					\\
					%							1					2					3					4					5					6					7					8			
					(1) {Total} 		&   -2.076\sym{**}	&	-1.581\sym{**}	&   -1.771\sym{***} &	-2.313\sym{*}	&  -2.293\sym{**}	&	 -0.203			&	0.154		\\
										&	(0.772)			&	(0.675)			&   (0.623)     	&	(1.127)			&  (0.987)			&	(0.962)			&	(0.469)		\\
					(2) {Female}		&   -1.742\sym{**}	&	-0.694			& 	-0.740      	&	-1.255			&  -1.559		    &	0.561			&	-0.396		\\
										&	(0.816)			&	(0.633)			&   (0.597)     	&	(1.231)			&  (1.112)			&	(0.964)			&	(0.503)		\\
					(3) {Male} 			&   -2.410\sym{**}	&	-2.462\sym{**}	&   -2.816\sym{***} &	-3.252\sym{**}	&  -3.007\sym{**}	&	-0.926 			&	0.593		\\
										&	(1.015)			&	(0.981)			&   (0.945)     	&	(1.310)			&  (1.135)			&	 (1.081) 		&	(0.714)		\\
					\midrule            																																							
					For total: 																																					\\							 
					Dependent mean 		&   121.1			&	92.22			&   98.66     		&	121.8			&  121.1			&	120.2			&	66.29		\\
					Effect in SDs [\%] 	&   18.88			&	16.21			&   4.750      		&	20.94			&  20.86			&	1.900			&	1.260		\\
					$N$ 				&   456				&	288				&   53,855    		&	912				&  672				&	456 			&	456			\\
					%Federal level		&   \checkmark		&	\checkmark		&   $\times$		& \checkmark		& \checkmark		&	\checkmark		&  \checkmark	\\ 
					\\
					MOB fixed effects 	&   \checkmark		&	\checkmark		&   \checkmark		& \checkmark		& \checkmark		&	\checkmark		&  \checkmark	\\ 
					Year fixed effects  &   \checkmark		&	\checkmark		&   \checkmark		& \checkmark		& \checkmark		&	\checkmark		&  \checkmark	\\ 
					\bottomrule
			\end{tabular}}
			\begin{tablenotes}
				\item \scriptsize \emph{Notes:} This table displays robustness checks for the effect of the 1979 maternity leave reform on hospital admissions. I perform the following checks (with reference to the column): (1) baseline specification that was used in previous parts of the paper, (2) for the outcome I use the number of diagnoses divided by the current number of individuals (approximation), (3) the analysis is carried out on the level of labor market regions, (4) triple difference model (the third difference stems from the former region of the GDR), (5) I use as control cohort not only the cohort before the reform, but also the cohort 2 years prior to the policy change, (6) first placebo, in which the entire analysis set-up is pushed back by one year, i.e. the placebo TG is the cohort prior to the real TG and the placebo CG is the cohort born 2 years before the reform took place, (7) second placebo, in which I run the normal DiD set-up in the area of the former GDR. \newline Significance levels: * p < 0.10, ** p < 0.05, *** p < 0.01. \newline
				\hspace*{15 pt}$^a$: level of analysis on Labor Market Regions: weighted regressions (by population), includes region fixed effects.\newline
				\hspace*{15 pt}$^b$: standard errors clustered on the month-of-birth$\times$birth-cohort$\times$East-West cell level.
			\end{tablenotes}
		\end{threeparttable} 
	\end{table} 
	\vspace*{\fill}\clearpage
\end{landscape}
\restoregeometry
% Welche Columns sind geupdated (einbinden von $LC) : 3,4,5,8,9,10






\restoregeometry




%--------------------------------------------

% Heterogeneity analysis - rural/urban
\newpage
\vspace*{\fill}
\begin{table}[htbp] \centering 
	\begin{threeparttable} \centering 
		\caption{Subgroup analysis} \label{tab: heterogeneity analysis} 
		{\def\sym#1{\ifmmode^{#1}\else\(^{#1}\)\fi} 
			\begin{tabular}{l*{2}{c}} \toprule 
				
				&  \multicolumn{2}{c}{Heterogeneity}\\
				\cmidrule(lr){2-3} 
				&\multicolumn{1}{c}{(1)}&\multicolumn{1}{c}{(2)}\\
				&\multicolumn{1}{c}{\clb{c}{rural}}&\multicolumn{1}{c}{\clb{c}{urban}}\\
				\midrule
				\\

				\textit{Panel A: Hospital admissions}\\
				DiD estimate 		&	-1.654		 &	-1.799\sym{***} \\
									&	(1.096)		 &	(0.598)			\\

				Dependent mean 		&	101.3		 &	96.50			\\
				Effect in SDs [\%] 	&	3.880		 &	5.600			\\
				$N$ 				&	24,287		 &	29,568			\\
				\\ \\


				\textit{Panel B: Mental and behavioral disorders}\\
				DiD estimate 		&	-0.241		&	-0.986\sym{***} 	\\
									&	(0.564)		&	(0.196)				\\							 
				Dependent mean 		&	17.00  		&	18.61				\\
				Effect in SDs [\%] 	&	1.310		&	7.100				\\
				$N$ 				&	24,287		&	29,568				\\

				\\
				\midrule
				MOB fixed effects 	&	\checkmark	&	\checkmark		    \\ 
				Year fixed effects  &	\checkmark	&	\checkmark		    \\
				Region fixed effects& 	\checkmark	&	\checkmark		    \\
				\bottomrule
		\end{tabular}}
	\end{threeparttable} 
	\begin{minipage}{0.7\linewidth}
		\scriptsize \emph{Notes:} This table contains a subgroup analysis for the effect of the 1979 maternity leave reform on different health outcomes. The DiD estimates stem from weighted regression (by population) over the entire pooled time frame, and a bandwidth of half a year around the cutoff. The level of analysis is on Labor Market Regions. Figure \ref{fig: AMR_regions_population_density} shows a map of Germany with the regions marked as rural/urban. \newline Significance levels: * p < 0.10, ** p < 0.05, *** p < 0.01. \newline
	\end{minipage}
\end{table} 
\vspace*{\fill}\clearpage


%--------------------------------------------
