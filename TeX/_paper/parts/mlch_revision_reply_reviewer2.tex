
\clearpage
\section*{Comments made by Reviewer 2}


%-----------------------------------------
% 
\begin{quote}
	\textit{"The empirical strategy provides results based on intention to treat estimations. The paper states that although eligibility for ML was universal among working women, the take up rate was low, as only about 40\% of women took advantage of the ML reform in 1979. This is an important fact, implying that only 60\% of the treatment group were really treated (if I understood well). This figure may even be higher, as children from non working mothers are also included. Perhaps some robustness checks can be added in order to deal with this problem, and it should be more clear for the reader."}
\end{quote}
\underline{Response:}

%-----------------------------------------
% 
\begin{quote}
	\textit{"The authors rule out strategic conceptions due to the short period between the draft bill and the implementation of the reform. Nevertheless, in cesareans there still could be place for strategic behavior, although its incidence may not be relevant at that time."}
\end{quote}
\underline{Response:} I fully agree that parents who postpone C-sections around the cutoff would be a serious threat to the validity of the identification scheme. For instance, \cite{gans2009born} find that during the introduction of a \$3.000 `Baby Bonus' in Australia, parents postponed the birth by as much as a week in order to be eligible for the benefits. In the appendix (page \pageref{sec_mlch:empirical_strategy_threats+validity}), I show that in my context the number of births is smooth around the threshold. Footnote \ref{rev_mlch: footnote_csections_germany} directly relates to your point and shows that the C-section rate in Germany (12.4\%) was considerably smaller than the Australian counterpart (30\%).

 


%-----------------------------------------
% 
\begin{quote}
	\textit{"Changes in household income are a potential channel trough which ML could have affected child health outcomes. This is discussed in section 7, but not linked to the previous findings in the literature about the effect of ML on mother's income (reported in section 3). By the way, the explanation about that increase is not very clear."}
\end{quote}
\underline{Response:}

%-----------------------------------------
% 
\begin{quote}
	\textit{"It  would be useful to provide the number of observations for each estimation in the tables"}
\end{quote}
\underline{Response:}

%-----------------------------------------
% 
\begin{quote}
	\textit{"There is no comment or discussion about why effects are found in men. May be other disciplines can bring some hypothesis about that?"}
\end{quote}
\underline{Response:}

%-----------------------------------------
% 
\begin{quote}
	\textit{"The paper focuses on MBD, but in table 3 significant and sizeable effects are found in Ext (may be related to risky behavior?) and Dig. It is not clear why these other chapters are not even discussed (even if they present unstable age patterns). "}
\end{quote}
\underline{Response:}

%-----------------------------------------
% structure of the paper
\begin{quote}
	\textit{"In terms of the structure of the paper, there is an overlapping between the discussion in Section 3, which presents the conceptual framework for long run health effects, and section 7, where the potential mechanisms through which the extension of ML could have affected health outcomes. Maybe those two sections can be merged and become just one (substituting section 3?), and also a more concise one. The first paragraphs in section 7 summarize the paper, they could excluded in the final version of the paper."}
\end{quote}
\underline{Response:}
