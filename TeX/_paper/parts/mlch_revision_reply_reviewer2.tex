
\clearpage
\section*{Comments made by Reviewer 2}


%-----------------------------------------
% ITT, LATE, Robustness check
\begin{quote}
	\textit{"The empirical strategy provides results based on intention to treat estimations. The paper states that although eligibility for ML was universal among working women, the take up rate was low, as only about 40\% of women took advantage of the ML reform in 1979. This is an important fact, implying that only 60\% of the treatment group were really treated (if I understood well). This figure may even be higher, as children from non working mothers are also included. Perhaps some robustness checks can be added in order to deal with this problem, and it should be more clear for the reader."}
\end{quote}
\underline{Response:} I dealt with this comment in two ways: 

\textbf{Make intention-to-treat (ITT) estimates clearer.---}You are entirely right in that the ITT estimates are diluted due to the non-perfect compliance rate. I address this issue in the empirical section by adding footnote \ref{rev_mlch: r2_itt_late} (on page \pageref{rev_mlch: r2_itt_late}). In that footnote, I explain how you have to scale the ITT effects to obtain the causal effect on the compliers, the local average treatment effect. The extra scaling factor of $0.835$ (the impact of the 1979 ML reform on maternal labor supply) is used to have the convenient interpretation of "one additional month away from work after childbirth". 
%The LATE represents the effect of treatment on the treated, if compliance is largely present in either treatment or control group (one-sided compliance). This seems unlikely, so the LATE identifies the causal effect on compliers.

\textbf{Robustness check exploiting heterogeneous eligibility for maternity leave.---} Section \ref{sec_mlch:discussion} (page \pageref{sec_mlch:discussion}) discusses potential mechanisms for how the reform affected children's health outcomes. In the first paragraph, it contains a robustness check addressing your comment. It was already part of the original submission, but I reformulated it to make it easier accessible for the reader. The idea of the robustness check is to run a subgroup analysis for urban and rural areas. These two regions differed in female labor force participation rates and hence in the share of mothers eligible for maternity leave. In urban areas, relatively more children were affected by the reform, which is reassuringly mirrored by larger ITT effects.





%-----------------------------------------
% C-Sections
\begin{quote}
	\textit{"The authors rule out strategic conceptions due to the short period between the draft bill and the implementation of the reform. Nevertheless, in cesareans there still could be place for strategic behavior, although its incidence may not be relevant at that time."}
\end{quote}
\underline{Response:} I fully agree that parents who postpone C-sections around the cutoff would be a severe threat to the validity of the identification scheme. For instance, \cite{gans2009born} find that during the introduction of a \$3.000 `Baby Bonus' in Australia, parents postponed the birth by as much as a week in order to be eligible for the benefits. In the appendix (page \pageref{sec_mlch:empirical_strategy_threats+validity}), I show that the number of births is smooth around the threshold in my context. Footnote \ref{rev_mlch: footnote_csections_germany} directly relates to your point and shows that Germany's C-section rate (12.4\%) was considerably smaller than the Australian counterpart (30\%).

 


%-----------------------------------------
% maternal income
\begin{quote}
	\textit{"Changes in household income are a potential channel trough which ML could have affected child health outcomes. This is discussed in section 7, but not linked to the previous findings in the literature about the effect of ML on mother's income (reported in section 3). By the way, the explanation about that increase is not very clear."}
\end{quote}
\underline{Response:} I added the missing link concerning the 1979 ML reform effect on maternal available income (see page \pageref{rev_mlch: r2_maternal_income_missing_link}). 

Furthermore, I also address the unclear explanation for the increase in average available income in section \ref{sec_mlch: info_ml_reform}. After having reformulated the reason, I moved it into Footnote \ref{rev_mlch: R2_increase_income} (page \pageref{rev_mlch: R2_increase_income}) to make the paragraph more concise. 


%-----------------------------------------
% number of observations in the tables
\begin{quote}
	\textit{"It  would be useful to provide the number of observations for each estimation in the tables"}
\end{quote}
\underline{Response:} I added the number of observations for each estimation in Tables \ref{tab_mlch: ITT_across_chapters_per_age_group_total} and \ref{tab_mlch: ITT_across_d5subcategories_per_age_group_total}. For Tables \ref{tab_mlch: DD_hopsital2_total}, \ref{tab_mlch: DD_hospital2_female_male}, \ref{tab_mlch: DD_d5_total}, and \ref{tab_mlch: DD_d5_female_male}, I refer to a separate Appendix Table \ref{tab_mlch: observations_age_brackets} in the footnotes. I thought that including the observations for each age bracket (in Panels B) would clutter up the Tables unnecessarily. Summing up the current format, the tables either contain the number of observations directly or provide a reference to where they can be found. If you feel that the number of observations should be in the aforementioned Tables, I can move them accordingly.




%-----------------------------------------
% 
\begin{quote}
	\textit{"There is no comment or discussion about why effects are found in men. May be other disciplines can bring some hypothesis about that?"}
\end{quote}
\underline{Response:}

%-----------------------------------------
% No discussion on other chapters
\begin{quote}
	\textit{"The paper focuses on MBD, but in table 3 significant and sizeable effects are found in Ext (may be related to risky behavior?) and Dig. It is not clear why these other chapters are not even discussed (even if they present unstable age patterns). "}
\end{quote}
\underline{Response:} I addressed your comment in two sections:

\textbf{Results section.---} When discussing the impact of the 1979 ML reform on other chapters, I added a short paragraph (page \pageref{rev_mlch: r2_results_other chapters}) briefly mentioning the effects on injuries, diseases of the digestive system, and respiratory maladies. Furthermore, I reformulated the first paragraph in the subsequent MBD chapter (section \ref{rev_mlch: r2_section_MBD}) to justify the focus on MBDs by emphasizing the importance of the diagnosis chapter for the overall results.

\textbf{Discussion section.---} In the discussion on potential mechanisms, I added footnote \ref{rev_mlch: r2_discussion_other_chapters} (page \pageref{rev_mlch: r2_discussion_other_chapters}), which is giving a potential explanation for the effects on the other diagnosis chapters (see above). In the footnote, I cite studies showing that breastfeeding is linked with lower willingness to take risks as well as lower likelihoods of asthma and inflammatory bowel disease.


XXX Unsure what you meant by the other chapters are not even discussed. More discussion of estimates or why do we see the effect for the other chapters?
Can I include such a statement? XXX

%-----------------------------------------
% structure of the paper
\begin{quote}
	\textit{"In terms of the structure of the paper, there is an overlapping between the discussion in Section 3, which presents the conceptual framework for long run health effects, and section 7, where the potential mechanisms through which the extension of ML could have affected health outcomes. Maybe those two sections can be merged and become just one (substituting section 3?), and also a more concise one. The first paragraphs in section 7 summarize the paper, they could excluded in the final version of the paper."}
\end{quote}
\underline{Response:} 

\textbf{Combining sections.---} Implemented as requested. I split up the discussion into two parts to have two independent topics. The first part (page \pageref{sec_mlch:discussion}, formerly section 7) covers mechanisms: suggestive evidence for maternal time using the urban/rural sample split and the theoretic discussion about other mechanisms. The second part (page \pageref{rev_mlch: restructure_discussion_framework}, formerly section 3) presents the conceptual framework of how changes in the mediating outcomes affect the children.

\textbf{Exclude first paragraphs of discussion.---} As mentioned above, the first paragraphs present evidence for maternal time as a potential channel by exploiting the urban/rural sample split. I restructured the paragraphs to show that they contain new content instead of a mere summary of the paper. Please also see my response to your first point.

 

