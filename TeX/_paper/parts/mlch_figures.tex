
%WMWMWMWMWMWMWMWMWMWMWMWMWMWMWMWMWMWMWMWM
% DATA
%WMWMWMWMWMWMWMWMWMWMWMWMWMWMWMWMWMWMWMWM

% figure: matrix descriptive hospital admission
\newpage
\vspace*{\fill}
\begin{figure}[H]\centering
	\caption{Hospital admissions}\label{fig_mlch: descriptive_hospital_admission}
	\includegraphics[width=\linewidth]{paper/descriptive_admission_TCG.pdf}
	\begin{minipage}{\linewidth}
		\scriptsize{\emph{Notes:} The figure depicts the evolution of key variables for the treatment and control cohort (only pre-threshold months, i.e. for individuals born between November 1977 and April 1978 as well as November 1978 and April 1979, respectively) over the period from 1996 to 2014. The dark lines correspond to the treatment cohorts, whereas control units are marked by light dashed lines. Hospital admissions are defined as the sum of all diagnosis chapters listed in Panel A of Table \ref{tab_mlch:outcomes_coding_main_chapters}. The x-axis shows in addition to the year the age of the treatment cohort in brackets.} 
	\end{minipage}
\end{figure}
\vspace*{\fill}\clearpage
%--------------------------------------------









%WMWMWMWMWMWMWMWMWMWMWMWMWMWMWMWMWMWMWMWM
% RESULTS
%WMWMWMWMWMWMWMWMWMWMWMWMWMWMWMWMWMWMWMWM
%--------------------------------------------
\newpage
%Life-course Hospital admission - commented out during revision1 to have parallel trends with the figures
%\begin{landscape}
%	\vspace*{\fill}
%	\begin{figure}[H]\centering
%		\caption{Life-course approach for \textbf{hospital admission}}\label{fig_mlch: lc_hospital2_frg_DD}
%		\begin{subfigure}[h]{0.31\linewidth}\centering\caption{Total}
%			\includegraphics[width=\linewidth]{paper/lc_hospital2_total_gdr.pdf}
%		\end{subfigure}
%		\begin{subfigure}[h]{0.31\linewidth}\centering\caption{Women}
%			\includegraphics[width=\linewidth]{paper/lc_hospital2_female_gdr.pdf}
%		\end{subfigure}
%		\begin{subfigure}[h]{0.31\linewidth}\centering\caption{Men}
%			\includegraphics[width=\linewidth]{paper/lc_hospital2_male_gdr.pdf}
%		\end{subfigure}
%		\scriptsize
%		\begin{minipage}{\linewidth}
%			\emph{Notes:} The figures plot DiD estimates (along with 90\% and 95\% confidence intervals) for the impact of the reform on hospital admission over the life-course. The light gray line in the background represents the baseline mean of the pre-reform treated cohort. The outcomes are defined as the number of cases per 1,000 individuals. Panel a shows the results for all admissions, whereas panel b and c show the estimates for females and males, respectively. The control group is comprised of children that are born in the same months, but one year before the reform (i.e. children born between November 1977 and October 1978).
%		\end{minipage}
%	\end{figure}
%	\vspace*{\fill}\clearpage
%\end{landscape}


%Life-course Hospital admission WITH PARALLEL TRENDS
\newgeometry{left=1cm,right=1cm,top=3cm,bottom=3cm} 
\begin{landscape}
	\vspace*{\fill}
	\begin{figure}[H]\centering
		\caption{Life-course approach for \textbf{hospital admission}}\label{fig_mlch: lc_hospital2_frg_DD}
		\begin{subfigure}[h]{0.31\linewidth}\centering\caption{Total}
			\includegraphics[width=\linewidth]{paper/mlch_lc_trends_hospital2.pdf}
		\end{subfigure}
		\begin{subfigure}[h]{0.31\linewidth}\centering\caption{Women}
			\includegraphics[width=\linewidth]{paper/mlch_lc_trends_hospital2_f.pdf}
		\end{subfigure}
		\begin{subfigure}[h]{0.31\linewidth}\centering\caption{Men}
			\includegraphics[width=\linewidth]{paper/mlch_lc_trends_hospital2_m.pdf}
		\end{subfigure}
	
		\scriptsize
		\begin{minipage}{\linewidth}
			\emph{Notes:} The top panels show pre-threshold (born November-April) means for the treatment (1978/79, dashed line) and control cohort (1977/78, solid line) across the years. The bottom panels plot DiD estimates (along with 90\% and 95\% confidence intervals) for the impact of the reform on hospital admissions over the life-course. For a given reporting year ($N=24$), I estimate the model in equation \ref{eq_mlch:DD_basline} (without $\rho_t$) and plot the DiD estimate and the corresponding confidence interval for that year. The outcomes are defined as the number of cases per 1,000 individuals. Column a shows the results for all admissions, whereas columns b and c show the estimates for females and males, respectively.
		\end{minipage}
	\end{figure}
	\vspace*{\fill}\clearpage
\end{landscape}
\restoregeometry


%--------------------------------------------
% HOPSITAL2 - RD plots
%
% Hospital - Reduced form pooled
\newgeometry{left=1cm,right=1cm,top=3cm,bottom=3cm} 
\begin{landscape}
	\vspace*{\fill}
	\begin{figure}
		[H]\centering
		\caption{RD plots for hospital admission (pooled)}\label{fig: rf_hospital2_pooled}
		\begin{subfigure}[h]{0.31\linewidth}\centering\caption{Total}
			\includegraphics[width=\linewidth]{paper/rd_hospital2_total_pooled.pdf}
		\end{subfigure}
		\begin{subfigure}[h]{0.31\linewidth}\centering\caption{Women}
			\includegraphics[width=\linewidth]{paper/rd_hospital2_female_pooled.pdf}
		\end{subfigure}
		\begin{subfigure}[h]{0.31\linewidth}\centering\caption{Men}
			\includegraphics[width=\linewidth]{paper/rd_hospital2_male_pooled.pdf}
		\end{subfigure}
		\scriptsize
		\begin{minipage}{0.95\linewidth}
			\emph{Notes:} The figure plots the average number of diagnoses per 1,000 individuals for month-of-birth cohorts born half a year around the cut-off date of the 1979 maternity leave expansion. The monthly averages are taken over the entire sample length from 1995 to 2014. The dashed lines represent linear fitted values along with 90\%/95\% confidence intervals. The solid vertical red line divides pre- and post-reform schemes (two vs. six months of leave).\newline
			\emph{Source:} Hospital registry data for individuals born between November 1978 and October 1979.
		\end{minipage}
	\end{figure}
	\vspace*{\fill}\clearpage
\end{landscape}
\restoregeometry  
%--------------------------------------------
% Hospital -Reduced form AGE groups 
% original 3 3 2 3
\newgeometry{left=1cm,right=1cm,top=3cm,bottom=3cm} 
\begin{landscape}
	\vspace*{\fill}
	\begin{figure}
		[H]\centering
		\caption{RD plots for hospital admission across age groups}\label{fig: rf_hospital2_agegroup}
		\includegraphics[width=0.85\linewidth]{paper/rd_r_fert_hospital2_overview_agegroups_CIfits.pdf}
		\scriptsize
		\begin{minipage}{0.9\linewidth}
			\emph{Notes:} The figure plots the number of diagnoses per 1,000 individuals for month-of-birth cohorts born half a year around the cut-off date of the 1979 maternity leave expansion across gender and different age groups. The first column reports the ratios for all patients, and the second and third column do so for women and men, respectively. The rows show the ratios across different age groups. The dashed lines represent linear fitted values along with 90\%/95\% confidence intervals. The solid vertical red line divides pre- and post-reform schemes (two vs. six months of leave).\newline
			\emph{Source:} Hospital registry data for individuals born between November 1978 and October 1979.
		\end{minipage}
	\end{figure}
	\vspace*{\fill}\clearpage
\end{landscape}
\restoregeometry

%--------------------------------------------
% Figure: effect sizes and frequency across chapters

\begin{landscape}
	\vspace*{\fill}
	\begin{figure}[H]\centering
		\caption{Intention-to-treat effects across \textbf{main diagnosis chapters}}\label{fig_mlch: DD_across_main chapters}
		\begin{subfigure}[h]{0.31\linewidth}\centering\caption{Total}
			\includegraphics[width=\linewidth]{paper/effect_chapters_frequency.pdf}
		\end{subfigure}
		\begin{subfigure}[h]{0.31\linewidth}\centering\caption{Women}
			\includegraphics[width=\linewidth]{paper/effect_chapters_frequency_f.pdf}
		\end{subfigure}
		\begin{subfigure}[h]{0.31\linewidth}\centering\caption{Men}
			\includegraphics[width=\linewidth]{paper/effect_chapters_frequency_m.pdf}
		\end{subfigure}
		\scriptsize
		\begin{minipage}{\linewidth}
			\emph{Notes:} The figures plot intention-to-treat estimates (along with 90\%/95\% confidence intervals) across the main diagnosis chapters. Furthermore, they indicate how often each chapter is diagnosed over the entire time frame (1995-2014). The outcomes are defined as the number of cases per 1,000 individuals. The point estimates are coming from a DiD regression as described in section \ref{sec_mlch:empirical_strategy}, with a bandwidth of six months, month-of-birth and year fixed effects, and clustered standard errors on the month-of-birth level. The control group is comprised of children that are born in the same months but one year before the reform (i.e. children born between November 1977 and October 1978). \newline
			\emph{Legend:} Infectious and parasitic diseases (IPD), neoplasms (Neo), mental and behavioral disorders (MBD), diseases of the nervous system (Ner), diseases of the sense organs (Sen), diseases of the circulatory system (Cir), diseases of the respiratory system (Res), diseases of the digestive system (Dig), diseases of the skin and subcutaneous tissue (SST), diseases of the musculoskeletal system (Mus), diseases of the genitourinary system (Gen), symptoms, signs, and ill-defined conditions (Sym), injury, poisoning and certain other consequences of external causes (Ext).
			
		\end{minipage}
	\end{figure}
	\vspace*{\fill}\clearpage
\end{landscape}
%--------------------------------------------

% D5 - LC Approach (Mental and behavioral disod)  - commented out during revision1 to have parallel trends with the figures
%\begin{landscape}
%	\vspace*{\fill}
%	\begin{figure}[H]\centering
%		\caption{Life-course approach for \textbf{mental and behavioral disorders}}\label{fig_mlch: lc_d5_frg_DD}
%		\begin{subfigure}[h]{0.31\linewidth}\centering\caption{Total}
%			\includegraphics[width=\linewidth]{paper/lc_d5_total_gdr.pdf}
%		\end{subfigure}
%		\begin{subfigure}[h]{0.31\linewidth}\centering\caption{Women}
%			\includegraphics[width=\linewidth]{paper/lc_d5_female_gdr.pdf}
%		\end{subfigure}
%		\quad
%		\begin{subfigure}[h]{0.31\linewidth}\centering\caption{Men}
%			\includegraphics[width=\linewidth]{paper/lc_d5_male_gdr.pdf}
%		\end{subfigure}
%		\scriptsize
%		\begin{minipage}{\linewidth}
%			\emph{Notes:} The figures plot DiD estimates (along with 90\% and 95\% confidence intervals) for the impact of the reform on mental and behavioral disorders over the life-course. The light gray line in the background represents the baseline mean of the pre-reform treated cohort. The outcomes are defined as the number of cases per 1,000 individuals. Panel a shows the results for all admissions, whereas panel b and c show the estimates for females and males, respectively. The control group is comprised of children	that are born in the same months but one year before (i.e. children born between November 1977 and October 1978).
%		\end{minipage}
%	\end{figure}
%	\vspace*{\fill}\clearpage
%\end{landscape}






% D5 - LC Approach (Mental and behavioral disod) - with parallel trends
\newgeometry{left=1cm,right=1cm,top=3cm,bottom=3cm} 
\begin{landscape}
	\vspace*{\fill}
	\begin{figure}[H]\centering
		\caption{Life-course approach for \textbf{mental and behavioral disorders}}\label{fig_mlch: lc_d5_frg_DD}
		\begin{subfigure}[h]{0.31\linewidth}\centering\caption{Total}
			\includegraphics[width=\linewidth]{paper/mlch_lc_trends_d5.pdf}
		\end{subfigure}
		\begin{subfigure}[h]{0.31\linewidth}\centering\caption{Women}
			\includegraphics[width=\linewidth]{paper/mlch_lc_trends_d5_f.pdf}
		\end{subfigure}
		\begin{subfigure}[h]{0.31\linewidth}\centering\caption{Men}
			\includegraphics[width=\linewidth]{paper/mlch_lc_trends_d5_m.pdf}
		\end{subfigure}
		\scriptsize
		\begin{minipage}{\linewidth}
			\emph{Notes:} The top panels show pre-threshold (born November-April) means for the treatment (1978/79, dashed line) and control cohort (1977/78, solid line) across the years. The bottom panels plot DiD estimates (along with 90\% and 95\% confidence intervals) for the impact of the reform on mental and behavioral disorders over the life-course. For a given reporting year ($N=24$), I estimate the model in equation \ref{eq_mlch:DD_basline} (without $\rho_t$) and plot the DiD estimate and the corresponding confidence interval for that year. The outcomes are defined as the number of cases per 1,000 individuals. Column a shows the results for all admissions, whereas columns b and c show the estimates for females and males, respectively.
		\end{minipage}
	\end{figure}
	\vspace*{\fill}\clearpage
\end{landscape}
\restoregeometry



%--------------------------------------------
% D5 RD plots
%
% d5 - RF pooled
\newgeometry{left=1cm,right=1cm,top=3cm,bottom=3cm} 
\begin{landscape}
	\vspace*{\fill}
	\begin{figure}
		[H]\centering
		\caption{RD plots for mental \& behavioral disorders (pooled)}\label{fig: rf_d5_pooled}
		\begin{subfigure}[h]{0.31\linewidth}\centering\caption{Total}
			\includegraphics[width=\linewidth]{paper/rd_d5_total_pooled.pdf}
		\end{subfigure}
		\begin{subfigure}[h]{0.31\linewidth}\centering\caption{Women}
			\includegraphics[width=\linewidth]{paper/rd_d5_female_pooled.pdf}
		\end{subfigure}
		\begin{subfigure}[h]{0.31\linewidth}\centering\caption{Men}
			\includegraphics[width=\linewidth]{paper/rd_d5_male_pooled.pdf}
		\end{subfigure}
		\scriptsize
		\begin{minipage}{0.95\linewidth}
			\emph{Notes:} The figure plots the average number of diagnoses per 1,000 individuals for month-of-birth cohorts born half a year around the cut-off date of the 1979 maternity leave expansion. The monthly averages are taken over the entire sample length from 1995 to 2014. The dashed lines represent linear fitted values along with 90\%/95\% confidence intervals. The solid vertical red line divides pre- and post-reform schemes (two vs. six months of leave).\newline
			\emph{Source:} Hospital registry data for individuals born between November 1978 and October 1979.
		\end{minipage}
	\end{figure}
	\vspace*{\fill}\clearpage
\end{landscape}
\restoregeometry 




%--------------------------------------------
% D5 - RF (age group)
\newgeometry{left=1cm,right=1cm,top=3cm,bottom=3cm} 
\begin{landscape}
	\vspace*{\fill}
	\begin{figure}
		[H]\centering
		\caption{RD plots for mental \& behavioral disorders across age groups}\label{fig: rf_d5_agegroup}
		\includegraphics[width=0.85\linewidth]{paper/rd_r_fert_d5_overview_agegroups_CIfits.pdf}
		\scriptsize
		\begin{minipage}{0.9\linewidth}
			\emph{Notes:} The figure plots the number of diagnoses per 1,000 individuals for month-of-birth cohorts born half a year around the cut-off date of the 1979 maternity leave expansion across gender and different age groups. The first column reports the ratios for all patients, and the second and third column do so for women and men, respectively. The rows show the ratios across different age groups. The dashed lines represent linear fitted values along with 90\%/95\% confidence intervals. The solid vertical red line divides pre- and post-reform schemes (two vs. six months of leave).\newline
			\emph{Source:} Hospital registry data for individuals born between November 1978 and October 1979.
		\end{minipage}
	\end{figure}
	\vspace*{\fill}\clearpage
\end{landscape}
\restoregeometry

 

%--------------------------------------------
% figure: iverview of d5 subcategories (effects + frequency)
\newgeometry{left=1cm,right=1cm,top=3cm,bottom=3cm} 
\begin{landscape}
	\vspace*{\fill}
	\begin{figure}
		[H]\centering
		\caption{ITT effect for \textbf{subcategories of mental and behavioral disorders (pooled)}}\label{fig_mlch: ITT_d5_subcategories}
		\begin{subfigure}[h]{0.31\linewidth}\centering\caption{Total}
			\includegraphics[width=\linewidth]{paper/effect_d5_frequency.pdf}
		\end{subfigure}
		\begin{subfigure}[h]{0.31\linewidth}\centering\caption{Women}
			\includegraphics[width=\linewidth]{paper/effect_d5_frequency_f.pdf}
		\end{subfigure}
		\begin{subfigure}[h]{0.31\linewidth}\centering\caption{Men}
			\includegraphics[width=\linewidth]{paper/effect_d5_frequency_m.pdf}
		\end{subfigure}
		\scriptsize
		\begin{minipage}{0.95\linewidth}
			\emph{Notes:} The figure plots ITT estimates (along with 90\%/95\% confidence intervals) across the five most common subcategories of MBDs. Moreover, they indicate how often each subcategory is diagnosed over the time window of 1995-2014. The outcomes are defined as the number of cases per 1,000 individuals. The point estimates are coming from a DiD regression as described in section \ref{sec_mlch:empirical_strategy}, with a bandwidth of six months, month-of-birth and year fixed effects, and standard errors clustered at the month-of-birth level. The control group is comprised of children that are born in the same months but one year before the reform (i.e. children born between November 1977 and October 1978).\newline
			% \emph{Source:} Hospital registry data.
		\end{minipage}
	\end{figure}
	\vspace*{\fill}\clearpage
\end{landscape}
\restoregeometry 
