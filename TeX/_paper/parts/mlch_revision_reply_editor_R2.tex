
\clearpage
\section{Comments made by the editor and reviewers}


%-----------------------------------------
% OECD Data Introduction
\phantomsection
\addcontentsline{toc}{subsection}{C.1: OECD Data Introduction}
\begin{quote}
\textit{C.1: "1st paragraph, Introduction: According to the data source you provide https://www.oecd.org/gender/data/length-of-maternity-leave-parental-leave-and-paid-father-specific-leave.htm, the average length of paid maternity and parental leave across all OECD countries was 19.08 weeks in 2016 (not 55.2 as you mention). Not sure what the average for these same countries was in 1970, as OECD countries have changed over time, but it seems to be below 16.8. Please revise and choose the same countries over time to make the comparison."}
\end{quote}

\underline{Response:} 





%-----------------------------------------
% Label of Figure
\phantomsection
\addcontentsline{toc}{subsection}{C.2: Label of Figure}
\bigskip
\begin{quote}
	\textit{C.2: "Figure 1. I would put “age” in the horizontal axis, rather than year (age). The year(age) labels are confusing, as they are different for treated and control groups."}
\end{quote}
\underline{Response:}











%-----------------------------------------
% Rearrange empirical strategy section
\phantomsection
\bigskip
\addcontentsline{toc}{subsection}{C.3: Rearrange Empirical Strategy Section}
\begin{quote}
	\textit{C.3: "Section 4. Empirical Strategy, 2nd paragraph. You claim that seasonality issues are the main reason why a RD deign is not feasible. But seasonality issues would easily be solved by a DiD RD. Your main problem is the imprecision of DiD RD estimates. I would rearrange this paragraph so this is clear (maybe shifting the details about the seasonality discussion to a footnote and shifting the discussion in footnote 28 to the paragraph."}
\end{quote}
\underline{Response:}











%-----------------------------------------
% Effect on SST
\phantomsection
\bigskip
\addcontentsline{toc}{subsection}{C.4: Effect on SST}
\begin{quote}
	\textit{C.4: "Table 3. You find a positive effect on SST, which is somehow puzzling. Could you provide any intuition for that result? You should at least acknowledge it is unexpected and goes counter to your hypothesis."}
\end{quote}
\underline{Response:}











%-----------------------------------------
% Effect of Foreigners
\phantomsection
\bigskip
\addcontentsline{toc}{subsection}{C.5: ITT and the Effect of Foreigners}
\begin{quote}
	\textit{C.5: "Reviewer 2’s comment about being unable to identify foreigners. You respond that seeking that data would take substantial time. I am fine with it, but you should mention this as a caveat in your discussion or concluding remarks. In any case, it would suggest results are larger than what you are finding due to the dilution effect of migrants."}
\end{quote}
\underline{Response:} 







%-----------------------------------------
% Formulation selective fertility
\phantomsection
\bigskip
\addcontentsline{toc}{subsection}{C.6: Reformulation in the Empirical Strategy Section}
\begin{quote}
	\textit{C.6: "Section 6.2 Potential Mechanisms. When addressing gender differences you mention several possible explanations that have to do with parents' differential attention according to gender, but leave aside one explanation you raised before when explaining the results by gender: that men are more likely to suffer from externalizing problem. Parents could have paid the same attention to boys and girls, but girls are more likely to suffer from internalizing problems, and thus may be less likely to end up being hospitalized for that reason (or the larger incidence of depression or anxiety may be seen later on, not as young as the age of 35). I think this latter explanation is more feasible than the former ones you raise in the paragraph."}
\end{quote}
\underline{Response:}



