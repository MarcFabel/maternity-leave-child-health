
\clearpage
\section{Comments}


%-----------------------------------------
% OECD Data Introduction
\phantomsection
\addcontentsline{toc}{subsection}{C.1: OECD Data in the Introduction}
\begin{quote}
\textit{C.1: "1st paragraph, Introduction: According to the data source you provide https://www.oecd.org/gender/data/length-of-maternity-leave-parental-leave-and-paid-father-specific-leave.htm, the average length of paid maternity and parental leave across all OECD countries was 19.08 weeks in 2016 (not 55.2 as you mention). Not sure what the average for these same countries was in 1970, as OECD countries have changed over time, but it seems to be below 16.8. Please revise and choose the same countries over time to make the comparison."}
\end{quote}

\underline{Response:} 

I am afraid I provided the intermediate link to the database I use. I apologize for the confusion this may have caused. In order to obtain the numbers I cite in the introduction, you have to click on \textbf{$\rightarrow$ Download data} at the bottom of the page. This forwards you to the database that provides the desired figures. Since my last visit, the OECD enlarged the set of available countries to 36 member states, which slightly changes the numbers (see Table below). I have adjusted the introduction accordingly (page 1).
 
\begin{table}[H] \centering 
	\begin{threeparttable} \centering \caption{Length of paid maternity and parental leave}\label{rev_mlch: }
		\begin{tabular*}{.7\linewidth}{@{\extracolsep{\fill}}l*{2}{c}}
			\toprule
			Year & Average & Median \\
			\midrule
			1970	&	14.0	&	12.0 \\
			2016	&	55.4	&	45.4 \\
			\bottomrule
		\end{tabular*}
		\begin{tablenotes} 
			\item \scriptsize \emph{Notes:} The table shows the average and median length of paid maternity and parental leave (in weeks) for the 36 OECD countries. The indicator is set to `Total length of paid maternity and parental leave'.\newline \emph{Source: OECD, Employment  : Length of maternity leave, parental leave, and paid father-specific leave.}\newline \url{https://stats.oecd.org/index.aspx?queryid=54760}
		\end{tablenotes}
	\end{threeparttable}
\end{table} 
 
 
The Database reports values for all 36 member states independent of when a country joined the organization. In other words, the averages refer to the same set of countries over time. 

To avoid this confusion, I updated the link provided in the list of references \citep{oecd_data_leave}.











%-----------------------------------------
% Label of Figure
\phantomsection
\addcontentsline{toc}{subsection}{C.2: Label of Figure 1}
\bigskip
\begin{quote}
	\textit{C.2: "Figure 1. I would put “age” in the horizontal axis, rather than year (age). The year(age) labels are confusing, as they are different for treated and control groups."}
\end{quote}
\underline{Response:}

Adjusted, as requested (see Figure \ref{fig_mlch: descriptive_hospital_admission}, page \pageref{fig_mlch: descriptive_hospital_admission}). Moreover, to have a coherent label throughout the paper, I have also adjusted the horizontal axis in the Figures \ref{fig_mlch: lc_hospital2_frg_DD} and \ref{fig_mlch: lc_d5_frg_DD}. 






%-----------------------------------------
% Rearrange empirical strategy section
\phantomsection
\bigskip
\addcontentsline{toc}{subsection}{C.3: Rearrange Empirical Strategy Section}
\begin{quote}
	\textit{C.3: "Section 4. Empirical Strategy, 2nd paragraph. You claim that seasonality issues are the main reason why a RD deign is not feasible. But seasonality issues would easily be solved by a DiD RD. Your main problem is the imprecision of DiD RD estimates. I would rearrange this paragraph so this is clear (maybe shifting the details about the seasonality discussion to a footnote and shifting the discussion in footnote 28 to the paragraph."}
\end{quote}
\underline{Response:}


As suggested, I have rearranged the empirical strategy section (page \pageref{sec_mlch:empirical_strategy}) by shifting the information on the precision problems of discontinuity designs (formerly footnote 28) to the main text and details about season of birth effects to a footnote (footnote \ref{rev_mlch: SOB_in_footnote}). When discussing season of birth effects, I explicitly mention that a difference-in-discontinuity design would be a candidate for identification, but must be dismissed in the end due to the lack of precision. I hope the new structure makes everything clearer for the reader.






%-----------------------------------------
% Effect on SST
\phantomsection
\bigskip
\addcontentsline{toc}{subsection}{C.4: Unexpected Positive Effect on SST}
\begin{quote}
	\textit{C.4: "Table 3. You find a positive effect on SST, which is somehow puzzling. Could you provide any intuition for that result? You should at least acknowledge it is unexpected and goes counter to your hypothesis."}
\end{quote}
\underline{Response:} 

Unfortunately, I cannot provide a satisfactory explanation for the increase in diseases of the skin and subcutaneous tissue. As suggested, I have acknowledged the unexpected increase in footnote \ref{rev_mlch: SST} (page \pageref{rev_mlch: SST}).








%-----------------------------------------
% Effect of Foreigners
\phantomsection
\bigskip
\addcontentsline{toc}{subsection}{C.5: ITT Estimates and the Effects of Foreigners}
\begin{quote}
	\textit{C.5: "Reviewer 2’s comment about being unable to identify foreigners. You respond that seeking that data would take substantial time. I am fine with it, but you should mention this as a caveat in your discussion or concluding remarks. In any case, it would suggest results are larger than what you are finding due to the dilution effect of migrants."}
\end{quote}
\underline{Response:} 



To cover the impact of unaffected individuals (e.g. foreigners) on the ITT estimates, I have added footnote \ref{rev_mlch:fn_ITT} (page \pageref{rev_mlch:fn_ITT}) in the conclusion.



%-----------------------------------------
% Formulation selective fertility
\phantomsection
\bigskip
\addcontentsline{toc}{subsection}{C.6: Reformulation in the Empirical Strategy Section}
\begin{quote}
	\textit{C.6: "Section 6.2 Potential Mechanisms. When addressing gender differences you mention several possible explanations that have to do with parents' differential attention according to gender, but leave aside one explanation you raised before when explaining the results by gender: that men are more likely to suffer from externalizing problem. Parents could have paid the same attention to boys and girls, but girls are more likely to suffer from internalizing problems, and thus may be less likely to end up being hospitalized for that reason (or the larger incidence of depression or anxiety may be seen later on, not as young as the age of 35). I think this latter explanation is more feasible than the former ones you raise in the paragraph."}
\end{quote}
\underline{Response:}

I have added your proposed explanation in section 6.2 (page \pageref{rev_mlch:gender_effects}). Furthermore, I have deleted the paragraph containing parents' differential attention according to gender.


