%--------------------------------------------
% Overview Hopsital admission
\newgeometry{left=0cm,right=0cm,top=0cm,bottom=3cm} 

\begin{landscape}
	\vspace*{\fill}
	\begin{table}[h] \centering % table environment for caption and label
		\begin{threeparttable}
			\centering % center the tabular
			\caption{Summary statistics for different diagnoses} % caption
			\label{tab_mlch:outcomes_coding_main_chapters} 
			\begin{tabular}{lrrrrrr} % alignment and number of columns of actual table
				\toprule % top thicker horizontal line (" rule ")
				&\multicolumn{1}{r}{(1)}& &\multicolumn{1}{r}{(2)}&\multicolumn{1}{r}{(3)} &\multicolumn{1}{r}{(4)}\\
				&\multicolumn{1}{r}{ICD-9} & & \multicolumn{1}{r}{ICD-10}&\multicolumn{1}{r}{Mean}&\multicolumn{1}{r}{SD} \\ 
				\midrule
				%-------------------------------------------------------------------------
				\\
\underline{\textit{A. Hospital admission}}									& 						& & 			 & 	 120.625     &  10.961\\ 
 \hspace{10pt} Infectious and parasitic diseases                           	&	001-139				& &		A00-B99  & 	   4.210     &   0.493\\
 \hspace{10pt} Neoplasms                                                   	&	140-239				& &		C00-D48  & 	   5.155     &   1.282\\
 \hspace{10pt} Mental \& behavioral  disorders                             	&	290-319				& &		F00-F99  & 	  18.956     &   5.548\\
 \hspace{10pt} Diseases of the nervous system                              	&	320-359				& &		G00-G99  & 	   4.500     &   1.264\\
 \hspace{10pt} Diseases of the sense organs                                	&	360-389				& &		H00-H95  & 	   2.404     &   0.348\\
 \hspace{10pt} Diseases of the circulatory system                          	&	390-459				& &		I00-I99  & 	   4.108     &   1.380\\
 \hspace{10pt} Diseases of the respiratory system                          	&	460-519				& &		J00-J99  & 	  10.994     &   1.939\\
 \hspace{10pt} Diseases of the digestive system                            	&	520-579				& &		K00-K93  & 	  16.746     &   2.079\\
 \hspace{10pt} Diseases of the skin and subcutaneous tissue                	&	680-709				& &		L00-L99  & 	   3.849     &   0.536\\
 \hspace{10pt} Diseases of the musculoskeletal system						&	710-739				& &		M00-M99  & 	   8.897     &   2.228\\
 \hspace{10pt} Diseases of the genitourinary system                        	&	580-629				& &		N00-N99  & 	  10.621     &   1.362\\
 \hspace{10pt} Symptoms, signs, and ill-defined conditions                 	&	780-799				& &		R00-R99  & 	   6.794     &   1.410\\
 \hspace{10pt} Injury, poisoning and certain other                          &	800-999				& &		S00-T98  & 	  21.196     &   5.978\\
 \hspace{18pt} consequences of external causes  \\
 
 %\hspace{4pt} Diseases of the blood and blood-forming organs              	&	280-289		& &		D50-D90  & \\
 %\hspace{4pt} Endocrine, nutritional and metabolic diseases				&	240-278		& &		E00-E90  & \\
 %\hspace{4pt} Pregnancy, childbirth, and the puerperium  					&	630-676		& &		O00-O99  & \\
 %\hspace{4pt} Certain conditions originating in the perinatal period      	&	760-779		& &		P00-P96  & \\
 %\hspace{4pt} Congenital anomalies                                        	&	740-759		& &		Q00-Q99  & \\
 
 
 % ------------------------------------------------------------------------
 % 							Mean           SD          Min          Max
 % ------------------------------------------------------------------------
 % Ratio using number~e      120.625       10.961        99.27       156.35
 % Ratio using number~e        4.210        0.493         2.87         5.65
 % Ratio using number~e        5.155        1.282         3.29         8.97
 % Ratio using number~e       18.956        5.548         6.68        30.17
 % Ratio using number~e        4.500        1.264         2.60         7.50
 % Ratio using number~e        2.404        0.348         1.62         3.07
 % Ratio using number~e        4.108        1.380         1.41         7.66
 % Ratio using number~e       10.994        1.939         8.30        15.55
 % Ratio using number~e       16.746        2.079        12.09        20.97
 % Ratio using number~e        3.849        0.536         2.63         5.12
 % Ratio using number~e        8.897        2.228         6.45        15.46
 % Ratio using number~e       10.621        1.362         6.79        13.64
 % Ratio using number~e        6.794        1.410         4.26        10.70
 % Ratio using number~e       21.196        5.978        14.41        34.06
 % ------------------------------------------------------------------------



				%-------------------------------------------------------------------------
				\bottomrule % bottom thicker horizontal line (" rule ")
			\end{tabular}
			\begin{tablenotes}
				\scriptsize{ \item \textit{Continued on next page}}
			\end{tablenotes}
		\end{threeparttable}
	\end{table}
	\vspace*{\fill}\clearpage 
	
	
	\vspace*{\fill}
	\begin{table}[h] \centering % table environment for caption and label
		\begin{threeparttable}
			\centering % center the tabular
			\caption*{\textit{Summary statistics for different diagnoses (continued)}} % caption

			\begin{tabular}{lrrrrrr} % alignment and number of columns of actual table
				\toprule % top thicker horizontal line (" rule ")
				&\multicolumn{1}{r}{(1)}& &\multicolumn{1}{r}{(2)}&\multicolumn{1}{r}{(3)} &\multicolumn{1}{r}{(4)}\\
				&\multicolumn{1}{r}{ICD-9} & & \multicolumn{1}{r}{ICD-10}&\multicolumn{1}{r}{Mean}&\multicolumn{1}{r}{SD} \\ 
				\midrule
				%-------------------------------------------------------------------------
				

\\
\underline{\textit{B. Mental \& behavioral disorders}}						&						& & 			 & 18.956		& 5.548		\\
 \hspace{10pt} Organic, including symptomatic, mental disorders				& 290,293,294,310		& & 	F00-F09	 &	0.115		& 0.056		\\
 \hspace{10pt} MBD due to psychoactive substance use$^1$					& 291,292,303,304,305	& & 	F10-F19	 &	6.366		& 2.232		\\
 \hspace{10pt} Schizophrenia, schizotypal and delusional disorders			& 295,297,298			& & 	F20-F29	 &	5.140		& 2.246		\\
 \hspace{10pt} Mood [affective] disorders									& 296,311				& & 	F30-F39	 &	2.339		& 1.673		\\
 \hspace{10pt} Neurotic, stress-related and somatoform disorders			& 300,306,308,309		& & 	F40-F48	 &	2.799		& 0.356		\\
 \hspace{10pt} Behavioural syndromes associated with 						& 316					& & 	F50-F59	 &	0.308		& 0.225		\\
 \hspace{18pt} physiological disturbances and physical factors 														 &				& 			\\
 \hspace{10pt} Disorders of adult personality and behavior					& 301,302				& & 	F60-F69	 &	1.375		& 0.511		\\
 \hspace{10pt} Mental retardation 											& 317,318,319			& &		F70-F79	 &	0.121		& 0.075		\\
 \hspace{10pt} Disorders of psychological development						& 299,315				& & 	F80-F89	 &	0.026		& 0.029		\\
 \hspace{10pt} Behavioural and emotional disorders with 					& 312,313,314,307		& & 	F90-F98	 &	0.320		& 0.535		\\
  \hspace{18pt} onset usually occurring in childhood and adolescence 																			\\


% var			mean	SD
% d5						18.956	5.548
% organic					0.115	0.056
% psychoactive substances	6.366	2.232
% schizophrenia			5.140	2.246
% affective				2.339	1.673
% neurosis				2.799	0.356
% physical factors		0.308	0.225
% personality				1.375	0.511
% retardation				0.121	0.075
% development				0.026	0.029
% childhood				0.320	0.535
		


				%-------------------------------------------------------------------------
				\bottomrule % bottom thicker horizontal line (" rule ")
			\end{tabular}
			\begin{tablenotes}
				\scriptsize{ \item \textit{Notes:} The table shows the classification of diseases according to the `International Statistical Classification of Diseases and Related Health Problems (ICD)', a medical classification list provided by the World Health Organization. For the main chapters, the mapping between ICD-9 and ICD-10 is taken from the European shortlist. The table reports next to the ICD codes summary statistics for the different diagnosis types. Column 3 and 4 show mean and standard deviation of the number of diagnoses per 1,000 individuals for the pre-reform treatment cohort. The data uses the modified version of the ICD system issued by DIMDI.\newline \textit{Source:} The ICD coding is taken from the World Health Organization (WHO), see for example: \href{http://www.who.int/classifications/icd/en/}{http://www.who.int/classifications/icd/en/} and the European shortlist \citep[p. 76]{statistisches2012diagnosedaten}, the summary statistics are obtained from the hospital registry data. \newline\hspace*{15 pt}$^1$ Psychoactive substances include alcohol, opioids, cannabinoids, sedatives or hypnotics, cocaine, other stimulants (including caffeine), hallucinogens, tobacco, volatile solvents, multiple drug use and use of other psychoactive substances. }
			\end{tablenotes}
		\end{threeparttable}
	\end{table}
	\vspace*{\fill}\clearpage 
\end{landscape}
\restoregeometry








%--------------------------------------------
% Interaction Treatment & Age Groups for Hospital2
\newgeometry{left=1cm,right=1cm,top=0cm,bottom=0cm} 
\vspace*{\fill}
\begin{table}[H] \centering 
	\begin{threeparttable} \centering \caption{Robustness: Interaction Treatment with Age Brackets (hospitalization)}\label{tab_mlch: interaction_TxA_agegroups_hospital2}
		{\def\sym#1{\ifmmode^{#1}\else\(^{#1}\)\fi} 
			\begin{tabular}{l*{6}{c}}
				\toprule 
				%\multicolumn{5}{l}{Dependant variable: \textbf{Hospital admission (total)}}\\ \\ 
				& \multicolumn{5}{c}{Estimation window} \\ 
				\cmidrule(lr){2-6}
				&\multicolumn{1}{c}{(1)}&\multicolumn{1}{c}{(2)}&\multicolumn{1}{c}{(3)}&\multicolumn{1}{c}{(4)}&\multicolumn{1}{c}{(5)}\\
				&\multicolumn{1}{c}{6M}&\multicolumn{1}{c}{5M}&\multicolumn{1}{c}{4M}&\multicolumn{1}{c}{3M}&\multicolumn{1}{c}{Donut}\\
				\midrule
				\multicolumn{5}{l}{\emph{Panel A. Total}} \\
				Treatment $\times$ Age 17-21&      -2.310\sym{***}&      -2.143\sym{**} &      -2.156\sym{*}  &      -2.386\sym{*}  &      -2.647\sym{***}\\
                    &     (0.729)         &     (0.829)         &     (1.039)         &     (1.309)         &     (0.715)         \\
Treatment $\times$ Age 22-26&     -0.0735         &      0.0501         &     -0.0501         &    -0.00689         &      -0.451         \\
                    &     (0.764)         &     (0.916)         &     (1.149)         &     (1.422)         &     (0.788)         \\
Treatment $\times$ Age 27-31&      -2.187\sym{**} &      -1.789         &      -2.350         &      -2.323         &      -2.644\sym{**} \\
                    &     (0.977)         &     (1.134)         &     (1.379)         &     (1.696)         &     (1.064)         \\
Treatment $\times$ Age 32-35&      -4.148\sym{***}&      -4.040\sym{***}&      -4.639\sym{***}&      -4.620\sym{**} &      -5.060\sym{***}\\
                    &     (1.021)         &     (1.164)         &     (1.390)         &     (1.747)         &     (1.015)         \\
\midrule Dependent mean&       121.1         &       121.0         &       121.5         &       123.3         &       121.9         \\
\(N\) (MOB $\times$ year)&         456         &         380         &         304         &         228         &         380         \\
 \\ \\
				
				\multicolumn{5}{l}{\emph{Panel B. Women}} \\
				Treatment $\times$ Age 17-21&      -0.879         &      -0.607         &       0.153         &      -0.292         &      -1.370         \\
                    &     (0.992)         &     (1.048)         &     (1.176)         &     (1.560)         &     (0.946)         \\
Treatment $\times$ Age 22-26&       0.206         &       0.880         &       1.487         &       1.234         &       0.177         \\
                    &     (0.943)         &     (1.058)         &     (1.278)         &     (1.698)         &     (0.872)         \\
Treatment $\times$ Age 27-31&      -3.083\sym{***}&      -2.204\sym{*}  &      -1.923         &      -2.429         &      -3.404\sym{***}\\
                    &     (1.060)         &     (1.119)         &     (1.338)         &     (1.790)         &     (1.035)         \\
Treatment $\times$ Age 32-35&      -3.580\sym{***}&      -3.401\sym{**} &      -2.921\sym{*}  &      -2.239         &      -4.533\sym{***}\\
                    &     (1.191)         &     (1.347)         &     (1.586)         &     (1.830)         &     (1.153)         \\
\midrule Dependent mean&       122.3         &       121.9         &       121.9         &       123.8         &       123.2         \\
\(N\) (MOB $\times$ year)&         456         &         380         &         304         &         228         &         380         \\
 \\ \\
				
				\multicolumn{5}{l}{\emph{Panel C. Men}} \\
				Treatment $\times$ Age 17-21&      -3.738\sym{***}&      -3.635\sym{***}&      -4.360\sym{***}&      -4.397\sym{**} &      -3.937\sym{***}\\
                    &     (0.984)         &     (1.110)         &     (1.317)         &     (1.561)         &     (1.133)         \\
Treatment $\times$ Age 22-26&      -0.352         &      -0.751         &      -1.513         &      -1.193         &      -1.060         \\
                    &     (1.072)         &     (1.264)         &     (1.521)         &     (1.696)         &     (1.212)         \\
Treatment $\times$ Age 27-31&      -1.325         &      -1.402         &      -2.759         &      -2.224         &      -1.908         \\
                    &     (1.318)         &     (1.585)         &     (1.811)         &     (1.914)         &     (1.548)         \\
Treatment $\times$ Age 32-35&      -4.679\sym{***}&      -4.648\sym{***}&      -6.275\sym{***}&      -6.887\sym{***}&      -5.551\sym{***}\\
                    &     (1.311)         &     (1.545)         &     (1.666)         &     (2.063)         &     (1.440)         \\
\midrule Dependent mean&       120.0         &       120.2         &       121.2         &       122.7         &       120.7         \\
\(N\) (MOB $\times$ year)&         456         &         380         &         304         &         228         &         380         \\
 				
				\bottomrule 
		\end{tabular}}
		\begin{tablenotes} 
			\item \scriptsize \emph{Notes:} The table reports DiD estimates when using interactions of $Treat \times After$ with the age brackets and relying on the full sample. This robustness check differs from the specification in Table \ref{tab_mlch: DD_hopsital2_total} in which there are different regressions for each age-bracket. 			
		\end{tablenotes}
	\end{threeparttable} 
\end{table}
\vspace*{\fill}\clearpage 
\restoregeometry













% % ITT effects nach chaptern (WOMEN)

% \newpage
% \newgeometry{left=3cm,right=3cm,top=1cm,bottom=2.5cm} 
% \vspace*{\fill}
% \begin{table}[H] \centering 
% 	\begin{threeparttable} \centering \caption{ITT effects on \textbf{hospital admission and main diagnoses chapters (women)}}\label{tab_mlch: ITT_across_chapters_per_age_group_women}
% 		{\def\sym#1{\ifmmode^{#1}\else\(^{#1}\)\fi} 
% 			\begin{tabular}{l*{5}{c}}
% 				\toprule 
% 				&\multicolumn{1}{c}{(1)}&\multicolumn{1}{c}{(2)}&\multicolumn{1}{c}{(3)}&\multicolumn{1}{c}{(4)}&\multicolumn{1}{c}{(5)}\\
% 				\midrule
% 				&\multirow{2}{*}{Overall} & \multicolumn{4}{c}{Age brackets [years]} \\ 
% 				\cmidrule(lr){3-6}
% 				&&\multicolumn{1}{c}{17-21}&\multicolumn{1}{c}{22-26}&\multicolumn{1}{c}{27-31}&\multicolumn{1}{c}{32-35}\\
				
% 				\midrule
				
% 				Hospital            &      -1.815\sym{**} &      -2.916\sym{***}&      0.0274         &      -2.762\sym{**} &      -1.212         \\
                    &     (0.807)         &     (0.935)         &     (1.267)         &     (1.004)         &     (0.866)         \\
IPD                 &     -0.0791         &       0.107         &      -0.193         &      -0.132         &     -0.0378         \\
                    &    (0.0519)         &     (0.111)         &     (0.158)         &    (0.0889)         &     (0.177)         \\
Neo                 &     -0.0484         &      -0.789\sym{***}&       0.387\sym{**} &       0.162         &       0.188         \\
                    &    (0.0920)         &     (0.229)         &     (0.162)         &     (0.126)         &     (0.234)         \\
MBD                 &      0.0599         &       0.388         &       0.205         &      -0.426         &      -0.163         \\
                    &     (0.266)         &     (0.313)         &     (0.466)         &     (0.418)         &     (0.388)         \\
Ner                 &    -0.00703         &      -0.274\sym{***}&       0.221\sym{*}  &       0.106         &       0.115         \\
                    &    (0.0753)         &    (0.0887)         &     (0.124)         &     (0.179)         &     (0.139)         \\
Sen                 &      -0.225\sym{***}&      -0.343\sym{***}&     -0.0591         &      -0.217\sym{**} &      -0.323\sym{**} \\
                    &    (0.0460)         &    (0.0705)         &    (0.0802)         &     (0.103)         &     (0.148)         \\
Cir                 &     -0.0410         &      -0.133         &       0.120         &     -0.0920         &      -0.172\sym{*}  \\
                    &    (0.0709)         &    (0.0843)         &     (0.147)         &    (0.0866)         &    (0.0886)         \\
Res                 &      -0.254\sym{**} &      -0.354         &     -0.0438         &      -0.149         &      -0.118         \\
                    &     (0.118)         &     (0.283)         &     (0.165)         &     (0.202)         &     (0.216)         \\
Dig                 &      -0.718\sym{***}&      -0.796\sym{***}&      -0.592         &      -1.069\sym{***}&      -0.734         \\
                    &     (0.182)         &     (0.258)         &     (0.408)         &     (0.282)         &     (0.438)         \\
SST                 &      0.0370         &     -0.0342         &       0.158         &      0.0960         &     -0.0972         \\
                    &    (0.0476)         &     (0.118)         &     (0.133)         &     (0.106)         &     (0.124)         \\
Mus                 &       0.124\sym{*}  &       0.119         &      -0.267         &       0.207         &       0.498\sym{**} \\
                    &    (0.0705)         &     (0.145)         &     (0.216)         &     (0.198)         &     (0.239)         \\
Gen                 &      -0.133         &      0.0523         &       0.147         &      -0.570         &       0.313         \\
                    &     (0.172)         &     (0.262)         &     (0.297)         &     (0.369)         &     (0.273)         \\
Sym                 &      -0.136         &      -0.214         &       0.139         &      -0.396\sym{**} &      -0.105         \\
                    &     (0.112)         &     (0.171)         &     (0.158)         &     (0.142)         &     (0.235)         \\
Ext                 &      -0.421\sym{***}&      -0.663\sym{***}&      -0.322         &      -0.219         &      -0.498\sym{**} \\
                    &     (0.102)         &     (0.124)         &     (0.220)         &     (0.190)         &     (0.185)         \\

				
% 				\bottomrule 
% 		\end{tabular}}
% 		% \begin{tablenotes} 
% 		% 	\item 
% 		% \end{tablenotes} 
% 	\end{threeparttable} 
% 	\begin{minipage}{0.9\linewidth}
% 		\scriptsize \emph{Notes:} This table reports intention-to-treat estimates across the main diagnosis chapters for the entire life-course or per age bracket. The outcomes are defined as the number of cases per 1,000 individuals (births). The point estimates are coming from a DiD regression as described in section \ref{sec_mlch:empirical_strategy}, with a bandwidth of six months, month-of-birth and year fixed effects, and clustered standard errors on the month-of-birth level. The control group is comprised of children that are born in the same months but one year before (i.e. children born between November 1977 and October 1978).\newline
% 		\emph{Legend:} Infectious and parasitic diseases (IPD), neoplasms (Neo), mental and behavioral disorders (MBD), diseases of the nervous system (Ner), diseases of the sense organs (Sen), diseases of the circulatory system (Cir), diseases of the respiratory system (Res), diseases of the digestive system (Dig), diseases of the skin and subcutaneous tissue (SST), diseases of the musculoskeletal system (Mus), diseases of the genitourinary system (Gen), symptoms, signs, and ill-defined conditions (Sym), injury, poisoning and certain other consequences of external causes (Ext).
% 	\end{minipage}
% \end{table} 
% \vspace*{\fill}\clearpage 
% \restoregeometry



% %--------------------------------------------
% % ITT effects nach chaptern (MEN)
% % original 3 3 2 3
% \newpage
% \newgeometry{left=3cm,right=3cm,top=1cm,bottom=2.5cm} 
% \vspace*{\fill}
% \begin{table}[H] \centering 
% 	\begin{threeparttable} \centering \caption{ITT effects on \textbf{hospital admission and main diagnoses chapters (men)}}\label{tab_mlch: ITT_across_chapters_per_age_group_men}
% 		{\def\sym#1{\ifmmode^{#1}\else\(^{#1}\)\fi} 
% 			\begin{tabular}{l*{5}{c}}
% 				\toprule 
% 				&\multicolumn{1}{c}{(1)}&\multicolumn{1}{c}{(2)}&\multicolumn{1}{c}{(3)}&\multicolumn{1}{c}{(4)}&\multicolumn{1}{c}{(5)}\\
% 				\midrule
% 				&\multirow{2}{*}{Overall} & \multicolumn{4}{c}{Age brackets [years]} \\ 
% 				\cmidrule(lr){3-6}
% 				&&\multicolumn{1}{c}{17-21}&\multicolumn{1}{c}{22-26}&\multicolumn{1}{c}{27-31}&\multicolumn{1}{c}{32-35}\\
				
% 				\midrule
				
% 				Hospital            &      -2.525\sym{**} &      -0.273         &      -1.230         &      -2.558\sym{*}  &      -6.373\sym{***}\\
                    &     (0.997)         &     (1.201)         &     (1.048)         &     (1.294)         &     (1.526)         \\
IPD                 &     -0.0912\sym{**} &     -0.0824         &      -0.135         &      -0.120         &    -0.00380         \\
                    &    (0.0418)         &    (0.0887)         &     (0.150)         &     (0.142)         &     (0.104)         \\
Neo                 &      0.0857         &       0.352\sym{*}  &       0.282         &      0.0677         &      -0.627\sym{***}\\
                    &     (0.120)         &     (0.203)         &     (0.176)         &     (0.172)         &     (0.162)         \\
MBD                 &      -1.267\sym{***}&     -0.0319         &      -0.180         &      -1.504\sym{***}&      -3.518\sym{***}\\
                    &     (0.292)         &     (0.262)         &     (0.485)         &     (0.507)         &     (0.515)         \\
Ner                 &      0.0186         &      0.0787         &       0.250\sym{*}  &     -0.0319         &      -0.150         \\
                    &    (0.0960)         &     (0.122)         &     (0.125)         &     (0.136)         &     (0.209)         \\
Sen                 &     -0.0688\sym{*}  &    -0.00121         &      -0.137         &      -0.133\sym{*}  &      -0.203         \\
                    &    (0.0400)         &    (0.0995)         &    (0.0944)         &    (0.0726)         &     (0.126)         \\
Cir                 &     -0.0481         &      0.0370         &     -0.0520         &      -0.301\sym{*}  &      -0.218         \\
                    &    (0.0929)         &     (0.122)         &    (0.0922)         &     (0.162)         &     (0.261)         \\
Res                 &      -0.326\sym{***}&      -0.406\sym{**} &      -0.351\sym{**} &      -0.391\sym{***}&     -0.0488         \\
                    &     (0.107)         &     (0.186)         &     (0.152)         &     (0.128)         &     (0.181)         \\
Dig                 &      -0.111         &      -0.110         &      -0.406\sym{*}  &       0.273         &      -0.266         \\
                    &     (0.159)         &     (0.172)         &     (0.208)         &     (0.396)         &     (0.369)         \\
SST                 &       0.152\sym{*}  &       0.144         &       0.219\sym{**} &       0.213         &      0.0690         \\
                    &    (0.0745)         &     (0.102)         &    (0.0843)         &     (0.134)         &     (0.141)         \\
Mus                 &      -0.184\sym{**} &      -0.167         &     -0.0445         &      -0.298         &      -0.168         \\
                    &    (0.0715)         &     (0.151)         &     (0.189)         &     (0.240)         &     (0.150)         \\
Gen                 &      0.0928         &       0.373\sym{*}  &       0.155         &       0.121         &      -0.406\sym{*}  \\
                    &     (0.117)         &     (0.200)         &     (0.213)         &     (0.192)         &     (0.211)         \\
Sym                 &     -0.0642         &      -0.158         &      0.0689         &      0.0486         &      -0.143         \\
                    &    (0.0639)         &     (0.184)         &    (0.0951)         &     (0.134)         &     (0.128)         \\
Ext                 &      -0.705\sym{**} &      -0.189         &      -0.965\sym{**} &      -0.580\sym{*}  &      -0.680\sym{**} \\
                    &     (0.288)         &     (0.570)         &     (0.411)         &     (0.294)         &     (0.265)         \\

				
% 				\bottomrule 
% 		\end{tabular}}
% 		% \begin{tablenotes} 
% 		% 	\item 
% 		% \end{tablenotes} 
% 	\end{threeparttable} 
% 	\begin{minipage}{0.9\linewidth}
% 		\scriptsize \emph{Notes:} This table reports intention-to-treat estimates across the main diagnosis chapters for the entire life-course or per age bracket. The outcomes are defined as the number of cases per 1,000 individuals (births). The point estimates are coming from a DiD regression as described in section \ref{sec_mlch:empirical_strategy}, with a bandwidth of six months, month-of-birth and year fixed effects, and clustered standard errors on the month-of-birth level. The control group is comprised of children that are born in the same months but one year before (i.e. children born between November 1977 and October 1978).\newline
% 		\emph{Legend:} Infectious and parasitic diseases (IPD), neoplasms (Neo), mental and behavioral disorders (MBD), diseases of the nervous system (Ner), diseases of the sense organs (Sen), diseases of the circulatory system (Cir), diseases of the respiratory system (Res), diseases of the digestive system (Dig), diseases of the skin and subcutaneous tissue (SST), diseases of the musculoskeletal system (Mus), diseases of the genitourinary system (Gen), symptoms, signs, and ill-defined conditions (Sym), injury, poisoning and certain other consequences of external causes (Ext).
% 	\end{minipage}
% \end{table} 
% \vspace*{\fill}\clearpage 
% \restoregeometry



%--------------------------------------------
% D5 SUBCATEGORIES
\newpage
\newgeometry{left=3cm,right=3cm,top=1cm,bottom=2.5cm} 
\vspace*{\fill}
\begin{table}[H] \centering 
	\begin{threeparttable} \centering \caption{ITT effects on the \textbf{subcategories of mental and behavioral disorders (total)}}\label{tab_mlch: ITT_across_d5subcategories_per_age_group_total}
		{\def\sym#1{\ifmmode^{#1}\else\(^{#1}\)\fi} 
			\begin{tabular}{l*{5}{c}}
				\toprule 
				&\multicolumn{1}{c}{(1)}&\multicolumn{1}{c}{(2)}&\multicolumn{1}{c}{(3)}&\multicolumn{1}{c}{(4)}&\multicolumn{1}{c}{(5)}\\
				\midrule
				&\multirow{2}{*}{Overall} & \multicolumn{4}{c}{Age brackets [years]} \\ 
				\cmidrule(lr){3-6}
				&&\multicolumn{1}{c}{17-21}&\multicolumn{1}{c}{22-26}&\multicolumn{1}{c}{27-31}&\multicolumn{1}{c}{32-35}\\
				
				\midrule
				
				MBD & -0.621\sym{**} & 0.174 & -0.008 & -1.000\sym{***} & -1.906\sym{***} \\
& (0.242) & (0.257) & (0.410) & (0.349) & (0.362) \\
Psychoactive substances & -0.483\sym{***} & -0.074 & -0.071 & -0.549\sym{***} & -1.428\sym{***} \\
& (0.110) & (0.123) & (0.136) & (0.156) & (0.270) \\
Schizophrenia & -0.272\sym{**} & 0.061 & 0.069 & -0.707\sym{***} & -0.572\sym{***} \\
& (0.119) & (0.087) & (0.230) & (0.155) & (0.170) \\
Affective & 0.093\sym{**} & -0.035 & 0.004 & 0.198\sym{***} & 0.235\sym{*} \\
& (0.035) & (0.042) & (0.054) & (0.068) & (0.128) \\
Neurosis & 0.066 & 0.001 & 0.108 & 0.213\sym{***} & -0.088 \\
& (0.040) & (0.086) & (0.102) & (0.066) & (0.054) \\
Personality & 0.013 & 0.172\sym{***} & 0.005 & -0.158\sym{*} & 0.039 \\
& (0.036) & (0.055) & (0.072) & (0.086) & (0.091) \\
				
				\bottomrule 
		\end{tabular}}
		% \begin{tablenotes} 
		% 	\item 
		% \end{tablenotes} 
	\end{threeparttable} 
	\begin{minipage}{0.9\linewidth}
		\scriptsize \emph{Notes:} The table shows DiD estimates of the 1979 ML reform on subcategories of MBDs. The first column shows the effect for the entire pooled time frame, whereas columns 2 to 5 display the impact per age group. The outcome variables are defined as the number of cases per thousand individuals (births). The point estimates are coming from a DiD regression as described in section \ref{sec_mlch:empirical_strategy}, with a bandwidth of six months, and month-of-birth and year fixed effects. The control group is comprised of children that are born in the same months but one year before the reform (i.e. children born between November 1977 and October 1978). Clustered standard errors are reported in parentheses. \newline Significance levels: * p < 0.10, ** p < 0.05, *** p < 0.01. \newline 	%\emph{Source:} Hospital registry data.
	\end{minipage}
\end{table} 
\vspace*{\fill}\clearpage 
\restoregeometry
%--------------------------------------------
% % D5 SUBCATEGORIES (WOMEN)
% \newpage
% \newgeometry{left=3cm,right=3cm,top=1cm,bottom=2.5cm} 
% \vspace*{\fill}
% \begin{table}[H] \centering 
% 	\begin{threeparttable} \centering \caption{ITT effects on the \textbf{subcategories of mental and behavioral disorders (women)}}\label{tab_mlch: ITT_across_d5subcategories_per_age_group_women}
% 		{\def\sym#1{\ifmmode^{#1}\else\(^{#1}\)\fi} 
% 			\begin{tabular}{l*{5}{c}}
% 				\toprule 
% 				&\multicolumn{1}{c}{(1)}&\multicolumn{1}{c}{(2)}&\multicolumn{1}{c}{(3)}&\multicolumn{1}{c}{(4)}&\multicolumn{1}{c}{(5)}\\
% 				\midrule
% 				&\multirow{2}{*}{Overall} & \multicolumn{4}{c}{Age brackets [years]} \\ 
% 				\cmidrule(lr){3-6}
% 				&&\multicolumn{1}{c}{17-21}&\multicolumn{1}{c}{22-26}&\multicolumn{1}{c}{27-31}&\multicolumn{1}{c}{32-35}\\
				
% 				\midrule
				
% 				MBD & 0.060 & 0.388 & 0.205 & -0.426 & -0.163 \\
& (0.266) & (0.305) & (0.455) & (0.408) & (0.378) \\
Psychoactive substances & -0.091 & -0.155 & 0.152 & 0.177 & -0.641\sym{***} \\
& (0.124) & (0.190) & (0.157) & (0.163) & (0.228) \\
Schizophrenia & -0.033 & 0.063 & 0.030 & -0.594\sym{***} & 0.280 \\
& (0.101) & (0.063) & (0.230) & (0.117) & (0.204) \\
Affective & 0.155\sym{**} & 0.031 & 0.030 & 0.189 & 0.276 \\
& (0.064) & (0.040) & (0.101) & (0.124) & (0.173) \\
Neurosis & 0.046 & 0.090 & 0.099 & 0.096 & -0.122 \\
& (0.080) & (0.180) & (0.130) & (0.132) & (0.128) \\
Personality & 0.032 & 0.272\sym{***} & 0.071 & -0.286 & 0.102 \\
& (0.057) & (0.089) & (0.104) & (0.170) & (0.166) \\
				
% 				\bottomrule 
% 		\end{tabular}}
% 		% \begin{tablenotes} 
% 		% 	\item 
% 		% \end{tablenotes} 
% 	\end{threeparttable} 
% 	\begin{minipage}{0.9\linewidth}
% 		\scriptsize \emph{Notes:} The table shows DiD estimates of the 1979 ML reform on subcategories of MBDs. The first column shows the effect for the entire pooled time frame, whereas columns 2 to 5 display the impact per age group. The outcome variables are defined as the number of cases per thousand individuals (births). The point estimates are coming from a DiD regression as described in section \ref{sec_mlch:empirical_strategy}, with a bandwidth of six months, and month-of-birth and year fixed effects. The control group is comprised of children that are born in the same months but one year before the reform (i.e. children born between November 1977 and October 1978). Clustered standard errors are reported in parentheses. \newline Significance levels: * p < 0.10, ** p < 0.05, *** p < 0.01. \newline 	\emph{Source:} Hospital registry data.
% 	\end{minipage}
% \end{table} 
% \vspace*{\fill}\clearpage 
% \restoregeometry

% %--------------------------------------------
% % D5 SUBCATEGORIES
% \newpage
% \newgeometry{left=3cm,right=3cm,top=1cm,bottom=2.5cm} 
% \vspace*{\fill}
% \begin{table}[H] \centering 
% 	\begin{threeparttable} \centering \caption{ITT effects on the \textbf{subcategories of mental and behavioral disorders (men)}}\label{tab_mlch: ITT_across_d5subcategories_per_age_group_men}
% 		{\def\sym#1{\ifmmode^{#1}\else\(^{#1}\)\fi} 
% 			\begin{tabular}{l*{5}{c}}
% 				\toprule 
% 				&\multicolumn{1}{c}{(1)}&\multicolumn{1}{c}{(2)}&\multicolumn{1}{c}{(3)}&\multicolumn{1}{c}{(4)}&\multicolumn{1}{c}{(5)}\\
% 				\midrule
% 				&\multirow{2}{*}{Overall} & \multicolumn{4}{c}{Age brackets [years]} \\ 
% 				\cmidrule(lr){3-6}
% 				&&\multicolumn{1}{c}{17-21}&\multicolumn{1}{c}{22-26}&\multicolumn{1}{c}{27-31}&\multicolumn{1}{c}{32-35}\\
				
% 				\midrule
				
% 				MBD & -1.267\sym{***} & -0.032 & -0.180 & -1.504\sym{***} & -3.518\sym{***} \\
& (0.292) & (0.255) & (0.473) & (0.495) & (0.502) \\
Psychoactive substances & -0.886\sym{***} & 0.012 & -0.262 & -1.205\sym{***} & -2.133\sym{***} \\
& (0.182) & (0.205) & (0.219) & (0.248) & (0.380) \\
Schizophrenia & -0.462\sym{**} & 0.067 & 0.135 & -0.787\sym{**} & -1.355\sym{***} \\
& (0.203) & (0.143) & (0.330) & (0.311) & (0.270) \\
Affective & 0.078\sym{***} & -0.099\sym{*} & -0.026 & 0.198\sym{***} & 0.184 \\
& (0.020) & (0.057) & (0.061) & (0.066) & (0.131) \\
Neurosis & 0.049 & -0.094 & 0.113 & 0.321\sym{***} & -0.060 \\
& (0.054) & (0.098) & (0.146) & (0.059) & (0.072) \\
Personality & -0.012 & 0.075\sym{*} & -0.064 & -0.042 & -0.025 \\
& (0.034) & (0.043) & (0.086) & (0.037) & (0.084) \\
				
% 				\bottomrule 
% 		\end{tabular}}
% 		% \begin{tablenotes} 
% 		% 	\item 
% 		% \end{tablenotes} 
% 	\end{threeparttable} 
% 	\begin{minipage}{0.9\linewidth}
% 		\scriptsize \emph{Notes:} The table shows DiD estimates of the 1979 ML reform on subcategories of MBDs. The first column shows the effect for the entire pooled time frame, whereas columns 2 to 5 display the impact per age group. The outcome variables are defined as the number of cases per thousand individuals (births). The point estimates are coming from a DiD regression as described in section \ref{sec_mlch:empirical_strategy}, with a bandwidth of six months, and month-of-birth and year fixed effects. The control group is comprised of children that are born in the same months but one year before the reform (i.e. children born between November 1977 and October 1978). Clustered standard errors are reported in parentheses. \newline Significance levels: * p < 0.10, ** p < 0.05, *** p < 0.01. \newline 	\emph{Source:} Hospital registry data.
% 	\end{minipage}
% \end{table} 
% \vspace*{\fill}\clearpage 
% \restoregeometry


%--------------------------------------------
% d5 - robustness in one table - blueprint for paper
\newpage
\newgeometry{left=1cm,right=1cm,top=1cm,bottom=2.5cm} 
% \begin{landscape}
	\vspace*{\fill}
	\begin{table}[htbp] \centering 
		\begin{threeparttable} \centering 
			\caption{Robustness checks for \textbf{mental and behavioral disorders}} \label{tab: robustness_d5} 
			{\def\sym#1{\ifmmode^{#1}\else\(^{#1}\)\fi} 
				\begin{tabular}{l*{10}{c}} \toprule 
					
					& & \multicolumn{2}{c}{Alternative specifications} & \multicolumn{3}{c}{\clb{c}{Alternative\\estimation}} & \multicolumn{2}{c}{Placebos}& \multicolumn{2}{c}{Heterogeneity}\\
					\cmidrule(lr){3-4} \cmidrule(lr){5-7} \cmidrule(lr){8-9} \cmidrule(lr){10-11}
					&\multicolumn{1}{c}{(1)}&\multicolumn{1}{c}{(2)}&\multicolumn{1}{c}{(3)}&\multicolumn{1}{c}{(4)}&\multicolumn{1}{c}{(5)}&\multicolumn{1}{c}{(6)}&\multicolumn{1}{c}{(7)}&\multicolumn{1}{c}{(8)}&\multicolumn{1}{c}{(9)}&\multicolumn{1}{c}{(10)}\\
					&\multicolumn{1}{c}{Baseline}&\multicolumn{1}{c}{\clb{c}{current\\population}}&\multicolumn{1}{c}{\clb{c}{LMR\\level$^a$}}&\multicolumn{1}{c}{\clb{c}{DDD$^b$}}&\multicolumn{1}{c}{\clb{c}{alt. DD$^b$}}&\multicolumn{1}{c}{add. CG}&\multicolumn{1}{c}{\clb{c}{temporal:\\cohort}}&\multicolumn{1}{c}{\clb{c}{spatial:\\ GDR}}&\multicolumn{1}{c}{\clb{c}{rural$^a$}}&\multicolumn{1}{c}{\clb{c}{urban$^a$}}\\
					\midrule
					\\
					(1) {total} 		&   -0.634\sym{**}	&	-0.832\sym{***}	&   -0.831\sym{**}  &	-0.834\sym{**}  & 	-0.558\sym{**}  & -0.553\sym{**}	&	0.162			&	0.200		&	-0.0987		&	-1.006\sym{***} 	\\
										&	(0.249)			&	(0.239)			&   (0.229)     	&	(0.321)			& 	(0.157)			& (0.269)			&	(0.304)			&	(0.141)		&	(0.588)		&	(0.202)				\\
					(2) {female}		&   0.0599			&	-0.0853			& 	-0.0724     	&	-0.0385			& 	-0.217			& 0.289			    &	0.457			&	0.0984		&	0.299		&	-0.161				\\
										&	(0.266)			&	(0.266)			&   (0.258)     	&	(0.320)			& 	(0.144)			& (0.287)			&	(0.369)			&	(0.196)		&	(0.816)		&	(0.234)				\\
					(3) {male} 			&   -1.267\sym{***}	&	-1.554\sym{***}	&   -1.598\sym{***} &	-1.533\sym{***} & 	-0.834\sym{***} & -1.323\sym{***}	&	-0.112			&	0.266		&	-0.414		&	-1.877\sym{***} 	\\
										&	(0.292)			&	(0.299)			&   (0.286)     	&	(0.388)			& 	(0.190)			& (0.322)			&	 (0.331) 		&	(0.198)		&	(0.487)		&	(0.333)				\\
					\midrule            																																																						
					For total: 																																																				\\							 
					Dependent mean 		&   18.96			&	17.28			&   17.65     		&	13.80			& 	13.80			& 18.96				&	18.67			&	8.640		&	16.76		&	18.38				\\
					Effect in SDs [\%] 	&   11.43			&	41.92			&   5.20      		&	12.53			& 	8.380			& 9.960				&	3.490			&	9.440		&	1.69		&	7.27				\\
					$N$ (MOB $\times$ year)  		&   480				&	288				&   58,751    		&	960				& 	480				& 720				&	480				&	480			&	26,495		&	32,256				\\
					%Federal level		&   \checkmark		&	\checkmark		&   $\times$		& \checkmark		&	\checkmark		& \checkmark		&	\checkmark		&  \checkmark	&	$\times$	&	$\times$			\\ 
					\\
					MOB fixed effects 	&   \checkmark		&	\checkmark		&   \checkmark		& \checkmark		&	\checkmark		& \checkmark		&	\checkmark		&  \checkmark	&	\checkmark	&	\checkmark		\\ 
					Year fixed effects  &   \checkmark		&	\checkmark		&   \checkmark		& \checkmark		&	\checkmark		& \checkmark		&	\checkmark		&  \checkmark	&	\checkmark	&	\checkmark		\\ 

					\bottomrule
			\end{tabular}}
	\end{threeparttable} 
		\begin{minipage}{0.87\linewidth}
		\scriptsize \emph{Notes:} This table displays robustness check for the effect of the 1979 maternity leave reform on mental and behavioral disorders. We perform the following checks (with reference to the column): (1) baseline specification that was used in previous parts of the paper, (2) for the outcome we use the number of diagnoses divided by the current number of individuals (approximation), (3) the analysis is carried out on the level of labor market regions, (4) triple difference model (the third difference stems from the former region of the GDR), (5) alternative difference-in-difference model which compares pre and post of the treatment cohort in West Germany with the respective values in East Germany, (6) we use as control cohort not only the cohort before the reform, but also the cohort 2 years prior to the policy change, (7) first placebo, in which the entire analysis set-up is pushed back by one year, i.e. the placebo TG is the cohort prior to the real TG and the placebo CG is the cohort born 2 years before the reform took place, (8) second placebo, in which we run the normal DD set-up in the area of the former GDR, (9) + (10)  DD carried out in rural and urban regions (compare with figure \ref{fig: AMR_regions_population_density} to see which regions are marked as rural/urban). \newline Significance levels: * p < 0.10, ** p < 0.05, *** p < 0.01. \newline
		\hspace*{15 pt}$^a$: level of analysis on Labor Market Regions: weighted regressions (by population), includes region fixed effects.\newline
		\hspace*{15 pt}$^b$: standard errors clustered on the month-of-birth$\times$birth-cohort$\times$East-West cell level.
	\end{minipage}
\end{table} 
	\vspace*{\fill}\clearpage
\end{landscape}

% Columns 1, 2 8 (lokal) ok 
% COLUMN 8 uses the r_popf specification
% sind DDD, alt DD, spatial placebo mit r_pop


%\newgeometry{left=2.5cm,right=2.5cm,top=3cm,bottom=3cm} 
\begin{landscape}
	\vspace*{\fill}
	\begin{table}[H] \centering 
		\begin{threeparttable} \centering 
			\caption{Robustness checks for mental and behavioral disorders} \label{tab_mlch: robustness_d5} 
			{\def\sym#1{\ifmmode^{#1}\else\(^{#1}\)\fi} 
				\begin{tabular}{l*{8}{c}} \toprule 
					
					& & \multicolumn{2}{c}{Alternative specifications} & \multicolumn{2}{c}{\clb{c}{Alternative\\estimation}} & \multicolumn{2}{c}{Placebos}\\
					\cmidrule(lr){3-4} \cmidrule(lr){5-6} \cmidrule(lr){7-8} 
					&\multicolumn{1}{c}{(1)}&\multicolumn{1}{c}{(2)}&\multicolumn{1}{c}{(3)}&\multicolumn{1}{c}{(4)}&\multicolumn{1}{c}{(5)}&\multicolumn{1}{c}{(6)}&\multicolumn{1}{c}{(7)}\\
					&\multicolumn{1}{c}{Baseline}&\multicolumn{1}{c}{\clb{c}{Current\\population}}&\multicolumn{1}{c}{\clb{c}{LMR\\level$^a$}}&\multicolumn{1}{c}{\clb{c}{DDD$^b$}}&\multicolumn{1}{c}{Add. CG}&\multicolumn{1}{c}{\clb{c}{Temporal:\\cohort}}&\multicolumn{1}{c}{\clb{c}{Spatial:\\ GDR}}\\
					\midrule
					\\
					%							1					2					3					4					5					6					7					8								
					(1) {Total} 		&   -0.621\sym{**}	&	-0.832\sym{***}	&   -0.844\sym{***} &	-0.872\sym{**}  &  -0.553\sym{**}	&	0.252			&	0.252			\\
										&	(0.242)			&	(0.239)			&   (0.219)     	&	(0.321)			&  (0.269)			&	(0.320)			&	(0.155)			\\
					(2) {Female}		&   0.010			&	-0.0853			& 	-0.130      	&	-0.0138			&  0.289			&	0.527			&	0.0235			\\
										&	(0.271)			&	(0.266)			&   (0.261)     	&	(0.326)			&  (0.287)			&	(0.392)			&	(0.215)			\\
					(3) {Male} 			&   -1.192\sym{***}	&	-1.554\sym{***}	&   -1.558\sym{***} &	-1.627\sym{***} &  -1.323\sym{***}	&	-0.005			&	0.434\sym{*}	\\
										&	(0.288)			&	(0.299)			&   (0.283)     	&	(0.392)			&  (0.322)			&	 (0.334) 		&	(0.225)			\\
					\midrule            																				 																													
					For total: 																							 																								\\							 
					Dependent mean 		&   19.57			&	17.28			&   17.88     		&	19.57			&  18.96			&	19.21			&	8.850			\\
					Effect in SDs [\%] 	&   12.44			&	41.92			&   5.230      		&	17.48			&  9.960			&	6.120			&	12.91			\\
					$N$ 				&   456				&	288				&   53,855    		&	912				&  720				&	456				&	456				\\
					%Federal level		&   \checkmark		&	\checkmark		&   $\times$		& \checkmark		&  \checkmark		&	\checkmark		&  \checkmark		\\ 
					\\
					MOB fixed effects 	&   \checkmark		&	\checkmark		&   \checkmark		& \checkmark		&  \checkmark		&	\checkmark		&  \checkmark	    \\ 
					Year fixed effects  &   \checkmark		&	\checkmark		&   \checkmark		& \checkmark		&  \checkmark		&	\checkmark		&  \checkmark	    \\ 
					\bottomrule
			\end{tabular}}
		\begin{tablenotes} 
			\item \scriptsize \emph{Notes:} This table displays robustness checks for the effect of the 1979 maternity leave reform on mental and behavioral disorders. I perform the following checks (with reference to the column): (1) baseline specification that was used in previous parts of the paper, (2) for the outcome I use the number of diagnoses divided by the current number of individuals (approximation), (3) the analysis is carried out on the level of labor market regions, (4) triple difference model (the third difference stems from the former region of the GDR), (5) I use as control cohort not only the cohort before the reform, but also the cohort 2 years prior to the policy change, (6) first placebo, in which the entire analysis set-up is pushed back by one year, i.e. the placebo TG is the cohort prior to the real TG and the placebo CG is the cohort born 2 years before the reform took place, (7) second placebo, in which I run the normal DD set-up in the area of the former GDR. \newline Significance levels: * p < 0.10, ** p < 0.05, *** p < 0.01. \newline
			\hspace*{15 pt}$^a$: level of analysis on Labor Market Regions: weighted regressions (by population), includes region fixed effects.\newline
			\hspace*{15 pt}$^b$: standard errors clustered on the month-of-birth$\times$birth-cohort$\times$East-West cell level.
		\end{tablenotes}
	\end{threeparttable} 
\end{table} 
	\vspace*{\fill}\clearpage
\end{landscape}
%\restoregeometry
% Welche Columns sind geupdated (einbinden von $LC) : 3,4,5,8,9,10




\restoregeometry
%--------------------------------------------


%t-tests
%%hospital total
\begin{table}[H] \centering 
	\begin{threeparttable} \centering \caption{t-test for \textbf{hospital admission (total)}}\label{tab:t-test_d5total}
		\begin{footnotesize}
			{\def\sym#1{\ifmmode^{#1}\else\(^{#1}\)\fi} 
				\begin{tabular}{l*{3}{c}}
					\toprule 
					& \multicolumn{1}{c}{Before} & \multicolumn{1}{c}{After} & \multicolumn{1}{c}{Difference} \\
					&\multicolumn{1}{c}{(1)}&\multicolumn{1}{c}{(2)}&\multicolumn{1}{c}{(1)-(2)}\\
					\midrule
					Overall (pooled)    &       120.6&       118.4&        2.22   \\
                    &      [11.0]&      [9.85]&      (1.35)   \\
\hspace{12pt}1995   &       111.1&       106.7&        4.48** \\
                    &      [3.65]&      [2.73]&      (1.86)   \\
\hspace{12pt}1996   &       117.6&       112.5&        5.08** \\
                    &      [5.13]&      [1.92]&      (2.24)   \\
\hspace{12pt}1997   &       125.9&       120.1&        5.76** \\
                    &      [5.12]&      [2.07]&      (2.25)   \\
\hspace{12pt}1998   &       126.1&       125.1&        1.00   \\
                    &      [3.99]&      [2.48]&      (1.92)   \\
\hspace{12pt}1999   &       124.7&       124.7&     -0.0025   \\
                    &      [4.76]&      [2.48]&      (2.19)   \\
\hspace{12pt}2000   &       120.1&       119.1&        0.92   \\
                    &      [4.39]&      [3.69]&      (2.34)   \\
\hspace{12pt}2001   &       119.4&       121.9&       -2.48   \\
                    &      [4.33]&      [3.47]&      (2.27)   \\
\hspace{12pt}2002   &       120.6&       118.6&        1.99   \\
                    &      [5.13]&      [1.76]&      (2.21)   \\
\hspace{12pt}2003   &       113.6&       112.5&        1.12   \\
                    &      [4.75]&      [0.77]&      (1.96)   \\
\hspace{12pt}2004   &       109.2&       109.0&        0.18   \\
                    &      [3.49]&      [2.61]&      (1.78)   \\
\hspace{12pt}2005   &       105.6&       105.5&       0.045   \\
                    &      [4.36]&      [3.64]&      (2.32)   \\
\hspace{12pt}2006   &       108.9&       106.0&        2.84   \\
                    &      [3.78]&      [1.64]&      (1.68)   \\
\hspace{12pt}2007   &       109.8&       108.2&        1.67   \\
                    &      [4.46]&      [2.47]&      (2.08)   \\
\hspace{12pt}2008   &       113.9&       112.0&        1.91   \\
                    &      [4.47]&      [1.37]&      (1.91)   \\
\hspace{12pt}2009   &       119.2&       117.9&        1.34   \\
                    &      [5.47]&      [2.10]&      (2.39)   \\
\hspace{12pt}2010   &       122.5&       118.8&        3.67   \\
                    &      [5.48]&      [2.92]&      (2.53)   \\
\hspace{12pt}2011   &       128.6&       123.7&        4.84*  \\
                    &      [5.69]&      [1.51]&      (2.40)   \\
\hspace{12pt}2012   &       133.8&       129.1&        4.67** \\
                    &      [4.72]&      [1.97]&      (2.09)   \\
\hspace{12pt}2013   &       136.4&       134.6&        1.80   \\
                    &      [4.85]&      [1.59]&      (2.09)   \\
\hspace{12pt}2014   &       145.5&       142.0&        3.51   \\
                    &      [7.17]&      [2.97]&      (3.17)   \\

					\bottomrule
			\end{tabular}}
		\end{footnotesize}
	\end{threeparttable} 
	\begin{minipage}{0.9\linewidth}
		\scriptsize \emph{Notes:} This table shows descriptive statistics for two samples: (i) cohorts born before May 1979; (ii) cohorts born after May 1979. Columns 1 and 2 show means with standard deviations in brackets. Column 3 report the difference in means between columns 1 and 2 with standard errors in parenthesis. For each sample we take a bandwidth of half a year around the cutoff date, i.e. individuals born between Nov78-Ap79 (May79-Oct79) in the 'before' ('after') sample.
	\end{minipage}
\end{table} 
%hospital women
\begin{table}[H] \centering 
	\begin{threeparttable} \centering \caption{t-test for \textbf{hospital admission (women)}}\label{tab:t-test_d5female}
		\begin{footnotesize}
			{\def\sym#1{\ifmmode^{#1}\else\(^{#1}\)\fi} 
				\begin{tabular}{l*{3}{c}}
					\toprule 
					& \multicolumn{1}{c}{Before} & \multicolumn{1}{c}{After} & \multicolumn{1}{c}{Difference} \\
					&\multicolumn{1}{c}{(1)}&\multicolumn{1}{c}{(2)}&\multicolumn{1}{c}{(1)-(2)}\\
					\midrule
					Overall (pooled)    &       122.4&       120.6&        1.85   \\
                    &      [11.1]&      [10.8]&      (1.42)   \\
\hspace{12pt}1995   &       124.5&       119.4&        5.14** \\
                    &      [4.21]&      [2.06]&      (1.91)   \\
\hspace{12pt}1996   &       129.7&       125.1&        4.61   \\
                    &      [6.62]&      [1.80]&      (2.80)   \\
\hspace{12pt}1997   &       135.0&       131.1&        3.90   \\
                    &      [5.25]&      [3.80]&      (2.65)   \\
\hspace{12pt}1998   &       131.9&       132.8&       -0.89   \\
                    &      [5.50]&      [4.08]&      (2.80)   \\
\hspace{12pt}1999   &       128.2&       129.0&       -0.81   \\
                    &      [4.89]&      [3.66]&      (2.49)   \\
\hspace{12pt}2000   &       123.9&       122.5&        1.37   \\
                    &      [4.76]&      [4.68]&      (2.72)   \\
\hspace{12pt}2001   &       122.6&       126.1&       -3.52   \\
                    &      [4.55]&      [4.36]&      (2.57)   \\
\hspace{12pt}2002   &       124.1&       121.1&        3.04   \\
                    &      [6.53]&      [2.35]&      (2.83)   \\
\hspace{12pt}2003   &       115.8&       115.1&        0.77   \\
                    &      [4.75]&      [1.55]&      (2.04)   \\
\hspace{12pt}2004   &       109.5&       109.8&       -0.37   \\
                    &      [4.92]&      [3.32]&      (2.42)   \\
\hspace{12pt}2005   &       103.5&       104.1&       -0.60   \\
                    &      [4.02]&      [3.50]&      (2.18)   \\
\hspace{12pt}2006   &       107.6&       104.6&        3.01   \\
                    &      [3.91]&      [3.43]&      (2.12)   \\
\hspace{12pt}2007   &       108.3&       105.8&        2.55   \\
                    &      [4.61]&      [2.82]&      (2.21)   \\
\hspace{12pt}2008   &       112.3&       109.4&        2.90   \\
                    &      [4.30]&      [2.60]&      (2.05)   \\
\hspace{12pt}2009   &       118.2&       114.7&        3.54   \\
                    &      [5.05]&      [2.28]&      (2.26)   \\
\hspace{12pt}2010   &       120.7&       116.6&        4.12   \\
                    &      [5.52]&      [3.49]&      (2.67)   \\
\hspace{12pt}2011   &       126.4&       121.8&        4.53   \\
                    &      [5.58]&      [3.14]&      (2.61)   \\
\hspace{12pt}2012   &       131.5&       126.4&        5.15** \\
                    &      [4.92]&      [2.12]&      (2.19)   \\
\hspace{12pt}2013   &       133.1&       133.8&       -0.69   \\
                    &      [4.11]&      [3.97]&      (2.33)   \\
\hspace{12pt}2014   &       141.5&       142.3&       -0.76   \\
                    &      [6.44]&      [2.95]&      (2.89)   \\

					\bottomrule
			\end{tabular}}
		\end{footnotesize}
	\end{threeparttable} 
	\begin{minipage}{0.9\linewidth}
		\scriptsize \emph{Notes:} This table shows descriptive statistics for two samples: (i) cohorts born before May 1979; (ii) cohorts born after May 1979. Columns 1 and 2 show means with standard deviations in brackets. Column 3 report the difference in means between columns 1 and 2 with standard errors in parenthesis. For each sample we take a bandwidth of half a year around the cutoff date, i.e. individuals born between Nov78-Ap79 (May79-Oct79) in the 'before' ('after') sample.
	\end{minipage}
\end{table} 
%hospital men
\begin{table}[H] \centering 
	\begin{threeparttable} \centering \caption{t-test for \textbf{hospital admission (men)}}\label{tab:t-test_d5male}
		\begin{footnotesize}
			{\def\sym#1{\ifmmode^{#1}\else\(^{#1}\)\fi} 
				\begin{tabular}{l*{3}{c}}
					\toprule 
					& \multicolumn{1}{c}{Before} & \multicolumn{1}{c}{After} & \multicolumn{1}{c}{Difference} \\
					&\multicolumn{1}{c}{(1)}&\multicolumn{1}{c}{(2)}&\multicolumn{1}{c}{(1)-(2)}\\
					\midrule
					Overall (pooled)    &       118.9&       116.4&        2.59   \\
                    &      [13.1]&      [11.5]&      (1.59)   \\
\hspace{12pt}1995   &        98.5&        94.5&        3.96*  \\
                    &      [3.22]&      [3.71]&      (2.01)   \\
\hspace{12pt}1996   &       106.1&       100.5&        5.61** \\
                    &      [4.06]&      [2.32]&      (1.91)   \\
\hspace{12pt}1997   &       117.3&       109.7&        7.62** \\
                    &      [5.86]&      [1.59]&      (2.48)   \\
\hspace{12pt}1998   &       120.6&       117.8&        2.84   \\
                    &      [5.58]&      [2.09]&      (2.43)   \\
\hspace{12pt}1999   &       121.4&       120.6&        0.80   \\
                    &      [4.82]&      [3.25]&      (2.37)   \\
\hspace{12pt}2000   &       116.4&       115.9&        0.52   \\
                    &      [4.88]&      [3.55]&      (2.46)   \\
\hspace{12pt}2001   &       116.4&       117.9&       -1.46   \\
                    &      [4.64]&      [4.33]&      (2.59)   \\
\hspace{12pt}2002   &       117.3&       116.3&        1.02   \\
                    &      [5.87]&      [1.40]&      (2.46)   \\
\hspace{12pt}2003   &       111.6&       110.1&        1.47   \\
                    &      [5.50]&      [1.82]&      (2.36)   \\
\hspace{12pt}2004   &       108.9&       108.2&        0.70   \\
                    &      [3.18]&      [3.56]&      (1.95)   \\
\hspace{12pt}2005   &       107.5&       106.9&        0.64   \\
                    &      [5.92]&      [5.48]&      (3.29)   \\
\hspace{12pt}2006   &       110.1&       107.4&        2.68   \\
                    &      [4.95]&      [2.67]&      (2.30)   \\
\hspace{12pt}2007   &       111.3&       110.5&        0.83   \\
                    &      [7.50]&      [3.44]&      (3.37)   \\
\hspace{12pt}2008   &       115.4&       114.4&        0.96   \\
                    &      [6.04]&      [2.28]&      (2.63)   \\
\hspace{12pt}2009   &       120.2&       121.0&       -0.76   \\
                    &      [6.70]&      [2.58]&      (2.93)   \\
\hspace{12pt}2010   &       124.2&       121.0&        3.22   \\
                    &      [6.45]&      [2.50]&      (2.83)   \\
\hspace{12pt}2011   &       130.7&       125.6&        5.13*  \\
                    &      [5.85]&      [1.83]&      (2.50)   \\
\hspace{12pt}2012   &       136.0&       131.8&        4.21   \\
                    &      [4.93]&      [2.85]&      (2.32)   \\
\hspace{12pt}2013   &       139.5&       135.4&        4.15   \\
                    &      [6.11]&      [2.77]&      (2.74)   \\
\hspace{12pt}2014   &       149.3&       141.8&        7.57*  \\
                    &      [8.14]&      [3.18]&      (3.57)   \\

					\bottomrule
			\end{tabular}}
		\end{footnotesize}
	\end{threeparttable} 
	\begin{minipage}{0.9\linewidth}
		\scriptsize \emph{Notes:} This table shows descriptive statistics for two samples: (i) cohorts born before May 1979; (ii) cohorts born after May 1979. Columns 1 and 2 show means with standard deviations in brackets. Column 3 report the difference in means between columns 1 and 2 with standard errors in parenthesis. For each sample we take a bandwidth of half a year around the cutoff date, i.e. individuals born between Nov78-Ap79 (May79-Oct79) in the 'before' ('after') sample.
	\end{minipage}
\end{table} 



%d5 total
\begin{table}[H] \centering 
	\begin{threeparttable} \centering \caption{t-test for \textbf{mental \& behavioral disorders (total)}}\label{tab:t-test_d5total}
		\begin{footnotesize}
			{\def\sym#1{\ifmmode^{#1}\else\(^{#1}\)\fi} 
			\begin{tabular}{l*{3}{c}}
				\toprule 
				& \multicolumn{1}{c}{Before} & \multicolumn{1}{c}{After} & \multicolumn{1}{c}{Difference} \\
				&\multicolumn{1}{c}{(1)}&\multicolumn{1}{c}{(2)}&\multicolumn{1}{c}{(1)-(2)}\\
				\midrule
				Overall (pooled)    &        19.0&        18.6&        0.38   \\
                    &      [5.55]&      [5.44]&      (0.71)   \\
\hspace{12pt}1995   &        7.38&        6.90&        0.48   \\
                    &      [0.61]&      [0.24]&      (0.27)   \\
\hspace{12pt}1996   &        8.64&        8.33&        0.31   \\
                    &      [0.79]&      [0.43]&      (0.37)   \\
\hspace{12pt}1997   &        11.1&        10.2&        0.91   \\
                    &      [1.18]&      [0.60]&      (0.54)   \\
\hspace{12pt}1998   &        13.4&        12.7&        0.69   \\
                    &      [1.17]&      [1.05]&      (0.64)   \\
\hspace{12pt}1999   &        15.1&        15.4&       -0.36   \\
                    &      [0.88]&      [0.55]&      (0.42)   \\
\hspace{12pt}2000   &        16.3&        16.2&        0.13   \\
                    &      [0.83]&      [0.62]&      (0.42)   \\
\hspace{12pt}2001   &        17.8&        18.8&       -1.03*  \\
                    &      [1.03]&      [0.74]&      (0.52)   \\
\hspace{12pt}2002   &        18.2&        18.3&      -0.059   \\
                    &      [0.88]&      [0.66]&      (0.45)   \\
\hspace{12pt}2003   &        19.0&        18.3&        0.68   \\
                    &      [1.51]&      [0.68]&      (0.68)   \\
\hspace{12pt}2004   &        19.6&        19.3&        0.28   \\
                    &      [1.18]&      [0.96]&      (0.62)   \\
\hspace{12pt}2005   &        20.1&        19.8&        0.36   \\
                    &      [1.08]&      [0.99]&      (0.60)   \\
\hspace{12pt}2006   &        20.8&        20.1&        0.73*  \\
                    &      [0.79]&      [0.51]&      (0.38)   \\
\hspace{12pt}2007   &        21.5&        21.1&        0.44   \\
                    &      [1.05]&      [0.44]&      (0.47)   \\
\hspace{12pt}2008   &        22.0&        21.5&        0.56   \\
                    &      [1.41]&      [0.61]&      (0.63)   \\
\hspace{12pt}2009   &        22.7&        22.7&      -0.058   \\
                    &      [1.42]&      [1.37]&      (0.80)   \\
\hspace{12pt}2010   &        23.7&        22.5&        1.22** \\
                    &      [0.96]&      [0.78]&      (0.51)   \\
\hspace{12pt}2011   &        23.9&        23.4&        0.50   \\
                    &      [1.31]&      [0.75]&      (0.62)   \\
\hspace{12pt}2012   &        24.7&        24.3&        0.35   \\
                    &      [0.78]&      [1.24]&      (0.60)   \\
\hspace{12pt}2013   &        25.8&        25.2&        0.64   \\
                    &      [1.20]&      [0.96]&      (0.63)   \\
\hspace{12pt}2014   &        27.3&        26.5&        0.88   \\
                    &      [1.53]&      [1.10]&      (0.77)   \\

				\bottomrule
			\end{tabular}}
		\end{footnotesize}
	\end{threeparttable} 
	\begin{minipage}{0.9\linewidth}
		\scriptsize \emph{Notes:} This table shows descriptive statistics for two samples: (i) cohorts born before May 1979; (ii) cohorts born after May 1979. Columns 1 and 2 show means with standard deviations in brackets. Column 3 report the difference in means between columns 1 and 2 with standard errors in parenthesis. For each sample we take a bandwidth of half a year around the cutoff date, i.e. individuals born between Nov78-Ap79 (May79-Oct79) in the 'before' ('after') sample.
	\end{minipage}
\end{table} 
%d5 women
\begin{table}[H] \centering 
	\begin{threeparttable} \centering \caption{t-test for \textbf{mental \& behavioral disorders (women)}}\label{tab:t-test_d5female}
		\begin{footnotesize}
			{\def\sym#1{\ifmmode^{#1}\else\(^{#1}\)\fi} 
				\begin{tabular}{l*{3}{c}}
					\toprule 
					& \multicolumn{1}{c}{Before} & \multicolumn{1}{c}{After} & \multicolumn{1}{c}{Difference} \\
					&\multicolumn{1}{c}{(1)}&\multicolumn{1}{c}{(2)}&\multicolumn{1}{c}{(1)-(2)}\\
					\midrule
					Overall (pooled)    &        15.7&        15.8&      -0.070   \\
                    &      [3.59]&      [3.56]&      (0.46)   \\
\hspace{12pt}1995   &        8.73&        8.62&        0.10   \\
                    &      [0.78]&      [0.58]&      (0.40)   \\
\hspace{12pt}1996   &        9.30&        9.79&       -0.49   \\
                    &      [1.16]&      [0.82]&      (0.58)   \\
\hspace{12pt}1997   &        11.6&        11.1&        0.48   \\
                    &      [1.07]&      [0.52]&      (0.49)   \\
\hspace{12pt}1998   &        13.0&        13.1&       -0.15   \\
                    &      [1.73]&      [1.62]&      (0.97)   \\
\hspace{12pt}1999   &        12.8&        13.6&       -0.75   \\
                    &      [0.61]&      [0.98]&      (0.47)   \\
\hspace{12pt}2000   &        13.1&        13.6&       -0.45   \\
                    &      [0.25]&      [1.10]&      (0.46)   \\
\hspace{12pt}2001   &        14.2&        16.5&       -2.32** \\
                    &      [1.36]&      [1.29]&      (0.76)   \\
\hspace{12pt}2002   &        15.3&        15.3&     -0.0090   \\
                    &      [1.28]&      [0.65]&      (0.59)   \\
\hspace{12pt}2003   &        16.1&        15.6&        0.55   \\
                    &      [1.85]&      [0.74]&      (0.81)   \\
\hspace{12pt}2004   &        15.7&        16.3&       -0.62   \\
                    &      [1.45]&      [1.15]&      (0.75)   \\
\hspace{12pt}2005   &        16.2&        15.9&        0.35   \\
                    &      [0.87]&      [1.62]&      (0.75)   \\
\hspace{12pt}2006   &        16.7&        16.3&        0.40   \\
                    &      [0.78]&      [1.20]&      (0.58)   \\
\hspace{12pt}2007   &        17.7&        16.9&        0.77   \\
                    &      [1.31]&      [0.98]&      (0.67)   \\
\hspace{12pt}2008   &        18.0&        16.9&        1.16   \\
                    &      [1.66]&      [1.23]&      (0.84)   \\
\hspace{12pt}2009   &        18.2&        17.4&        0.76   \\
                    &      [1.47]&      [1.58]&      (0.88)   \\
\hspace{12pt}2010   &        18.6&        18.1&        0.53   \\
                    &      [0.69]&      [1.24]&      (0.58)   \\
\hspace{12pt}2011   &        18.2&        18.3&       -0.13   \\
                    &      [1.05]&      [1.44]&      (0.73)   \\
\hspace{12pt}2012   &        20.0&        20.0&     0.00093   \\
                    &      [0.98]&      [1.26]&      (0.65)   \\
\hspace{12pt}2013   &        20.2&        20.8&       -0.56   \\
                    &      [1.02]&      [1.48]&      (0.73)   \\
\hspace{12pt}2014   &        21.2&        22.3&       -1.03   \\
                    &      [1.25]&      [1.04]&      (0.66)   \\

					\bottomrule
			\end{tabular}}
		\end{footnotesize}
	\end{threeparttable} 
	\begin{minipage}{0.9\linewidth}
		\scriptsize \emph{Notes:} This table shows descriptive statistics for two samples: (i) cohorts born before May 1979; (ii) cohorts born after May 1979. Columns 1 and 2 show means with standard deviations in brackets. Column 3 report the difference in means between columns 1 and 2 with standard errors in parenthesis. For each sample we take a bandwidth of half a year around the cutoff date, i.e. individuals born between Nov78-Ap79 (May79-Oct79) in the 'before' ('after') sample.
	\end{minipage}
\end{table} 
%d5 men
\begin{table}[H] \centering 
	\begin{threeparttable} \centering \caption{t-test for \textbf{mental \& behavioral disorders (men)}}\label{tab:t-test_d5male}
		\begin{footnotesize}
			{\def\sym#1{\ifmmode^{#1}\else\(^{#1}\)\fi} 
				\begin{tabular}{l*{3}{c}}
					\toprule 
					& \multicolumn{1}{c}{Before} & \multicolumn{1}{c}{After} & \multicolumn{1}{c}{Difference} \\
					&\multicolumn{1}{c}{(1)}&\multicolumn{1}{c}{(2)}&\multicolumn{1}{c}{(1)-(2)}\\
					\midrule
					Overall (pooled)    &        22.0&        21.2&        0.79   \\
                    &      [7.55]&      [7.45]&      (0.97)   \\
\hspace{12pt}1995   &        6.09&        5.26&        0.84** \\
                    &      [0.62]&      [0.67]&      (0.37)   \\
\hspace{12pt}1996   &        8.01&        6.93&        1.08***\\
                    &      [0.52]&      [0.45]&      (0.28)   \\
\hspace{12pt}1997   &        10.5&        9.20&        1.32*  \\
                    &      [1.31]&      [0.82]&      (0.63)   \\
\hspace{12pt}1998   &        13.8&        12.3&        1.48** \\
                    &      [0.95]&      [1.08]&      (0.59)   \\
\hspace{12pt}1999   &        17.2&        17.2&      0.0075   \\
                    &      [1.36]&      [1.05]&      (0.70)   \\
\hspace{12pt}2000   &        19.4&        18.7&        0.67   \\
                    &      [1.57]&      [0.71]&      (0.70)   \\
\hspace{12pt}2001   &        21.2&        21.0&        0.18   \\
                    &      [1.41]&      [1.35]&      (0.80)   \\
\hspace{12pt}2002   &        21.0&        21.2&       -0.13   \\
                    &      [0.93]&      [1.37]&      (0.68)   \\
\hspace{12pt}2003   &        21.7&        21.0&        0.78   \\
                    &      [1.48]&      [0.86]&      (0.70)   \\
\hspace{12pt}2004   &        23.3&        22.2&        1.10   \\
                    &      [1.33]&      [1.11]&      (0.71)   \\
\hspace{12pt}2005   &        23.8&        23.5&        0.34   \\
                    &      [1.48]&      [1.63]&      (0.90)   \\
\hspace{12pt}2006   &        24.7&        23.7&        1.02*  \\
                    &      [0.93]&      [0.92]&      (0.53)   \\
\hspace{12pt}2007   &        25.2&        25.1&       0.092   \\
                    &      [1.86]&      [1.22]&      (0.91)   \\
\hspace{12pt}2008   &        25.9&        25.9&      -0.033   \\
                    &      [1.96]&      [1.35]&      (0.97)   \\
\hspace{12pt}2009   &        26.9&        27.8&       -0.87   \\
                    &      [1.78]&      [1.36]&      (0.91)   \\
\hspace{12pt}2010   &        28.5&        26.7&        1.84*  \\
                    &      [2.02]&      [0.68]&      (0.87)   \\
\hspace{12pt}2011   &        29.3&        28.3&        1.06   \\
                    &      [1.79]&      [0.70]&      (0.78)   \\
\hspace{12pt}2012   &        29.1&        28.4&        0.66   \\
                    &      [0.66]&      [1.36]&      (0.62)   \\
\hspace{12pt}2013   &        31.2&        29.4&        1.75*  \\
                    &      [2.04]&      [1.16]&      (0.96)   \\
\hspace{12pt}2014   &        33.1&        30.5&        2.66** \\
                    &      [1.99]&      [1.44]&      (1.00)   \\

					\bottomrule
			\end{tabular}}
		\end{footnotesize}
	\end{threeparttable} 
	\begin{minipage}{0.9\linewidth}
		\scriptsize \emph{Notes:} This table shows descriptive statistics for two samples: (i) cohorts born before May 1979; (ii) cohorts born after May 1979. Columns 1 and 2 show means with standard deviations in brackets. Column 3 report the difference in means between columns 1 and 2 with standard errors in parenthesis. For each sample we take a bandwidth of half a year around the cutoff date, i.e. individuals born between Nov78-Ap79 (May79-Oct79) in the 'before' ('after') sample.
	\end{minipage}
\end{table} 

