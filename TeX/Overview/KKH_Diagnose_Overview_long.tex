%--------------------------------------------------------------------
%	DOCUMENT CLASS
%--------------------------------------------------------------------
\documentclass{scrartcl} % type of document (paper, presentation, book,...); scrartcl class with sans serif titles, European layout 
\usepackage{fullpage} % leaves less space at margins of page
\usepackage[onehalfspacing]{setspace} % determine line pitch to 1.5

%--------------------------------------------------------------------
%	INPUT
%--------------------------------------------------------------------
\usepackage[T1]{fontenc} % Use 8-bit encoding that has 256 glyphs
\usepackage[utf8]{inputenc} % Required for including letters with accents, Umlaute,...
\usepackage{float} % better control over placement of tables and figures in the text
\usepackage{graphicx} % input of graphics
\usepackage{xcolor} % advanced color package
\usepackage{url, hyperref} % include (clickable) URLs

%--------------------------------------------------------------------
%	TABLES, FIGURES, LISTS
%--------------------------------------------------------------------
\usepackage{booktabs} % better tables
\usepackage{threeparttable}
\renewcommand\TPTrlap{}
        \renewcommand\TPTnoteSettings{%
            \setlength\leftmargin{5pt}%  
            \setlength\rightmargin{5pt}%
          }

\usepackage[center, format=plain, font=normalsize, nooneline, labelfont={bf}]{caption} % change format of captions of tables and graphs 

% Allow line breaks with \\ in column headings of tables
\newcommand{\clb}[3][c]{%
	\begin{tabular}[#1]{@{}#2@{}}#3\end{tabular}}

% allow line breaks with \\ in row titles
\usepackage{multirow}

\newcommand{\lb}[3][c]{%
\multirow{2}{*}{\begin{tabular}[#1]{@{}#2@{}}#3\end{tabular}}}
% optional argument: b = bottom or t= top alignment

%--------------------------------------------------------------------
%	MATH
%--------------------------------------------------------------------
\usepackage{amsmath,amssymb} % more math symbols and commands

%--------------------------------------------------------------------
%	LANGUAGE SPECIFICS
%--------------------------------------------------------------------
\usepackage[american]{babel} % man­ages cul­tur­ally-de­ter­mined ty­po­graph­i­cal (and other) rules, and hy­phen­ation pat­terns
\usepackage{csquotes} % language specific quotations

%--------------------------------------------------------------------
%	PATHS
%--------------------------------------------------------------------
\makeatletter
\def\input@path{{../../analysis/tables/KKH/}}
%or: \def\input@path{{/path/to/folder/}{/path/to/other/folder/}}
\makeatother
\graphicspath{{../../analysis/graphs/KKH/}}




\usepackage[left=1cm,right=1cm,top=2cm,bottom=2cm]{geometry}


%\newgeometry{
%	top    = 3.5cm,
%	bottom = 2cm,
%	left   = 5.0cm,
%	right  = 2.8cm}
%	
\author{Marc Fabel}
\title{Overview of diagnoses outcomes}
\date{Last revision of this document: \today} 

%%%%%%%%%%%%%%%%%%%%%%%%%%%%%%%%%%%%%%%%%%%%%%%%%%%%%%%%%%%%
% BEGIN OF DOCUMENT
%%%%%%%%%%%%%%%%%%%%%%%%%%%%%%%%%%%%%%%%%%%%%%%%%%%%%%%%%%%%
\begin{document}
\maketitle
This document contains the results from the data in long format, i.e. per gender.

\newpage


%%%%%%%%%%%%%%%%%%%%%%%%%%%%%%%%%%%%%%%%%%%%%%%%%%%%%%%%%%%%%%%%%%%%%%%%%%%%%%%%%%%%%%%%%%%%%%%%%%%%%%%%%%%
\section{Meta Daten}
%=========================================
\input{summ_stay}
 \begin{table}[H] \begin{threeparttable} \centering \caption{Robustness with respect to the choice of \texttt{control group}} {\def\sym#1{\ifmmode^{#1}\else\(^{#1}\)\fi} \begin{tabular}{l*{10}{c}} \toprule & \multicolumn{9}{c}{Dependent variable: \textbf{Accumulated length of stay}} \\ \cmidrule(lr){2-10}
            &\multicolumn{3}{c}{Average Causal Effects}&\multicolumn{3}{c}{Women}             &\multicolumn{3}{c}{Men}               \\\cmidrule(lr){2-4}\cmidrule(lr){5-7}\cmidrule(lr){8-10}
            &\multicolumn{1}{c}{(1)}&\multicolumn{1}{c}{(2)}&\multicolumn{1}{c}{(3)}&\multicolumn{1}{c}{(4)}&\multicolumn{1}{c}{(5)}&\multicolumn{1}{c}{(6)}&\multicolumn{1}{c}{(7)}&\multicolumn{1}{c}{(8)}&\multicolumn{1}{c}{(9)}\\
            &\multicolumn{1}{c}{C2}&\multicolumn{1}{c}{C1+C2}&\multicolumn{1}{c}{C1-C3}&\multicolumn{1}{c}{C2}&\multicolumn{1}{c}{C1+C2}&\multicolumn{1}{c}{C1-C3}&\multicolumn{1}{c}{C2}&\multicolumn{1}{c}{C1+C2}&\multicolumn{1}{c}{C1-C3}\\
\midrule
 \multicolumn{10}{l}{\emph{Panel A. 2 Month bandwidth}} \\ Abs. numbers        &       436.1\sym{***}&      -107.7         &      -315.1         &       314.6         &       64.11         &      -230.0         &       121.5         &      -171.8         &      -85.11         \\
                    &     (113.6)         &     (300.1)         &     (425.3)         &     (266.3)         &     (214.3)         &     (425.5)         &     (364.3)         &     (358.9)         &     (374.2)         \\
 Ratio population    &      -33.95\sym{***}&      -15.12         &      -7.924         &      -48.13         &      -21.37         &      -35.37         &      -19.13\sym{*}  &      -7.981         &       12.47         \\
                    &     (6.650)         &     (25.96)         &     (24.38)         &     (28.53)         &     (40.39)         &     (51.30)         &     (8.867)         &     (20.22)         &     (19.38)         \\
 Ratio fertility     &       8.676\sym{*}  &       10.76         &       9.557         &       15.16         &       21.09         &       13.87         &       3.658         &       2.468         &       6.543         \\
                    &     (4.563)         &     (9.360)         &     (9.835)         &     (10.17)         &     (11.99)         &     (14.38)         &     (11.64)         &     (13.59)         &     (13.03)         \\
 \midrule\multicolumn{10}{l}{\emph{Panel B. 4 Month bandwidth}} \\ \input{summ_stay4Ma-cg} Ratio population    &      -35.31\sym{**} &      -24.73         &      -13.52         &       18.98         &       26.84         &       18.24         &      -69.52\sym{**} &      -60.65\sym{**} &      -37.56         \\
                    &     (13.59)         &     (20.38)         &     (18.37)         &     (39.93)         &     (37.19)         &     (39.71)         &     (28.88)         &     (29.04)         &     (29.19)         \\
 \input{summ_stay4Mc-cg} \midrule\multicolumn{10}{l}{\emph{Panel C. 6 Month bandwidth}} \\ \input{summ_stay6Ma-cg} \input{summ_stay6Mb-cg} \input{summ_stay6Mc-cg} \midrule\multicolumn{10}{l}{\emph{Panel D. Donut specification}} \\ \input{summ_stayDMa-cg} \input{summ_stayDMb-cg} \input{summ_stayDMc-cg} 
\bottomrule \end{tabular} } \begin{tablenotes} \item \scriptsize \emph{Notes:} Clustered standard errors in parentheses. All regressions contain Birthmonth FE. Ratios indicate cases per thousand; either approximated population or original number of births. \end{tablenotes} \end{threeparttable} \end{table} 

%=========================================
\input{hospital}
\input{hospital-cg}
%=========================================
\section{Hauptdiagnosekapitel}
 \begin{table}[H] \begin{threeparttable} \centering \caption{Robustness with respect to the inclusion of \texttt{fixed effects} and \texttt{covariates}} {\def\sym#1{\ifmmode^{#1}\else\(^{#1}\)\fi} \begin{tabular}{l*{7}{c}} \toprule & \multicolumn{6}{c}{Dependent variable: \textbf{Certain infectious and parasitic diseases}} \\ \cmidrule(lr){2-7}
            &\multicolumn{4}{c}{Average Causal Effects}         &\multicolumn{2}{c}{Heterogeneous Causal Effects}\\\cmidrule(lr){2-5}\cmidrule(lr){6-7}
            &\multicolumn{1}{c}{(1)}&\multicolumn{1}{c}{(2)}&\multicolumn{1}{c}{(3)}&\multicolumn{1}{c}{(4)}&\multicolumn{1}{c}{(5)}&\multicolumn{1}{c}{(6)}\\
            &\multicolumn{1}{c}{}&\multicolumn{1}{c}{}&\multicolumn{1}{c}{}&\multicolumn{1}{c}{}&\multicolumn{1}{c}{Women}&\multicolumn{1}{c}{Men}\\
\midrule
 \multicolumn{7}{l}{\emph{Panel A. 2 Month bandwidth}} \\ Abs. numbers        &      -10.00         &         -10\sym{**} &         -10\sym{**} &     -10.000\sym{**} &      -9.056\sym{***}&      -0.944         \\
                    &     (11.05)         &     (3.302)         &     (3.522)         &     (3.553)         &     (2.425)         &     (1.559)         \\
 Ratio fertility     &      -0.100         &      -0.100\sym{***}&      -0.100\sym{***}&      -0.100\sym{***}&      -0.162\sym{***}&     -0.0406\sym{*}  \\
                    &     (0.127)         &   (0.00598)         &   (0.00638)         &   (0.00641)         &    (0.0290)         &    (0.0183)         \\
 Ratio fertility     &      -0.136         &      -0.136\sym{***}&      -0.136\sym{**} &      -0.136\sym{**} &      -0.255\sym{***}&     -0.0225         \\
                    &     (0.139)         &    (0.0380)         &    (0.0405)         &    (0.0409)         &    (0.0675)         &    (0.0374)         \\
 \midrule\multicolumn{7}{l}{\emph{Panel B. 4 Month bandwidth}} \\ Abs. numbers        &       2.400         &       2.400         &       2.400         &       2.400         &       1.037         &       1.362         \\
                    &     (6.723)         &     (3.014)         &     (3.111)         &     (3.116)         &     (2.495)         &     (1.170)         \\
 Ratio population    &     -0.0664         &     -0.0664         &     -0.0664         &     -0.0664         &     -0.0233         &      -0.109         \\
                    &     (0.125)         &    (0.0712)         &    (0.0735)         &    (0.0738)         &     (0.147)         &    (0.0901)         \\
 Ratio fertility     &     0.00549         &     0.00549         &     0.00549         &     0.00549         &     -0.0551         &      0.0636         \\
                    &     (0.106)         &    (0.0450)         &    (0.0464)         &    (0.0466)         &    (0.0676)         &    (0.0431)         \\
 \midrule\multicolumn{7}{l}{\emph{Panel C. 6 Month bandwidth}} \\ Abs. numbers        &       4.537         &       4.537         &       4.537         &       4.710         &       2.685         &       1.852         \\
                    &     (5.329)         &     (3.000)         &     (3.061)         &     (3.082)         &     (2.279)         &     (1.560)         \\
 Ratio fertility     &    -0.00748         &    -0.00748         &    -0.00748         &    -0.00455         &      0.0154         &     -0.0292         \\
                    &    (0.0761)         &    (0.0208)         &    (0.0212)         &    (0.0217)         &    (0.0358)         &    (0.0287)         \\
 Ratio fertility     &     -0.0137         &     -0.0137         &     -0.0137         &     -0.0119         &    -0.00254         &     -0.0243         \\
                    &    (0.0901)         &    (0.0310)         &    (0.0316)         &    (0.0316)         &    (0.0494)         &    (0.0456)         \\
 \midrule\multicolumn{7}{l}{\emph{Panel D. Donut specification}} \\ Abs. numbers        &       7.580         &       7.580\sym{***}&       7.580\sym{***}&       7.765\sym{***}&       4.930\sym{**} &       2.650\sym{***}\\
                    &     (6.528)         &     (2.185)         &     (2.241)         &     (2.259)         &     (1.963)         &     (0.686)         \\
 Ratio fertility     &     0.00868         &     0.00868         &     0.00868         &      0.0115         &      0.0364         &     -0.0173         \\
                    &    (0.0691)         &    (0.0233)         &    (0.0238)         &    (0.0243)         &    (0.0394)         &    (0.0303)         \\
 Ratio population    &     0.00730         &     0.00730         &     0.00730         &     0.00587         &    -0.00902         &      0.0239         \\
                    &    (0.0714)         &    (0.0274)         &    (0.0281)         &    (0.0279)         &    (0.0465)         &    (0.0536)         \\
 \midrule \input{d1checks} 
\bottomrule \end{tabular} } \begin{tablenotes} \item \scriptsize \emph{Notes:} Clustered standard errors in parentheses. Personal covariates contain age and age squared. Ratios indicate cases per thousand; either approximated population or original number of births. \end{tablenotes} \end{threeparttable} \end{table} 

\input{d1-cg}
%=========================================
\input{d2}
\input{d2-cg}
%=========================================
 \begin{table}[H] \begin{threeparttable} \centering \caption{Robustness with respect to the inclusion of \texttt{fixed effects} and \texttt{covariates}} {\def\sym#1{\ifmmode^{#1}\else\(^{#1}\)\fi} \begin{tabular}{l*{7}{c}} \toprule & \multicolumn{6}{c}{Dependent variable: \textbf{Mental and behavioral disorders}} \\ \cmidrule(lr){2-7}
            &\multicolumn{4}{c}{Average Causal Effects}         &\multicolumn{2}{c}{Heterogeneous Causal Effects}\\\cmidrule(lr){2-5}\cmidrule(lr){6-7}
            &\multicolumn{1}{c}{(1)}&\multicolumn{1}{c}{(2)}&\multicolumn{1}{c}{(3)}&\multicolumn{1}{c}{(4)}&\multicolumn{1}{c}{(5)}&\multicolumn{1}{c}{(6)}\\
            &\multicolumn{1}{c}{}&\multicolumn{1}{c}{}&\multicolumn{1}{c}{}&\multicolumn{1}{c}{}&\multicolumn{1}{c}{Women}&\multicolumn{1}{c}{Men}\\
\midrule
 \multicolumn{7}{l}{\emph{Panel A. 2 Month bandwidth}} \\ Abs. numbers        &       15.28         &       15.28         &       15.28         &       15.28         &       13.23         &       2.050         \\
                    &     (39.76)         &     (9.196)         &     (9.822)         &     (9.858)         &     (10.55)         &     (3.535)         \\
 Ratio population    &      -1.316\sym{**} &      -1.316\sym{***}&      -1.316\sym{***}&      -1.316\sym{***}&      -0.943         &      -1.606\sym{***}\\
                    &     (0.514)         &     (0.192)         &     (0.205)         &     (0.207)         &     (0.560)         &     (0.446)         \\
 Ratio fertility     &      -0.466         &      -0.466\sym{***}&      -0.466\sym{**} &      -0.466\sym{**} &      -0.117         &      -0.808\sym{**} \\
                    &     (0.446)         &     (0.126)         &     (0.135)         &     (0.136)         &     (0.514)         &     (0.258)         \\
 \midrule\multicolumn{7}{l}{\emph{Panel B. 4 Month bandwidth}} \\ Abs. numbers        &      -12.25         &      -12.25         &      -12.25         &      -12.25         &       8.762         &      -21.01\sym{***}\\
                    &     (30.29)         &     (9.553)         &     (9.860)         &     (9.877)         &     (5.990)         &     (6.783)         \\
 Ratio population    &      -1.567\sym{**} &      -1.567\sym{***}&      -1.567\sym{***}&      -1.567\sym{***}&      -0.224         &      -3.260\sym{***}\\
                    &     (0.569)         &     (0.244)         &     (0.252)         &     (0.253)         &     (0.497)         &     (0.811)         \\
 Ratio population    &      -1.001         &      -1.001\sym{***}&      -1.001\sym{***}&      -1.001\sym{***}&      -0.199         &      -1.826\sym{***}\\
                    &     (0.618)         &     (0.242)         &     (0.249)         &     (0.250)         &     (0.267)         &     (0.364)         \\
 \midrule\multicolumn{7}{l}{\emph{Panel C. 6 Month bandwidth}} \\ Abs. numbers        &      -16.11         &      -16.11         &      -16.11         &      -16.44         &       10.19         &      -26.30\sym{**} \\
                    &     (34.44)         &     (14.92)         &     (15.23)         &     (15.16)         &     (8.336)         &     (9.967)         \\
 Ratio population    &      -1.051\sym{*}  &      -1.051\sym{***}&      -1.051\sym{**} &      -1.047\sym{**} &      -0.207         &      -1.929\sym{**} \\
                    &     (0.533)         &     (0.371)         &     (0.378)         &     (0.387)         &     (0.392)         &     (0.834)         \\
 Ratio fertility     &      -0.681\sym{*}  &      -0.681\sym{***}&      -0.681\sym{***}&      -0.691\sym{***}&     -0.0437         &      -1.280\sym{***}\\
                    &     (0.396)         &     (0.193)         &     (0.197)         &     (0.193)         &     (0.232)         &     (0.279)         \\
 \midrule\multicolumn{7}{l}{\emph{Panel D. Donut specification}} \\ Abs. numbers        &       8.840         &       8.840         &       8.840         &       10.61         &       22.97\sym{***}&      -14.13\sym{*}  \\
                    &     (32.69)         &     (11.62)         &     (11.91)         &     (12.33)         &     (5.369)         &     (7.281)         \\
 Ratio fertility     &      -0.326         &      -0.326         &      -0.326         &      -0.308         &       0.309         &      -0.928\sym{***}\\
                    &     (0.348)         &     (0.194)         &     (0.199)         &     (0.202)         &     (0.194)         &     (0.219)         \\
 Ratio fertility     &      -0.685         &      -0.685\sym{***}&      -0.685\sym{***}&      -0.692\sym{***}&       0.171         &      -1.494\sym{***}\\
                    &     (0.402)         &     (0.233)         &     (0.239)         &     (0.235)         &     (0.235)         &     (0.299)         \\
 \midrule \input{d5checks} 
\bottomrule \end{tabular} } \begin{tablenotes} \item \scriptsize \emph{Notes:} Clustered standard errors in parentheses. Personal covariates contain age and age squared. Ratios indicate cases per thousand; either approximated population or original number of births. \end{tablenotes} \end{threeparttable} \end{table} 

\input{d5-cg}
%=========================================
\input{d6}
 \begin{table}[H] \begin{threeparttable} \centering \caption{Robustness with respect to the choice of \texttt{control group}} {\def\sym#1{\ifmmode^{#1}\else\(^{#1}\)\fi} \begin{tabular}{l*{10}{c}} \toprule & \multicolumn{9}{c}{Dependent variable: \textbf{Diseases of the nervous system}} \\ \cmidrule(lr){2-10}
            &\multicolumn{3}{c}{Average Causal Effects}&\multicolumn{3}{c}{Women}             &\multicolumn{3}{c}{Men}               \\\cmidrule(lr){2-4}\cmidrule(lr){5-7}\cmidrule(lr){8-10}
            &\multicolumn{1}{c}{(1)}&\multicolumn{1}{c}{(2)}&\multicolumn{1}{c}{(3)}&\multicolumn{1}{c}{(4)}&\multicolumn{1}{c}{(5)}&\multicolumn{1}{c}{(6)}&\multicolumn{1}{c}{(7)}&\multicolumn{1}{c}{(8)}&\multicolumn{1}{c}{(9)}\\
            &\multicolumn{1}{c}{C2}&\multicolumn{1}{c}{C1+C2}&\multicolumn{1}{c}{C1-C3}&\multicolumn{1}{c}{C2}&\multicolumn{1}{c}{C1+C2}&\multicolumn{1}{c}{C1-C3}&\multicolumn{1}{c}{C2}&\multicolumn{1}{c}{C1+C2}&\multicolumn{1}{c}{C1-C3}\\
\midrule
 \multicolumn{10}{l}{\emph{Panel A. 2 Month bandwidth}} \\ Abs. numbers        &      -1.056         &      -5.556         &      -4.574         &       4.167         &       2.806         &       3.704         &      -5.222         &      -8.361         &      -8.278         \\
                    &     (7.827)         &     (7.681)         &     (8.355)         &     (4.585)         &     (4.226)         &     (5.311)         &     (6.294)         &     (6.329)         &     (5.657)         \\
 \input{d62Mb-cg} \input{d62Mc-cg} \midrule\multicolumn{10}{l}{\emph{Panel B. 4 Month bandwidth}} \\ \input{d64Ma-cg} Ratio population    &     -0.0807         &     -0.0833         &     -0.0441         &       0.122         &       0.121         &      0.0986         &      -0.274         &      -0.294         &      -0.199         \\
                    &     (0.147)         &     (0.160)         &     (0.161)         &     (0.244)         &     (0.218)         &     (0.221)         &     (0.208)         &     (0.211)         &     (0.215)         \\
 \input{d64Mc-cg} \midrule\multicolumn{10}{l}{\emph{Panel C. 6 Month bandwidth}} \\ \input{d66Ma-cg} Ratio population    &      0.0188         &     0.00383         &     0.00330         &       0.113         &      0.0840         &      0.0608         &     -0.0698         &     -0.0798         &     -0.0596         \\
                    &     (0.120)         &     (0.125)         &     (0.129)         &     (0.162)         &     (0.147)         &     (0.150)         &     (0.189)         &     (0.186)         &     (0.190)         \\
 Ratio fertility     &      0.0937         &      0.0979         &      0.0784         &       0.159\sym{*}  &       0.170\sym{*}  &       0.158\sym{*}  &      0.0308         &      0.0284         &     0.00179         \\
                    &    (0.0725)         &    (0.0782)         &    (0.0911)         &    (0.0886)         &    (0.0843)         &    (0.0907)         &     (0.104)         &     (0.116)         &     (0.129)         \\
 \midrule\multicolumn{10}{l}{\emph{Panel D. Donut specification}} \\ \input{d6DMa-cg} \input{d6DMb-cg} Ratio fertility     &      0.0537         &      0.0764         &      0.0525         &       0.123\sym{*}  &       0.139\sym{*}  &       0.116\sym{*}  &     -0.0132         &      0.0158         &    -0.00865         \\
                    &    (0.0817)         &    (0.0859)         &    (0.0998)         &    (0.0674)         &    (0.0697)         &    (0.0680)         &     (0.118)         &     (0.130)         &     (0.149)         \\
 
\bottomrule \end{tabular} } \begin{tablenotes} \item \scriptsize \emph{Notes:} Clustered standard errors in parentheses. All regressions contain Birthmonth FE. Ratios indicate cases per thousand; either approximated population or original number of births. \end{tablenotes} \end{threeparttable} \end{table} 

%=========================================
\input{d7}
\input{d7-cg}
%=========================================
\input{d8}
\input{d8-cg}
%=========================================
 \begin{table}[H] \begin{threeparttable} \centering \caption{Robustness with respect to the inclusion of \texttt{fixed effects} and \texttt{covariates}} {\def\sym#1{\ifmmode^{#1}\else\(^{#1}\)\fi} \begin{tabular}{l*{7}{c}} \toprule & \multicolumn{6}{c}{Dependent variable: \textbf{Diseases of the respiratory system}} \\ \cmidrule(lr){2-7}
            &\multicolumn{4}{c}{Average Causal Effects}         &\multicolumn{2}{c}{Heterogeneous Causal Effects}\\\cmidrule(lr){2-5}\cmidrule(lr){6-7}
            &\multicolumn{1}{c}{(1)}&\multicolumn{1}{c}{(2)}&\multicolumn{1}{c}{(3)}&\multicolumn{1}{c}{(4)}&\multicolumn{1}{c}{(5)}&\multicolumn{1}{c}{(6)}\\
            &\multicolumn{1}{c}{}&\multicolumn{1}{c}{}&\multicolumn{1}{c}{}&\multicolumn{1}{c}{}&\multicolumn{1}{c}{Women}&\multicolumn{1}{c}{Men}\\
\midrule
 \multicolumn{7}{l}{\emph{Panel A. 2 Month bandwidth}} \\ Abs. numbers        &       0.833         &       0.833         &       0.833         &       0.833         &       9.444\sym{**} &      -8.611         \\
                    &     (16.80)         &     (8.277)         &     (8.830)         &     (8.907)         &     (3.926)         &     (4.667)         \\
 Ratio population    &      -0.287         &      -0.287\sym{**} &      -0.287\sym{**} &      -0.287\sym{**} &     -0.0490         &      -0.508\sym{***}\\
                    &     (0.225)         &     (0.106)         &     (0.113)         &     (0.114)         &     (0.285)         &     (0.113)         \\
 Ratio population    &     -0.0202         &     -0.0202         &     -0.0202         &     -0.0202         &       0.221\sym{**} &      -0.264\sym{***}\\
                    &     (0.182)         &    (0.0697)         &    (0.0744)         &    (0.0749)         &    (0.0844)         &    (0.0741)         \\
 \midrule\multicolumn{7}{l}{\emph{Panel B. 4 Month bandwidth}} \\ Abs. numbers        &       7.861         &       7.861         &       7.861         &       7.861         &       11.53\sym{***}&      -3.667         \\
                    &     (13.74)         &     (4.848)         &     (4.999)         &     (5.019)         &     (2.179)         &     (3.034)         \\
 Ratio population    &      -0.171         &      -0.171\sym{*}  &      -0.171\sym{*}  &      -0.171\sym{*}  &       0.357         &      -0.744\sym{**} \\
                    &     (0.273)         &    (0.0933)         &    (0.0962)         &    (0.0966)         &     (0.253)         &     (0.282)         \\
 Ratio fertility     &   -0.000637         &   -0.000637         &   -0.000637         &   -0.000637         &       0.254\sym{***}&      -0.244\sym{**} \\
                    &     (0.229)         &    (0.0717)         &    (0.0739)         &    (0.0742)         &    (0.0686)         &    (0.0854)         \\
 \midrule\multicolumn{7}{l}{\emph{Panel C. 6 Month bandwidth}} \\ Abs. numbers        &       9.825         &       9.825\sym{*}  &       9.825\sym{*}  &       9.966\sym{*}  &       8.167\sym{***}&       1.658         \\
                    &     (16.11)         &     (5.381)         &     (5.494)         &     (5.467)         &     (2.575)         &     (3.663)         \\
 Ratio population    &     -0.0948         &     -0.0948         &     -0.0948         &     -0.0799         &      0.0896         &      -0.276         \\
                    &     (0.238)         &     (0.138)         &     (0.141)         &     (0.147)         &     (0.218)         &     (0.276)         \\
 Ratio fertility     &      0.0273         &      0.0273         &      0.0273         &      0.0354         &       0.145         &     -0.0852         \\
                    &     (0.172)         &    (0.0635)         &    (0.0647)         &    (0.0660)         &    (0.0876)         &    (0.0816)         \\
 \midrule\multicolumn{7}{l}{\emph{Panel D. Donut specification}} \\ Abs. numbers        &       13.21         &       13.21\sym{**} &       13.21\sym{**} &       13.31\sym{**} &       7.750\sym{**} &       5.460         \\
                    &     (18.73)         &     (5.740)         &     (5.886)         &     (5.856)         &     (2.801)         &     (3.713)         \\
 Ratio population    &      -0.109         &      -0.109         &      -0.109         &     -0.0924         &     -0.0140         &      -0.200         \\
                    &     (0.279)         &     (0.166)         &     (0.170)         &     (0.179)         &     (0.256)         &     (0.330)         \\
 Ratio fertility     &     -0.0320         &     -0.0320         &     -0.0320         &     -0.0218         &      0.0604         &      -0.121         \\
                    &     (0.183)         &    (0.0636)         &    (0.0652)         &    (0.0669)         &    (0.0907)         &    (0.0924)         \\
 \midrule \input{d9checks} 
\bottomrule \end{tabular} } \begin{tablenotes} \item \scriptsize \emph{Notes:} Clustered standard errors in parentheses. Personal covariates contain age and age squared. Ratios indicate cases per thousand; either approximated population or original number of births. \end{tablenotes} \end{threeparttable} \end{table} 

\input{d9-cg}
%=========================================
\input{d10}
\input{d10-cg}
%=========================================
\input{d11}
\input{d11-cg}
%=========================================
\input{d12}
\input{d12-cg}
%=========================================
\input{d13}
\input{d13-cg}
%=========================================
\input{d17}
\input{d17-cg}
%=========================================
\input{d18}
\input{d18-cg}
%=========================================
\section{Einzeldiagnosen}
\input{injuries}
\input{injuries-cg}
%=========================================
\input{neurosis}
\input{neurosis-cg}
%=========================================
\input{joints}
\input{joints-cg}
%=========================================
\input{kidneys}
\input{kidneys-cg}
%=========================================
\input{bile_pancreas}
 \begin{table}[H] \begin{threeparttable} \centering \caption{Robustness with respect to the choice of \texttt{control group}} {\def\sym#1{\ifmmode^{#1}\else\(^{#1}\)\fi} \begin{tabular}{l*{10}{c}} \toprule & \multicolumn{9}{c}{Dependent variable: \textbf{Diseases of the bile and pancreas}} \\ \cmidrule(lr){2-10}
            &\multicolumn{3}{c}{Average Causal Effects}&\multicolumn{3}{c}{Women}             &\multicolumn{3}{c}{Men}               \\\cmidrule(lr){2-4}\cmidrule(lr){5-7}\cmidrule(lr){8-10}
            &\multicolumn{1}{c}{(1)}&\multicolumn{1}{c}{(2)}&\multicolumn{1}{c}{(3)}&\multicolumn{1}{c}{(4)}&\multicolumn{1}{c}{(5)}&\multicolumn{1}{c}{(6)}&\multicolumn{1}{c}{(7)}&\multicolumn{1}{c}{(8)}&\multicolumn{1}{c}{(9)}\\
            &\multicolumn{1}{c}{C2}&\multicolumn{1}{c}{C1+C2}&\multicolumn{1}{c}{C1-C3}&\multicolumn{1}{c}{C2}&\multicolumn{1}{c}{C1+C2}&\multicolumn{1}{c}{C1-C3}&\multicolumn{1}{c}{C2}&\multicolumn{1}{c}{C1+C2}&\multicolumn{1}{c}{C1-C3}\\
\midrule
 \multicolumn{10}{l}{\emph{Panel A. 2 Month bandwidth}} \\ Abs. numbers        &      -1.056         &      -3.056         &      -6.241         &      -3.556\sym{***}&      -3.139         &      -4.741\sym{**} &       2.500         &      0.0833         &      -1.500         \\
                    &     (3.916)         &     (3.769)         &     (3.692)         &     (0.785)         &     (2.945)         &     (2.214)         &     (3.184)         &     (2.791)         &     (3.310)         \\
 \input{bile_pancreas2Mb-cg} Ratio fertility     &     -0.0129         &     -0.0138         &     -0.0540         &     -0.0897\sym{***}&     -0.0371         &     -0.0799         &      0.0634         &      0.0127         &     -0.0263         \\
                    &    (0.0502)         &    (0.0583)         &    (0.0714)         &    (0.0156)         &     (0.111)         &    (0.0910)         &    (0.0845)         &    (0.0693)         &    (0.0961)         \\
 \midrule\multicolumn{10}{l}{\emph{Panel B. 4 Month bandwidth}} \\ Abs. numbers        &      -4.056         &      -2.458         &      -3.546         &      -3.556         &      -2.681         &      -2.787         &      -0.500         &       0.222         &      -0.759         \\
                    &     (4.434)         &     (4.013)         &     (4.409)         &     (2.748)         &     (2.678)         &     (2.550)         &     (2.225)         &     (2.135)         &     (2.588)         \\
 Ratio population    &      -0.170\sym{**} &      -0.120         &      -0.107         &      -0.122         &     -0.0787         &     -0.0908         &      -0.144         &      -0.109         &      -0.103         \\
                    &    (0.0677)         &    (0.0737)         &    (0.0893)         &    (0.0792)         &     (0.100)         &     (0.106)         &     (0.105)         &    (0.0899)         &     (0.111)         \\
 Ratio fertility     &     -0.0996         &     -0.0447         &     -0.0551         &      -0.153         &     -0.0811         &     -0.0773         &     -0.0431         &    -0.00239         &     -0.0273         \\
                    &    (0.0838)         &    (0.0792)         &    (0.0915)         &     (0.106)         &     (0.107)         &     (0.106)         &    (0.0705)         &    (0.0688)         &    (0.0883)         \\
 \midrule\multicolumn{10}{l}{\emph{Panel C. 6 Month bandwidth}} \\ Abs. numbers        &      -3.537         &      -0.620         &      -0.988         &      -3.333         &      -1.426         &      -1.272         &      -0.204         &       0.806         &       0.284         \\
                    &     (3.036)         &     (3.059)         &     (3.583)         &     (1.970)         &     (2.047)         &     (2.012)         &     (1.700)         &     (1.644)         &     (2.147)         \\
 \input{bile_pancreas6Mb-cg} \input{bile_pancreas6Mc-cg} \midrule\multicolumn{10}{l}{\emph{Panel D. Donut specification}} \\ \input{bile_pancreasDMa-cg} \input{bile_pancreasDMb-cg} \input{bile_pancreasDMc-cg} 
\bottomrule \end{tabular} } \begin{tablenotes} \item \scriptsize \emph{Notes:} Clustered standard errors in parentheses. All regressions contain Birthmonth FE. Ratios indicate cases per thousand; either approximated population or original number of births. \end{tablenotes} \end{threeparttable} \end{table} 

%=========================================


\end{document}