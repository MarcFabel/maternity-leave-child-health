%--------------------------------------------------------------------
%	DOCUMENT CLASS
%--------------------------------------------------------------------
\documentclass{scrartcl} % type of document (paper, presentation, book,...); scrartcl class with sans serif titles, European layout 
\usepackage{fullpage} % leaves less space at margins of page
\usepackage[onehalfspacing]{setspace} % determine line pitch to 1.5

%--------------------------------------------------------------------
%	INPUT
%--------------------------------------------------------------------
\usepackage[T1]{fontenc} % Use 8-bit encoding that has 256 glyphs
\usepackage[utf8]{inputenc} % Required for including letters with accents, Umlaute,...
\usepackage{float} % better control over placement of tables and figures in the text
\usepackage{graphicx} % input of graphics
\usepackage{xcolor} % advanced color package
\usepackage{url, hyperref} % include (clickable) URLs

%--------------------------------------------------------------------
%	TABLES, FIGURES, LISTS
%--------------------------------------------------------------------
\usepackage{booktabs} % better tables
\usepackage{threeparttable}
\renewcommand\TPTrlap{}
        \renewcommand\TPTnoteSettings{%
            \setlength\leftmargin{5pt}%  
            \setlength\rightmargin{5pt}%
          }

\usepackage[center, format=plain, font=normalsize, nooneline, labelfont={bf}]{caption} % change format of captions of tables and graphs 

% Allow line breaks with \\ in column headings of tables
\newcommand{\clb}[3][c]{%
	\begin{tabular}[#1]{@{}#2@{}}#3\end{tabular}}

% allow line breaks with \\ in row titles
\usepackage{multirow}

\newcommand{\lb}[3][c]{%
\multirow{2}{*}{\begin{tabular}[#1]{@{}#2@{}}#3\end{tabular}}}
% optional argument: b = bottom or t= top alignment

%--------------------------------------------------------------------
%	MATH
%--------------------------------------------------------------------
\usepackage{amsmath,amssymb} % more math symbols and commands

%--------------------------------------------------------------------
%	LANGUAGE SPECIFICS
%--------------------------------------------------------------------
\usepackage[american]{babel} % man­ages cul­tur­ally-de­ter­mined ty­po­graph­i­cal (and other) rules, and hy­phen­ation pat­terns
\usepackage{csquotes} % language specific quotations

%--------------------------------------------------------------------
%	PATHS
%--------------------------------------------------------------------
\makeatletter
\def\input@path{{../../analysis/tables/KKH/}}
%or: \def\input@path{{/path/to/folder/}{/path/to/other/folder/}}
\makeatother
\graphicspath{{../../analysis/graphs/KKH/}}




\usepackage[left=1cm,right=1cm,top=2cm,bottom=2cm]{geometry}


%\newgeometry{
%	top    = 3.5cm,
%	bottom = 2cm,
%	left   = 5.0cm,
%	right  = 2.8cm}
%	
\author{Marc Fabel}
\title{Overview of diagnoses outcomes}
\date{Last revision of this document: \today} 

%%%%%%%%%%%%%%%%%%%%%%%%%%%%%%%%%%%%%%%%%%%%%%%%%%%%%%%%%%%%
% BEGIN OF DOCUMENT
%%%%%%%%%%%%%%%%%%%%%%%%%%%%%%%%%%%%%%%%%%%%%%%%%%%%%%%%%%%%
\begin{document}
\maketitle
This document contains the results from the data in long format, i.e. per gender.

\newpage


%%%%%%%%%%%%%%%%%%%%%%%%%%%%%%%%%%%%%%%%%%%%%%%%%%%%%%%%%%%%%%%%%%%%%%%%%%%%%%%%%%%%%%%%%%%%%%%%%%%%%%%%%%%
\section{Meta Daten}
%=========================================
 \begin{table}[H] \begin{threeparttable} \centering \caption{Robustness with respect to the inclusion of \texttt{fixed effects} and \texttt{covariates}} {\def\sym#1{\ifmmode^{#1}\else\(^{#1}\)\fi} \begin{tabular}{l*{7}{c}} \toprule & \multicolumn{6}{c}{Dependent variable: \textbf{Accumulated length of stay}} \\ \cmidrule(lr){2-7}
            &\multicolumn{4}{c}{Average Causal Effects}         &\multicolumn{2}{c}{Heterogeneous Causal Effects}\\\cmidrule(lr){2-5}\cmidrule(lr){6-7}
            &\multicolumn{1}{c}{(1)}&\multicolumn{1}{c}{(2)}&\multicolumn{1}{c}{(3)}&\multicolumn{1}{c}{(4)}&\multicolumn{1}{c}{(5)}&\multicolumn{1}{c}{(6)}\\
            &\multicolumn{1}{c}{}&\multicolumn{1}{c}{}&\multicolumn{1}{c}{}&\multicolumn{1}{c}{}&\multicolumn{1}{c}{Women}&\multicolumn{1}{c}{Men}\\
\midrule
 \multicolumn{7}{l}{\emph{Panel A. 2 Month bandwidth}} \\ Abs. numbers        &       976.8         &       976.8\sym{***}&       976.8\sym{***}&       976.8\sym{***}&       681.9\sym{***}&       294.9         \\
                    &    (2263.3)         &     (113.9)         &     (121.7)         &     (122.1)         &     (136.4)         &     (211.0)         \\
 Ratio population    &      -33.95         &      -33.95\sym{***}&      -33.95\sym{***}&      -33.95\sym{***}&      -48.13         &      -19.13\sym{*}  \\
                    &     (23.46)         &     (6.650)         &     (7.093)         &     (7.155)         &     (28.53)         &     (8.867)         \\
 Ratio population    &       11.41         &       11.41\sym{*}  &       11.41\sym{*}  &       11.41\sym{*}  &       25.63\sym{**} &      -1.758         \\
                    &     (19.22)         &     (5.545)         &     (5.918)         &     (5.956)         &     (8.682)         &     (12.09)         \\
 \midrule\multicolumn{7}{l}{\emph{Panel B. 4 Month bandwidth}} \\ Abs. numbers        &       769.7         &       769.7         &       769.7         &       769.7         &       813.6\sym{***}&      -43.87         \\
                    &    (1601.9)         &     (460.5)         &     (475.3)         &     (476.1)         &     (194.1)         &     (314.3)         \\
 Ratio population    &      -35.31         &      -35.31\sym{**} &      -35.31\sym{**} &      -35.31\sym{**} &       18.98         &      -69.52\sym{**} \\
                    &     (30.31)         &     (13.59)         &     (14.02)         &     (14.07)         &     (39.93)         &     (28.88)         \\
 Ratio fertility     &      -11.25         &      -11.25\sym{*}  &      -11.25\sym{*}  &      -11.25\sym{*}  &       2.287         &      -22.57\sym{**} \\
                    &     (23.74)         &     (6.188)         &     (6.381)         &     (6.407)         &     (6.326)         &     (9.378)         \\
 \midrule\multicolumn{7}{l}{\emph{Panel C. 6 Month bandwidth}} \\ Abs. numbers        &      1730.7         &      1730.7\sym{***}&      1730.7\sym{***}&      1822.8\sym{***}&      1101.0\sym{***}&       629.7\sym{*}  \\
                    &    (1617.8)         &     (483.5)         &     (493.7)         &     (502.1)         &     (205.9)         &     (316.4)         \\
 Ratio fertility     &       2.822         &       2.822         &       2.822         &       4.085         &       4.876         &       0.820         \\
                    &     (16.08)         &     (4.268)         &     (4.358)         &     (4.544)         &     (5.601)         &     (6.801)         \\
 Ratio population    &      -1.659         &      -1.659         &      -1.659         &      -0.493         &       3.482         &      -7.039         \\
                    &     (18.35)         &     (5.346)         &     (5.456)         &     (5.500)         &     (6.666)         &     (8.764)         \\
 \midrule\multicolumn{7}{l}{\emph{Panel D. Donut specification}} \\ Abs. numbers        &      1597.1         &      1597.1\sym{**} &      1597.1\sym{**} &      1687.5\sym{**} &      1235.0\sym{***}&       362.1         \\
                    &    (2016.1)         &     (656.4)         &     (672.6)         &     (689.8)         &     (293.0)         &     (418.9)         \\
 Ratio population    &      -17.84         &      -17.84         &      -17.84         &      -15.69         &      -15.56         &      -17.53         \\
                    &     (30.20)         &     (16.87)         &     (17.29)         &     (18.67)         &     (34.28)         &     (34.56)         \\
 Ratio fertility     &      -6.936         &      -6.936         &      -6.936         &      -5.738         &      -2.060         &      -11.40         \\
                    &     (18.82)         &     (5.254)         &     (5.384)         &     (5.633)         &     (6.880)         &     (9.654)         \\
 \midrule Birthmonth FE       &                     &  \checkmark         &  \checkmark         &  \checkmark         &  \checkmark         &  \checkmark         \\
Year FE             &                     &                     &  \checkmark         &  \checkmark         &                     &                     \\
Covariates          &                     &                     &                     &  \checkmark         &                     &                     \\
 
\bottomrule \end{tabular} } \begin{tablenotes} \item \scriptsize \emph{Notes:} Clustered standard errors in parentheses. Personal covariates contain age and age squared. Ratios indicate cases per thousand; either approximated population or original number of births. \end{tablenotes} \end{threeparttable} \end{table} 

 \begin{table}[H] \begin{threeparttable} \centering \caption{Robustness with respect to the choice of \texttt{control group}} {\def\sym#1{\ifmmode^{#1}\else\(^{#1}\)\fi} \begin{tabular}{l*{10}{c}} \toprule & \multicolumn{9}{c}{Dependent variable: \textbf{Accumulated length of stay}} \\ \cmidrule(lr){2-10}
            &\multicolumn{3}{c}{Average Causal Effects}&\multicolumn{3}{c}{Women}             &\multicolumn{3}{c}{Men}               \\\cmidrule(lr){2-4}\cmidrule(lr){5-7}\cmidrule(lr){8-10}
            &\multicolumn{1}{c}{(1)}&\multicolumn{1}{c}{(2)}&\multicolumn{1}{c}{(3)}&\multicolumn{1}{c}{(4)}&\multicolumn{1}{c}{(5)}&\multicolumn{1}{c}{(6)}&\multicolumn{1}{c}{(7)}&\multicolumn{1}{c}{(8)}&\multicolumn{1}{c}{(9)}\\
            &\multicolumn{1}{c}{C2}&\multicolumn{1}{c}{C1+C2}&\multicolumn{1}{c}{C1-C3}&\multicolumn{1}{c}{C2}&\multicolumn{1}{c}{C1+C2}&\multicolumn{1}{c}{C1-C3}&\multicolumn{1}{c}{C2}&\multicolumn{1}{c}{C1+C2}&\multicolumn{1}{c}{C1-C3}\\
\midrule
 \multicolumn{10}{l}{\emph{Panel A. 2 Month bandwidth}} \\ Abs. numbers        &       436.1\sym{***}&      -107.7         &      -315.1         &       314.6         &       64.11         &      -230.0         &       121.5         &      -171.8         &      -85.11         \\
                    &     (113.6)         &     (300.1)         &     (425.3)         &     (266.3)         &     (214.3)         &     (425.5)         &     (364.3)         &     (358.9)         &     (374.2)         \\
 Ratio population    &      -33.95\sym{***}&      -15.12         &      -7.924         &      -48.13         &      -21.37         &      -35.37         &      -19.13\sym{*}  &      -7.981         &       12.47         \\
                    &     (6.650)         &     (25.96)         &     (24.38)         &     (28.53)         &     (40.39)         &     (51.30)         &     (8.867)         &     (20.22)         &     (19.38)         \\
 Ratio fertility     &       8.676\sym{*}  &       10.76         &       9.557         &       15.16         &       21.09         &       13.87         &       3.658         &       2.468         &       6.543         \\
                    &     (4.563)         &     (9.360)         &     (9.835)         &     (10.17)         &     (11.99)         &     (14.38)         &     (11.64)         &     (13.59)         &     (13.03)         \\
 \midrule\multicolumn{10}{l}{\emph{Panel B. 4 Month bandwidth}} \\ Abs. numbers        &       263.9         &       379.5         &       283.3         &       589.0\sym{*}  &       708.5\sym{*}  &       514.3         &      -325.1         &      -329.0         &      -231.0         \\
                    &     (528.8)         &     (652.2)         &     (703.0)         &     (330.1)         &     (404.7)         &     (496.1)         &     (320.3)         &     (332.9)         &     (324.3)         \\
 Ratio population    &      -35.31\sym{**} &      -24.73         &      -13.52         &       18.98         &       26.84         &       18.24         &      -69.52\sym{**} &      -60.65\sym{**} &      -37.56         \\
                    &     (13.59)         &     (20.38)         &     (18.37)         &     (39.93)         &     (37.19)         &     (39.71)         &     (28.88)         &     (29.04)         &     (29.19)         \\
 Ratio fertility     &      -11.25\sym{*}  &       1.254         &       2.755         &       2.287         &       21.26\sym{**} &       18.51         &      -22.57\sym{**} &      -15.54         &      -10.35         \\
                    &     (6.188)         &     (7.434)         &     (9.319)         &     (6.326)         &     (9.437)         &     (12.16)         &     (9.378)         &     (9.160)         &     (10.43)         \\
 \midrule\multicolumn{10}{l}{\emph{Panel C. 6 Month bandwidth}} \\ Abs. numbers        &      1451.1\sym{**} &      1169.3\sym{**} &       922.0         &       979.9\sym{***}&       809.0\sym{**} &       553.8         &       471.3         &       360.3         &       368.2         \\
                    &     (611.7)         &     (568.3)         &     (594.3)         &     (329.7)         &     (329.3)         &     (403.4)         &     (361.2)         &     (325.4)         &     (304.4)         \\
 Ratio population    &      -20.85         &      -16.05         &      -8.724         &      -13.81         &      -14.22         &      -16.15         &      -20.77         &      -14.56         &      -1.610         \\
                    &     (14.97)         &     (16.45)         &     (15.20)         &     (29.21)         &     (28.87)         &     (30.05)         &     (28.88)         &     (26.40)         &     (25.28)         \\
 Ratio fertility     &      -3.314         &       5.149         &       7.681         &      -2.696         &       8.743         &       9.916         &      -3.964         &       1.891         &       5.995         \\
                    &     (4.858)         &     (5.786)         &     (7.527)         &     (6.051)         &     (8.574)         &     (10.00)         &     (8.784)         &     (8.602)         &     (9.578)         \\
 \midrule\multicolumn{10}{l}{\emph{Panel D. Donut specification}} \\ Abs. numbers        &      1597.1\sym{**} &      1327.2\sym{**} &      1046.5         &      1235.0\sym{***}&      1035.4\sym{***}&       746.4\sym{*}  &       362.1         &       291.8         &       300.1         \\
                    &     (656.4)         &     (597.5)         &     (636.3)         &     (293.0)         &     (317.6)         &     (391.1)         &     (418.9)         &     (368.7)         &     (354.0)         \\
 Ratio population    &      -17.84         &      -20.20         &      -14.86         &      -15.56         &      -20.50         &      -29.21         &      -17.53         &      -19.05         &      -4.559         \\
                    &     (16.87)         &     (15.76)         &     (14.46)         &     (34.28)         &     (31.18)         &     (32.86)         &     (34.56)         &     (30.47)         &     (29.27)         \\
 Ratio fertility     &      -6.936         &       2.027         &       6.172         &      -2.060         &       9.464         &       11.92         &      -11.40         &      -4.462         &       1.403         \\
                    &     (5.254)         &     (6.213)         &     (8.196)         &     (6.880)         &     (9.737)         &     (10.79)         &     (9.654)         &     (9.492)         &     (10.68)         \\
 
\bottomrule \end{tabular} } \begin{tablenotes} \item \scriptsize \emph{Notes:} Clustered standard errors in parentheses. All regressions contain Birthmonth FE. Ratios indicate cases per thousand; either approximated population or original number of births. \end{tablenotes} \end{threeparttable} \end{table} 

%=========================================
 \begin{table}[H] \begin{threeparttable} \centering \caption{Robustness with respect to the inclusion of \texttt{fixed effects} and \texttt{covariates}} {\def\sym#1{\ifmmode^{#1}\else\(^{#1}\)\fi} \begin{tabular}{l*{7}{c}} \toprule & \multicolumn{6}{c}{Dependent variable: \textbf{Hospital admission}} \\ \cmidrule(lr){2-7}
            &\multicolumn{4}{c}{Average Causal Effects}         &\multicolumn{2}{c}{Heterogeneous Causal Effects}\\\cmidrule(lr){2-5}\cmidrule(lr){6-7}
            &\multicolumn{1}{c}{(1)}&\multicolumn{1}{c}{(2)}&\multicolumn{1}{c}{(3)}&\multicolumn{1}{c}{(4)}&\multicolumn{1}{c}{(5)}&\multicolumn{1}{c}{(6)}\\
            &\multicolumn{1}{c}{}&\multicolumn{1}{c}{}&\multicolumn{1}{c}{}&\multicolumn{1}{c}{}&\multicolumn{1}{c}{Women}&\multicolumn{1}{c}{Men}\\
\midrule
 \multicolumn{7}{l}{\emph{Panel A. 2 Month bandwidth}} \\ Abs. numbers        &       15.00         &       15.00         &       15.00         &       15.00         &       46.78         &      -31.78         \\
                    &     (363.9)         &     (24.38)         &     (26.01)         &     (26.24)         &     (33.06)         &     (31.46)         \\
 Ratio population    &      -6.298\sym{*}  &      -6.298\sym{***}&      -6.298\sym{***}&      -6.298\sym{***}&      -8.808         &      -3.965         \\
                    &     (2.965)         &     (0.946)         &     (1.009)         &     (1.018)         &     (4.944)         &     (2.284)         \\
 Ratio population    &       1.043         &       1.043\sym{*}  &       1.043         &       1.043         &       3.430\sym{**} &      -1.097         \\
                    &     (2.757)         &     (0.527)         &     (0.563)         &     (0.567)         &     (1.199)         &     (0.613)         \\
 \midrule\multicolumn{7}{l}{\emph{Panel B. 4 Month bandwidth}} \\ Abs. numbers        &       33.83         &       33.83         &       33.83         &       33.83         &       80.81\sym{*}  &      -46.97\sym{**} \\
                    &     (275.9)         &     (51.74)         &     (53.36)         &     (53.57)         &     (42.63)         &     (21.52)         \\
 Ratio fertility     &      -0.668         &      -0.668         &      -0.668         &      -0.668         &       0.663         &      -1.664\sym{*}  \\
                    &     (3.122)         &     (0.892)         &     (0.920)         &     (0.922)         &     (0.875)         &     (0.893)         \\
 Ratio population    &      -1.724         &      -1.724         &      -1.724         &      -1.724         &       0.171         &      -3.264\sym{**} \\
                    &     (3.777)         &     (1.069)         &     (1.103)         &     (1.106)         &     (1.157)         &     (1.131)         \\
 \midrule\multicolumn{7}{l}{\emph{Panel C. 6 Month bandwidth}} \\ Abs. numbers        &       185.3         &       185.3\sym{***}&       185.3\sym{***}&       201.0\sym{***}&       131.6\sym{***}&       53.70\sym{**} \\
                    &     (229.5)         &     (46.42)         &     (47.39)         &     (49.43)         &     (29.13)         &     (21.49)         \\
 Ratio population    &      -4.517         &      -4.517\sym{**} &      -4.517\sym{**} &      -4.053\sym{*}  &      -3.872         &      -3.830         \\
                    &     (4.152)         &     (2.105)         &     (2.148)         &     (2.324)         &     (5.136)         &     (3.561)         \\
 Ratio fertility     &      -1.608         &      -1.608\sym{**} &      -1.608\sym{**} &      -1.299\sym{*}  &      -1.891\sym{*}  &      -1.371         \\
                    &     (2.777)         &     (0.655)         &     (0.669)         &     (0.700)         &     (0.970)         &     (0.890)         \\
 \midrule\multicolumn{7}{l}{\emph{Panel D. Donut specification}} \\ Abs. numbers        &       204.7         &       204.7\sym{***}&       204.7\sym{***}&       223.2\sym{***}&       143.2\sym{***}&       61.46\sym{**} \\
                    &     (262.4)         &     (49.00)         &     (50.25)         &     (53.53)         &     (28.52)         &     (24.97)         \\
 Ratio population    &      -3.830         &      -3.830         &      -3.830         &      -3.290         &      -4.534         &      -2.895         \\
                    &     (4.894)         &     (2.451)         &     (2.512)         &     (2.748)         &     (6.106)         &     (4.260)         \\
 Ratio fertility     &      -1.963         &      -1.963\sym{**} &      -1.963\sym{**} &      -1.585\sym{*}  &      -2.162\sym{*}  &      -1.757\sym{*}  \\
                    &     (3.163)         &     (0.744)         &     (0.762)         &     (0.805)         &     (1.159)         &     (0.989)         \\
 \midrule Birthmonth FE       &                     &  \checkmark         &  \checkmark         &  \checkmark         &  \checkmark         &  \checkmark         \\
Year FE             &                     &                     &  \checkmark         &  \checkmark         &                     &                     \\
Covariates          &                     &                     &                     &  \checkmark         &                     &                     \\
 
\bottomrule \end{tabular} } \begin{tablenotes} \item \scriptsize \emph{Notes:} Clustered standard errors in parentheses. Personal covariates contain age and age squared. Ratios indicate cases per thousand; either approximated population or original number of births. \end{tablenotes} \end{threeparttable} \end{table} 

 \begin{table}[H] \begin{threeparttable} \centering \caption{Robustness with respect to the choice of \texttt{control group}} {\def\sym#1{\ifmmode^{#1}\else\(^{#1}\)\fi} \begin{tabular}{l*{10}{c}} \toprule & \multicolumn{9}{c}{Dependent variable: \textbf{Hospital admission}} \\ \cmidrule(lr){2-10}
            &\multicolumn{3}{c}{Average Causal Effects}&\multicolumn{3}{c}{Women}             &\multicolumn{3}{c}{Men}               \\\cmidrule(lr){2-4}\cmidrule(lr){5-7}\cmidrule(lr){8-10}
            &\multicolumn{1}{c}{(1)}&\multicolumn{1}{c}{(2)}&\multicolumn{1}{c}{(3)}&\multicolumn{1}{c}{(4)}&\multicolumn{1}{c}{(5)}&\multicolumn{1}{c}{(6)}&\multicolumn{1}{c}{(7)}&\multicolumn{1}{c}{(8)}&\multicolumn{1}{c}{(9)}\\
            &\multicolumn{1}{c}{C2}&\multicolumn{1}{c}{C1+C2}&\multicolumn{1}{c}{C1-C3}&\multicolumn{1}{c}{C2}&\multicolumn{1}{c}{C1+C2}&\multicolumn{1}{c}{C1-C3}&\multicolumn{1}{c}{C2}&\multicolumn{1}{c}{C1+C2}&\multicolumn{1}{c}{C1-C3}\\
\midrule
 \multicolumn{10}{l}{\emph{Panel A. 2 Month bandwidth}} \\ Abs. numbers        &       15.00         &      -37.33         &      -52.93         &       46.78         &       5.056         &      -13.35         &      -31.78         &      -42.39         &      -39.57         \\
                    &     (24.38)         &     (66.02)         &     (82.88)         &     (33.06)         &     (61.69)         &     (78.55)         &     (31.46)         &     (29.95)         &     (30.20)         \\
 Ratio population    &      -6.298\sym{***}&      -2.745         &      -1.367         &      -8.808         &      -4.098         &      -5.599         &      -3.965         &      -1.683         &       0.750         \\
                    &     (0.946)         &     (3.753)         &     (3.366)         &     (4.944)         &     (6.896)         &     (8.558)         &     (2.284)         &     (2.417)         &     (2.409)         \\
 Ratio fertility     &       0.612         &       1.403         &       1.456         &       2.430         &       3.480\sym{*}  &       3.160\sym{*}  &      -0.865         &      -0.256         &      0.0606         \\
                    &     (0.386)         &     (1.123)         &     (1.084)         &     (1.426)         &     (1.923)         &     (1.679)         &     (0.645)         &     (0.902)         &     (1.092)         \\
 \midrule\multicolumn{10}{l}{\emph{Panel B. 4 Month bandwidth}} \\ Abs. numbers        &       33.83         &       10.69         &       2.963         &       80.81\sym{*}  &       54.24         &       48.49         &      -46.97\sym{**} &      -43.54\sym{*}  &      -45.53\sym{**} \\
                    &     (51.74)         &     (70.26)         &     (88.04)         &     (42.63)         &     (56.49)         &     (75.91)         &     (21.52)         &     (24.08)         &     (22.15)         \\
 Ratio population    &      -5.759\sym{***}&      -4.720         &      -2.837         &       2.505         &       2.399         &       1.697         &      -9.239\sym{**} &      -8.033\sym{**} &      -5.421         \\
                    &     (1.769)         &     (2.783)         &     (2.406)         &     (6.887)         &     (6.151)         &     (6.376)         &     (3.766)         &     (3.647)         &     (3.757)         \\
 Ratio fertility     &      -1.882\sym{*}  &      -0.560         &      -0.200         &      -0.271         &       1.490         &       1.862         &      -3.065\sym{***}&      -2.022\sym{*}  &      -1.747         \\
                    &     (0.949)         &     (1.203)         &     (1.262)         &     (1.119)         &     (1.499)         &     (1.449)         &     (1.019)         &     (1.080)         &     (1.283)         \\
 \midrule\multicolumn{10}{l}{\emph{Panel C. 6 Month bandwidth}} \\ Abs. numbers        &       158.3\sym{**} &       99.40         &       72.57         &       125.6\sym{***}&       78.04\sym{*}  &       61.17         &       32.72         &       21.36         &       11.40         \\
                    &     (62.82)         &     (60.80)         &     (73.93)         &     (41.30)         &     (43.84)         &     (60.70)         &     (31.35)         &     (28.42)         &     (25.96)         \\
 Ratio population    &      -4.517\sym{**} &      -3.982\sym{*}  &      -2.622         &      -3.872         &      -4.561         &      -4.058         &      -3.830         &      -2.902         &      -1.516         \\
                    &     (2.105)         &     (2.333)         &     (2.093)         &     (5.136)         &     (4.858)         &     (4.956)         &     (3.561)         &     (3.228)         &     (3.189)         \\
 Ratio fertility     &      -1.608\sym{**} &      -0.525         &      0.0800         &      -1.891\sym{*}  &      -0.554         &       0.518         &      -1.371         &      -0.482         &      -0.251         \\
                    &     (0.655)         &     (0.932)         &     (1.037)         &     (0.970)         &     (1.398)         &     (1.333)         &     (0.890)         &     (0.983)         &     (1.154)         \\
 \midrule\multicolumn{10}{l}{\emph{Panel D. Donut specification}} \\ Abs. numbers        &       196.3\sym{**} &       119.3\sym{*}  &       89.87         &       156.8\sym{***}&       92.78\sym{*}  &       76.51         &       39.47         &       26.49         &       13.36         \\
                    &     (68.81)         &     (68.94)         &     (85.33)         &     (39.36)         &     (45.94)         &     (68.05)         &     (37.56)         &     (33.97)         &     (30.95)         \\
 Ratio population    &      -3.830         &      -4.746\sym{*}  &      -3.657\sym{*}  &      -4.534         &      -6.395         &      -6.879         &      -2.895         &      -3.187         &      -1.696         \\
                    &     (2.451)         &     (2.394)         &     (2.116)         &     (6.106)         &     (5.386)         &     (5.445)         &     (4.260)         &     (3.787)         &     (3.762)         \\
 Ratio fertility     &      -1.963\sym{**} &      -1.084         &      -0.182         &      -2.162\sym{*}  &      -1.151         &       0.344         &      -1.757\sym{*}  &      -0.914         &      -0.542         \\
                    &     (0.744)         &     (1.043)         &     (1.199)         &     (1.159)         &     (1.615)         &     (1.554)         &     (0.989)         &     (1.095)         &     (1.327)         \\
 
\bottomrule \end{tabular} } \begin{tablenotes} \item \scriptsize \emph{Notes:} Clustered standard errors in parentheses. All regressions contain Birthmonth FE. Ratios indicate cases per thousand; either approximated population or original number of births. \end{tablenotes} \end{threeparttable} \end{table} 

%=========================================
\section{Hauptdiagnosekapitel}
 \begin{table}[H] \begin{threeparttable} \centering \caption{Robustness with respect to the inclusion of \texttt{fixed effects} and \texttt{covariates}} {\def\sym#1{\ifmmode^{#1}\else\(^{#1}\)\fi} \begin{tabular}{l*{7}{c}} \toprule & \multicolumn{6}{c}{Dependent variable: \textbf{Certain infectious and parasitic diseases}} \\ \cmidrule(lr){2-7}
            &\multicolumn{4}{c}{Average Causal Effects}         &\multicolumn{2}{c}{Heterogeneous Causal Effects}\\\cmidrule(lr){2-5}\cmidrule(lr){6-7}
            &\multicolumn{1}{c}{(1)}&\multicolumn{1}{c}{(2)}&\multicolumn{1}{c}{(3)}&\multicolumn{1}{c}{(4)}&\multicolumn{1}{c}{(5)}&\multicolumn{1}{c}{(6)}\\
            &\multicolumn{1}{c}{}&\multicolumn{1}{c}{}&\multicolumn{1}{c}{}&\multicolumn{1}{c}{}&\multicolumn{1}{c}{Women}&\multicolumn{1}{c}{Men}\\
\midrule
 \multicolumn{7}{l}{\emph{Panel A. 2 Month bandwidth}} \\ Abs. numbers        &      -7.575         &      -7.575\sym{***}&      -7.575\sym{***}&      -7.575\sym{***}&      -6.075\sym{***}&      -1.500         \\
                    &     (10.68)         &     (0.572)         &     (0.611)         &     (0.614)         &     (1.105)         &     (1.055)         \\
 Ratio fertility     &      -0.100         &      -0.100\sym{***}&      -0.100\sym{***}&      -0.100\sym{***}&      -0.162\sym{***}&     -0.0406\sym{*}  \\
                    &     (0.127)         &   (0.00598)         &   (0.00638)         &   (0.00641)         &    (0.0290)         &    (0.0183)         \\
 Ratio fertility     &      -0.136         &      -0.136\sym{***}&      -0.136\sym{**} &      -0.136\sym{**} &      -0.255\sym{***}&     -0.0225         \\
                    &     (0.139)         &    (0.0380)         &    (0.0405)         &    (0.0409)         &    (0.0675)         &    (0.0374)         \\
 \midrule\multicolumn{7}{l}{\emph{Panel B. 4 Month bandwidth}} \\ Abs. numbers        &       3.472         &       3.472         &       3.472         &       3.472         &      -0.500         &       3.972\sym{**} \\
                    &     (7.050)         &     (4.189)         &     (4.320)         &     (4.337)         &     (2.828)         &     (1.671)         \\
 Ratio fertility     &     -0.0148         &     -0.0148         &     -0.0148         &     -0.0148         &     -0.0142         &     -0.0134         \\
                    &    (0.0952)         &    (0.0273)         &    (0.0282)         &    (0.0283)         &    (0.0509)         &    (0.0378)         \\
 Ratio fertility     &     0.00549         &     0.00549         &     0.00549         &     0.00549         &     -0.0551         &      0.0636         \\
                    &     (0.106)         &    (0.0450)         &    (0.0464)         &    (0.0466)         &    (0.0676)         &    (0.0431)         \\
 \midrule\multicolumn{7}{l}{\emph{Panel C. 6 Month bandwidth}} \\ Abs. numbers        &       4.537         &       4.537         &       4.537         &       4.710         &       2.685         &       1.852         \\
                    &     (5.329)         &     (3.000)         &     (3.061)         &     (3.082)         &     (2.279)         &     (1.560)         \\
 Ratio population    &     -0.0687         &     -0.0687         &     -0.0687         &     -0.0642         &     -0.0365         &     -0.0987         \\
                    &     (0.116)         &    (0.0603)         &    (0.0615)         &    (0.0629)         &    (0.0998)         &    (0.0843)         \\
 Ratio population    &     -0.0242         &     -0.0242         &     -0.0242         &     -0.0241         &     -0.0443         &    -0.00399         \\
                    &    (0.0780)         &    (0.0283)         &    (0.0289)         &    (0.0284)         &    (0.0423)         &    (0.0502)         \\
 \midrule\multicolumn{7}{l}{\emph{Panel D. Donut specification}} \\ Abs. numbers        &       8.311         &       8.311\sym{***}&       8.311\sym{**} &       8.372\sym{**} &       5.133\sym{**} &       3.178\sym{*}  \\
                    &     (5.879)         &     (2.862)         &     (2.933)         &     (3.028)         &     (2.345)         &     (1.713)         \\
 Ratio fertility     &     0.00868         &     0.00868         &     0.00868         &      0.0115         &      0.0364         &     -0.0173         \\
                    &    (0.0691)         &    (0.0233)         &    (0.0238)         &    (0.0243)         &    (0.0394)         &    (0.0303)         \\
 Ratio fertility     &      0.0229         &      0.0229         &      0.0229         &      0.0228         &      0.0532         &    -0.00576         \\
                    &    (0.0862)         &    (0.0273)         &    (0.0280)         &    (0.0290)         &    (0.0473)         &    (0.0526)         \\
 \midrule Birthmonth FE       &                     &  \checkmark         &  \checkmark         &  \checkmark         &  \checkmark         &  \checkmark         \\
Year FE             &                     &                     &  \checkmark         &  \checkmark         &                     &                     \\
Covariates          &                     &                     &                     &  \checkmark         &                     &                     \\
 
\bottomrule \end{tabular} } \begin{tablenotes} \item \scriptsize \emph{Notes:} Clustered standard errors in parentheses. Personal covariates contain age and age squared. Ratios indicate cases per thousand; either approximated population or original number of births. \end{tablenotes} \end{threeparttable} \end{table} 

 \begin{table}[H] \begin{threeparttable} \centering \caption{Robustness with respect to the choice of \texttt{control group}} {\def\sym#1{\ifmmode^{#1}\else\(^{#1}\)\fi} \begin{tabular}{l*{10}{c}} \toprule & \multicolumn{9}{c}{Dependent variable: \textbf{Certain infectious and parasitic diseases}} \\ \cmidrule(lr){2-10}
            &\multicolumn{3}{c}{Average Causal Effects}&\multicolumn{3}{c}{Women}             &\multicolumn{3}{c}{Men}               \\\cmidrule(lr){2-4}\cmidrule(lr){5-7}\cmidrule(lr){8-10}
            &\multicolumn{1}{c}{(1)}&\multicolumn{1}{c}{(2)}&\multicolumn{1}{c}{(3)}&\multicolumn{1}{c}{(4)}&\multicolumn{1}{c}{(5)}&\multicolumn{1}{c}{(6)}&\multicolumn{1}{c}{(7)}&\multicolumn{1}{c}{(8)}&\multicolumn{1}{c}{(9)}\\
            &\multicolumn{1}{c}{C2}&\multicolumn{1}{c}{C1+C2}&\multicolumn{1}{c}{C1-C3}&\multicolumn{1}{c}{C2}&\multicolumn{1}{c}{C1+C2}&\multicolumn{1}{c}{C1-C3}&\multicolumn{1}{c}{C2}&\multicolumn{1}{c}{C1+C2}&\multicolumn{1}{c}{C1-C3}\\
\midrule
 \multicolumn{10}{l}{\emph{Panel A. 2 Month bandwidth}} \\ Abs. numbers        &         -10\sym{**} &      -7.778\sym{**} &      -5.130         &      -9.056\sym{***}&      -4.000         &      -2.185         &      -0.944         &      -3.778\sym{*}  &      -2.944         \\
                    &     (3.302)         &     (3.240)         &     (5.337)         &     (2.425)         &     (3.491)         &     (4.425)         &     (1.559)         &     (2.101)         &     (1.804)         \\
 Ratio population    &      -0.280\sym{***}&      -0.160         &     -0.0860         &      -0.444\sym{***}&      -0.197         &      -0.153         &      -0.118         &      -0.126         &     -0.0261         \\
                    &    (0.0422)         &    (0.0976)         &    (0.0811)         &    (0.0595)         &     (0.157)         &     (0.145)         &     (0.114)         &     (0.104)         &    (0.0880)         \\
 Ratio fertility     &      -0.136\sym{***}&     -0.0733         &     -0.0293         &      -0.255\sym{***}&     -0.0716         &     -0.0154         &     -0.0225         &     -0.0746         &     -0.0423         \\
                    &    (0.0380)         &    (0.0516)         &    (0.0547)         &    (0.0675)         &     (0.113)         &     (0.107)         &    (0.0374)         &    (0.0563)         &    (0.0424)         \\
 \midrule\multicolumn{10}{l}{\emph{Panel B. 4 Month bandwidth}} \\ Abs. numbers        &       3.472         &       3.000         &       4.056         &      -0.500         &       0.944         &       1.361         &       3.972\sym{**} &       2.056         &       2.694         \\
                    &     (4.189)         &     (3.653)         &     (4.668)         &     (2.828)         &     (2.477)         &     (3.054)         &     (1.671)         &     (1.964)         &     (2.232)         \\
 Ratio population    &     -0.0664         &     -0.0477         &     0.00972         &     -0.0233         &      0.0356         &      0.0439         &      -0.109         &      -0.135         &     -0.0328         \\
                    &    (0.0712)         &    (0.0708)         &    (0.0640)         &     (0.147)         &     (0.127)         &     (0.122)         &    (0.0901)         &    (0.0841)         &    (0.0797)         \\
 Ratio fertility     &     0.00549         &      0.0295         &      0.0548         &     -0.0551         &      0.0220         &      0.0439         &      0.0636         &      0.0376         &      0.0662         \\
                    &    (0.0450)         &    (0.0412)         &    (0.0374)         &    (0.0676)         &    (0.0654)         &    (0.0612)         &    (0.0431)         &    (0.0480)         &    (0.0451)         \\
 \midrule\multicolumn{10}{l}{\emph{Panel C. 6 Month bandwidth}} \\ Abs. numbers        &       4.537         &       2.787         &       3.173         &       2.685         &       2.056         &       2.444         &       1.852         &       0.731         &       0.728         \\
                    &     (3.000)         &     (2.626)         &     (3.456)         &     (2.279)         &     (1.912)         &     (2.332)         &     (1.560)         &     (1.520)         &     (1.841)         \\
 Ratio population    &     -0.0687         &     -0.0703         &     -0.0264         &     -0.0365         &     -0.0465         &     -0.0144         &     -0.0987         &     -0.0927         &     -0.0381         \\
                    &    (0.0603)         &    (0.0561)         &    (0.0512)         &    (0.0998)         &    (0.0954)         &    (0.0905)         &    (0.0843)         &    (0.0716)         &    (0.0758)         \\
 Ratio fertility     &     -0.0137         &    -0.00337         &      0.0231         &    -0.00254         &      0.0131         &      0.0500         &     -0.0243         &     -0.0188         &    -0.00181         \\
                    &    (0.0310)         &    (0.0322)         &    (0.0306)         &    (0.0494)         &    (0.0511)         &    (0.0475)         &    (0.0456)         &    (0.0415)         &    (0.0447)         \\
 \midrule\multicolumn{10}{l}{\emph{Panel D. Donut specification}} \\ Abs. numbers        &       8.311\sym{***}&       4.911\sym{*}  &       5.015         &       5.133\sym{**} &       3.378         &       3.637         &       3.178\sym{*}  &       1.533         &       1.378         \\
                    &     (2.862)         &     (2.838)         &     (3.726)         &     (2.345)         &     (2.100)         &     (2.488)         &     (1.713)         &     (1.764)         &     (2.140)         \\
 Ratio population    &    -0.00941         &     -0.0596         &     -0.0253         &      0.0188         &     -0.0384         &     -0.0258         &     -0.0386         &     -0.0824         &     -0.0277         \\
                    &    (0.0640)         &    (0.0584)         &    (0.0553)         &     (0.116)         &     (0.108)         &     (0.105)         &    (0.0954)         &    (0.0809)         &    (0.0880)         \\
 Ratio fertility     &      0.0229         &      0.0105         &      0.0391         &      0.0532         &      0.0369         &      0.0759         &    -0.00576         &     -0.0143         &     0.00505         \\
                    &    (0.0273)         &    (0.0351)         &    (0.0330)         &    (0.0473)         &    (0.0570)         &    (0.0506)         &    (0.0526)         &    (0.0482)         &    (0.0532)         \\
 
\bottomrule \end{tabular} } \begin{tablenotes} \item \scriptsize \emph{Notes:} Clustered standard errors in parentheses. All regressions contain Birthmonth FE. Ratios indicate cases per thousand; either approximated population or original number of births. \end{tablenotes} \end{threeparttable} \end{table} 

%=========================================
 \begin{table}[H] \begin{threeparttable} \centering \caption{Robustness with respect to the inclusion of \texttt{fixed effects} and \texttt{covariates}} {\def\sym#1{\ifmmode^{#1}\else\(^{#1}\)\fi} \begin{tabular}{l*{7}{c}} \toprule & \multicolumn{6}{c}{Dependent variable: \textbf{Neoplasms}} \\ \cmidrule(lr){2-7}
            &\multicolumn{4}{c}{Average Causal Effects}         &\multicolumn{2}{c}{Heterogeneous Causal Effects}\\\cmidrule(lr){2-5}\cmidrule(lr){6-7}
            &\multicolumn{1}{c}{(1)}&\multicolumn{1}{c}{(2)}&\multicolumn{1}{c}{(3)}&\multicolumn{1}{c}{(4)}&\multicolumn{1}{c}{(5)}&\multicolumn{1}{c}{(6)}\\
            &\multicolumn{1}{c}{}&\multicolumn{1}{c}{}&\multicolumn{1}{c}{}&\multicolumn{1}{c}{}&\multicolumn{1}{c}{Women}&\multicolumn{1}{c}{Men}\\
\midrule
 \multicolumn{7}{l}{\emph{Panel A. 2 Month bandwidth}} \\ Abs. numbers        &       7.611         &       7.611\sym{***}&       7.611\sym{***}&       7.611\sym{***}&       3.167         &       4.444\sym{**} \\
                    &     (14.61)         &     (0.829)         &     (0.885)         &     (0.892)         &     (2.114)         &     (1.704)         \\
 Ratio fertility     &      0.0837         &      0.0837         &      0.0837         &      0.0837         &      -0.161\sym{***}&       0.320\sym{**} \\
                    &    (0.0810)         &    (0.0754)         &    (0.0805)         &    (0.0808)         &    (0.0371)         &     (0.126)         \\
 Ratio population    &      0.0914\sym{*}  &      0.0914\sym{**} &      0.0914\sym{**} &      0.0914\sym{**} &      0.0582\sym{**} &       0.130\sym{*}  \\
                    &    (0.0453)         &    (0.0331)         &    (0.0353)         &    (0.0355)         &    (0.0241)         &    (0.0645)         \\
 \midrule\multicolumn{7}{l}{\emph{Panel B. 4 Month bandwidth}} \\ Abs. numbers        &       4.806         &       4.806         &       4.806         &       4.806         &       7.167\sym{**} &      -2.361         \\
                    &     (8.953)         &     (3.841)         &     (3.961)         &     (3.977)         &     (2.875)         &     (3.155)         \\
 Ratio population    &      -0.128         &      -0.128         &      -0.128         &      -0.128         &       0.153         &      -0.343\sym{*}  \\
                    &     (0.148)         &    (0.0773)         &    (0.0797)         &    (0.0800)         &     (0.241)         &     (0.165)         \\
 Ratio population    &     -0.0111         &     -0.0111         &     -0.0111         &     -0.0111         &       0.134\sym{***}&      -0.150         \\
                    &     (0.120)         &    (0.0623)         &    (0.0643)         &    (0.0644)         &    (0.0359)         &     (0.112)         \\
 \midrule\multicolumn{7}{l}{\emph{Panel C. 6 Month bandwidth}} \\ Abs. numbers        &       11.53         &       11.53\sym{**} &       11.53\sym{**} &       11.53\sym{**} &       5.942\sym{*}  &       5.583\sym{*}  \\
                    &     (9.034)         &     (5.147)         &     (5.256)         &     (5.264)         &     (3.399)         &     (2.883)         \\
 Ratio population    &     -0.0676         &     -0.0676         &     -0.0676         &     -0.0810         &      0.0867         &      -0.199         \\
                    &     (0.128)         &    (0.0837)         &    (0.0854)         &    (0.0899)         &     (0.162)         &     (0.153)         \\
 Ratio population    &      0.0997         &      0.0997\sym{*}  &      0.0997\sym{*}  &      0.0906         &       0.218\sym{***}&     -0.0210         \\
                    &    (0.0991)         &    (0.0553)         &    (0.0565)         &    (0.0567)         &    (0.0357)         &    (0.0908)         \\
 \midrule\multicolumn{7}{l}{\emph{Panel D. Donut specification}} \\ Abs. numbers        &       9.778         &       9.778\sym{*}  &       9.778\sym{*}  &       9.007         &       11.76\sym{***}&      -1.978         \\
                    &     (8.426)         &     (4.950)         &     (5.072)         &     (5.320)         &     (2.641)         &     (3.376)         \\
 Ratio population    &     -0.0536         &     -0.0536         &     -0.0536         &     -0.0659         &       0.117         &      -0.210         \\
                    &     (0.132)         &    (0.0901)         &    (0.0923)         &    (0.0990)         &     (0.192)         &     (0.177)         \\
 Ratio population    &       0.117         &       0.117\sym{*}  &       0.117\sym{*}  &       0.109         &       0.257\sym{***}&     -0.0240         \\
                    &     (0.109)         &    (0.0656)         &    (0.0672)         &    (0.0678)         &    (0.0366)         &     (0.109)         \\
 \midrule Birthmonth FE       &                     &  \checkmark         &  \checkmark         &  \checkmark         &  \checkmark         &  \checkmark         \\
Year FE             &                     &                     &  \checkmark         &  \checkmark         &                     &                     \\
Covariates          &                     &                     &                     &  \checkmark         &                     &                     \\
 
\bottomrule \end{tabular} } \begin{tablenotes} \item \scriptsize \emph{Notes:} Clustered standard errors in parentheses. Personal covariates contain age and age squared. Ratios indicate cases per thousand; either approximated population or original number of births. \end{tablenotes} \end{threeparttable} \end{table} 

 \begin{table}[H] \begin{threeparttable} \centering \caption{Robustness with respect to the choice of \texttt{control group}} {\def\sym#1{\ifmmode^{#1}\else\(^{#1}\)\fi} \begin{tabular}{l*{10}{c}} \toprule & \multicolumn{9}{c}{Dependent variable: \textbf{Neoplasms}} \\ \cmidrule(lr){2-10}
            &\multicolumn{3}{c}{Average Causal Effects}&\multicolumn{3}{c}{Women}             &\multicolumn{3}{c}{Men}               \\\cmidrule(lr){2-4}\cmidrule(lr){5-7}\cmidrule(lr){8-10}
            &\multicolumn{1}{c}{(1)}&\multicolumn{1}{c}{(2)}&\multicolumn{1}{c}{(3)}&\multicolumn{1}{c}{(4)}&\multicolumn{1}{c}{(5)}&\multicolumn{1}{c}{(6)}&\multicolumn{1}{c}{(7)}&\multicolumn{1}{c}{(8)}&\multicolumn{1}{c}{(9)}\\
            &\multicolumn{1}{c}{C2}&\multicolumn{1}{c}{C1+C2}&\multicolumn{1}{c}{C1-C3}&\multicolumn{1}{c}{C2}&\multicolumn{1}{c}{C1+C2}&\multicolumn{1}{c}{C1-C3}&\multicolumn{1}{c}{C2}&\multicolumn{1}{c}{C1+C2}&\multicolumn{1}{c}{C1-C3}\\
\midrule
 \multicolumn{10}{l}{\emph{Panel A. 2 Month bandwidth}} \\ Abs. numbers        &       7.611\sym{***}&       5.694         &       3.204         &       3.167         &       6.194         &       1.630         &       4.444\sym{**} &      -0.500         &       1.574         \\
                    &     (0.829)         &     (6.253)         &     (11.72)         &     (2.114)         &     (7.633)         &     (10.52)         &     (1.704)         &     (2.826)         &     (3.361)         \\
 Ratio population    &      -0.112\sym{***}&     0.00696         &      0.0106         &      -0.295\sym{**} &      0.0291         &      -0.118         &      0.0291         &     -0.0373         &      0.0952         \\
                    &    (0.0185)         &     (0.132)         &     (0.258)         &    (0.0898)         &     (0.282)         &     (0.469)         &    (0.0718)         &     (0.117)         &     (0.133)         \\
 Ratio fertility     &       0.118\sym{***}&       0.137         &       0.106         &       0.115\sym{*}  &       0.272         &       0.141         &       0.127\sym{*}  &      0.0187         &      0.0806         \\
                    &    (0.0119)         &    (0.0869)         &     (0.202)         &    (0.0519)         &     (0.220)         &     (0.355)         &    (0.0657)         &    (0.0884)         &     (0.117)         \\
 \midrule\multicolumn{10}{l}{\emph{Panel B. 4 Month bandwidth}} \\ Abs. numbers        &       4.806         &       1.389         &       1.222         &       7.167\sym{**} &       8.042         &       6.806         &      -2.361         &      -6.653\sym{*}  &      -5.583         \\
                    &     (3.841)         &     (5.835)         &     (8.758)         &     (2.875)         &     (5.230)         &     (7.380)         &     (3.155)         &     (3.639)         &     (3.952)         \\
 Ratio population    &      -0.128         &      -0.130         &     -0.0823         &       0.153         &       0.251         &       0.196         &      -0.343\sym{*}  &      -0.440\sym{**} &      -0.323         \\
                    &    (0.0773)         &     (0.120)         &     (0.186)         &     (0.241)         &     (0.246)         &     (0.340)         &     (0.165)         &     (0.194)         &     (0.218)         \\
 Ratio fertility     &      0.0104         &      0.0167         &      0.0155         &       0.159\sym{***}&       0.272\sym{*}  &       0.230         &      -0.124         &      -0.213         &      -0.178         \\
                    &    (0.0501)         &     (0.100)         &     (0.173)         &    (0.0425)         &     (0.155)         &     (0.267)         &     (0.108)         &     (0.130)         &     (0.151)         \\
 \midrule\multicolumn{10}{l}{\emph{Panel C. 6 Month bandwidth}} \\ Abs. numbers        &       9.074\sym{*}  &       4.685         &       2.827         &       9.926\sym{***}&       8.944\sym{*}  &       7.420         &      -0.852         &      -4.259         &      -4.593         \\
                    &     (4.646)         &     (6.430)         &     (8.259)         &     (2.475)         &     (4.831)         &     (6.712)         &     (3.178)         &     (3.237)         &     (3.129)         \\
 Ratio population    &     -0.0676         &     -0.0835         &     -0.0736         &      0.0867         &       0.106         &      0.0715         &      -0.199         &      -0.263         &      -0.221         \\
                    &    (0.0837)         &     (0.117)         &     (0.169)         &     (0.162)         &     (0.204)         &     (0.282)         &     (0.153)         &     (0.158)         &     (0.168)         \\
 Ratio fertility     &      0.0333         &      0.0374         &      0.0203         &       0.177\sym{***}&       0.244         &       0.209         &      -0.104         &      -0.158         &      -0.157         \\
                    &    (0.0480)         &    (0.0985)         &     (0.153)         &    (0.0356)         &     (0.149)         &     (0.242)         &    (0.0909)         &     (0.102)         &     (0.112)         \\
 \midrule\multicolumn{10}{l}{\emph{Panel D. Donut specification}} \\ Abs. numbers        &       9.778\sym{*}  &       3.767         &       1.533         &       11.76\sym{***}&       9.111         &       7.985         &      -1.978         &      -5.344         &      -6.452\sym{*}  \\
                    &     (4.950)         &     (7.693)         &     (9.482)         &     (2.641)         &     (5.711)         &     (7.740)         &     (3.376)         &     (3.733)         &     (3.404)         \\
 Ratio population    &     -0.0536         &      -0.133         &      -0.134         &       0.117         &      0.0563         &     0.00804         &      -0.210         &      -0.313         &      -0.288         \\
                    &    (0.0901)         &     (0.129)         &     (0.183)         &     (0.192)         &     (0.232)         &     (0.313)         &     (0.177)         &     (0.185)         &     (0.191)         \\
 Ratio fertility     &      0.0168         &    0.000182         &     -0.0142         &       0.200\sym{***}&       0.224         &       0.208         &      -0.157         &      -0.209\sym{*}  &      -0.222\sym{*}  \\
                    &    (0.0532)         &     (0.115)         &     (0.175)         &    (0.0407)         &     (0.170)         &     (0.277)         &    (0.0979)         &     (0.118)         &     (0.124)         \\
 
\bottomrule \end{tabular} } \begin{tablenotes} \item \scriptsize \emph{Notes:} Clustered standard errors in parentheses. All regressions contain Birthmonth FE. Ratios indicate cases per thousand; either approximated population or original number of births. \end{tablenotes} \end{threeparttable} \end{table} 

%=========================================
 \begin{table}[H] \begin{threeparttable} \centering \caption{Robustness with respect to the inclusion of \texttt{fixed effects} and \texttt{covariates}} {\def\sym#1{\ifmmode^{#1}\else\(^{#1}\)\fi} \begin{tabular}{l*{7}{c}} \toprule & \multicolumn{6}{c}{Dependent variable: \textbf{Mental and behavioral disorders}} \\ \cmidrule(lr){2-7}
            &\multicolumn{4}{c}{Average Causal Effects}         &\multicolumn{2}{c}{Heterogeneous Causal Effects}\\\cmidrule(lr){2-5}\cmidrule(lr){6-7}
            &\multicolumn{1}{c}{(1)}&\multicolumn{1}{c}{(2)}&\multicolumn{1}{c}{(3)}&\multicolumn{1}{c}{(4)}&\multicolumn{1}{c}{(5)}&\multicolumn{1}{c}{(6)}\\
            &\multicolumn{1}{c}{}&\multicolumn{1}{c}{}&\multicolumn{1}{c}{}&\multicolumn{1}{c}{}&\multicolumn{1}{c}{Women}&\multicolumn{1}{c}{Men}\\
\midrule
 \multicolumn{7}{l}{\emph{Panel A. 2 Month bandwidth}} \\ Abs. numbers        &       15.28         &       15.28         &       15.28         &       15.28         &       13.23         &       2.050         \\
                    &     (39.76)         &     (9.196)         &     (9.822)         &     (9.858)         &     (10.55)         &     (3.535)         \\
 Ratio fertility     &       0.262         &       0.262         &       0.262         &       0.262         &       0.465         &      0.0598         \\
                    &     (0.346)         &     (0.186)         &     (0.199)         &     (0.200)         &     (0.343)         &     (0.149)         \\
 Ratio fertility     &      -0.466         &      -0.466\sym{***}&      -0.466\sym{**} &      -0.466\sym{**} &      -0.117         &      -0.808\sym{**} \\
                    &     (0.446)         &     (0.126)         &     (0.135)         &     (0.136)         &     (0.514)         &     (0.258)         \\
 \midrule\multicolumn{7}{l}{\emph{Panel B. 4 Month bandwidth}} \\ Abs. numbers        &      -53.50         &      -53.50\sym{***}&      -53.50\sym{***}&      -53.50\sym{***}&      -4.972         &      -48.53\sym{***}\\
                    &     (33.63)         &     (9.326)         &     (9.618)         &     (9.656)         &     (9.401)         &     (7.820)         \\
 Ratio fertility     &      -0.417         &      -0.417\sym{*}  &      -0.417\sym{*}  &      -0.417\sym{*}  &      0.0951         &      -0.915\sym{***}\\
                    &     (0.493)         &     (0.224)         &     (0.231)         &     (0.231)         &     (0.222)         &     (0.292)         \\
 Ratio fertility     &      -1.065\sym{*}  &      -1.065\sym{***}&      -1.065\sym{***}&      -1.065\sym{***}&      -0.332         &      -1.776\sym{***}\\
                    &     (0.536)         &     (0.179)         &     (0.184)         &     (0.185)         &     (0.265)         &     (0.319)         \\
 \midrule\multicolumn{7}{l}{\emph{Panel C. 6 Month bandwidth}} \\ Abs. numbers        &       10.22         &       10.22         &       10.22         &       11.67         &       20.18\sym{***}&      -9.967         \\
                    &     (27.42)         &     (10.47)         &     (10.69)         &     (10.97)         &     (5.566)         &     (6.428)         \\
 Ratio fertility     &      -0.219         &      -0.219         &      -0.219         &      -0.204         &       0.292         &      -0.703\sym{***}\\
                    &     (0.352)         &     (0.170)         &     (0.173)         &     (0.175)         &     (0.172)         &     (0.211)         \\
 Ratio population    &      -0.609         &      -0.609\sym{**} &      -0.609\sym{**} &      -0.609\sym{**} &       0.147         &      -1.364\sym{***}\\
                    &     (0.445)         &     (0.220)         &     (0.225)         &     (0.222)         &     (0.237)         &     (0.298)         \\
 \midrule\multicolumn{7}{l}{\emph{Panel D. Donut specification}} \\ Abs. numbers        &       8.840         &       8.840         &       8.840         &       10.61         &       22.97\sym{***}&      -14.13\sym{*}  \\
                    &     (32.69)         &     (11.62)         &     (11.91)         &     (12.33)         &     (5.369)         &     (7.281)         \\
 Ratio population    &      -0.933         &      -0.933\sym{**} &      -0.933\sym{**} &      -0.925\sym{*}  &     0.00732         &      -1.850\sym{*}  \\
                    &     (0.597)         &     (0.435)         &     (0.446)         &     (0.459)         &     (0.426)         &     (1.005)         \\
 Ratio fertility     &      -0.685         &      -0.685\sym{***}&      -0.685\sym{***}&      -0.692\sym{***}&       0.171         &      -1.494\sym{***}\\
                    &     (0.402)         &     (0.233)         &     (0.239)         &     (0.235)         &     (0.235)         &     (0.299)         \\
 \midrule Birthmonth FE       &                     &  \checkmark         &  \checkmark         &  \checkmark         &  \checkmark         &  \checkmark         \\
Year FE             &                     &                     &  \checkmark         &  \checkmark         &                     &                     \\
Covariates          &                     &                     &                     &  \checkmark         &                     &                     \\
 
\bottomrule \end{tabular} } \begin{tablenotes} \item \scriptsize \emph{Notes:} Clustered standard errors in parentheses. Personal covariates contain age and age squared. Ratios indicate cases per thousand; either approximated population or original number of births. \end{tablenotes} \end{threeparttable} \end{table} 

 \begin{table}[H] \begin{threeparttable} \centering \caption{Robustness with respect to the choice of \texttt{control group}} {\def\sym#1{\ifmmode^{#1}\else\(^{#1}\)\fi} \begin{tabular}{l*{10}{c}} \toprule & \multicolumn{9}{c}{Dependent variable: \textbf{Mental and behavioral disorders}} \\ \cmidrule(lr){2-10}
            &\multicolumn{3}{c}{Average Causal Effects}&\multicolumn{3}{c}{Women}             &\multicolumn{3}{c}{Men}               \\\cmidrule(lr){2-4}\cmidrule(lr){5-7}\cmidrule(lr){8-10}
            &\multicolumn{1}{c}{(1)}&\multicolumn{1}{c}{(2)}&\multicolumn{1}{c}{(3)}&\multicolumn{1}{c}{(4)}&\multicolumn{1}{c}{(5)}&\multicolumn{1}{c}{(6)}&\multicolumn{1}{c}{(7)}&\multicolumn{1}{c}{(8)}&\multicolumn{1}{c}{(9)}\\
            &\multicolumn{1}{c}{C2}&\multicolumn{1}{c}{C1+C2}&\multicolumn{1}{c}{C1-C3}&\multicolumn{1}{c}{C2}&\multicolumn{1}{c}{C1+C2}&\multicolumn{1}{c}{C1-C3}&\multicolumn{1}{c}{C2}&\multicolumn{1}{c}{C1+C2}&\multicolumn{1}{c}{C1-C3}\\
\midrule
 \multicolumn{10}{l}{\emph{Panel A. 2 Month bandwidth}} \\ Abs. numbers        &      -36.06\sym{***}&      -34.22\sym{**} &      -44.50\sym{**} &      -6.944         &       3.222         &      -10.22         &      -29.11\sym{**} &      -37.44\sym{**} &      -34.28\sym{**} \\
                    &     (6.516)         &     (13.75)         &     (19.79)         &     (16.38)         &     (15.20)         &     (21.13)         &     (10.14)         &     (12.58)         &     (13.81)         \\
 Ratio population    &      -1.316\sym{***}&      -0.789         &      -0.748         &      -0.943         &      -0.220         &      -0.657         &      -1.606\sym{***}&      -1.261         &      -0.648         \\
                    &     (0.192)         &     (0.607)         &     (0.482)         &     (0.560)         &     (0.738)         &     (0.784)         &     (0.446)         &     (0.721)         &     (0.658)         \\
 Ratio fertility     &      -0.466\sym{***}&      -0.263         &      -0.372         &      -0.117         &       0.324         &     -0.0379         &      -0.808\sym{**} &      -0.831\sym{*}  &      -0.695\sym{*}  \\
                    &     (0.126)         &     (0.315)         &     (0.260)         &     (0.514)         &     (0.534)         &     (0.600)         &     (0.258)         &     (0.435)         &     (0.366)         \\
 \midrule\multicolumn{10}{l}{\emph{Panel B. 4 Month bandwidth}} \\ Abs. numbers        &      -53.50\sym{***}&      -31.22\sym{*}  &      -27.82         &      -4.972         &       10.75         &       4.787         &      -48.53\sym{***}&      -41.97\sym{***}&      -32.61\sym{***}\\
                    &     (9.326)         &     (16.87)         &     (19.68)         &     (9.401)         &     (11.79)         &     (15.46)         &     (7.820)         &     (9.695)         &     (9.696)         \\
 Ratio population    &      -1.567\sym{***}&      -1.061\sym{**} &      -0.772\sym{*}  &      -0.224         &       0.334         &       0.146         &      -3.260\sym{***}&      -2.800\sym{***}&      -1.920\sym{**} \\
                    &     (0.244)         &     (0.443)         &     (0.396)         &     (0.497)         &     (0.572)         &     (0.615)         &     (0.811)         &     (0.817)         &     (0.852)         \\
 Ratio fertility     &      -1.065\sym{***}&      -0.548\sym{**} &      -0.439\sym{*}  &      -0.332         &       0.310         &       0.183         &      -1.776\sym{***}&      -1.383\sym{***}&      -1.047\sym{***}\\
                    &     (0.179)         &     (0.255)         &     (0.216)         &     (0.265)         &     (0.347)         &     (0.374)         &     (0.319)         &     (0.336)         &     (0.317)         \\
 \midrule\multicolumn{10}{l}{\emph{Panel C. 6 Month bandwidth}} \\ Abs. numbers        &      -16.11         &      -10.21         &      -9.667         &       10.19         &       13.32         &       5.049         &      -26.30\sym{**} &      -23.54\sym{**} &      -14.72         \\
                    &     (14.92)         &     (15.27)         &     (17.02)         &     (8.336)         &     (9.111)         &     (11.38)         &     (9.967)         &     (9.501)         &     (10.10)         \\
 Ratio population    &      -1.051\sym{***}&      -0.806\sym{**} &      -0.583         &      -0.207         &     -0.0653         &      -0.234         &      -1.929\sym{**} &      -1.558\sym{**} &      -0.900         \\
                    &     (0.371)         &     (0.394)         &     (0.348)         &     (0.392)         &     (0.450)         &     (0.466)         &     (0.834)         &     (0.763)         &     (0.735)         \\
 Ratio fertility     &      -0.681\sym{***}&      -0.377\sym{*}  &      -0.247         &     -0.0437         &       0.220         &      0.0888         &      -1.280\sym{***}&      -0.940\sym{***}&      -0.566\sym{*}  \\
                    &     (0.193)         &     (0.215)         &     (0.200)         &     (0.232)         &     (0.272)         &     (0.276)         &     (0.279)         &     (0.292)         &     (0.307)         \\
 \midrule\multicolumn{10}{l}{\emph{Panel D. Donut specification}} \\ Abs. numbers        &      -9.133         &      -3.533         &      -2.452         &       20.09\sym{***}&       21.23\sym{**} &       12.03         &      -29.22\sym{**} &      -24.77\sym{**} &      -14.48         \\
                    &     (17.14)         &     (16.88)         &     (18.90)         &     (6.817)         &     (8.465)         &     (10.21)         &     (11.86)         &     (11.35)         &     (12.04)         \\
 Ratio population    &      -0.933\sym{**} &      -0.826\sym{*}  &      -0.623         &     0.00732         &      0.0376         &      -0.234         &      -1.850\sym{*}  &      -1.682\sym{*}  &      -0.930         \\
                    &     (0.435)         &     (0.417)         &     (0.377)         &     (0.426)         &     (0.466)         &     (0.503)         &     (1.005)         &     (0.895)         &     (0.869)         \\
 Ratio fertility     &      -0.685\sym{***}&      -0.380         &      -0.205         &       0.171         &       0.385         &       0.249         &      -1.494\sym{***}&      -1.107\sym{***}&      -0.639\sym{*}  \\
                    &     (0.233)         &     (0.251)         &     (0.230)         &     (0.235)         &     (0.284)         &     (0.251)         &     (0.299)         &     (0.335)         &     (0.352)         \\
 
\bottomrule \end{tabular} } \begin{tablenotes} \item \scriptsize \emph{Notes:} Clustered standard errors in parentheses. All regressions contain Birthmonth FE. Ratios indicate cases per thousand; either approximated population or original number of births. \end{tablenotes} \end{threeparttable} \end{table} 

%=========================================
 \begin{table}[H] \begin{threeparttable} \centering \caption{Robustness with respect to the inclusion of \texttt{fixed effects} and \texttt{covariates}} {\def\sym#1{\ifmmode^{#1}\else\(^{#1}\)\fi} \begin{tabular}{l*{7}{c}} \toprule & \multicolumn{6}{c}{Dependent variable: \textbf{Diseases of the nervous system}} \\ \cmidrule(lr){2-7}
            &\multicolumn{4}{c}{Average Causal Effects}         &\multicolumn{2}{c}{Heterogeneous Causal Effects}\\\cmidrule(lr){2-5}\cmidrule(lr){6-7}
            &\multicolumn{1}{c}{(1)}&\multicolumn{1}{c}{(2)}&\multicolumn{1}{c}{(3)}&\multicolumn{1}{c}{(4)}&\multicolumn{1}{c}{(5)}&\multicolumn{1}{c}{(6)}\\
            &\multicolumn{1}{c}{}&\multicolumn{1}{c}{}&\multicolumn{1}{c}{}&\multicolumn{1}{c}{}&\multicolumn{1}{c}{Women}&\multicolumn{1}{c}{Men}\\
\midrule
 \multicolumn{7}{l}{\emph{Panel A. 2 Month bandwidth}} \\ Abs. numbers        &       4.375         &       4.375         &       4.375         &       4.375         &       6.750\sym{**} &      -2.375         \\
                    &     (8.963)         &     (2.538)         &     (2.711)         &     (2.721)         &     (2.640)         &     (2.883)         \\
 Ratio population    &      -0.209         &      -0.209         &      -0.209         &      -0.209         &      -0.106         &      -0.273         \\
                    &     (0.234)         &     (0.125)         &     (0.133)         &     (0.134)         &     (0.247)         &     (0.170)         \\
 Ratio fertility     &    -0.00594         &    -0.00594         &    -0.00594         &    -0.00594         &       0.151         &      -0.155         \\
                    &     (0.228)         &     (0.119)         &     (0.127)         &     (0.128)         &     (0.142)         &     (0.176)         \\
 \midrule\multicolumn{7}{l}{\emph{Panel B. 4 Month bandwidth}} \\ Abs. numbers        &       6.389         &       6.389         &       6.389         &       6.389         &       4.611         &       1.778         \\
                    &     (13.54)         &     (7.955)         &     (8.203)         &     (8.236)         &     (5.085)         &     (4.367)         \\
 Ratio population    &     -0.0807         &     -0.0807         &     -0.0807         &     -0.0807         &       0.122         &      -0.274         \\
                    &     (0.224)         &     (0.147)         &     (0.152)         &     (0.153)         &     (0.244)         &     (0.208)         \\
 Ratio population    &      0.0392         &      0.0392         &      0.0392         &      0.0392         &       0.107         &     -0.0286         \\
                    &     (0.159)         &    (0.0750)         &    (0.0774)         &    (0.0776)         &     (0.126)         &     (0.112)         \\
 \midrule\multicolumn{7}{l}{\emph{Panel C. 6 Month bandwidth}} \\ Abs. numbers        &       13.17         &       13.17\sym{**} &       13.17\sym{**} &       13.23\sym{**} &       8.778\sym{**} &       4.389         \\
                    &     (10.63)         &     (5.799)         &     (5.917)         &     (5.881)         &     (3.677)         &     (3.573)         \\
 Ratio fertility     &      0.0622         &      0.0622         &      0.0622         &      0.0616         &      0.0694         &      0.0545         \\
                    &    (0.0882)         &    (0.0379)         &    (0.0387)         &    (0.0385)         &    (0.0493)         &    (0.0693)         \\
 Ratio population    &      0.0815         &      0.0815         &      0.0815         &      0.0793         &       0.159\sym{*}  &     0.00234         \\
                    &     (0.128)         &    (0.0647)         &    (0.0661)         &    (0.0652)         &    (0.0853)         &     (0.106)         \\
 \midrule\multicolumn{7}{l}{\emph{Panel D. Donut specification}} \\ Abs. numbers        &       12.02         &       12.02\sym{*}  &       12.02\sym{*}  &       12.11\sym{*}  &       8.400\sym{**} &       3.622         \\
                    &     (12.55)         &     (6.517)         &     (6.678)         &     (6.618)         &     (3.047)         &     (4.216)         \\
 Ratio population    &     0.00191         &     0.00191         &     0.00191         &     0.00465         &      0.0726         &     -0.0752         \\
                    &     (0.204)         &     (0.139)         &     (0.143)         &     (0.146)         &     (0.163)         &     (0.226)         \\
 Ratio fertility     &      0.0537         &      0.0537         &      0.0537         &      0.0524         &       0.123\sym{*}  &     -0.0132         \\
                    &     (0.132)         &    (0.0817)         &    (0.0837)         &    (0.0810)         &    (0.0674)         &     (0.118)         \\
 \midrule Birthmonth FE       &                     &  \checkmark         &  \checkmark         &  \checkmark         &  \checkmark         &  \checkmark         \\
Year FE             &                     &                     &  \checkmark         &  \checkmark         &                     &                     \\
Covariates          &                     &                     &                     &  \checkmark         &                     &                     \\
 
\bottomrule \end{tabular} } \begin{tablenotes} \item \scriptsize \emph{Notes:} Clustered standard errors in parentheses. Personal covariates contain age and age squared. Ratios indicate cases per thousand; either approximated population or original number of births. \end{tablenotes} \end{threeparttable} \end{table} 

 \begin{table}[H] \begin{threeparttable} \centering \caption{Robustness with respect to the choice of \texttt{control group}} {\def\sym#1{\ifmmode^{#1}\else\(^{#1}\)\fi} \begin{tabular}{l*{10}{c}} \toprule & \multicolumn{9}{c}{Dependent variable: \textbf{Diseases of the nervous system}} \\ \cmidrule(lr){2-10}
            &\multicolumn{3}{c}{Average Causal Effects}&\multicolumn{3}{c}{Women}             &\multicolumn{3}{c}{Men}               \\\cmidrule(lr){2-4}\cmidrule(lr){5-7}\cmidrule(lr){8-10}
            &\multicolumn{1}{c}{(1)}&\multicolumn{1}{c}{(2)}&\multicolumn{1}{c}{(3)}&\multicolumn{1}{c}{(4)}&\multicolumn{1}{c}{(5)}&\multicolumn{1}{c}{(6)}&\multicolumn{1}{c}{(7)}&\multicolumn{1}{c}{(8)}&\multicolumn{1}{c}{(9)}\\
            &\multicolumn{1}{c}{C2}&\multicolumn{1}{c}{C1+C2}&\multicolumn{1}{c}{C1-C3}&\multicolumn{1}{c}{C2}&\multicolumn{1}{c}{C1+C2}&\multicolumn{1}{c}{C1-C3}&\multicolumn{1}{c}{C2}&\multicolumn{1}{c}{C1+C2}&\multicolumn{1}{c}{C1-C3}\\
\midrule
 \multicolumn{10}{l}{\emph{Panel A. 2 Month bandwidth}} \\ Abs. numbers        &      -1.056         &      -5.556         &      -4.574         &       4.167         &       2.806         &       3.704         &      -5.222         &      -8.361         &      -8.278         \\
                    &     (7.827)         &     (7.681)         &     (8.355)         &     (4.585)         &     (4.226)         &     (5.311)         &     (6.294)         &     (6.329)         &     (5.657)         \\
 Ratio population    &      -0.209         &      -0.152         &     -0.0950         &      -0.106         &     -0.0161         &    -0.00724         &      -0.273         &      -0.258         &      -0.164         \\
                    &     (0.125)         &     (0.182)         &     (0.211)         &     (0.247)         &     (0.268)         &     (0.319)         &     (0.170)         &     (0.191)         &     (0.190)         \\
 Ratio fertility     &    -0.00594         &     -0.0240         &    -0.00230         &       0.151         &       0.159         &       0.190         &      -0.155         &      -0.198         &      -0.185         \\
                    &     (0.119)         &     (0.127)         &     (0.147)         &     (0.142)         &     (0.146)         &     (0.174)         &     (0.176)         &     (0.179)         &     (0.177)         \\
 \midrule\multicolumn{10}{l}{\emph{Panel B. 4 Month bandwidth}} \\ Abs. numbers        &       6.389         &       3.389         &       2.500         &       4.611         &       3.625         &       3.241         &       1.778         &      -0.236         &      -0.741         \\
                    &     (7.955)         &     (7.940)         &     (7.562)         &     (5.085)         &     (4.390)         &     (4.439)         &     (4.367)         &     (5.015)         &     (4.615)         \\
 Ratio population    &     -0.0807         &     -0.0833         &     -0.0441         &       0.122         &       0.121         &      0.0986         &      -0.274         &      -0.294         &      -0.199         \\
                    &     (0.147)         &     (0.160)         &     (0.161)         &     (0.244)         &     (0.218)         &     (0.221)         &     (0.208)         &     (0.211)         &     (0.215)         \\
 Ratio fertility     &      0.0266         &      0.0346         &      0.0331         &      0.0867         &       0.113         &       0.113         &     -0.0301         &     -0.0396         &     -0.0420         \\
                    &    (0.0891)         &     (0.101)         &     (0.116)         &     (0.130)         &     (0.121)         &     (0.132)         &     (0.109)         &     (0.130)         &     (0.141)         \\
 \midrule\multicolumn{10}{l}{\emph{Panel C. 6 Month bandwidth}} \\ Abs. numbers        &       13.17\sym{**} &       9.352         &       6.395         &       8.778\sym{**} &       7.148\sym{**} &       5.753\sym{*}  &       4.389         &       2.204         &       0.642         \\
                    &     (5.799)         &     (5.836)         &     (5.619)         &     (3.677)         &     (3.362)         &     (3.343)         &     (3.573)         &     (3.961)         &     (3.842)         \\
 Ratio population    &      0.0188         &     0.00383         &     0.00330         &       0.113         &      0.0840         &      0.0608         &     -0.0698         &     -0.0798         &     -0.0596         \\
                    &     (0.120)         &     (0.125)         &     (0.129)         &     (0.162)         &     (0.147)         &     (0.150)         &     (0.189)         &     (0.186)         &     (0.190)         \\
 Ratio fertility     &      0.0937         &      0.0979         &      0.0784         &       0.159\sym{*}  &       0.170\sym{*}  &       0.158\sym{*}  &      0.0308         &      0.0284         &     0.00179         \\
                    &    (0.0725)         &    (0.0782)         &    (0.0911)         &    (0.0886)         &    (0.0843)         &    (0.0907)         &     (0.104)         &     (0.116)         &     (0.129)         \\
 \midrule\multicolumn{10}{l}{\emph{Panel D. Donut specification}} \\ Abs. numbers        &       12.02\sym{*}  &       9.444         &       5.548         &       8.400\sym{**} &       6.822\sym{**} &       4.748\sym{*}  &       3.622         &       2.622         &       0.800         \\
                    &     (6.517)         &     (6.324)         &     (5.934)         &     (3.047)         &     (2.931)         &     (2.613)         &     (4.216)         &     (4.454)         &     (4.472)         \\
 Ratio population    &     0.00191         &     -0.0266         &     -0.0474         &      0.0726         &      0.0299         &     -0.0332         &     -0.0752         &     -0.0985         &     -0.0761         \\
                    &     (0.139)         &     (0.130)         &     (0.133)         &     (0.163)         &     (0.130)         &     (0.132)         &     (0.226)         &     (0.216)         &     (0.222)         \\
 Ratio fertility     &      0.0537         &      0.0764         &      0.0525         &       0.123\sym{*}  &       0.139\sym{*}  &       0.116\sym{*}  &     -0.0132         &      0.0158         &    -0.00865         \\
                    &    (0.0817)         &    (0.0859)         &    (0.0998)         &    (0.0674)         &    (0.0697)         &    (0.0680)         &     (0.118)         &     (0.130)         &     (0.149)         \\
 
\bottomrule \end{tabular} } \begin{tablenotes} \item \scriptsize \emph{Notes:} Clustered standard errors in parentheses. All regressions contain Birthmonth FE. Ratios indicate cases per thousand; either approximated population or original number of births. \end{tablenotes} \end{threeparttable} \end{table} 

%=========================================
 \begin{table}[H] \begin{threeparttable} \centering \caption{Robustness with respect to the inclusion of \texttt{fixed effects} and \texttt{covariates}} {\def\sym#1{\ifmmode^{#1}\else\(^{#1}\)\fi} \begin{tabular}{l*{7}{c}} \toprule & \multicolumn{6}{c}{Dependent variable: \textbf{Diseases of the eye and ear}} \\ \cmidrule(lr){2-7}
            &\multicolumn{4}{c}{Average Causal Effects}         &\multicolumn{2}{c}{Heterogeneous Causal Effects}\\\cmidrule(lr){2-5}\cmidrule(lr){6-7}
            &\multicolumn{1}{c}{(1)}&\multicolumn{1}{c}{(2)}&\multicolumn{1}{c}{(3)}&\multicolumn{1}{c}{(4)}&\multicolumn{1}{c}{(5)}&\multicolumn{1}{c}{(6)}\\
            &\multicolumn{1}{c}{}&\multicolumn{1}{c}{}&\multicolumn{1}{c}{}&\multicolumn{1}{c}{}&\multicolumn{1}{c}{Women}&\multicolumn{1}{c}{Men}\\
\midrule
 \multicolumn{7}{l}{\emph{Panel A. 2 Month bandwidth}} \\ Abs. numbers        &      -14.11\sym{*}  &      -14.11\sym{***}&      -14.11\sym{***}&      -14.11\sym{***}&      -4.500\sym{***}&      -9.611\sym{***}\\
                    &     (7.216)         &     (2.409)         &     (2.570)         &     (2.593)         &     (0.441)         &     (1.973)         \\
 Ratio population    &      -0.297\sym{***}&      -0.297\sym{***}&      -0.297\sym{***}&      -0.297\sym{***}&      -0.245\sym{***}&      -0.352\sym{***}\\
                    &    (0.0781)         &    (0.0340)         &    (0.0363)         &    (0.0366)         &    (0.0343)         &    (0.0989)         \\
 Ratio population    &      -0.138\sym{*}  &      -0.138\sym{***}&      -0.138\sym{***}&      -0.138\sym{***}&     -0.0755\sym{***}&      -0.200\sym{***}\\
                    &    (0.0608)         &    (0.0343)         &    (0.0366)         &    (0.0369)         &    (0.0147)         &    (0.0545)         \\
 \midrule\multicolumn{7}{l}{\emph{Panel B. 4 Month bandwidth}} \\ Abs. numbers        &      -9.194\sym{*}  &      -9.194\sym{**} &      -9.194\sym{**} &      -9.194\sym{**} &      -5.944\sym{***}&      -3.250         \\
                    &     (5.157)         &     (3.382)         &     (3.488)         &     (3.501)         &     (1.941)         &     (2.793)         \\
 Ratio population    &      -0.213\sym{***}&      -0.213\sym{***}&      -0.213\sym{***}&      -0.213\sym{***}&      -0.196\sym{**} &      -0.232\sym{*}  \\
                    &    (0.0619)         &    (0.0454)         &    (0.0468)         &    (0.0470)         &    (0.0853)         &     (0.117)         \\
 Ratio population    &      -0.123\sym{*}  &      -0.123\sym{**} &      -0.123\sym{**} &      -0.123\sym{**} &      -0.143\sym{**} &      -0.102         \\
                    &    (0.0589)         &    (0.0483)         &    (0.0499)         &    (0.0500)         &    (0.0570)         &    (0.0743)         \\
 \midrule\multicolumn{7}{l}{\emph{Panel C. 6 Month bandwidth}} \\ Abs. numbers        &      -5.944         &      -5.944\sym{**} &      -5.944\sym{**} &      -5.853\sym{**} &      -3.815\sym{**} &      -2.130         \\
                    &     (4.285)         &     (2.459)         &     (2.509)         &     (2.497)         &     (1.471)         &     (1.915)         \\
 Ratio fertility     &     -0.0736\sym{**} &     -0.0736\sym{***}&     -0.0736\sym{***}&     -0.0748\sym{***}&      -0.122\sym{***}&     -0.0281         \\
                    &    (0.0287)         &    (0.0216)         &    (0.0220)         &    (0.0218)         &    (0.0341)         &    (0.0292)         \\
 Ratio population    &      -0.106\sym{**} &      -0.106\sym{***}&      -0.106\sym{***}&      -0.105\sym{***}&      -0.144\sym{***}&     -0.0674         \\
                    &    (0.0451)         &    (0.0339)         &    (0.0346)         &    (0.0345)         &    (0.0441)         &    (0.0512)         \\
 \midrule\multicolumn{7}{l}{\emph{Panel D. Donut specification}} \\ Abs. numbers        &      -3.511         &      -3.511         &      -3.511         &      -3.414         &      -3.556\sym{**} &      0.0444         \\
                    &     (4.451)         &     (2.035)         &     (2.086)         &     (2.047)         &     (1.606)         &     (1.716)         \\
 Ratio fertility     &     -0.0719\sym{**} &     -0.0719\sym{***}&     -0.0719\sym{***}&     -0.0738\sym{***}&      -0.129\sym{***}&     -0.0174         \\
                    &    (0.0311)         &    (0.0212)         &    (0.0217)         &    (0.0214)         &    (0.0373)         &    (0.0319)         \\
 Ratio fertility     &      -0.103\sym{**} &      -0.103\sym{**} &      -0.103\sym{**} &      -0.102\sym{**} &      -0.161\sym{***}&     -0.0473         \\
                    &    (0.0451)         &    (0.0364)         &    (0.0373)         &    (0.0371)         &    (0.0508)         &    (0.0552)         \\
 \midrule Birthmonth FE       &                     &  \checkmark         &  \checkmark         &  \checkmark         &  \checkmark         &  \checkmark         \\
Year FE             &                     &                     &  \checkmark         &  \checkmark         &                     &                     \\
Covariates          &                     &                     &                     &  \checkmark         &                     &                     \\
 
\bottomrule \end{tabular} } \begin{tablenotes} \item \scriptsize \emph{Notes:} Clustered standard errors in parentheses. Personal covariates contain age and age squared. Ratios indicate cases per thousand; either approximated population or original number of births. \end{tablenotes} \end{threeparttable} \end{table} 

 \begin{table}[H] \begin{threeparttable} \centering \caption{Robustness with respect to the choice of \texttt{control group}} {\def\sym#1{\ifmmode^{#1}\else\(^{#1}\)\fi} \begin{tabular}{l*{10}{c}} \toprule & \multicolumn{9}{c}{Dependent variable: \textbf{Diseases of the eye and ear}} \\ \cmidrule(lr){2-10}
            &\multicolumn{3}{c}{Average Causal Effects}&\multicolumn{3}{c}{Women}             &\multicolumn{3}{c}{Men}               \\\cmidrule(lr){2-4}\cmidrule(lr){5-7}\cmidrule(lr){8-10}
            &\multicolumn{1}{c}{(1)}&\multicolumn{1}{c}{(2)}&\multicolumn{1}{c}{(3)}&\multicolumn{1}{c}{(4)}&\multicolumn{1}{c}{(5)}&\multicolumn{1}{c}{(6)}&\multicolumn{1}{c}{(7)}&\multicolumn{1}{c}{(8)}&\multicolumn{1}{c}{(9)}\\
            &\multicolumn{1}{c}{C2}&\multicolumn{1}{c}{C1+C2}&\multicolumn{1}{c}{C1-C3}&\multicolumn{1}{c}{C2}&\multicolumn{1}{c}{C1+C2}&\multicolumn{1}{c}{C1-C3}&\multicolumn{1}{c}{C2}&\multicolumn{1}{c}{C1+C2}&\multicolumn{1}{c}{C1-C3}\\
\midrule
 \multicolumn{10}{l}{\emph{Panel A. 2 Month bandwidth}} \\ Abs. numbers        &      -14.11\sym{***}&      -5.444         &      -4.185         &      -4.500\sym{***}&      -2.806\sym{*}  &      -0.759         &      -9.611\sym{***}&      -2.639         &      -3.426         \\
                    &     (2.409)         &     (4.293)         &     (3.514)         &     (0.441)         &     (1.461)         &     (1.824)         &     (1.973)         &     (3.562)         &     (2.614)         \\
 Ratio population    &      -0.297\sym{***}&      -0.113         &     -0.0773         &      -0.245\sym{***}&      -0.138         &     -0.0841         &      -0.352\sym{***}&     -0.0904         &     -0.0737         \\
                    &    (0.0340)         &    (0.0846)         &    (0.0815)         &    (0.0343)         &    (0.0811)         &     (0.117)         &    (0.0989)         &     (0.126)         &    (0.0973)         \\
 Ratio fertility     &      -0.198\sym{***}&     -0.0565         &     -0.0365         &      -0.124\sym{***}&     -0.0538         &     0.00550         &      -0.268\sym{***}&     -0.0585         &     -0.0758         \\
                    &    (0.0299)         &    (0.0648)         &    (0.0600)         &    (0.0128)         &    (0.0441)         &    (0.0640)         &    (0.0471)         &    (0.0981)         &    (0.0751)         \\
 \midrule\multicolumn{10}{l}{\emph{Panel B. 4 Month bandwidth}} \\ Abs. numbers        &      -9.194\sym{**} &      -1.653         &      -1.852         &      -5.944\sym{***}&      -3.056         &      -2.306         &      -3.250         &       1.403         &       0.454         \\
                    &     (3.382)         &     (3.231)         &     (2.915)         &     (1.941)         &     (2.118)         &     (2.212)         &     (2.793)         &     (2.526)         &     (2.097)         \\
 Ratio population    &      -0.213\sym{***}&     -0.0790         &     -0.0610         &      -0.196\sym{**} &     -0.0959         &     -0.0793         &      -0.232\sym{*}  &     -0.0678         &     -0.0512         \\
                    &    (0.0454)         &    (0.0497)         &    (0.0507)         &    (0.0853)         &    (0.0829)         &    (0.0930)         &     (0.117)         &     (0.105)         &    (0.0953)         \\
 Ratio fertility     &      -0.157\sym{**} &     -0.0269         &     -0.0264         &      -0.197\sym{**} &     -0.0851         &     -0.0623         &      -0.118         &      0.0286         &     0.00789         \\
                    &    (0.0603)         &    (0.0569)         &    (0.0574)         &    (0.0677)         &    (0.0744)         &    (0.0797)         &    (0.0871)         &    (0.0750)         &    (0.0676)         \\
 \midrule\multicolumn{10}{l}{\emph{Panel C. 6 Month bandwidth}} \\ Abs. numbers        &      -5.944\sym{**} &      -0.306         &      -1.605         &      -3.815\sym{**} &      -1.259         &      -1.241         &      -2.130         &       0.954         &      -0.364         \\
                    &     (2.459)         &     (2.331)         &     (2.241)         &     (1.471)         &     (1.550)         &     (1.568)         &     (1.915)         &     (1.799)         &     (1.641)         \\
 Ratio population    &      -0.170\sym{***}&     -0.0639         &     -0.0664         &      -0.193\sym{***}&      -0.104\sym{*}  &     -0.0959         &      -0.146         &     -0.0260         &     -0.0408         \\
                    &    (0.0392)         &    (0.0413)         &    (0.0446)         &    (0.0604)         &    (0.0601)         &    (0.0680)         &    (0.0904)         &    (0.0791)         &    (0.0768)         \\
 Ratio fertility     &      -0.128\sym{***}&     -0.0204         &     -0.0314         &      -0.159\sym{***}&     -0.0542         &     -0.0446         &     -0.0976         &      0.0116         &     -0.0189         \\
                    &    (0.0410)         &    (0.0424)         &    (0.0460)         &    (0.0484)         &    (0.0539)         &    (0.0581)         &    (0.0581)         &    (0.0552)         &    (0.0553)         \\
 \midrule\multicolumn{10}{l}{\emph{Panel D. Donut specification}} \\ Abs. numbers        &      -3.511         &       1.144         &      -1.148         &      -3.556\sym{**} &      -1.300         &      -1.785         &      0.0444         &       2.444         &       0.637         \\
                    &     (2.035)         &     (2.044)         &     (2.248)         &     (1.606)         &     (1.669)         &     (1.641)         &     (1.716)         &     (1.748)         &     (1.758)         \\
 Ratio population    &      -0.131\sym{***}&     -0.0536         &     -0.0735         &      -0.198\sym{**} &      -0.124\sym{*}  &      -0.138\sym{*}  &     -0.0647         &      0.0124         &     -0.0140         \\
                    &    (0.0335)         &    (0.0374)         &    (0.0449)         &    (0.0695)         &    (0.0674)         &    (0.0703)         &    (0.0935)         &    (0.0828)         &    (0.0862)         \\
 Ratio fertility     &      -0.103\sym{**} &    -0.00846         &     -0.0300         &      -0.161\sym{***}&     -0.0626         &     -0.0643         &     -0.0473         &      0.0430         &     0.00268         \\
                    &    (0.0364)         &    (0.0398)         &    (0.0491)         &    (0.0508)         &    (0.0571)         &    (0.0620)         &    (0.0552)         &    (0.0549)         &    (0.0612)         \\
 
\bottomrule \end{tabular} } \begin{tablenotes} \item \scriptsize \emph{Notes:} Clustered standard errors in parentheses. All regressions contain Birthmonth FE. Ratios indicate cases per thousand; either approximated population or original number of births. \end{tablenotes} \end{threeparttable} \end{table} 

%=========================================
 \begin{table}[H] \begin{threeparttable} \centering \caption{Robustness with respect to the inclusion of \texttt{fixed effects} and \texttt{covariates}} {\def\sym#1{\ifmmode^{#1}\else\(^{#1}\)\fi} \begin{tabular}{l*{7}{c}} \toprule & \multicolumn{6}{c}{Dependent variable: \textbf{Diseases of the circulatory system}} \\ \cmidrule(lr){2-7}
            &\multicolumn{4}{c}{Average Causal Effects}         &\multicolumn{2}{c}{Heterogeneous Causal Effects}\\\cmidrule(lr){2-5}\cmidrule(lr){6-7}
            &\multicolumn{1}{c}{(1)}&\multicolumn{1}{c}{(2)}&\multicolumn{1}{c}{(3)}&\multicolumn{1}{c}{(4)}&\multicolumn{1}{c}{(5)}&\multicolumn{1}{c}{(6)}\\
            &\multicolumn{1}{c}{}&\multicolumn{1}{c}{}&\multicolumn{1}{c}{}&\multicolumn{1}{c}{}&\multicolumn{1}{c}{Women}&\multicolumn{1}{c}{Men}\\
\midrule
 \multicolumn{7}{l}{\emph{Panel A. 2 Month bandwidth}} \\ Abs. numbers        &       1.889         &       1.889         &       1.889         &       1.889         &       4.389\sym{***}&      -2.500         \\
                    &     (16.19)         &     (2.951)         &     (3.148)         &     (3.175)         &     (0.598)         &     (3.256)         \\
 Ratio fertility     &       0.215\sym{**} &       0.215\sym{***}&       0.215\sym{***}&       0.215\sym{***}&       0.317\sym{***}&       0.119\sym{**} \\
                    &    (0.0766)         &    (0.0141)         &    (0.0151)         &    (0.0151)         &    (0.0244)         &    (0.0350)         \\
 Ratio fertility     &      0.0378         &      0.0378         &      0.0378         &      0.0378         &       0.144\sym{***}&     -0.0635         \\
                    &     (0.141)         &    (0.0583)         &    (0.0622)         &    (0.0627)         &    (0.0156)         &     (0.115)         \\
 \midrule\multicolumn{7}{l}{\emph{Panel B. 4 Month bandwidth}} \\ Abs. numbers        &       5.513         &       5.513\sym{*}  &       5.513         &       5.513         &       2.713         &       2.800         \\
                    &     (8.555)         &     (3.083)         &     (3.182)         &     (3.188)         &     (2.176)         &     (1.878)         \\
 Ratio population    &      -0.178         &      -0.178\sym{***}&      -0.178\sym{***}&      -0.178\sym{***}&      0.0218         &      -0.398\sym{**} \\
                    &     (0.150)         &    (0.0462)         &    (0.0476)         &    (0.0478)         &     (0.111)         &     (0.185)         \\
 Ratio fertility     &     -0.0637         &     -0.0637         &     -0.0637         &     -0.0637         &      0.0190         &      -0.144         \\
                    &     (0.132)         &    (0.0648)         &    (0.0668)         &    (0.0671)         &    (0.0402)         &     (0.110)         \\
 \midrule\multicolumn{7}{l}{\emph{Panel C. 6 Month bandwidth}} \\ Abs. numbers        &       1.704         &       1.704         &       1.704         &       1.628         &       3.407\sym{*}  &      -1.704         \\
                    &     (10.01)         &     (2.713)         &     (2.768)         &     (2.856)         &     (1.911)         &     (2.150)         \\
 Ratio population    &      -0.140         &      -0.140\sym{**} &      -0.140\sym{**} &      -0.141\sym{**} &     -0.0222         &      -0.262\sym{*}  \\
                    &     (0.136)         &    (0.0593)         &    (0.0605)         &    (0.0632)         &    (0.0826)         &     (0.149)         \\
 Ratio population    &     0.00114         &     0.00114         &     0.00114         &    -0.00454         &      0.0502         &     -0.0479         \\
                    &     (0.105)         &    (0.0560)         &    (0.0571)         &    (0.0574)         &    (0.0568)         &    (0.0913)         \\
 \midrule\multicolumn{7}{l}{\emph{Panel D. Donut specification}} \\ Abs. numbers        &       0.889         &       0.889         &       0.889         &       0.668         &       3.044         &      -2.156         \\
                    &     (10.20)         &     (2.966)         &     (3.039)         &     (3.143)         &     (2.169)         &     (2.521)         \\
 Ratio population    &      -0.153         &      -0.153\sym{**} &      -0.153\sym{**} &      -0.156\sym{**} &     -0.0534         &      -0.260         \\
                    &     (0.137)         &    (0.0678)         &    (0.0695)         &    (0.0727)         &    (0.0941)         &     (0.179)         \\
 Ratio population    &     -0.0727         &     -0.0727         &     -0.0727         &     -0.0800         &     -0.0382         &      -0.108         \\
                    &     (0.105)         &    (0.0520)         &    (0.0533)         &    (0.0530)         &    (0.0459)         &     (0.104)         \\
 \midrule Birthmonth FE       &                     &  \checkmark         &  \checkmark         &  \checkmark         &  \checkmark         &  \checkmark         \\
Year FE             &                     &                     &  \checkmark         &  \checkmark         &                     &                     \\
Covariates          &                     &                     &                     &  \checkmark         &                     &                     \\
 
\bottomrule \end{tabular} } \begin{tablenotes} \item \scriptsize \emph{Notes:} Clustered standard errors in parentheses. Personal covariates contain age and age squared. Ratios indicate cases per thousand; either approximated population or original number of births. \end{tablenotes} \end{threeparttable} \end{table} 

 \begin{table}[H] \begin{threeparttable} \centering \caption{Robustness with respect to the choice of \texttt{control group}} {\def\sym#1{\ifmmode^{#1}\else\(^{#1}\)\fi} \begin{tabular}{l*{10}{c}} \toprule & \multicolumn{9}{c}{Dependent variable: \textbf{Diseases of the circulatory system}} \\ \cmidrule(lr){2-10}
            &\multicolumn{3}{c}{Average Causal Effects}&\multicolumn{3}{c}{Women}             &\multicolumn{3}{c}{Men}               \\\cmidrule(lr){2-4}\cmidrule(lr){5-7}\cmidrule(lr){8-10}
            &\multicolumn{1}{c}{(1)}&\multicolumn{1}{c}{(2)}&\multicolumn{1}{c}{(3)}&\multicolumn{1}{c}{(4)}&\multicolumn{1}{c}{(5)}&\multicolumn{1}{c}{(6)}&\multicolumn{1}{c}{(7)}&\multicolumn{1}{c}{(8)}&\multicolumn{1}{c}{(9)}\\
            &\multicolumn{1}{c}{C2}&\multicolumn{1}{c}{C1+C2}&\multicolumn{1}{c}{C1-C3}&\multicolumn{1}{c}{C2}&\multicolumn{1}{c}{C1+C2}&\multicolumn{1}{c}{C1-C3}&\multicolumn{1}{c}{C2}&\multicolumn{1}{c}{C1+C2}&\multicolumn{1}{c}{C1-C3}\\
\midrule
 \multicolumn{10}{l}{\emph{Panel A. 2 Month bandwidth}} \\ Abs. numbers        &       1.889         &       2.944         &       3.296         &       4.389\sym{***}&       3.306         &       2.167         &      -2.500         &      -0.361         &       1.130         \\
                    &     (2.951)         &     (6.406)         &     (13.41)         &     (0.598)         &     (2.135)         &     (5.841)         &     (3.256)         &     (5.722)         &     (8.624)         \\
 Ratio population    &      -0.142\sym{**} &    -0.00522         &      0.0321         &     -0.0779         &      0.0189         &     -0.0337         &      -0.184\sym{**} &    -0.00705         &       0.121         \\
                    &    (0.0599)         &     (0.181)         &     (0.287)         &    (0.0899)         &     (0.149)         &     (0.278)         &    (0.0744)         &     (0.230)         &     (0.312)         \\
 Ratio fertility     &      0.0378         &      0.0952         &       0.102         &       0.144\sym{***}&       0.156\sym{**} &       0.122         &     -0.0635         &      0.0369         &      0.0826         \\
                    &    (0.0583)         &     (0.118)         &     (0.228)         &    (0.0156)         &    (0.0635)         &     (0.203)         &     (0.115)         &     (0.191)         &     (0.277)         \\
 \midrule\multicolumn{10}{l}{\emph{Panel B. 4 Month bandwidth}} \\ Abs. numbers        &      -1.167         &      -2.708         &      -3.130         &       1.778         &       1.042         &       1.278         &      -2.944         &      -3.750         &      -4.407         \\
                    &     (2.397)         &     (4.927)         &     (8.601)         &     (1.306)         &     (2.460)         &     (3.812)         &     (2.387)         &     (3.656)         &     (5.547)         \\
 Ratio population    &      -0.178\sym{***}&      -0.159         &      -0.125         &      0.0218         &      0.0388         &      0.0299         &      -0.398\sym{**} &      -0.387         &      -0.306         \\
                    &    (0.0462)         &     (0.115)         &     (0.178)         &     (0.111)         &     (0.126)         &     (0.180)         &     (0.185)         &     (0.226)         &     (0.272)         \\
 Ratio fertility     &     -0.0637         &     -0.0358         &     -0.0415         &      0.0190         &      0.0517         &      0.0565         &      -0.144         &      -0.121         &      -0.136         \\
                    &    (0.0648)         &     (0.101)         &     (0.170)         &    (0.0402)         &    (0.0921)         &     (0.153)         &     (0.110)         &     (0.138)         &     (0.206)         \\
 \midrule\multicolumn{10}{l}{\emph{Panel C. 6 Month bandwidth}} \\ Abs. numbers        &       1.704         &       0.139         &      -0.722         &       3.407\sym{*}  &       2.861         &       3.395         &      -1.704         &      -2.722         &      -4.117         \\
                    &     (2.713)         &     (4.589)         &     (6.845)         &     (1.911)         &     (2.574)         &     (3.236)         &     (2.150)         &     (2.877)         &     (4.229)         \\
 Ratio population    &      -0.140\sym{**} &      -0.120         &      -0.101         &     -0.0222         &     -0.0103         &      0.0132         &      -0.262\sym{*}  &      -0.236         &      -0.220         \\
                    &    (0.0593)         &     (0.101)         &     (0.147)         &    (0.0826)         &     (0.114)         &     (0.148)         &     (0.149)         &     (0.169)         &     (0.211)         \\
 Ratio fertility     &     -0.0552         &     -0.0174         &     -0.0207         &      0.0263         &      0.0719         &      0.0987         &      -0.133         &      -0.103         &      -0.134         \\
                    &    (0.0506)         &    (0.0869)         &     (0.137)         &    (0.0425)         &    (0.0907)         &     (0.129)         &    (0.0886)         &     (0.106)         &     (0.160)         \\
 \midrule\multicolumn{10}{l}{\emph{Panel D. Donut specification}} \\ Abs. numbers        &       0.889         &      -2.044         &      -3.267         &       3.044         &       1.944         &       3.141         &      -2.156         &      -3.989         &      -6.407         \\
                    &     (2.966)         &     (5.160)         &     (7.014)         &     (2.169)         &     (3.045)         &     (3.372)         &     (2.521)         &     (3.050)         &     (4.291)         \\
 Ratio population    &      -0.153\sym{**} &      -0.186\sym{*}  &      -0.176         &     -0.0534         &     -0.0743         &     -0.0446         &      -0.260         &      -0.311\sym{*}  &      -0.317         \\
                    &    (0.0678)         &     (0.101)         &     (0.146)         &    (0.0941)         &     (0.128)         &     (0.151)         &     (0.179)         &     (0.183)         &     (0.225)         \\
 Ratio fertility     &     -0.0896         &     -0.0696         &     -0.0700         &    -0.00421         &      0.0297         &      0.0819         &      -0.172         &      -0.165         &      -0.215         \\
                    &    (0.0575)         &    (0.0964)         &     (0.147)         &    (0.0475)         &     (0.105)         &     (0.137)         &     (0.104)         &     (0.115)         &     (0.171)         \\
 
\bottomrule \end{tabular} } \begin{tablenotes} \item \scriptsize \emph{Notes:} Clustered standard errors in parentheses. All regressions contain Birthmonth FE. Ratios indicate cases per thousand; either approximated population or original number of births. \end{tablenotes} \end{threeparttable} \end{table} 

%=========================================
 \begin{table}[H] \begin{threeparttable} \centering \caption{Robustness with respect to the inclusion of \texttt{fixed effects} and \texttt{covariates}} {\def\sym#1{\ifmmode^{#1}\else\(^{#1}\)\fi} \begin{tabular}{l*{7}{c}} \toprule & \multicolumn{6}{c}{Dependent variable: \textbf{Diseases of the respiratory system}} \\ \cmidrule(lr){2-7}
            &\multicolumn{4}{c}{Average Causal Effects}         &\multicolumn{2}{c}{Heterogeneous Causal Effects}\\\cmidrule(lr){2-5}\cmidrule(lr){6-7}
            &\multicolumn{1}{c}{(1)}&\multicolumn{1}{c}{(2)}&\multicolumn{1}{c}{(3)}&\multicolumn{1}{c}{(4)}&\multicolumn{1}{c}{(5)}&\multicolumn{1}{c}{(6)}\\
            &\multicolumn{1}{c}{}&\multicolumn{1}{c}{}&\multicolumn{1}{c}{}&\multicolumn{1}{c}{}&\multicolumn{1}{c}{Women}&\multicolumn{1}{c}{Men}\\
\midrule
 \multicolumn{7}{l}{\emph{Panel A. 2 Month bandwidth}} \\ Abs. numbers        &      -15.03         &      -15.03\sym{**} &      -15.03\sym{**} &      -15.02\sym{**} &      -0.225         &      -14.80\sym{***}\\
                    &     (20.67)         &     (4.513)         &     (4.820)         &     (4.838)         &     (4.791)         &     (1.221)         \\
 Ratio fertility     &      -0.192         &      -0.192\sym{*}  &      -0.192\sym{*}  &      -0.192\sym{*}  &      0.0382         &      -0.407\sym{***}\\
                    &     (0.169)         &    (0.0898)         &    (0.0959)         &    (0.0962)         &     (0.140)         &    (0.0572)         \\
 Ratio population    &     -0.0202         &     -0.0202         &     -0.0202         &     -0.0202         &       0.221\sym{**} &      -0.264\sym{***}\\
                    &     (0.182)         &    (0.0697)         &    (0.0744)         &    (0.0749)         &    (0.0844)         &    (0.0741)         \\
 \midrule\multicolumn{7}{l}{\emph{Panel B. 4 Month bandwidth}} \\ Abs. numbers        &       7.861         &       7.861         &       7.861         &       7.861         &       11.53\sym{***}&      -3.667         \\
                    &     (13.74)         &     (4.848)         &     (4.999)         &     (5.019)         &     (2.179)         &     (3.034)         \\
 Ratio fertility     &      -0.126         &      -0.126\sym{*}  &      -0.126\sym{*}  &      -0.126\sym{*}  &       0.115         &      -0.351\sym{***}\\
                    &     (0.201)         &    (0.0613)         &    (0.0633)         &    (0.0634)         &    (0.0767)         &    (0.0574)         \\
 Ratio population    &     -0.0335         &     -0.0335         &     -0.0335         &     -0.0335         &       0.215\sym{***}&      -0.283\sym{***}\\
                    &     (0.213)         &    (0.0459)         &    (0.0474)         &    (0.0475)         &    (0.0493)         &    (0.0545)         \\
 \midrule\multicolumn{7}{l}{\emph{Panel C. 6 Month bandwidth}} \\ Abs. numbers        &       14.43         &       14.43\sym{**} &       14.43\sym{**} &       14.96\sym{**} &       11.00\sym{***}&       3.426         \\
                    &     (14.87)         &     (5.240)         &     (5.347)         &     (5.421)         &     (2.983)         &     (3.246)         \\
 Ratio population    &     -0.0948         &     -0.0948         &     -0.0948         &     -0.0799         &      0.0896         &      -0.276         \\
                    &     (0.238)         &     (0.138)         &     (0.141)         &     (0.147)         &     (0.218)         &     (0.276)         \\
 Ratio fertility     &      0.0273         &      0.0273         &      0.0273         &      0.0354         &       0.145         &     -0.0852         \\
                    &     (0.172)         &    (0.0635)         &    (0.0647)         &    (0.0660)         &    (0.0876)         &    (0.0816)         \\
 \midrule\multicolumn{7}{l}{\emph{Panel D. Donut specification}} \\ Abs. numbers        &       13.04         &       13.04\sym{*}  &       13.04\sym{*}  &       13.70\sym{**} &       9.311\sym{**} &       3.733         \\
                    &     (17.61)         &     (6.266)         &     (6.421)         &     (6.512)         &     (3.461)         &     (3.907)         \\
 Ratio fertility     &     -0.0685         &     -0.0685         &     -0.0685         &     -0.0643         &     -0.0592         &     -0.0771         \\
                    &     (0.184)         &    (0.0566)         &    (0.0580)         &    (0.0589)         &    (0.0702)         &    (0.0850)         \\
 Ratio fertility     &     -0.0320         &     -0.0320         &     -0.0320         &     -0.0218         &      0.0604         &      -0.121         \\
                    &     (0.183)         &    (0.0636)         &    (0.0652)         &    (0.0669)         &    (0.0907)         &    (0.0924)         \\
 \midrule Birthmonth FE       &                     &  \checkmark         &  \checkmark         &  \checkmark         &  \checkmark         &  \checkmark         \\
Year FE             &                     &                     &  \checkmark         &  \checkmark         &                     &                     \\
Covariates          &                     &                     &                     &  \checkmark         &                     &                     \\
 
\bottomrule \end{tabular} } \begin{tablenotes} \item \scriptsize \emph{Notes:} Clustered standard errors in parentheses. Personal covariates contain age and age squared. Ratios indicate cases per thousand; either approximated population or original number of births. \end{tablenotes} \end{threeparttable} \end{table} 

 \begin{table}[H] \begin{threeparttable} \centering \caption{Robustness with respect to the choice of \texttt{control group}} {\def\sym#1{\ifmmode^{#1}\else\(^{#1}\)\fi} \begin{tabular}{l*{10}{c}} \toprule & \multicolumn{9}{c}{Dependent variable: \textbf{Diseases of the respiratory system}} \\ \cmidrule(lr){2-10}
            &\multicolumn{3}{c}{Average Causal Effects}&\multicolumn{3}{c}{Women}             &\multicolumn{3}{c}{Men}               \\\cmidrule(lr){2-4}\cmidrule(lr){5-7}\cmidrule(lr){8-10}
            &\multicolumn{1}{c}{(1)}&\multicolumn{1}{c}{(2)}&\multicolumn{1}{c}{(3)}&\multicolumn{1}{c}{(4)}&\multicolumn{1}{c}{(5)}&\multicolumn{1}{c}{(6)}&\multicolumn{1}{c}{(7)}&\multicolumn{1}{c}{(8)}&\multicolumn{1}{c}{(9)}\\
            &\multicolumn{1}{c}{C2}&\multicolumn{1}{c}{C1+C2}&\multicolumn{1}{c}{C1-C3}&\multicolumn{1}{c}{C2}&\multicolumn{1}{c}{C1+C2}&\multicolumn{1}{c}{C1-C3}&\multicolumn{1}{c}{C2}&\multicolumn{1}{c}{C1+C2}&\multicolumn{1}{c}{C1-C3}\\
\midrule
 \multicolumn{10}{l}{\emph{Panel A. 2 Month bandwidth}} \\ Abs. numbers        &       0.833         &      -6.444         &      -13.91         &       9.444\sym{**} &      -0.306         &      -0.389         &      -8.611         &      -6.139         &      -13.52\sym{**} \\
                    &     (8.277)         &     (7.891)         &     (9.809)         &     (3.926)         &     (4.961)         &     (5.723)         &     (4.667)         &     (4.650)         &     (6.228)         \\
 Ratio population    &      -0.287\sym{**} &      -0.200         &      -0.229         &     -0.0490         &      -0.167         &      -0.189         &      -0.508\sym{***}&      -0.227         &      -0.255\sym{*}  \\
                    &     (0.106)         &     (0.182)         &     (0.174)         &     (0.285)         &     (0.296)         &     (0.312)         &     (0.113)         &     (0.193)         &     (0.139)         \\
 Ratio fertility     &      0.0263         &    -0.00492         &     -0.0927         &       0.310\sym{**} &      0.0996         &       0.108         &      -0.244         &      -0.107         &      -0.285\sym{*}  \\
                    &     (0.117)         &     (0.100)         &    (0.0948)         &     (0.104)         &     (0.123)         &     (0.120)         &     (0.134)         &     (0.130)         &     (0.140)         \\
 \midrule\multicolumn{10}{l}{\emph{Panel B. 4 Month bandwidth}} \\ Abs. numbers        &       7.861         &       3.569         &      -0.954         &       11.53\sym{***}&       6.153\sym{*}  &       5.028         &      -3.667         &      -2.583         &      -5.981         \\
                    &     (4.848)         &     (6.398)         &     (7.931)         &     (2.179)         &     (3.516)         &     (4.167)         &     (3.034)         &     (3.696)         &     (4.836)         \\
 Ratio population    &      -0.171\sym{*}  &      -0.174         &      -0.142         &       0.357         &       0.203         &       0.160         &      -0.744\sym{**} &      -0.613\sym{*}  &      -0.499\sym{*}  \\
                    &    (0.0933)         &     (0.152)         &     (0.137)         &     (0.253)         &     (0.239)         &     (0.230)         &     (0.282)         &     (0.311)         &     (0.289)         \\
 Ratio fertility     &   -0.000637         &      0.0130         &     -0.0260         &       0.254\sym{***}&       0.178\sym{**} &       0.163\sym{*}  &      -0.244\sym{**} &      -0.146         &      -0.207\sym{*}  \\
                    &    (0.0717)         &    (0.0815)         &    (0.0748)         &    (0.0686)         &    (0.0787)         &    (0.0858)         &    (0.0854)         &     (0.111)         &     (0.106)         \\
 \midrule\multicolumn{10}{l}{\emph{Panel C. 6 Month bandwidth}} \\ Abs. numbers        &       14.43\sym{**} &       6.500         &       2.420         &       11.00\sym{***}&       4.824         &       3.790         &       3.426         &       1.676         &      -1.370         \\
                    &     (5.240)         &     (5.725)         &     (6.931)         &     (2.983)         &     (3.228)         &     (3.949)         &     (3.246)         &     (3.529)         &     (4.234)         \\
 Ratio population    &     -0.0948         &      -0.158         &      -0.129         &      0.0896         &     -0.0896         &     -0.0774         &      -0.276         &      -0.226         &      -0.180         \\
                    &     (0.138)         &     (0.148)         &     (0.139)         &     (0.218)         &     (0.206)         &     (0.206)         &     (0.276)         &     (0.276)         &     (0.266)         \\
 Ratio fertility     &      0.0273         &   -0.000813         &    -0.00798         &       0.145         &      0.0472         &      0.0737         &     -0.0852         &     -0.0469         &     -0.0858         \\
                    &    (0.0635)         &    (0.0701)         &    (0.0709)         &    (0.0876)         &    (0.0857)         &    (0.0912)         &    (0.0816)         &     (0.100)         &     (0.107)         \\
 \midrule\multicolumn{10}{l}{\emph{Panel D. Donut specification}} \\ Abs. numbers        &       13.04\sym{*}  &       5.622         &       2.881         &       9.311\sym{**} &       4.122         &       3.815         &       3.733         &       1.500         &      -0.933         \\
                    &     (6.266)         &     (6.765)         &     (7.840)         &     (3.461)         &     (3.680)         &     (4.370)         &     (3.907)         &     (4.201)         &     (4.942)         \\
 Ratio population    &      -0.109         &      -0.220         &      -0.183         &     -0.0140         &      -0.184         &      -0.184         &      -0.200         &      -0.259         &      -0.179         \\
                    &     (0.166)         &     (0.168)         &     (0.159)         &     (0.256)         &     (0.232)         &     (0.235)         &     (0.330)         &     (0.325)         &     (0.316)         \\
 Ratio fertility     &     -0.0320         &     -0.0517         &     -0.0248         &      0.0604         &    -0.00546         &      0.0548         &      -0.121         &     -0.0964         &      -0.101         \\
                    &    (0.0636)         &    (0.0764)         &    (0.0810)         &    (0.0907)         &    (0.0942)         &     (0.105)         &    (0.0924)         &     (0.113)         &     (0.125)         \\
 
\bottomrule \end{tabular} } \begin{tablenotes} \item \scriptsize \emph{Notes:} Clustered standard errors in parentheses. All regressions contain Birthmonth FE. Ratios indicate cases per thousand; either approximated population or original number of births. \end{tablenotes} \end{threeparttable} \end{table} 

%=========================================
 \begin{table}[H] \begin{threeparttable} \centering \caption{Robustness with respect to the inclusion of \texttt{fixed effects} and \texttt{covariates}} {\def\sym#1{\ifmmode^{#1}\else\(^{#1}\)\fi} \begin{tabular}{l*{7}{c}} \toprule & \multicolumn{6}{c}{Dependent variable: \textbf{Diseases of the digestive system}} \\ \cmidrule(lr){2-7}
            &\multicolumn{4}{c}{Average Causal Effects}         &\multicolumn{2}{c}{Heterogeneous Causal Effects}\\\cmidrule(lr){2-5}\cmidrule(lr){6-7}
            &\multicolumn{1}{c}{(1)}&\multicolumn{1}{c}{(2)}&\multicolumn{1}{c}{(3)}&\multicolumn{1}{c}{(4)}&\multicolumn{1}{c}{(5)}&\multicolumn{1}{c}{(6)}\\
            &\multicolumn{1}{c}{}&\multicolumn{1}{c}{}&\multicolumn{1}{c}{}&\multicolumn{1}{c}{}&\multicolumn{1}{c}{Women}&\multicolumn{1}{c}{Men}\\
\midrule
 \multicolumn{7}{l}{\emph{Panel A. 2 Month bandwidth}} \\ Abs. numbers        &       16.94         &       16.94         &       16.94         &       16.94         &       7.556\sym{*}  &       9.389         \\
                    &     (27.00)         &     (14.44)         &     (15.41)         &     (15.54)         &     (3.838)         &     (13.59)         \\
 Ratio population    &      -0.274         &      -0.274         &      -0.274         &      -0.274         &      -0.400         &      -0.111         \\
                    &     (0.202)         &     (0.152)         &     (0.162)         &     (0.163)         &     (0.423)         &     (0.497)         \\
 Ratio fertility     &       0.262         &       0.262         &       0.262         &       0.262         &       0.283\sym{**} &       0.246         \\
                    &     (0.242)         &     (0.174)         &     (0.186)         &     (0.188)         &     (0.106)         &     (0.332)         \\
 \midrule\multicolumn{7}{l}{\emph{Panel B. 4 Month bandwidth}} \\ Abs. numbers        &       10.14         &       10.14         &       10.14         &       10.14         &       6.288\sym{**} &       3.850         \\
                    &     (25.49)         &     (6.244)         &     (6.445)         &     (6.456)         &     (2.896)         &     (4.925)         \\
 Ratio population    &      -0.485         &      -0.485\sym{*}  &      -0.485\sym{*}  &      -0.485\sym{*}  &     -0.0114         &      -0.921         \\
                    &     (0.433)         &     (0.229)         &     (0.237)         &     (0.238)         &     (0.296)         &     (0.630)         \\
 Ratio population    &      -0.178         &      -0.178         &      -0.178         &      -0.178         &    -0.00552         &      -0.346         \\
                    &     (0.337)         &     (0.189)         &     (0.195)         &     (0.195)         &     (0.212)         &     (0.242)         \\
 \midrule\multicolumn{7}{l}{\emph{Panel C. 6 Month bandwidth}} \\ Abs. numbers        &       17.37         &       17.37\sym{***}&       17.37\sym{***}&       16.58\sym{***}&       6.000\sym{**} &       11.37\sym{***}\\
                    &     (24.73)         &     (5.007)         &     (5.113)         &     (5.060)         &     (2.482)         &     (3.996)         \\
 Ratio population    &      -0.307         &      -0.307         &      -0.307         &      -0.308         &      -0.500\sym{*}  &     -0.0695         \\
                    &     (0.332)         &     (0.195)         &     (0.199)         &     (0.208)         &     (0.249)         &     (0.534)         \\
 Ratio population    &     -0.0873         &     -0.0873         &     -0.0873         &      -0.105         &      -0.228         &      0.0533         \\
                    &     (0.247)         &     (0.136)         &     (0.138)         &     (0.137)         &     (0.164)         &     (0.214)         \\
 \midrule\multicolumn{7}{l}{\emph{Panel D. Donut specification}} \\ Abs. numbers        &       16.39         &       16.39\sym{**} &       16.39\sym{**} &       15.45\sym{**} &       3.910\sym{*}  &       12.48\sym{**} \\
                    &     (27.66)         &     (5.988)         &     (6.140)         &     (6.059)         &     (2.132)         &     (4.620)         \\
 Ratio population    &      -0.358         &      -0.358         &      -0.358         &      -0.360         &      -0.706\sym{**} &      0.0164         \\
                    &     (0.399)         &     (0.233)         &     (0.239)         &     (0.251)         &     (0.269)         &     (0.638)         \\
 Ratio population    &      -0.199         &      -0.199         &      -0.199         &      -0.220         &      -0.422\sym{**} &      0.0240         \\
                    &     (0.278)         &     (0.138)         &     (0.142)         &     (0.140)         &     (0.163)         &     (0.220)         \\
 \midrule Birthmonth FE       &                     &  \checkmark         &  \checkmark         &  \checkmark         &  \checkmark         &  \checkmark         \\
Year FE             &                     &                     &  \checkmark         &  \checkmark         &                     &                     \\
Covariates          &                     &                     &                     &  \checkmark         &                     &                     \\
 
\bottomrule \end{tabular} } \begin{tablenotes} \item \scriptsize \emph{Notes:} Clustered standard errors in parentheses. Personal covariates contain age and age squared. Ratios indicate cases per thousand; either approximated population or original number of births. \end{tablenotes} \end{threeparttable} \end{table} 

 \begin{table}[H] \begin{threeparttable} \centering \caption{Robustness with respect to the choice of \texttt{control group}} {\def\sym#1{\ifmmode^{#1}\else\(^{#1}\)\fi} \begin{tabular}{l*{10}{c}} \toprule & \multicolumn{9}{c}{Dependent variable: \textbf{Diseases of the digestive system}} \\ \cmidrule(lr){2-10}
            &\multicolumn{3}{c}{Average Causal Effects}&\multicolumn{3}{c}{Women}             &\multicolumn{3}{c}{Men}               \\\cmidrule(lr){2-4}\cmidrule(lr){5-7}\cmidrule(lr){8-10}
            &\multicolumn{1}{c}{(1)}&\multicolumn{1}{c}{(2)}&\multicolumn{1}{c}{(3)}&\multicolumn{1}{c}{(4)}&\multicolumn{1}{c}{(5)}&\multicolumn{1}{c}{(6)}&\multicolumn{1}{c}{(7)}&\multicolumn{1}{c}{(8)}&\multicolumn{1}{c}{(9)}\\
            &\multicolumn{1}{c}{C2}&\multicolumn{1}{c}{C1+C2}&\multicolumn{1}{c}{C1-C3}&\multicolumn{1}{c}{C2}&\multicolumn{1}{c}{C1+C2}&\multicolumn{1}{c}{C1-C3}&\multicolumn{1}{c}{C2}&\multicolumn{1}{c}{C1+C2}&\multicolumn{1}{c}{C1-C3}\\
\midrule
 \multicolumn{10}{l}{\emph{Panel A. 2 Month bandwidth}} \\ Abs. numbers        &       16.94         &       13.28         &       16.61         &       7.556\sym{*}  &       3.889         &       5.019         &       9.389         &       9.389         &       11.59         \\
                    &     (14.44)         &     (12.79)         &     (10.75)         &     (3.838)         &     (4.426)         &     (5.568)         &     (13.59)         &     (12.71)         &     (11.58)         \\
 Ratio population    &      -0.274         &      0.0125         &       0.191         &      -0.400         &      -0.156         &      -0.177         &      -0.111         &       0.203         &       0.555         \\
                    &     (0.152)         &     (0.241)         &     (0.287)         &     (0.423)         &     (0.512)         &     (0.576)         &     (0.497)         &     (0.383)         &     (0.438)         \\
 Ratio fertility     &       0.262         &       0.334\sym{**} &       0.401\sym{**} &       0.283\sym{**} &       0.312\sym{*}  &       0.357\sym{**} &       0.246         &       0.358         &       0.446         \\
                    &     (0.174)         &     (0.130)         &     (0.150)         &     (0.106)         &     (0.153)         &     (0.145)         &     (0.332)         &     (0.274)         &     (0.300)         \\
 \midrule\multicolumn{10}{l}{\emph{Panel B. 4 Month bandwidth}} \\ Abs. numbers        &       2.444         &       4.903         &       7.250         &       1.083         &       2.389         &       3.019         &       1.361         &       2.514         &       4.231         \\
                    &     (11.43)         &     (9.446)         &     (8.672)         &     (3.849)         &     (3.895)         &     (4.586)         &     (9.418)         &     (7.921)         &     (6.975)         \\
 Ratio population    &      -0.485\sym{*}  &      -0.330         &      -0.134         &     -0.0114         &      0.0806         &      0.0772         &      -0.921         &      -0.756         &      -0.378         \\
                    &     (0.229)         &     (0.250)         &     (0.247)         &     (0.296)         &     (0.303)         &     (0.326)         &     (0.630)         &     (0.541)         &     (0.549)         \\
 Ratio fertility     &      -0.165         &      0.0109         &      0.0801         &      -0.136         &      0.0635         &       0.114         &      -0.189         &     -0.0336         &      0.0530         \\
                    &     (0.212)         &     (0.193)         &     (0.200)         &     (0.190)         &     (0.194)         &     (0.180)         &     (0.274)         &     (0.239)         &     (0.255)         \\
 \midrule\multicolumn{10}{l}{\emph{Panel C. 6 Month bandwidth}} \\ Abs. numbers        &       16.69\sym{*}  &       15.62\sym{*}  &       13.48\sym{*}  &       0.130         &       2.741         &       0.642         &       16.56\sym{**} &       12.88\sym{*}  &       12.83\sym{**} \\
                    &     (8.919)         &     (7.759)         &     (7.088)         &     (3.909)         &     (3.915)         &     (4.465)         &     (7.935)         &     (6.580)         &     (5.850)         \\
 Ratio population    &      -0.307         &      -0.200         &     -0.0914         &      -0.500\sym{*}  &      -0.353         &      -0.350         &     -0.0695         &     -0.0368         &       0.166         \\
                    &     (0.195)         &     (0.198)         &     (0.198)         &     (0.249)         &     (0.251)         &     (0.271)         &     (0.534)         &     (0.449)         &     (0.443)         \\
 Ratio fertility     &     -0.0629         &      0.0837         &       0.127         &      -0.339\sym{**} &     -0.0829         &     -0.0541         &       0.199         &       0.242         &       0.299         \\
                    &     (0.149)         &     (0.145)         &     (0.151)         &     (0.150)         &     (0.159)         &     (0.144)         &     (0.233)         &     (0.201)         &     (0.217)         \\
 \midrule\multicolumn{10}{l}{\emph{Panel D. Donut specification}} \\ Abs. numbers        &       13.00         &       13.62         &       10.75         &      -3.289         &       0.344         &      -1.659         &       16.29\sym{*}  &       13.28\sym{*}  &       12.41\sym{*}  \\
                    &     (10.48)         &     (8.989)         &     (8.027)         &     (4.078)         &     (4.386)         &     (4.849)         &     (9.247)         &     (7.264)         &     (6.281)         \\
 Ratio population    &      -0.358         &      -0.314         &      -0.233         &      -0.706\sym{**} &      -0.563\sym{**} &      -0.613\sym{**} &      0.0164         &     -0.0740         &       0.128         \\
                    &     (0.233)         &     (0.217)         &     (0.208)         &     (0.269)         &     (0.243)         &     (0.262)         &     (0.638)         &     (0.532)         &     (0.524)         \\
 Ratio fertility     &      -0.185         &    -0.00899         &      0.0498         &      -0.512\sym{***}&      -0.214         &      -0.160         &       0.125         &       0.186         &       0.250         \\
                    &     (0.155)         &     (0.154)         &     (0.168)         &     (0.151)         &     (0.175)         &     (0.158)         &     (0.256)         &     (0.209)         &     (0.235)         \\
 
\bottomrule \end{tabular} } \begin{tablenotes} \item \scriptsize \emph{Notes:} Clustered standard errors in parentheses. All regressions contain Birthmonth FE. Ratios indicate cases per thousand; either approximated population or original number of births. \end{tablenotes} \end{threeparttable} \end{table} 

%=========================================
 \begin{table}[H] \begin{threeparttable} \centering \caption{Robustness with respect to the inclusion of \texttt{fixed effects} and \texttt{covariates}} {\def\sym#1{\ifmmode^{#1}\else\(^{#1}\)\fi} \begin{tabular}{l*{7}{c}} \toprule & \multicolumn{6}{c}{Dependent variable: \textbf{Diseases of the skin and subcutaneous tissue}} \\ \cmidrule(lr){2-7}
            &\multicolumn{4}{c}{Average Causal Effects}         &\multicolumn{2}{c}{Heterogeneous Causal Effects}\\\cmidrule(lr){2-5}\cmidrule(lr){6-7}
            &\multicolumn{1}{c}{(1)}&\multicolumn{1}{c}{(2)}&\multicolumn{1}{c}{(3)}&\multicolumn{1}{c}{(4)}&\multicolumn{1}{c}{(5)}&\multicolumn{1}{c}{(6)}\\
            &\multicolumn{1}{c}{}&\multicolumn{1}{c}{}&\multicolumn{1}{c}{}&\multicolumn{1}{c}{}&\multicolumn{1}{c}{Women}&\multicolumn{1}{c}{Men}\\
\midrule
 \multicolumn{7}{l}{\emph{Panel A. 2 Month bandwidth}} \\ Abs. numbers        &       15.25\sym{**} &       15.25\sym{***}&       15.25\sym{***}&       15.25\sym{***}&       1.025         &       14.23\sym{***}\\
                    &     (4.679)         &     (1.871)         &     (1.998)         &     (2.005)         &     (0.652)         &     (2.518)         \\
 Ratio fertility     &       0.226\sym{*}  &       0.226\sym{***}&       0.226\sym{***}&       0.226\sym{***}&      0.0439\sym{*}  &       0.396\sym{***}\\
                    &     (0.107)         &    (0.0179)         &    (0.0192)         &    (0.0192)         &    (0.0213)         &    (0.0503)         \\
 Ratio fertility     &       0.343\sym{**} &       0.343\sym{***}&       0.343\sym{***}&       0.343\sym{***}&      0.0506         &       0.617\sym{***}\\
                    &     (0.116)         &    (0.0874)         &    (0.0932)         &    (0.0941)         &    (0.0740)         &    (0.1000)         \\
 \midrule\multicolumn{7}{l}{\emph{Panel B. 4 Month bandwidth}} \\ Abs. numbers        &       12.99\sym{**} &       12.99\sym{***}&       12.99\sym{***}&       12.99\sym{***}&       4.250\sym{***}&       8.737\sym{***}\\
                    &     (5.588)         &     (1.554)         &     (1.604)         &     (1.607)         &     (1.051)         &     (2.277)         \\
 Ratio population    &       0.114         &       0.114         &       0.114         &       0.114         &       0.121\sym{**} &      0.0557         \\
                    &     (0.100)         &    (0.0716)         &    (0.0738)         &    (0.0741)         &    (0.0526)         &     (0.214)         \\
 Ratio population    &       0.180\sym{*}  &       0.180\sym{**} &       0.180\sym{**} &       0.180\sym{**} &       0.137\sym{**} &       0.220\sym{*}  \\
                    &     (0.101)         &    (0.0616)         &    (0.0635)         &    (0.0637)         &    (0.0481)         &     (0.110)         \\
 \midrule\multicolumn{7}{l}{\emph{Panel C. 6 Month bandwidth}} \\ Abs. numbers        &       13.02\sym{**} &       13.02\sym{***}&       13.02\sym{***}&       13.44\sym{***}&       4.767\sym{***}&       8.258\sym{***}\\
                    &     (6.249)         &     (1.075)         &     (1.097)         &     (1.063)         &     (0.787)         &     (1.632)         \\
 Ratio fertility     &       0.114         &       0.114\sym{***}&       0.114\sym{***}&       0.120\sym{***}&      0.0784\sym{**} &       0.148\sym{***}\\
                    &    (0.0781)         &    (0.0281)         &    (0.0287)         &    (0.0288)         &    (0.0306)         &    (0.0511)         \\
 Ratio fertility     &       0.125         &       0.125\sym{**} &       0.125\sym{**} &       0.122\sym{**} &      0.0775         &       0.171\sym{*}  \\
                    &    (0.0861)         &    (0.0556)         &    (0.0567)         &    (0.0573)         &    (0.0535)         &    (0.0842)         \\
 \midrule\multicolumn{7}{l}{\emph{Panel D. Donut specification}} \\ Abs. numbers        &       14.47         &       14.47\sym{***}&       14.47\sym{***}&       14.14\sym{***}&       6.000\sym{***}&       8.467\sym{**} \\
                    &     (8.422)         &     (3.545)         &     (3.633)         &     (3.731)         &     (1.789)         &     (3.049)         \\
 Ratio population    &       0.100         &       0.100         &       0.100         &      0.0970         &      0.0855         &       0.126         \\
                    &     (0.120)         &    (0.0669)         &    (0.0686)         &    (0.0698)         &    (0.0565)         &     (0.184)         \\
 Ratio fertility     &       0.118         &       0.118\sym{*}  &       0.118\sym{*}  &       0.113\sym{*}  &       0.109\sym{*}  &       0.127         \\
                    &    (0.0996)         &    (0.0616)         &    (0.0631)         &    (0.0641)         &    (0.0559)         &    (0.0949)         \\
 \midrule Birthmonth FE       &                     &  \checkmark         &  \checkmark         &  \checkmark         &  \checkmark         &  \checkmark         \\
Year FE             &                     &                     &  \checkmark         &  \checkmark         &                     &                     \\
Covariates          &                     &                     &                     &  \checkmark         &                     &                     \\
 
\bottomrule \end{tabular} } \begin{tablenotes} \item \scriptsize \emph{Notes:} Clustered standard errors in parentheses. Personal covariates contain age and age squared. Ratios indicate cases per thousand; either approximated population or original number of births. \end{tablenotes} \end{threeparttable} \end{table} 

 \begin{table}[H] \begin{threeparttable} \centering \caption{Robustness with respect to the choice of \texttt{control group}} {\def\sym#1{\ifmmode^{#1}\else\(^{#1}\)\fi} \begin{tabular}{l*{10}{c}} \toprule & \multicolumn{9}{c}{Dependent variable: \textbf{Diseases of the skin and subcutaneous tissue}} \\ \cmidrule(lr){2-10}
            &\multicolumn{3}{c}{Average Causal Effects}&\multicolumn{3}{c}{Women}             &\multicolumn{3}{c}{Men}               \\\cmidrule(lr){2-4}\cmidrule(lr){5-7}\cmidrule(lr){8-10}
            &\multicolumn{1}{c}{(1)}&\multicolumn{1}{c}{(2)}&\multicolumn{1}{c}{(3)}&\multicolumn{1}{c}{(4)}&\multicolumn{1}{c}{(5)}&\multicolumn{1}{c}{(6)}&\multicolumn{1}{c}{(7)}&\multicolumn{1}{c}{(8)}&\multicolumn{1}{c}{(9)}\\
            &\multicolumn{1}{c}{C2}&\multicolumn{1}{c}{C1+C2}&\multicolumn{1}{c}{C1-C3}&\multicolumn{1}{c}{C2}&\multicolumn{1}{c}{C1+C2}&\multicolumn{1}{c}{C1-C3}&\multicolumn{1}{c}{C2}&\multicolumn{1}{c}{C1+C2}&\multicolumn{1}{c}{C1-C3}\\
\midrule
 \multicolumn{10}{l}{\emph{Panel A. 2 Month bandwidth}} \\ Abs. numbers        &       23.56\sym{***}&       21.17\sym{***}&       16.04\sym{**} &       1.444         &       5.944         &       2.704         &       22.11\sym{***}&       15.22\sym{***}&       13.33\sym{***}\\
                    &     (6.655)         &     (5.336)         &     (7.412)         &     (2.544)         &     (3.561)         &     (4.068)         &     (4.157)         &     (4.117)         &     (4.522)         \\
 Ratio population    &       0.247\sym{**} &       0.290\sym{**} &       0.244\sym{***}&     -0.0604\sym{*}  &       0.143         &      0.0342         &       0.580\sym{**} &       0.468\sym{**} &       0.502\sym{***}\\
                    &    (0.0980)         &    (0.0984)         &    (0.0826)         &    (0.0276)         &     (0.107)         &     (0.110)         &     (0.210)         &     (0.176)         &     (0.159)         \\
 Ratio fertility     &       0.343\sym{***}&       0.338\sym{***}&       0.273\sym{***}&      0.0506         &       0.206\sym{*}  &       0.116         &       0.617\sym{***}&       0.462\sym{***}&       0.420\sym{***}\\
                    &    (0.0874)         &    (0.0697)         &    (0.0827)         &    (0.0740)         &    (0.0983)         &     (0.101)         &    (0.1000)         &     (0.109)         &     (0.108)         \\
 \midrule\multicolumn{10}{l}{\emph{Panel B. 4 Month bandwidth}} \\ Abs. numbers        &       14.94\sym{***}&       14.74\sym{***}&       12.09\sym{**} &       4.222\sym{*}  &       7.472\sym{***}&       5.333\sym{*}  &       10.72\sym{**} &       7.264\sym{**} &       6.759\sym{*}  \\
                    &     (4.311)         &     (3.613)         &     (4.709)         &     (2.223)         &     (2.556)         &     (2.882)         &     (3.779)         &     (3.192)         &     (3.600)         \\
 Ratio population    &       0.114         &       0.137\sym{*}  &       0.136\sym{**} &       0.121\sym{**} &       0.234\sym{***}&       0.164\sym{**} &      0.0557         &    -0.00710         &      0.0773         \\
                    &    (0.0716)         &    (0.0748)         &    (0.0657)         &    (0.0526)         &    (0.0681)         &    (0.0776)         &     (0.214)         &     (0.182)         &     (0.177)         \\
 Ratio fertility     &       0.166\sym{*}  &       0.194\sym{***}&       0.168\sym{***}&      0.0932         &       0.216\sym{**} &       0.160\sym{*}  &       0.231\sym{*}  &       0.170         &       0.172\sym{*}  \\
                    &    (0.0787)         &    (0.0664)         &    (0.0601)         &    (0.0710)         &    (0.0795)         &    (0.0795)         &     (0.123)         &     (0.103)         &    (0.0948)         \\
 \midrule\multicolumn{10}{l}{\emph{Panel C. 6 Month bandwidth}} \\ Abs. numbers        &       13.93\sym{***}&       13.53\sym{***}&       10.72\sym{**} &       4.574\sym{**} &       6.315\sym{***}&       4.537\sym{*}  &       9.352\sym{***}&       7.213\sym{***}&       6.185\sym{**} \\
                    &     (3.145)         &     (2.925)         &     (4.154)         &     (1.722)         &     (2.175)         &     (2.464)         &     (2.621)         &     (2.224)         &     (2.849)         \\
 Ratio population    &      0.0888         &       0.112         &       0.105         &      0.0601         &       0.125         &      0.0834         &       0.121         &       0.106         &       0.139         \\
                    &    (0.0573)         &    (0.0673)         &    (0.0653)         &    (0.0507)         &    (0.0775)         &    (0.0838)         &     (0.153)         &     (0.135)         &     (0.127)         \\
 Ratio fertility     &       0.125\sym{**} &       0.157\sym{***}&       0.139\sym{**} &      0.0775         &       0.158\sym{**} &       0.124\sym{*}  &       0.171\sym{*}  &       0.156\sym{**} &       0.152\sym{**} \\
                    &    (0.0556)         &    (0.0548)         &    (0.0547)         &    (0.0535)         &    (0.0688)         &    (0.0697)         &    (0.0842)         &    (0.0750)         &    (0.0735)         \\
 \midrule\multicolumn{10}{l}{\emph{Panel D. Donut specification}} \\ Abs. numbers        &       14.47\sym{***}&       13.02\sym{***}&       10.79\sym{**} &       6.000\sym{***}&       6.333\sym{***}&       5.163\sym{*}  &       8.467\sym{**} &       6.689\sym{**} &       5.622\sym{*}  \\
                    &     (3.545)         &     (3.269)         &     (4.495)         &     (1.789)         &     (2.192)         &     (2.567)         &     (3.049)         &     (2.598)         &     (3.205)         \\
 Ratio population    &       0.100         &      0.0866         &      0.0832         &      0.0855         &      0.0999         &      0.0657         &       0.126         &      0.0799         &       0.118         \\
                    &    (0.0669)         &    (0.0760)         &    (0.0751)         &    (0.0565)         &    (0.0842)         &    (0.0942)         &     (0.184)         &     (0.159)         &     (0.150)         \\
 Ratio fertility     &       0.118\sym{*}  &       0.136\sym{**} &       0.132\sym{**} &       0.109\sym{*}  &       0.151\sym{**} &       0.138\sym{*}  &       0.127         &       0.122         &       0.126         \\
                    &    (0.0616)         &    (0.0601)         &    (0.0600)         &    (0.0559)         &    (0.0693)         &    (0.0727)         &    (0.0949)         &    (0.0860)         &    (0.0836)         \\
 
\bottomrule \end{tabular} } \begin{tablenotes} \item \scriptsize \emph{Notes:} Clustered standard errors in parentheses. All regressions contain Birthmonth FE. Ratios indicate cases per thousand; either approximated population or original number of births. \end{tablenotes} \end{threeparttable} \end{table} 

%=========================================
 \begin{table}[H] \begin{threeparttable} \centering \caption{Robustness with respect to the inclusion of \texttt{fixed effects} and \texttt{covariates}} {\def\sym#1{\ifmmode^{#1}\else\(^{#1}\)\fi} \begin{tabular}{l*{7}{c}} \toprule & \multicolumn{6}{c}{Dependent variable: \textbf{Diseases of the musculoskeletal system and connective tissue}} \\ \cmidrule(lr){2-7}
            &\multicolumn{4}{c}{Average Causal Effects}         &\multicolumn{2}{c}{Heterogeneous Causal Effects}\\\cmidrule(lr){2-5}\cmidrule(lr){6-7}
            &\multicolumn{1}{c}{(1)}&\multicolumn{1}{c}{(2)}&\multicolumn{1}{c}{(3)}&\multicolumn{1}{c}{(4)}&\multicolumn{1}{c}{(5)}&\multicolumn{1}{c}{(6)}\\
            &\multicolumn{1}{c}{}&\multicolumn{1}{c}{}&\multicolumn{1}{c}{}&\multicolumn{1}{c}{}&\multicolumn{1}{c}{Women}&\multicolumn{1}{c}{Men}\\
\midrule
 \multicolumn{7}{l}{\emph{Panel A. 2 Month bandwidth}} \\ Abs. numbers        &       15.75         &       15.75\sym{**} &       15.75\sym{**} &       15.75\sym{**} &       10.25\sym{***}&       5.500\sym{**} \\
                    &     (18.98)         &     (4.767)         &     (5.091)         &     (5.110)         &     (2.501)         &     (2.318)         \\
 Ratio population    &      0.0239         &      0.0239         &      0.0239         &      0.0239         &       0.224         &      -0.135         \\
                    &     (0.217)         &     (0.121)         &     (0.129)         &     (0.130)         &     (0.130)         &     (0.368)         \\
 Ratio fertility     &       0.368\sym{*}  &       0.368\sym{***}&       0.368\sym{***}&       0.368\sym{**} &       0.664\sym{***}&      0.0878         \\
                    &     (0.156)         &    (0.0985)         &     (0.105)         &     (0.106)         &    (0.0470)         &     (0.149)         \\
 \midrule\multicolumn{7}{l}{\emph{Panel B. 4 Month bandwidth}} \\ Abs. numbers        &       15.31         &       15.31\sym{***}&       15.31\sym{***}&       15.31\sym{***}&       12.16\sym{***}&       3.150         \\
                    &     (13.75)         &     (4.771)         &     (4.924)         &     (4.933)         &     (3.274)         &     (1.999)         \\
 Ratio population    &      0.0165         &      0.0165         &      0.0165         &      0.0165         &       0.620\sym{*}  &      -0.625\sym{**} \\
                    &     (0.319)         &     (0.141)         &     (0.146)         &     (0.146)         &     (0.341)         &     (0.286)         \\
 Ratio fertility     &       0.215         &       0.215\sym{**} &       0.215\sym{**} &       0.215\sym{**} &       0.563\sym{***}&      -0.114         \\
                    &     (0.247)         &    (0.0834)         &    (0.0861)         &    (0.0864)         &     (0.141)         &     (0.139)         \\
 \midrule\multicolumn{7}{l}{\emph{Panel C. 6 Month bandwidth}} \\ Abs. numbers        &       25.06         &       25.06\sym{***}&       25.06\sym{***}&       25.22\sym{***}&       18.76\sym{***}&       6.296         \\
                    &     (15.63)         &     (7.216)         &     (7.364)         &     (7.336)         &     (4.428)         &     (4.796)         \\
 Ratio population    &      0.0469         &      0.0469         &      0.0469         &      0.0500         &       0.334         &      -0.209         \\
                    &     (0.269)         &     (0.147)         &     (0.150)         &     (0.153)         &     (0.242)         &     (0.320)         \\
 Ratio population    &      0.0984         &      0.0984\sym{*}  &      0.0984\sym{*}  &      0.0860         &       0.342\sym{***}&      -0.143         \\
                    &     (0.193)         &    (0.0535)         &    (0.0546)         &    (0.0557)         &    (0.0944)         &     (0.104)         \\
 \midrule\multicolumn{7}{l}{\emph{Panel D. Donut specification}} \\ Abs. numbers        &       29.16\sym{*}  &       29.16\sym{***}&       29.16\sym{***}&       29.41\sym{***}&       19.04\sym{***}&       10.11\sym{*}  \\
                    &     (16.68)         &     (7.974)         &     (8.171)         &     (8.165)         &     (4.580)         &     (5.340)         \\
 Ratio population    &       0.113         &       0.113         &       0.113         &       0.116         &       0.310         &     -0.0485         \\
                    &     (0.304)         &     (0.168)         &     (0.172)         &     (0.176)         &     (0.277)         &     (0.375)         \\
 Ratio population    &      0.0950         &      0.0950         &      0.0950         &      0.0802         &       0.324\sym{***}&      -0.132         \\
                    &     (0.220)         &    (0.0643)         &    (0.0659)         &    (0.0670)         &     (0.107)         &     (0.122)         \\
 \midrule Birthmonth FE       &                     &  \checkmark         &  \checkmark         &  \checkmark         &  \checkmark         &  \checkmark         \\
Year FE             &                     &                     &  \checkmark         &  \checkmark         &                     &                     \\
Covariates          &                     &                     &                     &  \checkmark         &                     &                     \\
 
\bottomrule \end{tabular} } \begin{tablenotes} \item \scriptsize \emph{Notes:} Clustered standard errors in parentheses. Personal covariates contain age and age squared. Ratios indicate cases per thousand; either approximated population or original number of births. \end{tablenotes} \end{threeparttable} \end{table} 

 \begin{table}[H] \begin{threeparttable} \centering \caption{Robustness with respect to the choice of \texttt{control group}} {\def\sym#1{\ifmmode^{#1}\else\(^{#1}\)\fi} \begin{tabular}{l*{10}{c}} \toprule & \multicolumn{9}{c}{Dependent variable: \textbf{Diseases of the musculoskeletal system and connective tissue}} \\ \cmidrule(lr){2-10}
            &\multicolumn{3}{c}{Average Causal Effects}&\multicolumn{3}{c}{Women}             &\multicolumn{3}{c}{Men}               \\\cmidrule(lr){2-4}\cmidrule(lr){5-7}\cmidrule(lr){8-10}
            &\multicolumn{1}{c}{(1)}&\multicolumn{1}{c}{(2)}&\multicolumn{1}{c}{(3)}&\multicolumn{1}{c}{(4)}&\multicolumn{1}{c}{(5)}&\multicolumn{1}{c}{(6)}&\multicolumn{1}{c}{(7)}&\multicolumn{1}{c}{(8)}&\multicolumn{1}{c}{(9)}\\
            &\multicolumn{1}{c}{C2}&\multicolumn{1}{c}{C1+C2}&\multicolumn{1}{c}{C1-C3}&\multicolumn{1}{c}{C2}&\multicolumn{1}{c}{C1+C2}&\multicolumn{1}{c}{C1-C3}&\multicolumn{1}{c}{C2}&\multicolumn{1}{c}{C1+C2}&\multicolumn{1}{c}{C1-C3}\\
\midrule
 \multicolumn{10}{l}{\emph{Panel A. 2 Month bandwidth}} \\ Abs. numbers        &       24.44\sym{**} &       17.83         &       15.98         &       21.50\sym{***}&       14.08\sym{**} &       9.852         &       2.944         &       3.750         &       6.130         \\
                    &     (8.119)         &     (13.46)         &     (18.33)         &     (1.635)         &     (5.147)         &     (8.527)         &     (6.544)         &     (9.580)         &     (11.30)         \\
 Ratio population    &      0.0239         &       0.160         &       0.199         &       0.224         &       0.242         &      0.0938         &      -0.135         &       0.116         &       0.352         \\
                    &     (0.121)         &     (0.243)         &     (0.391)         &     (0.130)         &     (0.208)         &     (0.459)         &     (0.368)         &     (0.362)         &     (0.475)         \\
 Ratio fertility     &       0.368\sym{***}&       0.356\sym{*}  &       0.338         &       0.664\sym{***}&       0.527\sym{***}&       0.408         &      0.0878         &       0.192         &       0.272         \\
                    &    (0.0985)         &     (0.186)         &     (0.321)         &    (0.0470)         &     (0.130)         &     (0.309)         &     (0.149)         &     (0.261)         &     (0.362)         \\
 \midrule\multicolumn{10}{l}{\emph{Panel B. 4 Month bandwidth}} \\ Abs. numbers        &       21.78\sym{**} &       13.69         &       8.648         &       21.44\sym{***}&       15.78\sym{**} &       11.54         &       0.333         &      -2.083         &      -2.889         \\
                    &     (7.643)         &     (10.52)         &     (12.71)         &     (6.120)         &     (6.234)         &     (7.043)         &     (4.047)         &     (5.968)         &     (7.380)         \\
 Ratio population    &      0.0165         &     -0.0302         &     -0.0188         &       0.620\sym{*}  &       0.499         &       0.349         &      -0.625\sym{**} &      -0.617\sym{*}  &      -0.431         \\
                    &     (0.141)         &     (0.204)         &     (0.270)         &     (0.341)         &     (0.292)         &     (0.352)         &     (0.286)         &     (0.323)         &     (0.396)         \\
 Ratio fertility     &       0.215\sym{**} &       0.192         &       0.129         &       0.563\sym{***}&       0.494\sym{***}&       0.373         &      -0.114         &     -0.0944         &      -0.102         \\
                    &    (0.0834)         &     (0.159)         &     (0.255)         &     (0.141)         &     (0.175)         &     (0.264)         &     (0.139)         &     (0.194)         &     (0.284)         \\
 \midrule\multicolumn{10}{l}{\emph{Panel C. 6 Month bandwidth}} \\ Abs. numbers        &       25.06\sym{***}&       19.05\sym{*}  &       12.55         &       18.76\sym{***}&       16.09\sym{***}&       10.62\sym{*}  &       6.296         &       2.954         &       1.926         \\
                    &     (7.216)         &     (9.407)         &     (11.27)         &     (4.428)         &     (4.747)         &     (5.352)         &     (4.796)         &     (5.792)         &     (7.181)         \\
 Ratio population    &      0.0469         &      0.0381         &     0.00944         &       0.334         &       0.293         &       0.148         &      -0.209         &      -0.198         &      -0.109         \\
                    &     (0.147)         &     (0.181)         &     (0.243)         &     (0.242)         &     (0.217)         &     (0.265)         &     (0.320)         &     (0.316)         &     (0.383)         \\
 Ratio fertility     &       0.193\sym{***}&       0.222         &       0.154         &       0.398\sym{***}&       0.433\sym{***}&       0.300         &     0.00241         &      0.0254         &      0.0186         \\
                    &    (0.0681)         &     (0.135)         &     (0.213)         &     (0.106)         &     (0.138)         &     (0.205)         &     (0.121)         &     (0.170)         &     (0.252)         \\
 \midrule\multicolumn{10}{l}{\emph{Panel D. Donut specification}} \\ Abs. numbers        &       29.16\sym{***}&       22.74\sym{**} &       14.30         &       19.04\sym{***}&       17.09\sym{***}&       10.50\sym{*}  &       10.11\sym{*}  &       5.656         &       3.800         \\
                    &     (7.974)         &     (10.76)         &     (12.67)         &     (4.580)         &     (5.486)         &     (6.006)         &     (5.340)         &     (6.437)         &     (8.086)         \\
 Ratio population    &       0.113         &      0.0368         &     -0.0312         &       0.310         &       0.263         &      0.0523         &     -0.0485         &      -0.174         &     -0.0900         \\
                    &     (0.168)         &     (0.193)         &     (0.258)         &     (0.277)         &     (0.241)         &     (0.284)         &     (0.375)         &     (0.359)         &     (0.439)         \\
 Ratio fertility     &       0.209\sym{**} &       0.235         &       0.154         &       0.368\sym{***}&       0.430\sym{**} &       0.277         &      0.0610         &      0.0515         &      0.0400         \\
                    &    (0.0812)         &     (0.156)         &     (0.241)         &     (0.114)         &     (0.164)         &     (0.231)         &     (0.137)         &     (0.189)         &     (0.286)         \\
 
\bottomrule \end{tabular} } \begin{tablenotes} \item \scriptsize \emph{Notes:} Clustered standard errors in parentheses. All regressions contain Birthmonth FE. Ratios indicate cases per thousand; either approximated population or original number of births. \end{tablenotes} \end{threeparttable} \end{table} 

%=========================================
 \begin{table}[H] \begin{threeparttable} \centering \caption{Robustness with respect to the inclusion of \texttt{fixed effects} and \texttt{covariates}} {\def\sym#1{\ifmmode^{#1}\else\(^{#1}\)\fi} \begin{tabular}{l*{7}{c}} \toprule & \multicolumn{6}{c}{Dependent variable: \textbf{Diseases of the genitourinary system}} \\ \cmidrule(lr){2-7}
            &\multicolumn{4}{c}{Average Causal Effects}         &\multicolumn{2}{c}{Heterogeneous Causal Effects}\\\cmidrule(lr){2-5}\cmidrule(lr){6-7}
            &\multicolumn{1}{c}{(1)}&\multicolumn{1}{c}{(2)}&\multicolumn{1}{c}{(3)}&\multicolumn{1}{c}{(4)}&\multicolumn{1}{c}{(5)}&\multicolumn{1}{c}{(6)}\\
            &\multicolumn{1}{c}{}&\multicolumn{1}{c}{}&\multicolumn{1}{c}{}&\multicolumn{1}{c}{}&\multicolumn{1}{c}{Women}&\multicolumn{1}{c}{Men}\\
\midrule
 \multicolumn{7}{l}{\emph{Panel A. 2 Month bandwidth}} \\ Abs. numbers        &       9.875         &       9.875\sym{**} &       9.875\sym{**} &       9.875\sym{**} &       11.68\sym{**} &      -1.800         \\
                    &     (23.52)         &     (3.675)         &     (3.925)         &     (3.939)         &     (3.383)         &     (3.697)         \\
 Ratio population    &      -0.156         &      -0.156         &      -0.156         &      -0.156         &      -0.129         &      -0.235\sym{***}\\
                    &     (0.198)         &     (0.151)         &     (0.161)         &     (0.163)         &     (0.418)         &    (0.0517)         \\
 Ratio fertility     &       0.205         &       0.205         &       0.205         &       0.205         &       0.518\sym{**} &     -0.0763         \\
                    &     (0.182)         &     (0.138)         &     (0.148)         &     (0.149)         &     (0.172)         &     (0.149)         \\
 \midrule\multicolumn{7}{l}{\emph{Panel B. 4 Month bandwidth}} \\ Abs. numbers        &       25.48         &       25.48\sym{**} &       25.48\sym{**} &       25.48\sym{**} &       17.99\sym{***}&       7.488         \\
                    &     (17.19)         &     (9.178)         &     (9.473)         &     (9.489)         &     (5.689)         &     (4.404)         \\
 Ratio population    &     -0.0721         &     -0.0721         &     -0.0721         &     -0.0721         &       0.366         &      -0.253         \\
                    &     (0.315)         &     (0.195)         &     (0.201)         &     (0.202)         &     (0.383)         &     (0.283)         \\
 Ratio population    &       0.195         &       0.195\sym{*}  &       0.195\sym{*}  &       0.195\sym{*}  &       0.314\sym{**} &      0.0993         \\
                    &     (0.244)         &    (0.0944)         &    (0.0974)         &    (0.0976)         &     (0.107)         &     (0.136)         \\
 \midrule\multicolumn{7}{l}{\emph{Panel C. 6 Month bandwidth}} \\ Abs. numbers        &       17.04         &       17.04\sym{**} &       17.04\sym{**} &       17.19\sym{**} &       11.22\sym{***}&       5.815         \\
                    &     (14.15)         &     (6.472)         &     (6.604)         &     (6.730)         &     (3.829)         &     (3.842)         \\
 Ratio population    &      -0.115         &      -0.115         &      -0.115         &      -0.108         &     -0.0950         &     -0.0739         \\
                    &     (0.258)         &     (0.166)         &     (0.170)         &     (0.176)         &     (0.303)         &     (0.218)         \\
 Ratio fertility     &      0.0450         &      0.0450         &      0.0450         &      0.0448         &      0.0374         &      0.0506         \\
                    &     (0.175)         &    (0.0788)         &    (0.0804)         &    (0.0828)         &     (0.108)         &    (0.0974)         \\
 \midrule\multicolumn{7}{l}{\emph{Panel D. Donut specification}} \\ Abs. numbers        &       23.71         &       23.71\sym{***}&       23.71\sym{***}&       24.64\sym{***}&       15.40\sym{***}&       8.310\sym{**} \\
                    &     (16.72)         &     (6.263)         &     (6.422)         &     (6.368)         &     (3.843)         &     (3.759)         \\
 Ratio population    &      -0.141         &      -0.141         &      -0.141         &      -0.138         &      -0.216         &     -0.0680         \\
                    &     (0.290)         &     (0.177)         &     (0.182)         &     (0.192)         &     (0.338)         &     (0.256)         \\
 Ratio population    &      0.0485         &      0.0485         &      0.0485         &      0.0442         &      0.0262         &      0.0694         \\
                    &     (0.192)         &    (0.0576)         &    (0.0590)         &    (0.0603)         &    (0.0875)         &     (0.101)         \\
 \midrule Birthmonth FE       &                     &  \checkmark         &  \checkmark         &  \checkmark         &  \checkmark         &  \checkmark         \\
Year FE             &                     &                     &  \checkmark         &  \checkmark         &                     &                     \\
Covariates          &                     &                     &                     &  \checkmark         &                     &                     \\
 
\bottomrule \end{tabular} } \begin{tablenotes} \item \scriptsize \emph{Notes:} Clustered standard errors in parentheses. Personal covariates contain age and age squared. Ratios indicate cases per thousand; either approximated population or original number of births. \end{tablenotes} \end{threeparttable} \end{table} 

 \begin{table}[H] \begin{threeparttable} \centering \caption{Robustness with respect to the choice of \texttt{control group}} {\def\sym#1{\ifmmode^{#1}\else\(^{#1}\)\fi} \begin{tabular}{l*{10}{c}} \toprule & \multicolumn{9}{c}{Dependent variable: \textbf{Diseases of the genitourinary system}} \\ \cmidrule(lr){2-10}
            &\multicolumn{3}{c}{Average Causal Effects}&\multicolumn{3}{c}{Women}             &\multicolumn{3}{c}{Men}               \\\cmidrule(lr){2-4}\cmidrule(lr){5-7}\cmidrule(lr){8-10}
            &\multicolumn{1}{c}{(1)}&\multicolumn{1}{c}{(2)}&\multicolumn{1}{c}{(3)}&\multicolumn{1}{c}{(4)}&\multicolumn{1}{c}{(5)}&\multicolumn{1}{c}{(6)}&\multicolumn{1}{c}{(7)}&\multicolumn{1}{c}{(8)}&\multicolumn{1}{c}{(9)}\\
            &\multicolumn{1}{c}{C2}&\multicolumn{1}{c}{C1+C2}&\multicolumn{1}{c}{C1-C3}&\multicolumn{1}{c}{C2}&\multicolumn{1}{c}{C1+C2}&\multicolumn{1}{c}{C1-C3}&\multicolumn{1}{c}{C2}&\multicolumn{1}{c}{C1+C2}&\multicolumn{1}{c}{C1-C3}\\
\midrule
 \multicolumn{10}{l}{\emph{Panel A. 2 Month bandwidth}} \\ Abs. numbers        &       12.61         &       2.333         &       2.630         &       15.50\sym{**} &       4.083         &       4.704         &      -2.889         &      -1.750         &      -2.074         \\
                    &     (7.649)         &     (8.501)         &     (7.271)         &     (4.974)         &     (6.184)         &     (4.728)         &     (4.633)         &     (4.660)         &     (4.214)         \\
 Ratio population    &      -0.156         &     -0.0841         &     0.00438         &      -0.129         &      -0.145         &      -0.178         &      -0.235\sym{***}&     -0.0707         &      0.0257         \\
                    &     (0.151)         &     (0.264)         &     (0.280)         &     (0.418)         &     (0.493)         &     (0.585)         &    (0.0517)         &     (0.174)         &     (0.165)         \\
 Ratio fertility     &       0.205         &       0.140         &       0.156         &       0.518\sym{**} &       0.310         &       0.333         &     -0.0763         &    -0.00267         &     0.00149         \\
                    &     (0.138)         &     (0.149)         &     (0.168)         &     (0.172)         &     (0.202)         &     (0.228)         &     (0.149)         &     (0.151)         &     (0.148)         \\
 \midrule\multicolumn{10}{l}{\emph{Panel B. 4 Month bandwidth}} \\ Abs. numbers        &       17.72\sym{*}  &       13.61         &       9.694         &       13.08\sym{**} &       6.069         &       5.444         &       4.639         &       7.542         &       4.250         \\
                    &     (8.907)         &     (8.552)         &     (7.314)         &     (4.511)         &     (4.465)         &     (3.464)         &     (5.698)         &     (5.547)         &     (4.999)         \\
 Ratio population    &     -0.0721         &     -0.0612         &     -0.0196         &       0.366         &       0.200         &       0.148         &      -0.253         &      -0.114         &     -0.0902         \\
                    &     (0.195)         &     (0.216)         &     (0.208)         &     (0.383)         &     (0.361)         &     (0.365)         &     (0.283)         &     (0.271)         &     (0.269)         \\
 Ratio fertility     &       0.132         &       0.169         &       0.130         &       0.246\sym{*}  &       0.190         &       0.190         &      0.0443         &       0.176         &      0.0959         \\
                    &     (0.109)         &     (0.125)         &     (0.146)         &     (0.122)         &     (0.151)         &     (0.176)         &     (0.146)         &     (0.150)         &     (0.154)         \\
 \midrule\multicolumn{10}{l}{\emph{Panel C. 6 Month bandwidth}} \\ Abs. numbers        &       17.04\sym{**} &       17.73\sym{***}&       10.51\sym{*}  &       11.22\sym{***}&       9.000\sym{**} &       5.809\sym{*}  &       5.815         &       8.731\sym{**} &       4.704         \\
                    &     (6.472)         &     (6.389)         &     (5.697)         &     (3.829)         &     (3.801)         &     (3.196)         &     (3.842)         &     (3.756)         &     (3.471)         \\
 Ratio population    &      -0.115         &     -0.0195         &     -0.0441         &     -0.0950         &     -0.0996         &      -0.152         &     -0.0739         &      0.0785         &      0.0304         \\
                    &     (0.166)         &     (0.172)         &     (0.171)         &     (0.303)         &     (0.279)         &     (0.288)         &     (0.218)         &     (0.205)         &     (0.204)         \\
 Ratio fertility     &      0.0450         &       0.172\sym{*}  &       0.109         &      0.0374         &       0.144         &       0.117         &      0.0506         &       0.199\sym{*}  &       0.104         \\
                    &    (0.0788)         &    (0.0972)         &     (0.116)         &     (0.108)         &     (0.130)         &     (0.146)         &    (0.0974)         &     (0.104)         &     (0.112)         \\
 \midrule\multicolumn{10}{l}{\emph{Panel D. Donut specification}} \\ Abs. numbers        &       15.31\sym{**} &       17.52\sym{**} &       9.504         &       10.09\sym{***}&       9.078\sym{**} &       5.230         &       5.222         &       8.444\sym{*}  &       4.274         \\
                    &     (5.602)         &     (6.755)         &     (6.038)         &     (2.958)         &     (3.857)         &     (3.314)         &     (4.231)         &     (4.324)         &     (3.939)         \\
 Ratio population    &      -0.141         &     -0.0821         &      -0.131         &      -0.216         &      -0.217         &      -0.340         &     -0.0680         &      0.0384         &    -0.00501         \\
                    &     (0.177)         &     (0.175)         &     (0.169)         &     (0.338)         &     (0.291)         &     (0.291)         &     (0.256)         &     (0.236)         &     (0.234)         \\
 Ratio fertility     &     -0.0287         &       0.125         &      0.0668         &     -0.0613         &      0.0923         &      0.0674         &     0.00330         &       0.162         &      0.0733         \\
                    &    (0.0680)         &     (0.110)         &     (0.127)         &    (0.0979)         &     (0.148)         &     (0.159)         &     (0.108)         &     (0.121)         &     (0.128)         \\
 
\bottomrule \end{tabular} } \begin{tablenotes} \item \scriptsize \emph{Notes:} Clustered standard errors in parentheses. All regressions contain Birthmonth FE. Ratios indicate cases per thousand; either approximated population or original number of births. \end{tablenotes} \end{threeparttable} \end{table} 

%=========================================
 \begin{table}[H] \begin{threeparttable} \centering \caption{Robustness with respect to the inclusion of \texttt{fixed effects} and \texttt{covariates}} {\def\sym#1{\ifmmode^{#1}\else\(^{#1}\)\fi} \begin{tabular}{l*{7}{c}} \toprule & \multicolumn{6}{c}{Dependent variable: \textbf{Symptoms, signs and abnormal clinical and laboratory findings, not elsewhere classified}} \\ \cmidrule(lr){2-7}
            &\multicolumn{4}{c}{Average Causal Effects}         &\multicolumn{2}{c}{Heterogeneous Causal Effects}\\\cmidrule(lr){2-5}\cmidrule(lr){6-7}
            &\multicolumn{1}{c}{(1)}&\multicolumn{1}{c}{(2)}&\multicolumn{1}{c}{(3)}&\multicolumn{1}{c}{(4)}&\multicolumn{1}{c}{(5)}&\multicolumn{1}{c}{(6)}\\
            &\multicolumn{1}{c}{}&\multicolumn{1}{c}{}&\multicolumn{1}{c}{}&\multicolumn{1}{c}{}&\multicolumn{1}{c}{Women}&\multicolumn{1}{c}{Men}\\
\midrule
 \multicolumn{7}{l}{\emph{Panel A. 2 Month bandwidth}} \\ Abs. numbers        &       10.85         &       10.85\sym{***}&       10.85\sym{***}&       10.85\sym{***}&       8.925\sym{***}&       1.925\sym{*}  \\
                    &     (15.66)         &     (1.292)         &     (1.380)         &     (1.385)         &     (2.106)         &     (0.815)         \\
 Ratio fertility     &       0.169         &       0.169\sym{***}&       0.169\sym{***}&       0.169\sym{***}&       0.292\sym{***}&      0.0567         \\
                    &     (0.118)         &    (0.0118)         &    (0.0126)         &    (0.0127)         &    (0.0583)         &    (0.0450)         \\
 Ratio population    &       0.154\sym{*}  &       0.154\sym{***}&       0.154\sym{***}&       0.154\sym{***}&       0.331\sym{***}&     -0.0209         \\
                    &    (0.0753)         &    (0.0266)         &    (0.0284)         &    (0.0286)         &    (0.0763)         &    (0.0499)         \\
 \midrule\multicolumn{7}{l}{\emph{Panel B. 4 Month bandwidth}} \\ Abs. numbers        &       6.278         &       6.278\sym{**} &       6.278\sym{*}  &       6.278\sym{*}  &       6.639\sym{**} &      -0.361         \\
                    &     (9.685)         &     (2.911)         &     (3.002)         &     (3.014)         &     (2.303)         &     (2.470)         \\
 Ratio population    &      -0.143         &      -0.143\sym{*}  &      -0.143\sym{*}  &      -0.143\sym{*}  &       0.171         &      -0.386\sym{**} \\
                    &     (0.157)         &    (0.0712)         &    (0.0735)         &    (0.0738)         &     (0.151)         &     (0.141)         \\
 Ratio fertility     &     0.00437         &     0.00437         &     0.00437         &     0.00437         &       0.113         &     -0.0930         \\
                    &     (0.137)         &    (0.0558)         &    (0.0575)         &    (0.0578)         &    (0.0935)         &    (0.0586)         \\
 \midrule\multicolumn{7}{l}{\emph{Panel C. 6 Month bandwidth}} \\ Abs. numbers        &       10.53         &       10.53\sym{***}&       10.53\sym{***}&       10.66\sym{***}&       7.075\sym{***}&       3.458\sym{**} \\
                    &     (9.059)         &     (2.512)         &     (2.565)         &     (2.581)         &     (2.003)         &     (1.557)         \\
 Ratio population    &      -0.101         &      -0.101         &      -0.101         &     -0.0974         &      -0.119         &     -0.0401         \\
                    &     (0.145)         &    (0.0846)         &    (0.0863)         &    (0.0896)         &     (0.137)         &     (0.189)         \\
 Ratio population    &      0.0328         &      0.0328         &      0.0328         &      0.0343         &      0.0395         &      0.0262         \\
                    &     (0.107)         &    (0.0336)         &    (0.0343)         &    (0.0349)         &    (0.0739)         &    (0.0565)         \\
 \midrule\multicolumn{7}{l}{\emph{Panel D. Donut specification}} \\ Abs. numbers        &       10.92         &       10.92\sym{***}&       10.92\sym{***}&       11.06\sym{***}&       7.450\sym{***}&       3.470\sym{*}  \\
                    &     (10.39)         &     (2.851)         &     (2.924)         &     (2.950)         &     (2.349)         &     (1.815)         \\
 Ratio fertility     &    -0.00756         &    -0.00756         &    -0.00756         &    -0.00686         &     0.00825         &     -0.0210         \\
                    &     (0.108)         &    (0.0515)         &    (0.0528)         &    (0.0530)         &    (0.0910)         &    (0.0549)         \\
 Ratio population    &     0.00821         &     0.00821         &     0.00821         &     0.00937         &     0.00707         &      0.0102         \\
                    &     (0.113)         &    (0.0374)         &    (0.0383)         &    (0.0390)         &    (0.0834)         &    (0.0665)         \\
 \midrule Birthmonth FE       &                     &  \checkmark         &  \checkmark         &  \checkmark         &  \checkmark         &  \checkmark         \\
Year FE             &                     &                     &  \checkmark         &  \checkmark         &                     &                     \\
Covariates          &                     &                     &                     &  \checkmark         &                     &                     \\
 
\bottomrule \end{tabular} } \begin{tablenotes} \item \scriptsize \emph{Notes:} Clustered standard errors in parentheses. Personal covariates contain age and age squared. Ratios indicate cases per thousand; either approximated population or original number of births. \end{tablenotes} \end{threeparttable} \end{table} 

 \begin{table}[H] \begin{threeparttable} \centering \caption{Robustness with respect to the choice of \texttt{control group}} {\def\sym#1{\ifmmode^{#1}\else\(^{#1}\)\fi} \begin{tabular}{l*{10}{c}} \toprule & \multicolumn{9}{c}{Dependent variable: \textbf{Symptoms, signs and abnormal clinical and laboratory findings, not elsewhere classified}} \\ \cmidrule(lr){2-10}
            &\multicolumn{3}{c}{Average Causal Effects}&\multicolumn{3}{c}{Women}             &\multicolumn{3}{c}{Men}               \\\cmidrule(lr){2-4}\cmidrule(lr){5-7}\cmidrule(lr){8-10}
            &\multicolumn{1}{c}{(1)}&\multicolumn{1}{c}{(2)}&\multicolumn{1}{c}{(3)}&\multicolumn{1}{c}{(4)}&\multicolumn{1}{c}{(5)}&\multicolumn{1}{c}{(6)}&\multicolumn{1}{c}{(7)}&\multicolumn{1}{c}{(8)}&\multicolumn{1}{c}{(9)}\\
            &\multicolumn{1}{c}{C2}&\multicolumn{1}{c}{C1+C2}&\multicolumn{1}{c}{C1-C3}&\multicolumn{1}{c}{C2}&\multicolumn{1}{c}{C1+C2}&\multicolumn{1}{c}{C1-C3}&\multicolumn{1}{c}{C2}&\multicolumn{1}{c}{C1+C2}&\multicolumn{1}{c}{C1-C3}\\
\midrule
 \multicolumn{10}{l}{\emph{Panel A. 2 Month bandwidth}} \\ Abs. numbers        &       12.06\sym{***}&       10.28\sym{**} &       7.500\sym{*}  &       10.50\sym{***}&       9.000\sym{*}  &       6.000         &       1.556         &       1.278         &       1.500         \\
                    &     (1.330)         &     (3.453)         &     (3.558)         &     (2.809)         &     (4.202)         &     (4.667)         &     (2.113)         &     (2.833)         &     (2.479)         \\
 Ratio population    &     -0.0625         &      0.0718         &      0.0869         &    -0.00162         &       0.140         &      0.0222         &     -0.0888         &      0.0340         &       0.148         \\
                    &    (0.0403)         &     (0.168)         &     (0.144)         &     (0.167)         &     (0.309)         &     (0.317)         &     (0.126)         &     (0.134)         &     (0.144)         \\
 Ratio fertility     &       0.188\sym{***}&       0.218\sym{**} &       0.189\sym{**} &       0.343\sym{**} &       0.368\sym{**} &       0.288\sym{*}  &      0.0438         &      0.0801         &      0.0978         \\
                    &    (0.0279)         &    (0.0776)         &    (0.0671)         &     (0.101)         &     (0.163)         &     (0.139)         &    (0.0417)         &    (0.0599)         &    (0.0758)         \\
 \midrule\multicolumn{10}{l}{\emph{Panel B. 4 Month bandwidth}} \\ Abs. numbers        &       6.278\sym{**} &       6.500         &       4.796         &       6.639\sym{**} &       6.917\sym{*}  &       4.972         &      -0.361         &      -0.417         &      -0.176         \\
                    &     (2.911)         &     (4.287)         &     (4.891)         &     (2.303)         &     (3.810)         &     (4.477)         &     (2.470)         &     (2.839)         &     (2.610)         \\
 Ratio population    &      -0.143\sym{*}  &     -0.0847         &     -0.0373         &       0.171         &       0.216         &       0.148         &      -0.386\sym{**} &      -0.337\sym{**} &      -0.198         \\
                    &    (0.0712)         &     (0.123)         &     (0.100)         &     (0.151)         &     (0.224)         &     (0.221)         &     (0.141)         &     (0.140)         &     (0.150)         \\
 Ratio fertility     &     0.00437         &      0.0701         &      0.0629         &       0.113         &       0.199\sym{*}  &       0.161\sym{*}  &     -0.0930         &     -0.0448         &     -0.0238         \\
                    &    (0.0558)         &    (0.0671)         &    (0.0574)         &    (0.0935)         &     (0.113)         &    (0.0932)         &    (0.0586)         &    (0.0888)         &    (0.0922)         \\
 \midrule\multicolumn{10}{l}{\emph{Panel C. 6 Month bandwidth}} \\ Abs. numbers        &       10.20\sym{***}&       6.602\sym{*}  &       3.907         &       4.148\sym{**} &       2.546         &       0.481         &       6.056\sym{**} &       4.056         &       3.426         \\
                    &     (2.633)         &     (3.543)         &     (3.954)         &     (1.833)         &     (3.050)         &     (3.577)         &     (2.902)         &     (2.750)         &     (2.462)         \\
 Ratio population    &      -0.101         &      -0.108         &     -0.0822         &      -0.119         &      -0.144         &      -0.170         &     -0.0401         &     -0.0457         &      0.0200         \\
                    &    (0.0846)         &    (0.0987)         &    (0.0850)         &     (0.137)         &     (0.192)         &     (0.187)         &     (0.189)         &     (0.155)         &     (0.151)         \\
 Ratio fertility     &     0.00825         &      0.0247         &      0.0218         &     -0.0494         &     -0.0160         &     -0.0279         &      0.0639         &      0.0641         &      0.0709         \\
                    &    (0.0399)         &    (0.0537)         &    (0.0472)         &    (0.0818)         &     (0.106)         &    (0.0905)         &    (0.0726)         &    (0.0754)         &    (0.0773)         \\
 \midrule\multicolumn{10}{l}{\emph{Panel D. Donut specification}} \\ Abs. numbers        &       10.51\sym{***}&       5.056         &       2.526         &       3.822\sym{**} &       0.978         &      -0.941         &       6.689\sym{*}  &       4.078         &       3.467         \\
                    &     (2.952)         &     (3.842)         &     (4.460)         &     (1.704)         &     (2.803)         &     (3.725)         &     (3.472)         &     (3.247)         &     (2.883)         \\
 Ratio population    &     -0.0946         &      -0.174\sym{*}  &      -0.153\sym{**} &      -0.170         &      -0.258         &      -0.308         &     0.00686         &     -0.0762         &    0.000944         \\
                    &    (0.0990)         &    (0.0884)         &    (0.0753)         &     (0.154)         &     (0.186)         &     (0.189)         &     (0.226)         &     (0.181)         &     (0.177)         \\
 Ratio fertility     &     -0.0192         &     -0.0282         &     -0.0171         &     -0.0953         &     -0.0955         &     -0.0912         &      0.0547         &      0.0375         &      0.0559         \\
                    &    (0.0453)         &    (0.0528)         &    (0.0475)         &    (0.0921)         &     (0.102)         &    (0.0876)         &    (0.0844)         &    (0.0840)         &    (0.0876)         \\
 
\bottomrule \end{tabular} } \begin{tablenotes} \item \scriptsize \emph{Notes:} Clustered standard errors in parentheses. All regressions contain Birthmonth FE. Ratios indicate cases per thousand; either approximated population or original number of births. \end{tablenotes} \end{threeparttable} \end{table} 

%=========================================
 \begin{table}[H] \begin{threeparttable} \centering \caption{Robustness with respect to the inclusion of \texttt{fixed effects} and \texttt{covariates}} {\def\sym#1{\ifmmode^{#1}\else\(^{#1}\)\fi} \begin{tabular}{l*{7}{c}} \toprule & \multicolumn{6}{c}{Dependent variable: \textbf{External causes of morbidity and mortality}} \\ \cmidrule(lr){2-7}
            &\multicolumn{4}{c}{Average Causal Effects}         &\multicolumn{2}{c}{Heterogeneous Causal Effects}\\\cmidrule(lr){2-5}\cmidrule(lr){6-7}
            &\multicolumn{1}{c}{(1)}&\multicolumn{1}{c}{(2)}&\multicolumn{1}{c}{(3)}&\multicolumn{1}{c}{(4)}&\multicolumn{1}{c}{(5)}&\multicolumn{1}{c}{(6)}\\
            &\multicolumn{1}{c}{}&\multicolumn{1}{c}{}&\multicolumn{1}{c}{}&\multicolumn{1}{c}{}&\multicolumn{1}{c}{Women}&\multicolumn{1}{c}{Men}\\
\midrule
 \multicolumn{7}{l}{\emph{Panel A. 2 Month bandwidth}} \\ Abs. numbers        &       1.225         &       1.225         &       1.225         &       1.225         &      -1.775         &       3.000         \\
                    &     (38.35)         &     (9.292)         &     (9.925)         &     (9.962)         &     (2.881)         &     (6.841)         \\
 Ratio population    &      -0.831\sym{*}  &      -0.831\sym{***}&      -0.831\sym{***}&      -0.831\sym{***}&      -0.572\sym{***}&      -1.022\sym{*}  \\
                    &     (0.366)         &     (0.218)         &     (0.233)         &     (0.235)         &     (0.162)         &     (0.475)         \\
 Ratio fertility     &      -0.276         &      -0.276\sym{*}  &      -0.276         &      -0.276         &      -0.103         &      -0.454\sym{**} \\
                    &     (0.313)         &     (0.144)         &     (0.154)         &     (0.155)         &     (0.205)         &     (0.168)         \\
 \midrule\multicolumn{7}{l}{\emph{Panel B. 4 Month bandwidth}} \\ Abs. numbers        &       1.400         &       1.400         &       1.400         &       1.400         &       1.913         &      -0.513         \\
                    &     (28.03)         &     (8.328)         &     (8.595)         &     (8.610)         &     (4.213)         &     (4.751)         \\
 Ratio fertility     &      -0.236         &      -0.236\sym{*}  &      -0.236\sym{*}  &      -0.236\sym{*}  &     -0.0844         &      -0.406         \\
                    &     (0.391)         &     (0.129)         &     (0.133)         &     (0.133)         &    (0.0820)         &     (0.243)         \\
 Ratio fertility     &      -0.337         &      -0.337\sym{***}&      -0.337\sym{***}&      -0.337\sym{***}&      -0.133         &      -0.553\sym{***}\\
                    &     (0.307)         &    (0.0799)         &    (0.0824)         &    (0.0828)         &     (0.147)         &     (0.143)         \\
 \midrule\multicolumn{7}{l}{\emph{Panel C. 6 Month bandwidth}} \\ Abs. numbers        &       10.91         &       10.91         &       10.91         &       9.840         &       3.833         &       7.074         \\
                    &     (19.67)         &     (8.414)         &     (8.586)         &     (8.512)         &     (4.066)         &     (5.478)         \\
 Ratio population    &      -0.383         &      -0.383\sym{*}  &      -0.383\sym{*}  &      -0.388\sym{*}  &      -0.197         &      -0.585         \\
                    &     (0.370)         &     (0.201)         &     (0.205)         &     (0.214)         &     (0.254)         &     (0.584)         \\
 Ratio fertility     &      -0.159         &      -0.159\sym{*}  &      -0.159         &      -0.175\sym{*}  &      -0.104         &      -0.209         \\
                    &     (0.268)         &    (0.0916)         &    (0.0935)         &    (0.0870)         &     (0.106)         &     (0.153)         \\
 \midrule\multicolumn{7}{l}{\emph{Panel D. Donut specification}} \\ Abs. numbers        &       23.01         &       23.01\sym{***}&       23.01\sym{***}&       22.16\sym{***}&       10.37\sym{***}&       12.64\sym{**} \\
                    &     (30.51)         &     (6.943)         &     (7.119)         &     (6.846)         &     (3.166)         &     (5.051)         \\
 Ratio population    &      -0.213         &      -0.213         &      -0.213         &      -0.218         &      -0.109         &      -0.256         \\
                    &     (0.409)         &     (0.214)         &     (0.219)         &     (0.231)         &     (0.300)         &     (0.673)         \\
 Ratio population    &      -0.132         &      -0.132         &      -0.132         &      -0.150         &     -0.0456         &      -0.214         \\
                    &     (0.299)         &     (0.116)         &     (0.119)         &     (0.113)         &     (0.103)         &     (0.224)         \\
 \midrule Birthmonth FE       &                     &  \checkmark         &  \checkmark         &  \checkmark         &  \checkmark         &  \checkmark         \\
Year FE             &                     &                     &  \checkmark         &  \checkmark         &                     &                     \\
Covariates          &                     &                     &                     &  \checkmark         &                     &                     \\
 
\bottomrule \end{tabular} } \begin{tablenotes} \item \scriptsize \emph{Notes:} Clustered standard errors in parentheses. Personal covariates contain age and age squared. Ratios indicate cases per thousand; either approximated population or original number of births. \end{tablenotes} \end{threeparttable} \end{table} 

 \begin{table}[H] \begin{threeparttable} \centering \caption{Robustness with respect to the choice of \texttt{control group}} {\def\sym#1{\ifmmode^{#1}\else\(^{#1}\)\fi} \begin{tabular}{l*{10}{c}} \toprule & \multicolumn{9}{c}{Dependent variable: \textbf{External causes of morbidity and mortality}} \\ \cmidrule(lr){2-10}
            &\multicolumn{3}{c}{Average Causal Effects}&\multicolumn{3}{c}{Women}             &\multicolumn{3}{c}{Men}               \\\cmidrule(lr){2-4}\cmidrule(lr){5-7}\cmidrule(lr){8-10}
            &\multicolumn{1}{c}{(1)}&\multicolumn{1}{c}{(2)}&\multicolumn{1}{c}{(3)}&\multicolumn{1}{c}{(4)}&\multicolumn{1}{c}{(5)}&\multicolumn{1}{c}{(6)}&\multicolumn{1}{c}{(7)}&\multicolumn{1}{c}{(8)}&\multicolumn{1}{c}{(9)}\\
            &\multicolumn{1}{c}{C2}&\multicolumn{1}{c}{C1+C2}&\multicolumn{1}{c}{C1-C3}&\multicolumn{1}{c}{C2}&\multicolumn{1}{c}{C1+C2}&\multicolumn{1}{c}{C1-C3}&\multicolumn{1}{c}{C2}&\multicolumn{1}{c}{C1+C2}&\multicolumn{1}{c}{C1-C3}\\
\midrule
 \multicolumn{10}{l}{\emph{Panel A. 2 Month bandwidth}} \\ Abs. numbers        &      -22.17\sym{**} &      -18.14\sym{*}  &      -17.96         &      -5.056         &      -6.722         &      -5.259         &      -17.11\sym{***}&      -11.42         &      -12.70         \\
                    &     (8.367)         &     (9.857)         &     (21.17)         &     (6.481)         &     (4.713)         &     (7.229)         &     (3.168)         &     (7.864)         &     (15.77)         \\
 Ratio population    &      -0.831\sym{***}&      -0.445         &      -0.288         &      -0.572\sym{***}&      -0.387         &      -0.367         &      -1.022\sym{*}  &      -0.427         &     0.00765         \\
                    &     (0.218)         &     (0.316)         &     (0.270)         &     (0.162)         &     (0.243)         &     (0.227)         &     (0.475)         &     (0.514)         &     (0.556)         \\
 Ratio fertility     &      -0.276\sym{*}  &      -0.102         &     -0.0708         &      -0.103         &     -0.0645         &    -0.00999         &      -0.454\sym{**} &      -0.158         &      -0.143         \\
                    &     (0.144)         &     (0.138)         &     (0.212)         &     (0.205)         &     (0.147)         &     (0.188)         &     (0.168)         &     (0.238)         &     (0.331)         \\
 \midrule\multicolumn{10}{l}{\emph{Panel B. 4 Month bandwidth}} \\ Abs. numbers        &      -10.11         &      -8.667         &      -12.12         &      -1.167         &      -1.569         &       0.204         &      -8.944\sym{**} &      -7.097         &      -12.32         \\
                    &     (8.221)         &     (9.249)         &     (17.10)         &     (5.677)         &     (4.895)         &     (6.275)         &     (3.617)         &     (5.786)         &     (12.12)         \\
 Ratio population    &      -0.650\sym{***}&      -0.528\sym{**} &      -0.399\sym{*}  &     -0.0474         &     -0.0327         &      0.0169         &      -1.619\sym{**} &      -1.362\sym{**} &      -1.028         \\
                    &     (0.186)         &     (0.244)         &     (0.230)         &     (0.363)         &     (0.306)         &     (0.293)         &     (0.614)         &     (0.619)         &     (0.670)         \\
 Ratio fertility     &      -0.337\sym{***}&      -0.193\sym{*}  &      -0.190         &      -0.133         &     -0.0476         &      0.0296         &      -0.553\sym{***}&      -0.361\sym{*}  &      -0.424\sym{*}  \\
                    &    (0.0799)         &    (0.0971)         &     (0.126)         &     (0.147)         &     (0.118)         &     (0.127)         &     (0.143)         &     (0.179)         &     (0.222)         \\
 \midrule\multicolumn{10}{l}{\emph{Panel C. 6 Month bandwidth}} \\ Abs. numbers        &       10.91         &       9.565         &       5.000         &       3.833         &       1.954         &       1.426         &       7.074         &       7.611         &       3.574         \\
                    &     (8.414)         &     (8.601)         &     (14.02)         &     (4.066)         &     (3.734)         &     (4.775)         &     (5.478)         &     (6.057)         &     (10.36)         \\
 Ratio population    &      -0.383\sym{*}  &      -0.298         &      -0.204         &      -0.197         &      -0.229         &      -0.192         &      -0.585         &      -0.356         &      -0.154         \\
                    &     (0.201)         &     (0.209)         &     (0.185)         &     (0.254)         &     (0.219)         &     (0.216)         &     (0.584)         &     (0.552)         &     (0.553)         \\
 Ratio fertility     &      -0.159\sym{*}  &     -0.0310         &    0.000380         &      -0.104         &     -0.0559         &    -0.00398         &      -0.209         &    -0.00799         &   -0.000176         \\
                    &    (0.0916)         &    (0.0995)         &     (0.117)         &     (0.106)         &    (0.0888)         &    (0.0919)         &     (0.153)         &     (0.179)         &     (0.216)         \\
 \midrule\multicolumn{10}{l}{\emph{Panel D. Donut specification}} \\ Abs. numbers        &       21.31\sym{**} &       15.54\sym{*}  &       11.16         &       8.889\sym{**} &       5.222         &       4.978         &       12.42\sym{**} &       10.32         &       6.185         \\
                    &     (7.658)         &     (9.007)         &     (14.83)         &     (3.811)         &     (3.885)         &     (4.930)         &     (5.531)         &     (6.497)         &     (11.20)         \\
 Ratio population    &      -0.213         &      -0.285         &      -0.211         &      -0.109         &      -0.221         &      -0.214         &      -0.256         &      -0.306         &     -0.0817         \\
                    &     (0.214)         &     (0.216)         &     (0.197)         &     (0.300)         &     (0.245)         &     (0.250)         &     (0.673)         &     (0.635)         &     (0.638)         \\
 Ratio fertility     &     -0.0831         &     -0.0126         &      0.0480         &    -0.00208         &   -0.000699         &      0.0745         &      -0.160         &     -0.0296         &      0.0153         \\
                    &     (0.101)         &     (0.109)         &     (0.124)         &     (0.114)         &    (0.0969)         &    (0.0942)         &     (0.182)         &     (0.201)         &     (0.240)         \\
 
\bottomrule \end{tabular} } \begin{tablenotes} \item \scriptsize \emph{Notes:} Clustered standard errors in parentheses. All regressions contain Birthmonth FE. Ratios indicate cases per thousand; either approximated population or original number of births. \end{tablenotes} \end{threeparttable} \end{table} 

%=========================================
\section{Einzeldiagnosen}
 \begin{table}[H] \begin{threeparttable} \centering \caption{Robustness with respect to the inclusion of \texttt{fixed effects} and \texttt{covariates}} {\def\sym#1{\ifmmode^{#1}\else\(^{#1}\)\fi} \begin{tabular}{l*{7}{c}} \toprule & \multicolumn{6}{c}{Dependent variable: \textbf{Injuries}} \\ \cmidrule(lr){2-7}
            &\multicolumn{4}{c}{Average Causal Effects}         &\multicolumn{2}{c}{Heterogeneous Causal Effects}\\\cmidrule(lr){2-5}\cmidrule(lr){6-7}
            &\multicolumn{1}{c}{(1)}&\multicolumn{1}{c}{(2)}&\multicolumn{1}{c}{(3)}&\multicolumn{1}{c}{(4)}&\multicolumn{1}{c}{(5)}&\multicolumn{1}{c}{(6)}\\
            &\multicolumn{1}{c}{}&\multicolumn{1}{c}{}&\multicolumn{1}{c}{}&\multicolumn{1}{c}{}&\multicolumn{1}{c}{Women}&\multicolumn{1}{c}{Men}\\
\midrule
 \multicolumn{7}{l}{\emph{Panel A. 2 Month bandwidth}} \\ Abs. numbers        &      -21.17         &      -21.17\sym{**} &      -21.17\sym{**} &      -21.17\sym{**} &      -5.667         &      -15.50\sym{***}\\
                    &     (22.75)         &     (7.427)         &     (7.923)         &     (7.992)         &     (4.474)         &     (3.226)         \\
 Ratio population    &      -0.712\sym{**} &      -0.712\sym{***}&      -0.712\sym{***}&      -0.712\sym{***}&      -0.470\sym{***}&      -0.900\sym{*}  \\
                    &     (0.283)         &     (0.170)         &     (0.181)         &     (0.183)         &    (0.0796)         &     (0.463)         \\
 Ratio population    &      -0.291         &      -0.291\sym{**} &      -0.291\sym{*}  &      -0.291\sym{*}  &      -0.142         &      -0.456\sym{**} \\
                    &     (0.272)         &     (0.122)         &     (0.130)         &     (0.131)         &     (0.143)         &     (0.135)         \\
 \midrule\multicolumn{7}{l}{\emph{Panel B. 4 Month bandwidth}} \\ Abs. numbers        &      -8.889         &      -8.889         &      -8.889         &      -8.889         &      -1.389         &      -7.500\sym{*}  \\
                    &     (16.86)         &     (7.504)         &     (7.739)         &     (7.770)         &     (4.207)         &     (3.887)         \\
 Ratio fertility     &      -0.280         &      -0.280\sym{***}&      -0.280\sym{***}&      -0.280\sym{***}&      -0.109         &      -0.464\sym{***}\\
                    &     (0.252)         &    (0.0628)         &    (0.0648)         &    (0.0650)         &     (0.108)         &    (0.0916)         \\
 Ratio population    &      -0.301         &      -0.301\sym{***}&      -0.301\sym{***}&      -0.301\sym{***}&      -0.114         &      -0.512\sym{***}\\
                    &     (0.270)         &    (0.0674)         &    (0.0695)         &    (0.0697)         &     (0.112)         &     (0.101)         \\
 \midrule\multicolumn{7}{l}{\emph{Panel C. 6 Month bandwidth}} \\ Abs. numbers        &       2.685         &       2.685         &       2.685         &       1.781         &       0.611         &       2.074         \\
                    &     (16.20)         &     (6.193)         &     (6.319)         &     (6.273)         &     (2.900)         &     (3.968)         \\
 Ratio population    &      -0.402         &      -0.402\sym{**} &      -0.402\sym{**} &      -0.407\sym{**} &      -0.212         &      -0.625         \\
                    &     (0.291)         &     (0.149)         &     (0.152)         &     (0.156)         &     (0.176)         &     (0.462)         \\
 Ratio population    &      -0.243         &      -0.243\sym{***}&      -0.243\sym{***}&      -0.257\sym{***}&      -0.148\sym{*}  &      -0.333\sym{***}\\
                    &     (0.230)         &    (0.0635)         &    (0.0648)         &    (0.0623)         &    (0.0798)         &     (0.101)         \\
 \midrule\multicolumn{7}{l}{\emph{Panel D. Donut specification}} \\ Abs. numbers        &       11.22         &       11.22\sym{**} &       11.22\sym{*}  &       10.35\sym{*}  &       4.133         &       7.089\sym{*}  \\
                    &     (18.47)         &     (5.239)         &     (5.369)         &     (5.263)         &     (2.808)         &     (3.476)         \\
 Ratio population    &      -0.265         &      -0.265         &      -0.265         &      -0.270         &      -0.151         &      -0.334         \\
                    &     (0.326)         &     (0.155)         &     (0.159)         &     (0.166)         &     (0.209)         &     (0.527)         \\
 Ratio population    &      -0.177         &      -0.177\sym{**} &      -0.177\sym{**} &      -0.191\sym{***}&     -0.0763         &      -0.276\sym{**} \\
                    &     (0.247)         &    (0.0662)         &    (0.0678)         &    (0.0642)         &    (0.0863)         &     (0.117)         \\
 \midrule Birthmonth FE       &                     &  \checkmark         &  \checkmark         &  \checkmark         &  \checkmark         &  \checkmark         \\
Year FE             &                     &                     &  \checkmark         &  \checkmark         &                     &                     \\
Covariates          &                     &                     &                     &  \checkmark         &                     &                     \\
 
\bottomrule \end{tabular} } \begin{tablenotes} \item \scriptsize \emph{Notes:} Clustered standard errors in parentheses. Personal covariates contain age and age squared. Ratios indicate cases per thousand; either approximated population or original number of births. \end{tablenotes} \end{threeparttable} \end{table} 

 \begin{table}[H] \begin{threeparttable} \centering \caption{Robustness with respect to the choice of \texttt{control group}} {\def\sym#1{\ifmmode^{#1}\else\(^{#1}\)\fi} \begin{tabular}{l*{10}{c}} \toprule & \multicolumn{9}{c}{Dependent variable: \textbf{Injuries}} \\ \cmidrule(lr){2-10}
            &\multicolumn{3}{c}{Average Causal Effects}&\multicolumn{3}{c}{Women}             &\multicolumn{3}{c}{Men}               \\\cmidrule(lr){2-4}\cmidrule(lr){5-7}\cmidrule(lr){8-10}
            &\multicolumn{1}{c}{(1)}&\multicolumn{1}{c}{(2)}&\multicolumn{1}{c}{(3)}&\multicolumn{1}{c}{(4)}&\multicolumn{1}{c}{(5)}&\multicolumn{1}{c}{(6)}&\multicolumn{1}{c}{(7)}&\multicolumn{1}{c}{(8)}&\multicolumn{1}{c}{(9)}\\
            &\multicolumn{1}{c}{C2}&\multicolumn{1}{c}{C1+C2}&\multicolumn{1}{c}{C1-C3}&\multicolumn{1}{c}{C2}&\multicolumn{1}{c}{C1+C2}&\multicolumn{1}{c}{C1-C3}&\multicolumn{1}{c}{C2}&\multicolumn{1}{c}{C1+C2}&\multicolumn{1}{c}{C1-C3}\\
\midrule
 \multicolumn{10}{l}{\emph{Panel A. 2 Month bandwidth}} \\ Abs. numbers        &      -21.17\sym{**} &      -14.19         &      -15.30         &      -5.667         &      -4.000         &      -2.870         &      -15.50\sym{***}&      -10.19         &      -12.43         \\
                    &     (7.427)         &     (10.44)         &     (20.11)         &     (4.474)         &     (3.227)         &     (5.798)         &     (3.226)         &     (9.218)         &     (15.47)         \\
 Ratio population    &      -0.712\sym{***}&      -0.352         &      -0.243         &      -0.470\sym{***}&      -0.251         &      -0.230         &      -0.900\sym{*}  &      -0.399         &     -0.0591         \\
                    &     (0.170)         &     (0.240)         &     (0.225)         &    (0.0796)         &     (0.202)         &     (0.187)         &     (0.463)         &     (0.451)         &     (0.519)         \\
 Ratio fertility     &      -0.270\sym{**} &     -0.0800         &     -0.0700         &      -0.136         &     -0.0235         &      0.0190         &      -0.412\sym{**} &      -0.152         &      -0.168         \\
                    &     (0.114)         &     (0.129)         &     (0.205)         &     (0.138)         &     (0.114)         &     (0.149)         &     (0.123)         &     (0.213)         &     (0.312)         \\
 \midrule\multicolumn{10}{l}{\emph{Panel B. 4 Month bandwidth}} \\ Abs. numbers        &      -8.889         &      -8.083         &      -12.40         &      -1.389         &      -2.056         &      -0.935         &      -7.500\sym{*}  &      -6.028         &      -11.46         \\
                    &     (7.504)         &     (8.772)         &     (15.90)         &     (4.207)         &     (3.908)         &     (5.101)         &     (3.887)         &     (6.113)         &     (11.74)         \\
 Ratio population    &      -0.527\sym{***}&      -0.438\sym{**} &      -0.356\sym{*}  &     -0.0578         &     -0.0577         &     -0.0221         &      -1.368\sym{**} &      -1.157\sym{**} &      -0.903         \\
                    &     (0.162)         &     (0.205)         &     (0.206)         &     (0.250)         &     (0.225)         &     (0.217)         &     (0.519)         &     (0.525)         &     (0.582)         \\
 Ratio fertility     &      -0.280\sym{***}&      -0.174\sym{*}  &      -0.190         &      -0.109         &     -0.0629         &    -0.00899         &      -0.464\sym{***}&      -0.310\sym{*}  &      -0.387\sym{*}  \\
                    &    (0.0628)         &    (0.0963)         &     (0.128)         &     (0.108)         &     (0.101)         &     (0.108)         &    (0.0916)         &     (0.158)         &     (0.211)         \\
 \midrule\multicolumn{10}{l}{\emph{Panel C. 6 Month bandwidth}} \\ Abs. numbers        &       2.685         &       4.454         &   -9.74e-14         &       0.611         &       0.759         &       0.364         &       2.074         &       3.694         &      -0.364         \\
                    &     (6.193)         &     (7.203)         &     (12.65)         &     (2.900)         &     (3.171)         &     (4.061)         &     (3.968)         &     (5.125)         &     (9.394)         \\
 Ratio population    &      -0.402\sym{**} &      -0.288\sym{*}  &      -0.221         &      -0.212         &      -0.183         &      -0.153         &      -0.625         &      -0.394         &      -0.238         \\
                    &     (0.149)         &     (0.168)         &     (0.157)         &     (0.176)         &     (0.159)         &     (0.158)         &     (0.462)         &     (0.452)         &     (0.460)         \\
 Ratio fertility     &      -0.219\sym{***}&     -0.0764         &     -0.0580         &      -0.137\sym{*}  &     -0.0592         &     -0.0193         &      -0.295\sym{***}&     -0.0928         &     -0.0990         \\
                    &    (0.0605)         &    (0.0895)         &     (0.110)         &    (0.0767)         &    (0.0838)         &    (0.0830)         &    (0.0929)         &     (0.147)         &     (0.186)         \\
 \midrule\multicolumn{10}{l}{\emph{Panel D. Donut specification}} \\ Abs. numbers        &       11.22\sym{**} &       9.700         &       5.119         &       4.133         &       2.533         &       2.407         &       7.089\sym{*}  &       7.167         &       2.711         \\
                    &     (5.239)         &     (7.302)         &     (13.29)         &     (2.808)         &     (3.420)         &     (4.237)         &     (3.476)         &     (4.844)         &     (9.929)         \\
 Ratio population    &      -0.265         &      -0.270         &      -0.223         &      -0.151         &      -0.193         &      -0.182         &      -0.334         &      -0.311         &      -0.146         \\
                    &     (0.155)         &     (0.177)         &     (0.170)         &     (0.209)         &     (0.171)         &     (0.179)         &     (0.527)         &     (0.520)         &     (0.530)         \\
 Ratio fertility     &      -0.156\sym{**} &     -0.0550         &     -0.0171         &     -0.0664         &     -0.0369         &      0.0218         &      -0.242\sym{**} &     -0.0778         &     -0.0614         \\
                    &    (0.0630)         &    (0.0956)         &     (0.114)         &    (0.0826)         &    (0.0922)         &    (0.0866)         &     (0.108)         &     (0.152)         &     (0.198)         \\
 
\bottomrule \end{tabular} } \begin{tablenotes} \item \scriptsize \emph{Notes:} Clustered standard errors in parentheses. All regressions contain Birthmonth FE. Ratios indicate cases per thousand; either approximated population or original number of births. \end{tablenotes} \end{threeparttable} \end{table} 

%=========================================
 \begin{table}[H] \begin{threeparttable} \centering \caption{Robustness with respect to the inclusion of \texttt{fixed effects} and \texttt{covariates}} {\def\sym#1{\ifmmode^{#1}\else\(^{#1}\)\fi} \begin{tabular}{l*{7}{c}} \toprule & \multicolumn{6}{c}{Dependent variable: \textbf{Neurosis}} \\ \cmidrule(lr){2-7}
            &\multicolumn{4}{c}{Average Causal Effects}         &\multicolumn{2}{c}{Heterogeneous Causal Effects}\\\cmidrule(lr){2-5}\cmidrule(lr){6-7}
            &\multicolumn{1}{c}{(1)}&\multicolumn{1}{c}{(2)}&\multicolumn{1}{c}{(3)}&\multicolumn{1}{c}{(4)}&\multicolumn{1}{c}{(5)}&\multicolumn{1}{c}{(6)}\\
            &\multicolumn{1}{c}{}&\multicolumn{1}{c}{}&\multicolumn{1}{c}{}&\multicolumn{1}{c}{}&\multicolumn{1}{c}{Women}&\multicolumn{1}{c}{Men}\\
\midrule
 \multicolumn{7}{l}{\emph{Panel A. 2 Month bandwidth}} \\ Abs. numbers        &       6.833         &       6.833         &       6.833         &       6.833         &       4.444         &       2.389         \\
                    &     (7.027)         &     (3.943)         &     (4.206)         &     (4.242)         &     (5.002)         &     (1.614)         \\
 Ratio population    &      0.0140         &      0.0140         &      0.0140         &      0.0140         &      0.0170         &      0.0151         \\
                    &     (0.105)         &    (0.0688)         &    (0.0734)         &    (0.0741)         &     (0.132)         &    (0.0543)         \\
 Ratio population    &       0.111         &       0.111         &       0.111         &       0.111         &       0.153         &      0.0708         \\
                    &     (0.104)         &    (0.0608)         &    (0.0648)         &    (0.0654)         &     (0.158)         &    (0.0419)         \\
 \midrule\multicolumn{7}{l}{\emph{Panel B. 4 Month bandwidth}} \\ Abs. numbers        &       13.17\sym{**} &       13.17\sym{***}&       13.17\sym{***}&       13.17\sym{***}&       7.444\sym{**} &       5.722\sym{***}\\
                    &     (5.663)         &     (3.158)         &     (3.257)         &     (3.270)         &     (3.435)         &     (1.430)         \\
 Ratio population    &       0.119         &       0.119\sym{**} &       0.119\sym{*}  &       0.119\sym{*}  &       0.234\sym{**} &      0.0369         \\
                    &     (0.104)         &    (0.0545)         &    (0.0562)         &    (0.0564)         &     (0.107)         &    (0.0469)         \\
 Ratio population    &       0.168         &       0.168\sym{***}&       0.168\sym{***}&       0.168\sym{***}&       0.194\sym{*}  &       0.144\sym{***}\\
                    &    (0.0980)         &    (0.0468)         &    (0.0482)         &    (0.0484)         &     (0.101)         &    (0.0477)         \\
 \midrule\multicolumn{7}{l}{\emph{Panel C. 6 Month bandwidth}} \\ Abs. numbers        &       10.19\sym{**} &       10.19\sym{***}&       10.19\sym{***}&       10.13\sym{***}&       4.093         &       6.093\sym{***}\\
                    &     (4.766)         &     (2.384)         &     (2.433)         &     (2.449)         &     (2.735)         &     (1.070)         \\
 Ratio population    &      0.0661         &      0.0661         &      0.0661         &      0.0671         &      0.0370         &       0.108\sym{*}  \\
                    &    (0.0855)         &    (0.0501)         &    (0.0511)         &    (0.0520)         &     (0.110)         &    (0.0562)         \\
 Ratio population    &      0.0974         &      0.0974\sym{**} &      0.0974\sym{**} &      0.0963\sym{**} &      0.0518         &       0.144\sym{***}\\
                    &    (0.0772)         &    (0.0436)         &    (0.0445)         &    (0.0449)         &    (0.0948)         &    (0.0333)         \\
 \midrule\multicolumn{7}{l}{\emph{Panel D. Donut specification}} \\ Abs. numbers        &       12.87\sym{**} &       12.87\sym{***}&       12.87\sym{***}&       12.80\sym{***}&       6.467\sym{**} &       6.400\sym{***}\\
                    &     (5.513)         &     (2.407)         &     (2.466)         &     (2.498)         &     (2.877)         &     (1.124)         \\
 Ratio fertility     &       0.119         &       0.119\sym{**} &       0.119\sym{**} &       0.118\sym{**} &       0.107         &       0.132\sym{***}\\
                    &    (0.0776)         &    (0.0463)         &    (0.0474)         &    (0.0480)         &     (0.104)         &    (0.0288)         \\
 Ratio fertility     &       0.119         &       0.119\sym{**} &       0.119\sym{**} &       0.118\sym{**} &       0.107         &       0.132\sym{***}\\
                    &    (0.0776)         &    (0.0463)         &    (0.0474)         &    (0.0480)         &     (0.104)         &    (0.0288)         \\
 \midrule Birthmonth FE       &                     &  \checkmark         &  \checkmark         &  \checkmark         &  \checkmark         &  \checkmark         \\
Year FE             &                     &                     &  \checkmark         &  \checkmark         &                     &                     \\
Covariates          &                     &                     &                     &  \checkmark         &                     &                     \\
 
\bottomrule \end{tabular} } \begin{tablenotes} \item \scriptsize \emph{Notes:} Clustered standard errors in parentheses. Personal covariates contain age and age squared. Ratios indicate cases per thousand; either approximated population or original number of births. \end{tablenotes} \end{threeparttable} \end{table} 

 \begin{table}[H] \begin{threeparttable} \centering \caption{Robustness with respect to the choice of \texttt{control group}} {\def\sym#1{\ifmmode^{#1}\else\(^{#1}\)\fi} \begin{tabular}{l*{10}{c}} \toprule & \multicolumn{9}{c}{Dependent variable: \textbf{Neurosis}} \\ \cmidrule(lr){2-10}
            &\multicolumn{3}{c}{Average Causal Effects}&\multicolumn{3}{c}{Women}             &\multicolumn{3}{c}{Men}               \\\cmidrule(lr){2-4}\cmidrule(lr){5-7}\cmidrule(lr){8-10}
            &\multicolumn{1}{c}{(1)}&\multicolumn{1}{c}{(2)}&\multicolumn{1}{c}{(3)}&\multicolumn{1}{c}{(4)}&\multicolumn{1}{c}{(5)}&\multicolumn{1}{c}{(6)}&\multicolumn{1}{c}{(7)}&\multicolumn{1}{c}{(8)}&\multicolumn{1}{c}{(9)}\\
            &\multicolumn{1}{c}{C2}&\multicolumn{1}{c}{C1+C2}&\multicolumn{1}{c}{C1-C3}&\multicolumn{1}{c}{C2}&\multicolumn{1}{c}{C1+C2}&\multicolumn{1}{c}{C1-C3}&\multicolumn{1}{c}{C2}&\multicolumn{1}{c}{C1+C2}&\multicolumn{1}{c}{C1-C3}\\
\midrule
 \multicolumn{10}{l}{\emph{Panel A. 2 Month bandwidth}} \\ Abs. numbers        &       6.833         &       3.889         &       0.963         &       4.444         &       3.833         &       2.519         &       2.389         &      0.0556         &      -1.556         \\
                    &     (3.943)         &     (3.624)         &     (5.032)         &     (5.002)         &     (4.473)         &     (4.536)         &     (1.614)         &     (2.090)         &     (2.778)         \\
 Ratio population    &      0.0140         &      0.0291         &     0.00969         &      0.0170         &      0.0688         &      0.0192         &      0.0151         &    -0.00529         &    -0.00762         \\
                    &    (0.0688)         &    (0.0824)         &    (0.0658)         &     (0.132)         &     (0.138)         &     (0.114)         &    (0.0543)         &    (0.0827)         &    (0.0822)         \\
 Ratio fertility     &       0.104         &      0.0841         &      0.0477         &       0.147         &       0.156         &       0.121         &      0.0643         &      0.0181         &     -0.0202         \\
                    &    (0.0565)         &    (0.0552)         &    (0.0653)         &     (0.151)         &     (0.137)         &     (0.138)         &    (0.0378)         &    (0.0615)         &    (0.0639)         \\
 \midrule\multicolumn{10}{l}{\emph{Panel B. 4 Month bandwidth}} \\ Abs. numbers        &       13.17\sym{***}&       8.292\sym{***}&       6.861\sym{*}  &       7.444\sym{**} &       5.625\sym{*}  &       5.269\sym{*}  &       5.722\sym{***}&       2.667         &       1.593         \\
                    &     (3.158)         &     (2.729)         &     (3.831)         &     (3.435)         &     (2.821)         &     (3.014)         &     (1.430)         &     (1.608)         &     (1.889)         \\
 Ratio population    &       0.119\sym{**} &      0.0621         &      0.0695         &       0.234\sym{**} &       0.186\sym{*}  &       0.171\sym{*}  &      0.0369         &     -0.0362         &     -0.0179         \\
                    &    (0.0545)         &    (0.0543)         &    (0.0553)         &     (0.107)         &    (0.0975)         &    (0.0854)         &    (0.0469)         &    (0.0646)         &    (0.0705)         \\
 Ratio fertility     &       0.157\sym{***}&       0.110\sym{**} &      0.0970\sym{**} &       0.187\sym{*}  &       0.164\sym{*}  &       0.161\sym{*}  &       0.131\sym{***}&      0.0625         &      0.0395         \\
                    &    (0.0436)         &    (0.0403)         &    (0.0474)         &    (0.0971)         &    (0.0810)         &    (0.0834)         &    (0.0431)         &    (0.0510)         &    (0.0469)         \\
 \midrule\multicolumn{10}{l}{\emph{Panel C. 6 Month bandwidth}} \\ Abs. numbers        &       10.19\sym{***}&       5.565\sym{**} &       4.377         &       4.093         &       2.519         &       2.012         &       6.093\sym{***}&       3.046\sym{**} &       2.364         \\
                    &     (2.384)         &     (2.214)         &     (2.926)         &     (2.735)         &     (2.327)         &     (2.549)         &     (1.070)         &     (1.253)         &     (1.501)         \\
 Ratio population    &      0.0661         &      0.0117         &      0.0200         &      0.0370         &    -0.00302         &    -0.00321         &       0.108\sym{*}  &      0.0341         &      0.0459         \\
                    &    (0.0501)         &    (0.0472)         &    (0.0479)         &     (0.110)         &     (0.100)         &    (0.0986)         &    (0.0562)         &    (0.0579)         &    (0.0623)         \\
 Ratio fertility     &      0.0930\sym{**} &      0.0519         &      0.0507         &      0.0528         &      0.0380         &      0.0445         &       0.132\sym{***}&      0.0657\sym{*}  &      0.0578         \\
                    &    (0.0405)         &    (0.0363)         &    (0.0387)         &    (0.0910)         &    (0.0771)         &    (0.0787)         &    (0.0301)         &    (0.0364)         &    (0.0348)         \\
 \midrule\multicolumn{10}{l}{\emph{Panel D. Donut specification}} \\ Abs. numbers        &       12.87\sym{***}&       7.089\sym{***}&       6.548\sym{**} &       6.467\sym{**} &       4.200         &       3.822         &       6.400\sym{***}&       2.889\sym{*}  &       2.726         \\
                    &     (2.407)         &     (2.487)         &     (3.089)         &     (2.877)         &     (2.499)         &     (2.757)         &     (1.124)         &     (1.442)         &     (1.658)         \\
 Ratio population    &       0.111\sym{*}  &      0.0219         &      0.0361         &      0.0904         &      0.0226         &      0.0143         &       0.135\sym{**} &      0.0237         &      0.0549         \\
                    &    (0.0543)         &    (0.0525)         &    (0.0546)         &     (0.126)         &     (0.115)         &     (0.115)         &    (0.0633)         &    (0.0663)         &    (0.0726)         \\
 Ratio fertility     &       0.119\sym{**} &      0.0619         &      0.0743\sym{*}  &       0.107         &      0.0740         &      0.0892         &       0.132\sym{***}&      0.0518         &      0.0619         \\
                    &    (0.0463)         &    (0.0431)         &    (0.0432)         &     (0.104)         &    (0.0888)         &    (0.0893)         &    (0.0288)         &    (0.0399)         &    (0.0379)         \\
 
\bottomrule \end{tabular} } \begin{tablenotes} \item \scriptsize \emph{Notes:} Clustered standard errors in parentheses. All regressions contain Birthmonth FE. Ratios indicate cases per thousand; either approximated population or original number of births. \end{tablenotes} \end{threeparttable} \end{table} 

%=========================================
 \begin{table}[H] \begin{threeparttable} \centering \caption{Robustness with respect to the inclusion of \texttt{fixed effects} and \texttt{covariates}} {\def\sym#1{\ifmmode^{#1}\else\(^{#1}\)\fi} \begin{tabular}{l*{7}{c}} \toprule & \multicolumn{6}{c}{Dependent variable: \textbf{Disease of the joints}} \\ \cmidrule(lr){2-7}
            &\multicolumn{4}{c}{Average Causal Effects}         &\multicolumn{2}{c}{Heterogeneous Causal Effects}\\\cmidrule(lr){2-5}\cmidrule(lr){6-7}
            &\multicolumn{1}{c}{(1)}&\multicolumn{1}{c}{(2)}&\multicolumn{1}{c}{(3)}&\multicolumn{1}{c}{(4)}&\multicolumn{1}{c}{(5)}&\multicolumn{1}{c}{(6)}\\
            &\multicolumn{1}{c}{}&\multicolumn{1}{c}{}&\multicolumn{1}{c}{}&\multicolumn{1}{c}{}&\multicolumn{1}{c}{Women}&\multicolumn{1}{c}{Men}\\
\midrule
 \multicolumn{7}{l}{\emph{Panel A. 2 Month bandwidth}} \\ Abs. numbers        &       10.44         &       10.44\sym{**} &       10.44\sym{**} &       10.44\sym{**} &       6.722\sym{***}&       3.722         \\
                    &     (7.422)         &     (3.744)         &     (3.994)         &     (4.029)         &     (0.829)         &     (2.954)         \\
 Ratio fertility     &       0.154\sym{**} &       0.154\sym{**} &       0.154\sym{**} &       0.154\sym{**} &       0.205\sym{***}&       0.106         \\
                    &    (0.0592)         &    (0.0483)         &    (0.0515)         &    (0.0520)         &    (0.0235)         &    (0.0730)         \\
 Ratio fertility     &       0.154\sym{**} &       0.154\sym{**} &       0.154\sym{**} &       0.154\sym{**} &       0.205\sym{***}&       0.106         \\
                    &    (0.0592)         &    (0.0483)         &    (0.0515)         &    (0.0520)         &    (0.0235)         &    (0.0730)         \\
 \midrule\multicolumn{7}{l}{\emph{Panel B. 4 Month bandwidth}} \\ Abs. numbers        &       8.083         &       8.083\sym{**} &       8.083\sym{**} &       8.083\sym{**} &       5.333\sym{**} &       2.750\sym{*}  \\
                    &     (4.929)         &     (2.984)         &     (3.077)         &     (3.089)         &     (2.177)         &     (1.550)         \\
 Ratio fertility     &      0.0842         &      0.0842\sym{**} &      0.0842\sym{**} &      0.0842\sym{**} &       0.133\sym{**} &      0.0380         \\
                    &    (0.0765)         &    (0.0342)         &    (0.0352)         &    (0.0354)         &    (0.0561)         &    (0.0430)         \\
 Ratio population    &      0.0904         &      0.0904\sym{**} &      0.0904\sym{**} &      0.0904\sym{**} &       0.138\sym{**} &      0.0426         \\
                    &    (0.0822)         &    (0.0366)         &    (0.0377)         &    (0.0379)         &    (0.0584)         &    (0.0473)         \\
 \midrule\multicolumn{7}{l}{\emph{Panel C. 6 Month bandwidth}} \\ Abs. numbers        &       8.389\sym{*}  &       8.389\sym{***}&       8.389\sym{***}&       8.092\sym{***}&       7.370\sym{***}&       1.019         \\
                    &     (4.161)         &     (2.617)         &     (2.670)         &     (2.615)         &     (1.688)         &     (1.587)         \\
 Ratio population    &      0.0410         &      0.0410         &      0.0410         &      0.0376         &       0.168\sym{**} &     -0.0818         \\
                    &    (0.0823)         &    (0.0589)         &    (0.0601)         &    (0.0603)         &    (0.0651)         &    (0.0957)         \\
 Ratio fertility     &      0.0723         &      0.0723\sym{*}  &      0.0723\sym{*}  &      0.0673\sym{*}  &       0.178\sym{***}&     -0.0270         \\
                    &    (0.0601)         &    (0.0352)         &    (0.0359)         &    (0.0351)         &    (0.0450)         &    (0.0474)         \\
 \midrule\multicolumn{7}{l}{\emph{Panel D. Donut specification}} \\ Abs. numbers        &       9.644\sym{*}  &       9.644\sym{***}&       9.644\sym{***}&       9.190\sym{***}&       7.911\sym{***}&       1.733         \\
                    &     (4.743)         &     (3.069)         &     (3.144)         &     (3.087)         &     (1.884)         &     (1.768)         \\
 Ratio population    &      0.0609         &      0.0609         &      0.0609         &      0.0549         &       0.172\sym{**} &     -0.0476         \\
                    &    (0.0924)         &    (0.0701)         &    (0.0718)         &    (0.0722)         &    (0.0752)         &     (0.112)         \\
 Ratio fertility     &      0.0790         &      0.0790\sym{*}  &      0.0790\sym{*}  &      0.0715         &       0.185\sym{***}&     -0.0202         \\
                    &    (0.0641)         &    (0.0419)         &    (0.0430)         &    (0.0420)         &    (0.0523)         &    (0.0521)         \\
 \midrule Birthmonth FE       &                     &  \checkmark         &  \checkmark         &  \checkmark         &  \checkmark         &  \checkmark         \\
Year FE             &                     &                     &  \checkmark         &  \checkmark         &                     &                     \\
Covariates          &                     &                     &                     &  \checkmark         &                     &                     \\
 
\bottomrule \end{tabular} } \begin{tablenotes} \item \scriptsize \emph{Notes:} Clustered standard errors in parentheses. Personal covariates contain age and age squared. Ratios indicate cases per thousand; either approximated population or original number of births. \end{tablenotes} \end{threeparttable} \end{table} 

 \begin{table}[H] \begin{threeparttable} \centering \caption{Robustness with respect to the choice of \texttt{control group}} {\def\sym#1{\ifmmode^{#1}\else\(^{#1}\)\fi} \begin{tabular}{l*{10}{c}} \toprule & \multicolumn{9}{c}{Dependent variable: \textbf{Disease of the joints}} \\ \cmidrule(lr){2-10}
            &\multicolumn{3}{c}{Average Causal Effects}&\multicolumn{3}{c}{Women}             &\multicolumn{3}{c}{Men}               \\\cmidrule(lr){2-4}\cmidrule(lr){5-7}\cmidrule(lr){8-10}
            &\multicolumn{1}{c}{(1)}&\multicolumn{1}{c}{(2)}&\multicolumn{1}{c}{(3)}&\multicolumn{1}{c}{(4)}&\multicolumn{1}{c}{(5)}&\multicolumn{1}{c}{(6)}&\multicolumn{1}{c}{(7)}&\multicolumn{1}{c}{(8)}&\multicolumn{1}{c}{(9)}\\
            &\multicolumn{1}{c}{C2}&\multicolumn{1}{c}{C1+C2}&\multicolumn{1}{c}{C1-C3}&\multicolumn{1}{c}{C2}&\multicolumn{1}{c}{C1+C2}&\multicolumn{1}{c}{C1-C3}&\multicolumn{1}{c}{C2}&\multicolumn{1}{c}{C1+C2}&\multicolumn{1}{c}{C1-C3}\\
\midrule
 \multicolumn{10}{l}{\emph{Panel A. 2 Month bandwidth}} \\ Abs. numbers        &       10.44\sym{**} &       8.750\sym{*}  &       10.61\sym{**} &       6.722\sym{***}&       3.556\sym{*}  &       2.389\sym{*}  &       3.722         &       5.194         &       8.222\sym{**} \\
                    &     (3.744)         &     (4.689)         &     (3.911)         &     (0.829)         &     (1.632)         &     (1.310)         &     (2.954)         &     (3.900)         &     (3.540)         \\
 Ratio population    &      0.0783         &       0.107         &       0.156\sym{**} &       0.107\sym{***}&      0.0650         &      0.0187         &      0.0633         &       0.160         &       0.304\sym{**} \\
                    &    (0.0564)         &    (0.0786)         &    (0.0682)         &    (0.0270)         &    (0.0675)         &    (0.0885)         &     (0.139)         &     (0.133)         &     (0.140)         \\
 Ratio fertility     &       0.154\sym{**} &       0.152\sym{**} &       0.180\sym{***}&       0.205\sym{***}&       0.135\sym{**} &       0.102\sym{**} &       0.106         &       0.168         &       0.254\sym{**} \\
                    &    (0.0483)         &    (0.0655)         &    (0.0605)         &    (0.0235)         &    (0.0442)         &    (0.0409)         &    (0.0730)         &     (0.104)         &     (0.101)         \\
 \midrule\multicolumn{10}{l}{\emph{Panel B. 4 Month bandwidth}} \\ Abs. numbers        &       8.083\sym{**} &       4.667         &       4.852         &       5.333\sym{**} &       3.167         &       2.269         &       2.750\sym{*}  &       1.500         &       2.583         \\
                    &     (2.984)         &     (3.791)         &     (3.748)         &     (2.177)         &     (2.318)         &     (2.325)         &     (1.550)         &     (2.424)         &     (2.593)         \\
 Ratio population    &      0.0396         &     0.00448         &      0.0338         &       0.153\sym{*}  &      0.0979         &      0.0661         &     -0.0903         &      -0.107         &     -0.0143         \\
                    &    (0.0545)         &    (0.0739)         &    (0.0721)         &    (0.0858)         &    (0.0875)         &    (0.0974)         &    (0.0994)         &     (0.120)         &     (0.134)         \\
 Ratio fertility     &      0.0842\sym{**} &      0.0587         &      0.0655         &       0.133\sym{**} &      0.0972         &      0.0736         &      0.0380         &      0.0219         &      0.0578         \\
                    &    (0.0342)         &    (0.0534)         &    (0.0600)         &    (0.0561)         &    (0.0688)         &    (0.0730)         &    (0.0430)         &    (0.0671)         &    (0.0798)         \\
 \midrule\multicolumn{10}{l}{\emph{Panel C. 6 Month bandwidth}} \\ Abs. numbers        &       8.389\sym{***}&       6.306\sym{**} &       4.358         &       7.370\sym{***}&       5.417\sym{***}&       3.210\sym{*}  &       1.019         &       0.889         &       1.148         \\
                    &     (2.617)         &     (2.899)         &     (2.914)         &     (1.688)         &     (1.769)         &     (1.773)         &     (1.587)         &     (1.932)         &     (2.033)         \\
 Ratio population    &      0.0410         &      0.0281         &      0.0193         &       0.168\sym{**} &       0.114\sym{*}  &      0.0518         &     -0.0818         &     -0.0555         &     -0.0107         \\
                    &    (0.0589)         &    (0.0614)         &    (0.0621)         &    (0.0651)         &    (0.0615)         &    (0.0706)         &    (0.0957)         &     (0.100)         &     (0.108)         \\
 Ratio fertility     &      0.0723\sym{*}  &      0.0711         &      0.0524         &       0.178\sym{***}&       0.149\sym{***}&      0.0924\sym{*}  &     -0.0270         &    -0.00165         &      0.0152         \\
                    &    (0.0352)         &    (0.0427)         &    (0.0486)         &    (0.0450)         &    (0.0502)         &    (0.0550)         &    (0.0474)         &    (0.0583)         &    (0.0657)         \\
 \midrule\multicolumn{10}{l}{\emph{Panel D. Donut specification}} \\ Abs. numbers        &       9.644\sym{***}&       7.578\sym{**} &       4.578         &       7.911\sym{***}&       5.967\sym{***}&       3.459\sym{*}  &       1.733         &       1.611         &       1.119         \\
                    &     (3.069)         &     (3.260)         &     (3.382)         &     (1.884)         &     (2.038)         &     (2.035)         &     (1.768)         &     (2.157)         &     (2.344)         \\
 Ratio population    &      0.0609         &      0.0329         &     0.00502         &       0.172\sym{**} &       0.112         &      0.0308         &     -0.0476         &     -0.0466         &     -0.0201         \\
                    &    (0.0701)         &    (0.0679)         &    (0.0687)         &    (0.0752)         &    (0.0672)         &    (0.0784)         &     (0.112)         &     (0.116)         &     (0.126)         \\
 Ratio fertility     &      0.0790\sym{*}  &      0.0783         &      0.0488         &       0.185\sym{***}&       0.157\sym{**} &      0.0945         &     -0.0202         &     0.00469         &     0.00601         \\
                    &    (0.0419)         &    (0.0486)         &    (0.0567)         &    (0.0523)         &    (0.0589)         &    (0.0639)         &    (0.0521)         &    (0.0653)         &    (0.0762)         \\
 
\bottomrule \end{tabular} } \begin{tablenotes} \item \scriptsize \emph{Notes:} Clustered standard errors in parentheses. All regressions contain Birthmonth FE. Ratios indicate cases per thousand; either approximated population or original number of births. \end{tablenotes} \end{threeparttable} \end{table} 

%=========================================
 \begin{table}[H] \begin{threeparttable} \centering \caption{Robustness with respect to the inclusion of \texttt{fixed effects} and \texttt{covariates}} {\def\sym#1{\ifmmode^{#1}\else\(^{#1}\)\fi} \begin{tabular}{l*{7}{c}} \toprule & \multicolumn{6}{c}{Dependent variable: \textbf{Diseases of the kidneys}} \\ \cmidrule(lr){2-7}
            &\multicolumn{4}{c}{Average Causal Effects}         &\multicolumn{2}{c}{Heterogeneous Causal Effects}\\\cmidrule(lr){2-5}\cmidrule(lr){6-7}
            &\multicolumn{1}{c}{(1)}&\multicolumn{1}{c}{(2)}&\multicolumn{1}{c}{(3)}&\multicolumn{1}{c}{(4)}&\multicolumn{1}{c}{(5)}&\multicolumn{1}{c}{(6)}\\
            &\multicolumn{1}{c}{}&\multicolumn{1}{c}{}&\multicolumn{1}{c}{}&\multicolumn{1}{c}{}&\multicolumn{1}{c}{Women}&\multicolumn{1}{c}{Men}\\
\midrule
 \multicolumn{7}{l}{\emph{Panel A. 2 Month bandwidth}} \\ Abs. numbers        &       5.500         &       5.500\sym{***}&       5.500\sym{**} &       5.500\sym{**} &       5.167         &       0.333         \\
                    &     (5.631)         &     (1.515)         &     (1.617)         &     (1.631)         &     (3.847)         &     (3.339)         \\
 Ratio population    &     -0.0517\sym{**} &     -0.0517\sym{***}&     -0.0517\sym{***}&     -0.0517\sym{***}&      0.0209         &     -0.0952\sym{*}  \\
                    &    (0.0212)         &   (0.00742)         &   (0.00792)         &   (0.00799)         &    (0.0542)         &    (0.0438)         \\
 Ratio fertility     &      0.0855\sym{*}  &      0.0855\sym{***}&      0.0855\sym{***}&      0.0855\sym{***}&       0.160         &      0.0132         \\
                    &    (0.0379)         &    (0.0122)         &    (0.0131)         &    (0.0132)         &     (0.113)         &     (0.109)         \\
 \midrule\multicolumn{7}{l}{\emph{Panel B. 4 Month bandwidth}} \\ Abs. numbers        &       6.583         &       6.583         &       6.583         &       6.583         &       3.167         &       3.417         \\
                    &     (6.243)         &     (4.497)         &     (4.637)         &     (4.656)         &     (2.238)         &     (4.922)         \\
 Ratio fertility     &      0.0451         &      0.0451         &      0.0451         &      0.0451         &      0.0566         &      0.0317         \\
                    &     (0.104)         &    (0.0581)         &    (0.0599)         &    (0.0602)         &    (0.0752)         &     (0.127)         \\
 Ratio fertility     &      0.0451         &      0.0451         &      0.0451         &      0.0451         &      0.0566         &      0.0317         \\
                    &     (0.104)         &    (0.0581)         &    (0.0599)         &    (0.0602)         &    (0.0752)         &     (0.127)         \\
 \midrule\multicolumn{7}{l}{\emph{Panel C. 6 Month bandwidth}} \\ Abs. numbers        &       6.796         &       6.796\sym{**} &       6.796\sym{**} &       6.646\sym{**} &       3.481\sym{**} &       3.315         \\
                    &     (5.488)         &     (2.986)         &     (3.047)         &     (3.107)         &     (1.619)         &     (3.315)         \\
 Ratio population    &     -0.0416         &     -0.0416         &     -0.0416         &     -0.0432         &     0.00810         &     -0.0858         \\
                    &     (0.103)         &    (0.0685)         &    (0.0699)         &    (0.0719)         &    (0.0626)         &     (0.170)         \\
 Ratio population    &      0.0206         &      0.0206         &      0.0206         &      0.0160         &      0.0324         &     0.00844         \\
                    &    (0.0809)         &    (0.0466)         &    (0.0475)         &    (0.0479)         &    (0.0618)         &    (0.0952)         \\
 \midrule\multicolumn{7}{l}{\emph{Panel D. Donut specification}} \\ Abs. numbers        &       6.956         &       6.956\sym{*}  &       6.956\sym{*}  &       6.804\sym{*}  &       4.689\sym{***}&       2.267         \\
                    &     (6.162)         &     (3.384)         &     (3.468)         &     (3.560)         &     (1.581)         &     (3.325)         \\
 Ratio fertility     &     0.00573         &     0.00573         &     0.00573         &     0.00109         &      0.0564         &     -0.0425         \\
                    &    (0.0879)         &    (0.0511)         &    (0.0524)         &    (0.0527)         &    (0.0609)         &    (0.0866)         \\
 Ratio fertility     &     0.00573         &     0.00573         &     0.00573         &     0.00109         &      0.0564         &     -0.0425         \\
                    &    (0.0879)         &    (0.0511)         &    (0.0524)         &    (0.0527)         &    (0.0609)         &    (0.0866)         \\
 \midrule Birthmonth FE       &                     &  \checkmark         &  \checkmark         &  \checkmark         &  \checkmark         &  \checkmark         \\
Year FE             &                     &                     &  \checkmark         &  \checkmark         &                     &                     \\
Covariates          &                     &                     &                     &  \checkmark         &                     &                     \\
 
\bottomrule \end{tabular} } \begin{tablenotes} \item \scriptsize \emph{Notes:} Clustered standard errors in parentheses. Personal covariates contain age and age squared. Ratios indicate cases per thousand; either approximated population or original number of births. \end{tablenotes} \end{threeparttable} \end{table} 

 \begin{table}[H] \begin{threeparttable} \centering \caption{Robustness with respect to the choice of \texttt{control group}} {\def\sym#1{\ifmmode^{#1}\else\(^{#1}\)\fi} \begin{tabular}{l*{10}{c}} \toprule & \multicolumn{9}{c}{Dependent variable: \textbf{Diseases of the kidneys}} \\ \cmidrule(lr){2-10}
            &\multicolumn{3}{c}{Average Causal Effects}&\multicolumn{3}{c}{Women}             &\multicolumn{3}{c}{Men}               \\\cmidrule(lr){2-4}\cmidrule(lr){5-7}\cmidrule(lr){8-10}
            &\multicolumn{1}{c}{(1)}&\multicolumn{1}{c}{(2)}&\multicolumn{1}{c}{(3)}&\multicolumn{1}{c}{(4)}&\multicolumn{1}{c}{(5)}&\multicolumn{1}{c}{(6)}&\multicolumn{1}{c}{(7)}&\multicolumn{1}{c}{(8)}&\multicolumn{1}{c}{(9)}\\
            &\multicolumn{1}{c}{C2}&\multicolumn{1}{c}{C1+C2}&\multicolumn{1}{c}{C1-C3}&\multicolumn{1}{c}{C2}&\multicolumn{1}{c}{C1+C2}&\multicolumn{1}{c}{C1-C3}&\multicolumn{1}{c}{C2}&\multicolumn{1}{c}{C1+C2}&\multicolumn{1}{c}{C1-C3}\\
\midrule
 \multicolumn{10}{l}{\emph{Panel A. 2 Month bandwidth}} \\ Abs. numbers        &       5.500\sym{***}&       0.500         &       1.019         &       5.167         &      -0.639         &       0.370         &       0.333         &       1.139         &       0.648         \\
                    &     (1.515)         &     (3.797)         &     (3.055)         &     (3.847)         &     (3.979)         &     (4.078)         &     (3.339)         &     (3.523)         &     (3.732)         \\
 Ratio population    &     -0.0517\sym{***}&     -0.0395         &   -0.000727         &      0.0209         &     -0.0821         &     -0.0622         &     -0.0952\sym{*}  &      0.0229         &      0.0752         \\
                    &   (0.00742)         &    (0.0735)         &    (0.0800)         &    (0.0542)         &    (0.0564)         &    (0.0658)         &    (0.0438)         &     (0.146)         &     (0.149)         \\
 Ratio fertility     &      0.0855\sym{***}&      0.0443         &      0.0569         &       0.160         &      0.0182         &      0.0499         &      0.0132         &      0.0668         &      0.0618         \\
                    &    (0.0122)         &    (0.0431)         &    (0.0484)         &     (0.113)         &     (0.104)         &     (0.104)         &     (0.109)         &     (0.120)         &     (0.132)         \\
 \midrule\multicolumn{10}{l}{\emph{Panel B. 4 Month bandwidth}} \\ Abs. numbers        &       6.583         &       7.208         &       4.343         &       3.167         &       0.889         &   -7.62e-16         &       3.417         &       6.319         &       4.343         \\
                    &     (4.497)         &     (4.535)         &     (4.148)         &     (2.238)         &     (2.280)         &     (2.559)         &     (4.922)         &     (4.440)         &     (4.111)         \\
 Ratio population    &     -0.0330         &     0.00577         &   -0.000799         &      0.0883         &      0.0293         &    -0.00423         &      -0.178         &     -0.0528         &     -0.0304         \\
                    &    (0.0892)         &    (0.0912)         &    (0.0894)         &    (0.0625)         &    (0.0661)         &    (0.0601)         &     (0.230)         &     (0.212)         &     (0.211)         \\
 Ratio fertility     &      0.0451         &      0.0904         &      0.0565         &      0.0566         &      0.0214         &     0.00307         &      0.0317         &       0.152         &       0.104         \\
                    &    (0.0581)         &    (0.0575)         &    (0.0660)         &    (0.0752)         &    (0.0633)         &    (0.0678)         &     (0.127)         &     (0.121)         &     (0.128)         \\
 \midrule\multicolumn{10}{l}{\emph{Panel C. 6 Month bandwidth}} \\ Abs. numbers        &       6.796\sym{**} &       9.361\sym{***}&       5.451\sym{*}  &       3.481\sym{**} &       2.657         &       1.148         &       3.315         &       6.704\sym{**} &       4.302         \\
                    &     (2.986)         &     (3.293)         &     (3.045)         &     (1.619)         &     (1.886)         &     (2.025)         &     (3.315)         &     (3.074)         &     (2.937)         \\
 Ratio population    &     -0.0416         &      0.0316         &     0.00393         &     0.00810         &    -0.00340         &     -0.0372         &     -0.0858         &      0.0682         &      0.0436         \\
                    &    (0.0685)         &    (0.0740)         &    (0.0720)         &    (0.0626)         &    (0.0689)         &    (0.0645)         &     (0.170)         &     (0.159)         &     (0.161)         \\
 Ratio fertility     &      0.0217         &       0.103\sym{**} &      0.0623         &      0.0338         &      0.0458         &      0.0212         &      0.0101         &       0.157\sym{*}  &       0.100         \\
                    &    (0.0435)         &    (0.0503)         &    (0.0553)         &    (0.0596)         &    (0.0615)         &    (0.0598)         &    (0.0864)         &    (0.0882)         &    (0.0974)         \\
 \midrule\multicolumn{10}{l}{\emph{Panel D. Donut specification}} \\ Abs. numbers        &       6.956\sym{*}  &       10.29\sym{**} &       5.956         &       4.689\sym{***}&       4.222\sym{**} &       2.296         &       2.267         &       6.067\sym{*}  &       3.659         \\
                    &     (3.384)         &     (3.836)         &     (3.582)         &     (1.581)         &     (1.839)         &     (2.058)         &     (3.325)         &     (3.342)         &     (3.112)         \\
 Ratio population    &     -0.0398         &      0.0240         &     -0.0148         &      0.0282         &      0.0200         &     -0.0396         &      -0.106         &      0.0230         &     0.00542         \\
                    &    (0.0794)         &    (0.0844)         &    (0.0805)         &    (0.0705)         &    (0.0810)         &    (0.0732)         &     (0.193)         &     (0.179)         &     (0.178)         \\
 Ratio fertility     &     0.00573         &       0.101         &      0.0597         &      0.0564         &      0.0815         &      0.0497         &     -0.0425         &       0.119         &      0.0682         \\
                    &    (0.0511)         &    (0.0604)         &    (0.0652)         &    (0.0609)         &    (0.0625)         &    (0.0621)         &    (0.0866)         &    (0.0982)         &     (0.106)         \\
 
\bottomrule \end{tabular} } \begin{tablenotes} \item \scriptsize \emph{Notes:} Clustered standard errors in parentheses. All regressions contain Birthmonth FE. Ratios indicate cases per thousand; either approximated population or original number of births. \end{tablenotes} \end{threeparttable} \end{table} 

%=========================================
 \begin{table}[H] \begin{threeparttable} \centering \caption{Robustness with respect to the inclusion of \texttt{fixed effects} and \texttt{covariates}} {\def\sym#1{\ifmmode^{#1}\else\(^{#1}\)\fi} \begin{tabular}{l*{7}{c}} \toprule & \multicolumn{6}{c}{Dependent variable: \textbf{Diseases of the bile and pancreas}} \\ \cmidrule(lr){2-7}
            &\multicolumn{4}{c}{Average Causal Effects}         &\multicolumn{2}{c}{Heterogeneous Causal Effects}\\\cmidrule(lr){2-5}\cmidrule(lr){6-7}
            &\multicolumn{1}{c}{(1)}&\multicolumn{1}{c}{(2)}&\multicolumn{1}{c}{(3)}&\multicolumn{1}{c}{(4)}&\multicolumn{1}{c}{(5)}&\multicolumn{1}{c}{(6)}\\
            &\multicolumn{1}{c}{}&\multicolumn{1}{c}{}&\multicolumn{1}{c}{}&\multicolumn{1}{c}{}&\multicolumn{1}{c}{Women}&\multicolumn{1}{c}{Men}\\
\midrule
 \multicolumn{7}{l}{\emph{Panel A. 2 Month bandwidth}} \\ Abs. numbers        &      -1.056         &      -1.056         &      -1.056         &      -1.056         &      -3.556\sym{***}&       2.500         \\
                    &     (6.663)         &     (3.916)         &     (4.177)         &     (4.214)         &     (0.785)         &     (3.184)         \\
 Ratio population    &      -0.124         &      -0.124\sym{**} &      -0.124\sym{*}  &      -0.124\sym{*}  &      -0.284\sym{**} &      0.0258         \\
                    &    (0.0886)         &    (0.0510)         &    (0.0544)         &    (0.0549)         &    (0.0941)         &    (0.0992)         \\
 Ratio population    &     -0.0140         &     -0.0140         &     -0.0140         &     -0.0140         &     -0.0932\sym{***}&      0.0693         \\
                    &     (0.100)         &    (0.0539)         &    (0.0575)         &    (0.0580)         &    (0.0164)         &    (0.0933)         \\
 \midrule\multicolumn{7}{l}{\emph{Panel B. 4 Month bandwidth}} \\ Abs. numbers        &      -4.056         &      -4.056         &      -4.056         &      -4.056         &      -3.556         &      -0.500         \\
                    &     (7.164)         &     (4.434)         &     (4.572)         &     (4.591)         &     (2.748)         &     (2.225)         \\
 Ratio fertility     &     -0.0996         &     -0.0996         &     -0.0996         &     -0.0996         &      -0.153         &     -0.0431         \\
                    &     (0.126)         &    (0.0838)         &    (0.0865)         &    (0.0868)         &     (0.106)         &    (0.0705)         \\
 Ratio fertility     &     -0.0996         &     -0.0996         &     -0.0996         &     -0.0996         &      -0.153         &     -0.0431         \\
                    &     (0.126)         &    (0.0838)         &    (0.0865)         &    (0.0868)         &     (0.106)         &    (0.0705)         \\
 \midrule\multicolumn{7}{l}{\emph{Panel C. 6 Month bandwidth}} \\ Abs. numbers        &      -3.537         &      -3.537         &      -3.537         &      -3.519         &      -3.333         &      -0.204         \\
                    &     (5.383)         &     (3.036)         &     (3.098)         &     (3.127)         &     (1.970)         &     (1.700)         \\
 Ratio fertility     &      -0.115         &      -0.115\sym{*}  &      -0.115\sym{*}  &      -0.117\sym{*}  &      -0.196\sym{**} &     -0.0391         \\
                    &    (0.0928)         &    (0.0601)         &    (0.0613)         &    (0.0616)         &    (0.0770)         &    (0.0565)         \\
 Ratio fertility     &      -0.115         &      -0.115\sym{*}  &      -0.115\sym{*}  &      -0.117\sym{*}  &      -0.196\sym{**} &     -0.0391         \\
                    &    (0.0928)         &    (0.0601)         &    (0.0613)         &    (0.0616)         &    (0.0770)         &    (0.0565)         \\
 \midrule\multicolumn{7}{l}{\emph{Panel D. Donut specification}} \\ Abs. numbers        &      -5.244         &      -5.244\sym{*}  &      -5.244\sym{*}  &      -5.058         &      -3.622         &      -1.622         \\
                    &     (5.940)         &     (2.862)         &     (2.933)         &     (2.953)         &     (2.276)         &     (1.253)         \\
 Ratio population    &      -0.202\sym{*}  &      -0.202\sym{***}&      -0.202\sym{***}&      -0.199\sym{***}&      -0.283\sym{***}&      -0.124         \\
                    &     (0.103)         &    (0.0538)         &    (0.0552)         &    (0.0570)         &    (0.0810)         &    (0.0929)         \\
 Ratio population    &      -0.168\sym{*}  &      -0.168\sym{**} &      -0.168\sym{**} &      -0.167\sym{**} &      -0.238\sym{**} &     -0.0974\sym{*}  \\
                    &    (0.0966)         &    (0.0619)         &    (0.0634)         &    (0.0636)         &    (0.0898)         &    (0.0480)         \\
 \midrule Birthmonth FE       &                     &  \checkmark         &  \checkmark         &  \checkmark         &  \checkmark         &  \checkmark         \\
Year FE             &                     &                     &  \checkmark         &  \checkmark         &                     &                     \\
Covariates          &                     &                     &                     &  \checkmark         &                     &                     \\
 
\bottomrule \end{tabular} } \begin{tablenotes} \item \scriptsize \emph{Notes:} Clustered standard errors in parentheses. Personal covariates contain age and age squared. Ratios indicate cases per thousand; either approximated population or original number of births. \end{tablenotes} \end{threeparttable} \end{table} 

 \begin{table}[H] \begin{threeparttable} \centering \caption{Robustness with respect to the choice of \texttt{control group}} {\def\sym#1{\ifmmode^{#1}\else\(^{#1}\)\fi} \begin{tabular}{l*{10}{c}} \toprule & \multicolumn{9}{c}{Dependent variable: \textbf{Diseases of the bile and pancreas}} \\ \cmidrule(lr){2-10}
            &\multicolumn{3}{c}{Average Causal Effects}&\multicolumn{3}{c}{Women}             &\multicolumn{3}{c}{Men}               \\\cmidrule(lr){2-4}\cmidrule(lr){5-7}\cmidrule(lr){8-10}
            &\multicolumn{1}{c}{(1)}&\multicolumn{1}{c}{(2)}&\multicolumn{1}{c}{(3)}&\multicolumn{1}{c}{(4)}&\multicolumn{1}{c}{(5)}&\multicolumn{1}{c}{(6)}&\multicolumn{1}{c}{(7)}&\multicolumn{1}{c}{(8)}&\multicolumn{1}{c}{(9)}\\
            &\multicolumn{1}{c}{C2}&\multicolumn{1}{c}{C1+C2}&\multicolumn{1}{c}{C1-C3}&\multicolumn{1}{c}{C2}&\multicolumn{1}{c}{C1+C2}&\multicolumn{1}{c}{C1-C3}&\multicolumn{1}{c}{C2}&\multicolumn{1}{c}{C1+C2}&\multicolumn{1}{c}{C1-C3}\\
\midrule
 \multicolumn{10}{l}{\emph{Panel A. 2 Month bandwidth}} \\ Abs. numbers        &      -1.056         &      -3.056         &      -6.241         &      -3.556\sym{***}&      -3.139         &      -4.741\sym{**} &       2.500         &      0.0833         &      -1.500         \\
                    &     (3.916)         &     (3.769)         &     (3.692)         &     (0.785)         &     (2.945)         &     (2.214)         &     (3.184)         &     (2.791)         &     (3.310)         \\
 Ratio population    &      -0.124\sym{**} &     -0.0834         &      -0.108         &      -0.284\sym{**} &      -0.173         &      -0.237         &      0.0258         &     -0.0109         &     -0.0310         \\
                    &    (0.0510)         &    (0.0855)         &    (0.0958)         &    (0.0941)         &     (0.189)         &     (0.181)         &    (0.0992)         &    (0.0757)         &     (0.108)         \\
 Ratio fertility     &     -0.0129         &     -0.0138         &     -0.0540         &     -0.0897\sym{***}&     -0.0371         &     -0.0799         &      0.0634         &      0.0127         &     -0.0263         \\
                    &    (0.0502)         &    (0.0583)         &    (0.0714)         &    (0.0156)         &     (0.111)         &    (0.0910)         &    (0.0845)         &    (0.0693)         &    (0.0961)         \\
 \midrule\multicolumn{10}{l}{\emph{Panel B. 4 Month bandwidth}} \\ Abs. numbers        &      -4.056         &      -2.458         &      -3.546         &      -3.556         &      -2.681         &      -2.787         &      -0.500         &       0.222         &      -0.759         \\
                    &     (4.434)         &     (4.013)         &     (4.409)         &     (2.748)         &     (2.678)         &     (2.550)         &     (2.225)         &     (2.135)         &     (2.588)         \\
 Ratio population    &      -0.170\sym{**} &      -0.120         &      -0.107         &      -0.122         &     -0.0787         &     -0.0908         &      -0.144         &      -0.109         &      -0.103         \\
                    &    (0.0677)         &    (0.0737)         &    (0.0893)         &    (0.0792)         &     (0.100)         &     (0.106)         &     (0.105)         &    (0.0899)         &     (0.111)         \\
 Ratio fertility     &     -0.0996         &     -0.0447         &     -0.0551         &      -0.153         &     -0.0811         &     -0.0773         &     -0.0431         &    -0.00239         &     -0.0273         \\
                    &    (0.0838)         &    (0.0792)         &    (0.0915)         &     (0.106)         &     (0.107)         &     (0.106)         &    (0.0705)         &    (0.0688)         &    (0.0883)         \\
 \midrule\multicolumn{10}{l}{\emph{Panel C. 6 Month bandwidth}} \\ Abs. numbers        &      -3.537         &      -0.620         &      -0.988         &      -3.333         &      -1.426         &      -1.272         &      -0.204         &       0.806         &       0.284         \\
                    &     (3.036)         &     (3.059)         &     (3.583)         &     (1.970)         &     (2.047)         &     (2.012)         &     (1.700)         &     (1.644)         &     (2.147)         \\
 Ratio population    &      -0.175\sym{***}&      -0.101\sym{*}  &     -0.0814         &      -0.247\sym{***}&      -0.167\sym{**} &      -0.146\sym{*}  &     -0.0879         &     -0.0371         &     -0.0337         \\
                    &    (0.0529)         &    (0.0571)         &    (0.0746)         &    (0.0707)         &    (0.0802)         &    (0.0847)         &    (0.0876)         &    (0.0738)         &    (0.0940)         \\
 Ratio fertility     &      -0.115\sym{*}  &     -0.0356         &     -0.0298         &      -0.196\sym{**} &     -0.0869         &     -0.0607         &     -0.0391         &      0.0130         &    0.000894         \\
                    &    (0.0601)         &    (0.0606)         &    (0.0746)         &    (0.0770)         &    (0.0808)         &    (0.0842)         &    (0.0565)         &    (0.0546)         &    (0.0745)         \\
 \midrule\multicolumn{10}{l}{\emph{Panel D. Donut specification}} \\ Abs. numbers        &      -5.244\sym{*}  &      -0.178         &     -0.0370         &      -3.622         &      -0.756         &      -0.356         &      -1.622         &       0.578         &       0.319         \\
                    &     (2.862)         &     (3.248)         &     (4.076)         &     (2.276)         &     (2.274)         &     (2.269)         &     (1.253)         &     (1.497)         &     (2.346)         \\
 Ratio population    &      -0.202\sym{***}&      -0.113\sym{*}  &     -0.0894         &      -0.283\sym{***}&      -0.184\sym{**} &      -0.171\sym{*}  &      -0.124         &     -0.0539         &     -0.0391         \\
                    &    (0.0538)         &    (0.0621)         &    (0.0842)         &    (0.0810)         &    (0.0816)         &    (0.0899)         &    (0.0929)         &    (0.0809)         &     (0.108)         \\
 Ratio fertility     &      -0.154\sym{**} &     -0.0425         &     -0.0248         &      -0.225\sym{**} &     -0.0845         &     -0.0453         &     -0.0863\sym{*}  &    -0.00102         &    -0.00316         \\
                    &    (0.0576)         &    (0.0661)         &    (0.0861)         &    (0.0860)         &    (0.0924)         &    (0.0976)         &    (0.0435)         &    (0.0512)         &    (0.0828)         \\
 
\bottomrule \end{tabular} } \begin{tablenotes} \item \scriptsize \emph{Notes:} Clustered standard errors in parentheses. All regressions contain Birthmonth FE. Ratios indicate cases per thousand; either approximated population or original number of births. \end{tablenotes} \end{threeparttable} \end{table} 

%=========================================


\end{document}