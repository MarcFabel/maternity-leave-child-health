%--------------------------------------------------------------------
%	DOCUMENT CLASS
%--------------------------------------------------------------------
\documentclass{scrartcl} % type of document (paper, presentation, book,...); scrartcl class with sans serif titles, European layout 
\usepackage{fullpage} % leaves less space at margins of page
\usepackage[onehalfspacing]{setspace} % determine line pitch to 1.5

%--------------------------------------------------------------------
%	INPUT
%--------------------------------------------------------------------
\usepackage[T1]{fontenc} % Use 8-bit encoding that has 256 glyphs
\usepackage[utf8]{inputenc} % Required for including letters with accents, Umlaute,...
\usepackage{float} % better control over placement of tables and figures in the text
\usepackage{graphicx} % input of graphics
\usepackage{xcolor} % advanced color package
\usepackage{url, hyperref} % include (clickable) URLs

%--------------------------------------------------------------------
%	TABLES, FIGURES, LISTS
%--------------------------------------------------------------------
\usepackage{booktabs} % better tables
\usepackage{threeparttable}
\renewcommand\TPTrlap{}
        \renewcommand\TPTnoteSettings{%
            \setlength\leftmargin{5pt}%  
            \setlength\rightmargin{5pt}%
          }

\usepackage[center, format=plain, font=normalsize, nooneline, labelfont={bf}]{caption} % change format of captions of tables and graphs 

% Allow line breaks with \\ in column headings of tables
\newcommand{\clb}[3][c]{%
	\begin{tabular}[#1]{@{}#2@{}}#3\end{tabular}}

% allow line breaks with \\ in row titles
\usepackage{multirow}

\newcommand{\lb}[3][c]{%
\multirow{2}{*}{\begin{tabular}[#1]{@{}#2@{}}#3\end{tabular}}}
% optional argument: b = bottom or t= top alignment

%--------------------------------------------------------------------
%	MATH
%--------------------------------------------------------------------
\usepackage{amsmath,amssymb} % more math symbols and commands

%--------------------------------------------------------------------
%	LANGUAGE SPECIFICS
%--------------------------------------------------------------------
\usepackage[american]{babel} % man­ages cul­tur­ally-de­ter­mined ty­po­graph­i­cal (and other) rules, and hy­phen­ation pat­terns
\usepackage{csquotes} % language specific quotations

%--------------------------------------------------------------------
%	PATHS
%--------------------------------------------------------------------
\makeatletter
\def\input@path{{../../analysis/tables/KKH/}}
%or: \def\input@path{{/path/to/folder/}{/path/to/other/folder/}}
\makeatother
\graphicspath{{../../analysis/graphs/KKH/}}






%\newgeometry{
%	top    = 3.5cm,
%	bottom = 2cm,
%	left   = 5.0cm,
%	right  = 2.8cm}
%	
\author{Marc Fabel}
\title{Overview of diagnoses outcomes}
\date{Last revision of this document: \today} 

%%%%%%%%%%%%%%%%%%%%%%%%%%%%%%%%%%%%%%%%%%%%%%%%%%%%%%%%%%%%
% BEGIN OF DOCUMENT
%%%%%%%%%%%%%%%%%%%%%%%%%%%%%%%%%%%%%%%%%%%%%%%%%%%%%%%%%%%%
\begin{document}
\maketitle
This document contains the results from the data in long format, i.e. per gender.

\newpage


%%%%%%%%%%%%%%%%%%%%%%%%%%%%%%%%%%%%%%%%%%%%%%%%%%%%%%%%%%%%%%%%%%%%%%%%%%%%%%%%%%%%%%%%%%%%%%%%%%%%%%%%%%%
\section{temp}



%\begin{table}[h]\centering
%\def\sym#1{\ifmmode^{#1}\else\(^{#1}\)\fi}
%\begin{tabular}{l*{3}{c}|cccc}
%\toprule
%% &\multicolumn{1}{c}{(1)}&\multicolumn{1}{c}{(2)}&\multicolumn{1}{c}{(3)}&\multicolumn{1}{c}{(4)}&\multicolumn{1}{c}{(5)}&\multicolumn{1}{c}{(6)}&\multicolumn{1}{c}{(7)} \\ 
%&\multicolumn{3}{c}{bandwidth of 6 months} & \multicolumn{4}{c}{different bandwidths} \\
% \cmidrule(lr{1em}){2-4} \cmidrule(lr{1em}){5-8}
% &\multicolumn{1}{c}{(1)}&\multicolumn{1}{c}{(2)}&\multicolumn{1}{c}{(3)}& 1 Month & 2 Months & 4 Months & 6M \& Donut \\
%\midrule 
%RD Linear           &    -0.00445         &    -0.00445         &    -0.00445         \\
                    &   (0.00435)         &   (0.00453)         &   (0.00453)         \\
RD Quadratic        &     0.00319         &     0.00319         &     0.00319         \\
                    &   (0.00861)         &   (0.00897)         &   (0.00897)         \\
RD Cubic            &     0.00146         &     0.00146         &     0.00146         \\
                    &    (0.0130)         &    (0.0136)         &    (0.0136)         \\
RD Linear Donut     &    -0.00881         &    -0.00881         &    -0.00881         \\
                    &   (0.00613)         &   (0.00643)         &   (0.00643)         \\
\midrule
DDRD: C1-C3 &    -0.00262         &    -0.00262         &    -0.00262         &     0.00255         &    -0.00137         &    -0.00284         &    -0.00366\sym{*}  \\
            &   (0.00209)         &   (0.00211)         &   (0.00211)         &   (0.00530)         &   (0.00337)         &   (0.00241)         &   (0.00212)         \\
DDRD: C2            &    -0.00452\sym{**} &    -0.00452\sym{**} &    -0.00452\sym{**} &    -0.00795\sym{***}&    -0.00630\sym{***}&    -0.00576\sym{***}&    -0.00383         \\
                    &   (0.00210)         &   (0.00215)         &   (0.00215)         &  (3.32e-17)         &  (0.000946)         &   (0.00177)         &   (0.00245)         \\
DDRD: C1+C2         &    -0.00398\sym{*}  &    -0.00398         &    -0.00398         &   -0.000162         &    -0.00275         &    -0.00472         &    -0.00475\sym{*}  \\
                    &   (0.00233)         &   (0.00236)         &   (0.00236)         &   (0.00496)         &   (0.00375)         &   (0.00278)         &   (0.00239)         \\
Birthmonth FE       &           X         &           X         &           X         &                     &                     &                     &                     \\
Time FE             &                     &           X         &           X         &                     &                     &                     &                     \\
Pers covar          &                     &                     &           X         &                     &                     &                     &                     \\
            &\multicolumn{1}{c}{Men}&\multicolumn{1}{c}{Women}\\
\midrule
DDRD: C1-C3 &    -0.00363         &    -0.00406         \\
            &   (0.00901)         &   (0.00496)         \\
DDRD: C2            &    -0.00823         &    -0.00456         \\
                    &   (0.00915)         &   (0.00486)         \\
DDRD: C1+C2         &     -0.0105         &    -0.00387         \\
                    &    (0.0101)         &   (0.00514)         \\

%\bottomrule
%\end{tabular}
%\end{table}
%
%
%
\begin{table}[h] % table environment for caption and label
\begin{threeparttable}
\centering % center the tabular
\caption{Regression Output with \texttt {esttab}} % caption
\label{table 1} % label to refer to it in the text
 \begin{table}[H] \begin{threeparttable} \centering \caption{Dep. variable: \textbf{Mental and behavioral disorders}} {\def\sym#1{\ifmmode^{#1}\else\(^{#1}\)\fi} \begin{tabular}{l*{13}{c}} \toprule year & \multicolumn{12}{c}{Month of birth} \\ \cmidrule(lr){2-13} 
            &          11&          12&           1&           2&           3&           4&           5&           6&           7&           8&           9&          10\\
1995        &         331&         318&         332&         354&         391&         323&         350&         336&         349&         341&         331&         359\\
1996        &         429&         398&         409&         391&         408&         361&         425&         425&         444&         420&         402&         378\\
1997        &         463&         459&         526&         588&         529&         507&         513&         475&         541&         559&         492&         463\\
1998        &         644&         633&         601&         634&         629&         576&         657&         613&         697&         687&         541&         617\\
1999        &         693&         671&         682&         697&         761&         679&         774&         740&         764&         828&         749&         763\\
2000        &         722&         745&         799&         763&         791&         718&         776&         754&         840&         812&         843&         828\\
2001        &         793&         794&         871&         835&         880&         773&         951&         938&         975&         976&         875&         925\\
2002        &         767&         830&         844&         854&         934&         840&         979&         869&         888&         929&         935&         878\\
2003        &         820&         885&         881&         914&         980&         799&         883&         856&         970&         912&         927&         941\\
2004        &         840&         958&         884&         890&         997&         879&         913&         965&         968&         937&         995&        1006\\
2005        &         816&         879&         977&         886&        1073&         977&         964&         944&         996&        1042&         931&        1040\\
2006        &         897&         950&         939&         957&        1039&        1000&         990&         956&         996&        1046&        1022&         999\\
2007        &         937&         946&         999&        1022&        1081&         998&        1063&        1044&        1101&        1043&        1043&        1022\\
2008        &         944&         955&        1007&        1086&        1084&        1049&        1102&         996&        1075&        1080&        1075&        1106\\
2009        &         935&         973&        1109&        1100&        1139&        1048&        1062&        1207&        1156&        1150&        1112&        1112\\
2010        &        1007&        1046&        1153&        1119&        1165&        1106&        1119&        1145&        1140&        1108&        1093&        1131\\
2011        &        1003&        1070&        1160&        1139&        1209&        1073&        1202&        1148&        1142&        1207&        1133&        1184\\
2012        &        1047&        1153&        1176&        1134&        1203&        1146&        1164&        1118&        1259&        1318&        1264&        1163\\
2013        &        1086&        1181&        1229&        1229&        1267&        1185&        1224&        1170&        1316&        1338&        1287&        1208\\
2014        &        1156&        1164&        1317&        1325&        1323&        1316&        1281&        1253&        1325&        1431&        1353&        1282\\
 \bottomrule \end{tabular} } \begin{tablenotes} \item \scriptsize \emph{Notes:} Number of cases per year and MOB. \end{tablenotes} \end{threeparttable} \end{table} 

\begin{tablenotes}
    \item \scriptsize 
\emph{Notes:} Clustered standard errors in parentheses. ***/**/* indicate significance at the 1\%/5\%/10\% level. Personal covariates contain age, age squared and a gender dummy. Ratios indicate cases per thousand; either approximated population or original number of births.
    \end{tablenotes}
  \end{threeparttable}
\end{table}



\end{document}