%--------------------------------------------------------------------
%	DOCUMENT CLASS
%--------------------------------------------------------------------
\documentclass[11pt, a4paper]{scrartcl} % type of document (paper, presentation, book,...); scrartcl class with sans serif titles, European layout 
\usepackage{fullpage} % leaves less space at margins of page
%\usepackage[left=3cm,right=3cm,top=1.5cm,bottom=1.5cm,includeheadfoot]{geometry}
\usepackage[onehalfspacing]{setspace} % determine line pitch to 1.5

%--------------------------------------------------------------------
%	INPUT
%--------------------------------------------------------------------
\usepackage[T1]{fontenc} 	% Use 8-bit encoding that has 256 glyphs
\usepackage[utf8]{inputenc} % Required for including letters with accents, Umlaute,...
\usepackage{float} 			% better control over placement of tables and figures in the text
\usepackage{graphicx} 		% input of graphics
\usepackage{xcolor} 		% advanced color package
\usepackage{url, hyperref} 	% include (clickable) URLs
\usepackage{pdfpages}		% insert pages of external pdf documents
\usepackage{rotating}		% rotating figures & tables

%--------------------------------------------------------------------
%	TABLES, FIGURES, LISTS
%--------------------------------------------------------------------
\usepackage{booktabs} 		% better tables
\usepackage{longtable}		% tables that may be continued on the next page
\usepackage{tabularx}		% modifies width of certain columns
\usepackage{threeparttable}
\renewcommand\TPTrlap{}
        \renewcommand\TPTnoteSettings{%
            \setlength\leftmargin{5pt}%  
            \setlength\rightmargin{5pt}%
          }

\usepackage[
center, format=plain,
font=normalsize,
nooneline,
labelfont={bf}
]{caption} 				% change format of captions of tables and graphs 
%USED IN MPHIL: \usepackage[labelfont=bf,labelsep = period, singlelinecheck=off,justification=raggedright]{caption}, other specifications which are nice: labelformat = parens -> number in paranthesis

\usepackage[
singlelinecheck=on
]{subcaption}%both together help to have subfigures

% Allow line breaks with \\ in column headings of tables
\newcommand{\clb}[3][c]{%
	\begin{tabular}[#1]{@{}#2@{}}#3\end{tabular}}

% allow line breaks with \\ in row titles
\usepackage{multirow}

\newcommand{\lb}[3][c]{%
\multirow{2}{*}{\begin{tabular}[#1]{@{}#2@{}}#3\end{tabular}}}
% optional argument: b = bottom or t= top alignment

\usepackage{wrapfig}				% wrap text around figure

\usepackage{enumerate}				% change appearance of the enumerator
\usepackage{paralist, enumitem}		% better enumerations
\setlist{noitemsep}					% no additional vertical spacing for enurations

%--------------------------------------------------------------------
%	MATH
%--------------------------------------------------------------------
\usepackage{amsmath,amssymb} % more math symbols and commands

%--------------------------------------------------------------------
%	LANGUAGE SPECIFICS
%--------------------------------------------------------------------
\usepackage[american]{babel} % man­ages cul­tur­ally-de­ter­mined ty­po­graph­i­cal (and other) rules, and hy­phen­ation pat­terns
\usepackage{csquotes} % language specific quotations

%--------------------------------------------------------------------
%	PATHS
%--------------------------------------------------------------------
\makeatletter
\def\input@path{{../../analysis/tables/KKH/}}	%PATH TO TABLES
%or: \def\input@path{{/path/to/folder/}{/path/to/other/folder/}}
\makeatother
\graphicspath{{../../analysis/graphs/KKH/}}		% PATH TO GRAPHS

%--------------------------------------------------------------------
%	LAYOUT
%--------------------------------------------------------------------
\usepackage[left=1cm,right=1cm,top=0.5cm,bottom=2cm]{geometry}
\usepackage{pdflscape} % lscape.sty Produce landscape pages in a (mainly) portrait document.

\definecolor{darkblue}{rgb}{0.0,0.0,0.6}
%--------------------------------------------------------------------
%	TITLE INFORMATION
%--------------------------------------------------------------------
\author{Marc Fabel}
\title{Overview of diagnoses outcomes}
\date{Last revision of this document: \today} 

%%%%%%%%%%%%%%%%%%%%%%%%%%%%%%%%%%%%%%%%%%%%%%%%%%%%%%%%%%%%
% BEGIN OF DOCUMENT
%%%%%%%%%%%%%%%%%%%%%%%%%%%%%%%%%%%%%%%%%%%%%%%%%%%%%%%%%%%%
\begin{document}
\maketitle
This document contains overview of results for different variables. 


%%--------------------------------------------------------------------
%%	PROTOTYP OVERVIEW
%%-------------------------------------------------------------------- 
%---------------------------------
% INPUT FOR VARIABLE: d5
%---------------------------------
%\subsection{d5}
%% RD overview
%\begin{landscape}
%\begin{figure}[H]
%	\centering
%\begin{minipage}{.5\linewidth}
%	\includegraphics[width=\linewidth]{rd_d5_overview_panel2}
%	{\scriptsize \emph{Notes:} The figures show monthly RD plots with a moving average window width of 3 months. The solid vertical line divides pre- and post-reform regime. The averages are taken over the period of at most 1995-2014. \par}
%\end{minipage}
%\end{figure}
%\end{landscape}
%
%%---------------------------------
%% TABELLEN
%\input{d5_DDRD_overview_tfm_scalars}
%%---------------------------------
%% Life-course figure (Panel1)
%\begin{landscape}
%\begin{figure}[htbp]
%\centering
%\begin{minipage}{.9\linewidth}
%\includegraphics[width=\linewidth]{lc_d5_overview_panel1}
%{\scriptsize \emph{Notes:} The figures depict DDRD estimates and 90\% confidence intervals over the life-course. The years are harmonized such that the cohorts are in the same age when they are compared. All regressions are carried out with month-of-birth FE and make use of clustered standard errors. Furthermore, we used a bandwidth of half a year and only the control cohort that was born one year prior to the reform. Ratios indicate cases per thousand; using in the denominator the approximated population (with weights coming from the original fertility distribution) or original number of births. \par}
%\end{minipage}
%\end{figure}
%\end{landscape}
%%---------------------------------
%% PLACEBO EXERCISES
%\newpage
%\begin{landscape}
%\begin{figure}[htbp]
%	\centering
%    \begin{minipage}{.9\linewidth}
%	\includegraphics[width=\linewidth]{placebo_graph_d5.pdf}
%    {\scriptsize \emph{Notes:} The figures depict DDRD estimates and 95\% confidence intervals when the treatment cohort is shifted over time. The date on the abscissa indicates the starting date of the treated.  All regressions are carried out with month-of-birth FE and make use of clustered standard errors. Furthermore, we used a bandwidth of half a year. Ratios indicate cases per thousand; using in the denominator the approximated population (with weights coming from the original fertility distribution) or original number of births. \par}
%    \end{minipage}
%\end{figure}
%\end{landscape}
%\newpage
%\begin{landscape}
%\begin{figure}[htbp]
%	\centering
%    \begin{minipage}{.9\linewidth}
%	\includegraphics[width=\linewidth]{d5_placebo_gdr_DD.pdf}
%    {\scriptsize \emph{Notes:} CORRESPONDS TO GWAP VERSION WITH JUST 1 CG. The figures depict DDRD estimates and 95\% confidence intervals when the treatment cohort is shifted over time. The date on the abscissa indicates the starting date of the treated.  All regressions are carried out with month-of-birth FE and make use of clustered standard errors. Furthermore, we used a bandwidth of half a year. Ratios indicate cases per thousand; using in the denominator the approximated population (with weights coming from the original fertility distribution) or original number of births. \par}
%    \end{minipage}
%\end{figure}
%\end{landscape}
% \begin{table}[H] \centering \begin{threeparttable} \caption{Placebo 2 (CONTROL2 ist TREAT) } {\def\sym#1{\ifmmode^{#1}\else\(^{#1}\)\fi} \begin{tabular}{l*{4}{c}} \toprule \multicolumn{4}{l}{Dep. variable: \textbf{Mental and behavioral disorders}} \\ & \multicolumn{3}{c}{Choice of control group} \\ \cmidrule(lr){2-4}
            &\multicolumn{1}{c}{(1)}&\multicolumn{1}{c}{(2)}&\multicolumn{1}{c}{(3)}\\
            &\multicolumn{1}{c}{C1}&\multicolumn{1}{c}{C3}&\multicolumn{1}{c}{C1+C3}\\
\midrule
 \multicolumn{4}{l}{\emph{Panel A. Average causal effects}} \\ Abs. numbers        &      -4.108         &      -4.758         &      -4.433         \\
                    &     (13.27)         &     (13.07)         &     (13.27)         \\
 Ratio fertility     &       0.162         &       0.898\sym{***}&       0.530         \\
                    &     (0.298)         &     (0.288)         &     (0.327)         \\
 Ratio population    &       0.329         &       0.736\sym{**} &       0.532\sym{*}  \\
                    &     (0.270)         &     (0.272)         &     (0.273)         \\
 Cum. numbers        &      -132.9         &      -91.10         &      -112.0         \\
                    &     (134.7)         &     (126.5)         &     (141.6)         \\
 Cum. ratio          &      -0.397         &       6.571\sym{**} &       3.087         \\
                    &     (2.887)         &     (2.692)         &     (4.005)         \\
 \midrule\multicolumn{4}{l}{\emph{Panel B. Treatment effect heterogeneity - Women}} \\ Abs. numbers        &       6.392         &      -14.47\sym{*}  &      -4.038         \\
                    &     (7.757)         &     (7.175)         &     (9.413)         \\
 Ratio fertility     &       0.457         &       0.285         &       0.371         \\
                    &     (0.362)         &     (0.318)         &     (0.342)         \\
 Ratio population    &       0.355         &     -0.0219         &       0.167         \\
                    &     (0.295)         &     (0.323)         &     (0.323)         \\
 Cum. numbers        &       56.85         &      -108.4         &      -25.80         \\
                    &     (81.94)         &     (64.91)         &     (83.55)         \\
 Cum. ratio          &       4.265         &       3.625         &       3.945         \\
                    &     (3.682)         &     (2.896)         &     (3.274)         \\
 \midrule\multicolumn{4}{l}{\emph{Panel C. Treatment effect heterogeneity - Men}} \\ Abs. numbers        &      -10.50         &       9.708         &      -0.396         \\
                    &     (7.832)         &     (8.916)         &     (9.015)         \\
 Ratio fertility     &      -0.112         &       1.479\sym{***}&       0.684         \\
                    &     (0.325)         &     (0.376)         &     (0.494)         \\
 Ratio population    &       0.294         &       1.494\sym{***}&       0.894\sym{**} \\
                    &     (0.320)         &     (0.346)         &     (0.353)         \\
 Cum. numbers        &      -189.7\sym{**} &       17.34         &      -86.19         \\
                    &     (83.38)         &     (84.90)         &     (128.0)         \\
 Cum. ratio          &      -4.749         &       9.372\sym{**} &       2.311         \\
                    &     (3.326)         &     (3.409)         &     (6.535)         \\
 
\bottomrule \end{tabular} } \begin{tablenotes} \item \scriptsize \emph{Notes:} Clustered standard errors in parentheses. All regression are run with month-of-birth FEs and control cohort 2 is assigned with the treatment status. All regressions are carried out with a window width of half a year. \end{tablenotes} \end{threeparttable} \end{table} 



%--------------------------------------------------------------------
%	MAIN OUTCOMES
%--------------------------------------------------------------------
\section{Main outcomes}
\input{d5_overview_input_graphs_and_figures_SHORT.tex}
\input{d5_DDRD_overview_tfm_scalars_popmzcw.tex}
\input{respiratory_index_overview_input_graphs_and_figures_SHORT.tex}
\input{drug_abuse_overview_input_graphs_and_figures_SHORT.tex}
\input{metabolic_syndrome_overview_input_graphs_and_figures_SHORT.tex}

%--------------------------------------------------------------------
%	META DATEN
%--------------------------------------------------------------------
\section{Meta Daten}
\input{hospital_overview_input_graphs_and_figures_SHORT.tex}
\input{hospital2_overview_input_graphs_and_figures_SHORT.tex}

%--------------------------------------------------------------------
%	HAUPTDIAGNOSEKAPITEL
%--------------------------------------------------------------------
\section{Hauptdiagnosekapitel}
\input{d1_overview_input_graphs_and_figures_SHORT.tex}
\input{d2_overview_input_graphs_and_figures_SHORT.tex}
\input{d6_overview_input_graphs_and_figures_SHORT.tex}
\input{d7_overview_input_graphs_and_figures_SHORT.tex}
\input{d8_overview_input_graphs_and_figures_SHORT.tex}
\input{d9_overview_input_graphs_and_figures_SHORT.tex}
\input{d10_overview_input_graphs_and_figures_SHORT.tex}
\input{d11_overview_input_graphs_and_figures_SHORT.tex}
\input{d12_overview_input_graphs_and_figures_SHORT.tex}
\input{d13_overview_input_graphs_and_figures_SHORT.tex}
\input{d17_overview_input_graphs_and_figures_SHORT.tex}
\input{d18_overview_input_graphs_and_figures_SHORT.tex}




\input{heart_overview_input_graphs_and_figures_SHORT.tex}




\end{document}
