%--------------------------------------------------------------------
%	DOCUMENT CLASS
%--------------------------------------------------------------------
\documentclass[11pt, a4paper]{scrartcl} % type of document (paper, presentation, book,...); scrartcl class with sans serif titles, European layout 
\usepackage{fullpage} % leaves less space at margins of page
%\usepackage[left=3cm,right=3cm,top=1.5cm,bottom=1.5cm,includeheadfoot]{geometry}
\usepackage[onehalfspacing]{setspace} % determine line pitch to 1.5

%--------------------------------------------------------------------
%	INPUT
%--------------------------------------------------------------------
\usepackage[T1]{fontenc} 	% Use 8-bit encoding that has 256 glyphs
\usepackage[utf8]{inputenc} % Required for including letters with accents, Umlaute,...
\usepackage{float} 			% better control over placement of tables and figures in the text
\usepackage{graphicx} 		% input of graphics
\usepackage{xcolor} 		% advanced color package
\usepackage{url, hyperref} 	% include (clickable) URLs
\usepackage{pdfpages}		% insert pages of external pdf documents
\usepackage{rotating}		% rotating figures & tables

%--------------------------------------------------------------------
%	TABLES, FIGURES, LISTS
%--------------------------------------------------------------------
\usepackage{booktabs} 		% better tables
\usepackage{longtable}		% tables that may be continued on the next page
\usepackage{tabularx}		% modifies width of certain columns
\usepackage{threeparttable}
\renewcommand\TPTrlap{}
        \renewcommand\TPTnoteSettings{%
            \setlength\leftmargin{5pt}%  
            \setlength\rightmargin{5pt}%
          }

\usepackage[
center, format=plain,
font=normalsize,
nooneline,
labelfont={bf}
]{caption} 				% change format of captions of tables and graphs 
%USED IN MPHIL: \usepackage[labelfont=bf,labelsep = period, singlelinecheck=off,justification=raggedright]{caption}, other specifications which are nice: labelformat = parens -> number in paranthesis

\usepackage[
singlelinecheck=on
]{subcaption}%both together help to have subfigures

% Allow line breaks with \\ in column headings of tables
\newcommand{\clb}[3][c]{%
	\begin{tabular}[#1]{@{}#2@{}}#3\end{tabular}}

% allow line breaks with \\ in row titles
\usepackage{multirow}

\newcommand{\lb}[3][c]{%
\multirow{2}{*}{\begin{tabular}[#1]{@{}#2@{}}#3\end{tabular}}}
% optional argument: b = bottom or t= top alignment

\usepackage{wrapfig}				% wrap text around figure

\usepackage{enumerate}				% change appearance of the enumerator
\usepackage{paralist, enumitem}		% better enumerations
\setlist{noitemsep}					% no additional vertical spacing for enurations

%--------------------------------------------------------------------
%	MATH
%--------------------------------------------------------------------
\usepackage{amsmath,amssymb} % more math symbols and commands

%--------------------------------------------------------------------
%	LANGUAGE SPECIFICS
%--------------------------------------------------------------------
\usepackage[american]{babel} % man­ages cul­tur­ally-de­ter­mined ty­po­graph­i­cal (and other) rules, and hy­phen­ation pat­terns
\usepackage{csquotes} % language specific quotations

%--------------------------------------------------------------------
%	PATHS
%--------------------------------------------------------------------
\makeatletter
\def\input@path{{../../analysis/tables/KKH/}}	%PATH TO TABLES
%or: \def\input@path{{/path/to/folder/}{/path/to/other/folder/}}
\makeatother
\graphicspath{{../../analysis/graphs/KKH/}}		% PATH TO GRAPHS

%--------------------------------------------------------------------
%	LAYOUT
%--------------------------------------------------------------------
\usepackage[left=1cm,right=1cm,top=2cm,bottom=2cm]{geometry}
\usepackage{pdflscape} % lscape.sty Produce landscape pages in a (mainly) portrait document.

\definecolor{darkblue}{rgb}{0.0,0.0,0.6}

%--------------------------------------------------------------------
%	TITLE INFORMATION
%--------------------------------------------------------------------
\author{Marc Fabel}
\title{Summary of Overview}
\date{Last revision of this document: \today} 

%%%%%%%%%%%%%%%%%%%%%%%%%%%%%%%%%%%%%%%%%%%%%%%%%%%%%%%%%%%%
% BEGIN OF DOCUMENT
%%%%%%%%%%%%%%%%%%%%%%%%%%%%%%%%%%%%%%%%%%%%%%%%%%%%%%%%%%%%
\begin{document}
\maketitle
This document contains overview of results for different variables. Long means here for the moment that the analysis is deeper than in the previous overview document.

\begin{landscape}
\input{summ_stay_ttest_overview.tex}
\input{hospital_ttest_overview.tex}
\input{d1_ttest_overview.tex}
\input{d2_ttest_overview.tex}
\input{d5_ttest_overview.tex}
\input{d6_ttest_overview.tex}
 \begin{table}[H] \begin{threeparttable} \centering \caption{\texttt{DIFFERENCE-IN-MEANS TESTS}} {\def\sym#1{\ifmmode^{#1}\else\(^{#1}\)\fi} \begin{tabular}{l*{13}{c}} \toprule & \multicolumn{12}{c}{Dependent variable: \textbf{Diseases of the eye and ear}} \\ \cmidrule(lr){2-13}
            &\multicolumn{3}{c}{Treatment (Nov78-Oct79)}&\multicolumn{3}{c}{Control 1 (Nov76-Oct77)}&\multicolumn{3}{c}{Control 2 (Nov77-Oct78)}&\multicolumn{3}{c}{Control 3 (Nov79-Oct80)}\\\cmidrule(lr){2-4}\cmidrule(lr){5-7}\cmidrule(lr){8-10}\cmidrule(lr){11-13}
            &\multicolumn{1}{c}{(1)}&\multicolumn{1}{c}{(2)}&\multicolumn{1}{c}{(3)}&\multicolumn{1}{c}{(4)}&\multicolumn{1}{c}{(5)}&\multicolumn{1}{c}{(6)}&\multicolumn{1}{c}{(7)}&\multicolumn{1}{c}{(8)}&\multicolumn{1}{c}{(9)}&\multicolumn{1}{c}{(10)}&\multicolumn{1}{c}{(11)}&\multicolumn{1}{c}{(12)}\\
            &\multicolumn{1}{c}{$\mathbb{E}_{Pre}[Y]$}&\multicolumn{1}{c}{$\mathbb{E}_{Post}[Y]$}&\multicolumn{1}{c}{$\Delta$}&\multicolumn{1}{c}{$\mathbb{E}_{Pre}[Y]$}&\multicolumn{1}{c}{$\mathbb{E}_{Post}[Y]$}&\multicolumn{1}{c}{$\Delta$}&\multicolumn{1}{c}{$\mathbb{E}_{Pre}[Y]$}&\multicolumn{1}{c}{$\mathbb{E}_{Post}[Y]$}&\multicolumn{1}{c}{$\Delta$}&\multicolumn{1}{c}{$\mathbb{E}_{Pre}[Y]$}&\multicolumn{1}{c}{$\mathbb{E}_{Post}[Y]$}&\multicolumn{1}{c}{$\Delta$}\\
\midrule
 \multicolumn{13}{l}{\emph{Panel A. 2 Month bandwidth}} \\ Abs. numbers&       120.2&       116.2&       -3.95         &       125.3&       119.4&       -5.95         &       119.0&       121.7&        2.63         &       115.6&       116.6&        1.02         \\
            &      [18.6]&      [16.8]&      (3.96)         &      [19.8]&      [15.8]&      (4.00)         &      [18.7]&      [15.9]&      (3.88)         &      [14.0]&      [21.2]&      (4.01)         \\
 Ratio fertility&        1.73&        1.68&      -0.050         &        1.81&        1.69&       -0.12\sym{**} &        1.69&        1.73&       0.041         &        1.59&        1.58&     -0.0086         \\
            &      [0.27]&      [0.24]&     (0.057)         &      [0.28]&      [0.22]&     (0.056)         &      [0.26]&      [0.23]&     (0.055)         &      [0.19]&      [0.29]&     (0.055)         \\
 Ratio population&        1.78&        1.74&      -0.045         &        1.90&        1.76&       -0.14\sym{*}  &        1.70&        1.80&       0.093         &        1.67&        1.62&      -0.051         \\
            &      [0.29]&      [0.26]&     (0.079)         &      [0.31]&      [0.26]&     (0.083)         &      [0.27]&      [0.24]&     (0.073)         &      [0.20]&      [0.28]&     (0.071)         \\
 \midrule\multicolumn{13}{l}{\emph{Panel B. 4 Month bandwidth}} \\ Abs. numbers&       114.9&       115.1&        0.19         &       121.4&       116.9&       -4.49         &       115.5&       120.9&        5.42\sym{**} &       114.2&       118.7&        4.52         \\
            &      [18.0]&      [16.5]&      (2.73)         &      [19.4]&      [16.4]&      (2.84)         &      [17.7]&      [14.5]&      (2.56)         &      [15.0]&      [20.6]&      (2.85)         \\
 Ratio fertility&        1.70&        1.64&      -0.060         &        1.82&        1.68&       -0.15\sym{***}&        1.69&        1.74&       0.042         &        1.60&        1.60&     -0.0073         \\
            &      [0.25]&      [0.24]&     (0.039)         &      [0.29]&      [0.23]&     (0.041)         &      [0.25]&      [0.21]&     (0.037)         &      [0.21]&      [0.28]&     (0.039)         \\
 Ratio population&        1.75&        1.70&      -0.051         &        1.89&        1.76&       -0.13\sym{**} &        1.75&        1.82&       0.072         &        1.67&        1.64&      -0.024         \\
            &      [0.26]&      [0.26]&     (0.053)         &      [0.32]&      [0.28]&     (0.061)         &      [0.28]&      [0.24]&     (0.052)         &      [0.22]&      [0.30]&     (0.053)         \\
 \midrule\multicolumn{13}{l}{\emph{Panel C. 6 Month bandwidth}} \\ Abs. numbers&       111.5&       114.2&        2.63         &       118.4&       117.5&       -0.90         &       113.7&       118.5&        4.77\sym{**} &       111.1&       118.4&        7.29\sym{***}\\
            &      [17.5]&      [15.5]&      (2.13)         &      [18.8]&      [16.0]&      (2.25)         &      [16.8]&      [15.9]&      (2.11)         &      [15.6]&      [19.7]&      (2.30)         \\
 Ratio fertility&        1.70&        1.64&      -0.057\sym{*}  &        1.83&        1.71&       -0.12\sym{***}&        1.71&        1.73&       0.016         &        1.61&        1.61&     -0.0026         \\
            &      [0.24]&      [0.22]&     (0.030)         &      [0.28]&      [0.24]&     (0.033)         &      [0.24]&      [0.23]&     (0.030)         &      [0.22]&      [0.27]&     (0.032)         \\
 Ratio population&        1.75&        1.69&      -0.064         &        1.90&        1.79&       -0.11\sym{**} &        1.78&        1.82&       0.041         &        1.65&        1.65&     -0.0061         \\
            &      [0.25]&      [0.23]&     (0.041)         &      [0.31]&      [0.28]&     (0.049)         &      [0.26]&      [0.25]&     (0.043)         &      [0.22]&      [0.29]&     (0.043)         \\
 \midrule\multicolumn{13}{l}{\emph{Panel D. Donut specification}} \\ Abs. numbers&       110.3&       113.2&        2.87         &       118.2&       117.0&       -1.14         &       113.4&       117.5&        4.18\sym{*}  &       111.1&       118.9&        7.78\sym{***}\\
            &      [17.0]&      [15.2]&      (2.28)         &      [18.2]&      [15.9]&      (2.42)         &      [16.4]&      [16.0]&      (2.29)         &      [16.3]&      [19.7]&      (2.56)         \\
 Ratio fertility&        1.69&        1.63&      -0.058\sym{*}  &        1.83&        1.72&       -0.12\sym{***}&        1.72&        1.74&       0.014         &        1.62&        1.62&     -0.0025         \\
            &      [0.24]&      [0.22]&     (0.032)         &      [0.27]&      [0.24]&     (0.036)         &      [0.24]&      [0.23]&     (0.033)         &      [0.23]&      [0.27]&     (0.035)         \\
 Ratio population&        1.75&        1.67&      -0.080\sym{*}  &        1.91&        1.80&       -0.11\sym{**} &        1.81&        1.83&       0.022         &        1.66&        1.66&     -0.0014         \\
            &      [0.25]&      [0.23]&     (0.043)         &      [0.30]&      [0.28]&     (0.053)         &      [0.26]&      [0.25]&     (0.047)         &      [0.23]&      [0.29]&     (0.047)         \\
 
\bottomrule \end{tabular} } \begin{tablenotes} \item \scriptsize \emph{Notes:} . \end{tablenotes} \end{threeparttable} \end{table} 

 \begin{table}[H] \begin{threeparttable} \centering \caption{\texttt{DIFFERENCE-IN-MEANS TESTS}} {\def\sym#1{\ifmmode^{#1}\else\(^{#1}\)\fi} \begin{tabular}{l*{13}{c}} \toprule & \multicolumn{12}{c}{Dependent variable: \textbf{Diseases of the circulatory system}} \\ \cmidrule(lr){2-13}
            &\multicolumn{3}{c}{Treatment (Nov78-Oct79)}&\multicolumn{3}{c}{Control 1 (Nov76-Oct77)}&\multicolumn{3}{c}{Control 2 (Nov77-Oct78)}&\multicolumn{3}{c}{Control 3 (Nov79-Oct80)}\\\cmidrule(lr){2-4}\cmidrule(lr){5-7}\cmidrule(lr){8-10}\cmidrule(lr){11-13}
            &\multicolumn{1}{c}{(1)}&\multicolumn{1}{c}{(2)}&\multicolumn{1}{c}{(3)}&\multicolumn{1}{c}{(4)}&\multicolumn{1}{c}{(5)}&\multicolumn{1}{c}{(6)}&\multicolumn{1}{c}{(7)}&\multicolumn{1}{c}{(8)}&\multicolumn{1}{c}{(9)}&\multicolumn{1}{c}{(10)}&\multicolumn{1}{c}{(11)}&\multicolumn{1}{c}{(12)}\\
            &\multicolumn{1}{c}{$\mathbb{E}_{Pre}[Y]$}&\multicolumn{1}{c}{$\mathbb{E}_{Post}[Y]$}&\multicolumn{1}{c}{$\Delta$}&\multicolumn{1}{c}{$\mathbb{E}_{Pre}[Y]$}&\multicolumn{1}{c}{$\mathbb{E}_{Post}[Y]$}&\multicolumn{1}{c}{$\Delta$}&\multicolumn{1}{c}{$\mathbb{E}_{Pre}[Y]$}&\multicolumn{1}{c}{$\mathbb{E}_{Post}[Y]$}&\multicolumn{1}{c}{$\Delta$}&\multicolumn{1}{c}{$\mathbb{E}_{Pre}[Y]$}&\multicolumn{1}{c}{$\mathbb{E}_{Post}[Y]$}&\multicolumn{1}{c}{$\Delta$}\\
\midrule
 \multicolumn{13}{l}{\emph{Panel A. 2 Month bandwidth}} \\ Abs. numbers&       195.3&       202.6&        7.27         &       230.6&       229.8&       -0.78         &       224.3&       217.0&       -7.33         &       191.8&       191.4&       -0.40         \\
            &      [69.7]&      [64.6]&      (15.0)         &      [75.6]&      [74.6]&      (16.8)         &      [73.6]&      [67.8]&      (15.8)         &      [64.0]&      [62.1]&      (14.1)         \\
 Ratio fertility&        2.81&        2.93&        0.12         &        3.32&        3.25&      -0.077         &        3.18&        3.08&      -0.098         &        2.64&        2.60&      -0.044         \\
            &      [1.00]&      [0.93]&      (0.22)         &      [1.09]&      [1.05]&      (0.24)         &      [1.03]&      [0.96]&      (0.22)         &      [0.88]&      [0.84]&      (0.19)         \\
 Ratio population&        3.59&        3.69&        0.10         &        4.15&        4.05&       -0.10         &        3.96&        3.85&       -0.11         &        3.34&        3.29&      -0.049         \\
            &      [0.93]&      [0.82]&      (0.25)         &      [1.09]&      [1.05]&      (0.31)         &      [1.02]&      [0.90]&      (0.28)         &      [0.78]&      [0.74]&      (0.22)         \\
 \midrule\multicolumn{13}{l}{\emph{Panel B. 4 Month bandwidth}} \\ Abs. numbers&       193.9&       196.2&        2.31         &       223.7&       226.5&        2.84         &       215.8&       212.6&       -3.20         &       188.5&       192.3&        3.79         \\
            &      [65.8]&      [61.4]&      (10.1)         &      [74.9]&      [74.0]&      (11.8)         &      [72.3]&      [66.6]&      (11.0)         &      [63.9]&      [63.2]&      (10.0)         \\
 Ratio fertility&        2.88&        2.80&      -0.078         &        3.36&        3.25&       -0.11         &        3.17&        3.05&       -0.11         &        2.64&        2.58&      -0.061         \\
            &      [0.98]&      [0.88]&      (0.15)         &      [1.12]&      [1.06]&      (0.17)         &      [1.05]&      [0.95]&      (0.16)         &      [0.89]&      [0.85]&      (0.14)         \\
 Ratio population&        3.63&        3.51&       -0.12         &        4.24&        4.07&       -0.17         &        3.98&        3.82&       -0.16         &        3.36&        3.28&      -0.086         \\
            &      [0.93]&      [0.80]&      (0.18)         &      [1.11]&      [1.04]&      (0.22)         &      [1.02]&      [0.90]&      (0.20)         &      [0.80]&      [0.75]&      (0.16)         \\
 \midrule\multicolumn{13}{l}{\emph{Panel C. 6 Month bandwidth}} \\ Abs. numbers&       190.2&       193.9&        3.68         &       221.1&       224.6&        3.49         &       211.3&       208.8&       -2.51         &       184.1&       189.2&        5.05         \\
            &      [63.9]&      [61.0]&      (8.06)         &      [72.0]&      [71.5]&      (9.26)         &      [68.9]&      [64.1]&      (8.59)         &      [61.8]&      [62.2]&      (8.01)         \\
 Ratio fertility&        2.90&        2.78&       -0.12         &        3.41&        3.26&       -0.15         &        3.18&        3.05&       -0.13         &        2.66&        2.56&       -0.10         \\
            &      [0.97]&      [0.88]&      (0.12)         &      [1.11]&      [1.04]&      (0.14)         &      [1.02]&      [0.93]&      (0.13)         &      [0.89]&      [0.84]&      (0.11)         \\
 Ratio population&        3.64&        3.46&       -0.18         &        4.26&        4.06&       -0.21         &        3.97&        3.79&       -0.18         &        3.38&        3.25&       -0.12         \\
            &      [0.93]&      [0.82]&      (0.15)         &      [1.08]&      [1.03]&      (0.18)         &      [0.99]&      [0.88]&      (0.16)         &      [0.80]&      [0.73]&      (0.13)         \\
 \midrule\multicolumn{13}{l}{\emph{Panel D. Donut specification}} \\ Abs. numbers&       190.9&       190.8&       -0.15         &       220.5&       223.2&        2.70         &       210.5&       206.1&       -4.42         &       184.4&       188.1&        3.69         \\
            &      [63.7]&      [59.7]&      (8.73)         &      [71.6]&      [70.4]&      (10.0)         &      [69.1]&      [63.2]&      (9.37)         &      [62.7]&      [62.0]&      (8.82)         \\
 Ratio fertility&        2.93&        2.75&       -0.18         &        3.43&        3.28&       -0.15         &        3.19&        3.04&       -0.15         &        2.69&        2.56&       -0.13         \\
            &      [0.97]&      [0.87]&      (0.13)         &      [1.11]&      [1.03]&      (0.15)         &      [1.03]&      [0.93]&      (0.14)         &      [0.90]&      [0.84]&      (0.12)         \\
 Ratio population&        3.69&        3.42&       -0.27\sym{*}  &        4.28&        4.06&       -0.22         &        3.99&        3.79&       -0.20         &        3.41&        3.25&       -0.16         \\
            &      [0.92]&      [0.81]&      (0.16)         &      [1.08]&      [1.04]&      (0.19)         &      [1.00]&      [0.89]&      (0.17)         &      [0.82]&      [0.72]&      (0.14)         \\
 
\bottomrule \end{tabular} } \begin{tablenotes} \item \scriptsize \emph{Notes:} . \end{tablenotes} \end{threeparttable} \end{table} 

 \begin{table}[H] \begin{threeparttable} \centering \caption{\texttt{DIFFERENCE-IN-MEANS TESTS}} {\def\sym#1{\ifmmode^{#1}\else\(^{#1}\)\fi} \begin{tabular}{l*{13}{c}} \toprule & \multicolumn{12}{c}{Dependent variable: \textbf{Diseases of the respiratory system}} \\ \cmidrule(lr){2-13}
            &\multicolumn{3}{c}{Treatment (Nov78-Oct79)}&\multicolumn{3}{c}{Control 1 (Nov76-Oct77)}&\multicolumn{3}{c}{Control 2 (Nov77-Oct78)}&\multicolumn{3}{c}{Control 3 (Nov79-Oct80)}\\\cmidrule(lr){2-4}\cmidrule(lr){5-7}\cmidrule(lr){8-10}\cmidrule(lr){11-13}
            &\multicolumn{1}{c}{(1)}&\multicolumn{1}{c}{(2)}&\multicolumn{1}{c}{(3)}&\multicolumn{1}{c}{(4)}&\multicolumn{1}{c}{(5)}&\multicolumn{1}{c}{(6)}&\multicolumn{1}{c}{(7)}&\multicolumn{1}{c}{(8)}&\multicolumn{1}{c}{(9)}&\multicolumn{1}{c}{(10)}&\multicolumn{1}{c}{(11)}&\multicolumn{1}{c}{(12)}\\
            &\multicolumn{1}{c}{$\mathbb{E}_{Pre}[Y]$}&\multicolumn{1}{c}{$\mathbb{E}_{Post}[Y]$}&\multicolumn{1}{c}{$\Delta$}&\multicolumn{1}{c}{$\mathbb{E}_{Pre}[Y]$}&\multicolumn{1}{c}{$\mathbb{E}_{Post}[Y]$}&\multicolumn{1}{c}{$\Delta$}&\multicolumn{1}{c}{$\mathbb{E}_{Pre}[Y]$}&\multicolumn{1}{c}{$\mathbb{E}_{Post}[Y]$}&\multicolumn{1}{c}{$\Delta$}&\multicolumn{1}{c}{$\mathbb{E}_{Pre}[Y]$}&\multicolumn{1}{c}{$\mathbb{E}_{Post}[Y]$}&\multicolumn{1}{c}{$\Delta$}\\
\midrule
 \multicolumn{13}{l}{\emph{Panel A. 2 Month bandwidth}} \\ Abs. numbers&       539.4&       527.2&       -12.1         &       517.2&       528.0&        10.8         &       530.9&       533.8&        2.90         &       551.0&       562.2&        11.3         \\
            &      [93.5]&      [85.1]&      (20.0)         &      [92.9]&      [86.8]&      (20.1)         &      [93.4]&     [102.5]&      (21.9)         &      [96.7]&      [89.3]&      (20.8)         \\
 Ratio fertility&        7.76&        7.62&       -0.14         &        7.45&        7.46&      0.0047         &        7.52&        7.57&       0.055         &        7.59&        7.63&       0.043         \\
            &      [1.32]&      [1.23]&      (0.29)         &      [1.33]&      [1.24]&      (0.29)         &      [1.31]&      [1.44]&      (0.31)         &      [1.32]&      [1.21]&      (0.28)         \\
 Ratio population&        7.34&        7.24&      -0.099         &        7.07&        7.13&       0.053         &        7.16&        7.09&      -0.078         &        7.26&        7.35&       0.091         \\
            &      [0.44]&      [0.47]&      (0.13)         &      [0.38]&      [0.50]&      (0.13)         &      [0.46]&      [0.54]&      (0.15)         &      [0.72]&      [0.56]&      (0.19)         \\
 \midrule\multicolumn{13}{l}{\emph{Panel B. 4 Month bandwidth}} \\ Abs. numbers&       524.5&       536.9&        12.4         &       497.8&         524&        26.3\sym{*}  &       519.5&       531.8&        12.3         &       541.4&       566.0&        24.6\sym{*}  \\
            &      [91.4]&      [88.0]&      (14.2)         &      [88.5]&      [90.2]&      (14.1)         &      [93.4]&      [96.8]&      (15.0)         &      [91.9]&      [91.2]&      (14.5)         \\
 Ratio fertility&        7.79&        7.67&       -0.12         &        7.48&        7.52&       0.043         &        7.63&        7.64&      0.0067         &        7.60&        7.61&      0.0096         \\
            &      [1.36]&      [1.25]&      (0.21)         &      [1.31]&      [1.30]&      (0.21)         &      [1.37]&      [1.38]&      (0.22)         &      [1.30]&      [1.24]&      (0.20)         \\
 Ratio population&        7.39&        7.29&      -0.100         &        7.13&        7.14&      0.0025         &        7.26&        7.19&      -0.066         &        7.29&        7.32&       0.035         \\
            &      [0.52]&      [0.46]&      (0.10)         &      [0.51]&      [0.55]&      (0.11)         &      [0.55]&      [0.52]&      (0.11)         &      [0.67]&      [0.62]&      (0.13)         \\
 \midrule\multicolumn{13}{l}{\emph{Panel C. 6 Month bandwidth}} \\ Abs. numbers&       509.6&       537.8&        28.2\sym{**} &       486.2&       524.2&        38.0\sym{***}&       507.6&       526.0&        18.4         &       527.5&       563.6&        36.1\sym{***}\\
            &      [91.9]&      [89.7]&      (11.7)         &      [86.2]&      [90.6]&      (11.4)         &      [92.6]&      [95.3]&      (12.1)         &      [91.1]&      [90.7]&      (11.7)         \\
 Ratio fertility&        7.76&        7.72&      -0.036         &        7.50&        7.62&        0.12         &        7.65&        7.68&       0.032         &        7.63&        7.64&      0.0085         \\
            &      [1.36]&      [1.29]&      (0.17)         &      [1.29]&      [1.33]&      (0.17)         &      [1.37]&      [1.39]&      (0.18)         &      [1.29]&      [1.24]&      (0.16)         \\
 Ratio population&        7.32&        7.32&     -0.0074         &        7.12&        7.23&        0.12         &        7.25&        7.23&      -0.023         &        7.30&        7.36&       0.069         \\
            &      [0.52]&      [0.52]&     (0.087)         &      [0.48]&      [0.53]&     (0.084)         &      [0.52]&      [0.50]&     (0.085)         &      [0.61]&      [0.64]&      (0.10)         \\
 \midrule\multicolumn{13}{l}{\emph{Panel D. Donut specification}} \\ Abs. numbers&       507.7&       538.0&        30.3\sym{**} &       482.4&       523.6&        41.2\sym{***}&       505.4&       522.5&        17.1         &       526.5&       562.6&        36.1\sym{***}\\
            &      [92.9]&      [90.6]&      (13.0)         &      [84.4]&      [91.6]&      (12.5)         &      [92.9]&      [91.5]&      (13.0)         &      [91.8]&      [90.7]&      (12.9)         \\
 Ratio fertility&        7.79&        7.77&      -0.026         &        7.49&        7.68&        0.19         &        7.68&        7.72&       0.042         &        7.68&        7.66&      -0.022         \\
            &      [1.38]&      [1.31]&      (0.19)         &      [1.27]&      [1.35]&      (0.19)         &      [1.38]&      [1.35]&      (0.19)         &      [1.30]&      [1.25]&      (0.18)         \\
 Ratio population&        7.35&        7.35&      0.0049         &        7.12&        7.28&        0.16\sym{*}  &        7.27&        7.29&       0.014         &        7.33&        7.38&       0.048         \\
            &      [0.53]&      [0.53]&     (0.097)         &      [0.52]&      [0.52]&     (0.095)         &      [0.54]&      [0.47]&     (0.092)         &      [0.60]&      [0.65]&      (0.11)         \\
 
\bottomrule \end{tabular} } \begin{tablenotes} \item \scriptsize \emph{Notes:} . \end{tablenotes} \end{threeparttable} \end{table} 

\input{d10_ttest_overview.tex}
\input{d11_ttest_overview.tex}
\input{d12_ttest_overview.tex}
\input{d13_ttest_overview.tex}
\input{d17_ttest_overview.tex}
\input{d18_ttest_overview.tex}
 \begin{table}[H] \begin{threeparttable} \centering \caption{\texttt{DIFFERENCE-IN-MEANS TESTS}} {\def\sym#1{\ifmmode^{#1}\else\(^{#1}\)\fi} \begin{tabular}{l*{13}{c}} \toprule & \multicolumn{12}{c}{Dependent variable: \textbf{Incidence of metabolic Syndrome}} \\ \cmidrule(lr){2-13}
            &\multicolumn{3}{c}{Treatment (Nov78-Oct79)}&\multicolumn{3}{c}{Control 1 (Nov76-Oct77)}&\multicolumn{3}{c}{Control 2 (Nov77-Oct78)}&\multicolumn{3}{c}{Control 3 (Nov79-Oct80)}\\\cmidrule(lr){2-4}\cmidrule(lr){5-7}\cmidrule(lr){8-10}\cmidrule(lr){11-13}
            &\multicolumn{1}{c}{(1)}&\multicolumn{1}{c}{(2)}&\multicolumn{1}{c}{(3)}&\multicolumn{1}{c}{(4)}&\multicolumn{1}{c}{(5)}&\multicolumn{1}{c}{(6)}&\multicolumn{1}{c}{(7)}&\multicolumn{1}{c}{(8)}&\multicolumn{1}{c}{(9)}&\multicolumn{1}{c}{(10)}&\multicolumn{1}{c}{(11)}&\multicolumn{1}{c}{(12)}\\
            &\multicolumn{1}{c}{$\mathbb{E}_{Pre}[Y]$}&\multicolumn{1}{c}{$\mathbb{E}_{Post}[Y]$}&\multicolumn{1}{c}{$\Delta$}&\multicolumn{1}{c}{$\mathbb{E}_{Pre}[Y]$}&\multicolumn{1}{c}{$\mathbb{E}_{Post}[Y]$}&\multicolumn{1}{c}{$\Delta$}&\multicolumn{1}{c}{$\mathbb{E}_{Pre}[Y]$}&\multicolumn{1}{c}{$\mathbb{E}_{Post}[Y]$}&\multicolumn{1}{c}{$\Delta$}&\multicolumn{1}{c}{$\mathbb{E}_{Pre}[Y]$}&\multicolumn{1}{c}{$\mathbb{E}_{Post}[Y]$}&\multicolumn{1}{c}{$\Delta$}\\
\midrule
 \multicolumn{13}{l}{\emph{Panel A. 2 Month bandwidth}} \\ Abs. numbers&        71.6&        74.2&        2.63         &        80.7&        82.4&        1.70         &        77.4&        79.0&        1.55         &          72&        71.5&       -0.53         \\
            &      [26.7]&      [28.7]&      (6.20)         &      [34.7]&      [34.8]&      (7.77)         &      [34.4]&      [29.3]&      (7.14)         &      [25.9]&      [25.4]&      (5.73)         \\
 Ratio fertility&        1.03&        1.07&       0.043         &        1.16&        1.16&      0.0015         &        1.10&        1.12&       0.025         &        0.99&        0.97&      -0.021         \\
            &      [0.38]&      [0.41]&     (0.089)         &      [0.49]&      [0.49]&      (0.11)         &      [0.49]&      [0.42]&      (0.10)         &      [0.36]&      [0.34]&     (0.078)         \\
 Ratio population&        1.24&        1.30&       0.062         &        1.44&        1.47&       0.029         &        1.36&        1.34&      -0.020         &        1.17&        1.16&      -0.018         \\
            &      [0.47]&      [0.50]&      (0.14)         &      [0.59]&      [0.56]&      (0.17)         &      [0.58]&      [0.52]&      (0.16)         &      [0.45]&      [0.43]&      (0.13)         \\
 \midrule\multicolumn{13}{l}{\emph{Panel B. 4 Month bandwidth}} \\ Abs. numbers&        69.5&        73.9&        4.42         &        79.1&        80.4&        1.30         &        77.4&        79.2&        1.85         &        70.8&        74.6&        3.81         \\
            &      [26.6]&      [26.6]&      (4.21)         &      [33.1]&      [34.1]&      (5.31)         &      [32.0]&      [30.8]&      (4.97)         &      [25.9]&      [25.9]&      (4.10)         \\
 Ratio fertility&        1.03&        1.06&       0.025         &        1.19&        1.15&      -0.035         &        1.14&        1.14&      0.0017         &        0.99&        1.00&      0.0090         \\
            &      [0.39]&      [0.38]&     (0.061)         &      [0.50]&      [0.49]&     (0.078)         &      [0.47]&      [0.45]&     (0.072)         &      [0.36]&      [0.34]&     (0.056)         \\
 Ratio population&        1.26&        1.27&       0.016         &        1.48&        1.45&      -0.036         &        1.41&        1.37&      -0.037         &        1.18&        1.18&     -0.0018         \\
            &      [0.47]&      [0.45]&     (0.094)         &      [0.58]&      [0.57]&      (0.12)         &      [0.55]&      [0.55]&      (0.11)         &      [0.44]&      [0.43]&     (0.089)         \\
 \midrule\multicolumn{13}{l}{\emph{Panel C. 6 Month bandwidth}} \\ Abs. numbers&        68.0&        72.3&        4.26         &        78.7&        81.0&        2.33         &        76.3&        76.8&        0.53         &        69.2&        74.3&        5.10         \\
            &      [25.4]&      [25.8]&      (3.31)         &      [32.0]&      [33.4]&      (4.22)         &      [29.7]&      [29.9]&      (3.85)         &      [25.0]&      [25.1]&      (3.24)         \\
 Ratio fertility&        1.03&        1.04&      0.0024         &        1.21&        1.18&      -0.037         &        1.15&        1.12&      -0.028         &        1.00&        1.01&      0.0057         \\
            &      [0.38]&      [0.37]&     (0.049)         &      [0.49]&      [0.49]&     (0.063)         &      [0.44]&      [0.44]&     (0.057)         &      [0.36]&      [0.34]&     (0.045)         \\
 Ratio population&        1.27&        1.24&      -0.027         &        1.51&        1.45&      -0.065         &        1.40&        1.36&      -0.040         &        1.17&        1.18&      0.0098         \\
            &      [0.44]&      [0.45]&     (0.074)         &      [0.57]&      [0.58]&     (0.095)         &      [0.53]&      [0.53]&     (0.088)         &      [0.43]&      [0.42]&     (0.071)         \\
 \midrule\multicolumn{13}{l}{\emph{Panel D. Donut specification}} \\ Abs. numbers&          68&        71.5&        3.54         &        79.2&        80.2&        0.97         &        76.6&        76.6&      -0.020         &        69.2&        74.8&        5.65         \\
            &      [24.9]&      [25.0]&      (3.53)         &      [32.4]&      [33.4]&      (4.65)         &      [29.0]&      [30.7]&      (4.22)         &      [25.7]&      [24.7]&      (3.57)         \\
 Ratio fertility&        1.04&        1.03&      -0.011         &        1.23&        1.18&      -0.054         &        1.16&        1.13&      -0.033         &        1.01&        1.02&      0.0090         \\
            &      [0.38]&      [0.36]&     (0.052)         &      [0.50]&      [0.49]&     (0.070)         &      [0.43]&      [0.45]&     (0.063)         &      [0.37]&      [0.33]&     (0.050)         \\
 Ratio population&        1.28&        1.23&      -0.049         &        1.54&        1.44&      -0.099         &        1.41&        1.37&      -0.039         &        1.18&        1.19&       0.015         \\
            &      [0.43]&      [0.44]&     (0.079)         &      [0.57]&      [0.59]&      (0.11)         &      [0.52]&      [0.55]&     (0.097)         &      [0.44]&      [0.41]&     (0.078)         \\
 
\bottomrule \end{tabular} } \begin{tablenotes} \item \scriptsize \emph{Notes:} . \end{tablenotes} \end{threeparttable} \end{table} 

\input{respiratory_index_ttest_overview.tex}
\input{drug_abuse_ttest_overview.tex}
\input{heart_ttest_overview.tex}
\end{landscape}


\end{document}
