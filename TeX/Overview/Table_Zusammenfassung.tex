%--------------------------------------------------------------------
%	DOCUMENT CLASS
%--------------------------------------------------------------------
\documentclass[11pt, a4paper]{scrartcl} % type of document (paper, presentation, book,...); scrartcl class with sans serif titles, European layout 
\usepackage{fullpage} % leaves less space at margins of page
%\usepackage[left=3cm,right=3cm,top=1.5cm,bottom=1.5cm,includeheadfoot]{geometry}
\usepackage[onehalfspacing]{setspace} % determine line pitch to 1.5

%--------------------------------------------------------------------
%	INPUT
%--------------------------------------------------------------------
\usepackage[T1]{fontenc} 	% Use 8-bit encoding that has 256 glyphs
\usepackage[utf8]{inputenc} % Required for including letters with accents, Umlaute,...
\usepackage{float} 			% better control over placement of tables and figures in the text
\usepackage{graphicx} 		% input of graphics
\usepackage{xcolor} 		% advanced color package
\usepackage{url, hyperref} 	% include (clickable) URLs
\usepackage{pdfpages}		% insert pages of external pdf documents
\usepackage{rotating}		% rotating figures & tables

%--------------------------------------------------------------------
%	TABLES, FIGURES, LISTS
%--------------------------------------------------------------------
\usepackage{booktabs} 		% better tables
\usepackage{longtable}		% tables that may be continued on the next page
\usepackage{tabularx}		% modifies width of certain columns
\usepackage{threeparttable}
\renewcommand\TPTrlap{}
        \renewcommand\TPTnoteSettings{%
            \setlength\leftmargin{5pt}%  
            \setlength\rightmargin{5pt}%
          }

\usepackage[
center, format=plain,
font=normalsize,
nooneline,
labelfont={bf}
]{caption} 				% change format of captions of tables and graphs 
%USED IN MPHIL: \usepackage[labelfont=bf,labelsep = period, singlelinecheck=off,justification=raggedright]{caption}, other specifications which are nice: labelformat = parens -> number in paranthesis

\usepackage[
singlelinecheck=on
]{subcaption}%both together help to have subfigures

% Allow line breaks with \\ in column headings of tables
\newcommand{\clb}[3][c]{%
	\begin{tabular}[#1]{@{}#2@{}}#3\end{tabular}}

% allow line breaks with \\ in row titles
\usepackage{multirow}

\newcommand{\lb}[3][c]{%
\multirow{2}{*}{\begin{tabular}[#1]{@{}#2@{}}#3\end{tabular}}}
% optional argument: b = bottom or t= top alignment

\usepackage{wrapfig}				% wrap text around figure

\usepackage{enumerate}				% change appearance of the enumerator
\usepackage{paralist, enumitem}		% better enumerations
\setlist{noitemsep}					% no additional vertical spacing for enurations

%--------------------------------------------------------------------
%	MATH
%--------------------------------------------------------------------
\usepackage{amsmath,amssymb} % more math symbols and commands

%--------------------------------------------------------------------
%	LANGUAGE SPECIFICS
%--------------------------------------------------------------------
\usepackage[american]{babel} % man­ages cul­tur­ally-de­ter­mined ty­po­graph­i­cal (and other) rules, and hy­phen­ation pat­terns
\usepackage{csquotes} % language specific quotations

%--------------------------------------------------------------------
%	PATHS
%--------------------------------------------------------------------
\makeatletter
\def\input@path{{../../analysis/tables/KKH/}}	%PATH TO TABLES
%or: \def\input@path{{/path/to/folder/}{/path/to/other/folder/}}
\makeatother
\graphicspath{{../../analysis/graphs/KKH/}}		% PATH TO GRAPHS

%--------------------------------------------------------------------
%	LAYOUT
%--------------------------------------------------------------------
\usepackage[left=1cm,right=1cm,top=2cm,bottom=2cm]{geometry}
\usepackage{pdflscape} % lscape.sty Produce landscape pages in a (mainly) portrait document.

\definecolor{darkblue}{rgb}{0.0,0.0,0.6}

%--------------------------------------------------------------------
%	TITLE INFORMATION
%--------------------------------------------------------------------
\author{Marc Fabel}
\title{Summary of Overview}
\date{Last revision of this document: \today} 

%%%%%%%%%%%%%%%%%%%%%%%%%%%%%%%%%%%%%%%%%%%%%%%%%%%%%%%%%%%%
% BEGIN OF DOCUMENT
%%%%%%%%%%%%%%%%%%%%%%%%%%%%%%%%%%%%%%%%%%%%%%%%%%%%%%%%%%%%
\begin{document}
\maketitle
This document contains overview of results for different variables. Long means here for the moment that the analysis is deeper than in the previous overview document.

\begin{landscape}
 \begin{table}[H] \begin{threeparttable} \centering \caption{\texttt{DIFFERENCE-IN-MEANS TESTS}} {\def\sym#1{\ifmmode^{#1}\else\(^{#1}\)\fi} \begin{tabular}{l*{13}{c}} \toprule & \multicolumn{12}{c}{Dependent variable: \textbf{Accumulated length of stay}} \\ \cmidrule(lr){2-13}
            &\multicolumn{3}{c}{Treatment (Nov78-Oct79)}&\multicolumn{3}{c}{Control 1 (Nov76-Oct77)}&\multicolumn{3}{c}{Control 2 (Nov77-Oct78)}&\multicolumn{3}{c}{Control 3 (Nov79-Oct80)}\\\cmidrule(lr){2-4}\cmidrule(lr){5-7}\cmidrule(lr){8-10}\cmidrule(lr){11-13}
            &\multicolumn{1}{c}{(1)}&\multicolumn{1}{c}{(2)}&\multicolumn{1}{c}{(3)}&\multicolumn{1}{c}{(4)}&\multicolumn{1}{c}{(5)}&\multicolumn{1}{c}{(6)}&\multicolumn{1}{c}{(7)}&\multicolumn{1}{c}{(8)}&\multicolumn{1}{c}{(9)}&\multicolumn{1}{c}{(10)}&\multicolumn{1}{c}{(11)}&\multicolumn{1}{c}{(12)}\\
            &\multicolumn{1}{c}{$\mathbb{E}_{Pre}[Y]$}&\multicolumn{1}{c}{$\mathbb{E}_{Post}[Y]$}&\multicolumn{1}{c}{$\Delta$}&\multicolumn{1}{c}{$\mathbb{E}_{Pre}[Y]$}&\multicolumn{1}{c}{$\mathbb{E}_{Post}[Y]$}&\multicolumn{1}{c}{$\Delta$}&\multicolumn{1}{c}{$\mathbb{E}_{Pre}[Y]$}&\multicolumn{1}{c}{$\mathbb{E}_{Post}[Y]$}&\multicolumn{1}{c}{$\Delta$}&\multicolumn{1}{c}{$\mathbb{E}_{Pre}[Y]$}&\multicolumn{1}{c}{$\mathbb{E}_{Post}[Y]$}&\multicolumn{1}{c}{$\Delta$}\\
\midrule
 \multicolumn{13}{l}{\emph{Panel A. 2 Month bandwidth}} \\ Abs. numbers&     57621.9&     57531.6&       -90.3         &     61474.6&     61369.6&      -105.1         &     60499.7&     59432.5&     -1067.2         &     57620.1&     58673.1&      1053.0         \\
            &    [5752.1]&    [5916.8]&    (1304.8)         &    [3756.4]&    [3300.1]&     (790.6)         &    [4173.1]&    [4429.2]&     (962.2)         &    [7834.9]&    [7986.3]&    (1768.9)         \\
 Ratio fertility&       829.1&       831.7&        2.63         &       885.7&       866.8&       -18.9\sym{*}  &       856.7&       843.2&       -13.5         &       794.0&       796.6&        2.67         \\
            &      [79.2]&      [84.5]&      (18.3)         &      [44.9]&      [45.1]&      (10.1)         &      [54.1]&      [61.6]&      (13.0)         &     [106.3]&     [107.6]&      (23.9)         \\
 Ratio population&       935.3&       940.4&        5.12         &       954.1&       944.0&       -10.1         &       945.0&       938.7&       -6.28         &       929.3&       931.9&        2.65         \\
            &      [44.7]&      [44.7]&      (12.9)         &      [32.5]&      [28.8]&      (8.87)         &      [29.8]&      [37.5]&      (9.77)         &      [56.2]&      [54.8]&      (16.0)         \\
 \midrule\multicolumn{13}{l}{\emph{Panel B. 4 Month bandwidth}} \\ Abs. numbers&     56688.6&     58328.0&      1639.4\sym{*}  &     59456.2&     60681.2&      1225.0\sym{**} &     58747.2&     59616.9&       869.7         &     57174.9&     58656.1&      1481.1         \\
            &    [5709.0]&    [6052.8]&     (930.2)         &    [3829.8]&    [3216.3]&     (559.2)         &    [4450.7]&    [4681.1]&     (722.2)         &    [7391.9]&    [7769.3]&    (1199.0)         \\
 Ratio fertility&       842.6&       833.7&       -8.88         &       893.5&       870.9&       -22.6\sym{***}&       862.6&       856.4&       -6.21         &       802.5&       788.3&       -14.2         \\
            &      [86.1]&      [85.3]&      (13.5)         &      [51.8]&      [43.5]&      (7.56)         &      [60.7]&      [67.5]&      (10.1)         &     [105.0]&     [104.6]&      (16.6)         \\
 Ratio population&       949.1&       942.8&       -6.26         &       962.6&       944.4&       -18.3\sym{**} &       950.5&       953.5&        3.03         &       933.4&       922.3&       -11.0         \\
            &      [53.2]&      [45.9]&      (10.1)         &      [41.0]&      [30.1]&      (7.34)         &      [40.5]&      [37.2]&      (7.94)         &      [63.9]&      [57.9]&      (12.4)         \\
 \midrule\multicolumn{13}{l}{\emph{Panel C. 6 Month bandwidth}} \\ Abs. numbers&     55368.3&     58210.4&      2842.1\sym{***}&     58195.0&     60464.6&      2269.6\sym{***}&     57536.8&     58648.2&      1111.4\sym{*}  &     55887.8&     58287.6&      2399.9\sym{**} \\
            &    [5691.2]&    [6313.3]&     (775.9)         &    [3835.0]&    [3408.8]&     (468.4)         &    [4364.1]&    [4865.7]&     (596.7)         &    [7249.2]&    [7685.6]&     (964.5)         \\
 Ratio fertility&       843.3&       835.7&       -7.60         &       897.4&       878.8&       -18.6\sym{***}&       866.7&       856.2&       -10.4         &       808.9&       790.2&       -18.7         \\
            &      [82.4]&      [90.5]&      (11.2)         &      [47.8]&      [52.6]&      (6.49)         &      [56.9]&      [69.7]&      (8.22)         &     [102.3]&     [104.6]&      (13.4)         \\
 Ratio population&       949.2&       949.5&        0.28         &       961.5&       954.6&       -6.82         &       952.2&       954.1&        1.94         &       934.7&       924.1&       -10.5         \\
            &      [47.4]&      [46.8]&      (7.85)         &      [37.2]&      [38.9]&      (6.35)         &      [36.8]&      [41.5]&      (6.53)         &      [61.2]&      [58.9]&      (10.0)         \\
 \midrule\multicolumn{13}{l}{\emph{Panel D. Donut specification}} \\ Abs. numbers&     55289.0&     58110.7&      2821.7\sym{***}&     57961.9&     60101.6&      2139.7\sym{***}&     57282.1&     58281.1&       999.0         &     55921.3&     57994.9&      2073.5\sym{**} \\
            &    [5876.1]&    [6401.9]&     (869.0)         &    [3878.7]&    [3351.2]&     (512.6)         &    [4477.9]&    [4852.4]&     (660.3)         &    [7197.4]&    [7664.7]&    (1051.4)         \\
 Ratio fertility&       848.9&       838.6&       -10.3         &       900.2&       882.2&       -18.0\sym{**} &       869.8&       860.7&       -9.09         &       815.5&       789.1&       -26.4\sym{*}  \\
            &      [83.9]&      [92.0]&      (12.5)         &      [46.8]&      [53.7]&      (7.12)         &      [57.6]&      [70.4]&      (9.10)         &     [100.6]&     [105.1]&      (14.5)         \\
 Ratio population&       955.6&       953.6&       -2.01         &       962.3&       958.0&       -4.26         &       955.2&       959.3&        4.15         &       939.1&       923.5&       -15.6         \\
            &      [46.4]&      [46.3]&      (8.46)         &      [38.2]&      [40.9]&      (7.22)         &      [38.2]&      [41.1]&      (7.24)         &      [62.2]&      [59.0]&      (11.1)         \\
 
\bottomrule \end{tabular} } \begin{tablenotes} \item \scriptsize \emph{Notes:} . \end{tablenotes} \end{threeparttable} \end{table} 

 \begin{table}[H] \begin{threeparttable} \centering \caption{\texttt{DIFFERENCE-IN-MEANS TESTS}} {\def\sym#1{\ifmmode^{#1}\else\(^{#1}\)\fi} \begin{tabular}{l*{13}{c}} \toprule & \multicolumn{12}{c}{Dependent variable: \textbf{Hospital admission}} \\ \cmidrule(lr){2-13}
            &\multicolumn{3}{c}{Treatment (Nov78-Oct79)}&\multicolumn{3}{c}{Control 1 (Nov76-Oct77)}&\multicolumn{3}{c}{Control 2 (Nov77-Oct78)}&\multicolumn{3}{c}{Control 3 (Nov79-Oct80)}\\\cmidrule(lr){2-4}\cmidrule(lr){5-7}\cmidrule(lr){8-10}\cmidrule(lr){11-13}
            &\multicolumn{1}{c}{(1)}&\multicolumn{1}{c}{(2)}&\multicolumn{1}{c}{(3)}&\multicolumn{1}{c}{(4)}&\multicolumn{1}{c}{(5)}&\multicolumn{1}{c}{(6)}&\multicolumn{1}{c}{(7)}&\multicolumn{1}{c}{(8)}&\multicolumn{1}{c}{(9)}&\multicolumn{1}{c}{(10)}&\multicolumn{1}{c}{(11)}&\multicolumn{1}{c}{(12)}\\
            &\multicolumn{1}{c}{$\mathbb{E}_{Pre}[Y]$}&\multicolumn{1}{c}{$\mathbb{E}_{Post}[Y]$}&\multicolumn{1}{c}{$\Delta$}&\multicolumn{1}{c}{$\mathbb{E}_{Pre}[Y]$}&\multicolumn{1}{c}{$\mathbb{E}_{Post}[Y]$}&\multicolumn{1}{c}{$\Delta$}&\multicolumn{1}{c}{$\mathbb{E}_{Pre}[Y]$}&\multicolumn{1}{c}{$\mathbb{E}_{Post}[Y]$}&\multicolumn{1}{c}{$\Delta$}&\multicolumn{1}{c}{$\mathbb{E}_{Pre}[Y]$}&\multicolumn{1}{c}{$\mathbb{E}_{Post}[Y]$}&\multicolumn{1}{c}{$\Delta$}\\
\midrule
 \multicolumn{13}{l}{\emph{Panel A. 2 Month bandwidth}} \\ Abs. numbers&      8691.4&      8727.2&        35.8         &      9064.2&      9094.1&        29.9         &      9002.8&      8936.4&       -66.4         &      8757.1&      8824.4&        67.3         \\
            &    [1476.0]&    [1481.1]&     (330.6)         &     [923.1]&     [966.3]&     (211.3)         &    [1159.1]&    [1246.0]&     (269.1)         &    [1739.6]&    [1787.7]&     (394.4)         \\
 Ratio fertility&       125.1&       126.2&        1.13         &       130.6&       128.5&       -2.16         &       127.5&       126.8&       -0.69         &       120.7&       119.8&       -0.85         \\
            &      [21.0]&      [21.4]&      (4.73)         &      [12.8]&      [13.6]&      (2.96)         &      [16.0]&      [17.6]&      (3.76)         &      [23.8]&      [24.2]&      (5.37)         \\
 Ratio population&       148.4&       149.9&        1.44         &       148.1&       146.7&       -1.49         &       147.6&       148.0&        0.40         &       146.9&       146.5&       -0.43         \\
            &      [12.0]&      [11.9]&      (3.46)         &      [5.72]&      [5.68]&      (1.64)         &      [8.33]&      [9.60]&      (2.59)         &      [14.6]&      [15.0]&      (4.27)         \\
 \midrule\multicolumn{13}{l}{\emph{Panel B. 4 Month bandwidth}} \\ Abs. numbers&      8508.4&      8763.9&       255.5         &      8764.7&      9006.9&       242.1         &      8742.6&      8902.0&       159.4         &      8617.9&      8854.3&       236.4         \\
            &    [1406.2]&    [1484.8]&     (228.6)         &     [921.0]&     [963.8]&     (149.0)         &    [1125.4]&    [1250.7]&     (188.1)         &    [1673.8]&    [1790.2]&     (274.0)         \\
 Ratio fertility&       126.5&       125.3&       -1.18         &       131.7&       129.3&       -2.44         &       128.4&       127.9&       -0.51         &       120.9&       119.0&       -1.95         \\
            &      [21.0]&      [21.2]&      (3.34)         &      [13.4]&      [13.7]&      (2.14)         &      [16.1]&      [17.9]&      (2.70)         &      [23.6]&      [24.0]&      (3.76)         \\
 Ratio population&       149.6&       148.7&       -0.91         &       149.5&       147.6&       -1.93         &       148.6&       149.4&        0.81         &       147.1&       145.4&       -1.62         \\
            &      [12.8]&      [12.2]&      (2.56)         &      [7.30]&      [6.54]&      (1.41)         &      [9.09]&      [9.44]&      (1.89)         &      [14.8]&      [15.3]&      (3.08)         \\
 \midrule\multicolumn{13}{l}{\emph{Panel C. 6 Month bandwidth}} \\ Abs. numbers&      8323.0&      8721.6&       398.6\sym{**} &      8599.6&      8963.9&       364.3\sym{***}&      8576.0&      8789.2&       213.3         &      8401.8&      8792.8&       391.0\sym{*}  \\
            &    [1363.6]&    [1509.0]&     (185.7)         &     [886.7]&     [967.2]&     (119.8)         &    [1089.8]&    [1245.6]&     (151.1)         &    [1632.2]&    [1773.3]&     (220.0)         \\
 Ratio fertility&       126.8&       125.2&       -1.55         &       132.6&       130.3&       -2.35         &       129.2&       128.3&       -0.87         &       121.6&       119.2&       -2.38         \\
            &      [20.4]&      [21.7]&      (2.72)         &      [12.8]&      [14.3]&      (1.76)         &      [15.7]&      [18.1]&      (2.19)         &      [23.2]&      [24.0]&      (3.05)         \\
 Ratio population&       149.6&       149.0&       -0.67         &       149.5&       148.9&       -0.61         &       149.1&       149.9&        0.85         &       147.0&       145.6&       -1.39         \\
            &      [12.3]&      [12.8]&      (2.09)         &      [6.81]&      [7.51]&      (1.19)         &      [8.42]&      [9.92]&      (1.53)         &      [14.5]&      [15.5]&      (2.50)         \\
 \midrule\multicolumn{13}{l}{\emph{Panel D. Donut specification}} \\ Abs. numbers&      8304.8&      8690.7&       385.9\sym{*}  &      8559.4&      8919.2&       359.7\sym{***}&      8545.7&      8727.0&       181.3         &      8383.9&      8757.7&       373.8         \\
            &    [1355.1]&    [1511.1]&     (203.0)         &     [881.1]&     [963.6]&     (130.6)         &    [1085.5]&    [1240.6]&     (164.8)         &    [1621.5]&    [1775.1]&     (240.4)         \\
 Ratio fertility&       127.5&       125.4&       -2.09         &       133.0&       130.9&       -2.05         &       129.8&       128.9&       -0.88         &       122.2&       119.1&       -3.10         \\
            &      [20.3]&      [21.9]&      (2.98)         &      [12.7]&      [14.4]&      (1.93)         &      [15.7]&      [18.3]&      (2.41)         &      [23.1]&      [24.2]&      (3.34)         \\
 Ratio population&       150.3&       149.3&       -1.03         &       149.6&       149.6&       0.057         &       149.6&       150.6&        1.02         &       147.6&       145.7&       -1.92         \\
            &      [12.4]&      [13.0]&      (2.32)         &      [7.03]&      [7.69]&      (1.34)         &      [8.32]&      [9.94]&      (1.67)         &      [14.6]&      [15.6]&      (2.75)         \\
 
\bottomrule \end{tabular} } \begin{tablenotes} \item \scriptsize \emph{Notes:} . \end{tablenotes} \end{threeparttable} \end{table} 

 \begin{table}[H] \begin{threeparttable} \centering \caption{\texttt{DIFFERENCE-IN-MEANS TESTS}} {\def\sym#1{\ifmmode^{#1}\else\(^{#1}\)\fi} \begin{tabular}{l*{13}{c}} \toprule & \multicolumn{12}{c}{Dependent variable: \textbf{Certain infectious and parasitic diseases}} \\ \cmidrule(lr){2-13}
            &\multicolumn{3}{c}{Treatment (Nov78-Oct79)}&\multicolumn{3}{c}{Control 1 (Nov76-Oct77)}&\multicolumn{3}{c}{Control 2 (Nov77-Oct78)}&\multicolumn{3}{c}{Control 3 (Nov79-Oct80)}\\\cmidrule(lr){2-4}\cmidrule(lr){5-7}\cmidrule(lr){8-10}\cmidrule(lr){11-13}
            &\multicolumn{1}{c}{(1)}&\multicolumn{1}{c}{(2)}&\multicolumn{1}{c}{(3)}&\multicolumn{1}{c}{(4)}&\multicolumn{1}{c}{(5)}&\multicolumn{1}{c}{(6)}&\multicolumn{1}{c}{(7)}&\multicolumn{1}{c}{(8)}&\multicolumn{1}{c}{(9)}&\multicolumn{1}{c}{(10)}&\multicolumn{1}{c}{(11)}&\multicolumn{1}{c}{(12)}\\
            &\multicolumn{1}{c}{$\mathbb{E}_{Pre}[Y]$}&\multicolumn{1}{c}{$\mathbb{E}_{Post}[Y]$}&\multicolumn{1}{c}{$\Delta$}&\multicolumn{1}{c}{$\mathbb{E}_{Pre}[Y]$}&\multicolumn{1}{c}{$\mathbb{E}_{Post}[Y]$}&\multicolumn{1}{c}{$\Delta$}&\multicolumn{1}{c}{$\mathbb{E}_{Pre}[Y]$}&\multicolumn{1}{c}{$\mathbb{E}_{Post}[Y]$}&\multicolumn{1}{c}{$\Delta$}&\multicolumn{1}{c}{$\mathbb{E}_{Pre}[Y]$}&\multicolumn{1}{c}{$\mathbb{E}_{Post}[Y]$}&\multicolumn{1}{c}{$\Delta$}\\
\midrule
 \multicolumn{13}{l}{\emph{Panel A. 2 Month bandwidth}} \\ Abs. numbers&       204.2&       201.3&       -2.92         &       198.6&       198.4&       -0.15         &       200.3&       204.9&        4.65         &       207.8&       212.2&        4.33         \\
            &      [26.1]&      [20.9]&      (5.28)         &      [25.8]&      [20.8]&      (5.24)         &      [26.0]&      [24.2]&      (5.61)         &      [25.0]&      [26.9]&      (5.81)         \\
 Ratio fertility&        2.94&        2.91&      -0.026         &        2.86&        2.80&      -0.058         &        2.83&        2.91&       0.074         &        2.86&        2.88&       0.017         \\
            &      [0.36]&      [0.31]&     (0.075)         &      [0.36]&      [0.29]&     (0.073)         &      [0.35]&      [0.35]&     (0.079)         &      [0.34]&      [0.36]&     (0.078)         \\
 Ratio population&        3.08&        3.07&      -0.016         &        2.91&        2.94&       0.032         &        2.88&        3.01&        0.12         &        3.10&        3.07&      -0.035         \\
            &      [0.46]&      [0.32]&      (0.11)         &      [0.35]&      [0.35]&      (0.10)         &      [0.37]&      [0.37]&      (0.11)         &      [0.35]&      [0.39]&      (0.11)         \\
 \midrule\multicolumn{13}{l}{\emph{Panel B. 4 Month bandwidth}} \\ Abs. numbers&       199.0&       205.9&        6.83\sym{*}  &       191.8&       196.3&        4.55         &       197.4&       201.8&        4.42         &       205.4&       212.8&        7.36\sym{*}  \\
            &      [25.1]&      [24.1]&      (3.89)         &      [23.8]&      [22.3]&      (3.65)         &      [23.6]&      [24.4]&      (3.79)         &      [23.9]&      [27.2]&      (4.05)         \\
 Ratio fertility&        2.96&        2.94&      -0.015         &        2.88&        2.82&      -0.064         &        2.90&        2.90&    -0.00016         &        2.88&        2.86&      -0.024         \\
            &      [0.37]&      [0.34]&     (0.056)         &      [0.35]&      [0.32]&     (0.053)         &      [0.35]&      [0.35]&     (0.055)         &      [0.34]&      [0.36]&     (0.055)         \\
 Ratio population&        3.06&        3.09&       0.029         &        2.95&        2.91&      -0.042         &        2.96&        3.01&       0.044         &        3.11&        3.05&      -0.063         \\
            &      [0.43]&      [0.34]&     (0.079)         &      [0.37]&      [0.34]&     (0.073)         &      [0.36]&      [0.36]&     (0.073)         &      [0.35]&      [0.40]&     (0.077)         \\
 \midrule\multicolumn{13}{l}{\emph{Panel C. 6 Month bandwidth}} \\ Abs. numbers&       195.1&       206.3&        11.2\sym{***}&       188.9&       196.8&        7.88\sym{***}&       194.2&       200.2&        6.05\sym{**} &       199.7&       211.6&        11.9\sym{***}\\
            &      [23.4]&      [23.2]&      (3.01)         &      [23.2]&      [22.1]&      (2.92)         &      [23.3]&      [23.0]&      (2.98)         &      [24.2]&      [26.7]&      (3.30)         \\
 Ratio fertility&        2.97&        2.96&     -0.0094         &        2.91&        2.86&      -0.053         &        2.93&        2.92&     -0.0019         &        2.89&        2.87&      -0.021         \\
            &      [0.34]&      [0.33]&     (0.044)         &      [0.35]&      [0.33]&     (0.044)         &      [0.34]&      [0.34]&     (0.044)         &      [0.33]&      [0.36]&     (0.045)         \\
 Ratio population&        3.09&        3.11&       0.021         &        2.98&        2.95&      -0.025         &        2.98&        3.03&       0.045         &        3.08&        3.04&      -0.043         \\
            &      [0.38]&      [0.33]&     (0.059)         &      [0.37]&      [0.35]&     (0.060)         &      [0.34]&      [0.34]&     (0.057)         &      [0.34]&      [0.38]&     (0.060)         \\
 \midrule\multicolumn{13}{l}{\emph{Panel D. Donut specification}} \\ Abs. numbers&       195.2&         207&        11.8\sym{***}&       188.2&       196.1&        7.94\sym{**} &       195.0&       199.2&        4.26         &       199.6&       210.9&        11.3\sym{***}\\
            &      [23.6]&      [22.8]&      (3.28)         &      [23.3]&      [22.3]&      (3.22)         &      [22.5]&      [22.7]&      (3.20)         &      [24.0]&      [26.9]&      (3.61)         \\
 Ratio fertility&        3.00&        2.99&     -0.0090         &        2.92&        2.88&      -0.045         &        2.96&        2.94&      -0.018         &        2.91&        2.87&      -0.041         \\
            &      [0.34]&      [0.33]&     (0.047)         &      [0.35]&      [0.34]&     (0.048)         &      [0.32]&      [0.34]&     (0.047)         &      [0.32]&      [0.36]&     (0.049)         \\
 Ratio population&        3.10&        3.13&       0.029         &        2.98&        2.97&      -0.017         &        3.02&        3.05&       0.022         &        3.09&        3.04&      -0.050         \\
            &      [0.36]&      [0.31]&     (0.062)         &      [0.37]&      [0.35]&     (0.066)         &      [0.32]&      [0.34]&     (0.060)         &      [0.34]&      [0.38]&     (0.066)         \\
 
\bottomrule \end{tabular} } \begin{tablenotes} \item \scriptsize \emph{Notes:} . \end{tablenotes} \end{threeparttable} \end{table} 

 \begin{table}[H] \begin{threeparttable} \centering \caption{\texttt{DIFFERENCE-IN-MEANS TESTS}} {\def\sym#1{\ifmmode^{#1}\else\(^{#1}\)\fi} \begin{tabular}{l*{13}{c}} \toprule & \multicolumn{12}{c}{Dependent variable: \textbf{Neoplasms}} \\ \cmidrule(lr){2-13}
            &\multicolumn{3}{c}{Treatment (Nov78-Oct79)}&\multicolumn{3}{c}{Control 1 (Nov76-Oct77)}&\multicolumn{3}{c}{Control 2 (Nov77-Oct78)}&\multicolumn{3}{c}{Control 3 (Nov79-Oct80)}\\\cmidrule(lr){2-4}\cmidrule(lr){5-7}\cmidrule(lr){8-10}\cmidrule(lr){11-13}
            &\multicolumn{1}{c}{(1)}&\multicolumn{1}{c}{(2)}&\multicolumn{1}{c}{(3)}&\multicolumn{1}{c}{(4)}&\multicolumn{1}{c}{(5)}&\multicolumn{1}{c}{(6)}&\multicolumn{1}{c}{(7)}&\multicolumn{1}{c}{(8)}&\multicolumn{1}{c}{(9)}&\multicolumn{1}{c}{(10)}&\multicolumn{1}{c}{(11)}&\multicolumn{1}{c}{(12)}\\
            &\multicolumn{1}{c}{$\mathbb{E}_{Pre}[Y]$}&\multicolumn{1}{c}{$\mathbb{E}_{Post}[Y]$}&\multicolumn{1}{c}{$\Delta$}&\multicolumn{1}{c}{$\mathbb{E}_{Pre}[Y]$}&\multicolumn{1}{c}{$\mathbb{E}_{Post}[Y]$}&\multicolumn{1}{c}{$\Delta$}&\multicolumn{1}{c}{$\mathbb{E}_{Pre}[Y]$}&\multicolumn{1}{c}{$\mathbb{E}_{Post}[Y]$}&\multicolumn{1}{c}{$\Delta$}&\multicolumn{1}{c}{$\mathbb{E}_{Pre}[Y]$}&\multicolumn{1}{c}{$\mathbb{E}_{Post}[Y]$}&\multicolumn{1}{c}{$\Delta$}\\
\midrule
 \multicolumn{13}{l}{\emph{Panel A. 2 Month bandwidth}} \\ Abs. numbers&       244.3&       242.0&       -2.28         &       275.9&       276.8&        0.85         &       279.2&       271.8&       -7.43         &       235.3&       249.6&        14.3         \\
            &      [55.1]&      [63.0]&      (13.2)         &      [73.0]&      [67.9]&      (15.8)         &      [65.1]&      [76.7]&      (15.9)         &      [62.4]&      [52.9]&      (12.9)         \\
 Ratio fertility&        3.52&        3.50&      -0.017         &        3.98&        3.91&      -0.066         &        3.95&        3.85&       -0.10         &        3.24&        3.39&        0.15         \\
            &      [0.79]&      [0.91]&      (0.19)         &      [1.04]&      [0.96]&      (0.22)         &      [0.92]&      [1.08]&      (0.22)         &      [0.86]&      [0.71]&      (0.18)         \\
 Ratio population&        4.09&        4.13&       0.039         &        4.71&        4.62&      -0.090         &        4.48&        4.43&      -0.053         &        3.88&        3.92&       0.036         \\
            &      [0.90]&      [1.02]&      (0.28)         &      [1.20]&      [1.08]&      (0.33)         &      [1.10]&      [1.35]&      (0.35)         &      [0.95]&      [0.81]&      (0.25)         \\
 \midrule\multicolumn{13}{l}{\emph{Panel B. 4 Month bandwidth}} \\ Abs. numbers&       241.0&       244.3&        3.30         &       270.1&       276.0&        5.85         &       269.5&         268&       -1.46         &       237.2&       245.9&        8.70         \\
            &      [59.0]&      [60.8]&      (9.47)         &      [75.8]&      [68.0]&      (11.4)         &      [64.0]&      [70.3]&      (10.6)         &      [60.1]&      [53.9]&      (9.02)         \\
 Ratio fertility&        3.58&        3.49&      -0.090         &        4.06&        3.96&      -0.100         &        3.96&        3.85&       -0.11         &        3.33&        3.30&      -0.025         \\
            &      [0.89]&      [0.87]&      (0.14)         &      [1.15]&      [0.98]&      (0.17)         &      [0.93]&      [1.00]&      (0.15)         &      [0.85]&      [0.73]&      (0.12)         \\
 Ratio population&        4.19&        4.10&      -0.088         &        4.85&        4.72&       -0.13         &        4.53&        4.46&      -0.077         &        3.95&        3.80&       -0.15         \\
            &      [1.01]&      [0.96]&      (0.20)         &      [1.31]&      [1.07]&      (0.24)         &      [1.13]&      [1.20]&      (0.24)         &      [0.97]&      [0.84]&      (0.19)         \\
 \midrule\multicolumn{13}{l}{\emph{Panel C. 6 Month bandwidth}} \\ Abs. numbers&       238.8&       243.8&        5.05         &       267.3&       274.8&        7.57         &       265.0&       258.5&       -6.47         &       231.9&       245.4&        13.5\sym{*}  \\
            &      [59.1]&      [61.1]&      (7.76)         &      [73.3]&      [68.9]&      (9.19)         &      [65.1]&      [69.3]&      (8.68)         &      [58.0]&      [53.1]&      (7.18)         \\
 Ratio fertility&        3.64&        3.50&       -0.14         &        4.13&        4.00&       -0.13         &        3.99&        3.77&       -0.22\sym{*}  &        3.36&        3.33&      -0.030         \\
            &      [0.91]&      [0.88]&      (0.12)         &      [1.14]&      [1.01]&      (0.14)         &      [0.97]&      [0.99]&      (0.13)         &      [0.83]&      [0.72]&      (0.10)         \\
 Ratio population&        4.22&        4.12&       -0.10         &        4.94&        4.76&       -0.18         &        4.60&        4.40&       -0.20         &        3.96&        3.78&       -0.18         \\
            &      [1.04]&      [0.95]&      (0.17)         &      [1.24]&      [1.12]&      (0.20)         &      [1.17]&      [1.15]&      (0.19)         &      [0.95]&      [0.86]&      (0.15)         \\
 \midrule\multicolumn{13}{l}{\emph{Panel D. Donut specification}} \\ Abs. numbers&       238.5&       243.4&        4.93         &       267.2&       274.4&        7.21         &       264.1&       253.3&       -10.8         &       232.2&       243.1&        10.9         \\
            &      [59.9]&      [60.5]&      (8.51)         &      [74.4]&      [69.6]&      (10.2)         &      [64.3]&      [66.7]&      (9.26)         &      [57.5]&      [53.5]&      (7.85)         \\
 Ratio fertility&        3.66&        3.51&       -0.15         &        4.15&        4.03&       -0.13         &        4.01&        3.74&       -0.27\sym{*}  &        3.39&        3.31&      -0.079         \\
            &      [0.92]&      [0.87]&      (0.13)         &      [1.16]&      [1.03]&      (0.15)         &      [0.97]&      [0.98]&      (0.14)         &      [0.83]&      [0.73]&      (0.11)         \\
 Ratio population&        4.25&        4.12&       -0.13         &        4.99&        4.81&       -0.17         &        4.63&        4.38&       -0.25         &        3.97&        3.74&       -0.23         \\
            &      [1.06]&      [0.96]&      (0.18)         &      [1.25]&      [1.12]&      (0.22)         &      [1.18]&      [1.12]&      (0.21)         &      [0.96]&      [0.87]&      (0.17)         \\
 
\bottomrule \end{tabular} } \begin{tablenotes} \item \scriptsize \emph{Notes:} . \end{tablenotes} \end{threeparttable} \end{table} 

 \begin{table}[H] \begin{threeparttable} \centering \caption{\texttt{DIFFERENCE-IN-MEANS TESTS}} {\def\sym#1{\ifmmode^{#1}\else\(^{#1}\)\fi} \begin{tabular}{l*{13}{c}} \toprule & \multicolumn{12}{c}{Dependent variable: \textbf{Mental and behavioral disorders}} \\ \cmidrule(lr){2-13}
            &\multicolumn{3}{c}{Treatment (Nov78-Oct79)}&\multicolumn{3}{c}{Control 1 (Nov76-Oct77)}&\multicolumn{3}{c}{Control 2 (Nov77-Oct78)}&\multicolumn{3}{c}{Control 3 (Nov79-Oct80)}\\\cmidrule(lr){2-4}\cmidrule(lr){5-7}\cmidrule(lr){8-10}\cmidrule(lr){11-13}
            &\multicolumn{1}{c}{(1)}&\multicolumn{1}{c}{(2)}&\multicolumn{1}{c}{(3)}&\multicolumn{1}{c}{(4)}&\multicolumn{1}{c}{(5)}&\multicolumn{1}{c}{(6)}&\multicolumn{1}{c}{(7)}&\multicolumn{1}{c}{(8)}&\multicolumn{1}{c}{(9)}&\multicolumn{1}{c}{(10)}&\multicolumn{1}{c}{(11)}&\multicolumn{1}{c}{(12)}\\
            &\multicolumn{1}{c}{$\mathbb{E}_{Pre}[Y]$}&\multicolumn{1}{c}{$\mathbb{E}_{Post}[Y]$}&\multicolumn{1}{c}{$\Delta$}&\multicolumn{1}{c}{$\mathbb{E}_{Pre}[Y]$}&\multicolumn{1}{c}{$\mathbb{E}_{Post}[Y]$}&\multicolumn{1}{c}{$\Delta$}&\multicolumn{1}{c}{$\mathbb{E}_{Pre}[Y]$}&\multicolumn{1}{c}{$\mathbb{E}_{Post}[Y]$}&\multicolumn{1}{c}{$\Delta$}&\multicolumn{1}{c}{$\mathbb{E}_{Pre}[Y]$}&\multicolumn{1}{c}{$\mathbb{E}_{Post}[Y]$}&\multicolumn{1}{c}{$\Delta$}\\
\midrule
 \multicolumn{13}{l}{\emph{Panel A. 2 Month bandwidth}} \\ Abs. numbers&       905.9&       908.6&        2.70         &       926.2&       935.7&        9.50         &       927.2&       914.6&       -12.6         &       909.5&       945.6&        36.1         \\
            &     [275.3]&     [263.0]&      (60.2)         &     [217.5]&     [217.7]&      (48.6)         &     [230.5]&     [256.3]&      (54.5)         &     [293.8]&     [319.8]&      (68.7)         \\
 Ratio fertility&        13.0&        13.1&        0.11         &        13.3&        13.2&       -0.14         &        13.1&        13.0&       -0.16         &        12.5&        12.8&        0.31         \\
            &      [3.94]&      [3.81]&      (0.87)         &      [3.12]&      [3.04]&      (0.69)         &      [3.24]&      [3.62]&      (0.77)         &      [4.04]&      [4.34]&      (0.94)         \\
 Ratio population&        16.8&        16.7&       -0.15         &        16.4&        16.3&      -0.072         &        16.3&        16.4&        0.14         &        16.4&        17.0&        0.62         \\
            &      [1.78]&      [1.85]&      (0.52)         &      [1.67]&      [1.49]&      (0.46)         &      [1.62]&      [2.08]&      (0.54)         &      [2.05]&      [2.01]&      (0.59)         \\
 \midrule\multicolumn{13}{l}{\emph{Panel B. 4 Month bandwidth}} \\ Abs. numbers&       900.6&       930.6&        30.0         &       902.7&       925.1&        22.4         &       900.2&       942.5&        42.3         &       911.5&       942.5&        31.0         \\
            &     [268.8]&     [270.2]&      (42.6)         &     [211.9]&     [222.5]&      (34.4)         &     [225.1]&     [261.7]&      (38.6)         &     [288.6]&     [310.0]&      (47.4)         \\
 Ratio fertility&        13.4&        13.3&      -0.089         &        13.6&        13.3&       -0.30         &        13.2&        13.5&        0.33         &        12.8&        12.7&       -0.13         \\
            &      [4.02]&      [3.86]&      (0.62)         &      [3.19]&      [3.18]&      (0.50)         &      [3.28]&      [3.79]&      (0.56)         &      [4.06]&      [4.17]&      (0.65)         \\
 Ratio population&        17.2&        16.9&       -0.32         &        16.7&        16.4&       -0.26         &        16.4&        17.1&        0.68\sym{*}  &        16.7&        16.7&      -0.030         \\
            &      [2.06]&      [1.95]&      (0.41)         &      [1.85]&      [1.77]&      (0.37)         &      [1.69]&      [2.11]&      (0.39)         &      [1.93]&      [2.02]&      (0.40)         \\
 \midrule\multicolumn{13}{l}{\emph{Panel C. 6 Month bandwidth}} \\ Abs. numbers&       878.2&       927.1&        48.9         &       883.5&       926.4&        42.8         &       887.5&       926.2&        38.7         &       895.6&       939.0&        43.5         \\
            &     [258.5]&     [271.9]&      (34.2)         &     [208.3]&     [227.1]&      (28.1)         &     [223.1]&     [259.1]&      (31.2)         &     [281.8]&     [307.6]&      (38.1)         \\
 Ratio fertility&        13.4&        13.3&      -0.067         &        13.6&        13.5&       -0.16         &        13.4&        13.5&        0.15         &        13.0&        12.7&       -0.23         \\
            &      [3.91]&      [3.90]&      (0.50)         &      [3.19]&      [3.32]&      (0.42)         &      [3.34]&      [3.79]&      (0.46)         &      [4.08]&      [4.18]&      (0.53)         \\
 Ratio population&        17.1&        17.0&       -0.15         &        16.7&        16.7&       0.041         &        16.6&        17.1&        0.46         &        16.8&        16.8&      -0.061         \\
            &      [1.95]&      [1.87]&      (0.32)         &      [1.94]&      [1.88]&      (0.32)         &      [1.77]&      [2.14]&      (0.33)         &      [1.94]&      [2.01]&      (0.33)         \\
 \midrule\multicolumn{13}{l}{\emph{Panel D. Donut specification}} \\ Abs. numbers&       880.3&       928.7&        48.3         &       879.6&       918.2&        38.6         &       884.6&       924.1&        39.5         &       898.8&       935.6&        36.8         \\
            &     [256.6]&     [274.6]&      (37.6)         &     [206.7]&     [228.0]&      (30.8)         &     [224.5]&     [259.4]&      (34.3)         &     [281.6]&     [305.8]&      (41.6)         \\
 Ratio fertility&        13.5&        13.4&       -0.12         &        13.7&        13.5&       -0.19         &        13.4&        13.6&        0.21         &        13.1&        12.7&       -0.38         \\
            &      [3.89]&      [3.95]&      (0.55)         &      [3.18]&      [3.38]&      (0.46)         &      [3.38]&      [3.83]&      (0.51)         &      [4.09]&      [4.17]&      (0.58)         \\
 Ratio population&        17.3&        17.1&       -0.14         &        16.7&        16.7&       0.065         &        16.7&        17.2&        0.49         &        17.0&        16.8&       -0.23         \\
            &      [1.93]&      [1.87]&      (0.35)         &      [2.01]&      [1.97]&      (0.36)         &      [1.76]&      [2.14]&      (0.36)         &      [1.86]&      [1.97]&      (0.35)         \\
 
\bottomrule \end{tabular} } \begin{tablenotes} \item \scriptsize \emph{Notes:} . \end{tablenotes} \end{threeparttable} \end{table} 

 \begin{table}[H] \begin{threeparttable} \centering \caption{\texttt{DIFFERENCE-IN-MEANS TESTS}} {\def\sym#1{\ifmmode^{#1}\else\(^{#1}\)\fi} \begin{tabular}{l*{13}{c}} \toprule & \multicolumn{12}{c}{Dependent variable: \textbf{Diseases of the nervous system}} \\ \cmidrule(lr){2-13}
            &\multicolumn{3}{c}{Treatment (Nov78-Oct79)}&\multicolumn{3}{c}{Control 1 (Nov76-Oct77)}&\multicolumn{3}{c}{Control 2 (Nov77-Oct78)}&\multicolumn{3}{c}{Control 3 (Nov79-Oct80)}\\\cmidrule(lr){2-4}\cmidrule(lr){5-7}\cmidrule(lr){8-10}\cmidrule(lr){11-13}
            &\multicolumn{1}{c}{(1)}&\multicolumn{1}{c}{(2)}&\multicolumn{1}{c}{(3)}&\multicolumn{1}{c}{(4)}&\multicolumn{1}{c}{(5)}&\multicolumn{1}{c}{(6)}&\multicolumn{1}{c}{(7)}&\multicolumn{1}{c}{(8)}&\multicolumn{1}{c}{(9)}&\multicolumn{1}{c}{(10)}&\multicolumn{1}{c}{(11)}&\multicolumn{1}{c}{(12)}\\
            &\multicolumn{1}{c}{$\mathbb{E}_{Pre}[Y]$}&\multicolumn{1}{c}{$\mathbb{E}_{Post}[Y]$}&\multicolumn{1}{c}{$\Delta$}&\multicolumn{1}{c}{$\mathbb{E}_{Pre}[Y]$}&\multicolumn{1}{c}{$\mathbb{E}_{Post}[Y]$}&\multicolumn{1}{c}{$\Delta$}&\multicolumn{1}{c}{$\mathbb{E}_{Pre}[Y]$}&\multicolumn{1}{c}{$\mathbb{E}_{Post}[Y]$}&\multicolumn{1}{c}{$\Delta$}&\multicolumn{1}{c}{$\mathbb{E}_{Pre}[Y]$}&\multicolumn{1}{c}{$\mathbb{E}_{Post}[Y]$}&\multicolumn{1}{c}{$\Delta$}\\
\midrule
 \multicolumn{13}{l}{\emph{Panel A. 2 Month bandwidth}} \\ Abs. numbers&       216.9&       226.5&        9.60         &       237.6&       249.1&        11.4         &       226.9&       232.2&        5.22         &       232.8&       237.4&        4.63         \\
            &      [65.6]&      [62.2]&      (14.3)         &      [75.2]&      [65.2]&      (15.7)         &      [66.8]&      [71.0]&      (15.4)         &      [63.9]&      [66.3]&      (14.6)         \\
 Ratio fertility&        3.12&        3.28&        0.16         &        3.42&        3.52&       0.095         &        3.21&        3.30&       0.083         &        3.21&        3.22&       0.014         \\
            &      [0.93]&      [0.91]&      (0.21)         &      [1.07]&      [0.92]&      (0.22)         &      [0.94]&      [1.02]&      (0.22)         &      [0.88]&      [0.90]&      (0.20)         \\
 Ratio population&        3.94&        4.06&        0.12         &        4.33&        4.34&       0.010         &        4.08&        4.14&       0.060         &        4.02&        4.02&      0.0049         \\
            &      [0.81]&      [0.81]&      (0.23)         &      [0.94]&      [0.76]&      (0.25)         &      [0.74]&      [0.97]&      (0.25)         &      [0.81]&      [0.81]&      (0.23)         \\
 \midrule\multicolumn{13}{l}{\emph{Panel B. 4 Month bandwidth}} \\ Abs. numbers&       212.9&       230.8&        18.0\sym{*}  &       231.6&       245.9&        14.3         &       218.1&       229.0&        10.9         &       222.3&       234.1&        11.8         \\
            &      [62.5]&      [61.2]&      (9.78)         &      [65.4]&      [64.7]&      (10.3)         &      [62.3]&      [64.4]&      (10.0)         &      [59.5]&      [66.6]&      (9.99)         \\
 Ratio fertility&        3.16&        3.30&        0.14         &        3.48&        3.53&       0.049         &        3.20&        3.29&       0.089         &        3.12&        3.14&       0.026         \\
            &      [0.93]&      [0.88]&      (0.14)         &      [0.96]&      [0.93]&      (0.15)         &      [0.90]&      [0.92]&      (0.14)         &      [0.83]&      [0.89]&      (0.14)         \\
 Ratio population&        3.98&        4.09&        0.12         &        4.32&        4.33&       0.013         &        4.01&        4.09&       0.079         &        3.89&        3.95&       0.065         \\
            &      [0.82]&      [0.73]&      (0.16)         &      [0.83]&      [0.83]&      (0.17)         &      [0.75]&      [0.86]&      (0.16)         &      [0.76]&      [0.81]&      (0.16)         \\
 \midrule\multicolumn{13}{l}{\emph{Panel C. 6 Month bandwidth}} \\ Abs. numbers&       208.6&       227.7&        19.1\sym{**} &       226.5&       241.7&        15.2\sym{*}  &       216.9&       226.0&        9.08         &       216.8&       232.5&        15.8\sym{*}  \\
            &      [59.1]&      [61.4]&      (7.78)         &      [64.4]&      [64.7]&      (8.33)         &      [61.0]&      [62.7]&      (7.99)         &      [58.9]&      [67.2]&      (8.15)         \\
 Ratio fertility&        3.18&        3.27&       0.093         &        3.49&        3.51&       0.019         &        3.27&        3.30&       0.031         &        3.14&        3.15&       0.016         \\
            &      [0.89]&      [0.88]&      (0.11)         &      [0.97]&      [0.94]&      (0.12)         &      [0.91]&      [0.91]&      (0.12)         &      [0.84]&      [0.91]&      (0.11)         \\
 Ratio population&        3.97&        4.07&        0.10         &        4.32&        4.32&     0.00038         &        4.09&        4.11&       0.019         &        3.91&        3.97&       0.057         \\
            &      [0.79]&      [0.73]&      (0.13)         &      [0.84]&      [0.83]&      (0.14)         &      [0.77]&      [0.83]&      (0.13)         &      [0.77]&      [0.84]&      (0.13)         \\
 \midrule\multicolumn{13}{l}{\emph{Panel D. Donut specification}} \\ Abs. numbers&       208.9&       228.1&        19.2\sym{**} &       226.3&       239.9&        13.7         &       216.1&       225.2&        9.15         &       214.1&       230.0&        15.9\sym{*}  \\
            &      [59.7]&      [61.4]&      (8.56)         &      [63.5]&      [64.2]&      (9.04)         &      [60.7]&      [61.7]&      (8.65)         &      [57.9]&      [68.2]&      (8.95)         \\
 Ratio fertility&        3.21&        3.29&       0.086         &        3.51&        3.52&      0.0071         &        3.28&        3.33&       0.043         &        3.12&        3.13&      0.0070         \\
            &      [0.90]&      [0.89]&      (0.13)         &      [0.96]&      [0.94]&      (0.13)         &      [0.92]&      [0.91]&      (0.13)         &      [0.84]&      [0.93]&      (0.13)         \\
 Ratio population&        4.01&        4.09&       0.077         &        4.33&        4.33&      0.0065         &        4.10&        4.14&       0.042         &        3.88&        3.96&       0.079         \\
            &      [0.79]&      [0.74]&      (0.14)         &      [0.84]&      [0.83]&      (0.15)         &      [0.78]&      [0.81]&      (0.14)         &      [0.78]&      [0.86]&      (0.15)         \\
 
\bottomrule \end{tabular} } \begin{tablenotes} \item \scriptsize \emph{Notes:} . \end{tablenotes} \end{threeparttable} \end{table} 

 \begin{table}[H] \begin{threeparttable} \centering \caption{\texttt{DIFFERENCE-IN-MEANS TESTS}} {\def\sym#1{\ifmmode^{#1}\else\(^{#1}\)\fi} \begin{tabular}{l*{13}{c}} \toprule & \multicolumn{12}{c}{Dependent variable: \textbf{Diseases of the eye and ear}} \\ \cmidrule(lr){2-13}
            &\multicolumn{3}{c}{Treatment (Nov78-Oct79)}&\multicolumn{3}{c}{Control 1 (Nov76-Oct77)}&\multicolumn{3}{c}{Control 2 (Nov77-Oct78)}&\multicolumn{3}{c}{Control 3 (Nov79-Oct80)}\\\cmidrule(lr){2-4}\cmidrule(lr){5-7}\cmidrule(lr){8-10}\cmidrule(lr){11-13}
            &\multicolumn{1}{c}{(1)}&\multicolumn{1}{c}{(2)}&\multicolumn{1}{c}{(3)}&\multicolumn{1}{c}{(4)}&\multicolumn{1}{c}{(5)}&\multicolumn{1}{c}{(6)}&\multicolumn{1}{c}{(7)}&\multicolumn{1}{c}{(8)}&\multicolumn{1}{c}{(9)}&\multicolumn{1}{c}{(10)}&\multicolumn{1}{c}{(11)}&\multicolumn{1}{c}{(12)}\\
            &\multicolumn{1}{c}{$\mathbb{E}_{Pre}[Y]$}&\multicolumn{1}{c}{$\mathbb{E}_{Post}[Y]$}&\multicolumn{1}{c}{$\Delta$}&\multicolumn{1}{c}{$\mathbb{E}_{Pre}[Y]$}&\multicolumn{1}{c}{$\mathbb{E}_{Post}[Y]$}&\multicolumn{1}{c}{$\Delta$}&\multicolumn{1}{c}{$\mathbb{E}_{Pre}[Y]$}&\multicolumn{1}{c}{$\mathbb{E}_{Post}[Y]$}&\multicolumn{1}{c}{$\Delta$}&\multicolumn{1}{c}{$\mathbb{E}_{Pre}[Y]$}&\multicolumn{1}{c}{$\mathbb{E}_{Post}[Y]$}&\multicolumn{1}{c}{$\Delta$}\\
\midrule
 \multicolumn{13}{l}{\emph{Panel A. 2 Month bandwidth}} \\ Abs. numbers&       120.2&       116.2&       -3.95         &       125.3&       119.4&       -5.95         &       119.0&       121.7&        2.63         &       115.6&       116.6&        1.02         \\
            &      [18.6]&      [16.8]&      (3.96)         &      [19.8]&      [15.8]&      (4.00)         &      [18.7]&      [15.9]&      (3.88)         &      [14.0]&      [21.2]&      (4.01)         \\
 Ratio fertility&        1.73&        1.68&      -0.050         &        1.81&        1.69&       -0.12\sym{**} &        1.69&        1.73&       0.041         &        1.59&        1.58&     -0.0086         \\
            &      [0.27]&      [0.24]&     (0.057)         &      [0.28]&      [0.22]&     (0.056)         &      [0.26]&      [0.23]&     (0.055)         &      [0.19]&      [0.29]&     (0.055)         \\
 Ratio population&        1.78&        1.74&      -0.045         &        1.90&        1.76&       -0.14\sym{*}  &        1.70&        1.80&       0.093         &        1.67&        1.62&      -0.051         \\
            &      [0.29]&      [0.26]&     (0.079)         &      [0.31]&      [0.26]&     (0.083)         &      [0.27]&      [0.24]&     (0.073)         &      [0.20]&      [0.28]&     (0.071)         \\
 \midrule\multicolumn{13}{l}{\emph{Panel B. 4 Month bandwidth}} \\ Abs. numbers&       114.9&       115.1&        0.19         &       121.4&       116.9&       -4.49         &       115.5&       120.9&        5.42\sym{**} &       114.2&       118.7&        4.52         \\
            &      [18.0]&      [16.5]&      (2.73)         &      [19.4]&      [16.4]&      (2.84)         &      [17.7]&      [14.5]&      (2.56)         &      [15.0]&      [20.6]&      (2.85)         \\
 Ratio fertility&        1.70&        1.64&      -0.060         &        1.82&        1.68&       -0.15\sym{***}&        1.69&        1.74&       0.042         &        1.60&        1.60&     -0.0073         \\
            &      [0.25]&      [0.24]&     (0.039)         &      [0.29]&      [0.23]&     (0.041)         &      [0.25]&      [0.21]&     (0.037)         &      [0.21]&      [0.28]&     (0.039)         \\
 Ratio population&        1.75&        1.70&      -0.051         &        1.89&        1.76&       -0.13\sym{**} &        1.75&        1.82&       0.072         &        1.67&        1.64&      -0.024         \\
            &      [0.26]&      [0.26]&     (0.053)         &      [0.32]&      [0.28]&     (0.061)         &      [0.28]&      [0.24]&     (0.052)         &      [0.22]&      [0.30]&     (0.053)         \\
 \midrule\multicolumn{13}{l}{\emph{Panel C. 6 Month bandwidth}} \\ Abs. numbers&       111.5&       114.2&        2.63         &       118.4&       117.5&       -0.90         &       113.7&       118.5&        4.77\sym{**} &       111.1&       118.4&        7.29\sym{***}\\
            &      [17.5]&      [15.5]&      (2.13)         &      [18.8]&      [16.0]&      (2.25)         &      [16.8]&      [15.9]&      (2.11)         &      [15.6]&      [19.7]&      (2.30)         \\
 Ratio fertility&        1.70&        1.64&      -0.057\sym{*}  &        1.83&        1.71&       -0.12\sym{***}&        1.71&        1.73&       0.016         &        1.61&        1.61&     -0.0026         \\
            &      [0.24]&      [0.22]&     (0.030)         &      [0.28]&      [0.24]&     (0.033)         &      [0.24]&      [0.23]&     (0.030)         &      [0.22]&      [0.27]&     (0.032)         \\
 Ratio population&        1.75&        1.69&      -0.064         &        1.90&        1.79&       -0.11\sym{**} &        1.78&        1.82&       0.041         &        1.65&        1.65&     -0.0061         \\
            &      [0.25]&      [0.23]&     (0.041)         &      [0.31]&      [0.28]&     (0.049)         &      [0.26]&      [0.25]&     (0.043)         &      [0.22]&      [0.29]&     (0.043)         \\
 \midrule\multicolumn{13}{l}{\emph{Panel D. Donut specification}} \\ Abs. numbers&       110.3&       113.2&        2.87         &       118.2&       117.0&       -1.14         &       113.4&       117.5&        4.18\sym{*}  &       111.1&       118.9&        7.78\sym{***}\\
            &      [17.0]&      [15.2]&      (2.28)         &      [18.2]&      [15.9]&      (2.42)         &      [16.4]&      [16.0]&      (2.29)         &      [16.3]&      [19.7]&      (2.56)         \\
 Ratio fertility&        1.69&        1.63&      -0.058\sym{*}  &        1.83&        1.72&       -0.12\sym{***}&        1.72&        1.74&       0.014         &        1.62&        1.62&     -0.0025         \\
            &      [0.24]&      [0.22]&     (0.032)         &      [0.27]&      [0.24]&     (0.036)         &      [0.24]&      [0.23]&     (0.033)         &      [0.23]&      [0.27]&     (0.035)         \\
 Ratio population&        1.75&        1.67&      -0.080\sym{*}  &        1.91&        1.80&       -0.11\sym{**} &        1.81&        1.83&       0.022         &        1.66&        1.66&     -0.0014         \\
            &      [0.25]&      [0.23]&     (0.043)         &      [0.30]&      [0.28]&     (0.053)         &      [0.26]&      [0.25]&     (0.047)         &      [0.23]&      [0.29]&     (0.047)         \\
 
\bottomrule \end{tabular} } \begin{tablenotes} \item \scriptsize \emph{Notes:} . \end{tablenotes} \end{threeparttable} \end{table} 

 \begin{table}[H] \begin{threeparttable} \centering \caption{\texttt{DIFFERENCE-IN-MEANS TESTS}} {\def\sym#1{\ifmmode^{#1}\else\(^{#1}\)\fi} \begin{tabular}{l*{13}{c}} \toprule & \multicolumn{12}{c}{Dependent variable: \textbf{Diseases of the circulatory system}} \\ \cmidrule(lr){2-13}
            &\multicolumn{3}{c}{Treatment (Nov78-Oct79)}&\multicolumn{3}{c}{Control 1 (Nov76-Oct77)}&\multicolumn{3}{c}{Control 2 (Nov77-Oct78)}&\multicolumn{3}{c}{Control 3 (Nov79-Oct80)}\\\cmidrule(lr){2-4}\cmidrule(lr){5-7}\cmidrule(lr){8-10}\cmidrule(lr){11-13}
            &\multicolumn{1}{c}{(1)}&\multicolumn{1}{c}{(2)}&\multicolumn{1}{c}{(3)}&\multicolumn{1}{c}{(4)}&\multicolumn{1}{c}{(5)}&\multicolumn{1}{c}{(6)}&\multicolumn{1}{c}{(7)}&\multicolumn{1}{c}{(8)}&\multicolumn{1}{c}{(9)}&\multicolumn{1}{c}{(10)}&\multicolumn{1}{c}{(11)}&\multicolumn{1}{c}{(12)}\\
            &\multicolumn{1}{c}{$\mathbb{E}_{Pre}[Y]$}&\multicolumn{1}{c}{$\mathbb{E}_{Post}[Y]$}&\multicolumn{1}{c}{$\Delta$}&\multicolumn{1}{c}{$\mathbb{E}_{Pre}[Y]$}&\multicolumn{1}{c}{$\mathbb{E}_{Post}[Y]$}&\multicolumn{1}{c}{$\Delta$}&\multicolumn{1}{c}{$\mathbb{E}_{Pre}[Y]$}&\multicolumn{1}{c}{$\mathbb{E}_{Post}[Y]$}&\multicolumn{1}{c}{$\Delta$}&\multicolumn{1}{c}{$\mathbb{E}_{Pre}[Y]$}&\multicolumn{1}{c}{$\mathbb{E}_{Post}[Y]$}&\multicolumn{1}{c}{$\Delta$}\\
\midrule
 \multicolumn{13}{l}{\emph{Panel A. 2 Month bandwidth}} \\ Abs. numbers&       195.3&       202.6&        7.27         &       230.6&       229.8&       -0.78         &       224.3&       217.0&       -7.33         &       191.8&       191.4&       -0.40         \\
            &      [69.7]&      [64.6]&      (15.0)         &      [75.6]&      [74.6]&      (16.8)         &      [73.6]&      [67.8]&      (15.8)         &      [64.0]&      [62.1]&      (14.1)         \\
 Ratio fertility&        2.81&        2.93&        0.12         &        3.32&        3.25&      -0.077         &        3.18&        3.08&      -0.098         &        2.64&        2.60&      -0.044         \\
            &      [1.00]&      [0.93]&      (0.22)         &      [1.09]&      [1.05]&      (0.24)         &      [1.03]&      [0.96]&      (0.22)         &      [0.88]&      [0.84]&      (0.19)         \\
 Ratio population&        3.59&        3.69&        0.10         &        4.15&        4.05&       -0.10         &        3.96&        3.85&       -0.11         &        3.34&        3.29&      -0.049         \\
            &      [0.93]&      [0.82]&      (0.25)         &      [1.09]&      [1.05]&      (0.31)         &      [1.02]&      [0.90]&      (0.28)         &      [0.78]&      [0.74]&      (0.22)         \\
 \midrule\multicolumn{13}{l}{\emph{Panel B. 4 Month bandwidth}} \\ Abs. numbers&       193.9&       196.2&        2.31         &       223.7&       226.5&        2.84         &       215.8&       212.6&       -3.20         &       188.5&       192.3&        3.79         \\
            &      [65.8]&      [61.4]&      (10.1)         &      [74.9]&      [74.0]&      (11.8)         &      [72.3]&      [66.6]&      (11.0)         &      [63.9]&      [63.2]&      (10.0)         \\
 Ratio fertility&        2.88&        2.80&      -0.078         &        3.36&        3.25&       -0.11         &        3.17&        3.05&       -0.11         &        2.64&        2.58&      -0.061         \\
            &      [0.98]&      [0.88]&      (0.15)         &      [1.12]&      [1.06]&      (0.17)         &      [1.05]&      [0.95]&      (0.16)         &      [0.89]&      [0.85]&      (0.14)         \\
 Ratio population&        3.63&        3.51&       -0.12         &        4.24&        4.07&       -0.17         &        3.98&        3.82&       -0.16         &        3.36&        3.28&      -0.086         \\
            &      [0.93]&      [0.80]&      (0.18)         &      [1.11]&      [1.04]&      (0.22)         &      [1.02]&      [0.90]&      (0.20)         &      [0.80]&      [0.75]&      (0.16)         \\
 \midrule\multicolumn{13}{l}{\emph{Panel C. 6 Month bandwidth}} \\ Abs. numbers&       190.2&       193.9&        3.68         &       221.1&       224.6&        3.49         &       211.3&       208.8&       -2.51         &       184.1&       189.2&        5.05         \\
            &      [63.9]&      [61.0]&      (8.06)         &      [72.0]&      [71.5]&      (9.26)         &      [68.9]&      [64.1]&      (8.59)         &      [61.8]&      [62.2]&      (8.01)         \\
 Ratio fertility&        2.90&        2.78&       -0.12         &        3.41&        3.26&       -0.15         &        3.18&        3.05&       -0.13         &        2.66&        2.56&       -0.10         \\
            &      [0.97]&      [0.88]&      (0.12)         &      [1.11]&      [1.04]&      (0.14)         &      [1.02]&      [0.93]&      (0.13)         &      [0.89]&      [0.84]&      (0.11)         \\
 Ratio population&        3.64&        3.46&       -0.18         &        4.26&        4.06&       -0.21         &        3.97&        3.79&       -0.18         &        3.38&        3.25&       -0.12         \\
            &      [0.93]&      [0.82]&      (0.15)         &      [1.08]&      [1.03]&      (0.18)         &      [0.99]&      [0.88]&      (0.16)         &      [0.80]&      [0.73]&      (0.13)         \\
 \midrule\multicolumn{13}{l}{\emph{Panel D. Donut specification}} \\ Abs. numbers&       190.9&       190.8&       -0.15         &       220.5&       223.2&        2.70         &       210.5&       206.1&       -4.42         &       184.4&       188.1&        3.69         \\
            &      [63.7]&      [59.7]&      (8.73)         &      [71.6]&      [70.4]&      (10.0)         &      [69.1]&      [63.2]&      (9.37)         &      [62.7]&      [62.0]&      (8.82)         \\
 Ratio fertility&        2.93&        2.75&       -0.18         &        3.43&        3.28&       -0.15         &        3.19&        3.04&       -0.15         &        2.69&        2.56&       -0.13         \\
            &      [0.97]&      [0.87]&      (0.13)         &      [1.11]&      [1.03]&      (0.15)         &      [1.03]&      [0.93]&      (0.14)         &      [0.90]&      [0.84]&      (0.12)         \\
 Ratio population&        3.69&        3.42&       -0.27\sym{*}  &        4.28&        4.06&       -0.22         &        3.99&        3.79&       -0.20         &        3.41&        3.25&       -0.16         \\
            &      [0.92]&      [0.81]&      (0.16)         &      [1.08]&      [1.04]&      (0.19)         &      [1.00]&      [0.89]&      (0.17)         &      [0.82]&      [0.72]&      (0.14)         \\
 
\bottomrule \end{tabular} } \begin{tablenotes} \item \scriptsize \emph{Notes:} . \end{tablenotes} \end{threeparttable} \end{table} 

 \begin{table}[H] \begin{threeparttable} \centering \caption{\texttt{DIFFERENCE-IN-MEANS TESTS}} {\def\sym#1{\ifmmode^{#1}\else\(^{#1}\)\fi} \begin{tabular}{l*{13}{c}} \toprule & \multicolumn{12}{c}{Dependent variable: \textbf{Diseases of the respiratory system}} \\ \cmidrule(lr){2-13}
            &\multicolumn{3}{c}{Treatment (Nov78-Oct79)}&\multicolumn{3}{c}{Control 1 (Nov76-Oct77)}&\multicolumn{3}{c}{Control 2 (Nov77-Oct78)}&\multicolumn{3}{c}{Control 3 (Nov79-Oct80)}\\\cmidrule(lr){2-4}\cmidrule(lr){5-7}\cmidrule(lr){8-10}\cmidrule(lr){11-13}
            &\multicolumn{1}{c}{(1)}&\multicolumn{1}{c}{(2)}&\multicolumn{1}{c}{(3)}&\multicolumn{1}{c}{(4)}&\multicolumn{1}{c}{(5)}&\multicolumn{1}{c}{(6)}&\multicolumn{1}{c}{(7)}&\multicolumn{1}{c}{(8)}&\multicolumn{1}{c}{(9)}&\multicolumn{1}{c}{(10)}&\multicolumn{1}{c}{(11)}&\multicolumn{1}{c}{(12)}\\
            &\multicolumn{1}{c}{$\mathbb{E}_{Pre}[Y]$}&\multicolumn{1}{c}{$\mathbb{E}_{Post}[Y]$}&\multicolumn{1}{c}{$\Delta$}&\multicolumn{1}{c}{$\mathbb{E}_{Pre}[Y]$}&\multicolumn{1}{c}{$\mathbb{E}_{Post}[Y]$}&\multicolumn{1}{c}{$\Delta$}&\multicolumn{1}{c}{$\mathbb{E}_{Pre}[Y]$}&\multicolumn{1}{c}{$\mathbb{E}_{Post}[Y]$}&\multicolumn{1}{c}{$\Delta$}&\multicolumn{1}{c}{$\mathbb{E}_{Pre}[Y]$}&\multicolumn{1}{c}{$\mathbb{E}_{Post}[Y]$}&\multicolumn{1}{c}{$\Delta$}\\
\midrule
 \multicolumn{13}{l}{\emph{Panel A. 2 Month bandwidth}} \\ Abs. numbers&       539.4&       527.2&       -12.1         &       517.2&       528.0&        10.8         &       530.9&       533.8&        2.90         &       551.0&       562.2&        11.3         \\
            &      [93.5]&      [85.1]&      (20.0)         &      [92.9]&      [86.8]&      (20.1)         &      [93.4]&     [102.5]&      (21.9)         &      [96.7]&      [89.3]&      (20.8)         \\
 Ratio fertility&        7.76&        7.62&       -0.14         &        7.45&        7.46&      0.0047         &        7.52&        7.57&       0.055         &        7.59&        7.63&       0.043         \\
            &      [1.32]&      [1.23]&      (0.29)         &      [1.33]&      [1.24]&      (0.29)         &      [1.31]&      [1.44]&      (0.31)         &      [1.32]&      [1.21]&      (0.28)         \\
 Ratio population&        7.34&        7.24&      -0.099         &        7.07&        7.13&       0.053         &        7.16&        7.09&      -0.078         &        7.26&        7.35&       0.091         \\
            &      [0.44]&      [0.47]&      (0.13)         &      [0.38]&      [0.50]&      (0.13)         &      [0.46]&      [0.54]&      (0.15)         &      [0.72]&      [0.56]&      (0.19)         \\
 \midrule\multicolumn{13}{l}{\emph{Panel B. 4 Month bandwidth}} \\ Abs. numbers&       524.5&       536.9&        12.4         &       497.8&         524&        26.3\sym{*}  &       519.5&       531.8&        12.3         &       541.4&       566.0&        24.6\sym{*}  \\
            &      [91.4]&      [88.0]&      (14.2)         &      [88.5]&      [90.2]&      (14.1)         &      [93.4]&      [96.8]&      (15.0)         &      [91.9]&      [91.2]&      (14.5)         \\
 Ratio fertility&        7.79&        7.67&       -0.12         &        7.48&        7.52&       0.043         &        7.63&        7.64&      0.0067         &        7.60&        7.61&      0.0096         \\
            &      [1.36]&      [1.25]&      (0.21)         &      [1.31]&      [1.30]&      (0.21)         &      [1.37]&      [1.38]&      (0.22)         &      [1.30]&      [1.24]&      (0.20)         \\
 Ratio population&        7.39&        7.29&      -0.100         &        7.13&        7.14&      0.0025         &        7.26&        7.19&      -0.066         &        7.29&        7.32&       0.035         \\
            &      [0.52]&      [0.46]&      (0.10)         &      [0.51]&      [0.55]&      (0.11)         &      [0.55]&      [0.52]&      (0.11)         &      [0.67]&      [0.62]&      (0.13)         \\
 \midrule\multicolumn{13}{l}{\emph{Panel C. 6 Month bandwidth}} \\ Abs. numbers&       509.6&       537.8&        28.2\sym{**} &       486.2&       524.2&        38.0\sym{***}&       507.6&       526.0&        18.4         &       527.5&       563.6&        36.1\sym{***}\\
            &      [91.9]&      [89.7]&      (11.7)         &      [86.2]&      [90.6]&      (11.4)         &      [92.6]&      [95.3]&      (12.1)         &      [91.1]&      [90.7]&      (11.7)         \\
 Ratio fertility&        7.76&        7.72&      -0.036         &        7.50&        7.62&        0.12         &        7.65&        7.68&       0.032         &        7.63&        7.64&      0.0085         \\
            &      [1.36]&      [1.29]&      (0.17)         &      [1.29]&      [1.33]&      (0.17)         &      [1.37]&      [1.39]&      (0.18)         &      [1.29]&      [1.24]&      (0.16)         \\
 Ratio population&        7.32&        7.32&     -0.0074         &        7.12&        7.23&        0.12         &        7.25&        7.23&      -0.023         &        7.30&        7.36&       0.069         \\
            &      [0.52]&      [0.52]&     (0.087)         &      [0.48]&      [0.53]&     (0.084)         &      [0.52]&      [0.50]&     (0.085)         &      [0.61]&      [0.64]&      (0.10)         \\
 \midrule\multicolumn{13}{l}{\emph{Panel D. Donut specification}} \\ Abs. numbers&       507.7&       538.0&        30.3\sym{**} &       482.4&       523.6&        41.2\sym{***}&       505.4&       522.5&        17.1         &       526.5&       562.6&        36.1\sym{***}\\
            &      [92.9]&      [90.6]&      (13.0)         &      [84.4]&      [91.6]&      (12.5)         &      [92.9]&      [91.5]&      (13.0)         &      [91.8]&      [90.7]&      (12.9)         \\
 Ratio fertility&        7.79&        7.77&      -0.026         &        7.49&        7.68&        0.19         &        7.68&        7.72&       0.042         &        7.68&        7.66&      -0.022         \\
            &      [1.38]&      [1.31]&      (0.19)         &      [1.27]&      [1.35]&      (0.19)         &      [1.38]&      [1.35]&      (0.19)         &      [1.30]&      [1.25]&      (0.18)         \\
 Ratio population&        7.35&        7.35&      0.0049         &        7.12&        7.28&        0.16\sym{*}  &        7.27&        7.29&       0.014         &        7.33&        7.38&       0.048         \\
            &      [0.53]&      [0.53]&     (0.097)         &      [0.52]&      [0.52]&     (0.095)         &      [0.54]&      [0.47]&     (0.092)         &      [0.60]&      [0.65]&      (0.11)         \\
 
\bottomrule \end{tabular} } \begin{tablenotes} \item \scriptsize \emph{Notes:} . \end{tablenotes} \end{threeparttable} \end{table} 

 \begin{table}[H] \begin{threeparttable} \centering \caption{\texttt{DIFFERENCE-IN-MEANS TESTS}} {\def\sym#1{\ifmmode^{#1}\else\(^{#1}\)\fi} \begin{tabular}{l*{13}{c}} \toprule & \multicolumn{12}{c}{Dependent variable: \textbf{Diseases of the digestive system}} \\ \cmidrule(lr){2-13}
            &\multicolumn{3}{c}{Treatment (Nov78-Oct79)}&\multicolumn{3}{c}{Control 1 (Nov76-Oct77)}&\multicolumn{3}{c}{Control 2 (Nov77-Oct78)}&\multicolumn{3}{c}{Control 3 (Nov79-Oct80)}\\\cmidrule(lr){2-4}\cmidrule(lr){5-7}\cmidrule(lr){8-10}\cmidrule(lr){11-13}
            &\multicolumn{1}{c}{(1)}&\multicolumn{1}{c}{(2)}&\multicolumn{1}{c}{(3)}&\multicolumn{1}{c}{(4)}&\multicolumn{1}{c}{(5)}&\multicolumn{1}{c}{(6)}&\multicolumn{1}{c}{(7)}&\multicolumn{1}{c}{(8)}&\multicolumn{1}{c}{(9)}&\multicolumn{1}{c}{(10)}&\multicolumn{1}{c}{(11)}&\multicolumn{1}{c}{(12)}\\
            &\multicolumn{1}{c}{$\mathbb{E}_{Pre}[Y]$}&\multicolumn{1}{c}{$\mathbb{E}_{Post}[Y]$}&\multicolumn{1}{c}{$\Delta$}&\multicolumn{1}{c}{$\mathbb{E}_{Pre}[Y]$}&\multicolumn{1}{c}{$\mathbb{E}_{Post}[Y]$}&\multicolumn{1}{c}{$\Delta$}&\multicolumn{1}{c}{$\mathbb{E}_{Pre}[Y]$}&\multicolumn{1}{c}{$\mathbb{E}_{Post}[Y]$}&\multicolumn{1}{c}{$\Delta$}&\multicolumn{1}{c}{$\mathbb{E}_{Pre}[Y]$}&\multicolumn{1}{c}{$\mathbb{E}_{Post}[Y]$}&\multicolumn{1}{c}{$\Delta$}\\
\midrule
 \multicolumn{13}{l}{\emph{Panel A. 2 Month bandwidth}} \\ Abs. numbers&       822.6&       825.9&        3.27         &       812.1&       817.0&        4.88         &       834.9&       827.1&       -7.80         &       865.7&       864.5&       -1.15         \\
            &     [103.8]&      [97.8]&      (22.5)         &     [101.9]&      [81.2]&      (20.6)         &     [106.7]&      [91.0]&      (22.2)         &     [109.2]&     [112.3]&      (24.8)         \\
 Ratio fertility&        11.8&        11.9&        0.10         &        11.7&        11.5&       -0.16         &        11.8&        11.7&      -0.085         &        11.9&        11.7&       -0.19         \\
            &      [1.47]&      [1.39]&      (0.32)         &      [1.38]&      [1.15]&      (0.28)         &      [1.48]&      [1.31]&      (0.31)         &      [1.48]&      [1.49]&      (0.33)         \\
 Ratio population&        12.0&        12.1&        0.12         &        12.2&        12.1&       -0.14         &        12.1&        12.1&      0.0078         &        12.0&        11.8&       -0.25         \\
            &      [1.42]&      [1.27]&      (0.39)         &      [1.57]&      [1.37]&      (0.43)         &      [1.56]&      [1.42]&      (0.43)         &      [1.23]&      [1.21]&      (0.35)         \\
 \midrule\multicolumn{13}{l}{\emph{Panel B. 4 Month bandwidth}} \\ Abs. numbers&       797.2&       829.3&        32.2\sym{**} &       785.2&       816.9&        31.7\sym{**} &       801.9&       823.9&        22.0         &       839.7&       870.4&        30.7\sym{*}  \\
            &      [98.8]&      [97.5]&      (15.5)         &      [98.3]&      [87.8]&      (14.7)         &     [109.4]&      [92.3]&      (16.0)         &     [106.5]&     [110.2]&      (17.1)         \\
 Ratio fertility&        11.8&        11.9&      0.0083         &        11.8&        11.7&      -0.071         &        11.8&        11.8&       0.067         &        11.8&        11.7&      -0.086         \\
            &      [1.46]&      [1.39]&      (0.22)         &      [1.41]&      [1.28]&      (0.21)         &      [1.53]&      [1.33]&      (0.23)         &      [1.47]&      [1.46]&      (0.23)         \\
 Ratio population&        12.0&        12.0&      -0.050         &        12.4&        12.2&       -0.12         &        12.1&        12.2&        0.13         &        12.0&        11.8&       -0.23         \\
            &      [1.44]&      [1.25]&      (0.27)         &      [1.66]&      [1.49]&      (0.32)         &      [1.69]&      [1.47]&      (0.32)         &      [1.36]&      [1.18]&      (0.26)         \\
 \midrule\multicolumn{13}{l}{\emph{Panel C. 6 Month bandwidth}} \\ Abs. numbers&       776.2&       825.1&        48.9\sym{***}&       770.0&       811.4&        41.4\sym{***}&       783.0&       814.6&        31.6\sym{**} &         813&       866.7&        53.7\sym{***}\\
            &     [100.7]&      [97.9]&      (12.8)         &      [96.5]&      [88.7]&      (12.0)         &     [105.8]&      [96.1]&      (13.0)         &     [106.7]&     [108.5]&      (13.9)         \\
 Ratio fertility&        11.8&        11.8&       0.028         &        11.9&        11.8&      -0.079         &        11.8&        11.9&        0.11         &        11.8&        11.7&     -0.0082         \\
            &      [1.45]&      [1.41]&      (0.18)         &      [1.41]&      [1.32]&      (0.18)         &      [1.48]&      [1.41]&      (0.19)         &      [1.42]&      [1.47]&      (0.19)         \\
 Ratio population&        12.0&        12.0&     -0.0080         &        12.3&        12.3&      -0.066         &        12.1&        12.2&       0.079         &        11.9&        11.8&       -0.13         \\
            &      [1.37]&      [1.25]&      (0.22)         &      [1.60]&      [1.49]&      (0.26)         &      [1.63]&      [1.48]&      (0.26)         &      [1.30]&      [1.19]&      (0.21)         \\
 \midrule\multicolumn{13}{l}{\emph{Panel D. Donut specification}} \\ Abs. numbers&       770.7&       821.4&        50.6\sym{***}&       768.4&       809.1&        40.7\sym{***}&       776.6&       810.9&        34.2\sym{**} &       806.8&       862.9&        56.1\sym{***}\\
            &     [101.3]&      [97.5]&      (14.1)         &      [99.4]&      [90.3]&      (13.4)         &     [107.0]&      [96.5]&      (14.4)         &     [107.9]&     [106.3]&      (15.2)         \\
 Ratio fertility&        11.8&        11.9&       0.023         &        11.9&        11.9&      -0.057         &        11.8&        12.0&        0.19         &        11.8&        11.7&      -0.018         \\
            &      [1.47]&      [1.41]&      (0.20)         &      [1.44]&      [1.35]&      (0.20)         &      [1.50]&      [1.42]&      (0.21)         &      [1.44]&      [1.45]&      (0.20)         \\
 Ratio population&        12.0&        11.9&      -0.039         &        12.4&        12.3&      -0.064         &        12.1&        12.3&        0.16         &        11.9&        11.8&       -0.13         \\
            &      [1.41]&      [1.23]&      (0.24)         &      [1.65]&      [1.51]&      (0.29)         &      [1.65]&      [1.48]&      (0.29)         &      [1.33]&      [1.19]&      (0.23)         \\
 
\bottomrule \end{tabular} } \begin{tablenotes} \item \scriptsize \emph{Notes:} . \end{tablenotes} \end{threeparttable} \end{table} 

 \begin{table}[H] \begin{threeparttable} \centering \caption{\texttt{DIFFERENCE-IN-MEANS TESTS}} {\def\sym#1{\ifmmode^{#1}\else\(^{#1}\)\fi} \begin{tabular}{l*{13}{c}} \toprule & \multicolumn{12}{c}{Dependent variable: \textbf{Diseases of the skin and subcutaneous tissue}} \\ \cmidrule(lr){2-13}
            &\multicolumn{3}{c}{Treatment (Nov78-Oct79)}&\multicolumn{3}{c}{Control 1 (Nov76-Oct77)}&\multicolumn{3}{c}{Control 2 (Nov77-Oct78)}&\multicolumn{3}{c}{Control 3 (Nov79-Oct80)}\\\cmidrule(lr){2-4}\cmidrule(lr){5-7}\cmidrule(lr){8-10}\cmidrule(lr){11-13}
            &\multicolumn{1}{c}{(1)}&\multicolumn{1}{c}{(2)}&\multicolumn{1}{c}{(3)}&\multicolumn{1}{c}{(4)}&\multicolumn{1}{c}{(5)}&\multicolumn{1}{c}{(6)}&\multicolumn{1}{c}{(7)}&\multicolumn{1}{c}{(8)}&\multicolumn{1}{c}{(9)}&\multicolumn{1}{c}{(10)}&\multicolumn{1}{c}{(11)}&\multicolumn{1}{c}{(12)}\\
            &\multicolumn{1}{c}{$\mathbb{E}_{Pre}[Y]$}&\multicolumn{1}{c}{$\mathbb{E}_{Post}[Y]$}&\multicolumn{1}{c}{$\Delta$}&\multicolumn{1}{c}{$\mathbb{E}_{Pre}[Y]$}&\multicolumn{1}{c}{$\mathbb{E}_{Post}[Y]$}&\multicolumn{1}{c}{$\Delta$}&\multicolumn{1}{c}{$\mathbb{E}_{Pre}[Y]$}&\multicolumn{1}{c}{$\mathbb{E}_{Post}[Y]$}&\multicolumn{1}{c}{$\Delta$}&\multicolumn{1}{c}{$\mathbb{E}_{Pre}[Y]$}&\multicolumn{1}{c}{$\mathbb{E}_{Post}[Y]$}&\multicolumn{1}{c}{$\Delta$}\\
\midrule
 \multicolumn{13}{l}{\emph{Panel A. 2 Month bandwidth}} \\ Abs. numbers&       187.5&       195.8&        8.35         &       191.2&       186.9&       -4.28         &       193.7&       186.8&       -6.90         &       193.3&       198.7&        5.38         \\
            &      [28.7]&      [26.2]&      (6.14)         &      [25.2]&      [24.6]&      (5.57)         &      [21.1]&      [21.9]&      (4.81)         &      [31.8]&      [32.2]&      (7.16)         \\
 Ratio fertility&        2.70&        2.83&        0.13         &        2.76&        2.64&       -0.12         &        2.74&        2.65&      -0.091         &        2.66&        2.70&       0.034         \\
            &      [0.41]&      [0.39]&     (0.090)         &      [0.35]&      [0.35]&     (0.078)         &      [0.29]&      [0.32]&     (0.069)         &      [0.44]&      [0.44]&     (0.098)         \\
 Ratio population&        2.89&        3.06&        0.18\sym{*}  &        2.77&        2.64&       -0.13         &        2.83&        2.70&       -0.13\sym{**} &        2.93&        3.03&        0.10         \\
            &      [0.35]&      [0.29]&     (0.092)         &      [0.28]&      [0.30]&     (0.084)         &      [0.23]&      [0.22]&     (0.064)         &      [0.24]&      [0.28]&     (0.076)         \\
 \midrule\multicolumn{13}{l}{\emph{Panel B. 4 Month bandwidth}} \\ Abs. numbers&       182.3&       196.3&        13.9\sym{***}&       183.6&       186.6&        3.01         &       187.3&       188.3&        0.94         &       188.7&       196.7&        7.93         \\
            &      [26.5]&      [24.8]&      (4.06)         &      [23.7]&      [23.0]&      (3.69)         &      [23.1]&      [23.9]&      (3.71)         &      [28.5]&      [32.6]&      (4.84)         \\
 Ratio fertility&        2.71&        2.81&       0.096         &        2.76&        2.68&      -0.080         &        2.75&        2.70&      -0.045         &        2.65&        2.64&     -0.0044         \\
            &      [0.40]&      [0.36]&     (0.060)         &      [0.34]&      [0.33]&     (0.053)         &      [0.33]&      [0.35]&     (0.053)         &      [0.40]&      [0.45]&     (0.067)         \\
 Ratio population&        2.89&        3.01&        0.13\sym{*}  &        2.79&        2.69&       -0.10\sym{*}  &        2.82&        2.77&      -0.054         &        2.92&        2.96&       0.042         \\
            &      [0.37]&      [0.29]&     (0.069)         &      [0.28]&      [0.29]&     (0.059)         &      [0.28]&      [0.22]&     (0.052)         &      [0.29]&      [0.31]&     (0.061)         \\
 \midrule\multicolumn{13}{l}{\emph{Panel C. 6 Month bandwidth}} \\ Abs. numbers&       178.3&       193.3&        15.0\sym{***}&       181.7&       186.4&        4.71\sym{*}  &       182.8&       184.8&        1.93         &       185.9&       195.7&        9.75\sym{**} \\
            &      [25.3]&      [25.4]&      (3.28)         &      [22.3]&      [21.7]&      (2.84)         &      [23.9]&      [24.5]&      (3.12)         &      [27.5]&      [33.6]&      (3.96)         \\
 Ratio fertility&        2.72&        2.78&       0.059         &        2.80&        2.71&      -0.093\sym{**} &        2.75&        2.70&      -0.055         &        2.69&        2.65&      -0.038         \\
            &      [0.38]&      [0.37]&     (0.048)         &      [0.33]&      [0.32]&     (0.043)         &      [0.34]&      [0.36]&     (0.045)         &      [0.40]&      [0.46]&     (0.056)         \\
 Ratio population&        2.86&        2.97&        0.11\sym{**} &        2.82&        2.73&      -0.090\sym{*}  &        2.80&        2.76&      -0.036         &        2.95&        2.96&       0.016         \\
            &      [0.33]&      [0.30]&     (0.053)         &      [0.27]&      [0.28]&     (0.046)         &      [0.29]&      [0.26]&     (0.046)         &      [0.28]&      [0.30]&     (0.048)         \\
 \midrule\multicolumn{13}{l}{\emph{Panel D. Donut specification}} \\ Abs. numbers&       176.9&       192.9&        16.0\sym{***}&       180.7&       186.1&        5.41\sym{*}  &       181.8&       184.6&        2.81         &       184.8&       194.9&        10.1\sym{**} \\
            &      [24.0]&      [25.2]&      (3.48)         &      [23.1]&      [20.6]&      (3.09)         &      [24.2]&      [24.6]&      (3.45)         &      [27.1]&      [33.9]&      (4.34)         \\
 Ratio fertility&        2.72&        2.78&       0.067         &        2.81&        2.73&      -0.076         &        2.76&        2.73&      -0.033         &        2.70&        2.65&      -0.044         \\
            &      [0.36]&      [0.37]&     (0.052)         &      [0.35]&      [0.31]&     (0.047)         &      [0.34]&      [0.36]&     (0.050)         &      [0.39]&      [0.47]&     (0.061)         \\
 Ratio population&        2.85&        2.97&        0.12\sym{**} &        2.81&        2.76&      -0.044         &        2.80&        2.79&     -0.0095         &        2.94&        2.95&      0.0095         \\
            &      [0.32]&      [0.30]&     (0.057)         &      [0.28]&      [0.27]&     (0.050)         &      [0.29]&      [0.26]&     (0.050)         &      [0.28]&      [0.31]&     (0.054)         \\
 
\bottomrule \end{tabular} } \begin{tablenotes} \item \scriptsize \emph{Notes:} . \end{tablenotes} \end{threeparttable} \end{table} 

 \begin{table}[H] \begin{threeparttable} \centering \caption{\texttt{DIFFERENCE-IN-MEANS TESTS}} {\def\sym#1{\ifmmode^{#1}\else\(^{#1}\)\fi} \begin{tabular}{l*{13}{c}} \toprule & \multicolumn{12}{c}{Dependent variable: \textbf{Diseases of the musculoskeletal system and connective tissue}} \\ \cmidrule(lr){2-13}
            &\multicolumn{3}{c}{Treatment (Nov78-Oct79)}&\multicolumn{3}{c}{Control 1 (Nov76-Oct77)}&\multicolumn{3}{c}{Control 2 (Nov77-Oct78)}&\multicolumn{3}{c}{Control 3 (Nov79-Oct80)}\\\cmidrule(lr){2-4}\cmidrule(lr){5-7}\cmidrule(lr){8-10}\cmidrule(lr){11-13}
            &\multicolumn{1}{c}{(1)}&\multicolumn{1}{c}{(2)}&\multicolumn{1}{c}{(3)}&\multicolumn{1}{c}{(4)}&\multicolumn{1}{c}{(5)}&\multicolumn{1}{c}{(6)}&\multicolumn{1}{c}{(7)}&\multicolumn{1}{c}{(8)}&\multicolumn{1}{c}{(9)}&\multicolumn{1}{c}{(10)}&\multicolumn{1}{c}{(11)}&\multicolumn{1}{c}{(12)}\\
            &\multicolumn{1}{c}{$\mathbb{E}_{Pre}[Y]$}&\multicolumn{1}{c}{$\mathbb{E}_{Post}[Y]$}&\multicolumn{1}{c}{$\Delta$}&\multicolumn{1}{c}{$\mathbb{E}_{Pre}[Y]$}&\multicolumn{1}{c}{$\mathbb{E}_{Post}[Y]$}&\multicolumn{1}{c}{$\Delta$}&\multicolumn{1}{c}{$\mathbb{E}_{Pre}[Y]$}&\multicolumn{1}{c}{$\mathbb{E}_{Post}[Y]$}&\multicolumn{1}{c}{$\Delta$}&\multicolumn{1}{c}{$\mathbb{E}_{Pre}[Y]$}&\multicolumn{1}{c}{$\mathbb{E}_{Post}[Y]$}&\multicolumn{1}{c}{$\Delta$}\\
\midrule
 \multicolumn{13}{l}{\emph{Panel A. 2 Month bandwidth}} \\ Abs. numbers&       425.8&       434.4&        8.60         &       482.9&       478.9&          -4         &       461.4&       454.2&       -7.15         &       428.1&       421.9&       -6.20         \\
            &     [108.6]&     [109.2]&      (24.4)         &     [136.3]&     [129.3]&      (29.7)         &     [121.8]&     [118.3]&      (26.8)         &     [106.8]&     [106.0]&      (23.8)         \\
 Ratio fertility&        6.13&        6.28&        0.15         &        6.96&        6.77&       -0.19         &        6.53&        6.44&      -0.090         &        5.90&        5.73&       -0.17         \\
            &      [1.55]&      [1.57]&      (0.35)         &      [1.94]&      [1.83]&      (0.42)         &      [1.71]&      [1.67]&      (0.38)         &      [1.46]&      [1.43]&      (0.32)         \\
 Ratio population&        7.23&        7.47&        0.24         &        8.24&        8.04&       -0.20         &        7.74&        7.69&      -0.053         &        7.01&        6.93&      -0.082         \\
            &      [1.82]&      [1.78]&      (0.52)         &      [2.31]&      [2.14]&      (0.64)         &      [2.02]&      [1.89]&      (0.56)         &      [1.73]&      [1.58]&      (0.48)         \\
 \midrule\multicolumn{13}{l}{\emph{Panel B. 4 Month bandwidth}} \\ Abs. numbers&       418.5&       434.6&        16.2         &       464.8&       474.6&        9.74         &       448.0&       448.9&        0.85         &       416.9&       428.3&        11.4         \\
            &     [105.4]&     [104.9]&      (16.6)         &     [129.0]&     [128.1]&      (20.3)         &     [119.0]&     [114.6]&      (18.5)         &     [101.8]&     [106.8]&      (16.5)         \\
 Ratio fertility&        6.22&        6.21&     -0.0084         &        6.98&        6.81&       -0.17         &        6.58&        6.45&       -0.14         &        5.85&        5.75&      -0.095         \\
            &      [1.58]&      [1.50]&      (0.24)         &      [1.91]&      [1.84]&      (0.30)         &      [1.75]&      [1.64]&      (0.27)         &      [1.42]&      [1.43]&      (0.23)         \\
 Ratio population&        7.35&        7.32&      -0.036         &        8.32&        8.10&       -0.22         &        7.82&        7.63&       -0.19         &        6.96&        6.92&      -0.044         \\
            &      [1.84]&      [1.74]&      (0.37)         &      [2.21]&      [2.13]&      (0.44)         &      [2.04]&      [1.90]&      (0.40)         &      [1.67]&      [1.62]&      (0.34)         \\
 \midrule\multicolumn{13}{l}{\emph{Panel C. 6 Month bandwidth}} \\ Abs. numbers&       412.0&       431.1&        19.1         &       455.9&       466.8&        10.9         &       439.4&       441.7&        2.31         &       405.7&       425.1&        19.4         \\
            &     [103.1]&     [106.1]&      (13.5)         &     [127.6]&     [124.5]&      (16.3)         &     [115.6]&     [110.9]&      (14.6)         &      [98.9]&     [103.8]&      (13.1)         \\
 Ratio fertility&        6.28&        6.19&      -0.089         &        7.03&        6.78&       -0.25         &        6.62&        6.45&       -0.17         &        5.87&        5.76&       -0.11         \\
            &      [1.57]&      [1.52]&      (0.20)         &      [1.94]&      [1.80]&      (0.24)         &      [1.73]&      [1.61]&      (0.22)         &      [1.41]&      [1.40]&      (0.18)         \\
 Ratio population&        7.41&        7.30&       -0.11         &        8.37&        8.06&       -0.31         &        7.83&        7.62&       -0.21         &        6.94&        6.88&      -0.064         \\
            &      [1.83]&      [1.77]&      (0.30)         &      [2.22]&      [2.08]&      (0.36)         &      [2.03]&      [1.86]&      (0.32)         &      [1.66]&      [1.62]&      (0.27)         \\
 \midrule\multicolumn{13}{l}{\emph{Panel D. Donut specification}} \\ Abs. numbers&       411.3&       429.0&        17.6         &       454.7&       463.9&        9.19         &       438.2&       436.4&       -1.83         &       404.0&       423.9&        19.9         \\
            &     [103.9]&     [104.1]&      (14.7)         &     [128.3]&     [122.9]&      (17.8)         &     [115.9]&     [108.9]&      (15.9)         &      [98.8]&     [103.8]&      (14.3)         \\
 Ratio fertility&        6.32&        6.19&       -0.13         &        7.06&        6.81&       -0.25         &        6.66&        6.44&       -0.21         &        5.89&        5.77&       -0.12         \\
            &      [1.59]&      [1.51]&      (0.22)         &      [1.96]&      [1.80]&      (0.27)         &      [1.74]&      [1.61]&      (0.24)         &      [1.41]&      [1.41]&      (0.20)         \\
 Ratio population&        7.46&        7.29&       -0.17         &        8.43&        8.08&       -0.35         &        7.88&        7.61&       -0.27         &        6.96&        6.88&      -0.080         \\
            &      [1.86]&      [1.75]&      (0.33)         &      [2.22]&      [2.08]&      (0.39)         &      [2.05]&      [1.87]&      (0.36)         &      [1.67]&      [1.64]&      (0.30)         \\
 
\bottomrule \end{tabular} } \begin{tablenotes} \item \scriptsize \emph{Notes:} . \end{tablenotes} \end{threeparttable} \end{table} 

 \begin{table}[H] \begin{threeparttable} \centering \caption{\texttt{DIFFERENCE-IN-MEANS TESTS}} {\def\sym#1{\ifmmode^{#1}\else\(^{#1}\)\fi} \begin{tabular}{l*{13}{c}} \toprule & \multicolumn{12}{c}{Dependent variable: \textbf{Diseases of the genitourinary system}} \\ \cmidrule(lr){2-13}
            &\multicolumn{3}{c}{Treatment (Nov78-Oct79)}&\multicolumn{3}{c}{Control 1 (Nov76-Oct77)}&\multicolumn{3}{c}{Control 2 (Nov77-Oct78)}&\multicolumn{3}{c}{Control 3 (Nov79-Oct80)}\\\cmidrule(lr){2-4}\cmidrule(lr){5-7}\cmidrule(lr){8-10}\cmidrule(lr){11-13}
            &\multicolumn{1}{c}{(1)}&\multicolumn{1}{c}{(2)}&\multicolumn{1}{c}{(3)}&\multicolumn{1}{c}{(4)}&\multicolumn{1}{c}{(5)}&\multicolumn{1}{c}{(6)}&\multicolumn{1}{c}{(7)}&\multicolumn{1}{c}{(8)}&\multicolumn{1}{c}{(9)}&\multicolumn{1}{c}{(10)}&\multicolumn{1}{c}{(11)}&\multicolumn{1}{c}{(12)}\\
            &\multicolumn{1}{c}{$\mathbb{E}_{Pre}[Y]$}&\multicolumn{1}{c}{$\mathbb{E}_{Post}[Y]$}&\multicolumn{1}{c}{$\Delta$}&\multicolumn{1}{c}{$\mathbb{E}_{Pre}[Y]$}&\multicolumn{1}{c}{$\mathbb{E}_{Post}[Y]$}&\multicolumn{1}{c}{$\Delta$}&\multicolumn{1}{c}{$\mathbb{E}_{Pre}[Y]$}&\multicolumn{1}{c}{$\mathbb{E}_{Post}[Y]$}&\multicolumn{1}{c}{$\Delta$}&\multicolumn{1}{c}{$\mathbb{E}_{Pre}[Y]$}&\multicolumn{1}{c}{$\mathbb{E}_{Post}[Y]$}&\multicolumn{1}{c}{$\Delta$}\\
\midrule
 \multicolumn{13}{l}{\emph{Panel A. 2 Month bandwidth}} \\ Abs. numbers&       508.9&       511.3&        2.42         &       555.4&       555.1&       -0.23         &       546.2&       538.7&       -7.45         &       504.5&       506.8&        2.30         \\
            &      [66.9]&      [68.6]&      (15.1)         &      [60.2]&      [54.6]&      (12.9)         &      [59.7]&      [62.7]&      (13.7)         &      [80.4]&      [91.3]&      (19.2)         \\
 Ratio fertility&        7.32&        7.39&       0.071         &        8.00&        7.84&       -0.16         &        7.73&        7.64&      -0.090         &        6.95&        6.88&      -0.071         \\
            &      [0.93]&      [0.99]&      (0.21)         &      [0.83]&      [0.78]&      (0.18)         &      [0.82]&      [0.89]&      (0.19)         &      [1.10]&      [1.23]&      (0.26)         \\
 Ratio population&        7.97&        8.10&        0.14         &        8.24&        8.20&      -0.035         &        8.25&        8.21&      -0.041         &        7.84&        7.84&    -0.00011         \\
            &      [0.87]&      [0.84]&      (0.25)         &      [0.90]&      [0.80]&      (0.25)         &      [0.85]&      [0.89]&      (0.25)         &      [0.61]&      [0.76]&      (0.20)         \\
 \midrule\multicolumn{13}{l}{\emph{Panel B. 4 Month bandwidth}} \\ Abs. numbers&       500.4&       519.1&        18.8\sym{*}  &       541.7&       547.3&        5.61         &       530.7&       524.0&       -6.73         &       497.5&       506.7&        9.20         \\
            &      [65.2]&      [71.0]&      (10.8)         &      [58.1]&      [53.0]&      (8.79)         &      [58.9]&      [60.2]&      (9.42)         &      [82.2]&      [91.3]&      (13.7)         \\
 Ratio fertility&        7.44&        7.42&      -0.018         &        8.14&        7.86&       -0.29\sym{**} &        7.79&        7.52&       -0.27\sym{**} &        6.98&        6.81&       -0.17         \\
            &      [0.98]&      [1.01]&      (0.16)         &      [0.86]&      [0.75]&      (0.13)         &      [0.84]&      [0.84]&      (0.13)         &      [1.16]&      [1.22]&      (0.19)         \\
 Ratio population&        8.12&        8.08&      -0.043         &        8.51&        8.19&       -0.32\sym{*}  &        8.31&        8.08&       -0.24         &        7.88&        7.78&       -0.10         \\
            &      [0.93]&      [0.77]&      (0.17)         &      [1.00]&      [0.85]&      (0.19)         &      [0.93]&      [0.82]&      (0.18)         &      [0.79]&      [0.76]&      (0.16)         \\
 \midrule\multicolumn{13}{l}{\emph{Panel C. 6 Month bandwidth}} \\ Abs. numbers&       491.9&       513.4&        21.4\sym{**} &       534.7&       541.6&        6.95         &       519.1&       517.7&       -1.44         &       485.1&       502.8&        17.7         \\
            &      [63.2]&      [72.6]&      (8.78)         &      [55.8]&      [55.4]&      (7.18)         &      [57.7]&      [58.7]&      (7.51)         &      [79.9]&      [88.8]&      (10.9)         \\
 Ratio fertility&        7.50&        7.37&       -0.13         &        8.25&        7.87&       -0.38\sym{***}&        7.82&        7.56&       -0.26\sym{**} &        7.02&        6.82&       -0.20         \\
            &      [0.96]&      [1.04]&      (0.13)         &      [0.85]&      [0.81]&      (0.11)         &      [0.82]&      [0.84]&      (0.11)         &      [1.12]&      [1.20]&      (0.15)         \\
 Ratio population&        8.14&        8.06&      -0.086         &        8.58&        8.20&       -0.39\sym{**} &        8.30&        8.11&       -0.19         &        7.85&        7.80&      -0.054         \\
            &      [0.86]&      [0.79]&      (0.14)         &      [0.99]&      [0.87]&      (0.16)         &      [0.88]&      [0.81]&      (0.14)         &      [0.78]&      [0.72]&      (0.13)         \\
 \midrule\multicolumn{13}{l}{\emph{Panel D. Donut specification}} \\ Abs. numbers&       493.0&       511.6&        18.6\sym{*}  &       533.6&       538.6&        4.97         &       516.9&       511.8&       -5.10         &       484.3&       500.2&        15.9         \\
            &      [62.5]&      [74.1]&      (9.69)         &      [57.1]&      [54.7]&      (7.91)         &      [57.7]&      [56.0]&      (8.04)         &      [80.0]&      [88.0]&      (11.9)         \\
 Ratio fertility&        7.57&        7.38&       -0.19         &        8.29&        7.90&       -0.39\sym{***}&        7.85&        7.56&       -0.29\sym{**} &        7.06&        6.81&       -0.26         \\
            &      [0.93]&      [1.07]&      (0.14)         &      [0.87]&      [0.81]&      (0.12)         &      [0.81]&      [0.84]&      (0.12)         &      [1.12]&      [1.20]&      (0.16)         \\
 Ratio population&        8.22&        8.07&       -0.15         &        8.64&        8.22&       -0.42\sym{**} &        8.31&        8.11&       -0.20         &        7.88&        7.78&       -0.10         \\
            &      [0.83]&      [0.77]&      (0.15)         &      [1.02]&      [0.87]&      (0.17)         &      [0.88]&      [0.80]&      (0.15)         &      [0.80]&      [0.70]&      (0.14)         \\
 
\bottomrule \end{tabular} } \begin{tablenotes} \item \scriptsize \emph{Notes:} . \end{tablenotes} \end{threeparttable} \end{table} 

 \begin{table}[H] \begin{threeparttable} \centering \caption{\texttt{DIFFERENCE-IN-MEANS TESTS}} {\def\sym#1{\ifmmode^{#1}\else\(^{#1}\)\fi} \begin{tabular}{l*{13}{c}} \toprule & \multicolumn{12}{c}{Dependent variable: \textbf{Symptoms, signs and abnormal clinical and laboratory findings, not elsewhere classified}} \\ \cmidrule(lr){2-13}
            &\multicolumn{3}{c}{Treatment (Nov78-Oct79)}&\multicolumn{3}{c}{Control 1 (Nov76-Oct77)}&\multicolumn{3}{c}{Control 2 (Nov77-Oct78)}&\multicolumn{3}{c}{Control 3 (Nov79-Oct80)}\\\cmidrule(lr){2-4}\cmidrule(lr){5-7}\cmidrule(lr){8-10}\cmidrule(lr){11-13}
            &\multicolumn{1}{c}{(1)}&\multicolumn{1}{c}{(2)}&\multicolumn{1}{c}{(3)}&\multicolumn{1}{c}{(4)}&\multicolumn{1}{c}{(5)}&\multicolumn{1}{c}{(6)}&\multicolumn{1}{c}{(7)}&\multicolumn{1}{c}{(8)}&\multicolumn{1}{c}{(9)}&\multicolumn{1}{c}{(10)}&\multicolumn{1}{c}{(11)}&\multicolumn{1}{c}{(12)}\\
            &\multicolumn{1}{c}{$\mathbb{E}_{Pre}[Y]$}&\multicolumn{1}{c}{$\mathbb{E}_{Post}[Y]$}&\multicolumn{1}{c}{$\Delta$}&\multicolumn{1}{c}{$\mathbb{E}_{Pre}[Y]$}&\multicolumn{1}{c}{$\mathbb{E}_{Post}[Y]$}&\multicolumn{1}{c}{$\Delta$}&\multicolumn{1}{c}{$\mathbb{E}_{Pre}[Y]$}&\multicolumn{1}{c}{$\mathbb{E}_{Post}[Y]$}&\multicolumn{1}{c}{$\Delta$}&\multicolumn{1}{c}{$\mathbb{E}_{Pre}[Y]$}&\multicolumn{1}{c}{$\mathbb{E}_{Post}[Y]$}&\multicolumn{1}{c}{$\Delta$}\\
\midrule
 \multicolumn{13}{l}{\emph{Panel A. 2 Month bandwidth}} \\ Abs. numbers&       326.6&       335.8&        9.25         &       328.6&       331.9&        3.30         &       335.1&       333.5&       -1.60         &       341.8&       346.6&        4.80         \\
            &      [68.6]&      [63.3]&      (14.8)         &      [68.7]&      [64.6]&      (14.9)         &      [60.5]&      [66.3]&      (14.2)         &      [63.7]&      [64.7]&      (14.4)         \\
 Ratio fertility&        4.70&        4.85&        0.16         &        4.74&        4.69&      -0.050         &        4.74&        4.73&      -0.012         &        4.71&        4.71&     -0.0045         \\
            &      [0.98]&      [0.91]&      (0.21)         &      [1.00]&      [0.91]&      (0.21)         &      [0.84]&      [0.94]&      (0.20)         &      [0.87]&      [0.87]&      (0.20)         \\
 Ratio population&        5.16&        5.38&        0.22         &        5.29&        5.25&      -0.038         &        5.20&        5.27&       0.065         &        5.22&        5.26&       0.043         \\
            &      [1.15]&      [1.11]&      (0.33)         &      [1.25]&      [1.11]&      (0.34)         &      [1.08]&      [1.17]&      (0.33)         &      [1.09]&      [1.04]&      (0.31)         \\
 \midrule\multicolumn{13}{l}{\emph{Panel B. 4 Month bandwidth}} \\ Abs. numbers&       322.0&       336.8&        14.8         &       317.9&       328.4&        10.4         &       323.9&       330.6&        6.71         &       339.6&       351.2&        11.7         \\
            &      [68.2]&      [62.7]&      (10.4)         &      [69.6]&      [66.1]&      (10.7)         &      [62.7]&      [66.1]&      (10.2)         &      [61.2]&      [66.3]&      (10.1)         \\
 Ratio fertility&        4.79&        4.81&       0.027         &        4.78&        4.71&      -0.067         &        4.75&        4.75&     -0.0048         &        4.77&        4.72&      -0.047         \\
            &      [1.03]&      [0.89]&      (0.15)         &      [1.05]&      [0.94]&      (0.16)         &      [0.91]&      [0.95]&      (0.15)         &      [0.87]&      [0.89]&      (0.14)         \\
 Ratio population&        5.30&        5.34&       0.048         &        5.36&        5.26&      -0.095         &        5.26&        5.28&       0.019         &        5.31&        5.24&      -0.061         \\
            &      [1.19]&      [1.09]&      (0.23)         &      [1.31]&      [1.17]&      (0.25)         &      [1.12]&      [1.18]&      (0.23)         &      [1.09]&      [1.10]&      (0.22)         \\
 \midrule\multicolumn{13}{l}{\emph{Panel C. 6 Month bandwidth}} \\ Abs. numbers&       314.7&       335.4&        20.7\sym{**} &       310.2&       328.3&        18.1\sym{**} &       317.4&       327.6&        10.2         &       329.6&       350.1&        20.5\sym{**} \\
            &      [65.4]&      [62.3]&      (8.25)         &      [66.6]&      [65.5]&      (8.53)         &      [60.0]&      [65.2]&      (8.09)         &      [62.6]&      [64.6]&      (8.21)         \\
 Ratio fertility&        4.79&        4.82&       0.021         &        4.78&        4.77&      -0.012         &        4.78&        4.78&      0.0032         &        4.77&        4.75&      -0.023         \\
            &      [0.99]&      [0.90]&      (0.12)         &      [1.01]&      [0.96]&      (0.13)         &      [0.88]&      [0.95]&      (0.12)         &      [0.89]&      [0.88]&      (0.11)         \\
 Ratio population&        5.31&        5.35&       0.035         &        5.32&        5.33&       0.014         &        5.29&        5.30&      0.0018         &        5.29&        5.27&      -0.015         \\
            &      [1.17]&      [1.09]&      (0.19)         &      [1.27]&      [1.18]&      (0.20)         &      [1.09]&      [1.19]&      (0.19)         &      [1.11]&      [1.08]&      (0.18)         \\
 \midrule\multicolumn{13}{l}{\emph{Panel D. Donut specification}} \\ Abs. numbers&       314.6&       333.9&        19.3\sym{**} &       307.3&       326.6&        19.3\sym{**} &       316.7&       325.1&        8.41         &       328.6&       349.4&        20.8\sym{**} \\
            &      [65.6]&      [61.0]&      (8.96)         &      [65.5]&      [64.2]&      (9.17)         &      [60.1]&      [64.4]&      (8.81)         &      [62.5]&      [64.9]&      (9.00)         \\
 Ratio fertility&        4.83&        4.82&      -0.012         &        4.77&        4.79&       0.020         &        4.81&        4.80&     -0.0047         &        4.79&        4.75&      -0.037         \\
            &      [1.00]&      [0.88]&      (0.13)         &      [1.01]&      [0.95]&      (0.14)         &      [0.88]&      [0.95]&      (0.13)         &      [0.89]&      [0.88]&      (0.13)         \\
 Ratio population&        5.36&        5.34&      -0.014         &        5.29&        5.35&       0.053         &        5.32&        5.30&      -0.022         &        5.31&        5.27&      -0.043         \\
            &      [1.16]&      [1.08]&      (0.20)         &      [1.27]&      [1.17]&      (0.22)         &      [1.08]&      [1.20]&      (0.21)         &      [1.11]&      [1.10]&      (0.20)         \\
 
\bottomrule \end{tabular} } \begin{tablenotes} \item \scriptsize \emph{Notes:} . \end{tablenotes} \end{threeparttable} \end{table} 

 \begin{table}[H] \begin{threeparttable} \centering \caption{\texttt{DIFFERENCE-IN-MEANS TESTS}} {\def\sym#1{\ifmmode^{#1}\else\(^{#1}\)\fi} \begin{tabular}{l*{13}{c}} \toprule & \multicolumn{12}{c}{Dependent variable: \textbf{External causes of morbidity and mortality}} \\ \cmidrule(lr){2-13}
            &\multicolumn{3}{c}{Treatment (Nov78-Oct79)}&\multicolumn{3}{c}{Control 1 (Nov76-Oct77)}&\multicolumn{3}{c}{Control 2 (Nov77-Oct78)}&\multicolumn{3}{c}{Control 3 (Nov79-Oct80)}\\\cmidrule(lr){2-4}\cmidrule(lr){5-7}\cmidrule(lr){8-10}\cmidrule(lr){11-13}
            &\multicolumn{1}{c}{(1)}&\multicolumn{1}{c}{(2)}&\multicolumn{1}{c}{(3)}&\multicolumn{1}{c}{(4)}&\multicolumn{1}{c}{(5)}&\multicolumn{1}{c}{(6)}&\multicolumn{1}{c}{(7)}&\multicolumn{1}{c}{(8)}&\multicolumn{1}{c}{(9)}&\multicolumn{1}{c}{(10)}&\multicolumn{1}{c}{(11)}&\multicolumn{1}{c}{(12)}\\
            &\multicolumn{1}{c}{$\mathbb{E}_{Pre}[Y]$}&\multicolumn{1}{c}{$\mathbb{E}_{Post}[Y]$}&\multicolumn{1}{c}{$\Delta$}&\multicolumn{1}{c}{$\mathbb{E}_{Pre}[Y]$}&\multicolumn{1}{c}{$\mathbb{E}_{Post}[Y]$}&\multicolumn{1}{c}{$\Delta$}&\multicolumn{1}{c}{$\mathbb{E}_{Pre}[Y]$}&\multicolumn{1}{c}{$\mathbb{E}_{Post}[Y]$}&\multicolumn{1}{c}{$\Delta$}&\multicolumn{1}{c}{$\mathbb{E}_{Pre}[Y]$}&\multicolumn{1}{c}{$\mathbb{E}_{Post}[Y]$}&\multicolumn{1}{c}{$\Delta$}\\
\midrule
 \multicolumn{13}{l}{\emph{Panel A. 2 Month bandwidth}} \\ Abs. numbers&      1026.8&      1039.0&        12.2         &         994&       991.9&       -2.10         &      1014.8&      1025.8&        11.0         &      1091.2&      1107.6&        16.5         \\
            &     [288.7]&     [290.3]&      (64.7)         &     [293.6]&     [279.5]&      (64.1)         &     [310.6]&     [293.4]&      (67.5)         &     [303.8]&     [305.9]&      (68.2)         \\
 Ratio fertility&        14.8&        15.0&        0.25         &        14.3&        14.0&       -0.31         &        14.4&        14.6&        0.18         &        15.0&        15.0&      0.0031         \\
            &      [4.13]&      [4.22]&      (0.93)         &      [4.19]&      [3.96]&      (0.91)         &      [4.38]&      [4.15]&      (0.95)         &      [4.17]&      [4.14]&      (0.93)         \\
 Ratio population&        12.7&        12.8&        0.14         &        12.2&        12.1&       -0.14         &        12.1&        12.4&        0.37         &        13.0&        13.0&      0.0074         \\
            &      [1.41]&      [1.33]&      (0.39)         &      [1.10]&      [0.99]&      (0.30)         &      [1.00]&      [1.17]&      (0.31)         &      [1.71]&      [1.54]&      (0.47)         \\
 \midrule\multicolumn{13}{l}{\emph{Panel B. 4 Month bandwidth}} \\ Abs. numbers&      1006.1&      1046.8&        40.7         &       951.5&       991.2&        39.7         &       983.8&      1023.1&        39.3         &      1065.9&      1121.5&        55.6         \\
            &     [282.8]&     [293.4]&      (45.6)         &     [275.5]&     [286.0]&      (44.4)         &     [295.5]&     [292.7]&      (46.5)         &     [294.3]&     [307.4]&      (47.6)         \\
 Ratio fertility&        15.0&        15.0&       0.013         &        14.3&        14.2&      -0.060         &        14.4&        14.7&        0.25         &        15.0&        15.1&        0.12         \\
            &      [4.21]&      [4.20]&      (0.67)         &      [4.07]&      [4.12]&      (0.65)         &      [4.31]&      [4.20]&      (0.67)         &      [4.12]&      [4.13]&      (0.65)         \\
 Ratio population&        12.8&        12.7&      -0.062         &        12.3&        12.1&       -0.12         &        12.2&        12.5&        0.28         &        13.0&        13.0&       0.073         \\
            &      [1.39]&      [1.27]&      (0.27)         &      [1.04]&      [0.97]&      (0.21)         &      [1.05]&      [1.14]&      (0.22)         &      [1.63]&      [1.58]&      (0.33)         \\
 \midrule\multicolumn{13}{l}{\emph{Panel C. 6 Month bandwidth}} \\ Abs. numbers&       982.1&      1051.9&        69.8\sym{*}  &       936.6&       992.1&        55.5         &       965.7&      1017.6&        51.8         &      1039.0&      1117.3&        78.3\sym{**} \\
            &     [278.3]&     [292.9]&      (36.9)         &     [272.7]&     [286.1]&      (36.1)         &     [285.3]&     [292.1]&      (37.3)         &     [288.3]&     [306.8]&      (38.4)         \\
 Ratio fertility&        15.0&        15.1&        0.15         &        14.4&        14.4&      -0.016         &        14.5&        14.9&        0.31         &        15.0&        15.1&        0.12         \\
            &      [4.21]&      [4.22]&      (0.54)         &      [4.16]&      [4.19]&      (0.54)         &      [4.26]&      [4.28]&      (0.55)         &      [4.12]&      [4.17]&      (0.53)         \\
 Ratio population&        12.8&        12.9&        0.10         &        12.3&        12.3&       0.012         &        12.4&        12.6&        0.27         &        13.0&        13.1&       0.098         \\
            &      [1.31]&      [1.33]&      (0.22)         &      [0.98]&      [1.10]&      (0.17)         &      [1.09]&      [1.19]&      (0.19)         &      [1.54]&      [1.63]&      (0.26)         \\
 \midrule\multicolumn{13}{l}{\emph{Panel D. Donut specification}} \\ Abs. numbers&       980.2&      1053.3&        73.1\sym{*}  &       932.6&         993&        60.4         &       961.5&      1011.7&        50.1         &      1035.5&      1115.1&        79.6\sym{*}  \\
            &     [279.9]&     [293.8]&      (40.6)         &     [271.7]&     [285.9]&      (39.4)         &     [282.3]&     [290.9]&      (40.5)         &     [288.7]&     [306.4]&      (42.1)         \\
 Ratio fertility&        15.1&        15.2&        0.15         &        14.5&        14.6&       0.095         &        14.6&        14.9&        0.34         &        15.1&        15.2&       0.077         \\
            &      [4.26]&      [4.25]&      (0.60)         &      [4.16]&      [4.22]&      (0.59)         &      [4.24]&      [4.31]&      (0.60)         &      [4.15]&      [4.18]&      (0.59)         \\
 Ratio population&        12.8&        12.9&        0.12         &        12.3&        12.4&        0.12         &        12.4&        12.7&        0.26         &        13.0&        13.1&       0.066         \\
            &      [1.29]&      [1.30]&      (0.24)         &      [0.96]&      [1.10]&      (0.19)         &      [1.09]&      [1.21]&      (0.21)         &      [1.52]&      [1.67]&      (0.29)         \\
 
\bottomrule \end{tabular} } \begin{tablenotes} \item \scriptsize \emph{Notes:} . \end{tablenotes} \end{threeparttable} \end{table} 

 \begin{table}[H] \begin{threeparttable} \centering \caption{\texttt{DIFFERENCE-IN-MEANS TESTS}} {\def\sym#1{\ifmmode^{#1}\else\(^{#1}\)\fi} \begin{tabular}{l*{13}{c}} \toprule & \multicolumn{12}{c}{Dependent variable: \textbf{Incidence of metabolic Syndrome}} \\ \cmidrule(lr){2-13}
            &\multicolumn{3}{c}{Treatment (Nov78-Oct79)}&\multicolumn{3}{c}{Control 1 (Nov76-Oct77)}&\multicolumn{3}{c}{Control 2 (Nov77-Oct78)}&\multicolumn{3}{c}{Control 3 (Nov79-Oct80)}\\\cmidrule(lr){2-4}\cmidrule(lr){5-7}\cmidrule(lr){8-10}\cmidrule(lr){11-13}
            &\multicolumn{1}{c}{(1)}&\multicolumn{1}{c}{(2)}&\multicolumn{1}{c}{(3)}&\multicolumn{1}{c}{(4)}&\multicolumn{1}{c}{(5)}&\multicolumn{1}{c}{(6)}&\multicolumn{1}{c}{(7)}&\multicolumn{1}{c}{(8)}&\multicolumn{1}{c}{(9)}&\multicolumn{1}{c}{(10)}&\multicolumn{1}{c}{(11)}&\multicolumn{1}{c}{(12)}\\
            &\multicolumn{1}{c}{$\mathbb{E}_{Pre}[Y]$}&\multicolumn{1}{c}{$\mathbb{E}_{Post}[Y]$}&\multicolumn{1}{c}{$\Delta$}&\multicolumn{1}{c}{$\mathbb{E}_{Pre}[Y]$}&\multicolumn{1}{c}{$\mathbb{E}_{Post}[Y]$}&\multicolumn{1}{c}{$\Delta$}&\multicolumn{1}{c}{$\mathbb{E}_{Pre}[Y]$}&\multicolumn{1}{c}{$\mathbb{E}_{Post}[Y]$}&\multicolumn{1}{c}{$\Delta$}&\multicolumn{1}{c}{$\mathbb{E}_{Pre}[Y]$}&\multicolumn{1}{c}{$\mathbb{E}_{Post}[Y]$}&\multicolumn{1}{c}{$\Delta$}\\
\midrule
 \multicolumn{13}{l}{\emph{Panel A. 2 Month bandwidth}} \\ Abs. numbers&        71.6&        74.2&        2.63         &        80.7&        82.4&        1.70         &        77.4&        79.0&        1.55         &          72&        71.5&       -0.53         \\
            &      [26.7]&      [28.7]&      (6.20)         &      [34.7]&      [34.8]&      (7.77)         &      [34.4]&      [29.3]&      (7.14)         &      [25.9]&      [25.4]&      (5.73)         \\
 Ratio fertility&        1.03&        1.07&       0.043         &        1.16&        1.16&      0.0015         &        1.10&        1.12&       0.025         &        0.99&        0.97&      -0.021         \\
            &      [0.38]&      [0.41]&     (0.089)         &      [0.49]&      [0.49]&      (0.11)         &      [0.49]&      [0.42]&      (0.10)         &      [0.36]&      [0.34]&     (0.078)         \\
 Ratio population&        1.24&        1.30&       0.062         &        1.44&        1.47&       0.029         &        1.36&        1.34&      -0.020         &        1.17&        1.16&      -0.018         \\
            &      [0.47]&      [0.50]&      (0.14)         &      [0.59]&      [0.56]&      (0.17)         &      [0.58]&      [0.52]&      (0.16)         &      [0.45]&      [0.43]&      (0.13)         \\
 \midrule\multicolumn{13}{l}{\emph{Panel B. 4 Month bandwidth}} \\ Abs. numbers&        69.5&        73.9&        4.42         &        79.1&        80.4&        1.30         &        77.4&        79.2&        1.85         &        70.8&        74.6&        3.81         \\
            &      [26.6]&      [26.6]&      (4.21)         &      [33.1]&      [34.1]&      (5.31)         &      [32.0]&      [30.8]&      (4.97)         &      [25.9]&      [25.9]&      (4.10)         \\
 Ratio fertility&        1.03&        1.06&       0.025         &        1.19&        1.15&      -0.035         &        1.14&        1.14&      0.0017         &        0.99&        1.00&      0.0090         \\
            &      [0.39]&      [0.38]&     (0.061)         &      [0.50]&      [0.49]&     (0.078)         &      [0.47]&      [0.45]&     (0.072)         &      [0.36]&      [0.34]&     (0.056)         \\
 Ratio population&        1.26&        1.27&       0.016         &        1.48&        1.45&      -0.036         &        1.41&        1.37&      -0.037         &        1.18&        1.18&     -0.0018         \\
            &      [0.47]&      [0.45]&     (0.094)         &      [0.58]&      [0.57]&      (0.12)         &      [0.55]&      [0.55]&      (0.11)         &      [0.44]&      [0.43]&     (0.089)         \\
 \midrule\multicolumn{13}{l}{\emph{Panel C. 6 Month bandwidth}} \\ Abs. numbers&        68.0&        72.3&        4.26         &        78.7&        81.0&        2.33         &        76.3&        76.8&        0.53         &        69.2&        74.3&        5.10         \\
            &      [25.4]&      [25.8]&      (3.31)         &      [32.0]&      [33.4]&      (4.22)         &      [29.7]&      [29.9]&      (3.85)         &      [25.0]&      [25.1]&      (3.24)         \\
 Ratio fertility&        1.03&        1.04&      0.0024         &        1.21&        1.18&      -0.037         &        1.15&        1.12&      -0.028         &        1.00&        1.01&      0.0057         \\
            &      [0.38]&      [0.37]&     (0.049)         &      [0.49]&      [0.49]&     (0.063)         &      [0.44]&      [0.44]&     (0.057)         &      [0.36]&      [0.34]&     (0.045)         \\
 Ratio population&        1.27&        1.24&      -0.027         &        1.51&        1.45&      -0.065         &        1.40&        1.36&      -0.040         &        1.17&        1.18&      0.0098         \\
            &      [0.44]&      [0.45]&     (0.074)         &      [0.57]&      [0.58]&     (0.095)         &      [0.53]&      [0.53]&     (0.088)         &      [0.43]&      [0.42]&     (0.071)         \\
 \midrule\multicolumn{13}{l}{\emph{Panel D. Donut specification}} \\ Abs. numbers&          68&        71.5&        3.54         &        79.2&        80.2&        0.97         &        76.6&        76.6&      -0.020         &        69.2&        74.8&        5.65         \\
            &      [24.9]&      [25.0]&      (3.53)         &      [32.4]&      [33.4]&      (4.65)         &      [29.0]&      [30.7]&      (4.22)         &      [25.7]&      [24.7]&      (3.57)         \\
 Ratio fertility&        1.04&        1.03&      -0.011         &        1.23&        1.18&      -0.054         &        1.16&        1.13&      -0.033         &        1.01&        1.02&      0.0090         \\
            &      [0.38]&      [0.36]&     (0.052)         &      [0.50]&      [0.49]&     (0.070)         &      [0.43]&      [0.45]&     (0.063)         &      [0.37]&      [0.33]&     (0.050)         \\
 Ratio population&        1.28&        1.23&      -0.049         &        1.54&        1.44&      -0.099         &        1.41&        1.37&      -0.039         &        1.18&        1.19&       0.015         \\
            &      [0.43]&      [0.44]&     (0.079)         &      [0.57]&      [0.59]&      (0.11)         &      [0.52]&      [0.55]&     (0.097)         &      [0.44]&      [0.41]&     (0.078)         \\
 
\bottomrule \end{tabular} } \begin{tablenotes} \item \scriptsize \emph{Notes:} . \end{tablenotes} \end{threeparttable} \end{table} 

 \begin{table}[H] \begin{threeparttable} \centering \caption{\texttt{DIFFERENCE-IN-MEANS TESTS}} {\def\sym#1{\ifmmode^{#1}\else\(^{#1}\)\fi} \begin{tabular}{l*{13}{c}} \toprule & \multicolumn{12}{c}{Dependent variable: \textbf{Index respiratory system}} \\ \cmidrule(lr){2-13}
            &\multicolumn{3}{c}{Treatment (Nov78-Oct79)}&\multicolumn{3}{c}{Control 1 (Nov76-Oct77)}&\multicolumn{3}{c}{Control 2 (Nov77-Oct78)}&\multicolumn{3}{c}{Control 3 (Nov79-Oct80)}\\\cmidrule(lr){2-4}\cmidrule(lr){5-7}\cmidrule(lr){8-10}\cmidrule(lr){11-13}
            &\multicolumn{1}{c}{(1)}&\multicolumn{1}{c}{(2)}&\multicolumn{1}{c}{(3)}&\multicolumn{1}{c}{(4)}&\multicolumn{1}{c}{(5)}&\multicolumn{1}{c}{(6)}&\multicolumn{1}{c}{(7)}&\multicolumn{1}{c}{(8)}&\multicolumn{1}{c}{(9)}&\multicolumn{1}{c}{(10)}&\multicolumn{1}{c}{(11)}&\multicolumn{1}{c}{(12)}\\
            &\multicolumn{1}{c}{$\mathbb{E}_{Pre}[Y]$}&\multicolumn{1}{c}{$\mathbb{E}_{Post}[Y]$}&\multicolumn{1}{c}{$\Delta$}&\multicolumn{1}{c}{$\mathbb{E}_{Pre}[Y]$}&\multicolumn{1}{c}{$\mathbb{E}_{Post}[Y]$}&\multicolumn{1}{c}{$\Delta$}&\multicolumn{1}{c}{$\mathbb{E}_{Pre}[Y]$}&\multicolumn{1}{c}{$\mathbb{E}_{Post}[Y]$}&\multicolumn{1}{c}{$\Delta$}&\multicolumn{1}{c}{$\mathbb{E}_{Pre}[Y]$}&\multicolumn{1}{c}{$\mathbb{E}_{Post}[Y]$}&\multicolumn{1}{c}{$\Delta$}\\
\midrule
 \multicolumn{13}{l}{\emph{Panel A. 2 Month bandwidth}} \\ Abs. numbers&        66.3&        64.2&       -2.08         &        66.8&        68.8&           2         &        67.2&        67.2&           0         &        67.8&        65.6&       -2.22         \\
            &      [30.2]&      [24.0]&      (6.11)         &      [29.2]&      [26.3]&      (6.22)         &      [30.2]&      [31.8]&      (6.94)         &      [29.3]&      [27.0]&      (6.30)         \\
 Ratio fertility&        0.95&        0.93&      -0.024         &        0.96&        0.97&      0.0086         &        0.95&        0.95&      0.0012         &        0.94&        0.89&      -0.043         \\
            &      [0.43]&      [0.35]&     (0.088)         &      [0.42]&      [0.37]&     (0.089)         &      [0.43]&      [0.45]&     (0.098)         &      [0.40]&      [0.37]&     (0.086)         \\
 Ratio population&        0.79&        0.82&       0.031         &        0.83&        0.84&      0.0060         &        0.79&        0.76&      -0.025         &        0.77&        0.75&      -0.024         \\
            &      [0.19]&      [0.16]&     (0.052)         &      [0.16]&      [0.17]&     (0.048)         &      [0.16]&      [0.17]&     (0.047)         &      [0.13]&      [0.16]&     (0.042)         \\
 \midrule\multicolumn{13}{l}{\emph{Panel B. 4 Month bandwidth}} \\ Abs. numbers&        67.9&        65.0&       -2.90         &        64.6&        68.4&        3.80         &        67.9&        69.2&        1.29         &        67.6&        67.6&      -0.037         \\
            &      [28.1]&      [24.6]&      (4.18)         &      [26.7]&      [27.7]&      (4.30)         &      [27.2]&      [30.8]&      (4.59)         &      [28.5]&      [28.2]&      (4.48)         \\
 Ratio fertility&        1.01&        0.93&      -0.081         &        0.97&        0.98&       0.011         &        1.00&        0.99&     -0.0047         &        0.95&        0.91&      -0.041         \\
            &      [0.42]&      [0.35]&     (0.062)         &      [0.40]&      [0.40]&     (0.063)         &      [0.40]&      [0.44]&     (0.067)         &      [0.40]&      [0.38]&     (0.062)         \\
 Ratio population&        0.89&        0.82&      -0.066         &        0.85&        0.82&      -0.035         &        0.85&        0.82&      -0.028         &        0.79&        0.75&      -0.036         \\
            &      [0.28]&      [0.18]&     (0.047)         &      [0.16]&      [0.16]&     (0.032)         &      [0.15]&      [0.19]&     (0.035)         &      [0.15]&      [0.15]&     (0.031)         \\
 \midrule\multicolumn{13}{l}{\emph{Panel C. 6 Month bandwidth}} \\ Abs. numbers&        66.1&        65.7&       -0.40         &        64.8&        67.8&        2.96         &        66.6&        67.7&        1.09         &        65.2&        68.5&        3.32         \\
            &      [27.7]&      [26.0]&      (3.47)         &      [25.5]&      [28.9]&      (3.52)         &      [27.1]&      [29.5]&      (3.66)         &      [26.9]&      [28.2]&      (3.56)         \\
 Ratio fertility&        1.01&        0.94&      -0.064         &        1.00&        0.98&      -0.016         &        1.00&        0.99&      -0.016         &        0.94&        0.93&      -0.014         \\
            &      [0.42]&      [0.38]&     (0.052)         &      [0.39]&      [0.42]&     (0.053)         &      [0.41]&      [0.43]&     (0.054)         &      [0.39]&      [0.39]&     (0.050)         \\
 Ratio population&        0.86&        0.82&      -0.049         &        0.88&        0.82&      -0.053\sym{**} &        0.85&        0.82&      -0.034         &        0.80&        0.77&      -0.026         \\
            &      [0.25]&      [0.19]&     (0.037)         &      [0.16]&      [0.16]&     (0.027)         &      [0.17]&      [0.18]&     (0.029)         &      [0.15]&      [0.15]&     (0.025)         \\
 \midrule\multicolumn{13}{l}{\emph{Panel D. Donut specification}} \\ Abs. numbers&        66.7&        66.3&       -0.43         &        64.7&        67.2&        2.48         &        66.6&        67.4&        0.88         &        64.8&        69.5&        4.66         \\
            &      [28.0]&      [26.9]&      (3.88)         &      [24.8]&      [29.0]&      (3.81)         &      [26.3]&      [28.6]&      (3.88)         &      [26.7]&      [28.7]&      (3.92)         \\
 Ratio fertility&        1.03&        0.96&      -0.068         &        1.01&        0.99&      -0.021         &        1.01&        1.00&      -0.016         &        0.95&        0.95&     0.00075         \\
            &      [0.43]&      [0.39]&     (0.058)         &      [0.38]&      [0.42]&     (0.057)         &      [0.40]&      [0.42]&     (0.058)         &      [0.39]&      [0.39]&     (0.055)         \\
 Ratio population&        0.89&        0.82&      -0.063         &        0.88&        0.82&      -0.058\sym{*}  &        0.87&        0.84&      -0.029         &        0.79&        0.78&      -0.015         \\
            &      [0.26]&      [0.19]&     (0.041)         &      [0.16]&      [0.17]&     (0.030)         &      [0.17]&      [0.18]&     (0.032)         &      [0.16]&      [0.14]&     (0.027)         \\
 
\bottomrule \end{tabular} } \begin{tablenotes} \item \scriptsize \emph{Notes:} . \end{tablenotes} \end{threeparttable} \end{table} 

 \begin{table}[H] \begin{threeparttable} \centering \caption{\texttt{DIFFERENCE-IN-MEANS TESTS}} {\def\sym#1{\ifmmode^{#1}\else\(^{#1}\)\fi} \begin{tabular}{l*{13}{c}} \toprule & \multicolumn{12}{c}{Dependent variable: \textbf{Psychoactive substance abuse}} \\ \cmidrule(lr){2-13}
            &\multicolumn{3}{c}{Treatment (Nov78-Oct79)}&\multicolumn{3}{c}{Control 1 (Nov76-Oct77)}&\multicolumn{3}{c}{Control 2 (Nov77-Oct78)}&\multicolumn{3}{c}{Control 3 (Nov79-Oct80)}\\\cmidrule(lr){2-4}\cmidrule(lr){5-7}\cmidrule(lr){8-10}\cmidrule(lr){11-13}
            &\multicolumn{1}{c}{(1)}&\multicolumn{1}{c}{(2)}&\multicolumn{1}{c}{(3)}&\multicolumn{1}{c}{(4)}&\multicolumn{1}{c}{(5)}&\multicolumn{1}{c}{(6)}&\multicolumn{1}{c}{(7)}&\multicolumn{1}{c}{(8)}&\multicolumn{1}{c}{(9)}&\multicolumn{1}{c}{(10)}&\multicolumn{1}{c}{(11)}&\multicolumn{1}{c}{(12)}\\
            &\multicolumn{1}{c}{$\mathbb{E}_{Pre}[Y]$}&\multicolumn{1}{c}{$\mathbb{E}_{Post}[Y]$}&\multicolumn{1}{c}{$\Delta$}&\multicolumn{1}{c}{$\mathbb{E}_{Pre}[Y]$}&\multicolumn{1}{c}{$\mathbb{E}_{Post}[Y]$}&\multicolumn{1}{c}{$\Delta$}&\multicolumn{1}{c}{$\mathbb{E}_{Pre}[Y]$}&\multicolumn{1}{c}{$\mathbb{E}_{Post}[Y]$}&\multicolumn{1}{c}{$\Delta$}&\multicolumn{1}{c}{$\mathbb{E}_{Pre}[Y]$}&\multicolumn{1}{c}{$\mathbb{E}_{Post}[Y]$}&\multicolumn{1}{c}{$\Delta$}\\
\midrule
 \multicolumn{13}{l}{\emph{Panel A. 2 Month bandwidth}} \\ Abs. numbers&       312.8&       316.2&        3.40         &       321.1&       329.2&        8.13         &       321.3&       333.8&        12.6         &       305.9&       314.8&        8.93         \\
            &     [104.4]&      [97.8]&      (22.6)         &      [90.7]&      [82.0]&      (19.3)         &      [90.9]&     [109.5]&      (22.5)         &     [100.0]&     [112.0]&      (23.7)         \\
 Ratio fertility&        4.50&        4.57&       0.073         &        4.63&        4.65&       0.020         &        4.55&        4.74&        0.19         &        4.22&        4.28&       0.060         \\
            &      [1.50]&      [1.42]&      (0.33)         &      [1.30]&      [1.15]&      (0.27)         &      [1.29]&      [1.55]&      (0.32)         &      [1.37]&      [1.52]&      (0.32)         \\
 Ratio population&        5.74&        5.68&      -0.062         &        5.66&        5.59&      -0.067         &        5.61&        6.02&        0.41         &        5.36&        5.55&        0.19         \\
            &      [1.27]&      [1.25]&      (0.36)         &      [1.22]&      [1.02]&      (0.32)         &      [1.18]&      [1.32]&      (0.36)         &      [1.12]&      [1.27]&      (0.35)         \\
 \midrule\multicolumn{13}{l}{\emph{Panel B. 4 Month bandwidth}} \\ Abs. numbers&       313.0&       319.5&        6.56         &       316.1&       327.9&        11.8         &       310.6&       336.9&        26.3\sym{*}  &       301.2&       311.6&        10.4         \\
            &     [102.3]&     [101.8]&      (16.1)         &      [87.7]&      [86.3]&      (13.8)         &      [84.4]&     [105.4]&      (15.1)         &      [98.5]&     [112.0]&      (16.7)         \\
 Ratio fertility&        4.66&        4.57&      -0.087         &        4.76&        4.71&      -0.050         &        4.56&        4.84&        0.28         &        4.23&        4.19&      -0.038         \\
            &      [1.54]&      [1.46]&      (0.24)         &      [1.33]&      [1.24]&      (0.20)         &      [1.22]&      [1.51]&      (0.22)         &      [1.39]&      [1.51]&      (0.23)         \\
 Ratio population&        5.89&        5.71&       -0.18         &        5.83&        5.70&       -0.13         &        5.58&        6.11&        0.53\sym{**} &        5.40&        5.43&       0.032         \\
            &      [1.37]&      [1.28]&      (0.27)         &      [1.24]&      [1.17]&      (0.25)         &      [1.08]&      [1.26]&      (0.24)         &      [1.15]&      [1.30]&      (0.25)         \\
 \midrule\multicolumn{13}{l}{\emph{Panel C. 6 Month bandwidth}} \\ Abs. numbers&       306.7&       313.2&        6.55         &       311.1&       327.5&        16.4         &       313.8&       331.1&        17.3         &       299.6&       309.5&        9.93         \\
            &      [98.7]&     [100.0]&      (12.8)         &      [85.7]&      [90.1]&      (11.4)         &      [86.2]&     [105.4]&      (12.4)         &      [99.9]&     [110.2]&      (13.6)         \\
 Ratio fertility&        4.67&        4.50&       -0.18         &        4.80&        4.76&      -0.042         &        4.73&        4.84&        0.10         &        4.34&        4.20&       -0.14         \\
            &      [1.50]&      [1.43]&      (0.19)         &      [1.32]&      [1.32]&      (0.17)         &      [1.32]&      [1.54]&      (0.18)         &      [1.46]&      [1.50]&      (0.19)         \\
 Ratio population&        5.88&        5.64&       -0.24         &        5.85&        5.80&      -0.052         &        5.77&        6.07&        0.30         &        5.54&        5.43&       -0.11         \\
            &      [1.34]&      [1.23]&      (0.21)         &      [1.25]&      [1.26]&      (0.21)         &      [1.21]&      [1.36]&      (0.21)         &      [1.26]&      [1.26]&      (0.21)         \\
 \midrule\multicolumn{13}{l}{\emph{Panel D. Donut specification}} \\ Abs. numbers&       306.8&       312.5&        5.71         &       310.3&       325.4&        15.1         &       312.9&       329.7&        16.9         &       299.8&       308.2&        8.38         \\
            &      [97.2]&     [100.6]&      (14.0)         &      [85.9]&      [90.7]&      (12.5)         &      [84.3]&     [103.3]&      (13.3)         &     [100.7]&     [110.5]&      (15.0)         \\
 Ratio fertility&        4.71&        4.51&       -0.20         &        4.82&        4.78&      -0.047         &        4.76&        4.87&        0.11         &        4.38&        4.19&       -0.18         \\
            &      [1.49]&      [1.45]&      (0.21)         &      [1.34]&      [1.34]&      (0.19)         &      [1.30]&      [1.52]&      (0.20)         &      [1.49]&      [1.51]&      (0.21)         \\
 Ratio population&        5.90&        5.67&       -0.24         &        5.88&        5.83&      -0.049         &        5.78&        6.09&        0.30         &        5.61&        5.44&       -0.17         \\
            &      [1.36]&      [1.24]&      (0.24)         &      [1.27]&      [1.29]&      (0.23)         &      [1.19]&      [1.37]&      (0.23)         &      [1.27]&      [1.27]&      (0.23)         \\
 
\bottomrule \end{tabular} } \begin{tablenotes} \item \scriptsize \emph{Notes:} . \end{tablenotes} \end{threeparttable} \end{table} 

 \begin{table}[H] \begin{threeparttable} \centering \caption{\texttt{DIFFERENCE-IN-MEANS TESTS}} {\def\sym#1{\ifmmode^{#1}\else\(^{#1}\)\fi} \begin{tabular}{l*{13}{c}} \toprule & \multicolumn{12}{c}{Dependent variable: \textbf{Non-ischemic heart diseases}} \\ \cmidrule(lr){2-13}
            &\multicolumn{3}{c}{Treatment (Nov78-Oct79)}&\multicolumn{3}{c}{Control 1 (Nov76-Oct77)}&\multicolumn{3}{c}{Control 2 (Nov77-Oct78)}&\multicolumn{3}{c}{Control 3 (Nov79-Oct80)}\\\cmidrule(lr){2-4}\cmidrule(lr){5-7}\cmidrule(lr){8-10}\cmidrule(lr){11-13}
            &\multicolumn{1}{c}{(1)}&\multicolumn{1}{c}{(2)}&\multicolumn{1}{c}{(3)}&\multicolumn{1}{c}{(4)}&\multicolumn{1}{c}{(5)}&\multicolumn{1}{c}{(6)}&\multicolumn{1}{c}{(7)}&\multicolumn{1}{c}{(8)}&\multicolumn{1}{c}{(9)}&\multicolumn{1}{c}{(10)}&\multicolumn{1}{c}{(11)}&\multicolumn{1}{c}{(12)}\\
            &\multicolumn{1}{c}{$\mathbb{E}_{Pre}[Y]$}&\multicolumn{1}{c}{$\mathbb{E}_{Post}[Y]$}&\multicolumn{1}{c}{$\Delta$}&\multicolumn{1}{c}{$\mathbb{E}_{Pre}[Y]$}&\multicolumn{1}{c}{$\mathbb{E}_{Post}[Y]$}&\multicolumn{1}{c}{$\Delta$}&\multicolumn{1}{c}{$\mathbb{E}_{Pre}[Y]$}&\multicolumn{1}{c}{$\mathbb{E}_{Post}[Y]$}&\multicolumn{1}{c}{$\Delta$}&\multicolumn{1}{c}{$\mathbb{E}_{Pre}[Y]$}&\multicolumn{1}{c}{$\mathbb{E}_{Post}[Y]$}&\multicolumn{1}{c}{$\Delta$}\\
\midrule
 \multicolumn{13}{l}{\emph{Panel A. 2 Month bandwidth}} \\ Abs. numbers&        54.1&        54.0&      -0.075         &        61.7&        61.5&       -0.20         &        57.8&        58.6&        0.77         &        52.8&        53.1&        0.30         \\
            &      [21.6]&      [18.2]&      (4.46)         &      [24.0]&      [22.6]&      (5.21)         &      [20.4]&      [20.6]&      (4.59)         &      [18.5]&      [18.3]&      (4.11)         \\
 Ratio fertility&        0.78&        0.78&      0.0018         &        0.89&        0.87&      -0.021         &        0.82&        0.83&       0.013         &        0.73&        0.72&     -0.0063         \\
            &      [0.31]&      [0.26]&     (0.064)         &      [0.35]&      [0.32]&     (0.075)         &      [0.29]&      [0.29]&     (0.065)         &      [0.25]&      [0.25]&     (0.056)         \\
 Ratio population&        1.01&        0.97&      -0.046         &        1.15&        1.09&      -0.062         &        1.03&        1.07&       0.044         &        0.92&        0.93&      0.0060         \\
            &      [0.29]&      [0.24]&     (0.077)         &      [0.34]&      [0.31]&     (0.094)         &      [0.29]&      [0.26]&     (0.079)         &      [0.22]&      [0.21]&     (0.061)         \\
 \midrule\multicolumn{13}{l}{\emph{Panel B. 4 Month bandwidth}} \\ Abs. numbers&        53.3&        51.6&       -1.70         &        59.2&        60.4&        1.17         &        57.5&        56.3&       -1.24         &        52.0&        53.5&        1.52         \\
            &      [20.2]&      [16.9]&      (2.94)         &      [22.9]&      [23.1]&      (3.64)         &      [20.9]&      [19.8]&      (3.23)         &      [19.5]&      [18.8]&      (3.03)         \\
 Ratio fertility&        0.79&        0.74&      -0.055         &        0.89&        0.87&      -0.024         &        0.84&        0.81&      -0.037         &        0.73&        0.72&     -0.0098         \\
            &      [0.30]&      [0.24]&     (0.043)         &      [0.35]&      [0.33]&     (0.053)         &      [0.31]&      [0.28]&     (0.047)         &      [0.27]&      [0.25]&     (0.042)         \\
 Ratio population&        1.01&        0.91&       -0.10\sym{*}  &        1.14&        1.10&      -0.041         &        1.08&        1.03&      -0.052         &        0.94&        0.93&     -0.0073         \\
            &      [0.29]&      [0.23]&     (0.053)         &      [0.34]&      [0.32]&     (0.068)         &      [0.29]&      [0.27]&     (0.057)         &      [0.26]&      [0.21]&     (0.048)         \\
 \midrule\multicolumn{13}{l}{\emph{Panel C. 6 Month bandwidth}} \\ Abs. numbers&        52.4&        51.1&       -1.24         &        57.6&        60.4&        2.74         &        56.0&        57.1&        1.16         &        50.8&        52.3&        1.50         \\
            &      [19.3]&      [16.8]&      (2.33)         &      [21.9]&      [21.3]&      (2.79)         &      [20.5]&      [19.4]&      (2.58)         &      [19.2]&      [18.8]&      (2.45)         \\
 Ratio fertility&        0.80&        0.73&      -0.064\sym{*}  &        0.89&        0.88&      -0.011         &        0.84&        0.83&     -0.0083         &        0.74&        0.71&      -0.026         \\
            &      [0.29]&      [0.24]&     (0.035)         &      [0.34]&      [0.31]&     (0.042)         &      [0.31]&      [0.28]&     (0.038)         &      [0.28]&      [0.25]&     (0.034)         \\
 Ratio population&        1.00&        0.90&       -0.10\sym{**} &        1.14&        1.10&      -0.034         &        1.07&        1.05&      -0.021         &        0.94&        0.92&      -0.028         \\
            &      [0.29]&      [0.24]&     (0.044)         &      [0.32]&      [0.30]&     (0.052)         &      [0.30]&      [0.27]&     (0.047)         &      [0.26]&      [0.21]&     (0.039)         \\
 \midrule\multicolumn{13}{l}{\emph{Panel D. Donut specification}} \\ Abs. numbers&        52.2&        50.1&       -2.15         &        56.8&        60.0&        3.26         &        56.2&        56.6&        0.41         &        50.8&        52.2&        1.38         \\
            &      [19.3]&      [16.6]&      (2.55)         &      [20.9]&      [20.7]&      (2.94)         &      [20.7]&      [19.2]&      (2.82)         &      [19.8]&      [19.1]&      (2.75)         \\
 Ratio fertility&        0.80&        0.72&      -0.080\sym{**} &        0.88&        0.88&    -0.00088         &        0.85&        0.84&      -0.016         &        0.74&        0.71&      -0.030         \\
            &      [0.29]&      [0.24]&     (0.038)         &      [0.32]&      [0.30]&     (0.044)         &      [0.31]&      [0.29]&     (0.042)         &      [0.29]&      [0.26]&     (0.039)         \\
 Ratio population&        1.01&        0.89&       -0.12\sym{**} &        1.12&        1.10&      -0.018         &        1.08&        1.05&      -0.031         &        0.96&        0.92&      -0.036         \\
            &      [0.30]&      [0.24]&     (0.049)         &      [0.30]&      [0.30]&     (0.055)         &      [0.30]&      [0.27]&     (0.052)         &      [0.27]&      [0.22]&     (0.045)         \\
 
\bottomrule \end{tabular} } \begin{tablenotes} \item \scriptsize \emph{Notes:} . \end{tablenotes} \end{threeparttable} \end{table} 

\end{landscape}


\end{document}
