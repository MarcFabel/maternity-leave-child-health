% 	TO DO: 
%	Wie passen die neu eingefügten Bilder (RF,fertilty hist)
 

%	MZ Tabelle nicht einfügen: DDRD estimate von Regression mit 3 CG


%--------------------------------------------------------------------
%	DOCUMENT CLASS
%--------------------------------------------------------------------
\documentclass[11pt, a4paper]{article} % type of document (paper, presentation, book,...); scrartcl class with sans serif titles, European layout 
\usepackage{fullpage} % leaves less space at margins of page
\usepackage[onehalfspacing]{setspace} % determine line pitch to 1.5

%--------------------------------------------------------------------
%	INPUT
%--------------------------------------------------------------------
\usepackage[T1]{fontenc} 	% Use 8-bit encoding that has 256 glyphs
\usepackage[utf8]{inputenc} % Required for including letters with accents, Umlaute,...
\usepackage{float} 			% better control over placement of tables and figures in the text
\usepackage{graphicx} 		% input of graphics
\usepackage{xcolor} 		% advanced color package
\usepackage{url, hyperref} 	% include (clickable) URLs
\usepackage{pdfpages}		% insert pages of external pdf documents
\setlength{\parskip}{0em}	% vertical spacing for paragraphs
\setlength{\parindent}{0em}	% horizonzal spacing for paragraphs
\usepackage{tikz}
\usepackage{tikzscale}		% helps to adjust tikz pictures to textwidth/linewidth
\usetikzlibrary{decorations.pathreplacing}
\usetikzlibrary{patterns}
\usetikzlibrary{arrows}

% Have sections in TOC, but not in text
\usepackage{xparse}% for easier management of optional arguments
\ExplSyntaxOn
\NewDocumentCommand{\TODO}{msom}
{
	\IfBooleanF{#1}% do nothing if it's starred
	{
		\cs_if_eq:NNT #1 \chapter { \cleardoublepage\mbox{} }
		\refstepcounter{\cs_to_str:N #1}
		\IfNoValueTF{#3}
		{
			\addcontentsline{toc}{\cs_to_str:N #1}{\protect\numberline{\use:c{the\cs_to_str:N #1}}#4}
		}
		{
			\addcontentsline{toc}{\cs_to_str:N #1}{\protect\numberline{\use:c{the\cs_to_str:N #1}}#3}
		}
	}
	\cs_if_eq:NNF #1 \chapter { \mbox{} }% allow page breaks after sections
}
\ExplSyntaxOff

%--------------------------------------------------------------------
%	TABLES, FIGURES, LISTS
%--------------------------------------------------------------------
\usepackage{booktabs} 		% better tables
\usepackage{longtable}		% tables that may be continued on the next page
\usepackage{threeparttable} % add notes below tables
\renewcommand\TPTrlap{}		% add margins on the side of the notes
	\renewcommand\TPTnoteSettings{%
	\setlength\leftmargin{5 pt}%
	\setlength\rightmargin{5 pt}%
}
\usepackage[
center, format=plain,
font=normalsize,
nooneline,
labelfont={bf}
]{caption} 				% change format of captions of tables and graphs 
%USED IN MPHIL: \usepackage[labelfont=bf,labelsep = period, singlelinecheck=off,justification=raggedright]{caption}, other specifications which are nice: labelformat = parens -> number in paranthesis 


%\usepackage{threeparttablex} % for "ThreePartTable" environment, helps to combine threepart and longtable

% Allow line breaks with \\ in column headings of tables
\newcommand{\clb}[3][c]{%
	\begin{tabular}[#1]{@{}#2@{}}#3\end{tabular}}

% allow line breaks with \\ in row titles
\usepackage{multirow}

\newcommand{\rlb}[3][c]{%
\multirow{2}{*}{\begin{tabular}[#1]{@{}#2@{}}#3\end{tabular}}}% optional argument: b = bottom or t= top alignment


\usepackage[singlelinecheck=on]{subcaption}%both together help to have subfigures
\usepackage{wrapfig}				% wrap text around figure


\usepackage{rotating}				% rotating figures & tables
\usepackage{enumerate}				% change appearance of the enumerator
\usepackage{paralist, enumitem}		% better enumerations
\setlist{noitemsep}					% no additional vertical spacing for enurations
%--------------------------------------------------------------------
%	MATH
%--------------------------------------------------------------------
\usepackage{amsmath,amssymb,amsfonts} % more math symbols and commands
\let\vec\mathbf				 % make vector bold, with no arrow and not in italic

%--------------------------------------------------------------------
%	LANGUAGE SPECIFICS
%--------------------------------------------------------------------
\usepackage[american]{babel} % man­ages cul­tur­ally-de­ter­mined ty­po­graph­i­cal (and other) rules, and hy­phen­ation pat­terns
\usepackage{csquotes} % language specific quotations

%--------------------------------------------------------------------
%	BIBLIOGRAPHY & CITATIONS
%--------------------------------------------------------------------
\usepackage{csquotes} % language specific quotations
\usepackage{etex}		% some more Tex functionality
\usepackage[nottoc]{tocbibind} %add bibliography to TOC
\usepackage[authoryear, round, comma]{natbib} %biblatex

%--------------------------------------------------------------------
%	PATHS
%--------------------------------------------------------------------
\makeatletter
\def\input@path{{../../analysis/tables/}}	%PATH TO TABLES
%or: \def\input@path{{/path/to/folder/}{/path/to/other/folder/}}
\makeatother
\graphicspath{{../../analysis/graphs/}}		% PATH TO GRAPHS

%--------------------------------------------------------------------
%	LAYOUT
%--------------------------------------------------------------------
\usepackage[left=3cm,right=3cm,top=2cm,bottom=3cm]{geometry}
\usepackage{pdflscape} % lscape.sty Produce landscape pages in a (mainly) portrait document.

\definecolor{darkblue}{rgb}{0.0,0.0,0.6}

% CAPTIAL LETTERS FOR SECTION CAPTIONS
%\usepackage{sectsty}
%\sectionfont{\normalfont\scshape\centering\textbf}
%\renewcommand{\thesection}{\Roman{section}.}
%\renewcommand{\thesubsection}{\Alph{subsection}.}%\thesection\Alph{subsection}.
%\subsectionfont{\itshape}
%\subsubsectionfont{\scshape}
%\newcommand\relphantom[1]{\mathrel{\phantom{#1}}}
%\setlength\topmargin{0.1in} \setlength\headheight{0.1in}
%\setlength\headsep{0in} \setlength\textheight{9.2in}
%\setlength\textwidth{6.3in} \setlength\oddsidemargin{0.1in}
%\setlength\evensidemargin{0.1in}

\hypersetup{
  colorlinks  = true,
  citecolor   = darkblue,
 	linkcolor   = darkblue,
  urlcolor    = darkblue 
} % macht die URLS blau   
     
\usepackage{lettrine}	% First letter capitalized

% have date in month year format (i.e. omit the day in dates)
\usepackage{datetime}
\newdateformat{monthyeardate}{%
  \monthname[\THEMONTH], \THEYEAR}
%--------------------------------------------------------------------
%	AUTHOR & TITLE
%--------------------------------------------------------------------
\title{Maternity Leave and Long-Term Health Outcomes of Children\footnote{We are grateful to Maarten Lindeboom, Erik Plug, Helmut Rainer and participants at several conferences for helpful comments and suggestions. The authors gratefully acknowledge financial support from the Leibniz association. All errors and omissions are our own.
[Q to ND:  other people which we would like to thank: Daniel Kühnle, Anna Raute, Mathias Hübener, Tanya Wilson]}}
\author{Natalia Danzer \& Marc Fabel \thanks{Marc Fabel (corresponding author): Munich Graduate School of Economics (MGSE) and ifo Institute for Economic Research, ifo Center for Labor and Demographic Economics (email: \href{mailto:fabel@ifo.de}{fabel@ifo.de}).\newline Natalia Danzer: Free University of Berlin, CESifo and IZA (email: \href{mailto:natalia.danzer@fu-berlin.de}{natalia.danzer@fu-berlin.de}).}}

\date{\monthyeardate\today}








%--------------------------------------------------------------------
%	BEGIN DOCUMENT
%--------------------------------------------------------------------
\begin{document}
\setcounter{page}{0}  
\tableofcontents
\newpage
\setcounter{page}{1}    
\maketitle

\textbf{\color{red} Preliminary and incomplete draft\newline Please do not cite or circulate without the authors' permission}
\renewcommand{\abstractname}{\vspace{-\baselineskip}} % GET RID OF ABSTRACT TITLE

  \begin{abstract}\noindent 
   \footnotesize{\begin{center}\textbf{Abstract}\end{center} This paper assesses the impact of the length of maternity leave on children’s long-run health outcomes. Our quasi-experimental design evaluates an expansion in maternity leave coverage from two to six months, which occurred in the Federal Republic of Germany in 1979. The expansion came into effect after a sharp cutoff date and significantly increased the time working mothers stayed at home with their newborns. In our analysis, we exploit the German Micro Census and hospital registry data, containing detailed information about the universe of inpatients' diagnoses and treatment for the years 1995 to 2014. By tracking the health of treated and control children from age 16 up to age 35, we provide new insights into the trajectory of health differentials over the life-cycle.
   	We find a positive effect of the legislative change on several measures of long-term child health. Our intention-to-treat estimates suggest that children who were born shortly after the implementation of the reform experience fewer hospital admissions and are less likely to be diagnosed with mental and behavioral disorders.\\\newline \textbf{Keywords:} Early childhood development, health, paid maternity leave, life-cycle approach \newline \textbf{JEL codes:} I10, J13, J18}
    \end{abstract}

\newpage



%\printbibliography

%--------------------------------------------------------------------
% BIBLIOGRAPHY
%--------------------------------------------------------------------
\newpage


%\bibliographystyle{ecca_edited}%previous style-chicago
%\bibliography{mlch_bibliography}
%--------------------------------------------------------------------
% FIGURES AND TABLES
%--------------------------------------------------------------------
%\newpage
%\section{Figures and tables}
\newpage
\TODO\section{Figures}
\vspace*{\fill}
{\Huge \begin{center}\textbf{FIGURES}\end{center}}
\vspace*{\fill}\clearpage
%--------------------------------------------

%--------------------------------------------


%WMWMWMWMWMWMWMWMWMWMWMWMWMWMWMWMWMWMWMWM
% RESULTS
%WMWMWMWMWMWMWMWMWMWMWMWMWMWMWMWMWMWMWMWM
%--------------------------------------------
\newpage
%Life-course Hospital admission
\begin{landscape}
	\vspace*{\fill}
	\begin{figure}[H]\centering
		\caption{Life-course approach for hospital admission}\label{fig: lc_hospital2_frg_DD}
		\begin{subfigure}[h]{0.31\linewidth}\centering\caption{Total}
			\includegraphics[width=\linewidth]{paper/lc_r_popf_hospital2_total_gdr.pdf}
		\end{subfigure}
		\begin{subfigure}[h]{0.31\linewidth}\centering\caption{Women}
			\includegraphics[width=\linewidth]{paper/lc_r_popf_hospital2_female_gdr.pdf}
		\end{subfigure}
		\begin{subfigure}[h]{0.31\linewidth}\centering\caption{Men}
			\includegraphics[width=\linewidth]{paper/lc_r_popf_hospital2_male_gdr.pdf}
		\end{subfigure}
		\scriptsize
		\begin{minipage}{\linewidth}
			\emph{Notes:} The figures plot DD estimates (along with 90\% and 95\% confidence intervals) for the impact of the reform on hospital admission over the life-course. The light gray line in the background represents the baseline mean of the pre-reform treated cohort. Panel a shows the results for all admissions, whereas panel b and c show the estimates for females and males respectively. The control group is comprised of children	that are born in the same months but one year before (i.e. children born between November 1977 and October 1978).
		\end{minipage}
	\end{figure}
	\vspace*{\fill}\clearpage
\end{landscape}
%--------------------------------------------
% Hospital - Reduced form pooled
\newgeometry{left=1cm,right=1cm,top=3cm,bottom=3cm} 
\begin{landscape}
	\vspace*{\fill}
	\begin{figure}
		[H]\centering
		\caption{RD plots for hospital admission (pooled)}\label{fig: rf_hospital2_pooled}
		\begin{subfigure}[h]{0.31\linewidth}\centering\caption{Total}
			\includegraphics[width=\linewidth]{paper/rd_r_popf_hospital2_total_pooled.pdf}
		\end{subfigure}
		\begin{subfigure}[h]{0.31\linewidth}\centering\caption{Women}
			\includegraphics[width=\linewidth]{paper/rd_r_popf_hospital2_female_pooled.pdf}
		\end{subfigure}
		\begin{subfigure}[h]{0.31\linewidth}\centering\caption{Men}
			\includegraphics[width=\linewidth]{paper/rd_r_popf_hospital2_male_pooled.pdf}
		\end{subfigure}
		\scriptsize
		\begin{minipage}{0.95\linewidth}
			\emph{Notes:} The figure plots the average number of diagnoses per 1,000 individuals for month-of-birth cohorts born half a year around the cut-off date of the 1979 maternity leave expansion. The monthly averages are taken over the entire sample length from 1995 to 2014. The dashed lines represent linear fitted values along with 90\%/95\% confidence intervals. The solid vertical red line divides pre- and post-reform schemes (two vs. six months of leave).\newline
			\emph{Source:} Hospital registry data for individuals born between November 1978 and October 1979.
		\end{minipage}
	\end{figure}
	\vspace*{\fill}\clearpage
\end{landscape}
\restoregeometry  
%--------------------------------------------
% Hospital -Reduced form AGE groups 
% original 3 3 2 3
\newgeometry{left=1cm,right=1cm,top=3cm,bottom=3cm} 
\begin{landscape}
	\vspace*{\fill}
	\begin{figure}
		[H]\centering
		\caption{RD plots for hospital admission across age groups}\label{fig: rf_hospital2_agegroup}
		\includegraphics[width=0.85\linewidth]{paper/rd_r_popf_hospital2_overview_agegroups_CIfits.pdf}
		\scriptsize
		\begin{minipage}{0.9\linewidth}
			\emph{Notes:} The figure plots the number of diagnoses per 1,000 individuals for month-of-birth cohorts born half a year around the cut-off date of the 1979 maternity leave expansion across gender and different age groups. The first column reports the ratios for all patients, and the second and third column do so for women and men, respectively. The rows show the ratios across different age groups. The dashed lines represent linear fitted values along with 90\%/95\% confidence intervals. The solid vertical red line divides pre- and post-reform schemes (two vs. six months of leave).\newline
			\emph{Source:} Hospital registry data for individuals born between November 1978 and October 1979.
		\end{minipage}
	\end{figure}
	\vspace*{\fill}\clearpage
\end{landscape}
\restoregeometry
%--------------------------------------------
% Figure: effect sizes and frequency across chapters

\begin{landscape}
	\vspace*{\fill}
	\begin{figure}[H]\centering
		\caption{Intention-to-treat effects across main chapters}\label{fig: DD_across_main chapters}
		\begin{subfigure}[h]{0.31\linewidth}\centering\caption{Total}
			\includegraphics[width=\linewidth]{paper/effect_chapters_frequency.pdf}
		\end{subfigure}
		\begin{subfigure}[h]{0.31\linewidth}\centering\caption{Women}
			\includegraphics[width=\linewidth]{paper/effect_chapters_frequency_f.pdf}
		\end{subfigure}
		\begin{subfigure}[h]{0.31\linewidth}\centering\caption{Men}
			\includegraphics[width=\linewidth]{paper/effect_chapters_frequency_m.pdf}
		\end{subfigure}
		\scriptsize
		\begin{minipage}{\linewidth}
			\emph{Notes:} The figures plots intention-to-treat estimates (along with 90\%/95\% confidence intervals) across the main diagnosis chapters. Furthermore, they indicate how often each distribution occurs across the entire length (1995-2014). The outcomes are defined as the number of cases per 1,000 individuals (births). The point estimates are coming from a DiD regression as described in section \ref{sec:empirical_strategy}, with a bandwidth of six months, month-of-birth and year fixed effects, and clustered standard errors on the month-of-birth level. The control group is comprised of children	that are born in the same months but one year before (i.e. children born between November 1977 and October 1978). \newline
			\emph{Legend:} Infectious and parasitic diseases (IPD), neoplasms (Neo), mental \& behavioral disorders (MBD), diseases of the nervous system (Ner), diseases of the sense organs (Sen), diseases of the circulatory system (Cir), diseases of the respiratory system (Res), diseases of the digestive system (Dig), diseases of the skin and subcutaneous tissue (SST), diseases of the musculoskeletal system (Mus), diseases of the genitourinary system (Gen), "symptoms, signs, and ill-defined conditions" (Sym), "injury, poisoning and certain other consequences of external causes" (Ext).
			
		\end{minipage}
	\end{figure}
	\vspace*{\fill}\clearpage
\end{landscape}
%--------------------------------------------

% D5 - LC Approach (Mental and behavioral disod)
\begin{landscape}
	\vspace*{\fill}
	\begin{figure}[H]\centering
		\caption{Life-course approach for mental and behavioral disorders}\label{fig: lc_d5_frg_DD}r	\begin{subfigure}[h]{0.31\linewidth}\centering\caption{Total}
			\includegraphics[width=\linewidth]{paper/lc_r_popf_d5_total_gdr.pdf}
		\end{subfigure}
		\begin{subfigure}[h]{0.31\linewidth}\centering\caption{Women}
			\includegraphics[width=\linewidth]{paper/lc_r_popf_d5_female_gdr.pdf}
		\end{subfigure}
		\quad
		\begin{subfigure}[h]{0.31\linewidth}\centering\caption{Men}
			\includegraphics[width=\linewidth]{paper/lc_r_popf_d5_male_gdr.pdf}
		\end{subfigure}
		\scriptsize
		\begin{minipage}{\linewidth}
			\emph{Notes:} The figures plot DD estimates (along with 90\% and 95\% confidence intervals) for the impact of the reform on mental and behavioral disorders over the life-course. The light gray line in the background represents the baseline mean of the pre-reform treated cohort. Panel a shows the results for all admissions, whereas panel b and c show the estimates for females and males respectively. The control group is comprised of children	that are born in the same months but one year before (i.e. children born between November 1977 and October 1978).
		\end{minipage}
	\end{figure}
	\vspace*{\fill}\clearpage
\end{landscape}

%--------------------------------------------
% d5 - RF pooled
\newgeometry{left=1cm,right=1cm,top=3cm,bottom=3cm} 
\begin{landscape}
	\vspace*{\fill}
	\begin{figure}
		[H]\centering
		\caption{RD plots for mental \& behavioral disorders (pooled)}\label{fig: rf_d5_pooled}
		\begin{subfigure}[h]{0.31\linewidth}\centering\caption{Total}
			\includegraphics[width=\linewidth]{paper/rd_r_popf_d5_total_pooled.pdf}
		\end{subfigure}
		\begin{subfigure}[h]{0.31\linewidth}\centering\caption{Women}
			\includegraphics[width=\linewidth]{paper/rd_r_popf_d5_female_pooled.pdf}
		\end{subfigure}
		\begin{subfigure}[h]{0.31\linewidth}\centering\caption{Men}
			\includegraphics[width=\linewidth]{paper/rd_r_popf_d5_male_pooled.pdf}
		\end{subfigure}
		\scriptsize
		\begin{minipage}{0.95\linewidth}
			\emph{Notes:} The figure plots the average number of diagnoses per 1,000 individuals for month-of-birth cohorts born half a year around the cut-off date of the 1979 maternity leave expansion. The monthly averages are taken over the entire sample length from 1995 to 2014. The dashed lines represent linear fitted values along with 90\%/95\% confidence intervals. The solid vertical red line divides pre- and post-reform schemes (two vs. six months of leave).\newline
			\emph{Source:} Hospital registry data for individuals born between November 1978 and October 1979.
		\end{minipage}
	\end{figure}
	\vspace*{\fill}\clearpage
\end{landscape}
\restoregeometry 

%--------------------------------------------
% D5 - RF (age group)
\newgeometry{left=1cm,right=1cm,top=3cm,bottom=3cm} 
\begin{landscape}
	\vspace*{\fill}
	\begin{figure}
		[H]\centering
		\caption{RD plots for mental \& behavioral disorders across age groups}\label{fig: rf_d5_agegroup}
		\includegraphics[width=0.85\linewidth]{paper/rd_r_popf_d5_overview_agegroups_CIfits.pdf}
		\scriptsize
		\begin{minipage}{0.9\linewidth}
			\emph{Notes:} The figure plots the number of diagnoses per 1,000 individuals for month-of-birth cohorts born half a year around the cut-off date of the 1979 maternity leave expansion across gender and different age groups. The first column reports the ratios for all patients, and the second and third column do so for women and men, respectively. The rows show the ratios across different age groups. The dashed lines represent linear fitted values along with 90\%/95\% confidence intervals. The solid vertical red line divides pre- and post-reform schemes (two vs. six months of leave).\newline
			\emph{Source:} Hospital registry data for individuals born between November 1978 and October 1979.
		\end{minipage}
	\end{figure}
	\vspace*{\fill}\clearpage
\end{landscape}
\restoregeometry  
%--------------------------------------------
% Graph D5 partition - Subcategories
\vspace*{\fill}
\begin{figure}[H]\centering
	\caption{The top five subcategories of mental \& behavioral diagnoses}\label{fig: d5partition}
	\includegraphics[width=0.8\linewidth]{../../analysis/graphs/paper/d5partition_lfstat.pdf}
	\scriptsize
	\begin{minipage}{0.9\linewidth}
	\emph{Notes:} This figure plots the yearly number of diagnoses for the treatment cohort (i.e. the individuals born between November 1978 and October 1979). The subcategories are ordered by their occurrence in 2014 (from the most to the least frequent diagnosis), which also coincides by chance with the ordering in the ICD-10 classification system. The five most frequent subcategories - as shown here - comprise more than 95\% of all mental and behavioral disorders. 
	\end{minipage}
\end{figure}
\vspace*{\fill}\clearpage%--------------------------------------------
%% figure: iverview of d5 subcategories (effects + frequency)
%\newgeometry{left=1cm,right=1cm,top=3cm,bottom=3cm} 
%\begin{landscape}
%	\vspace*{\fill}
%	\begin{figure}
%		[H]\centering
%		\caption{ITT effect for subcategories of mental \& behavioral disorders (pooled)}\label{fig: ITT_d5_subcategories}
%		\begin{subfigure}[h]{0.31\linewidth}\centering\caption{Total}
%			\includegraphics[width=\linewidth]{paper/effect_d5_frequency.pdf}
%		\end{subfigure}
%		\begin{subfigure}[h]{0.31\linewidth}\centering\caption{Women}
%			\includegraphics[width=\linewidth]{paper/effect_d5_frequency_f.pdf}
%		\end{subfigure}
%		\begin{subfigure}[h]{0.31\linewidth}\centering\caption{Men}
%			\includegraphics[width=\linewidth]{paper/effect_d5_frequency_m.pdf}
%		\end{subfigure}
%		\scriptsize
%		\begin{minipage}{0.95\linewidth}
%			\emph{Notes:} The figure plots ITT effects for the five most common subcategories of mental and behavioral disorders....
%
%			%the average number of diagnoses per 1,000 individuals for month-of-birth cohorts born half a year around the cut-off date of the 1979 maternity leave expansion. The monthly averages are taken over the entire sample length from 1995 to 2014. The dashed lines represent linear fitted values along with 90\%/95\% confidence intervals. The solid vertical red line divides pre- and post-reform schemes (two vs. six months of leave).\newline
%			\emph{Source:} Hospital registry data for individuals born between November 1978 and October 1979.
%		\end{minipage}
%	\end{figure}
%	\vspace*{\fill}\clearpage
%\end{landscape}
%\restoregeometry 
%--------------------------------------------
% map: population density per AMR in Germany
%\newpage
%
%\vspace*{\fill}
%\begin{figure}[H]\centering
%	\caption{Region-level population density}\label{fig: AMR_regions_population_density}
%	\includegraphics[width=0.8 \linewidth]{paper/AMR_popdensity.png}
%	\scriptsize
%	\begin{minipage}{0.9\linewidth}
%		\emph{Notes:} This map shows the regional variation of population density across German regions. \emph{Source:} Own representation with data from the Federal Institute for Research on Building, Urban Affairs and Spatial Development (BBSR) and the Regional Database Germany.
%	\end{minipage}
%\end{figure}
%\vspace*{\fill}\clearpage





%--------------------------------------------------------------------
% TABLES
%--------------------------------------------------------------------
\newpage
\TODO\section{Tables}
\vspace*{\fill}
{\Huge \begin{center}\textbf{TABLES}\end{center}}
\vspace*{\fill}\clearpage

%--------------------------------------------




%--------------------------------------------
% HOSPITAL - Difference in Difference tables for 
\newpage
\vspace*{\fill}
\begin{table}[H] \centering 
 \begin{threeparttable} \centering \caption{ITT effects on \textbf{hospital admission (total)}}\label{tab: DD_hopsital2_total}
  {\def\sym#1{\ifmmode^{#1}\else\(^{#1}\)\fi} 
 	\begin{tabular}{l*{8}{c}}
 		\toprule 
 		%\multicolumn{5}{l}{Dependant variable: \textbf{Hospital admission (total)}}\\ \\ 
 		& \multicolumn{7}{c}{Estimation window} \\ 
 		\cmidrule(lr){2-8}
 		&\multicolumn{1}{c}{(1)}&\multicolumn{1}{c}{(2)}&\multicolumn{1}{c}{(3)}&\multicolumn{1}{c}{(4)}&\multicolumn{1}{c}{(5)}&\multicolumn{1}{c}{(6)}&\multicolumn{1}{c}{(7)}\\
 		&\multicolumn{1}{c}{1M}&\multicolumn{1}{c}{2M}&\multicolumn{1}{c}{3M}&\multicolumn{1}{c}{4M}&\multicolumn{1}{c}{5M}&\multicolumn{1}{c}{6M}&\multicolumn{1}{c}{Donut}\\
 		\midrule
 		\multicolumn{5}{l}{\emph{Panel A. Over entire length of the life-course}} \\
 		\hspace*{10pt}Overall&     -0.0878\sym{***}&      -0.897         &      -2.730\sym{**} &      -2.819\sym{***}&      -2.259\sym{**} &      -2.066\sym{***}&      -2.462\sym{***}\\
                    &  (2.72e-14)         &     (0.923)         &     (1.087)         &     (0.932)         &     (0.801)         &     (0.719)         &     (0.831)         \\
\midrule Dependent mean&       91.63         &       94.22         &       96.53         &       95.22         &       94.77         &       94.77         &       95.40         \\
Effect in SDs [\%]  &       0.920         &       8.980         &       26.20         &       27.48         &       22.74         &       21.25         &       25.28         \\
Observations        &          38         &          76         &         114         &         152         &         190         &         228         &         190         \\
 \\ \\
 		\multicolumn{5}{l}{\emph{Panel B. Age brackets}} \\
 		 \hspace*{10pt}Age 27-29&      -0.564\sym{***}&     -0.0352         &      -1.582\sym{*}  &      -2.196\sym{**} &      -1.907\sym{**} &      -1.462\sym{**} &      -1.641\sym{*}  \\
                    &  (2.15e-14)         &     (0.594)         &     (0.880)         &     (0.833)         &     (0.694)         &     (0.687)         &     (0.826)         \\
 \hspace*{10pt}Age 30-32&      -0.343\sym{***}&      -0.687         &      -2.705\sym{*}  &      -2.906\sym{**} &      -2.456\sym{**} &      -2.303\sym{***}&      -2.695\sym{***}\\
                    &  (1.99e-14)         &     (1.172)         &     (1.346)         &     (1.019)         &     (0.858)         &     (0.803)         &     (0.880)         \\
 \hspace*{10pt}Age 33-35&      -0.159\sym{***}&      -2.253         &      -4.085\sym{**} &      -3.646\sym{***}&      -2.623\sym{**} &      -2.657\sym{***}&      -3.157\sym{***}\\
                    &  (8.11e-15)         &     (1.303)         &     (1.335)         &     (1.160)         &     (1.132)         &     (0.945)         &     (1.098)         \\
 
 		\bottomrule 
 	\end{tabular}}
 	\begin{tablenotes} 
 		\item \scriptsize \emph{Notes:} Clustered standard errors in parentheses. All regression are run with CG2 (i.e. the cohort prior to the reform) and with month-of-birth FEs. The 'overall' specification includes year fixed effects, as well. The outcome variables are defined as the number of cases per thousand individuals (births). Dependent mean and the effect size correspond to pre-reform values of the treated group.
 	\end{tablenotes} 
 \end{threeparttable} 
 \end{table}
\vspace*{\fill}\clearpage 
\vspace*{\fill}
 \begin{table}[H] \centering 
 	\begin{threeparttable} \centering \caption{ITT effects on \textbf{hospital admission (women)}}\label{tab: DD_hopsital2_female} {\def\sym#1{\ifmmode^{#1}\else\(^{#1}\)\fi} 
 			\begin{tabular}{l*{6}{c}}
 				\toprule 
 				%\multicolumn{5}{l}{Dependant variable: \textbf{Hospital admission (total)}}\\ \\ 
 				& \multicolumn{5}{c}{Estimation window} \\ 
 				\cmidrule(lr){2-6}
 				&\multicolumn{1}{c}{(1)}&\multicolumn{1}{c}{(2)}&\multicolumn{1}{c}{(3)}&\multicolumn{1}{c}{(4)}&\multicolumn{1}{c}{(5)}\\
 				&\multicolumn{1}{c}{6M}&\multicolumn{1}{c}{5M}&\multicolumn{1}{c}{4M}&\multicolumn{1}{c}{3M}&\multicolumn{1}{c}{Donut}\\
 				\midrule
 				\multicolumn{5}{l}{\emph{Panel A.Over entire length of the life-course}} \\
 				\hspace*{10pt}Overall&       0.149         &      -0.766         &      -1.815\sym{**} &      -2.278\sym{***}\\
                    &     (2.038)         &     (1.088)         &     (0.807)         &     (0.751)         \\
\midrule Dependent mean&       122.3         &       122.0         &       122.4         &       123.3         \\
Effect in SDs [\%]  &       1.350         &       6.530         &       16.29         &       20.28         \\
Observations        &         160         &         320         &         480         &         400         \\
 \\ \\
 				\multicolumn{5}{l}{\emph{Panel B. Age brackets}} \\
 				\hspace*{10pt}Age 17-21&      -2.121\sym{+}  &      -1.274         &      -1.931\sym{*}  &      -2.916\sym{***}&      -3.322\sym{***}\\
                    &     (1.269)         &     (1.018)         &     (0.959)         &     (0.935)         &     (0.985)         \\
 \hspace*{10pt}Age 22-26&       1.661         &       1.126         &      0.0274         &      -0.510         \\
                    &     (3.675)         &     (1.806)         &     (1.267)         &     (1.117)         \\
 \hspace*{10pt}Age 27-31&      -1.379         &      -1.669         &      -2.605\sym{**} &      -2.762\sym{**} &      -2.944\sym{***}\\
                    &     (1.765)         &     (1.336)         &     (1.163)         &     (1.004)         &     (0.917)         \\
 \hspace*{10pt}Age 32-35&      -0.605         &      -1.004         &      -0.841         &      -1.212         &      -1.810\sym{*}  \\
                    &     (1.300)         &     (1.165)         &     (1.024)         &     (0.866)         &     (0.941)         \\
 
 				\bottomrule 
 		\end{tabular}}
 		\begin{tablenotes} 
 			\item \scriptsize \emph{Notes:} The table shows DiD estimates of the 1979 maternity leave reform on hospital admission for different estimation windows around the cutoff. The \textit{`Donut'} specification uses a bandwidth of half a year and excludes children born in April and May. Panel A shows the effect for the entire pooled time frame and panel B breaks the life-course up in age brackets. The outcome variables are defined as the number of cases per thousand individuals (births). All regressions control for year and month-of-birth fixed effects. The control group is comprised of children that are born in the same months but one year before the reform (i.e. children born between November 1977 and October 1978). In order to compare the two birth cohorts at the same age, I shift the control cohort from wave $t$ to wave $t+1$. The dependent mean and the effect size in standard deviation units correspond to pre-reform values of the treated group. Clustered standard errors are reported in parentheses. \newline Significance levels: + p < 0.15, * p < 0.10, ** p < 0.05, *** p < 0.01. \newline 	\emph{Source:} Hospital registry data.
 		\end{tablenotes} 
 	\end{threeparttable} 
 \end{table}
\vspace*{\fill}\clearpage 
\vspace*{\fill}
\begin{table}[H] \centering 
	\begin{threeparttable} \centering \caption{ITT effects on \textbf{hospital admission (men)}}\label{tab: DD_hopsital2_total}
		{\def\sym#1{\ifmmode^{#1}\else\(^{#1}\)\fi} 
			\begin{tabular}{l*{8}{c}}
				\toprule 
				%\multicolumn{5}{l}{Dependant variable: \textbf{Hospital admission (total)}}\\ \\ 
				& \multicolumn{7}{c}{Estimation window} \\ 
				\cmidrule(lr){2-8}
				&\multicolumn{1}{c}{(1)}&\multicolumn{1}{c}{(2)}&\multicolumn{1}{c}{(3)}&\multicolumn{1}{c}{(4)}&\multicolumn{1}{c}{(5)}&\multicolumn{1}{c}{(6)}&\multicolumn{1}{c}{(7)}\\
				&\multicolumn{1}{c}{1M}&\multicolumn{1}{c}{2M}&\multicolumn{1}{c}{3M}&\multicolumn{1}{c}{4M}&\multicolumn{1}{c}{5M}&\multicolumn{1}{c}{6M}&\multicolumn{1}{c}{Donut}\\
				\midrule
				\multicolumn{5}{l}{\emph{Panel A. Over entire length of the life-course}} \\
				\hspace*{10pt}Overall&      -0.265\sym{***}&      -2.400\sym{**} &      -4.769\sym{***}&      -4.733\sym{***}&      -3.471\sym{***}&      -3.016\sym{***}&      -3.566\sym{***}\\
                    &  (1.56e-14)         &     (0.983)         &     (1.336)         &     (1.276)         &     (1.183)         &     (1.045)         &     (1.220)         \\
\midrule Dependent mean&       96.05         &       98.57         &       101.5         &       100.2         &       99.27         &       99.07         &       99.67         \\
Effect in SDs [\%]  &       2.440         &       21.45         &       41.15         &       42.01         &       31.06         &       27.38         &       32.24         \\
Observations        &          38         &          76         &         114         &         152         &         190         &         228         &         190         \\
 \\ \\
				\multicolumn{5}{l}{\emph{Panel B. Age brackets}} \\
				\hspace*{10pt}Age 27-29&      -0.327\sym{***}&      -0.952         &      -3.046\sym{**} &      -4.157\sym{**} &      -2.673\sym{*}  &      -1.677         &      -1.947         \\
                    &  (4.14e-14)         &     (0.825)         &     (1.195)         &     (1.556)         &     (1.466)         &     (1.336)         &     (1.517)         \\
 \hspace*{10pt}Age 30-32&       0.867\sym{***}&      -0.528         &      -2.982\sym{*}  &      -3.115\sym{**} &      -2.617\sym{**} &      -2.303\sym{**} &      -2.937\sym{**} \\
                    &  (1.99e-14)         &     (1.027)         &     (1.453)         &     (1.141)         &     (0.996)         &     (0.980)         &     (1.104)         \\
 \hspace*{10pt}Age 33-35&      -2.043\sym{***}&      -5.517\sym{**} &      -8.421\sym{***}&      -7.154\sym{***}&      -5.190\sym{***}&      -5.186\sym{***}&      -5.815\sym{***}\\
                    &  (2.93e-14)         &     (1.762)         &     (1.917)         &     (1.726)         &     (1.733)         &     (1.462)         &     (1.729)         \\
 
				\bottomrule 
		\end{tabular}}
		\begin{tablenotes} 
			\item \scriptsize \emph{Notes:} Clustered standard errors in parentheses. All regression are run with CG2 (i.e. the cohort prior to the reform) and with month-of-birth FEs. The 'overall' specification includes year fixed effects, as well. The outcome variables are defined as the number of cases per thousand individuals (births). Dependent mean and the effect size correspond to pre-reform values of the treated group.
		\end{tablenotes} 
	\end{threeparttable} 
\end{table}
\vspace*{\fill}\clearpage 

%--------------------------------------------
% ITT effects nach chaptern
% original 3 3 2 3
%\newpage
%\newgeometry{left=3cm,right=3cm,top=1cm,bottom=2.5cm} 
%\vspace*{\fill}
%\begin{table}[H] \centering 
%	\begin{threeparttable} \centering \caption{ITT effects on hospital admission and different diagnoses chapters (total)}\label{tab: ITT_across_chapters_per_age_group_total}
%		{\def\sym#1{\ifmmode^{#1}\else\(^{#1}\)\fi} 
%			\begin{tabular}{l*{5}{c}}
%				\toprule 
%				&\multicolumn{1}{c}{(1)}&\multicolumn{1}{c}{(2)}&\multicolumn{1}{c}{(3)}&\multicolumn{1}{c}{(4)}&\multicolumn{1}{c}{(5)}\\
%				\midrule
%				&\multirow{2}{*}{Overall} & \multicolumn{4}{c}{Age brackets} \\ 
%				\cmidrule(lr){3-6}
%				&&\multicolumn{1}{c}{17-21}&\multicolumn{1}{c}{22-26}&\multicolumn{1}{c}{27-31}&\multicolumn{1}{c}{32-35}\\
%				
%				\midrule
%				
%				Hospital            &      -2.168\sym{**} &      -1.517         &      -0.611         &      -2.665\sym{***}&      -3.869\sym{***}\\
                    &     (0.782)         &     (0.946)         &     (0.937)         &     (0.826)         &     (1.083)         \\
IPD                 &     -0.0838\sym{**} &      0.0130         &      -0.162         &      -0.126\sym{**} &     -0.0197         \\
                    &    (0.0334)         &    (0.0680)         &     (0.108)         &    (0.0548)         &     (0.111)         \\
Neo                 &      0.0269         &      -0.198         &       0.336\sym{**} &       0.121         &      -0.217         \\
                    &    (0.0821)         &     (0.155)         &     (0.139)         &     (0.102)         &     (0.164)         \\
MBD                 &      -0.634\sym{**} &       0.174         &    -0.00769         &      -1.000\sym{**} &      -1.906\sym{***}\\
                    &     (0.249)         &     (0.263)         &     (0.420)         &     (0.357)         &     (0.372)         \\
Ner                 &     0.00791         &     -0.0919         &       0.238\sym{***}&      0.0374         &     -0.0190         \\
                    &    (0.0561)         &    (0.0558)         &    (0.0751)         &    (0.0963)         &     (0.126)         \\
Sen                 &      -0.144\sym{***}&      -0.168\sym{**} &     -0.0990\sym{*}  &      -0.172\sym{**} &      -0.261\sym{**} \\
                    &    (0.0272)         &    (0.0626)         &    (0.0518)         &    (0.0640)         &    (0.0998)         \\
Cir                 &     -0.0453         &     -0.0466         &      0.0311         &      -0.199\sym{*}  &      -0.198         \\
                    &    (0.0678)         &    (0.0782)         &    (0.0812)         &    (0.0994)         &     (0.162)         \\
Res                 &      -0.287\sym{***}&      -0.369\sym{**} &      -0.199\sym{***}&      -0.273\sym{**} &     -0.0842         \\
                    &    (0.0787)         &     (0.173)         &    (0.0694)         &     (0.107)         &     (0.167)         \\
Dig                 &      -0.395\sym{***}&      -0.421\sym{**} &      -0.485\sym{*}  &      -0.376         &      -0.494         \\
                    &     (0.131)         &     (0.177)         &     (0.257)         &     (0.227)         &     (0.361)         \\
SST                 &      0.0936\sym{**} &      0.0567         &       0.186\sym{**} &       0.152\sym{*}  &     -0.0149         \\
                    &    (0.0422)         &    (0.0947)         &    (0.0721)         &    (0.0882)         &     (0.120)         \\
Mus                 &     -0.0356         &     -0.0260         &      -0.155         &     -0.0562         &       0.152         \\
                    &    (0.0556)         &    (0.0884)         &     (0.172)         &     (0.141)         &     (0.158)         \\
Gen                 &     0.00619         &       0.241         &       0.180         &      -0.194         &     -0.0356         \\
                    &     (0.109)         &     (0.159)         &     (0.207)         &     (0.235)         &     (0.159)         \\
Sym                 &     -0.0922         &      -0.175         &       0.112         &      -0.164\sym{**} &      -0.122         \\
                    &    (0.0671)         &     (0.151)         &    (0.0988)         &    (0.0604)         &     (0.103)         \\
Ext                 &      -0.597\sym{***}&      -0.460         &      -0.685\sym{***}&      -0.429\sym{**} &      -0.612\sym{***}\\
                    &     (0.177)         &     (0.328)         &     (0.230)         &     (0.193)         &     (0.135)         \\
 
%				
%				\bottomrule 
%		\end{tabular}}
%		% \begin{tablenotes} 
%		% 	\item 
%		% \end{tablenotes} 
%	\end{threeparttable} 
%	\begin{minipage}{0.9\linewidth}
%		\scriptsize \emph{Notes:} This table reports intention-to-treat estimates across the main diagnosis chapters for the entire life-course or per age bracket. The outcomes are defined as the number of cases per 1,000 individuals (births). The point estimates are coming from a DiD regression as described in section \ref{sec:empirical_strategy}, with a bandwidth of six months, month-of-birth and year fixed effects, and clustered standard errors on the month-of-birth level. The control group is comprised of children that are born in the same months but one year before (i.e. children born between November 1977 and October 1978).\newline
%		\emph{Legend:} Infectious and parasitic diseases (IPD), neoplasms (Neo), mental \& behavioral disorders (MBD), diseases of the nervous system (Ner), diseases of the sense organs (Sen), diseases of the circulatory system (Cir), diseases of the respiratory system (Res), diseases of the digestive system (Dig), diseases of the skin and subcutaneous tissue (SST), diseases of the musculoskeletal system (Mus), diseases of the genitourinary system (Gen), "symptoms, signs, and ill-defined conditions" (Sym), "injury, poisoning and certain other consequences of external causes" (Ext).
%	\end{minipage}
%\end{table} 
%\vspace*{\fill}\clearpage 
%\restoregeometry
%--------------------------------------------
% d5 - Difference in Difference table 
\vspace*{\fill}
\begin{table}[H] \centering 
	\begin{threeparttable} \centering \caption{ITT effects on \textbf{mental and behavioral disorders (total)}}\label{tab: DD_hopsital2_total}
		{\def\sym#1{\ifmmode^{#1}\else\(^{#1}\)\fi} 
			\begin{tabular}{l*{8}{c}}
				\toprule 
				%\multicolumn{5}{l}{Dependant variable: \textbf{Hospital admission (total)}}\\ \\ 
				& \multicolumn{7}{c}{Estimation window} \\ 
				\cmidrule(lr){2-8}
				&\multicolumn{1}{c}{(1)}&\multicolumn{1}{c}{(2)}&\multicolumn{1}{c}{(3)}&\multicolumn{1}{c}{(4)}&\multicolumn{1}{c}{(5)}&\multicolumn{1}{c}{(6)}&\multicolumn{1}{c}{(7)}\\
				&\multicolumn{1}{c}{1M}&\multicolumn{1}{c}{2M}&\multicolumn{1}{c}{3M}&\multicolumn{1}{c}{4M}&\multicolumn{1}{c}{5M}&\multicolumn{1}{c}{6M}&\multicolumn{1}{c}{Donut}\\
				\midrule
				\multicolumn{5}{l}{\emph{Panel A. Over entire length of the life-course}} \\
				\hspace*{10pt}Overall&      -0.861\sym{***}&      -0.668\sym{**} &      -1.230\sym{***}&      -1.381\sym{***}&      -1.180\sym{***}&      -0.982\sym{***}&      -1.006\sym{***}\\
                    &  (1.55e-15)         &     (0.240)         &     (0.323)         &     (0.268)         &     (0.235)         &     (0.247)         &     (0.297)         \\
\midrule Dependent mean&       17.37         &       17.81         &       18.37         &       18.20         &       18.07         &          18         &       18.13         \\
Effect in SDs [\%]  &       56.97         &          45         &       70.16         &       77.65         &       67.49         &       57.94         &       58.62         \\
Observations        &          38         &          76         &         114         &         152         &         190         &         228         &         190         \\
 \\ \\
				\multicolumn{5}{l}{\emph{Panel B. Age brackets}} \\
				\hspace*{10pt}Age 27-29&      -0.681\sym{***}&      -0.212         &      -0.901\sym{**} &      -1.035\sym{**} &      -0.878\sym{***}&      -0.604\sym{**} &      -0.589\sym{*}  \\
                    &  (1.43e-15)         &     (0.240)         &     (0.381)         &     (0.366)         &     (0.299)         &     (0.287)         &     (0.334)         \\
 \hspace*{10pt}Age 30-32&     0.00497\sym{***}&      -0.163         &      -1.096\sym{*}  &      -1.349\sym{***}&      -1.112\sym{***}&      -0.916\sym{***}&      -1.100\sym{***}\\
                    &  (5.07e-15)         &     (0.407)         &     (0.537)         &     (0.413)         &     (0.351)         &     (0.319)         &     (0.359)         \\
 \hspace*{10pt}Age 33-35&      -2.187         &      -1.849\sym{***}&      -1.930\sym{***}&      -1.926\sym{***}&      -1.619\sym{***}&      -1.525\sym{***}&      -1.393\sym{***}\\
                    &         (.)         &     (0.324)         &     (0.381)         &     (0.296)         &     (0.284)         &     (0.342)         &     (0.384)         \\
 
				\bottomrule 
		\end{tabular}}
		\begin{tablenotes} 
			\item \scriptsize \emph{Notes:} Clustered standard errors in parentheses. All regression are run with CG2 (i.e. the cohort prior to the reform) and with month-of-birth FEs. The 'overall' specification includes year fixed effects, as well. The outcome variables are defined as the number of cases per thousand individuals (births). Dependent mean and the effect size correspond to pre-reform values of the treated group.
		\end{tablenotes} 
	\end{threeparttable} 
\end{table}
\vspace*{\fill}\clearpage 
\vspace*{\fill}
\begin{table}[H] \centering 
	\begin{threeparttable} \centering \caption{ITT effects on \textbf{mental and behavioral disorders (women)}}\label{tab: DD_hopsital2_total}
		{\def\sym#1{\ifmmode^{#1}\else\(^{#1}\)\fi} 
			\begin{tabular}{l*{8}{c}}
				\toprule 
				%\multicolumn{5}{l}{Dependant variable: \textbf{Hospital admission (total)}}\\ \\ 
				& \multicolumn{7}{c}{Estimation window} \\ 
				\cmidrule(lr){2-8}
				&\multicolumn{1}{c}{(1)}&\multicolumn{1}{c}{(2)}&\multicolumn{1}{c}{(3)}&\multicolumn{1}{c}{(4)}&\multicolumn{1}{c}{(5)}&\multicolumn{1}{c}{(6)}&\multicolumn{1}{c}{(7)}\\
				&\multicolumn{1}{c}{1M}&\multicolumn{1}{c}{2M}&\multicolumn{1}{c}{3M}&\multicolumn{1}{c}{4M}&\multicolumn{1}{c}{5M}&\multicolumn{1}{c}{6M}&\multicolumn{1}{c}{Donut}\\
				\midrule
				\multicolumn{5}{l}{\emph{Panel A. Over entire length of the life-course}} \\
				\hspace*{10pt}Overall&      -1.063\sym{***}&      -0.144         &      -0.427         &      -0.474         &      -0.250         &     -0.0901         &       0.104         \\
                    &  (4.12e-15)         &     (0.540)         &     (0.375)         &     (0.278)         &     (0.247)         &     (0.266)         &     (0.291)         \\
\midrule Dependent mean&       13.24         &       13.69         &       14.08         &       13.93         &       13.93         &       13.89         &       14.03         \\
Effect in SDs [\%]  &       70.05         &       11.45         &       30.88         &       34.64         &       19.06         &       7.260         &       9.060         \\
Observations        &          38         &          76         &         114         &         152         &         190         &         228         &         190         \\
 \\ \\
				\multicolumn{5}{l}{\emph{Panel B. Age brackets}} \\
				\hspace*{10pt}Age 27-29&      -0.438\sym{***}&       0.391         &      -0.301         &      -0.363         &      -0.274         &      -0.144         &     -0.0854         \\
                    &  (2.03e-15)         &     (0.752)         &     (0.587)         &     (0.428)         &     (0.340)         &     (0.335)         &     (0.376)         \\
 \hspace*{10pt}Age 30-32&      -0.924\sym{***}&      -0.133         &      -0.692         &      -0.912\sym{**} &      -0.571         &      -0.465         &      -0.373         \\
                    &  (3.21e-15)         &     (0.646)         &     (0.498)         &     (0.379)         &     (0.381)         &     (0.322)         &     (0.369)         \\
 \hspace*{10pt}Age 33-35&      -2.280\sym{***}&      -0.915         &      -0.518         &      -0.403         &      -0.158         &      0.0671         &       0.537         \\
                    &  (2.48e-15)         &     (0.652)         &     (0.524)         &     (0.388)         &     (0.331)         &     (0.384)         &     (0.352)         \\
 
				\bottomrule 
		\end{tabular}}
		\begin{tablenotes} 
			\item \scriptsize \emph{Notes:} Clustered standard errors in parentheses. All regression are run with CG2 (i.e. the cohort prior to the reform) and with month-of-birth FEs. The 'overall' specification includes year fixed effects, as well. The outcome variables are defined as the number of cases per thousand individuals (births). Dependent mean and the effect size correspond to pre-reform values of the treated group.
		\end{tablenotes} 
	\end{threeparttable} 
\end{table}
\vspace*{\fill}\clearpage 
\vspace*{\fill}
\begin{table}[H] \centering 
	\begin{threeparttable} \centering \caption{ITT effects on \textbf{mental and behavioral disorders (men)}}\label{tab: DD_hopsital2_total}
		{\def\sym#1{\ifmmode^{#1}\else\(^{#1}\)\fi} 
			\begin{tabular}{l*{8}{c}}
				\toprule 
				%\multicolumn{5}{l}{Dependant variable: \textbf{Hospital admission (total)}}\\ \\ 
				& \multicolumn{7}{c}{Estimation window} \\ 
				\cmidrule(lr){2-8}
				&\multicolumn{1}{c}{(1)}&\multicolumn{1}{c}{(2)}&\multicolumn{1}{c}{(3)}&\multicolumn{1}{c}{(4)}&\multicolumn{1}{c}{(5)}&\multicolumn{1}{c}{(6)}&\multicolumn{1}{c}{(7)}\\
				&\multicolumn{1}{c}{1M}&\multicolumn{1}{c}{2M}&\multicolumn{1}{c}{3M}&\multicolumn{1}{c}{4M}&\multicolumn{1}{c}{5M}&\multicolumn{1}{c}{6M}&\multicolumn{1}{c}{Donut}\\
				\midrule
				\multicolumn{5}{l}{\emph{Panel A. Over entire length of the life-course}} \\
				\hspace*{10pt}Overall&      -0.614\sym{***}&      -1.178\sym{***}&      -2.025\sym{***}&      -2.285\sym{***}&      -2.098\sym{***}&      -1.850\sym{***}&      -2.097\sym{***}\\
                    &  (2.75e-15)         &     (0.250)         &     (0.456)         &     (0.396)         &     (0.328)         &     (0.306)         &     (0.340)         \\
\midrule Dependent mean&       21.49         &       21.93         &       22.69         &       22.48         &       22.23         &       22.13         &       22.25         \\
Effect in SDs [\%]  &       37.46         &       59.50         &       85.57         &       95.52         &       85.46         &       76.51         &       82.54         \\
Observations        &          38         &          76         &         114         &         152         &         190         &         228         &         190         \\
 \\ \\
				\multicolumn{5}{l}{\emph{Panel B. Age brackets}} \\
				\hspace*{10pt}Age 27-29&      -0.884\sym{***}&      -0.800         &      -1.489\sym{**} &      -1.697\sym{**} &      -1.466\sym{**} &      -1.035\sym{*}  &      -1.065\sym{**} \\
                    &  (3.51e-15)         &     (0.832)         &     (0.638)         &     (0.660)         &     (0.533)         &     (0.509)         &     (0.508)         \\
 \hspace*{10pt}Age 30-32&       0.978\sym{***}&      -0.179         &      -1.491\sym{*}  &      -1.780\sym{***}&      -1.634\sym{***}&      -1.336\sym{***}&      -1.799\sym{***}\\
                    &  (2.03e-15)         &     (0.608)         &     (0.764)         &     (0.579)         &     (0.465)         &     (0.429)         &     (0.424)         \\
 \hspace*{10pt}Age 33-35&      -2.044\sym{***}&      -2.769\sym{***}&      -3.338\sym{***}&      -3.453\sym{***}&      -3.075\sym{***}&      -3.104\sym{***}&      -3.316\sym{***}\\
                    &  (1.39e-14)         &     (0.557)         &     (0.600)         &     (0.467)         &     (0.419)         &     (0.416)         &     (0.435)         \\
 
				\bottomrule 
		\end{tabular}}
		\begin{tablenotes} 
			\item \scriptsize \emph{Notes:} Clustered standard errors in parentheses. All regression are run with CG2 (i.e. the cohort prior to the reform) and with month-of-birth FEs. The 'overall' specification includes year fixed effects, as well. The outcome variables are defined as the number of cases per thousand individuals (births). Dependent mean and the effect size correspond to pre-reform values of the treated group.
		\end{tablenotes} 
	\end{threeparttable} 
\end{table}
\vspace*{\fill}\clearpage 
%--------------------------------------------
% d5 - robustness in one table - blueprint for paper
\newpage
%\begin{landscape}
	\vspace*{\fill}
	\begin{table}[htbp] \centering 
		\begin{threeparttable} \centering 
			\caption{Robustness checks for \textbf{mental and behavioral disorders}} \label{tab: robustness_d5} 
			{\def\sym#1{\ifmmode^{#1}\else\(^{#1}\)\fi} 
				\begin{tabular}{l*{10}{c}} \toprule 
					
					& & \multicolumn{2}{c}{Alternative specifications} & \multicolumn{3}{c}{\clb{c}{Alternative\\estimation}} & \multicolumn{2}{c}{Placebos}& \multicolumn{2}{c}{Heterogeneity}\\
					\cmidrule(lr){3-4} \cmidrule(lr){5-7} \cmidrule(lr){8-9} \cmidrule(lr){10-11}
					&\multicolumn{1}{c}{(1)}&\multicolumn{1}{c}{(2)}&\multicolumn{1}{c}{(3)}&\multicolumn{1}{c}{(4)}&\multicolumn{1}{c}{(5)}&\multicolumn{1}{c}{(6)}&\multicolumn{1}{c}{(7)}&\multicolumn{1}{c}{(8)}&\multicolumn{1}{c}{(9)}&\multicolumn{1}{c}{(10)}\\
					&\multicolumn{1}{c}{Baseline}&\multicolumn{1}{c}{\clb{c}{current\\population}}&\multicolumn{1}{c}{\clb{c}{LMR\\level$^a$}}&\multicolumn{1}{c}{\clb{c}{DDD$^b$}}&\multicolumn{1}{c}{\clb{c}{alt. DD$^b$}}&\multicolumn{1}{c}{add. CG}&\multicolumn{1}{c}{\clb{c}{temporal:\\cohort}}&\multicolumn{1}{c}{\clb{c}{spatial:\\ GDR}}&\multicolumn{1}{c}{\clb{c}{rural$^a$}}&\multicolumn{1}{c}{\clb{c}{urban$^a$}}\\
					\midrule
					\\
					(1) {total} 		&   -0.634\sym{**}	&	-0.832\sym{***}	&   -0.831\sym{**}  &	-0.834\sym{**}  & 	-0.558\sym{**}  & -0.553\sym{**}	&	0.162			&	0.200		&	-0.0987		&	-1.006\sym{***} 	\\
										&	(0.249)			&	(0.239)			&   (0.229)     	&	(0.321)			& 	(0.157)			& (0.269)			&	(0.304)			&	(0.141)		&	(0.588)		&	(0.202)				\\
					(2) {female}		&   0.0599			&	-0.0853			& 	-0.0724     	&	-0.0385			& 	-0.217			& 0.289			    &	0.457			&	0.0984		&	0.299		&	-0.161				\\
										&	(0.266)			&	(0.266)			&   (0.258)     	&	(0.320)			& 	(0.144)			& (0.287)			&	(0.369)			&	(0.196)		&	(0.816)		&	(0.234)				\\
					(3) {male} 			&   -1.267\sym{***}	&	-1.554\sym{***}	&   -1.598\sym{***} &	-1.533\sym{***} & 	-0.834\sym{***} & -1.323\sym{***}	&	-0.112			&	0.266		&	-0.414		&	-1.877\sym{***} 	\\
										&	(0.292)			&	(0.299)			&   (0.286)     	&	(0.388)			& 	(0.190)			& (0.322)			&	 (0.331) 		&	(0.198)		&	(0.487)		&	(0.333)				\\
					\midrule            																																																						
					For total: 																																																				\\							 
					Dependent mean 		&   18.96			&	17.28			&   17.65     		&	13.80			& 	13.80			& 18.96				&	18.67			&	8.640		&	16.76		&	18.38				\\
					Effect in SDs [\%] 	&   11.43			&	41.92			&   5.20      		&	12.53			& 	8.380			& 9.960				&	3.490			&	9.440		&	1.69		&	7.27				\\
					$N$ (MOB $\times$ year)  		&   480				&	288				&   58,751    		&	960				& 	480				& 720				&	480				&	480			&	26,495		&	32,256				\\
					%Federal level		&   \checkmark		&	\checkmark		&   $\times$		& \checkmark		&	\checkmark		& \checkmark		&	\checkmark		&  \checkmark	&	$\times$	&	$\times$			\\ 
					\\
					MOB fixed effects 	&   \checkmark		&	\checkmark		&   \checkmark		& \checkmark		&	\checkmark		& \checkmark		&	\checkmark		&  \checkmark	&	\checkmark	&	\checkmark		\\ 
					Year fixed effects  &   \checkmark		&	\checkmark		&   \checkmark		& \checkmark		&	\checkmark		& \checkmark		&	\checkmark		&  \checkmark	&	\checkmark	&	\checkmark		\\ 

					\bottomrule
			\end{tabular}}
	\end{threeparttable} 
		\begin{minipage}{0.87\linewidth}
		\scriptsize \emph{Notes:} This table displays robustness check for the effect of the 1979 maternity leave reform on mental and behavioral disorders. We perform the following checks (with reference to the column): (1) baseline specification that was used in previous parts of the paper, (2) for the outcome we use the number of diagnoses divided by the current number of individuals (approximation), (3) the analysis is carried out on the level of labor market regions, (4) triple difference model (the third difference stems from the former region of the GDR), (5) alternative difference-in-difference model which compares pre and post of the treatment cohort in West Germany with the respective values in East Germany, (6) we use as control cohort not only the cohort before the reform, but also the cohort 2 years prior to the policy change, (7) first placebo, in which the entire analysis set-up is pushed back by one year, i.e. the placebo TG is the cohort prior to the real TG and the placebo CG is the cohort born 2 years before the reform took place, (8) second placebo, in which we run the normal DD set-up in the area of the former GDR, (9) + (10)  DD carried out in rural and urban regions (compare with figure \ref{fig: AMR_regions_population_density} to see which regions are marked as rural/urban). \newline Significance levels: * p < 0.10, ** p < 0.05, *** p < 0.01. \newline
		\hspace*{15 pt}$^a$: level of analysis on Labor Market Regions: weighted regressions (by population), includes region fixed effects.\newline
		\hspace*{15 pt}$^b$: standard errors clustered on the month-of-birth$\times$birth-cohort$\times$East-West cell level.
	\end{minipage}
\end{table} 
	\vspace*{\fill}\clearpage
\end{landscape}

% Columns 1, 2 8 (lokal) ok 
% COLUMN 8 uses the r_popf specification
% sind DDD, alt DD, spatial placebo mit r_pop







%
%
%%--------------------------------------------------------------------
%% APPENDIX
%%--------------------------------------------------------------------
%\newpage
%\section{Appendix}
%
%
%% LC MATRIX FOR ALL CHAPTERS
%
%% Part 1 of LC matrix
%\begin{figure}[H]\centering
%	\caption{Life-course approach for all chapters}\label{fig: appendix_lc_matrix_chapters}
%	\includegraphics[width=1.1\linewidth]{paper/lc_matrix_chapters_1.pdf}
%		\scriptsize
%		\begin{minipage}{\linewidth}
%			\emph{Continued on next page}
%		\end{minipage}
%\end{figure}
%
%%Part 2 of LC matrix
%\begin{figure}[H]\centering
%		\begin{minipage}{\linewidth}\scriptsize
%		\begin{center} \emph{Life-course approach for all chapters (continued)}\end{center}
%	\end{minipage}
%	\includegraphics[width=1.1\linewidth]{paper/lc_matrix_chapters_2.pdf}
%		\begin{minipage}{\linewidth}
%		\scriptsize \emph{Notes:} This figure plots intention-to-treat estimates (along with confidence intervals) across the main diagnosis chapters for the entire life-course. The outcomes are defined as the number of cases per 1,000 individuals (births). The point estimates are coming from a DiD regression as described in section \ref{sec:empirical_strategy}, with a bandwidth of six months, month-of-birth fixed effects, and clustered standard errors on the month-of-birth level. The control group is comprised of children that are born in the same months but one year before (i.e. children born between November 1977 and October 1978).
%	\end{minipage}
%\end{figure}
%%--------------------------------------------
%% LC MATRIX D5 SUBCATGEORIES
%\begin{figure}[H]\centering
%	\caption{Life-course approach for the subcategories of mental and behavioral disorders.}\label{fig: appendix_lc_matrix_d5_subcateg}
%	\includegraphics[width=1.1\linewidth]{paper/lc_matrix_d5subcategories.pdf}
%		\begin{minipage}{\linewidth}
%		\scriptsize \emph{Notes:} This figure plots intention-to-treat estimates (along with confidence intervals) across the subcategories of mental and behavioral disorders. The outcomes are defined as the number of cases per 1,000 individuals (births). The point estimates are coming from a DiD regression as described in section \ref{sec:empirical_strategy}, with a bandwidth of six months, month-of-birth fixed effects, and clustered standard errors on the month-of-birth level. The control group is comprised of children that are born in the same months but one year before (i.e. children born between November 1977 and October 1978).
%	\end{minipage}
%\end{figure}
%%--------------------------------------------
%
%% TABLES
%
%%t-tests
%%%hospital total
\begin{table}[H] \centering 
	\begin{threeparttable} \centering \caption{t-test for \textbf{hospital admission (total)}}\label{tab:t-test_d5total}
		\begin{footnotesize}
			{\def\sym#1{\ifmmode^{#1}\else\(^{#1}\)\fi} 
				\begin{tabular}{l*{3}{c}}
					\toprule 
					& \multicolumn{1}{c}{Before} & \multicolumn{1}{c}{After} & \multicolumn{1}{c}{Difference} \\
					&\multicolumn{1}{c}{(1)}&\multicolumn{1}{c}{(2)}&\multicolumn{1}{c}{(1)-(2)}\\
					\midrule
					Overall (pooled)    &       120.6&       118.4&        2.22   \\
                    &      [11.0]&      [9.85]&      (1.35)   \\
\hspace{12pt}1995   &       111.1&       106.7&        4.48** \\
                    &      [3.65]&      [2.73]&      (1.86)   \\
\hspace{12pt}1996   &       117.6&       112.5&        5.08** \\
                    &      [5.13]&      [1.92]&      (2.24)   \\
\hspace{12pt}1997   &       125.9&       120.1&        5.76** \\
                    &      [5.12]&      [2.07]&      (2.25)   \\
\hspace{12pt}1998   &       126.1&       125.1&        1.00   \\
                    &      [3.99]&      [2.48]&      (1.92)   \\
\hspace{12pt}1999   &       124.7&       124.7&     -0.0025   \\
                    &      [4.76]&      [2.48]&      (2.19)   \\
\hspace{12pt}2000   &       120.1&       119.1&        0.92   \\
                    &      [4.39]&      [3.69]&      (2.34)   \\
\hspace{12pt}2001   &       119.4&       121.9&       -2.48   \\
                    &      [4.33]&      [3.47]&      (2.27)   \\
\hspace{12pt}2002   &       120.6&       118.6&        1.99   \\
                    &      [5.13]&      [1.76]&      (2.21)   \\
\hspace{12pt}2003   &       113.6&       112.5&        1.12   \\
                    &      [4.75]&      [0.77]&      (1.96)   \\
\hspace{12pt}2004   &       109.2&       109.0&        0.18   \\
                    &      [3.49]&      [2.61]&      (1.78)   \\
\hspace{12pt}2005   &       105.6&       105.5&       0.045   \\
                    &      [4.36]&      [3.64]&      (2.32)   \\
\hspace{12pt}2006   &       108.9&       106.0&        2.84   \\
                    &      [3.78]&      [1.64]&      (1.68)   \\
\hspace{12pt}2007   &       109.8&       108.2&        1.67   \\
                    &      [4.46]&      [2.47]&      (2.08)   \\
\hspace{12pt}2008   &       113.9&       112.0&        1.91   \\
                    &      [4.47]&      [1.37]&      (1.91)   \\
\hspace{12pt}2009   &       119.2&       117.9&        1.34   \\
                    &      [5.47]&      [2.10]&      (2.39)   \\
\hspace{12pt}2010   &       122.5&       118.8&        3.67   \\
                    &      [5.48]&      [2.92]&      (2.53)   \\
\hspace{12pt}2011   &       128.6&       123.7&        4.84*  \\
                    &      [5.69]&      [1.51]&      (2.40)   \\
\hspace{12pt}2012   &       133.8&       129.1&        4.67** \\
                    &      [4.72]&      [1.97]&      (2.09)   \\
\hspace{12pt}2013   &       136.4&       134.6&        1.80   \\
                    &      [4.85]&      [1.59]&      (2.09)   \\
\hspace{12pt}2014   &       145.5&       142.0&        3.51   \\
                    &      [7.17]&      [2.97]&      (3.17)   \\

					\bottomrule
			\end{tabular}}
		\end{footnotesize}
	\end{threeparttable} 
	\begin{minipage}{0.9\linewidth}
		\scriptsize \emph{Notes:} This table shows descriptive statistics for two samples: (i) cohorts born before May 1979; (ii) cohorts born after May 1979. Columns 1 and 2 show means with standard deviations in brackets. Column 3 report the difference in means between columns 1 and 2 with standard errors in parenthesis. For each sample we take a bandwidth of half a year around the cutoff date, i.e. individuals born between Nov78-Ap79 (May79-Oct79) in the 'before' ('after') sample.
	\end{minipage}
\end{table} 
%hospital women
\begin{table}[H] \centering 
	\begin{threeparttable} \centering \caption{t-test for \textbf{hospital admission (women)}}\label{tab:t-test_d5female}
		\begin{footnotesize}
			{\def\sym#1{\ifmmode^{#1}\else\(^{#1}\)\fi} 
				\begin{tabular}{l*{3}{c}}
					\toprule 
					& \multicolumn{1}{c}{Before} & \multicolumn{1}{c}{After} & \multicolumn{1}{c}{Difference} \\
					&\multicolumn{1}{c}{(1)}&\multicolumn{1}{c}{(2)}&\multicolumn{1}{c}{(1)-(2)}\\
					\midrule
					Overall (pooled)    &       122.4&       120.6&        1.85   \\
                    &      [11.1]&      [10.8]&      (1.42)   \\
\hspace{12pt}1995   &       124.5&       119.4&        5.14** \\
                    &      [4.21]&      [2.06]&      (1.91)   \\
\hspace{12pt}1996   &       129.7&       125.1&        4.61   \\
                    &      [6.62]&      [1.80]&      (2.80)   \\
\hspace{12pt}1997   &       135.0&       131.1&        3.90   \\
                    &      [5.25]&      [3.80]&      (2.65)   \\
\hspace{12pt}1998   &       131.9&       132.8&       -0.89   \\
                    &      [5.50]&      [4.08]&      (2.80)   \\
\hspace{12pt}1999   &       128.2&       129.0&       -0.81   \\
                    &      [4.89]&      [3.66]&      (2.49)   \\
\hspace{12pt}2000   &       123.9&       122.5&        1.37   \\
                    &      [4.76]&      [4.68]&      (2.72)   \\
\hspace{12pt}2001   &       122.6&       126.1&       -3.52   \\
                    &      [4.55]&      [4.36]&      (2.57)   \\
\hspace{12pt}2002   &       124.1&       121.1&        3.04   \\
                    &      [6.53]&      [2.35]&      (2.83)   \\
\hspace{12pt}2003   &       115.8&       115.1&        0.77   \\
                    &      [4.75]&      [1.55]&      (2.04)   \\
\hspace{12pt}2004   &       109.5&       109.8&       -0.37   \\
                    &      [4.92]&      [3.32]&      (2.42)   \\
\hspace{12pt}2005   &       103.5&       104.1&       -0.60   \\
                    &      [4.02]&      [3.50]&      (2.18)   \\
\hspace{12pt}2006   &       107.6&       104.6&        3.01   \\
                    &      [3.91]&      [3.43]&      (2.12)   \\
\hspace{12pt}2007   &       108.3&       105.8&        2.55   \\
                    &      [4.61]&      [2.82]&      (2.21)   \\
\hspace{12pt}2008   &       112.3&       109.4&        2.90   \\
                    &      [4.30]&      [2.60]&      (2.05)   \\
\hspace{12pt}2009   &       118.2&       114.7&        3.54   \\
                    &      [5.05]&      [2.28]&      (2.26)   \\
\hspace{12pt}2010   &       120.7&       116.6&        4.12   \\
                    &      [5.52]&      [3.49]&      (2.67)   \\
\hspace{12pt}2011   &       126.4&       121.8&        4.53   \\
                    &      [5.58]&      [3.14]&      (2.61)   \\
\hspace{12pt}2012   &       131.5&       126.4&        5.15** \\
                    &      [4.92]&      [2.12]&      (2.19)   \\
\hspace{12pt}2013   &       133.1&       133.8&       -0.69   \\
                    &      [4.11]&      [3.97]&      (2.33)   \\
\hspace{12pt}2014   &       141.5&       142.3&       -0.76   \\
                    &      [6.44]&      [2.95]&      (2.89)   \\

					\bottomrule
			\end{tabular}}
		\end{footnotesize}
	\end{threeparttable} 
	\begin{minipage}{0.9\linewidth}
		\scriptsize \emph{Notes:} This table shows descriptive statistics for two samples: (i) cohorts born before May 1979; (ii) cohorts born after May 1979. Columns 1 and 2 show means with standard deviations in brackets. Column 3 report the difference in means between columns 1 and 2 with standard errors in parenthesis. For each sample we take a bandwidth of half a year around the cutoff date, i.e. individuals born between Nov78-Ap79 (May79-Oct79) in the 'before' ('after') sample.
	\end{minipage}
\end{table} 
%hospital men
\begin{table}[H] \centering 
	\begin{threeparttable} \centering \caption{t-test for \textbf{hospital admission (men)}}\label{tab:t-test_d5male}
		\begin{footnotesize}
			{\def\sym#1{\ifmmode^{#1}\else\(^{#1}\)\fi} 
				\begin{tabular}{l*{3}{c}}
					\toprule 
					& \multicolumn{1}{c}{Before} & \multicolumn{1}{c}{After} & \multicolumn{1}{c}{Difference} \\
					&\multicolumn{1}{c}{(1)}&\multicolumn{1}{c}{(2)}&\multicolumn{1}{c}{(1)-(2)}\\
					\midrule
					Overall (pooled)    &       118.9&       116.4&        2.59   \\
                    &      [13.1]&      [11.5]&      (1.59)   \\
\hspace{12pt}1995   &        98.5&        94.5&        3.96*  \\
                    &      [3.22]&      [3.71]&      (2.01)   \\
\hspace{12pt}1996   &       106.1&       100.5&        5.61** \\
                    &      [4.06]&      [2.32]&      (1.91)   \\
\hspace{12pt}1997   &       117.3&       109.7&        7.62** \\
                    &      [5.86]&      [1.59]&      (2.48)   \\
\hspace{12pt}1998   &       120.6&       117.8&        2.84   \\
                    &      [5.58]&      [2.09]&      (2.43)   \\
\hspace{12pt}1999   &       121.4&       120.6&        0.80   \\
                    &      [4.82]&      [3.25]&      (2.37)   \\
\hspace{12pt}2000   &       116.4&       115.9&        0.52   \\
                    &      [4.88]&      [3.55]&      (2.46)   \\
\hspace{12pt}2001   &       116.4&       117.9&       -1.46   \\
                    &      [4.64]&      [4.33]&      (2.59)   \\
\hspace{12pt}2002   &       117.3&       116.3&        1.02   \\
                    &      [5.87]&      [1.40]&      (2.46)   \\
\hspace{12pt}2003   &       111.6&       110.1&        1.47   \\
                    &      [5.50]&      [1.82]&      (2.36)   \\
\hspace{12pt}2004   &       108.9&       108.2&        0.70   \\
                    &      [3.18]&      [3.56]&      (1.95)   \\
\hspace{12pt}2005   &       107.5&       106.9&        0.64   \\
                    &      [5.92]&      [5.48]&      (3.29)   \\
\hspace{12pt}2006   &       110.1&       107.4&        2.68   \\
                    &      [4.95]&      [2.67]&      (2.30)   \\
\hspace{12pt}2007   &       111.3&       110.5&        0.83   \\
                    &      [7.50]&      [3.44]&      (3.37)   \\
\hspace{12pt}2008   &       115.4&       114.4&        0.96   \\
                    &      [6.04]&      [2.28]&      (2.63)   \\
\hspace{12pt}2009   &       120.2&       121.0&       -0.76   \\
                    &      [6.70]&      [2.58]&      (2.93)   \\
\hspace{12pt}2010   &       124.2&       121.0&        3.22   \\
                    &      [6.45]&      [2.50]&      (2.83)   \\
\hspace{12pt}2011   &       130.7&       125.6&        5.13*  \\
                    &      [5.85]&      [1.83]&      (2.50)   \\
\hspace{12pt}2012   &       136.0&       131.8&        4.21   \\
                    &      [4.93]&      [2.85]&      (2.32)   \\
\hspace{12pt}2013   &       139.5&       135.4&        4.15   \\
                    &      [6.11]&      [2.77]&      (2.74)   \\
\hspace{12pt}2014   &       149.3&       141.8&        7.57*  \\
                    &      [8.14]&      [3.18]&      (3.57)   \\

					\bottomrule
			\end{tabular}}
		\end{footnotesize}
	\end{threeparttable} 
	\begin{minipage}{0.9\linewidth}
		\scriptsize \emph{Notes:} This table shows descriptive statistics for two samples: (i) cohorts born before May 1979; (ii) cohorts born after May 1979. Columns 1 and 2 show means with standard deviations in brackets. Column 3 report the difference in means between columns 1 and 2 with standard errors in parenthesis. For each sample we take a bandwidth of half a year around the cutoff date, i.e. individuals born between Nov78-Ap79 (May79-Oct79) in the 'before' ('after') sample.
	\end{minipage}
\end{table} 



%d5 total
\begin{table}[H] \centering 
	\begin{threeparttable} \centering \caption{t-test for \textbf{mental \& behavioral disorders (total)}}\label{tab:t-test_d5total}
		\begin{footnotesize}
			{\def\sym#1{\ifmmode^{#1}\else\(^{#1}\)\fi} 
			\begin{tabular}{l*{3}{c}}
				\toprule 
				& \multicolumn{1}{c}{Before} & \multicolumn{1}{c}{After} & \multicolumn{1}{c}{Difference} \\
				&\multicolumn{1}{c}{(1)}&\multicolumn{1}{c}{(2)}&\multicolumn{1}{c}{(1)-(2)}\\
				\midrule
				Overall (pooled)    &        19.0&        18.6&        0.38   \\
                    &      [5.55]&      [5.44]&      (0.71)   \\
\hspace{12pt}1995   &        7.38&        6.90&        0.48   \\
                    &      [0.61]&      [0.24]&      (0.27)   \\
\hspace{12pt}1996   &        8.64&        8.33&        0.31   \\
                    &      [0.79]&      [0.43]&      (0.37)   \\
\hspace{12pt}1997   &        11.1&        10.2&        0.91   \\
                    &      [1.18]&      [0.60]&      (0.54)   \\
\hspace{12pt}1998   &        13.4&        12.7&        0.69   \\
                    &      [1.17]&      [1.05]&      (0.64)   \\
\hspace{12pt}1999   &        15.1&        15.4&       -0.36   \\
                    &      [0.88]&      [0.55]&      (0.42)   \\
\hspace{12pt}2000   &        16.3&        16.2&        0.13   \\
                    &      [0.83]&      [0.62]&      (0.42)   \\
\hspace{12pt}2001   &        17.8&        18.8&       -1.03*  \\
                    &      [1.03]&      [0.74]&      (0.52)   \\
\hspace{12pt}2002   &        18.2&        18.3&      -0.059   \\
                    &      [0.88]&      [0.66]&      (0.45)   \\
\hspace{12pt}2003   &        19.0&        18.3&        0.68   \\
                    &      [1.51]&      [0.68]&      (0.68)   \\
\hspace{12pt}2004   &        19.6&        19.3&        0.28   \\
                    &      [1.18]&      [0.96]&      (0.62)   \\
\hspace{12pt}2005   &        20.1&        19.8&        0.36   \\
                    &      [1.08]&      [0.99]&      (0.60)   \\
\hspace{12pt}2006   &        20.8&        20.1&        0.73*  \\
                    &      [0.79]&      [0.51]&      (0.38)   \\
\hspace{12pt}2007   &        21.5&        21.1&        0.44   \\
                    &      [1.05]&      [0.44]&      (0.47)   \\
\hspace{12pt}2008   &        22.0&        21.5&        0.56   \\
                    &      [1.41]&      [0.61]&      (0.63)   \\
\hspace{12pt}2009   &        22.7&        22.7&      -0.058   \\
                    &      [1.42]&      [1.37]&      (0.80)   \\
\hspace{12pt}2010   &        23.7&        22.5&        1.22** \\
                    &      [0.96]&      [0.78]&      (0.51)   \\
\hspace{12pt}2011   &        23.9&        23.4&        0.50   \\
                    &      [1.31]&      [0.75]&      (0.62)   \\
\hspace{12pt}2012   &        24.7&        24.3&        0.35   \\
                    &      [0.78]&      [1.24]&      (0.60)   \\
\hspace{12pt}2013   &        25.8&        25.2&        0.64   \\
                    &      [1.20]&      [0.96]&      (0.63)   \\
\hspace{12pt}2014   &        27.3&        26.5&        0.88   \\
                    &      [1.53]&      [1.10]&      (0.77)   \\

				\bottomrule
			\end{tabular}}
		\end{footnotesize}
	\end{threeparttable} 
	\begin{minipage}{0.9\linewidth}
		\scriptsize \emph{Notes:} This table shows descriptive statistics for two samples: (i) cohorts born before May 1979; (ii) cohorts born after May 1979. Columns 1 and 2 show means with standard deviations in brackets. Column 3 report the difference in means between columns 1 and 2 with standard errors in parenthesis. For each sample we take a bandwidth of half a year around the cutoff date, i.e. individuals born between Nov78-Ap79 (May79-Oct79) in the 'before' ('after') sample.
	\end{minipage}
\end{table} 
%d5 women
\begin{table}[H] \centering 
	\begin{threeparttable} \centering \caption{t-test for \textbf{mental \& behavioral disorders (women)}}\label{tab:t-test_d5female}
		\begin{footnotesize}
			{\def\sym#1{\ifmmode^{#1}\else\(^{#1}\)\fi} 
				\begin{tabular}{l*{3}{c}}
					\toprule 
					& \multicolumn{1}{c}{Before} & \multicolumn{1}{c}{After} & \multicolumn{1}{c}{Difference} \\
					&\multicolumn{1}{c}{(1)}&\multicolumn{1}{c}{(2)}&\multicolumn{1}{c}{(1)-(2)}\\
					\midrule
					Overall (pooled)    &        15.7&        15.8&      -0.070   \\
                    &      [3.59]&      [3.56]&      (0.46)   \\
\hspace{12pt}1995   &        8.73&        8.62&        0.10   \\
                    &      [0.78]&      [0.58]&      (0.40)   \\
\hspace{12pt}1996   &        9.30&        9.79&       -0.49   \\
                    &      [1.16]&      [0.82]&      (0.58)   \\
\hspace{12pt}1997   &        11.6&        11.1&        0.48   \\
                    &      [1.07]&      [0.52]&      (0.49)   \\
\hspace{12pt}1998   &        13.0&        13.1&       -0.15   \\
                    &      [1.73]&      [1.62]&      (0.97)   \\
\hspace{12pt}1999   &        12.8&        13.6&       -0.75   \\
                    &      [0.61]&      [0.98]&      (0.47)   \\
\hspace{12pt}2000   &        13.1&        13.6&       -0.45   \\
                    &      [0.25]&      [1.10]&      (0.46)   \\
\hspace{12pt}2001   &        14.2&        16.5&       -2.32** \\
                    &      [1.36]&      [1.29]&      (0.76)   \\
\hspace{12pt}2002   &        15.3&        15.3&     -0.0090   \\
                    &      [1.28]&      [0.65]&      (0.59)   \\
\hspace{12pt}2003   &        16.1&        15.6&        0.55   \\
                    &      [1.85]&      [0.74]&      (0.81)   \\
\hspace{12pt}2004   &        15.7&        16.3&       -0.62   \\
                    &      [1.45]&      [1.15]&      (0.75)   \\
\hspace{12pt}2005   &        16.2&        15.9&        0.35   \\
                    &      [0.87]&      [1.62]&      (0.75)   \\
\hspace{12pt}2006   &        16.7&        16.3&        0.40   \\
                    &      [0.78]&      [1.20]&      (0.58)   \\
\hspace{12pt}2007   &        17.7&        16.9&        0.77   \\
                    &      [1.31]&      [0.98]&      (0.67)   \\
\hspace{12pt}2008   &        18.0&        16.9&        1.16   \\
                    &      [1.66]&      [1.23]&      (0.84)   \\
\hspace{12pt}2009   &        18.2&        17.4&        0.76   \\
                    &      [1.47]&      [1.58]&      (0.88)   \\
\hspace{12pt}2010   &        18.6&        18.1&        0.53   \\
                    &      [0.69]&      [1.24]&      (0.58)   \\
\hspace{12pt}2011   &        18.2&        18.3&       -0.13   \\
                    &      [1.05]&      [1.44]&      (0.73)   \\
\hspace{12pt}2012   &        20.0&        20.0&     0.00093   \\
                    &      [0.98]&      [1.26]&      (0.65)   \\
\hspace{12pt}2013   &        20.2&        20.8&       -0.56   \\
                    &      [1.02]&      [1.48]&      (0.73)   \\
\hspace{12pt}2014   &        21.2&        22.3&       -1.03   \\
                    &      [1.25]&      [1.04]&      (0.66)   \\

					\bottomrule
			\end{tabular}}
		\end{footnotesize}
	\end{threeparttable} 
	\begin{minipage}{0.9\linewidth}
		\scriptsize \emph{Notes:} This table shows descriptive statistics for two samples: (i) cohorts born before May 1979; (ii) cohorts born after May 1979. Columns 1 and 2 show means with standard deviations in brackets. Column 3 report the difference in means between columns 1 and 2 with standard errors in parenthesis. For each sample we take a bandwidth of half a year around the cutoff date, i.e. individuals born between Nov78-Ap79 (May79-Oct79) in the 'before' ('after') sample.
	\end{minipage}
\end{table} 
%d5 men
\begin{table}[H] \centering 
	\begin{threeparttable} \centering \caption{t-test for \textbf{mental \& behavioral disorders (men)}}\label{tab:t-test_d5male}
		\begin{footnotesize}
			{\def\sym#1{\ifmmode^{#1}\else\(^{#1}\)\fi} 
				\begin{tabular}{l*{3}{c}}
					\toprule 
					& \multicolumn{1}{c}{Before} & \multicolumn{1}{c}{After} & \multicolumn{1}{c}{Difference} \\
					&\multicolumn{1}{c}{(1)}&\multicolumn{1}{c}{(2)}&\multicolumn{1}{c}{(1)-(2)}\\
					\midrule
					Overall (pooled)    &        22.0&        21.2&        0.79   \\
                    &      [7.55]&      [7.45]&      (0.97)   \\
\hspace{12pt}1995   &        6.09&        5.26&        0.84** \\
                    &      [0.62]&      [0.67]&      (0.37)   \\
\hspace{12pt}1996   &        8.01&        6.93&        1.08***\\
                    &      [0.52]&      [0.45]&      (0.28)   \\
\hspace{12pt}1997   &        10.5&        9.20&        1.32*  \\
                    &      [1.31]&      [0.82]&      (0.63)   \\
\hspace{12pt}1998   &        13.8&        12.3&        1.48** \\
                    &      [0.95]&      [1.08]&      (0.59)   \\
\hspace{12pt}1999   &        17.2&        17.2&      0.0075   \\
                    &      [1.36]&      [1.05]&      (0.70)   \\
\hspace{12pt}2000   &        19.4&        18.7&        0.67   \\
                    &      [1.57]&      [0.71]&      (0.70)   \\
\hspace{12pt}2001   &        21.2&        21.0&        0.18   \\
                    &      [1.41]&      [1.35]&      (0.80)   \\
\hspace{12pt}2002   &        21.0&        21.2&       -0.13   \\
                    &      [0.93]&      [1.37]&      (0.68)   \\
\hspace{12pt}2003   &        21.7&        21.0&        0.78   \\
                    &      [1.48]&      [0.86]&      (0.70)   \\
\hspace{12pt}2004   &        23.3&        22.2&        1.10   \\
                    &      [1.33]&      [1.11]&      (0.71)   \\
\hspace{12pt}2005   &        23.8&        23.5&        0.34   \\
                    &      [1.48]&      [1.63]&      (0.90)   \\
\hspace{12pt}2006   &        24.7&        23.7&        1.02*  \\
                    &      [0.93]&      [0.92]&      (0.53)   \\
\hspace{12pt}2007   &        25.2&        25.1&       0.092   \\
                    &      [1.86]&      [1.22]&      (0.91)   \\
\hspace{12pt}2008   &        25.9&        25.9&      -0.033   \\
                    &      [1.96]&      [1.35]&      (0.97)   \\
\hspace{12pt}2009   &        26.9&        27.8&       -0.87   \\
                    &      [1.78]&      [1.36]&      (0.91)   \\
\hspace{12pt}2010   &        28.5&        26.7&        1.84*  \\
                    &      [2.02]&      [0.68]&      (0.87)   \\
\hspace{12pt}2011   &        29.3&        28.3&        1.06   \\
                    &      [1.79]&      [0.70]&      (0.78)   \\
\hspace{12pt}2012   &        29.1&        28.4&        0.66   \\
                    &      [0.66]&      [1.36]&      (0.62)   \\
\hspace{12pt}2013   &        31.2&        29.4&        1.75*  \\
                    &      [2.04]&      [1.16]&      (0.96)   \\
\hspace{12pt}2014   &        33.1&        30.5&        2.66** \\
                    &      [1.99]&      [1.44]&      (1.00)   \\

					\bottomrule
			\end{tabular}}
		\end{footnotesize}
	\end{threeparttable} 
	\begin{minipage}{0.9\linewidth}
		\scriptsize \emph{Notes:} This table shows descriptive statistics for two samples: (i) cohorts born before May 1979; (ii) cohorts born after May 1979. Columns 1 and 2 show means with standard deviations in brackets. Column 3 report the difference in means between columns 1 and 2 with standard errors in parenthesis. For each sample we take a bandwidth of half a year around the cutoff date, i.e. individuals born between Nov78-Ap79 (May79-Oct79) in the 'before' ('after') sample.
	\end{minipage}
\end{table} 
%
%
%
%
%
%%--------------------------------------------------------------------
%% Useful links for the paper
%%-------------------------------------------------------------------
%\newpage
%\subsection*{Useful links for the paper}
%\begin{itemize}
%\item \textbf{breastfeeding \& metabolism:}
%
%\begin{itemize}
%\item[-]\href{http://www.who.int/elena/titles/bbc/breastfeeding_childhood_obesity/en/}{WHO link})\\ 
%\textit{"The positive impact of breastfeeding on lowering the risk of death from infectious diseases in the first two years of life is now well-established (1). A mounting body of evidence suggests that breastfeeding may also play a role in programming noncommunicable disease risk later in life (2-13) including protection against overweight and obesity in childhood (2-6)."}
%
%\item[-]\href{http://articles.latimes.com/2011/may/02/news/la-heb-infant-feeding-20110502}{LA Times article}\\
%\textit{"The study showed that children who received breast milk for the first four months had a specific pattern of growth and metabolic profile that differed from the formula-fed babies. Even at 15 days of life, the breast-fed infants had blood insulin levels that were lower than the formula-fed infants.\newline
%By 3 years of age, many of the metabolic and growth differences between the breast-fed and formula-fed infants had disappeared. However, blood pressure readings were higher in the infants who had been fed the high-protein formula compared with breast-fed infants. The blood pressure rates were still within the normal range."}
%\end{itemize}
%\end{itemize}






% NOT USED ANY MORE
%-----------------------------------------------------
% Earlier version of descriptive, zum Teil 2 lines in einem Graph zusammengefasst
%\begin{figure}[H]\centering
%	\caption{Evolution of key variables}\label{fig: hospital_admissions_lengthstay_surgery}
%	\begin{subfigure}[h]{0.48\linewidth}\centering
%		\caption{Hospital admissions}
%		\includegraphics[width=\linewidth]{paper/hospitaladmission_genderpercent_total}
%	\end{subfigure}
%	\quad
%	\begin{subfigure}[h]{0.48\linewidth}\centering
%		\caption{Length of stay \& surgery}
%		\includegraphics[width=\linewidth]{paper/surgery_lengthofstay}
%	\end{subfigure}
%	\begin{minipage}{\linewidth}
%		\scriptsize{\emph{Notes:} The figure depicts the evolution of key variables for the treatment cohort (i.e. for individuals born between November 1978 and October 1979). In the left figure, the bold gray line shows the total number of hospital admissions (in thousand, right axis), whereas the red (blue) line indicates the relative frequency of female (male) hospital admissions (on the left axis). Hospital admissions are defined as the sum of all diagnoses, except for diagnoses of the "O" chapter (pregnancy, childbirth, and the puerperium). The right figure reports the relative frequency of surgeries (green line, left axis) and the average number of days the patient stayed in the hospital (yellow line, right axis).} 
%	\end{minipage}
%\end{figure}
%-----------------------------------------------------
% Figure: effect sizes and frequency across chapters
% \vspace*{\fill}
% \begin{figure}[H]\centering
% 	\includegraphics[width = 0.9 \linewidth]{paper/effect_chapters_frequency.pdf}
% 	\begin{minipage}
% 		{\linewidth}
% 		\emph{Notes:} %The figures plot DD estimates (along with 90\% and 95\% confidence intervals) for the impact of the reform on hospital admission over the life-course. The light gray line in the background represents the number of diagnoses that were made for both the treatment and control year. Panel a shows the results for all admissions, whereas panel b and c show the estimates for females and males respectively. The control group is comprised of children	that are born in the same months but one year before (i.e. children born between November 1977 and October 1978.)
% 	\end{minipage}
% \end{figure}
% \vspace*{\fill}\clearpage
%-----------------------------------------------------

\end{document}